\chapter{Year 2011}
\label{2011}

\section{January 15: Chateau Chillon}
\label{2011:Chillon}

Not having had a trip over Christmas break both my brother and I were eager to get out somewhere. The chateau de Chillon is a castle located on a little island just a couple of metres away from the shores of lake Geneva. Still conserved in its original shape, the rooms are full of old frescoes and also some old furniture had been conserved. The cellars were once used as prison, one sees clearly the rock of the island. The belfry has four floors and offers quite a view. The area around the castle is really great too, particularly during sunset. This view was featured prominently on the album cover of Queen's Made in Heaven album, the last one featuring recordings by Freddy Mercury, whose statue is in the harbour of nearby Montreux as well.\\

Chateau Chillon***

\section{January 16: Lausanne}
\label{Lausanne2011}

Another day, another outing with my brother, this time we just wanted to see a bit of Lausanne. The cathedral is an early gothic cathedral, also with pretty decent stained glass windows. The cathedral is up a hillside close to the two castles of the former archbishops of Lausanne. Nowadays protestants dominate the region, and the cathedral is not the home of a bishop anymore. The portals are decent as well. On our way back we stopped by Nyon. This little town was founded by the Romans, remains can be seen in the Roman museum as well as three columns of an ancient temple on a terrace overlooking Lake Geneva. My brother was rather disappointed.\\

Lausanne: Cathedrale**
Nyon: Roman Temple*

\section{January 23: Milano}
\label{Milano2011}

My very first time in Milan, a nice day trip ahead of us (considering it is just a 4 h train ride away). Getting on a 5:42 early train with brother, we arrived in Milan shortly after 10 (train had a bit of a delay, also happening often on this track). The old Castello Sforzesco houses many pieces of arts, among them the last big sculpture by Michelangelo, the so-called Pieta Rondanini. Certain halls of the castle have been painted by Leonardo da Vinci as well. After a walk through the old town, including a visit of the Galleria Vittorio Emanuele we visited the Duomo. Back then the cathedral was free for visits, and there was no special security check, nowadays it can be over an hour to get in just based on the very slow security checks. We also got a ticket for the roof terrace, really magnificent to see the decorations and pillars up-close as well as the (relatively small) dome and the golden statue on top of it. The square by the basilica of San Lorenzo is dominated by old roman remains, the so-called Colonne di San Lorenzo, the basilica itself is alright, but I would recommend to see the side chapel of San Aquilino with magnificent mosaics on the walls and the roof. Then we queued for the Palazzo Reale for an exhibition of Salvador Dali, but after hardly moving more than 2 m within a 45 min wait we gave up and rather had dinner somewhere else before taking the last train home shortly after 7 pm.\\

Castello Sforzesco***, Duomo***, Duomo Rooftop***, Galleria Vittorio Emanuele II**, Basilica San Lorenzo**, Sant' Ambrogio*, San Sebastiano*, Colonne di San Lorenzo*

\section{February 13--February 19: Chamonix \& Les Houches}
\label{LesHouches2011}

Shortly after the LHC had started data taking first papers were published. Their findings were the main discussion point of a winter workshop at the Les Houches physics centre in the valley of Chamonix. My colleague Andreas and I were invited to talk about our measurements of QCD processes at CMS. We decided that it would be nice to see a bit of the valley before the workshop.\\

February 13: Flegere \& Bossons\\
Although we weren't skiing on that day, we went all the way up to the Flegere cable car stop, offering magnificent views of the snow-covered mountains and the Mer de Glace as well as Mont Blanc just opposite of us. We did a small walk around the area, but naturally that's hard to do considering we didn't want to walk along snow pistes and normal hiking trails were not prepared for winter either. So pretty soon after getting up we decided that we could try a bit of a small hike by the Glacier des Bossons. That glacier still reaches almost down to the valley, thus we parked the car shortly before the finale moraines which the glacier left in the 19th century. From there it was a short 45 minute hike to reach the current tongue of the glacier. Although it retreated quite a bit already in 2011, it was still a really impressive scenery.\\

Flegere**, Glacier des Bossons**\\

February 17: Argentiere\\
Having failed at getting good views around the Grands Montets area the first time around back in 2010 (see \ref{Chamonix2010}) I made my second attempt getting all up to the mountain top by cable car. Indeed this time it was sunny and I got great views of the Glacier des Rognons and the Glacier d'Argentiere. From the middle station of Lognon, the ice-fall is the Argentiere Glacier looked pretty great too, but I didn't walk over there, going along a ski piste didn't seem like a brilliant idea after all.\\

Glacier des Rognons***, Glacier d'Argentiere***\\

\section{April 3: Jungfraujoch}
\label{Jungfrau2011}

Once again a trip to Jungfraujoch, this time all alone. This time I made sure to get off already at Eiger north face. It is quite a sight, but I think the view from bottom up to the top is far more impressive. After the usual visit of the glacier palace as well as the views of the Aletschglacier from the Sphinx platform I made my way over the paved passage to the M\"onchsjochh\"utte. The walk is pretty relaxed even considering the high altitude, besides the very last 10-20 metres which were a tad slippery. I had a nice plate plate of various sorts of cheese as lunch snack, before making my way back to the Jungfraujoch station, and then over Lauterbrunnen and the Staubachfall back all the way to Geneva.\\

Jungfraujoch***, M\"onchsjochh\"utte***, Staubachfall*

\section{April 22: Ravennaschlucht \& Freiburg}
\label{Freiburg2011}

Ravennaschlucht**, Freiburger M\"unster**

\section{April 25: Feldberg \& Feldsee}
\label{Feldsee}

Being back home in Germany over the long Easter weekend offers the opportunity to explore the close-by hills of the Black Forest. The largest mountain is the Feldberg, once home to the Feldberg glacier. Down the rock-face is the cute little lake of Feldsee, also start of the Gutach/Wutach which a few km down reaches my home village. Little pine forests around the lake, little waterfalls here and there, the outflow of the river at one of the moraines of the former Feldberg glacier, switchbacks to get up the mountains, all in all a cute afternoon hike at the start of the hiking season.\\

Feldsee***

\section{May 15: Martigny \& Vernayaz}
\label{Martigny2011}

This time my brother and I decided to spend some more time in Valais. Stopping in Martigny we opted for the Roman amphitheatre, clearly a smaller even venue compared to the big arenas everybody knows off. And on to our second stop the village of Vernayaz with the heavily advertised Gorge du Trient. The gorge is a nice slot canyon, also water levels were great. It was just a bit short, about 1 km and that was it. I don't know if it is planned to extend the pathway, the gorge itself is quite a bit longer at least. A bit disappointed we continued to walk along the village's main road until we reached the waterfall of Pissevache. Clearly visible from the train racks all year round, particularly impressive in winter when the water levels are rather large.\\

Martign:y Amphitheatre*\\
Vernayaz: Gorge du Trient*, Pissevache**\\

\section{May 21: Grottes des Vallorbe}
\label{Vallorbe2011}

My parents are visiting me and my brother. Thus we decided to plan a bit to make their stay worthwhile. We decided to meet halfway at Vallorbe. This border town is not only home to one of the several mountain forts, but it is also home to one of the largest caves in Switzerland. The Jura mountain has several big caves, though mainly on the French side. Several halls can be visited, an underground stream leads through some of the chambers, before it appears as the Orbe overground. The halls are full of stalactites and stalagmites, in a couple of rooms at the end of the way several crystals and gemstones can be admired. Back then photography was forbidden, but that has changed in the meanwhile (status 2019). Crossing the Orbe and walking up the hills there is another smaller cave -- the Grotte de Fees. One can go in at ones own risk with large flashlights, helmets are not a must, since the cave has a decent height, I thought it is still nice to do.\\

Grottes des Vallorbe***, Grotte de Fees*

\section{May 22: Lyon}
\label{Lyon2011}

After a night in Geneva we spent the second day of my parents' visit in Lyon. As one of the three largest cities in France (way behind Paris and roughly on the same scale like Marseille I believe), Lyon has a nice old town sitting between the Rhone and Saone rivers. Its roots range back into Roman times, the very well conserved Odeon and Theatre are witnesses of that time. Not much is left of the amphitheatre though. The neo-byzantine basilica of Notre Dame de Fourviere towers on a hill side over old town. You can either walk up, or take a small cable car up the hill. The view of the city from the terrace in front of the church is really nice, on good days even the Mont Blanc can be seen (it was not a good day, but pretty hazy). The church itself is full of mosaics, definitely worth a visit. While the gothic cathedral of St Jean is nice, it pales a bit after a visit of the Fourviere basilica. The astronomic clock is the highlight of the cathedral, also stained glass can be found. Then we had lunch, Lyon is famous for its good food and the mussels we had were very nice indeed. Then we strolled over nice town squares, looking at facades and courtyards of Renaissance and Baroque buildings, like the city hall, and we had a short look into two more churches, although these didn't offer something special.\\

Notre Dame de Fourviere***, Odeon \& Theatre**, Cathedrale St Jean**, St Nizier*, St Georges

\section{May 29: Grindelwald}
\label{Grindelwald2011}

After our two trips to the Zermatt the year before, my brother and I decided that we should see a bit of Grindelwald as well. We started the day taking the gondola up to First. My brother had gotten himself a Nikkon SLR and taught me how to properly use it, since I did already then consider getting myself a new camera after getting my PhD (it took me still a couple of months to get my own SLR, opting for a Canon EOS600D instead). The gondola ride offers superb views of the Upper Grindelwaldglaciers including both icefalls. Nowdays the second ice-fall has collapsed completely, leaving about 2 km of dead ice detached from the glacier. Once we were up on First we walked the nice and easy path to Bachalpsee. Since it was very windy, we had the nice panorama of the Bernese alps in front of us, albeit without their reflections in the water of the lake. 

First \& Bachalpsee***, Pfingstegg***, Kleine Scheidegg***, Staubachfall**

\section{June 3: Basel Zoo \& Bad S\"ackingen}
\label{Basel2011}

Zoo Basel***, Bad S\"ackingen: Wooden Bridge*, Fridolinsm\"unster **

\section{June 4: Mainau}
\label{Mainau2011}

Mainau***

\section{June 19: Zermatt}
\label{Zermatt2011}

Gornergrat***, Gandeggh\"utte***

\section{June 26: Trift- \& Steingletscher}
\label{Gadmen}

Trifgletscher \& Triftbr\"ucke***, Steingletscher***

\section{July 3: Aletschgletscher}
\label{Aletsch2011}

Eggishorn***, Fiescherlgletscher***, Aletschgletscher***

\section{July 31: Scharzbergkopf}
\label{Saasalmagell2011}

Scharzbergkopf***

\section{August 28: Ludwigsburg}
\label{Ludwigsburg2011}

Ludwigsburg Palace***, Favorite Palace**

\section{September 3--September 5: F\"ugen}
\label{Tirol2011}

September 3: Hintertux:\\
Hintertuxer Glacier***, Hintertuxer Falls**\\

September 4: Zell am Ziller:\\
St Maria Rast*, Gold Mine*

September 5: Innbruck:\\
Dom***, Hofkirche***, Hofburg**

\section{September 29-October 1: Franken}
\label{Franken2011}

September 29: W\"urzburg:\\
Residenz***, Neum\"unster**, Festung Marienberg*, K\"apelle***\\

September 30:\\
Bamberg: Kaiserdom***, Neue Residenz***, St Michael**, Bad Staffelstein: Vierzehnheiligen***, Banz: Monastery**, Coburg: Veste*\\

October 1:\\
Pommersfelden: Schloss Weissenstein***, W\"urzburg: Residenz***

\section{October 16: Chamonix}
\label{Chamonix2011}

Mer de Glace***, Glacier de Nantillons**, Glacier des Bossons***

\section{December 27--December 30: Vienna}
\label{Vienna2011}

December 27:\\
Schloss Sch\"onbrunn***, Stephansdom**\\

December 28:\\
Upper Belvedere***, Lower Belvedere***, Karlskirche*, Albertina**, Hofdepot*\\

December 29:\\
Hofburg**, National Library***, Augustinerkirche*, Peterskirche**, Jesuitenkirche**, Dominikanerkirche**, Greek-Orthodox Cathedral*, Maria am Gestade*, Schottenkirche*, Minoritenkirche, Parliament**, Burgtheater**

December 30:\\
Hofburg: Neue Burg*, Hofburgkapelle, Hofburg: Treasury***, Hofburg: Silverchamber, City Hall**, Votivkirche*, University, Kunsthistorisches Museum***
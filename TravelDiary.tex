\documentclass[11pt,a4paper,twoside]{book}
\usepackage{epsfig}
\usepackage{amsmath}
\usepackage{amssymb}
%\usepackage{german}
%\usepackage{umlaut}
\usepackage{graphicx}

\usepackage{colordvi}
\usepackage{multirow}
\usepackage{hhline}
\usepackage{tabularx}
\usepackage{cancel}
\usepackage{hyperref}

\pagestyle{plain}

%\usepackage{fancyhdr}
%\setlength{\headheight}{15.2pt}
%\pagestyle{fancy}


%\linespread{1.3}
%\textheight 245mm             %     Einstellungen fuer A4 Papier
%\hoffset -4mm
%\voffset -4mm

\textwidth 150mm   %
\topmargin -10mm

\oddsidemargin 6mm
\textheight 240mm
\evensidemargin 7mm
%\usepackage[top=2.5cm,bottom=2.5cm,left=2.5cm,right=2.5cm]{geometry}
%\pagestyle{plain}
\parindent 0mm

%----------------------------------------
%\fancyhead{} %erst mal leer machen
%\fancyfoot{} % -��-
%\fancyhead[OR]{\rightmark} %die Section-Name
%\fancyhead[EL]{\leftmark} % Chapter-Name


\begin{document}
\frontmatter


  \begin{titlepage}
      \setlength{\baselineskip}{8mm}
      \begin{center}
      \Huge{Travel Diary}
      \end{center}
    \vspace{1.5 cm}
      \begin{center}
        \huge{\textbf{whatever name}} 
      \end{center}
      \begin{center}
      \large{\textbf{- 2019-}}
      \end{center}
%\cleardoublepage
    \end{titlepage}

%\addtolength{\oddsidemargin}{-.875in}
%\addtolength{\evensidemargin}{-.875in}
%\oddsidemargin 10mm
%\evensidemargin 10mm

%short abstract

%\begin{abstract}
%In this thesis the measurement of hadronic event shapes in proton-proton collisions at $\sqrt{s}=7$ TeV is presented. The data sample is collected with the Compact Muon Solenoid (CMS) detector at the Large Hadron Collider (LHC) and corresponds to an integrated luminosity of 3.2 pb$^{-1}$. Jets constructed using particle flow techniques are used as input for the event shape calculation. Normalized event-shape distributions are shown to be robust against detector resolution effects and various sources of systematic uncertainty. Several event-shape distributions are compared with five models of QCD multijet production after detector simulation. For two selected event shapes - central transverse thrust and central thrust minor - the distributions are corrected for detector effects and compared to six models of QCD using the SVD regularization method. The model dependence of the unfolding technique and the error propagation are discussed in detail. Results are cross-checked using an alternative unfolding method and an alternative method of jet reconstruction.
%\end{abstract}

%long abstract for 1 1/2 pages

\frontmatter
\chapter{Preamble}

When you hear about someone's travel you often hear how awesome it was and amazing (or the opposite: everything was bad, from the hotel, the food in restaurants, overcrowded beaches etc). Unlike those stories, trips are in general a fun time to see or discover something interesting while spending time friends or family merged with unfortunate things like public transport on strike, broken down cars, getting lost, overpriced sights, and other mishaps. You won't only hear about the sunny days, beautiful mountain sceneries, deserted incredible beaches, or days without smog in the huge metropolis, but also about days getting completely soaked in pouring rain, flooded metro stations. The current version takes you to four continents (no Australia, Antarctica or South America as of yet), obviously still open for that to happen in future updates. You will get to hear a huge amount about the beauty of places and the fun you can experience in about 700 places in close to 40 countries (and still counting). What is your take on a comment along the lines of: the Vatican is not a real country? or countries you have only been at an airport don't count?\\

Now I also have a point system going on, which is of course subjective, depending on the type of visit, expectations, changes of the visits: I have been in several place more than once, but if a tour changes from over an hour to just 15 minutes going along the same route, then you know something is wrong. On the other hand there have been places where I expected a lot, was disappointed, lowered my expectations. Going in with these expectations I started to enjoy that place again. It also depends what you did the days before, standalone a place might be great, but just not as great as something else, so you maybe downvote it just by comparison, while by its own you might have been wow'ed by it. Anyways the point system:

Five Star Attractions: essentials, you should really do that\\
Four Star Attractions: great to do, but not essential\\
Three Star Attractions: in the context of the place this is the typical good bread \& butter you get here\\
Two Stars: step in if you have time, some disappointing aspects, sometimes I am also just bored since it is not my thing I enjoy to do\\
One Star: disappointing, maybe step in if you just in the area, not for spending many time there\\
Zero Star: just avoid, to get to that place, something must have gone really wrong, or it was amazingly over expensive\\


\tableofcontents

%2003 Sigi, Boeni, Gianrico
%2004 Anne
%2009 Anne-Kathrin
%2010 HJ
%2011 A & H
%2012 Brant, Dan, Jacob, Jesse
%Eric
%2013 Andrew
%Pieter
%Rosana
%Cameron, Eve
%Indara, James
%Chris L, Chris C
%Manuel, Bing
%Chris M
%2015 Siyi
%Canyou, Laser, Nate, Kennedy, Devin, Aaron
%AJ
%Line,Jonas
%Riju
%Evan
%2016 Amin
%Keyvan
%Grace & Chris S
%2017 Sarah
%Reyer
%Rachel
%Tony
%2018
%Chris B
%Sasha
%Santona
%Yao, Graham
%2018
%Grace
%Christian
%Ulrike, Francesca, Ivan,Jose
%Andres,David,Janina,Sam
%Yaroslava
%Nhi,Michael,Vivan
%Xuan, Will
%Elga,Juno,Artem


%Sigi,Boeni,Gianrico,Anne,Eric,Andrew,ChrisLee,BeardChris,Chris,Manuel,AJ,Stephan,Laser,Devin,Kennedy,Candice,Nate,Aaron,Riju,Reyer,Rachel,Evan,Siyi,Santona,Christine,Indara,Amin,Jason,ChrisS,GraceS,Pieter,Line,Jonas,Cameron,David,Will,Xuan,Janina,Sam,Michael,Vivan,Nhi,Dylan,DavidYu,Francesca,Ulrike,Artem,Sasha,Juno,Elge,ChrisB,Andres,Yaroslava,Jose,Ivan,Family

\mainmatter
\chapter{Year 1985}
\label{1985}

\section{June 26--July 9: Ibiza (Cala Azul)}
\label{1985:Ibiza}

The first bits of holidays I remember are from Ibiza. There were beaches, lots of little stones, which you could use to built ``high'' castle mountains (aka a couple of cm). My favourite toy car -- an Unimog -- lost its front tyre. I blamed that on my sister (nobody remembers if she was actually in any way responsible though). Oh and if you should be looking at the year itself, indeed I am not the youngest spring chicken in the world anymore.

\chapter{Year 1987}
\label{1987}

\section{Summer: Mallorca (Calas de Mallorca)}
\label{1987:Mallorca}

Another summer outing to Spain, this time with a DC10 from Zurich. My parents though a king size bed could be enough space for the kids to be happy. Unfortunately for them their three kids who were supposed to share the bed didn't think so, thus continuously hitting and beating below the pillows. As consequence we had to get into bed in shifts. The beach was quite some distance away, so we had to take the bus to get there. The hotel pool was nice at least for us kids. My dad and I also checked out a larger pool which was close-by. Clearly targeting families there was also a kid's disco, where we spent out evenings at. We also collected shells by the beach. We also went to a little bird zoo. The highlight was the Cuevas del Drach (at least for me as kid). The cave has a lot of Stalactites and Stalagmites all over, and an underground lake where boats are running on surrounded by classical music. Little me was impressed. By the way should you miss the year 1986, I have no memory or idea what was going on in that year, for sure we went somewhere doing something but even I can forget things.

Cuevas del Drach*****, Bird Zoo***

\section{October 24-31: Flumserberg}
\label{1987:Flums}

There was a time, when Switzerland was not that amazingly more expensive than the rest of Europe. In those times my parents decided it would be nice to spend some late Autumn days in the alps around Flumserberg. I don't have that many memories of the trip besides the following: Almost every morning the valley was covered in fog, and we had clear skies up over all of this soup of condensed water. Then my parents took us to the Tamina Gorge - or well, that was the actual plan. But seems due to heavy rainfalls the days before, the gorge had already closed its gates. Thus we had to turn back, but collected chestnuts and Slate tablets on our way back, which were then put together into animals to play with, or drew on it with chalk. We also looked for knights in the castle of Sargans. Unfortunately there weren't that many of them around anymore. We also hiked up to the Heidi Alp, in fact the Heidi movies had been shot there, and the place had inspired the famous Japanese Heidi anime as well. Besides the hike we kids were more interested into the helicopter transporting trees down the valley. We also took a gondola up the mountains, where we did long hikes (I am sure they were actually not that long, but for the kid of me they seemed long). \\

Heidialp hike****, Sargens Castle***

\chapter{Year 1988}
\label{1988}

%horn, July 
%aachquelle, hohentwiel, lake constance, unteruhldingen

\section{October 24-October 29: Beatenberg}
\label{1988:Beatenberg}

%truemmelbach, Grindelwaldgltscher, Aareschlucht
Another year, another holiday spent in Switzerland, this time in Beatenberg which is a village overlooking a cliff of Lake Thun. Being opposite of Jungfrau you have quite a view of the Berner Dreigestirn -- Eiger, M"onch, and Jungfrau. Now there is also a chair lift going up the close-by mountain, the Niederhorn, which we had to see too. Most of the days we spent in the indoor swimming pool or in the kindergarden which was part of the hotel complex. We also visited the waterfalls of Tr\"ummelbach. I remember more of the lift through the mountain than the actual waterfalls though. We also visited the glacier cave of the Upper Grindelwaldglacier. For me it was very impressive to talk through the ice to reach the cave. 1988 is roughly the time when the glacier reached one of its maxima over the last 500 years, it was very impressive to see. Unfortunately by now the glacier tongue retreaded quite substantially by a couple of kilometres, it cannot be seen from the valley anymore and it is so far less accessible that people gave up on carving a glacier cave over a decade ago. Even climbing up several ladders and crossing the glacier gorge doesn't even get you to the dead ice formed by the nowadays detached lower tongue below the once existing second lower icefall of the glacier.\\

Tr\"ummelbach Waterfalls*****, Upper Grindelwaldglacier Cave*****


 %1985,1987,1988 done
\chapter{Year 1989}
\label{1989}

\section{May: Beatenberg}
\label{1989:Beatenberg}

%ballenberg here

Another morning we went to the Beatus caves, one of the larger caves in Switzerland with an entrance just next to a waterfall. I remember we did a longer hike, and we walked over meadows with lots of cows on. My mum reminded us that we should stay away from cow poop, but she used the dialect word (Kuhdeitsche), which I never had heard of. So I asked her what that stuff is, notably just stepping into the cow shit at that very same moment. Clearly mum wasn't too happy, even my younger brother who was just 2 years old at that point can tell that story. Anyways the hike went on for longer and longer and longer, thus my parents decided to cut it short and take the bus back at the first occasion without getting to the place which was originally foreseen as destination\\

Beatush\"ohle**** and close-by waterfall****

\section{June: Horn}
\label{1989:Horn}

What can you do with young kids: what about camping by a lake. My dad, my older brother and I built the tent (I doubt I was really that much of help). Clearly it was a nice setting with the camping ground distributed throughout a forest area, where we could play or let snails run against each other. On the other side the camping ground was very close to the lake with a sand beach. We did also go to Reichenau by boat, where we also the churches. I remember I was particularly impressed how large the Minster was (in fact it is not all that large, but having had not seen any large church at this point, I didn't know that). We also went to Meersburg to see the Zeppelin museum with many models and also burnt remains of the LZ129 Hindenburg. Clearly I became a fan of air ships and Zeppelins in particular for a while. We also visited the Old Castle (Altes Schloss), which has some of the oldest and mightiest walls from medieval times in Germany. And last but not least we visited Hohenzollern, the original family seat of the former Imperial family. In fact most of the castle is a neogothic interpretation of the old castle from the 19th century, when the castle was rebuilt after centuries of decay. Only one chapel is in fact from the medieval era. I liked the slippers they gave us which were fun to slide over the floors.

Reichenau: churches****\\
Meersburg: Zeppelinmuseum****, Altes Schloss***\\
Hechingen: Hohenzollern Castle***

%missen willams November  --> Wanderung lost in woods, Nebelhorn
\section{November: Missen Willams}
\label{1989:MissenWillams}

In 1989 we even did a third little holiday, this time in the Allg\"au region in the German state of Bavaria in a place which had two floors, the upper room just below the roof was my and my older brother's room. We did mainly walks and hikes in the forest. On one of those hikes we messed up and took the wrong turn at one point. Realising that the hike didn't lead to our originally foreseen place, my parents decided to cross the meadows and just to try to get down to the valley again, and clearly we made it back to the hotel after that detour. The highlight of the trip was our cable car ride up the Nebelhorn by Oberstdorf where we enjoyed the mountain panorama and got some Schupfnudeln.\\

Oberstdorf: Nebelhorn**** %missing
\chapter{Year 1990}
\label{1990}

\section{May 25--June 8: Ibiza}
\label{1990:Ibiza}

What I didn't know back then: This marked the last flight for a while, taking a Belair DC9 to get to Ibiza and back. This time we stayed in San Antonio. I don't remember that much of the trips, but we were part of the daily morning jogging animation, as well as hanging around the pool. There was a daily kid's disco with some games being played (which was one by my little brother). Then we also participated in a treasure hunt where we had to split into teams. This time my sister's team won. She wasn't that happy that her team won biscuits while we others (as loosers) won bonbons. I was responsible for watering the plants on the roof terrace and the balcony. We also went to church once, I was surprised how the Spanish children were allowed to run around without anybody telling them to sit down and behave. Clearly there was a lot more of singing going on that in Germany, not that I understood any of it. This time the beaches were small but nice (and also close) with lots of seashells, and also fish swimming around. At some point we had dinner somewhere in the middle of the island - for sure chicken was part of the menu. I also wrote my first postcard to my godmother, going for a postcard depicting fireworks over the citadel of Ibiza City. Well things one does, once one learnt how to read and write (in some countries kid's start school kinda late). Ibiza City was also the only major tourist sight we did, climbing up all the way to the top of the citadel and the cathedral. From there we had a superb view of the harbour and the huge car ferry, at least I thought back then it was huge. For once my older brother and I had our own room with a bathroom attached (yes we were the big kids).\\

Ibiza City: Citadel \& Cathedral***

%August: Habsburg, Wildegg (zu)
%August: Einsiedeln

\section{October 27--November 3: Lechbruck}
\label{1990: Lechbruck}

%28 Lechsee
%29 Hohenschwangau, Neuschwanstein
%30 Patrozinium Lechbruck
%31 Wieskirche (renovation)
%1 Auerberg (moving clouds, church), Bernbeuren (church)

The first of two visits in Lechbruck which is home of a holiday village where each family can rent a small house. It is close to a natural small pond in a forest and only a short walk away is the local indoor swimming pool with an attached hot pool outdoors (don't know if this is from a real thermal spring though). The town of Lechbruck is next to the reservoir of Lechsee, which is nice for some walks. Close by is the Auerberg with a pilgrimage church on top. I enjoyed particularly the fact that one could just lie in the grass on the mountain top and see how the clouds move just by looking up the church tower. Then we visited the Wieskirche, but I only remember all the scaffolding of a large renovation campaign at that time and not much more. A highlight of the trip was a visit of the Hohenschwangau and Neuschwanstein castles. I remember I even enjoyed the visit of Hohenschwangau more, feeling all the history of knights around me (in fact the castle is though a neogothic rebuilt of a much older medieval castle).\\

Schwangau: Hohenschwangau*****, Neuschwanstein****
Lechbruck: Lechsee***
Steingaden: Wiekirche (in renovation)**
Bernbeuren: Auerberg Hike \& Auerberg Pilgrimage Church**** %done
\chapter{Year 1991}
\label{1991}

\section{May18--June 1; Kell am See}
\label{1991:Kell}

%21 Porta Nigra, Kaiserthermen, Aula, Dom, Matthias
%23 Flugzeugmuseum Hermeskeil --Concorde
%24 Luxembourg 
%26 Saarburg: old castle, waterfall
%28 hike to Eltz and Cochem
%31 ruwerquelle (second attempt)
This summer holiday brought us to a holiday village by Kell am See in the Hunsbr\"uck mountains. We played also a couple of times with dutch kids next door, although we didn't understand all which they tried to tell us.\\
We did also plenty of trips to nearby Trier with its famous Roman gateway, the Porta Nigra, and other ruins such as the impressive Imperial Baths or the former audience chamber of Emperor Contantine. We also saw other medieval churches such as the cathedral and the basilica of apostle Matthias. In Luxembourg we walked along the city walls, which rather looked like an old ruined castle to me. We did also more castles, such as Burg Eltz, famous since it featured prominently on Germany's 500 DM bank note. I remember that we had to walk quite a bit from parking to reach the castle through forests and then we took part of the tour. Later on we stopped on one day as well by the former Imperial castle of Cochem. We did also two hikes to reach the sspring of the Ruwer river. On our first attempt we took a wrong turn and then ended up doing a forest lake walk instead, which was still pretty enough. On our second attempt we made it to the spring which was in summer rather a small creek than a river. A further trip lead us to the town of Saarburg where we walked up on the castle hill, and enjoyed view of the waterfall which sits in the middle of town.\\

Trier: Porta Nigra*****, Constantine Basilica***, Imperial Baths****, Dom*****, Matthiasbasilika****\\
Luxemburg: Fortifications***, Cathedral**\\
Wierschem: Burg Eltz*****\\

\section{August 3--August 13: \"Uberlingen}
\label{1991:Uberlingen}

And we had another camping holiday, this time with the extended family including my aunt, two uncles, three cousins, and my grand parents. In \"Uberlingen we had a couple of walks through the parks of the town and the minster, then we did walk through the reconstructed pile dwellings of Unteruhldingen, had a tour of the Old Castle in Meersburg, one of the oldest still inhabited buildings in the country with walls which are more than one m thick on average. Last but not least we took a ferry across the lake to reach the gardens and parks of the Mainau island.\\

Meersburg: Old Castle****\\
Ueberlingen: M\"unster***\\
Unteruhldingen: Pile Dwelling*****\\
 Mainau: Gardens***** %missing
\chapter{Year 1992}
\label{1992}

\section{July: Daun}
\label{1992:Daun}

My parents decided to go for our family trip to a place in Germany once again, this time opting for a place in Rhineland-Palatine with the option to see also a big of the northern Rhineland. \\ 

The town where we stayed at was Daun, a town with an old castle and three close-by Maars, lakes which are remains of a volcanic steam explosions. In fact the local german word Maar were used to scientifically name those type of lakes.. 

%Weinfeldener Maar, Totenmaar, Schalkenmeerer Maar
%7.7 Maria Laach
%8.7. Koblenz, Ehrenbreitstein
%Strohn: Lavabombe, ROemergrag
%koeln: Dom, Aachen: Dom
%Manderscheid (Burgen), Saegewerg
%Kasselburg
%wWallenderborn




\section{October 25--November 1: Lechbruck}
\label{1992:Lechbruck}

Since we loved it in Lechbruck two years ago this town was chosen again as late autumn destination.\\

Once more we all enjoyed the large indoor pool with the adjacent hot outdoor pool - still fascinating to watch the steam over the pool. Other than that the church of Lechbruck received a couple of new bells which were lifted up to the tower on this weekend with a large celebration going with it. This time the renovation of Wieskirche had been finished, thus at least something could be seen of the church. We once again visited one of Ludwig's II castles, this time the beautiful little palace of Linderhof, which is in fact the only of his large projects which was finished. On our way back we stopped by the close-by Baroque monastery of Ettal. And I bought my first guidebook about castles myself, outlining all realised project and plans the king had in mind concerning castles, theatres etc. Naturally we also did a little hike, once again going up the Auerberg. And we visited F\"ussen with its castle and churches. I thought back then the church of St Mang was really big, little did I know how large churches can be at this point. Don't get me wrong, St Mang is indeed a fine church but when I visited about 20 years later it was just a normal sized Baroque Bavarian church. Anyways having read up a bit on the projects of Ludwig II., I convinced my parents that it would be nice to hike up the ruin of Falkenstein by Pfronten, the place where the king started his last castle project. Besides building a road and some foundations for a future castle nothing more had been finished. The ruin itself is one of the highest located in Germany, so it was quite a substantial hike for us kids, but I enjoyed it.\\

Lechbruck: Church\\
Ettal: Monastery \& Linderhof\\
Auerberg\\
Steingaden: Wieskirche\\
F\"ussen: St Mang, Hohes Schloss (Castle), Heilig-Geist-Kirche, Franziskanerkirche\\
Pfronten: Burg Falkenstein*\\


\chapter{Year 1993}
\label{1993}

\section{October 30--November 6: Telfs}
\label{1993:Telfs}

The house was really nice, my brother and I had the basement all for us. In the morning the Hohe Munde mountain greeted us through a large window. The pool was decent enough (with a hair dryer that only blew a bit of warm air out, but didn't dry in any useful way).
The highlight was the visit of Zugspitze from the Austrian side via a cable car from Ehrwald. From there we could also walk over to the German side (after all the mountain is the tallest one in Germany). I also realised i would need stronger glasses once i ran into a glass door not realising it was shut. On the way back we stopped by the Fernstein mountain pass, walking around the lake and visiting the Sigmundsburg on the peninsula.\\
By Stams is a large Baroque monastery with a giant gilded wood carved main altar. Close-by is an old suspension bridge leading over the green waters of the Inn and a nice waterfall.\\
We also visited Innsbruck with its beautiful Baroque cathedral (surprisingly there are many Baroque style cathedrals in Austria unlike in Germany, where gothic style is dominating). Interesting to visit is also Hofkirche, where the emperor had a tomb constructed with life size bronze statues of ancestors and family members (as far as I know his body isn't in the tomb though). The Imperial palace of Hofburg was largely remodelled in Baroque times for a royal wedding, the large Riesensaal (Hall of Giants) has huge frescoes and paintings on the ceiling and the walls. We saw more churches in the Wilten suburb, had more little hikes to old ruined castles, and did tour a silver mine by Schwaz, visited then the parish church and had fish nuggets and fries.
Not to forget: since everybody could choose something to do, my older brother wanted to go to McDonalds. Thus this was my first time I had McDonalds food, he didn't know that in a couple of future trips that would be my Mum's default to go to if she was very hungry or needed a large drink.

%Hohe Munde, 30 pfarrkirche
%31 Zugspitze, Fernstein Sigmundsburg
%1 Stams, zugbruecke
%2 Insbruck Dom Hofburg, Hofkirche
%3 Burg Hoertenberg
%4 Schwaz, Wilten, McDonalds
%5 Telfs Kirchen

\chapter{Year 1994}
\label{1994}

\section{July 9--July 23: Westerwaldtreff in Oberlahr}
\label{1994:Westerwald}

In Daun we had close encounters with ants, this time heavy rain fall had flooded parts along the Rhine and we had to go on a daily hunt for flies, trust me I got really good at catching them after this trip. Our house had two floors, once again my brother and I occupied the upper floor.\\



%Limburg: Dom, Dietkirchen Lubentius
%Burg Runkel
%Montabaur: Schloss
%Bonn Haus der Geschichte, Beethoven
%Drachenfels
%Limes bei Rheinbrohl
%Linz, Burg Rathaus
%Grube Georg

\section{October-November: Telfs}
\label{1994:Telfs}

Once again we drove to Telf and Austria up the little house with the two floors, and the little tiny indoor pool. One of our visits lead us to the beautiful Renaissance castle of Ambras which is just on the suburbs of Innsbruck by a large forrest. The Knight's hall (Rittersaal) has many paintings of the Habsburg ancestors all over the place, as well as a beautiful courtyard in the main castle with an armoury. We visited once again an old mining town, this time the town of Hall. We walked up a hill to get to the church of Maria Locherboden, built in gothic revival style. The alps are a good place to find gorges with little waterfalls, also in Tirol we hiked through a slot canyon and enjoyed the little ladders and wooden paths over the rapid waters and the waterfalls between the rock walls. This time we took a cable car up the Karwendel massif. We even had first snow up there, which I was particularly excited about. On one of the last days we spent hours in the fortress of Kufstein just by the border of Austria and Bavaria.\\

Innsbruck: Schloss Ambras\\
Hall\\
Maria Locherboden\\
Karwendel Mountain peak\\
Fortress Kufstein\\

%Oct-Nov Telfs
%Schloss Ambras
%Hall
% Wallfahrtskirche Maria Locherboden
%Imst, Rosengartenschlucht
%Festung Kufstein
%Karwendel
\chapter{Year 1995}
\label{1995}

\section{August: Seepark Kirchheim}
\label{1995:Kirchheim}

The last trip I did without having a camera myself. We stayed at the Seepark Kirchheim resort, a holiday village next to a lake in Kirchheim. Clearly having a lake is very nice for swimming in hot summer days. Kirchheim is very close to Bad Hersfeld which is famous for its large ruined Romanesque abbey, one of the largest in Germany. But it is also very close to Eastern Germany. After the fall of the wall and later reunification this was my first time in this part of Germany. Many infrastructure works had been going on at that time, but old cobble stone was still the main road surface in the inner parts of the towns. We did see the Wartburg castle in Eisenach and the impressive cathedral in Erfurt. In Weimar we only walked around and saw Goethe's and Schiller's houses, but I was not too impressed at that point. I did enjoy a trip to Fulda with its large Baroque cathedral. In Frankfurt I was particularly impressed by its skyline. Not many towns in Europe had proper skyscrapers at that point, and the banking district of Frankfurt was very unique in that send. I was a bit disappointed by Paulskirche, it is more famous though as the meeting place of Germany's first elected parliament back in 1848. The Dom in Frankfurt has in fact never been the seat of a bishop, but it has been the coronation church of the German emperors several times,\\\

Erfurt: Dom*****\\
Weimar: Old Town with G\"othe and Schiller House**
Eisenach: Wartburg*****\\
Bad Hersfeld: Abbey****\\
Fulda: Dom*****\\
Frankfurt: Dom***, Paulskirche**, Flughafen (Plane Spotting)****%done
\chapter{Year 1996}
\label{1996}

For Christmas 1995 I got my first camera, an analogue one, which was then with me on all big and little trips until about 2006. Photos are an excellent way to keep memories alive thus the following chapters will continue to be more and more elaborate (naturally being older helps quite a bit too).

\section{Beatenberg}
\label{1996:Beatenberg}

Every couple of years the orchestra of my village does a little trip. The one of this year was an outing to the Swiss alps by Beatenberg. Starting with a boat ride over lake Thun we arrived about lunch time at the resort. My dad, my sister, my aunt and other people decided to do a bit of hiking. While I enjoyed it my aunt went out of steam shortly before the midstation, thus we decided to call it a day and get down with the chair lift again. Meeting my uncles and cousins we had a couple of drinks and cake in a closeby cafe. On the second day we went to the museum village of Ballenberg. There old houses from everywhere in Switzerland (like old schools, old mills, old bakeries etc) have been transferred and rebuilt to give you a flavour of old traditional Swiss life. \\\

Niederhorn **, Ballenberg**

\section{August: London}
\label{1996:London}

%Westminster Cathedral & Abbey
%Hampton Court
%Tower
%Westminster Abbey
%Greenwich
%St Paul's
%British Museum

This was the first flight in many many years taking a BAE flight from Zurich to London Stanstead. Plan was to take the Stanstead express to Liverpool station and then distribute ourselves over taxis. But since we were one person too much I had to do my first underground ride on my own not that this was anything scary for me. Since I didn't keep track of dates back then and analogue photos don't include useful EXIF data in this respect I don't know how to distribute certain events to certain dates, the order is correct though. Since the Hotel President is by Russel Square the first place to visit was the British Museum, starting out with the East Asian section. On the second day we visited Westminster Abbey. I was very surprised how white the church was, all pictures I had seen of it showed it in black, but it seemed just on that year they finally cleaned all the stones from the dirt of many decades. Notable difference to nowadays, the front section could be visited free of charge. Then we went to the Westminster Cathedral. Back then I was rather impressed by this church (that changed by now), also with the great fact that you could take the elevator. Early morning next day we queued up at Madame Tussauds. Even in the mid 90s the queues were really large (keep in mind online tickets hadn't been a thing back then). While we queued another family came up to us, asking if we present to be a group (aka 10 people and more) and just stand in the much shorter tour group queue. And that's exactly what we did and less than 5 mins later we were in. I still don't see why people are so amazed by wax figures, but at least I did it once. We also saw Trafalgar Square with St Martin in the fields and the National Gallery, which didn't leave any impression on me back then. Next morning I had to realise how huge the crowd was to witness the Changing Guards by Buckingham Palace, not that I got why that was supposed to be amazing either. Since we had found out how expensive it was to buy tickets for Musicals back in Germany my parents decided to try to get cheap last minute tickets in London. Our preferred choice of Les Miserables was not available, so we decided to go to Starlight Express instead. And seems we got front row in the upper section. Quite an experience to see people passing you on high speed while on roller skates. Since I was already back then fascinated by palaces I took the whole family over to Hampton Court Palace. I was clearly impressed by both the gothic and the baroque section. My siblings enjoyed at least the kitchen section and the park. In the Tower the crown jewels were amazing, but I had expected more of the rest. Then we went to the London Monument (my sister even went up), and then it was shopping time in Harrods and Selfridges, not my kind of thing necessarily, but my sister got a dress out and I got myself new running shoes which I used for almost 10 years until they broke down. Then my dad and I went to the back side of Westminster Abbey with all the royal tombs, the only part you had to pay for in 96. The second notable difference was that back in 1996 on Wednesday afternoon for 2 GBP photography was allowed in Westminster Abbey. Thus I do have in fact a couple of pictures of the inside, unfortunately 1996 analogue camera quality but better than nothing. On Thursday we took the regional train to Greenwich. There we visited the National Maritime Museum, the kinda boring Queen's House and then walked up to the old Observatory. Standing on the zero meridian was nothing special. I was more interested in the Thames Barrier (and the shopping carts in the Thames close to it). During the afternoons we spend quite a bit in the British Museum, I enjoyed particularly the Egyptian and Assyrian artefacts. It was one of the first years Buckingham Palace opened its doors to the public in Summer. Photography of the interior is even up to now not allowed. but back then you were allowed to have a view of the courtyard, which is not possible anymore. Still I was impressed. The last day we spent in St Paul's where photography was in fact allowed back then. Once again that policy has changed and by now it is not allowed anymore (similarly to Westminster Abbey). First they claimed it on the bullshit excuse of protecting the art, then they went on to say it would disturb the guards, at the moment the official excuse of a reason is that it disturbs the spiritual experience (of course unlike paying the almost 20 GBP entrance fee). Anyways the church itself is great (still second to Westminster Abbey), but the view of the City is the best.\\

Hampton Court Palace***, Tower of London**, Greenwich*, Buckingham Palace***, Star Light Express***, Westminster Abbey***, Westminster Cathedral***, St Paul's Cathedral***, British Museum***, Madame Tussauds*, National Gallery*

\section{Hartmannswillerkopf}
\label{1996:Hartmannswillerkopf}

In World War I, the Alsace region had been a major point of operations. At Hartmannswillerkopf is one of the largest soldier cemetery of this particular battle, together with a brutalist church and crypt. The remains of the ditches and fortifications are slowly claimed back by the forests, but it was quite solemn and touching if you think back then what had happened and how many young men lost their lives in a rather pointless war, which might have been avoidable if politics back then would have known a bit more of diplomacy.\\

Hartmannwillerkopf***

\section{K\"ussaburg}
\label{1996:Kuessaburg}

Quite close to my home village is the ruined castle of K\"ussaberg. Overlooking an old road which had been used already in Roman times for trading, the castle dominates the nearby mountains. Destroyed and given up centuries ago, almost everything of the inner decoration is gone, but some of the enclosure is still there, as well as some ruined walls of the palace or the fortified outer towers. One of the former towers can be climbed on, as well as the gate fortifications.\\

Hartmannwillerkopf***

\section{Hohentwiel}
\label{1996:Hohentwiel}

Hohentwiel is an extinct volcano (actually the core of the Volcano is the only part which remains after all those years of erosion) close to lake constance by the town of Singen. Originally once one of the largest fortresses in Europe, it was left in ruins by French soldiers. There is a lower and an upper Fort, the ruins still give you an impression how mighty it originally was. I did enjoy the ruins of the church and the underground layers the most. For sure a good trip for a summer family outing. The ramps between the lower and the upper Fort are a tad steep, maybe also slippery in rain.\\

Hohentwiel***

\section{Unteruhldingen}
\label{1996:Unteruhldingen}

Along the shores of Lake Constance many remains of old pile dwellings can be found. One of the largest archeological sites is by Unteruhldingen, where the artefacts have been placed in the local museum. To give visitors an idea about the original state and life in pile dwellings a whole village has been reconstructed. I always enjoy to walk along, look into the houses, where locals also set up re-enaction of life back then based on the latest historical evidences.\\

Pile Dwellings**


\section{October: Stromberg}
\label{1996:Stromberg}

For the fall holidays we stayed in Germany, in Stromberg. On our trip to the holiday resort we stopped by the romanesque cathedral of Speyer. Partially destroyed by the French in the 17th century the eastern part is still mainly the original from the 11th century, including the imperial tombs in the crypt of the Salian dynasty. This is one of the three imperial cathedral of the Rhineland which we visited on this trip. Arriving, we did a small hike along a small riverbed ending up by the castle of Stromburg, which had been partially converted into a fancy restaurant. On the second day we went to Bingen, a nice little town with an old castle in the middle of the city. In the Rhine is another little tower and this one marks the start of the Rhine Gorge. In Heidelberg is the old Castle, which was destroyed in the nine year war by the French. The castle was a mixture of Gothic and mainly Renaissance buildings. Most of the towers have been split in two by explosions. My sister tried her best to photobomb my pictures too. The castle is one of the most famous romanticised castles in Germany, only a very little part has been rebuilt. The castle is also home of one of the largest vine tuns with a capacity of 219 thousand litres. On the third day we visited the city of Worms, home to the second imperial cathedral. Unlike in Speyer the cathedral had not seen such turbulent times, the baroque high alter and choir decoration survived all these years. The interior is though a lot darker and feels more like being in a castle. Afterwards we saw the old synagogue and the adjacent rooms depicting the history of the jewish community in Worms. The last of the three imperial cathedrals is the one in Mainz, which we saw the day after. In Mainz Johannes Gutenberg invented the print press, which still can be seen in a specialised museum along with one of the printed bibles. in a separate rooms we kids were allowed to try to print out own text.\\ 

Speyer: Dom***\\
Stromberg: Stromburg\\
Bingen: Old Town \& M\"auseturm**\\
Heidelberg: Schloss***\\
Worms: Dom***, Synagoge**\
Mainz: Dom**, Gutenbergmuseum*%done
\chapter{Year 1997}
\label{1997}

\section{May 21--May 25: M\"unchen}
\label{1997:Muenchen}

%22 Glockenspiel, Asamkirche, Frauenkirc he
%23: Nymphenburg, Olympiapark
%24: Deutsches Museum
%25 Dachau

Nymphenburg with Park Palaces*****,Theatinerkiche****, Michaelskirche**** ,Asamkirche**** , Concentration Camp Dachau**** , Olympiapark with Olympia Tower****, Deutsches Museum*****

\section{August 24--August 31: Rom}
\label{1997:Rom}

After we did a big family trip to London the year before, my mum decided it was time to take on Rome. Considering my interest for anything in old architecture my aunt got me a book about Rome as present. I had a very old second book which originally belonged to my grandmother as second source of information. Once we arrived in Rome we took a taxi to our hotel which was pretty close to Termini. The is one thing I remember from the hotel: it had a pretty old elevator, which you could stop during the ride by slightly opening the doors, and getting it to continue via closing it again (besides the one time, when it got stuck - well teenage years I guess). \\

August 24: Saturday:\\
The first church I ever had seen in Rome was Santa Maria Maggiore, one of the big four churches. The nave is decorated with ancient mosaics, the Sistine and the Borghese Chapels are and will certainly remain my favourites. After a short stop by the Trevi fountain the day already came to an end in the church of San Ignacio, which is home to one of the largest ceiling frescoes in the world.\\

Santa Maria Maggiore***** , Trevi Fountain****, San Ignazio*****\\

August 25: Sunday:\\
On the last Sunday of the months the Forum Romanum and the Palatine hill can be visited for free. I had a board game about the Forum Romanum at home, but clearly the forum had the best times behind it. The palatine hill was nice too. We only saw the outside of the Colosseum. St Paul's is another one of the four big churches, though in my opinion the least impressive. Most of the church burnt down in the 19th century.\\

Colosseum***, Forum Romanum \& Palatine****, San Paolo fuori le Mura****, San Pietro in Vincoli****\\

%26: Il Gesu, Santa Maria Sopra Minerva, Pantheon, Piazza Navona, Diokletians, Santa Maria d'Aracoeli, Himmelsleiter

August 27: St Peter's Basilica:\\
In the morning I remember that big news hit the eternal city that Princess Diana had passed away in a car accident. People were discussing it all around, while we were queuing for the Vatican Museum (free entrance day). I was very impressed by the rooms and halls, particularly the Stanze di Raffaelo and the Sistine Chapel. I also enjoyed the large library. Nowadays only the entrance of the library can be visited, the rest of the hall is off limits for the everyday tourist. Other than that the Apollo del Belvedere, the Laokoon as well as the Pinacotheca never fail to be exciting. Although it was a very rainy day we climbed the dome of St Peter's (not really worth it due to the bad conditions), and then we visited the interior of the gigantic basilica. Nowadays most of the original paintings of all the altars have been replaced by mosaics. The many monuments and tombs of the popes are clear highlights, as well as Michelangelo's Pieta.\\  

Petersdom*****\\

%28 Laterno, San CLemente

%29 Ostia Antica

%30 Catacombs

%31 Vatican Museum, Sapnish Stairs

Ostia Antica**** , Baths of Caracalla*****, Basilica Laterano***** , San Sebastian Catacombs***,  Il Gesu*****, Pantheon*****, Santa Maria degli Angeli****, San Andrea al Quirinale****
\chapter{Year 1998}
\label{1998}

\section{April 17--April 18: Ulm}
\label{1998:Ulm}

%17 Ulm Rundgang
%18 Blaubeuren, Wiblingen
April 17: Ulm:
Another family trip, but this time my older brother decided not to join us on this weekend. It lead us to the town of Ulm, which is famous for its M\"unster. This protestant church still has the tallest church tower in the world, although in a few years Sagrada Familia in Barcelona will surpass it. The minster is a giant gothic hall church, with typical medieval sculptures and altar pieces. We walked across old town with plenty of medieval timber houses, but also modern churches, but well a usual medieval town. In the evening we went to the university area with its sculpture trail\\

Ulm: M\"unster*****, Kornhaus***, Rathaus***, Schw\"orhaus***, Schiefes Haus**, Reichenauer Hof**, Einsteinbrunnen*,* St Jakob***, Garnerkirche****\\

April 18: Blaubeuren \& Wilblingen
On the second day we started with a visit of the medieval gothic style Blaubeuren Abbey which sits next to the crystal clear blue water of Blautopf, which is a Karst well connected to one of Germany's largest but also most dangerous underwater cave. Since one monastery is not sufficient enough we continued our day at the monastery of Wiblingen with its spectacular late Baroque library (rivalling the one of St Gallen in beauty), and its classicist church. Last but not least we climbed the tower of Ulm minster for some city views.\\

Ulm: M\"unster tower****\\
Blaubeuren: Blautopf***, Abtei****\\
Wiblingen: Klosterkirche****, Klosterbibliothek*****

\section{August 22--August 30: Paris}
\label{1998:Paris}

The yearly city trip of this year heads for Paris. Since Air France offers particularly cheap tickets for flights within France, we flew out of Basel-Mulhouse, by the way the one and only flight to Paris with the destination of Paris itself. The RER from Charles-de-Gaulle just stops at the Garde du Nord, where we had our hotel.\\

August 22: The Tour d'Eiffel:\\
Where would you go on your first day in Paris. My parents decided it should be the Eiffel Tower. Instead of taking the elevator up we took the stairs. These offer more detailed views of the structure of the pilons, it is a lot cheaper, and even faster during summer, when the queues for the elevators are quite sizeable. We enjoyed the views from the first two floors. The third floor can be only visited using an elevator.\\

Paris: Eiffel Tower*****\\

August 23: The Louvre\\
The Louvre: one of the oldest museums of the world, and one of the largest if not the largest of the world. Not convinced yet -- free for everyone up to and including age 18, even free up to and including age 25 should you be a resident or citizen of the European Union (Switzerland seems to work too). We first started with a visit by Mona Lisa, back then not that many people gathered around the painting, it was also behind a secured glass which was tinted a bit on the purple side. I was not impressed. On my later visit the presentation was largely improved, at the expense of more and more visitors heading for the painting, but still it leaves me a bit underwhelmed. I heard in 2019 the outfit of the room was once again changed though. I like old paintings, but after about an hour I get usually bored a bit by those. But the Louvre has much more to offer than just old paintings: the sculptures are amazing, particularly the ancient statues like the Venus de Milo, but also more classical pieces, like Michelangelo's slaves or Canova's Psyche and Amor. And all the beautifully decorated rooms. A couple even with the original ceilings, like the former Royal Bedchamber or the presence chamber of the queens.And as pinnacle of Baroque decorations -- the gallery of Apollo, one of the predecessors of the Hall of Mirrors in Versailles. Other ceilings have been remodelled later, but still with impressive carvings and frescoes, some by none other than Delacroix -- thus really amazing as well. During the second empire, a wing of the Louvre was used as meeting room for the Ministers, these so-called Napoleon III apartments with their gilded walls never fail to impress me. A couple of pieces from the former royal palace of the Tuileries are on display, notable the bedchamber and the throne. The British Museum has the Rosetta Stone, the Louvre has the Codex of Hammurabi, the eldest surviving scripture we know off. Should you care about ancient art, the exhibition of the old Assyrian palace walls are amazing, or the big top of a pillar in the audience hall of the imperial Persian palace of Persepolis. Naturally Egyptian mummies, statues and sculptures can be seen as well. Considering that I got into the museum for free I spend a couple of more afternoons and evenings in the museum in the next couple of days.\\

Paris: The Louvre*****\\

August 24: Ile de Cite:\\
The Ile de Cite is the larger of the two islands of old town Paris, the old centre of power with the remains of the old medieval royal palace -- Conciergerie and Sainte Chapelle, as well as the secular centre piece of Paris -- the Notre Dame de Paris. The guard hall of the former palace is one of the largest conserved gothic halls in France, and impressive considering that on top of it used to be the actual Great Hall of the Palace. A couple of windows of St Chapelle had been in renovation, but what could be seen was still very impressive. Being an early gothic cathedral I remember I thought that Notre Dame was quite dark for a gothic church, which I had come to know as particularly bright and illuminated. The Rose windows of Notre Dame have slighter darker colours than most other cathedrals I have been at as well. Oh and I remember the that all of Notre Dames photos got out in shiny red -- unfortunately there was no photoshop available to save these negatives and consequently analogue photos, the scanned versions well those are photoshopped to far better quality nowadays.\\ 

Paris: Notre Dame*****,Sainte Chapelle*****, Conciergerie***\\

August 25: Versailles:\\
Three former royal and imperial palaces are close to Paris: Versailles, Fontainebleau, and Compiegne. Clearly the most popular out of those is Versailles. Nowadays crowded to the maximum, particularly on weekends, when the fountains are running for about 2 hours during the day, come early or get stuck. Back in 1998 things were peaceful, the chateau was popular, but on a reasonable scale. Having read about the palace I was convinced I would love it, and indeed it was love at first sight, particularly the Hall of Mirrors and the chapel. The gardens are so large, that one needs more than a day to see all of it, I have never walked along the whole Grand Canal myself for example, but the plan shows only large forests once you cross the perpendicular canal. On the side is the private garden with the Grand Trianon (also with its own little Hall of Mirrors) and the private castle of Queen Marie Antoinette, the Petit Trianon (indeed far less sumptuous than the other two palaces). For those who don't feel like walking along the cobble stone, golf cards and a little train are available for your comfort.\\

Versailles: Chateau with both Trianons*****\\

August 26: St Denis:\\
Just a couple of weeks earlier France won the FIFA world champion title against Brazil, and this in the Stade de France which is close to Paris in the suburb of St Denis: Clearly I was excited to see the inside of a real large stadium for the first time. Impressive but maybe not the entrance fee, but none of the stadiums are particularly inexpensive. Anyway in St Denis itself is the basilica of St Denis, nowadays even a cathedral. So what's so special about this church: Besides the apparent collapsed left tower the basilica houses the tombs of most Kings and Queens in France, and the choir is the birthplace of gothic art. The stained glass is very impressive, but the tombs are magnificent, the only church I've been to which can compete in this respect is Westminster Abbey. Considering the French tombs are typically larger and more decorated I would opt to see those (particularly the ban on photography at Westminster Abbey adds one more icing on the cake). As if we didn't see enough on this day yet we continued with the Opera Garnier. One gigantic neo-Baroque theatre with modern elements like Marc Chagall's large ceiling fresco in the amphitheatre.\\

St Denis: Basilica and Royal Tombs*****,Stade de France**\\
Paris: Opera Garnier*****\\

August 27: The Pantheon:\\
On this rather quiet day we saw the Pantheon, originally a church build in the classical style during the reign of Louis XVI., the church was transformed into the French Pantheon holding the remains of the most important French people, among those e.g. Victor Hugo. The building itself was in a quite desperate state, a lot of ceilings were covered hidden behind nets due to falling rocks, scaffolding was all over the walls as well, clearly in dire need for renovation. Afterwards we spent some time in the vast park of the Jardin Luxembourg.\\

Paris: Pantheon****\\

August 28: Champs-Elysees\\
I never got the craze for fancy boulevards turned into shopping avenues. The Place de la Concorde is a nice superb square, also the church of Madelaine is just next to it. A bit on the side is the Palais d'Elysee. Back then you were at least allowed to view the gate and the courtyard. On normal days nowadays you even have to stick to the other side of the road. The only days when normal people like us can see the interior of the palace is the heritage weekend of the France (typically the third weekend in September). Since so many people want to see the palace on those two days the tickets are nowadays limited. Anyways back to the Champs-Elysees, the facades are nice, but otherwise almost all buildings are nothing else but shops (or McDonalds etc etc). The Arc de Triomphe is OK, but not really an absolute must either. SThe afternoon we spent by the quite interesting Parc des Buttes Chaumont.\\

Paris: Arc de Triomphe***, Place de la Concorde*****, Champs-Elysees**, Parc des Buttes Chaumont****\\

August 29: Montmartre \& Dome d'Invalides:\\
I will never understand what people see in going to Montmartre. I agree the Sacre-Couer is interesting from the outside, but an absolute plain uninteresting insides (maybe that's why photography is not allowed inside). Not to forget the zillions of street vendors hunting for possible customers around the stairs leading up to the mountain top. The views are nothing to get crazy about either. Anyways on to the next stop, the Galleries Lafayette is a giant shopping mall as well, but instead of others in a really old building, thus I go there to for looking at the building rather than shopping, don't forget to get up the roof terrace with an interesting view of the nearby Opera Garnier and the Eiffel Tower. And via the Place Vendome on to the Army Museum, where they exhibit all generations of French tanks (not that they had been leading to such an amazing success in most occasions). The centre piece of the Invalides complex is the Dome d'Invalides with Napoleon I's tomb, once again another occasion for a red tilted photograph.\\

Pais: Montmartre with Sacre-Coeur**, Galleries Lafayette***, Dome d'Invalides****\\

August 30: The last day:\\ %Park (Porte Ludovico Magno
Not that much to do on this very last day: we walked past Centre Pompidou, ate at a Pizza Hut (first time on that Paris trip), but then all the way back to the airport and home it goes. And then it took roughly 12 years for me to return, but more about that later on.


\chapter{Year 1999}
\label{1999}

\section{August 27-September8: Greece}
\label{1999:Greece}

%27 Athen Nacht Akropolis
%28 Akropolis, Nationalmuseum, Katholische Kathedrale
%29 Korinth Kanal, Mykene (Atrus, Akropolis, Graeberfeld, Epidauros: Museuum Theater, Nauplia
%30 Olympia: Museum, Stadion, Temple, Basilika
%31 Mega Spileo, Rio-Andirio Faehre
%1 Delphi: Museum etc
%2 Athen Dionysos Theater, Olympeion, Agora, Olympiastadion
%5 Eubaoa

Japanese Tourists know how to deal with earth quakes - everybody else not so, dancing on the evening show

Athens: Acropolis*****, National Museum*****\\
Nauplia: Harbour****\\
Olympia: Excavations*****\\
Delphi: Excavations*****\\
 Epidauros: Theatre and Excavations****\\
 Mykene: Acropolis*****\\
 Corinth: Canal of Corinth****, Temple***\\
 Eretria: greek ruins**


\chapter{Year 2000}
\label{2000}

\section{May 18-May21: Munich}
\label{2000:Munich}

This trip was part of the so-called Hochrheinseminar, a school outing for folks particularly interested in STEM subjects as extracurricular activity besides school. My topic was back then chemistry (a bit ironic I then later went to focus on physics instead). The main part of the science outings was a visit of the Graching FRM II reactor which was in construction at that point, we were also told about the old reactor. For sure I thought it is a very exciting topic and maybe one of the many reasons I went for a career in research a couple of years later.\\
Now did I only do what highschool students do (aka celebrate too late, going to restaurants or bars), but the whole group also did an Opera outing, watching Bizet's Carmen at Staatsoper. OK I only saw half of it, since on the very same day Bayer Leverkusen lost a bit unexpectedly vs Unterhaching, while Bayern M\"unchen won an unexpected Bundesliga championship. Clearly I had to be present at the impromptu Meisterfeier at Marienplatz.\\
Otherwise my friend Vera and I did a bit of tourism too, visiting Frauenkirche and Theatinerkirche, as well as the Munich Residence, the royal winter palace which is one of Germany's largest palaces. It was really impressive, and I enjoyed that you could take photographs (and you still can up to now, one of the few places in Bavaria which still allows it).\\ 

Munich: Residenz*****, Frauenkirche***, Theatinerkirche****, Chinesischer Turm****, Garching FRM II*****, Opera****
FC Bayern, FC Bayern*****

\section{June 3--June6: Hannover}
\label{2000:Hannover}

Hannover: EXPO2000*****

\section{August 25--September 7: Turkey}
\label{2000:Turkey}

%25 Antalya
%26 Bursa Green Moschee, Ulu Camii
%27 ferry Asia Europe, Topkapi, Hagia Sophia
%28 Bazar, Bosporus Cruise, Sultan-Ahmed, Zisterne
%29 Ferry, Troy, Canakkale
%30 Pergamon, Ephesos
%31: Pamukkale, Hierapolis

Antalya: Harbour****\\
Istanbul: Topkapi Palace*****, Sultan-Ahmed-Mosque****,Grand Bazaar***,Hagia Sophia*****,Cisterne*****, Bosporus Cruise*****\\
Ephesos: Excavations*****, Pergamon: Acropolis*****, Lower Town***,
Pammukkale: Teracces*****, Tombs of Hierapolis***\\
Bursa: Grand Mosque****, Green Mosque*****\\
Troy: Excavations*****
\chapter{Year 2001}
\label{2001}

\section{Spring 2001 -- Barcelona}
\label{2001:Barcelona}

Montserrat*, Sagrada Familia***, Cathedral***, Aquarium*, Casa Mila**, Beach in Sitges*, Miro Museum**, Figures: Teatro Dali***, Pont du Garde***

\section{Summer 2001 -- Berlin}
\label{2001:Berlin}

Reichstag***, Sanssouci***, Egypt Museum**, Charlottenburg***, Kaiser-Wilhelm-Gedaechtniskirche*, Dom***, Deutscher Dom**, Zitadelle Spandau*

\section{August 22 - September 3: Andalusia}
\label{2001:Andalusia}

My family really loves spending time in Spain, both to see the culture and to be on the beach. Thus Andalusia offers both of those worlds and thus one destination which seemed to be ideal. After we had good experience the two years before we opted for a guided tour of Andalusia.\\

August 22: Where is our hotel?\\
Arriving late in the evening on we were already awaited by a shuttle, which was supposed to bring my family and two fellow travellers to our hotel. We got out and proceeded to check-in -- AND were informed that we were dropped of by the wrong hotel. Unfortunately nobody knew though were the correct hotel would have been. Thus my father called my older brother, who had opted out of the trip and decided to stay at home. Calling the agency in Germany forth and back, detailing that we were lost at this particular hotel - some 30 minutes later another shuttle arrived taking us to the correct hotel. Our tour guide for the next few days claimed she was already looking for us, when in fact she even didn't initiate anything and most probably even didn't bother to check if everybody was properly picked up. At least in the following days she showed once and once again that she wasn't up to the standards we had so far travelling with this agency.\\

August 23: Nerja\\
Our first stop was the Balcon de Europa, a viewpoint over the Costa del Sol in the village of Nerja, it has a couple of cliffs around, but the weather was not that sunny, thus maybe not as impressive as it could have been. Then we stopped by the very impressive caves of Nerja, which stretch for a couple of km. Huge halls with gigantic stalactites and Stalagmites. Then our bus ride continued to Granada, where we explored the arab quarter of Albayzin with the best view of the Alhambra. Our guide claimed that every day tourist would get lost in the labyrinth of the quarter, which was for sure a bit exaggerated. We didn't go into any of the old houses, courtyards etc, if I should be back at some point, that's something I'd like to explore again. Nowadays google would lead the way anyway.\\

%Arcos de la Frontera/Ronda 1, Marbella 28 etc ? 14 nights

Nerja: Balcon de Europa*, Caves of Nerja***\\
Granada: Albayzin*\\

August 24: Granada\\
The Alhambra is one of the most impressive palaces in Europe. The palace was built over centuries, with the most important part being build by the Nasrid Kings of Granada. After the conquest of Granada by the Spanish, king Carlos V built a renaissance palace on the plateau. The palace was left unfinished though. We had a tour of the Nasrid palace with its many courtyards, tile and carved stone decorations. Most of the paint faded away through the times, I was most impressed by the Lion Courtyard. Afterwards we went over to the gardens of the Generalife palace. Since we had a short coffee stop by the cathedral, we asked out guide if we could have a self payed look into the cathedral as well. We were brushed off that 15 minutes wouldn't be enough and it would be utterly boring for anybody to see (Once again she was wrong and proofed here incompetence). Once we arrived in Cordoba we stopped by the Roman bridge for a photo.\\

Granada: Alhambra***, Generalife***\\

August 25: Cordoba\\
The Mezquita-Cathedral is two sights for the price of one. It is the old mosque of the Caliphate of Cordoba, and if you look from most sides, it still looks to be completely intact. The Mihrab is still there, the forrest of columns with white and red double arches is not that much affected by the modifications of the Spanish kings, it is brilliantly done. In the middle of the mosque is the cathedral part, built in Baroque and Renaissance style. Obviously a completely different style and not going well together, the king is said to have announced his displeasure of not being told how awesome the mosque was, before agreeing to the construction of the cathedral. Nonetheless the cathedral is very beautiful itself as well. Back in 2001 though in a dire state and in clear need for renovation (which was finished by mid 2010s though). Impressive choir stalls with lots of wood carving, impressive main altar piece, beautiful ceiling. The side chapels are held in moorish style and blend in with the mosque. While my dad took a photo of the rood screen and the retro choir his flash decided to go up with a bang (scary but nothing more happened). The combination of the Mezquita and Cathedral is clearly my most favourite religious building i have been to so far (status being of 2020). Afterwards we walked through old town having a look at courtyards and flowers over the whitewashed walls. And then off to Sevilla, where we took a bus tour of the remaining pavilions of the EXPO 1992 (500 years of the Columbus Journey to the Americas) and the Puente del Alamillo\\ 

Cordoba: Mezquita-Cathedral***\\

August 26: Sevilla\\
We started our day by the Plaza de Espana, a square set up in the 1920s as exhibition space for the Ibero-American exposition in 1929. Nowadays the square with its canal and it towers is used as prominent background for filming. Then we visited the Alcazar, the Royal Spanish palace in Sevilla, with vast gardens and even rooms which have been built in the 11th century by the Abbabid dynasty while Sevilla was the seat of muslim kings (really beautiful). Only a couple of metres away is the gigantic gothic cathedral of Sevilla, still the largest cathedral in the world and only a bit smaller than St Peter's, absolutely magnificent. It is also the home of the tomb of Christopher Columbus. The Retablo of the main altar is the largest in the world, and the pinnacle of Spanish medieval wood carving. Then we climbed the Giralda, the bell tower which is in fact the former Minaret of the previous mosque with Renaissance additions. \\

Sevilla: Alcazar***, Cathedral***\\

August 27: Cadiz, Jerez and Arcos de la Frontera\\
Driving along the Guadalquivir and the Isla Del Trocadero we arrived by the old city of Cadiz, where we stopped over lunch time. The cathedral was closed at that point and there wasn't much more to do than getting a snack. Then we drove through olive fields over to Jerez de la Frontera. On our way our guide praised the quality of Andalusian olives and talked down the horrible olives from Italy and Greece. Then she continued to tell us how amazing the Formula One Grand Prix of 2000 had been. Talking to German formula one fans, who all clearly do remember the Jerez Grand Prix in 1997, when Schumacher tried to kick out Villeneuve unsuccessfully and lost the championship, she clearly had no idea ONCE again. Or she just believed we would be stupid. After being told that she got her facts wrong she insisted that she is right. Clearly once again showing her incompetence. And we stopped by the cathedral and the Alcazar (not that we got to see any of them), but we had a sherry tasting by the Gonzalez Byass Sherry Bodegas, which was alright (having not developed a proper taste for alcoholic drinks at that point). The final stop was Arcos de la Frontera with a visit of the Santa Maria basilica.\\

Jerez de la Frontera: Gonzalez Byass Sherry Bodegas*, Arcos de la Frontera: Santa Maria**\\

August 28: Ronda:\\
The last stop of the Andalusia tour where we walked over the El Tajo Canyon, saw a couple of gardens, the church of Santa Maria la Mayor and one of the oldest bullfighting arenas in Spain. And then back to Marbella, where we transferred to our holiday club for the next couple of days.

Ronda: El Tajo Canyon**,  Santa Maria la Mayor**, Bullfighting Arena\\

September 1: Marbella\\
What do you do as good German in a club full of tourists, pools, drinks and a closeby beach: watch football (just like the English families in that club did too). But what a horrible match it was: Germany 1 - England 5. Half of the holiday resort was celebrating and running around singing, whereas the other half is dead silent. All English fans felt pretty sorry for us. But then in the aftermath: Germany finished second in the group, in fact Germany qualified against Ukraine in the relegation matches, and made it into the final of the world cup 2002 - only losing out to Brazil.
\chapter{Year 2002}
\label{2002}

\section{May: Tuscany}
\label{2002:Tuscany}

%San Gimignano
%Siena
%Florenz: San Lorenzo, Duomo, Santa Croce, Ponte Vecchio, Palazzo Pitti, Santo Spirito
%Pisa
%Volterra
%Montaione-San Vivaldo (place where we stayed)

San Gimignano**, Siena Duomo***, Pisa Duomo***, Firenze: Duomo**, Palazzo Pitti***, San Lorenzo**, Santa Croce**

\section{August 20-September 3: Morocco}
\label{2002:Morocco}

%20: Agadir
%21: Marrakesh: Saadian Tombs, Fnar el, El Badi, Menara Gardens
%22: Ouaoumana Dam, Atlas
%23 Fes Mdina
%24 Fes, Medersa Attarine, Gerberviertel Meknes: Moulay Ismail, Volubilis
%25 Rabat: Mausoleum, Hasan Mosque, Casablanca
%26 Essauira

After our good experiences in Greece and Turkey my parents decided that we should see a bit of northern Africa, deciding to visit Morocco.\\

August 20: Agadir\\
We flew into Agadir, where we were in a hotel with a nice pool, other than that we got to know out tour group. Our family was put together with a French woman and her 12 year old son. My mum does understand French but is typically missing words when trying to talk. The French lady had the opposite issue, she understood German pretty well, but speaking was more of a problem. Thus they did have some French-German conversations with each other. Having had latin at school, I was of no help anyway.\\

August 21: Marrakech:\\
Starting out very early, once we arrived in Marrakesh we saw the ruined palace of El Badi as well as the Saadian tombs, which are very nice. Then we drove over to the Menara Gardens where we saw the former Royal pavilion. Then we had dinner on a roof top restaurant overlooking the bazaar of the Djemaa el Fna square.\\

Marrakesch: Saadian Tomb***, El Badi Palace**, Menara Gardens*\\

August 22: Atlas:\\
From Marrakech we drove through the Atlas mountains, passing the Ouaoumana dam on our way, having arrived in Fez we stopped to have a look at the large town on the food of the hills. Fez had been one of the largest old towns of Africa, actually two old towns - the Fes el-Bali from the 9th century and the Fez el-Jdid in the 13th century. Then the new city was founded by the French. The old town is part of world UNESCO heritage and even nowadays the old part of the city has over 100'000 citizens alone. \\

Marrakesch: Saadian Tomb***, El Badi Palace**, Fez: Gates***, Chouara Tannery**, Al-Atarine Madrasa***, Mausoleum Moulay Idris II**, Meknes: Moulay Ismail Mausoleum**, Volubilis***, Rabat: Mausoleum Mohammed V***, Casablanca: Mosque Hassan II**, Essauira*

\chapter{Year 2003}
\label{2003}

\section{July: Bodrum}
\label{2003:Bodrum}

Gianrico, Sigi, B\"oni and I wondered where we should relax after our hard first year physics exams at ETH. Back then we had to go through catalogues to find suitable places. My mum suggested that maybe we could go to Turkey, and why not to Bodrum. We were easily convinced and ready for some days by the beach.\\

Co-travellers:\\
Gianrico, Sigi \& B\"oni: just like me first year undergrads studying physics at ETH Zurich.\\

My dad volunteered to pick everybody else up in Zurich and drive us to Stuttgart airport. Apparently the three of them had already a wild evening, B\"oni at least thought it was a brilliant idea to have a few drinks to forget about his flight fright. We also passed a couple of accidents on our nightly drive to Stuttgart. We arrived well in time, even the check-in lines were not operating yet. On the flight we met my local village's political figureheads who went on a sailing trip in Turkey. After a short stop at another airport we finally arrived in Bodrum and took a small bus to our resort. It was indeed very nicely located, overlooking the sea and a small island, a bit removed from the city with a large pool and a small apartment with two bedrooms. Since our all inclusive package included all kind of drinks we had a pretty good time by the pool, on the bar, as well as checking out the beach. Gianrico unfortunately caught an ear infection, thus he was not allowed to really dive deep into the pool anymore. On our first day we wandered through Bodrum and opted for a night out on a club boat. We soon found out that the music was not really suited for our style, but then we were stuck on the boat for more than 3 h anyway. In our club we also had a couple of nice late evening concerts, which I enjoyed quite a bit, same with the sunset over the mountains. Unfortunately on one day we were affected by one local wildfire which ravaged for about a day before they managed to put it out. Getting back into Bodrum frequently for dinner and night activities we found a nice bar just by the sea with nice views of the old harbour and the castle. Since it was a nice time out with music according to our taste we went there about three times, each time meeting the same fun dutch ladies as well. For our cultural program we visited the close-by ruins of Ephesus with the famous theatre and the breathtaking library, the merchant houses were just opened, but we didn't go to those. On our way back we had a short photo stopped by the one remaining column of the temple of Artemis, which is in fact patched together using fragments from different columns. Although I knew that Bodrum is the former site of the Mausoleum of Halikarnassos (another one of the former ancient wonders of the world), we didn't see the local archaeology museum either. Overall we had a lot of fun swimming, eating good food and having a couple of drinks day in day out. Well-needed after the stressful examination period.\\

Ephesos***\\
\chapter{Year 2004}
\label{2004}

\section{March 2004: Frankfurt}
\label{2004:Frankfurt}

My dad had a business meeting in Frankfurt. Being on spring break without any exams following (these happen at ETH only in Summer, unless you have to repeat exams), I decided to join him, or rather join him on the road to Frankfurt, and then go on my own way. What can you do in Frankfurt: besides seeing Paulskirche, the first meeting place of an elected parliament in Germany, and the Dom, the former coronation church of German Emperors, there is the Bankenviertel. The only business district in Germany with plenty of skyscrapers, the tallest one in Germany and some of the tallest in Europe. But other than that Frankfurt is just a big city, without that much more to do

\section{July 2--July 9: Tunisia}
\label{2004:Tunisia}

How this trip came into existence: After studying real hard for another year, we decided to go on another beach holiday. This time we followed recommendations from Sigi's mum and ended up in Tunisia. Fan fact: my sister thought it looks awesome as well, and booked the same resort with her friends a week later.\\

Co-travellers:\\
Sigi and Gianrico: having shrunk after B\"oni's departure after last year's examination this time we booked a larger room for three wild physics students from the holy trinity of Germany (myself), Switzerland (Gianrico), and Austria (Sigi).

Unfortunately I don't have particular dates recorded for this trip, besides the fact that there were two pools in the resort, a large disco, which came in handy, after we found out that the bars and clubs in Sousse and Monastir were not on par with those we encountered in Bodrum the year before. On the first day we made our way to Sousse, with its old city walls and Medina, but after a nice dinner all music we could hear were more traditional and the city was more tame than Bodrum, thus in the next few days we always stayed at the resort and enjoyed the disco there.\\
The pool was nice too, this time nobody had to suffer through an ear infection. But the best was still to come: two or three days in we got hit up by a german Neckerman employee Heike: She asked us, if we wanna join. We were -- alright, join what -- Getting totally SMASHED! -- Ohhh, well, no, thanks, but no. Seems that was not really the expected reaction: ``What, i don't understand, you don't wanna get drunk and puke all over -- you are on holidays, you can see an authentic tunisian show - with me fire-spitting'' (clearly authentic). ``We even don't care if you throw up in the bus, even that is fine. And you still don't want to come and get smashed''. We must have been the strangest young people to her, and nope we didn't join her.\\
We had though one dinner at a restaurant, where they serves us all vine we wanted. Clearly splitting two bottles between the three of us, wasn't the smartest idea we had, but considering we stayed in the resort nothing much happened otherwise. Clearly we also did a trip to see other places in Tunisia, notable the capital Tunis itself, visiting the really nice , and then we went on to the ruins of Carthage. Little is left from the punic times, besides a few findings put in museums, but the romans built up quite a sizeable again, particularly the Roman Baths were impressive.\\

Carthage: Excavations****\\
 Tunis: Medina with National Museum*****\\
 Sousse: Medina***

\section{October 15--October 20: London}
\label{2004:London}

Learning for this year's exams Gianrico realised on this weekend the last easyjet flight to London (actually London Luton) would fly (they did come back later on though). Thus he decided we couldn't let such an opportunity go by, and booked tickets.\\

Co-travellers:\\
This time it was four of us: Gianrico, myself, Sigi, and Anne, a ETH math student in our year from Luxembourg, who happened to be Sigi's girlfriend too (so clearly two rooms it was for this trip):\\

Unfortunately once again I have no detailed recollection of when we went, besides the fact that it was after the ETH examination period, and most probably late September. We started our trips with the Dinosaurs at the museum of natural history. We also did pay a visit to the British Museum (amazing as always). Our tourist programme was rounded of with the Tower of London and St Paul's. Now this trip wasn't supposed to be cultural only: The first night we decided to spend at the club of the Swiss house by Leicester Square: but that R\&B music (Destiny's Child etc etc), wasn't so much to our liking. The next days we went to other dance clubs, playing rather dance and pop music (or a pop cover night for that matter). After all it is great to dance until 3 or 4 am the next day. By coincidence while waiting at the half fare ticket booth, we realised there were so many people waiting all over, and we found out pretty quickly, that it was the premier of ``Finding Neverland'', and indeed we later saw Johnny Depp and Kate Winslet walk by and shaking hands with fans.\\

London: Museum of Natural History*****, Tower of London****, St Paul's Cathedral*****, British Museum*****

\chapter{Year 2005}
\label{2005}

\section{February:Diavolezza}
\label{2005:Diavolezza}

Each winter ETH Z\"urich offers the possibility to observe the sky for a week with fancy little telescopes under the supervision of experts to get practical experience on how to perform little projects. Many fellow students in my friends group were interested too. Sigi, Sibylle and I set up an observation plan, we learnt to get to know the telescopes previously as well. This year's observation point was Diavolezza, a mountain hut at almost 3000 m in the canton of Graub\"unden high in the alps. Chances for a clear sky are higher in winter, the Bernina massive is also far away from big cities, though only a distant light cone of Milan should spoil the sky. Most of the groups had projects during night time, while other projects studied the sun (thus day time projects). Anyway since we were quite a bunch of people who knew each other I set up a little group booking, which should have given us a nice ride in free seats etc -- or that was the plan. My dad got me to Zurich main station and there we realised that nothing would be as straight forward as hoped. While most of us were indeed around at the planned meeting point, the traffic control system of Zurich main station had major issues on this very day (and Sigi had a major cold, so he was out of the trip as well). Mind you Zurich is one of central europe's busiest train station, thus imagine the gigantic chaos when almost no train can be routed to go in and out as planned. Some of my friends were stuck for over an hour, although being merely a few km away from the central station (second biggest train chaos I have ever seen in Switzerland, where things go pretty smoothly almost always, the largest chaos being a failure of the whole power network for the full train system of Switzerland). Also we others were stuck as well, as almost no train left the station either -- not that we were too worried, as long as it would be possible to get up to the hut once we would arrive (and yes we were even interviewed by Swiss newspapers how we felt about it). Now at some point a train was leaving for Chur, and we jumped on it. From there train traffic was just working fine again, we (actually Frank) had to organise a cable car ride up since we would miss the regular hours for a cable car ride. Since it was not in our hands to arrive though thankfully SBB (Swiss Train company) agreed to take over the typically extra fee for the out of hours ride.\\
 Once we arrived in the hut we were split up into the hostel rooms, had dinner and a first view at the sky, which was though a bit cloudy on that night, thus not ideal for observations. The hut was very cute, Diavolezza offers a very popular skiing piste as well as superb views of the Bernina massif (highest peaks in the eastern alps) and the Pers and Moreratsch glacier. Being a non-skier myself I rather enjoyed the mountain scenery (and sleeping in on mornings, since the observations were going on at night). During day time we had set up other telescopes for the sun observations, which we also showed to other tourists after all it is great to convince others (also young kids coming for skiing) that astrophysics is a pretty great thing to do. I also learnt about all playing modes of the traditional Swiss card game Jassen (the one we played at home in Germany didn't include all options Swiss people came up with).\\
 Other unexpected issues popped up during our observations run - it is pretty cold high up in the alps at night (no shit -- not THAT unexpected I know). At about - 20 degrees C some mechanism to automatically move the telescope to tackle earth's rotation froze in, and laptops went on strike as well. Thus we had to construct heating devices (aka enclosures which we heated up using hot water packets, which we had to exchange regularly) and we used hair driers to blow hot air at the mechanics of the telescope on a regular basis as well. Thank god the air was dry, thus the other end didn't freeze over with a film of ice like your windshield might do in winter. The next couple of nights were pretty clear, at one point we also saw the ISS pass over us (quite awesome if you ask me). Since we had to deal with these telescopes in a manner that we didn't expect to be that elaborate and two nights were lost to none-clear skies, some groups -- including mine -- merged with others who had a similar project in mind in order to distribute the tasks and speed up things. 
 %something about the project, brightness curve
 Later we found out that although we were convinced to have caught the star constellation we wanted to study we had in fact not but studied another double-star system close-by. Thus it was not the ideal conditions for the curves as we had hoped for, but it was still doable (tough luck).\\
 What happened otherwise: the showers broke for a few days and we had to use the sink (plumbing doesn't get easier up there than down in the valley). Gianrico visited as well over the weekend and we hiked over to a nearby mountain, my first time crossing the 3000 m altitude (Germany's highest mountain peak -- the Zugspitze -- is only 2963 m or 2964 m, depending if you ask Germany or Austria). By now (begin of 2020) I crossed 3000 m a couple of times, in Switzerland, France, Italy, as well as in the US). Swiss alps are always very cute and particularly magical in wintery white snow coverage, nevertheless also very impressive in Summer, when you can still admire all the glaciers (please do that while you can, in 50-100 years from now ALMOST all of them will be sadly gone). And then it was time to get back home. For the train ride from Diavolezzo to Chur we decided to take the route of the R\"athische Eisenbahn (Raethian Railways) which has been included into World Unesco Heritage since - with many tunnels and viaducts crossing canyons and valleys. This time all trains were running on their regular schedule.\\
 
 Diavolezza \& Bernina Massif with Pers and Morteratsch Glaciers***, R\"athische Eisenbahn**

\section{March 20--March 28: Egypt}
\label{2005:Egypt}

Getting to Egypt has been a dream for me and my parents for quite some time. In my last years at highschool I wondered if I should inscribe for archaeology, but the dire job situation made me reconsider. How I managed to end up in particle physics instead with uncertain job aspects instead still remains a mystery. Anyway finally my dad decided that we should tackle the organisational matters and go for it. He found a tour provider for Egypt in Zurich with who he set up an itinerary. We had to hand in our passports to check visa regulations. It was a bit chaotic but then we were ready to go. This time it was my parents, my younger brother and I.\\

March 20: Arriving in Cairo:\\
Having arrived in Cairo we met our contact, who welcomed us and handed over our flight tickets and our tickets for the Nile cruise. But something didn't really work out. We were supposed to fly from Cairo to Aswan and back from Luxor to Cairo, but the boat was going the other way, from Luxor to Aswan. Something for sure was messed up. A couple of calls later we were told that something went wrong with the inner Egyptian flights and in a couple of hours we would get the proper tickets, thus we decided to walk around Cairo for a bit. Our hotel was located just next to the Hilton Ramses directly by the river nile and its island. Back in pre google times we had to rely on a rough city map we had received at the hotel and we chose to explore a bit the Nile Island. Which turned out to be a nice walk also with a view of Tower. Once we came back from our walk, the new flight tickets were awaiting us.\\

Nile Island**\\

March 21: Pyramids and the Egyptian Museum:\\
Our guide took us first to the graveyard of Sakkara. Here we saw the complex of Djoser's pyramid, built in several steps it is the first of the large Egyptian pyramids. Sakkara was located south of the ancient capital of Memphis. Not only Djoser's pryamid is located there, but we saw also the pyramids of Userkaf and Unas as well as the walls where old boats had been buried with ancient treasures. After Sakkara we moved on to the necropolis of Gizeh with the famous large three pyramids from Chufu (also known as Cheops in Germany), Chephren and Menkaure. We walked around all of the pyramids also sitting one one of the stones just in front of the big pyramid. Our guide took us inside one of the queen's pyramids which are located next to the Pharaoh's pyramid. Chufu's pyramid is the largest but Chephrens pyramid is built on a slightly higher plateau, thus it appears to be a bit higher. On top of that the polished surface of the latter is still conserved. One of Cairo's later rulers tried to destroy the pyramids starting with the smallest one of Menkaure, he didn't get very far but a big hole can be seen about midway up where stones had been removed. Although the Sphinx of Gizeh is gigantic it just pales in comparison to the size of the pyramids which completely dominate the view. Not to forget that this was my very first time in a desert. It was very impressive to see nothing but sand and rock as far as your ice could see. After our visit by the pyramids we headed back to our hotel which was just a couple of metres away from the Egyptian Museum. It is absolutely amazing what sculptures, gift, treasures etc are housed in this rather tiny building. The whole floor and foyer were full of art of thousands of years. And the out of this world gilded chamber of Tunankamun with the sarcophagus and the gold mask. Even photographs don't do justice to the amazing incredible piece of world history you can find here.\\

Pyramids of Sakkara***, Pyramids of Gizeh***, Egyptian Museum***\\

March 22: Karnak and Luxor:\\
We got up very early on to make our flight to Luxor. When our flight was called we realised that our tickets had a different colour than most other passengers, and indeed we were told not to follow everybody else but move to the left. We got our own private van bringing us to our Air Egypt flight. Seems the mess up with our tickets lead to us flying in Business class on this flight. Unfortunately it was just a bit less than an hour. But we tasted all nice pastries and jam and fruits and juices which we were served. After all it didn't happen to me that often to be upgraded to Business at least up to now. Once we arrived in Luxor we met our guide for the Nile cruise part of this trip who brought us first to our ship. And then we went to the largest temple of Egypt, the Amun temple in Karnak. After an alley of sphinx, a forrest of columns appears with lots of depicted religious and battle scenes, some even with their colour preserved. There is a vast area of courtyards and a little lake behind the main halls, where excavations are still going on. The second large temple of the region is located in the South of Luxor, both temples are thought to have been once fix points of a procession street through the old Egyptian capital of Thebes. The temple of Luxor has a large hall with columns as well, inside the temple is also an old mosque, and a lot of colossal statues.\\

Temple of Luxor***, Temple of Karnak***\\

March 23: Valley of the Kings:\\
Crossing over the other shore of the Nile, we visited the valley of the kings and one of the tombs of a Pharaoh, which was very richly decorated. It might have been the tomb of Seti I. Afterwards we stopped by the giant mortuary temple of Hatshepsut, one of the view female Pharaohs, which rised up from the desert in terraces leading to the main chamber. Nearby another temple is being excavated right now. On our way back to the ship we passed the temple of Ramses and stopped by the so-called Colossi of Memnon, the giant status of Pharaoh Amenhotep III in front of his temple, which was destroyed in an earth quake. And then we got on the boat and the cruise started. We spent the afternoon playing cards on the sun deck, the pool was a tad small. In the evening we passed a lock where already other ships had lined up ahead of us.\\
  
Valley of the Kings***, Temple of Hatshepsut***, Colossal statues of Amenhotep III**\\

March 24: Edfu and Khom Ombo\\
One of the late large temples was built in Edfu during the Ptolemaic period. Already the front depicts battle scenes. The temple is in a superb shape, but the frieze, sculptures, and paintings are less colourful, but still very impressive to see. And on to the boat again, where my mum decided to give the tiny pool a chance. But there wasn't much swimming to do, due to it only being about 10 m long. It was already dark when we arrived by
 the double temple of Khom Ombo. Brightly illuminated blue seemed to be the dominating colour of the ceiling decorations. It was a real nice experience visiting the sight by night. Coming back to our cabin we realised that our towels had been folded into Elephants, butterflies, and crocodiles, very cute and nice to look at.\\
 
 Temple of Edfu***, Temple of Khom Ombo***\\
 

%25 Aswan Temple
%26 Staudamm, bootsfahrt
%27 palast bei zitadelle Zitadelle

Citadel***
Aswan Temple of Isis***

\section{August 28--September 4: Lisbon}
\label{2005:lisbon}

%28:
% grosser platz: strand/hafen mit Bruecke und Christus im hintergrund
%29
%Castelo/Le Seo
%30 Belem, Weltkarte
%31 Aquarium
%01 Queluz & Sintra
%02 Stadion, Altstadt
%03 Christus
%04 heimflug

Castelo de Sao Jorge*, Monasteiro de Geronimo***, Tower of Belem**, Aquarium***, Estadio Jose Avalade**, Estadio da Luz*
Palacio de Queluz***, Palacio de Sintra**
\chapter{Year 2006}
\label{2006}


\section{July 11-July 18: London}
\label{2006:London}

I like London, so does my family. My brother had just graduated high school and my last semester had just finished. Since exam season starts in late August, I rather preferred to have a trip right away, before starting studying. Thus my parents, my brother, and I decided to go to London once again. We had all been in London previously, thus it should be easy for us to get around.\\

July 11:\\
We started our tourist program of this London trip in Westminster Cathedral, the large catholic cathedral built in neo-Byzantine style, since the bishop didn't appreciate the gothic revival anymore. The side chapels are full of mosaics, on the main body of the church only one mosaic has been installed. If ever others had been planned, none of those have been realised, and it doesn't look like that would change anytime soon.\\

London: Westminster Cathedral*****\\

July 12:\\
Last time we had seen Hampton Court Palace, thus we opted to visit Windsor Castle instead this time around. Windsor Castle is still inhabited by the royal family, but the state apartments can be visited all year round. The medieval keep still remind the visitor of the old roots of the castle, so does the high gothic St George's Chapel as well. In the afternoon we admired the royal tombs in Westminster Abbey. By this time they had already forbidden photography, abolished the free Wednesday afternoon, and increased the ticket from 2.50 pounds to over 15, but at least the nave could be visited for free. Already in the 2010s that has been abolished, the ticket price has gone to almost 20 pounds, photography is all forbidden - all claimed to be a sign for spirituality. Seems to be the norm for anglican cathedrals, just saying... Then we wanted to walk around the parliament. Once we passed Westminster Hall, my mum saw, that we could witness a parliamentarian debate. Only having passports on us was suited enough to let us in, I cannot imagine that happening anymore. We also had time to admire the great ness of the medieval Westminster Hall. Just like for all parliaments in the world, we had to lock in our phones and cameras, sit quietly on the visitor benches and listen to the debates for about 30-40 minutes. This time the debate happened in the House of Lords. While this chamber is the less powerful of the bicameral UK system, the meeting hall is clearly the more beautiful, the original of the House of Commons had been lost in the 40s, when German bombs hit Westminster Palace.\\

Windosr: Windsor Castle*****\\
London: Westminster Abbey*****, Houses of Parliament*****\\

July 13:\\
And we decided to have a quiet day in between, spending it in the parks of London, Hyde Park in particular, starting out by speaker's corner and the Marble Arch. The park is really large, with lots of squirrels running around, little lakes, large lawns. The Diana fountain is a bit of a let down (or not as beautiful as I hoped it would be). We also walked over to the Albert Memorial and Kensington Gardens, and then we finished our walk by the Wellington Arch. Meeting up with my friend Clive the evening for drinks and dinner.\\

London: Hyde Park***\\

July 14:\\
After having been in the Tower of London already twice before, I decided to skip it this time, handed over my camera to my younger brother, and decided to stroll a bit through the City of London instead. Afterwards we all met up and walk across to the city hall, visiting with the Southwark cathedral another large gothic church, before we ended up by the Tate modern gallery, which is a transformed former power station. I do enjoy spending time in this modern art museum quite a bit, whenever I can in London.\\

London: Southwark Cathedral****,  Tate Modern****\\

July 15:\\
One of the most important classical baroque ensembles in England is the so-called Maritime Greenwich ensemble, comprising of the former Marine hospital and Queen's House. Both the chapel and the Painted Hall had been open for the open days of the university/college of Greenwich. My mum assumed that you could only go inside should you be a guest of the open days and announced her desire to study either history or politics. Since german and english A-levels are treated equivalently in theory she could have gone for it (didn't show much more enthusiasm for starting late studies shortly after though). It was my first time in the Painted Hall, indeed one very fine hall with giant frescoes. Nowadays (2019), the hall is always open for visits. The chapel was alright, I still don't think Queen's House is something really special, most of the interior is long gone anyway. And then we walked over to the observatory, standing on the Greenwich Meridian (not that I get the craze for that one, unlike the equator there is no physical reason behind it being in that particular spot).\\

London: Greenwich (including the Painted Hall, the Chapel, and the Royal Observatory)****\\

July 16:\\
The British Museum is always a highlight of a London trip. No matter if you like old Egyptian, Greek, Mesopotamian, Roman, Buddhist, Islamic, or Persian art, the British museum has artefacts for you. Also rooms full of marbles from the Parthenon, which Athens would love to get back to put them into the acropolis museum though. And on top of it all, you don't need to worry about ticket prices, since the museum is all for free (just like the large natural science museum, or also the national gallery).\\

London: British Museum*****\\

July 17:\\
The second largest dome of a church in Europe, offering the most superb views of the City of London. St Paul's used to dominate the London cityscape for centuries. Now overshadowed by many skyscrapers, the interior is still very nice to see (albeit a bit pricey, after all it is an Anglican cathedral). Back then photos could be taken, that has changed as well nowadays. Anyways you can also find Wellington's and Nelson's tombs in the large crypt.\\

London: St Paul's Cathedral*****

\section{October xyz: move to Meyrin and the CERN hostel}
\label{moveMeyrin}

And this marks the first move to a different place in my life: For my master thesis I stayed at the CERN hostel in Meyrin in order to be close to the group I would be working with over the course of the next 5-6 months. I had my single room, with a shared kitchen, no TV but internet (the usual a physicist needs). Watching youtube got me a warning since the signature at least back then was similar to p2p file share usage, and since that was mainly used to illegally share music or movies it was clearly not allowed. It felt like starting a job and all of that abroad.

\section{October 28: Geneva}
\label{2006:Geneva}

Having arrived in Meyrin just a few days before, the closest big destination is Geneva. The harbour of the city is characterised by it large 150 m high water fountain, the so called Jet d'Eau. It is usually on from 6 am to midnight, in winter or in heavy winds it is usually switched off. In summer time the sun often creates a rainbow. The cathedral is one of the largest churches in Switzerland, particularly the early Romanesque stone masonry and capitals are of value. Beautiful neogothic windows can be found in a side chapel, the towers can be climbed too with nice views of the harbour.\\

Geneva: cathedral****, Jet d'Eau*****

\section{October 29: Geneva}
\label{2006:GenevaII}

Geneva has many parks with former Villas, one of those is the Villa Ariana, containing the local porcelain and ceramic's museum. More interesting if the Palais de Nations constructed as headquarters of the League of Nations it is still the second largest seat of its successor the United Nations. Many meeting rooms can be visited, the old former main assembly of the league, as well as the current general assembly hall. I enjoyed the modern conference hall with its beautiful painted and sculptured ceiling,\\

Geneva: Villa Ariana***, Palais de Nations*****

\section{November 10: Geneva}
\label{2006:GenevaIII}

Geneva's former city hall is now used by the Great Council of the canton. Several meeting rooms in Baroque or Baroque revival style exists and can be visited on tours during the local Escalade city festival. Otherwise no tours of the Hotel de Ville are offered, but the nice courtyard is always accessible. The large neogothic church of Notre Dame is the largest catholic church in town. Although once famed as protestant Rome nowadays catholics are in the majority, and there are even efforts ongoing to create a bishop seat in Geneva again.\\

Geneva: Hotel de Ville****, Basilique Notre Dame***

\section{November 25: Chamonix - Mer de Glace}
\label{2006:Chamonix}

After having started to work at CERN only a couple of weeks early, my parents and my younger brother visited me once more, this time they had a visit of Chamonix in mind. I'll never forget the first up-close view of the snow covered Mont Blanc massive just after getting through the tunnel by Les Houches dominating the full view just ahead of you. The Aiguille du Midi cable car was closed for its yearly inspection, thus we opted to take the train up to Montenvers by the Mer de Glace. The view of the glacier looked absolutely magnificent and the glacier ice was still shiny blue (far shy from the dirt and grey gravel you see nowadays). As the ice cave had been closed down as well, we all decided to walk along the glacier edges and to take the ladders down as far as we could (only to climb them up again and then getting back into Geneva). All in all a really great introductory visit to the valley of Chamonix.\\

Chamonix: Mer de Glace*****\\%done
\chapter{Year 2007}
\label{2007}

\section{March 9: Z\"urich}
\label{2007:Zuerich}

Kreuzkirche*, Peterskirche*, Fraum\"unster**, Grossm\"unster*

\section{April 9: Zurzach}
\label{2007:Zurzach}

Verenam\"unster**, Roman Castel*

\section{April 14: Inner Switzerland}
\label{2007:Switzerland}

Monastery Einsiedeln***, Schwyz: Old City*

\section{April 21: Meersburg \& Birnau}
\label{2007:Meersburg}

Schlosskirche Meersburg*, Wallfahrtskirche Birnau**

\section{June 3: Saleve}
\label{2007:Saleve}

Saleve**

\section{September 9: Chateau Chillon \& Lausanne}
\label{2007:Chillon}

Chateau Chillon***, Lausanne: Cathedral**, Cathedral Tower**

\section{December 16: Zermatt}
\label{2007:Zermatt}

After having been in Geneva for over half a year but without doing barely anything I decided it was time to take on Swiss mountains, this time going to Zermatt on the last weekend before Christmas. On the way I thought that Sion looks so nice with its castles, that i should maybe get off and pay a visit, but then I decided to stick to the original plan. I thought the Santa Claus church tower of St Niklas looked very cute. I enjoyed a first view of the Bisgletscher over Randa before finally arriving in Zermatt where I was blown away by seeing the Matterhorn for the first time. Clearly I had made my choice early that the Kleinmatterhorn was the peak to take the cable car up to, and indeed it didn't disappoint. I visited the glacier palace, did enjoy the view from the platform and had lunch in the cafeteria. Once I was down in the village again i had a quick view of Zermatt's church (nothing interesting to see there though).\\

Kleinmatterhorn***, Zermatt: church%done
\chapter{Year 2008}
\label{2008}

\section{January 2--January 4: Dresden}
\label{2008:Dresden}

January 2: Arriving in Dresden:\\

Kreuzkirche*, Frauenkirche**\\

January 3: Meissen:\\

Meissen: Dom*, Albrechtsburg**, Dresden: Gl\"aserne Manufaktur\

January 4: Dresden:\\

Zwinger***, Hofkirche**, Frauenkirche**, Gr\"unes Gew\"olbe**, Festung Dresden*

\section{January 17: CHIPP School 2008 and St Gallen}
\label{2008:StGallen}

In January 2008 the first ever CHIPP winter school took place at the Linth Arena in Naefels with its own attached large indoor swimming pool. The lectures focused on LHC physics, but covered other topics as well, the neutrino lecture was given by Takaaki Kajita, who received the Nobel Prize in 2015 for his discovery of Neutrino Oscillations. On one of the days we had a free afternoon which I decided to use to go to St Gallen and see its beautiful cathedral. It had been over a decade since I had last seen this church. One of the late large Baroque churches in central Europe it is very ornate, classical influences can be already seen in the choir area. It was built as the church for the Abbey of St Gall, and later pronounced the cathedral of the newly founded Diocese of St Gallen. Unlike other Baroque churches a large wide rotunda dominates the main nave with a large fresco depicting the return of god.\\

St Gallen: Cathedral***\\

\section{May 3: Neuchatel, Solothurn, Bern, Sion and Lausanne}
\label{2008:Swisstrip}

Neuchatel: Collegiate**, Solothurn: Ursenkathedrale**, Jesuitenkirche**, Bern: M\"unster**, Sion: Tourbillon**, Lausanne: Hafen*

\section{July 18: Annecy}
\label{2008:Annecy}

Cathedral*

\section{July 27- August 6: Philadelphia}
\label{2008:Philadelphia}

How this trip came into existence:\\
My project was supposed to start in autumn 2008. Thus there was a vital interest in studies, which could be performed in first data. Each experiment of the project typically hands in a couple of abstracts. Some of those are accepted for a talk, others might be merged, or offered to be presented at poster. From my working group we had the offer of one talk covering several topics and one poster, covering one. Since the conference -- ICHEP 2008 -- was taking place in Philadelphia, I was eager to go to the US for the first time in my life. This meant though hard 3 months of very long hours, dedication and write up of public document, and we managed to get the approval from the collaboration that the results were mature enough to go public. Having started to work in early 2007 this was also the time to get my first own credit card (even late for European standards), but something got stuck and I received the card only two weeks after coming back from the US. Since the trip would be reimbursed later anyway, the costs were taken over directly by the institute. Thank God the institute's secretary was a very friendly lady, thus i wasn't too much yelled at for causing this extra work for her.\\

Co-travellers:\\
My first work trip, thus many many physicists, alone from my university my professor, one postdoc and two other grad students were attending.\\ 

July 28: My first ever transatlantic flight:\\
Connecting over Frankfurt, the first Lufthansa flight still had Sandwiches although Lufthansa catering was on strike. On the bus we passed and airplane and I thought already, that airplane looks pretty dirty compared to those around it -- and guess what, that was the US Airways Boeing, which I had to sit in later on. Up to now still the worst transatlantic flight I ever had, not that the crew was unfriendly or not attentive, they just claimed because of the Lufthansa catering strike they had for each passenger only one sandwich as food for the whole flight -- underwhelming. Immigration was pretty quick, by then the Immigration officer already even knew about our conference - I also got the obligatorily ``Swedish people are always so nice'', when I told the officer I came from Switzerland. Once I arrived at the hotel I was very surprised and pleased to see it had a rooftop pool.\\

July 29: Time to explore the city:\\
Philadelphia is the sixth largest city of the US. Pretty nice for a European to see a real skyline, which is something not many cities here back in Europe have to offer. That comes for most US cities usually at the expense of having any real old buildings or history of hundreds of years attached to it. I stopped by the cathedral, which was built in mid 19th century in a baroque revival style, and the city hall (known from many movies, e.g. 12 Monkeys). Philadelphia used to serve as first capital of the United States. It was also here, where the US Declaration of Independence had been signed. Unfortunately all slots to go inside Independence Hall had been filled, thus I didn't get to go inside. but saw e.g. Liberty Bell, the Second Bank of the United States and further buildings of the historical district. Most buildings there can be visited free of charge, check out Carpenter's Hall, the meeting place of the first Continental Congress, also home to one of the first US flags.\\

Liberty Bell*, Carpenter's Hall**, Historical District*, Cathedral*\\

July 30-August 5:\\
The conference itself took place at the University of Pennsylvania. In the evenings we hung out by the White Dog Cafe and the Black Cat tavern. This was my first major conference, so everything was new and exciting. I learnt a lot about all different aspects of my field, met many new interesting people. The conference has one day free for social excursions, some of my colleagues decided to rent a car and go up all the way to New York City, I decided to stay in Philadelphia and check out the Philadelphia Museum of Art (also the Rocky Statue on the side). It was a nice museum, but pretty much the same you can see in European places. The conference dinner was also a buffet in said museum. Overall I spend some time in coffee shops but rather focused on the conference itself than doing too much touristy things. Overall it was fun though. The flight home was nice and relaxing, particularly on the flight from Frankfurt there were only a dozen of people in this A320, so quite some space and nice views of the alps.\\

Philadelphia Museum of Art**

\section{September 3--September 9: Madrid}
\label{2008:Madrid}

September 4:\\
Royal Palace***,San Isidro**, Almudena Cathedral***, Campo del Moro*\\

September 5:\\
El Escorial: Habsburg Palace \& Monastery**\\

September 6:\\
Segovia: Aqueduct***, Cathedral***, Alcazar**, Madrid: Prado***\\

September 7:\\
Monasterio de las Descalzas Reales**, Iglesia de San Miguel*, Museo Reina Sofia***, San Francisco el Grande***\\

September 8:\\
El Retiro Park**, San Jeronimo el Real*\\

September 9:\\
San Antonio de los Alemanes**, San Andrea**, Nicolas de los Servites*


\section{October 5: Jungfraujoch}
\label{2008:Jungfrau}

Even back in the 80s my parents always discussed if we should go up Jungfraujoch. Even back then being the most expensive ride in the alps, my parents decided it was just not affordable for a family of 6. Since I started now with my first regular job, I thought it would be a nice opportunity for a weekend out. I asked Sigi, Anne and Gianrico if they would have time as well. While Gianrico had others plans, Anne and Sigi were on board.\\

Co-travellers:\\
Sigi and Anne, while Sigi was now working as a PhD student in condensed matter physics, Anne had started here PhD in mathematics, so for all of us a day out in the mountains is a refreshing getaway thinking about work.\\

We were supposed to meet up in the train, but since the weekend was one of the late nice autumn weekends trains were expected to be so full, that an additional train was added for people coming from Bern. Thus we actually only met up in Interlaken where we decided to get to Grindelwald and get out return tickets up there. Even with the half fare card it is still 90 CHF for the last part (both from Grindelwald and Lauterbrunnen). I didn't inform myself properly about the procedure, thus at the Eiger Nordwand station I stayed on the train. On the way up the train stops at every station for around 10 minutes, this gives you time to leave the train take a couple of photos by the giant windows of the station and then getting back in time. Later trips up to Jungfrau always confirm the Nordwand station is not as impressive as the stop later - the Eismeer station. By this station you see the start and the icefall of the Grindwald-Fiescher glacier with the Eismeer glacier in the distance just below Schreckhorn. By the year 2000 both glaciers united to form the Lower Grindwald glacier by now this tongue has almost completely collapsed, thus this glacier will be only a note in history after the next 10 years or even earlier. Anyway we got to the final destination, the Jungfraujoch, on a saddle between the Jungfrau and the Moench mountain peaks. We first enjoyed the view down the valley as well as the impressive Aletsch glacier, with over 25 km length the largest of the alpine glaciers. Next we went up the Sphinx observatory, still used as a weather station nowadays, and then all the way down to the large glacier palace carved inside the ``eternal'' ice of the Jungfraufirn. Then we had lunch and a slow walk along the paved path. On our way back we asked at Kleine Scheidegg if we could exchange our ticket for Grindelwald to Lauterbrunnen instead. Indeed the valley of Lauterbrunnen is by far the more beautiful valley, in fact on most postcards and advertisements of Switzerland it is shown most prominently as THE alpine valley to see, dominated by falling cliffs and the Staubach waterfall. Rumour has it that Tolkien was inspired by the valley to come up with his descriptions of Rivendell. In Bern our paths split up again, but it was nice to see Anne and Sigi once again, not knowing that these occasions would get a bit more rare in the future to come.\\

Jungfraujoch***, Kleine Scheidegg**

\section{November 29--November 31: N\"urnberg}
\label{2008:Nuernberg}

November 29:\\
Kaiserburg**, Sch\"oner Brunnen**\\

November 30:\\
Jakobskirche*, Elisabethkirche*, St Lorenz***, Frauenkirche*, St Sebald**, Egidienkirche, Lochgef\"angnis, Weihnachtsmarkt**\\

November 31:\\
Kongresshalle***, Haupttrib\"une Zepellinfeld**

\section{December 3: St Blasien}
\label{2008:StBlasien}

Dom**

\chapter{Year 2009}
\label{2009}

\section{January 3: Reichenau \& Konstanz}
\label{2009:Reichenau}

M\"unster Konstanz**, Reichenau: St Paul**, Kloster**, St Georg***

\section{February 1: Nyon}
\label{2009:Nyon}

Nyon: Chateau***\\
Lausanne: Cathedral****

\section{March 21: Switzerland}
\label{2009:Switzerland}

Basel: M\"unster****, Fribourg: Cathedral***

\section{April 11: Rheinfall \& Rheinau}
\label{2009:Rheinfall}

Rheinfall*****, Rheinau***

\section{April 25--April 30: Madrid DIS2009}
\label{2009:Madrid}

The DIS2009 (Deep-Inelastic Scattering) conference wanted to hear about the prospects of the LHC regarding event-shape measurements at the LHC. Since that was the topic of my PhD thesis and we all hoped that we indeed would get collision data this year, my boss nominated me for that talk. I was selected and got myself a hotel by Madrid Chamartin. The conference itself took place at the congress centre just opposite of the Bernabeu stadium of Real Madrid. Having been in Madrid just the year before for the first time, I didn't stay for any additional days.\\

April 25: Saturday:\\
Having arrived a bit earlier I decided to go to the city centre, visiting the Almudena cathedral. The cathedral took about 110 years to be finished only in 1993. The outside is of a neobaroque style while the interior leans more towards a gothic outlook. The ceiling is very colourful, the same hold for the square cupola with a large traditional style apse painting. The crypt was free to visit back then, built in a neo-romanesque style.\\

Almudena Cathedral*****\\

April 29: Wednesday:\\
This was the free afternoon, I decided to go to Toledo. Since that train wouldn't leave until later I chose to do a short trip to the outskirts of Getafe, to see the cathedral there. The retabel was really nice, the church itself nice too, particular considering it had been a normal city church for the longest time. Back by the Atocha station i still had time to walk through the palm gardens which have been set up in the middle of the station hall, pretty unusual and unique. Once I arrived in Toledo after a short train ride, I climbed the stairs up to the Alcazar and then walked along nice little roads to the cathedral, on of the largest of the country. Back then photography was not allowed (that changed in the meanwhile). The decoration is breathtaking, from side altars, a large main altar (though that one can only be viewed through a giant screen) and the fantastic baroque El Transparente altarpiece on the outside of the choir screen, covering the full height of the cathedral with statues literally climbing up the altar walls up to a window. The sacristy is also pretty nice, filled with several paintings by El Greco. Another church which I enjoy visiting in Toledo is the large baroque Jesuit church. There you can climb on the roof with nice views of the old town. On my way back I walked up a little hill top to see the Castillo de San Servando, which overlooks the old down of Toledo on the other side of the Tajo river.\\

Getafe: Cathedral****\\
Madrid: Atocha Palmgarden****\\
Toledo: Cathedral*****, Jesuitchurch****, Castillo de San Servando***\\

During the evenings I had a couple of dinners in the old town of Madrid, for example the Plaza Mayor, and walked through the little parks by the Royal palace. The conference dinner was at a fancy roof top restaurant but other than that I just flew home on the last day of the conference, so no time to see anything else.

\section{June 6: K\"ussaburg}
\label{2009:Kuessaburg}

Once again I went home to see my family again, and since there is only so much variety of very near sights, we walked up to the ruined castle of K\"ussaburg once more. This time of the year all the walls were covered in fully green ivy. The views were as usually nice both of Klettgau and the Rhine valley, particularly since this time dark rainy clouds (before the storm) created a pretty eerie athmosphere. We walked around the inner and outer moat for the complete experience, always nice to return.\\

K\"ussaburg****

\section{June 23--July 8:CTEQ09 \& New York City}
\label{2009:US}

June 23:\\
I didn't want to take a bus from Chicago to Madison, so I decided to fly instead via Paris and Cincinnati to Madison. Having arrived in Cincinnati I had to spend quite boring 6 hours at the airport, and then about 30 minutes before our flight was supposed to depart, Delta cancelled our flight due to a technical problem. We were told we could talk to the call center or queue by the counter. I decided to go to the counter, just seconds later people found out the call center didn't work anyway. I was told I could get on a flight to Chicago O'Hare, then taking the bus from there (exactly what I had wanted to avoid originally), but what can you do. I was told to rush the flight would leave in 20 minutes, I still hoped for my luggage to get on board, at least they were happy to hear an Air France checked bag would look very distinct. Three more guys were from my original flight were on the one to Chicago. Surprisingly my bag did make it on the plane so once I got it, I got on the bus to Madison, but when the driver asked me where I wanted to get off, I just had no idea. Seems there are three stations, and since I didn't plan being on a bus, I didn't inform myself about where bus stations are located at. I told him I would get somewhere close to the University, so I was told to get off at Union station, the last one of this bus. About 3 1/2 h later, by this time I had been awake longer than 24 h, we finally arrived at Union station. I got off, got into a taxi with 2 other folks, told the driver where i wanted to go, he drove for about 45 minutes, I got off, gave a generous tip, and finally checked in (having told them I would arrive about 5 h earlier), and finally got some sleep. Waking up the next day I walked to the auditorium ONLY to realise that the bus station was only 500 m or a 5 minute walk away from the hotel. Since then I have sworn to myself never to get on a taxi again, unless there is absolutely no other possibility.\\

June 24 \& 25: Wednesday \& Thursday\\
Madison didn't feel like a big city to me, it clearly does have a large amount of citizens, the Madison area has roughly the same inhabitants as the canton of Geneva, but they all seem to be scattered over a much larger surface. Old town is also only a couple of streets and it felt like the giant Wisconsin State Capitol rises out in the middle of like nowhere. it is a neoclassical building erected between1906-1917 with four wings and a central 57 m high dome, with pediments on each of the four front facades.The rotunda has two levels plus the dome on top of it, where the upper level is decorated with several dark pillars with golden leaved capitals and four large mosaics on the vaults. Since no building in the vicinity is allowed to surpass the capitol's heigh there are not many high rises in Madison downtown. I tried to find the cathedral of Madison as well - unsuccessfully and read up back home in the hostel that indeed the cathedral had been burnt down just a year previously and the remains were torn down thus no wonder I only saw green grass where google maps told me to look for it.\\
The following day I returned to the State Capitol for a couple of more photos, this time also going into the Senate assembly hall. I also did enjoy some beers out with my colleague J\"urg, Andrea, Cosmin, and other people I had met at the CTEQ summer school, particularly sitting in sunset by the calming water of Lake Mendota.\\

Madison: Wisconsin State Capitol****\\

June 28, Sunday: Chicago:\\
After having sat on the bus from Chicago to Madison I wasn't scared to do the same bus ride again. Unfortunately everybody else had seen Chicago at that point already, thus I was all on my own. Once I arrived in Chicago I tried to get up on Willis Tower, but I was told the observatory would open late due to heavy winds at that time. Thus I walked across the loop to the Shedd Aquarium, one of the best aquariums I have been to, with turtles, sharks, see horses, artificial coral reefs (yes also with clown fishes), frogs, leguans, crabs, sea stars, lobsters, just everything, and also Beluga Whales.\\
I had a short lunch before taking a ferry across Lake Michigan to reach Navy Pier, where i had another drink in the beergarden walking through the indoor gardens there, ending up on the top of John Hancock Center and enjoying the skyline from up there. I passed by the cathedral, which was though closed for renovation works. I walked a bit along the Chicago river taking snapshots of Trump Tower (who would know back in 2009 what we would be up to in 2016, that name didn't age well).\\
Having arrived by Willis Tower again, the observatory was open, and clearly I had to go up and enjoy the Loop are of Chicago downtown from a different perspective while admiring the wide sea of house beneath me to the west and south. Then I sat in Union Station for another half an hour waiting for the bus.\\

Chicago: Willis Tower*****, John Hancock Center*****, Navy Pier***, Shedd Aquarium*****, Millenium Park**\\

July 3: New York City:\\
At least this flights went without major issues, the flight into Detroit departed and landed on time. On my flight to NYC I was bumped up to business class (well better snack for me that is), but then our landing was delayed by over an hour. A light rail and then public train to Philadelphia Station later soon after I arrived by the Hilton Times Square hotel in a gigantic room (for my taste) with a nice view of Empire State Building. But once in New York you have better things to do then stay in a hotel. Thus I walked straight over to Times Square, iconic, and also a wonderful feeling to be in a busy metropolis. By that time the sun was setting, and I walked over to St Patrick's Cathedral, another neo-gothic large church with nice stained-glass windows. I had already booked tickets for the Empire State Building, but unfortunately I was not really alone on the Eve before Independence Day, thus over an hour wait later I arrived by the platform, and I was blown away. Midtown Manhattan is just breathtaking, a gigantic wonderful shiny city on the bottom of your feet from up there. Also the streets which go like bright shiny veins over the island and the illuminated bridges looked just wonderful. And all below an almost full moon. I did enjoy it a lot, no matter the wait.\\

New York City: St Patrick's Cathedral****, Empire State Building*****, Times Square*****\\

4th of July: New York City:\\
Independence day 2009 in New York City, sounds exiting, isn't it. Since hovering around in NYC all on my own sounded like a bad idea, I already tried beforehand to find someone local to show me around, after all what is the internet there for. And I found a wonderful guide, even a local artist, Kendal, who thought it would be fun to show a nerdy physics PhD student the wonders of the town. First I made my way to Grand Central Terminal admiring the beautiful train station and hub of public transportation in NYC. The Main Hall is indeed very impressive, also from a European perspective curious to see the many large US flags hanging all over the hall, maybe also just due to it being the Independence Day weekend. \\
And this is where I met up with Kendal. As I expressed my interest in skyscrapers before he brought me to the lobby of the Chrysler Building, another one of the iconic NYC skyscrapers which had for a short moment been the tallest tower on Earth before it was surpassed by the neighbouring Empire State Building. While it doesn't have an observatory, the lobby itself was really nice, with art deco style marble walls, murals of the Chrysler building on the ceiling, maybe make that extra mile to see it. Then Kendal and I had breakfast at Bryant Park, and we had to take photos of Empire State Building as well. We stopped by Madison Square Park to get a glimpse of the triangular shaped 87 m high Flatiron building from 1902, which some consider the first skyscraper ever built. After one more snack at Union Square we walked past New York City Hall over to Ground Zero, where construction of One World Trade Center had started. \\
Then it was time to say goodbye to Kendal for now, while I walked across Brooklyn Bridge with nice views of neighbouring Manhattan Bridge and freighters on East River. Passing by St Paul's Chapel (nothing too special), Trinity Church is another well-known neo-gothic church in NYC, but if you know European churches it all feels a bit underwhelming. Walking past Wall Street, Federal Hall, the Standard Oil (nice art deco) building and Fort Clinton I ended up at Battery Park where I enjoyed some moments. I asked people where this year's fireworks would take place, since I had read it does sometimes happen over East river, at other time over Hudson River, I was told it would be by Hudson River, thus I moved already over there, where many folks had already gathered by then.\\
About two hours lighter boats paraded over the river shooting out fountains of water during sunset. About an hour later the fireworks started. Clearly the fireworks were outstanding and gigantic, spanning along the river. I was impressed even after being spoiled by the Fete de Geneve fireworks already the year before (and the following years to come). There wasn't a display of music coming with it but all the formations and patterns and colours were breathtaking, sure up there with the best.\\

New York City: Grand Central Terminal****, New York Public Library***, Brooklyn Bridge*****, Manhatten Bridge****, St Paul's Chapel**, Trinity Church***, New York Stock Exchange**, Battery Park**, Independence Day Fireworks*****\\

July 5: New York City:\\
On my second day I had book the morning sunset ticket of the Rockefeller observatory, the Top of the Rock. Unfortunately unlike for the Empire State Building glass walls surround the terrace nowadays, as high as people size. Clearly if the sun is shining against them or they are not cleaned properly (and who would do that every day), the view is a bit obstructed. On the other hand this is the best view you can get of the Empire State Building, and also the iconic birds-eye view of St Patrick's Cathedral, and not to forget the view of Central park which just gives you a feeling why it is called the green lung of NYC. After a short visiting intermezzo at St Patrick's Cathedral, I took the Metro up north to see St John the Divine. This neogothic anglican cathedral is in fact still unfinished, missing the top of its towers as well as the southern transept. The church is with a length of 183 m and a height of 54 m one of the largest in the world.\\
 Afterwards I visited the American Museum of Natural History withs its Meteors and crystal collection as well the dinosaur collection. I then had a walk through Central park with its forests, ponds, and squirrels, before visiting the Metropolitan Museum of Art. The MET has tons of things to see, the screen of the Spanish cathedral of Valladolid, period rooms all the way from Paris, the egyptian temple of Dendur, islamic art such as the Jain Audience Hall, sculptures from Canova, Mesopotamian gates, and many paintings, such as the Cypress painting by Van Gogh, and I finished the day with a further walk across central park. And then came the second part of the Top of the Rock ticket, the night view to admire the Empire State building in its epic Independence Day outfit of red, blue, and white.\\

New York City: Rockefeller Center*****, St Patrick's Cathedral****, Cathedral St John the Divine****, American Museum of Natural History****, Central Park****, Metropolitan Museum of Art*****\\

July 6: New York City:\\
Featured in several movies the New York Public Library is one of the largest of its kind, with nice coffered and painted ceilings the reading rooms, galleries, and entrance halls. I was looking forward to seeing the Headquarters of the United Nations, bought a ticket which listed 5 different places I could see, only to be informed after getting through security that due to ongoing meetings solely the main assembly hall would be open today. Clearly they either could have given us a ticket at a reduced price or told us to come back maybe a day later. After that disappointment I admired the Museum of Modern Art, a fantastic collection including the who is who each with multiple paintings, a must to visit for any lover of modern paintings, you might have heard of Van Gogh's Starry Night, Warhol's Campbell's Soup Cans, Picasso's Demoiselles d'Avignon or Matisse's Dance (and many many more). Since there wasn't much more to do I decided to see a couple of churches instead, like another neo-gothic church of St Thomas (beautiful sculptured high altar), or the neo romanesque/gothic style mix of St Bartholomew.\\
When in a big city you want to do something special, for me that was watching a movie in IMAX, which didn't yet made it across the ocean yet, the only movie they showed at IMAX at that point was Transformers II. The clerk at the cinema told me if I really want to pay that much money for IMAX, considering the price was only half of what a normal ticket in Geneva was clearly I wanted to do that - also got my giant cup (for sure about 1 l inside) of coke and hot dogs for a good real American cinema experience - well besides the fact that it still was Transformers II, oh well you know what you get by Michael Bay.\\

New York City: New York Public Library***, United Nations**, Museum of Modern Art*****, St Thomas Church****, St Bartholomew****\\

July 7: New York City:\\
When in New York City you might want to visit Liberty Island. Clearly I wasn't the only person with that idea. Consequently all tickets to reach the island from Manhattan had been booked. Nevertheless there were several tickets available to do the same trip from the New Jersey side. Thus I took a ferry from Financial District over to Jersey City. Since I had a bit of time I walked into the Central Railroad of New Jersey Terminal, an abandoned large train station where I got myself breakfast and coffee before taking a ferry to Ellis Island. There the museum elaborated on former times of immigration when people were scanned and their immigration forms processed at Ellis Island Terminal, and then the Ferry took us around Liberty Island. Since I had booked the platform ticket I climbed the State of Liberty monument up to the feet of the statue, from there I had a view up the inner skeleton. The tickets to walk up to the crown had been sold out way in advance though. And back to Jersey City and then to Ground Zero it was. \\
That afternoon I met up with Kendal again and we walked through the Meat Packing District along the old refurbished track which make up High Line Park. A very beautiful way to revitalise the area, besides adding green and refreshing spaces in that part of the city, also art installations are scattered throughout the park. Clearly I had to finish the day taking some night photos of Times Square.\\

Jersey City: Central Railroad of New Jersey Terminal***\\
New York City: Ellis Island****, Statue of Liberty****, High Line Park*****\\

July 8: New York City:\\
Seems I had not enough planned for the last day, thus I just did a curiosity museum which was just next to the hotel and advertised by the hotel itself, in the end it delivered also nice items like ivory carved ships from China, wood-carved ships from Japan, parts of the Berlin Wall, Napoleon's death mask among other things. Having planned nothing more I just went to airport already now where I was informed that it was far too early to receive my bag, thus I had to walk around with that clunky suitcase or take it with me while having dinner. The flight in the Air France Boeing 747 was pretty uneventful besides flying into Charles de Gaulle with the best view of Paris I ever had from a plane, unfortunately stupid me didn't kept the camera out of the hand luggage area.\\

New York City: Curiosity Museum**

\section{July 26: Valais}
\label{2009:Valais}

One more weekend in Switzerland which I decided to spent in the Rhone valley. I started out in St Maurice. In this village you find not only an ancient abbey and one of a couple of mountain forts of the Swiss army, but also a nice little cute cave with no flowstones, but with the largest cave waterfall within Switzerland which had still quite a bit of water even in July. A few stops later I arrived in Sion, an old bishop seat with an old but pretty plain cathedral. In fact I was a bit disappointed, but then a couple of visits later I warmed up to it a bit again, after all the altar piece is alright, but there is hardly any other decoration available. On two hillsides close to old town are two castles, one which is more or less a fortified old basilica with one of the oldest functioning organs in Europe, and a ruined former castle of the bishops of Sion on Tourbillon. This time I managed to get inside unlike last time where doors were shut basically in our face. In St Leonhard the largest underground lake in the Europe can be found. One wall of a cave collapsed and the cave filled up with rain water after a while, still nice to have a cute little boat tour on it. Only at the end of the cabe you have permanent lights installed, but on the boats are a couple of spotlights which did give enough lights for decent pictures, even with my old digital camera which for sure isn't anywhere close to the quality of mobile phone cameras or even an SLR. And last but not least the castle of Stockalperpalast, a private palace, which is in fact the largest Renaissance palace in Switzerland with three big towers and a nice courtyard, with its arcaded walks it reminded me of Italy. Stockalper in fact also made his money on the trade route of the Simplon leading after all to Lombardy, so for sure the inspiration came from places south of the alps. Nowadays the local court and the city hall of Brig are situated here, only on guided tours the general public can see some of the rooms nowadays.\\

St Maurice: Grotte de Fees****\\
Sion: Cathedral**, Tourbillon****, Valere****\\
 St Leonhard: Lac Souterrain****\\
Brig: Stockalperpalast***

\section{August 16: Wutachschlucht}
\label{2009:Wutachschlucht}

One of the longest gorges and canyons in Germany is the Wutachschlucht, where the river dug a deep valley of up to 80 metres. There are plenty of trails which altogether are claimed to be 50 km long. The gorge has several side gorge and multiple entries which can be also done in multiple short hikes. There are plenty of waterfalls around, on one spot the river even disappears and reappears again after about 20-30 metres in hot summer at least. There are also slot canyon type places from time to time (small, but they do exist). In ancient times the river continued its flow to the Danube, but digging so deep the connection to this valley was lost and nowadays the river flows into the Rhine. From the former river valley a creek falls down in several cascades into a little side canyon of the Wutachschlucht, but that side part I've only done as late as 2021.\\

Wutachschlucht****

\section{August 23: Zermatt}
\label{2009:Zermatt}



Zermatt: Gornergrat*****

\section{August 30: Aareschlucht \& Tr\"ummelbach Falls}
\label{2009:Aareschlucht}

Meiringen: Aareschlucht*****\\
Stechelberg: Tr\"ummelbach Falls*****

\section{September 1--September 5: Innsbruck SPS \& OEPG 09}
\label{2009:Innsbruck}

September 1:\\

Hofburg****, Hofkirche*****, Dom*****, Jesuitenkirche***, Basilika Wilten****, Stiftskirche Wilten****

September 5:\\

Schloss Ambras****, Nordkette*****

\section{December 18--December: Karlsruhe et al}

December 18:\\
Schloss Karlsruhe****\\

December 19: Mainz \& Frankfurt\\
Mainz Dom****, Frankfurt: Bankenviertel****, Dom****, Paulskirche***\\

December 20: Speyer, Mannheim, and Worms:\\
Speyer Dom*****, Mannheim: Schloss****, Jesuitenkirche****, Worms: Dom*****\\

December 21: Heidelberg:\\
Heidelberg: Schloss*****, Heilig-Geist-Kirche****, Jesuitenkirche***
\chapter{Year 2010: the year without any flight}
\label{2010}

Since 1996 i had at least one flight per year but this year was one without any flight. This was the year when i stayed so close to home, that no flight had been necessary. Something which I never repeated up to now (2020).

\section{January 19--January 25: Ancona, CHIPP 2010}
\label{Ancona2010}

This is the second ever CHIPP school taking place in the Monte Verita centre on a hillside over Ancona. Almost all fellow ETH PhD students were going as well. The lectures focused on neutrino physics this time.\\

January 19: Ancona arriving at Monte Verita\\
Starting out early in the morning I took hopped on the train with direction to Milan, only to get off at Domodossola transferring to the Cento Valli train over to Locarno. The wintery landscape was very nice and snow was all over. By the time we arrived in Locarno it was already night, then a short bus ride and a bit of a walk later I arrived up on the Monte Verita centre. It was not too late though to go down to the town for dinner, enjoying the nightly views of the harbour and Lago Maggiore.\\

Ancona: Monte Verita***, Harbour***\\

January 22: Bellinzona\\
The social program included a trip to Bellinzona, where we could either have the day to explore the town, or spent the day at the SBB Cargo centre. I opted for the town: Bellinzona is dominated by three castles which are part of the World UNESCO heritage, the largest being the quite impressive Castelgrande. Castel Montebello has a more interesting layout, unfortunately the interior is closed in winter, but it is still possible to climb the walls, at Castel Sasso Corbaro even that is not possible. The church of Lugano is an ordinary Baroque church.\\

Bellinzona: Castelgrande****, Castel Montebello****, Collegiata dei Santi Pietro i Stefano***, Castel Sasso Corbaro***\\

January 255: Lugano \& Como\\
Since a day ticket was only 2 CHF more expensive than a direct ticket back to Geneva, I decided to see a bit more of Ticino, this time Lugano the largest city of the canton. It was very rainy, the mountain overlooking Lago Lugano was covered in clouds. Thus I just stepped into the cathedral (as most of Switzerland's cathedrals, rather small and nothing special), and then wondered what I could do next. Checking the schedule I realised I would have enough time to get to Como and from there back to Geneva. Getting rid of my stuff in one of the lockers I jumped on the next train to Chiasso and from there on to Como. Clearly the weather didn't improve on this short ride, so I didn't get any good views out of the Lago di Como. But the cathedral in Como is a lot larger and more stunning than the one in Lugano, the Duomo is a richly decorated Baroque church wtih one large Dome. Afterwards I was pleasantly surprised by the church of San Fedele. And then I got on my way home including five transfers (and getting my luggage in Lugano) as well as a ride over the (nowadays old) Gotthard tunnel.\\

Lugano: Cattedrale**, Como: Duomo****, San Fedele****

\section{February 6--February 7: Chamonix}
\label{Chamonix2010}

February 5:\\
Grand Montets (too foggy)*, Mer de Glace (too foggy)*, Church*, Museum of Crystals***\\

February 6:\\
Aiguille du Midi*****

\section{April 5--April 9: Bavaria}
\label{2010:Bavaria}

April 5: Schwangau \& F\"ussen:\\
Schwangau: Hohenschwangau***, Steingaden: Wieskirche*****, Neuschwanstein****, F\"ussen: St Mang****, Spitalkirche***, Hohes Schloss***, Lechfall***, Lechklamm****\\

April 6: Garmisch-Partenkirchen \& Ettal:\\
Ehrwald: Zugspitze*****, Ettal: Kloster****, Linderhof*****, Venusgrotte*****, Garmisch-Partenkirchen: Eibsee*****, Skisprung-Schanze***\\

April 7: M\"unchen\\
Nymphenburg mit Badenburg,Amalienburg,Pagodenburg \& Magdalenenklause*****, Olympiastadion****, Asamkirche*****, St Peter****, Heilig-Geist-Kirche***, Michaelskirche****, Frauenkirche***, Theatinerkirche***\\

April 8: M\"unchen\\
Residenz*****, Schatzkammer*****, Cuviliestheater****, Deutsches Museum*****, St Anna im Lehel**, Klosterkirche St Anna im Lehel***, Dreifaltigkeitskirche***\\

April 9: Augsburg \& Ulm:\\
Augsburg: Rathaus*****, St Afra** (in renovation), Baroque Gallery in Palais Sch\"atzlein**, Dom***, Ulm: M\"unster*****

\section{May 23: Luzern \& Rosenlaui}
\label{2010:Luzern}

Luzern: Spreuerbr\"ucke****, Jesuitenkirche****, Kapellbr\"ucke****, Weeping Lion***, Rosenlaui: Rosenlauischlucht*****

\section{June 27: Fiesch}
\label{Fiesch2010}

Aletschgletscher*****, Eggishorn*****, Fieschergletscher*****, Moosfluh*****

\section{July 11: Rhonegletscher}
\label{Rhonegletscher2010}

Rhonegletscher*****

\section{July 21--August 4: Paris, ICHEP2010}
\label{Paris2010}

The largest conference of the year happened to be taking place in Paris. After having presented a poster on simulated data in 2008, I made sure to have my analysis ready on real data in early 2010. Naturally my poster abstract had been accepted without any problems. Since Paris is quite close to Geneva as well, my boss encouraged me to show my results. The results were also shown by more senior people in other talks. Since I had not been in Paris since 1998 I asked for a couple of days off, in order to see the city as well as the surrounding place once again.\\

Co-travellers:\\
For the conference part most of my colleagues and friends had been around, but other than that I spent most of the other days with my mum and dad.\\

July 21: Paris:\\
Having arrived at Gare de Lyon with the earliest train possible I decided to walk the two stations to my hotel, where I had a room to myself just below the roof-top (elevator was not existing), so just walked up the four floors, dropped my stuff and on to discover Paris. The hotel was in the quarter of Marais, thus only a short walk away from the Ile de Cite. On my way to the island I stopped by the church of St Paul \& St Louis. Then we got to real business: The Conciergerie is the only surviving part of the old royal medieval palace, the hall of the guards is still impressive, but it is more or less one interesting room. OK there is also the cell of Marie Antoinette and other people where they spent their last days before beheading by the Guillotine. One of the most beautiful gems of Paris is just a couple of metres near the Conciergerie, the Sainte Chapelle, also part of the former royal palace. The stained glass is just breathtaking, absolutely magnificent. Still after all these centuries a place to impress people. Many times people tried to replicate this beauty, Stephen's Hall in the Houses of Parliament is claimed to be modelled after Sainte Chapelle for example. But all of those attempts cannot recreate the special atmosphere of the upper chapel. The lower chapel is nice, pales though compare to the upper chapel. And once we saw the little gem, right on to the large cathedral, located on the island as well. Notre Dame is an early gothic cathedral, which was largely renovated in the 19th century. Whenever I was in Paris I paid my visit - until disaster struck and the whole roof went up in flames in 2019. Whatever happens I will make sure to be back once it should be possible again. Still refusing to take the Metro I made my way to the Pantheon, at least unlike in 1998 no scaffolding had to be placed on certain parts due to instability of the ceilings. It is one of the late classical buildings in Paris, but the more I visited the less impress I was. On my way to the Dome d'Invalides I strolled through the Jardins du Luxembourg and had a short look into St Francois-Xavier, before getting to the Hotel d'Invalides. Originally built for the wounded of the wars it now hosts the Army Museum (a bit meeh, was more impressed by the tanks back as teenager in 98). The Dome d'Invalides houses Napoleon I's tomb, as well as remains of his family members. The main altar is nice, as well as the large dome. The cathedral of the army, St Louis d'Invalides is attached to the Dome and has a lot of flags being flown. On my way to the old opera house I stopped shortly by St Madelaine, a church in classic style, modelled after ancient temples on the outside. Opera Garnier never fails to disappoint, the staircase, the eclectic upper Foyer as well as the amazing amphitheatre, particularly the giant fresco by Marc Chagall, one item more beautiful than the other. The most beautiful opera house I have seen so far.\\

St Paul \& St Louis**, Conciergerie***, Sainte Chapelle*****, Notre Dame****, Pantheon****, St Francois-Xavier***, Dome d'Invalides****, Hotel d'Invalides: Musee d'Armee***, St Louis d'Invalides***, St Madeleine****, Opera Garnier*****, St. Germain l'Auxerrois***\\

July 25: Versailles:\\
Chateau*****, Trianons*****, Cathedrale***\\

July 26: ICHEP 2010\\
President Sarkozy decided that ICHEP was important enough to give a speech. Rumour has it that the organisers wanted the minister of science to give a speech, but she was quite reluctant. Once Sarkozy heard about it, he decided to give the speech himself and the minister was relegated to sitting in the audience and clap. For the speech itself it was energetic, obviously in French, we had to go through security and metal detectors. Sarkozy told us how important basic science is. how much his government increased the funding. Our French colleagues commented on that by saying he counts also the military developments as part of that so the increase was far lower than claimed though. It was nice at least to witness a speech by a world leader for once.\\

Paris: Presidential Speech****\\

July 29: Chartres:\\
Chartres: Cathedrale*****, St Pierre****, St Aignan****\\

July 30: Fontainebleau:\\
Fontainebleau: Chateau*****\\
Paris: Tour Montparnasse*****\\

July 31: \\
Vincennes: Chateau***\\
St Denis: Stade de France***, Basilica*****,\\
Puteaux: La Defense****\\
Paris: Arc de Triomphe****\\

August 1: Paris\\
Paris: Louvre*****, Musee d'Orsay***, Musee de l'Orangerie****, Centre Pompidou*****\\

August 2: Paris\\
Paris: Sacre-Coeur**, Madelaine****, Place de la Concorde*****, Grand Palais***,  St Sulpice****, St Germain des Pres****, Tour Eiffel*****\\

August 3: Reims\\
Reims: Palais du Tau****, Cathedrale*****, Roof \& Towers of the Cathedral*****, St Remi****\\

August 4: Paris\\
Paris: Petit Palais****, Hotel de Cluny****, St-Gervais-et-St-Protais*, St Eustache****

\section{August 8: Grindelwald}
\label{2010Grindelwald}

Weather forecast claimed the day could be cloudy, but it should be a dry day and the clouds would be high up. Indeed most of the clouds were high up, thus one could see most of the North face of the Eiger. I took the gondola up to First and walked over to Bachalpsee. It is a pretty easy hike, which definitely can be done in a family outing or for non-experienced hikers. On the way one can enjoy the views of the Lower Grindelwaldglacier, while on the Firstbahn the Upper Grindelwaldglacier is in perfect view. From First itself the view over to Titlis and Grosse Scheidegg is nice too. On the way back I got into a 5 minute heavy rain fall, even my jacket didn't keep me fully dry. Still I took the bus to the Gletscherschlucht restaurant (had a little snack there) and walked through the Glacier Gorge of the Lower Grindelwaldglacier. A pretty nice cute canyon, where some people even took a zip line down the rock face, or climbed up. Some people even cross the touristic area and try to climb up to the glacier. That might have been suspended now that the glacier walls became unstable. As last bit I took the bus to the Upper Grindelwaldglacier.. Back in the 80s my parents took us kids to the glacier cave of the upper Grindelwaldglacier, back then the glacier had reached a maximum of the last 60 years and was very impressive. Digging a glacier cave had been given up since, even climbing multiple ladders and crossing over the deep gorge of the melting water, the glacier tongue was unreachable. Ten years later the lower part of the glacier has split of at the lower icefall and the glacier lost another 2 km of its length, so go there while you still can.\\

First \& Bachalpsee*****, Glacier Gorge of the Lower Grindelwaldglacier****, Ladders up the Upper Grindelwaldglacier****

\section{August 21: Hasslachh\"ohle}
\label{2010Hasslach}

Erdmannsh\"ohle****

\section{August 22: Hohenzollern Castle \& Rottweil}
\label{2010Hohenzollern}

Hohenzollern Castle***, Rottweil: Kapellenkirche**, Predigerkirche****, Heilig-Kreuz-M\"unster****

\section{August 27: Jura}
\label{jura2010}

Lac des Brenets*****, Saut du Doubs*****, Le Locle: Moulins Souterrains***, La-Chaux-de-Fonds**

\section{August 28: Zermatt}
\label{Zermatt2010}

Unterrothorn*****, Kleinmatterhorn*****, Matterhorn Glacier Walk*****, Gornerschlucht****

\section{September 4: Saas Fee}
\label{2010SaasFee}

Saas-Fee: L\"angfluh*****, Mittelallalin*****, Eisgrotte Feegletscher*****, Saas Grund: Triftgletscher*****

\section{September 5: Engelberg}
\label{2010Titlis}

Titlis \& Titlisgletscher*****, Kloster Engelberg****

\section{September 12: Chamonix I}
\label{2010ChamonixI}

Aiguille du Midi*****, Panoramique Mont-Blanc*****, Pointe Helbronner*****, Glacier d'Argentiere*****, Glacier des Grands Montets****

\section{October 2: Chamonix II}
\label{2010ChamonixII}

Aiguille du Midi*****

\section{November 13: St Blasien}
\label{2010StBlasien}

Dom****

\section{November 27: Zermatt \& Bern}
\label{2010ZermattBern}

Swiss Meteo claimed after snow fall really early in the morning everything should clear up and a we could have a beautiful afternoon in Zermatt, so time to take my brother to Zermatt. We decided that we should go up to Unterrothorn. The side of Strahlhorn and the Findelgletscher were in superb view, but the Matterhorn was completely covered in clouds. So time to built a snow man (which proofed to be pretty hard considering the consistency of the snow). Still no Matterhorn to be seen, so we were mad and gave up and decided to take the train to Bern, enjoying at least free views of the Bisgletscher by Randa. In Bern they just had set up the Christmas market, but not too many stands had been opened thus we just went to the M\"unster instead, Switzerlands largest church in gothic style, always cute to see.\\

Zermatt: Unterrothorn****, Bisgletscher****\\
Bern: M\"unster****

\section{December 4: Zermatt}
\label{2010ZermattII}

Gornergrat*****, Kleinmatterhorn*****
\chapter{Year 2011}
\label{2011}

\section{January 15: Chateau Chillon}
\label{2011:Chillon}

Not having had a trip over Christmas break both my brother and I were eager to get out somewhere. The chateau de Chillon is a castle located on a little island just a couple of metres away from the shores of lake Geneva. Still conserved in its original shape, the rooms are full of old frescoes and also some old furniture had been conserved. The cellars were once used as prison, one sees clearly the rock of the island. The belfry has four floors and offers quite a view. The area around the castle is really great too, particularly during sunset. This view was featured prominently on the album cover of Queen's Made in Heaven album, the last one featuring recordings by Freddy Mercury, whose statue is in the harbour of nearby Montreux as well.\\

Veytaux: Chateau Chillon*****

\section{January 16: Lausanne}
\label{Lausanne2011}

Another day, another outing with my brother, this time we just wanted to see a bit of Lausanne. The cathedral is an early gothic cathedral, also with pretty decent stained glass windows. The cathedral is up a hillside close to the two castles of the former archbishops of Lausanne. Nowadays protestants dominate the region, and the cathedral is not the home of a bishop anymore. The portals are decent as well. On our way back we stopped by Nyon. This little town was founded by the Romans, remains can be seen in the Roman museum as well as three columns of an ancient temple on a terrace overlooking Lake Geneva. My brother was rather disappointed.\\

Lausanne: Cathedrale****\\
Nyon: Roman Temple***

\section{January 23: Milano}
\label{Milano2011}

My very first time in Milan, a nice day trip ahead of us (considering it is just a 4 h train ride away). Getting on a 5:42 early train with brother, we arrived in Milan shortly after 10 (train had a bit of a delay, also happening often on this track). The old Castello Sforzesco houses many pieces of arts, among them the last big sculpture by Michelangelo, the so-called Pieta Rondanini. Certain halls of the castle have been painted by Leonardo da Vinci as well. After a walk through the old town, including a visit of the Galleria Vittorio Emanuele we visited the Duomo. Back then the cathedral was free for visits, and there was no special security check, nowadays it can be over an hour to get in just based on the very slow security checks. We also got a ticket for the roof terrace, really magnificent to see the decorations and pillars up-close as well as the (relatively small) dome and the golden statue on top of it. The square by the basilica of San Lorenzo is dominated by old roman remains, the so-called Colonne di San Lorenzo, the basilica itself is alright, but I would recommend to see the side chapel of San Aquilino with magnificent mosaics on the walls and the roof. Then we queued for the Palazzo Reale for an exhibition of Salvador Dali, but after hardly moving more than 2 m within a 45 min wait we gave up and rather had dinner somewhere else before taking the last train home shortly after 7 pm.\\

Milano: Castello Sforzesco*****, Duomo*****, Duomo Rooftop*****, Galleria Vittorio Emanuele II****, Basilica San Lorenzo****, Sant' Ambrogio***, San Sebastiano***, Colonne di San Lorenzo***

\section{February 13--February 19: Chamonix \& Les Houches}
\label{LesHouches2011}

Shortly after the LHC had started data taking first papers were published. Their findings were the main discussion point of a winter workshop at the Les Houches physics centre in the valley of Chamonix. My colleague Andreas and I were invited to talk about our measurements of QCD processes at CMS. We decided that it would be nice to see a bit of the valley before the workshop.\\

February 13: Flegere \& Bossons\\
Although we weren't skiing on that day, we went all the way up to the Flegere cable car stop, offering magnificent views of the snow-covered mountains and the Mer de Glace as well as Mont Blanc just opposite of us. We did a small walk around the area, but naturally that's hard to do considering we didn't want to walk along snow pistes and normal hiking trails were not prepared for winter either. So pretty soon after getting up we decided that we could try a bit of a small hike by the Glacier des Bossons. That glacier still reaches almost down to the valley, thus we parked the car shortly before the finale moraines which the glacier left in the 19th century. From there it was a short 45 minute hike to reach the current tongue of the glacier. Although it retreated quite a bit already in 2011, it was still a really impressive scenery.\\

Chamonix: Flegere****, Glacier des Bossons****\\

February 17: Argentiere\\
Having failed at getting good views around the Grands Montets area the first time around back in 2010 (see \ref{Chamonix2010}) I made my second attempt getting all up to the mountain top by cable car. Indeed this time it was sunny and I got great views of the Glacier des Rognons and the Glacier d'Argentiere. From the middle station of Lognon, the ice-fall is the Argentiere Glacier looked pretty great too, but I didn't walk over there, going along a ski piste didn't seem like a brilliant idea after all.\\

Argentiere: Glacier des Rognons*****, Glacier d'Argentiere*****\\

\section{April 3: Jungfraujoch}
\label{Jungfrau2011}

Once again a trip to Jungfraujoch, this time all alone. This time I made sure to get off already at Eiger north face. It is quite a sight, but I think the view from bottom up to the top is far more impressive. After the usual visit of the glacier palace as well as the views of the Aletschglacier from the Sphinx platform I made my way over the paved passage to the M\"onchsjochh\"utte. The walk is pretty relaxed even considering the high altitude, besides the very last 10-20 metres which were a tad slippery. I had a nice plate plate of various sorts of cheese as lunch snack, before making my way back to the Jungfraujoch station, and then over Lauterbrunnen and the Staubachfall back all the way to Geneva.\\

Lauterbrunnen: Jungfraujoch*****, M\"onchsjochh\"utte*****, Staubachfall***

\section{April 22: Ravennaschlucht \& Freiburg}
\label{Freiburg2011}

Breitnau: Ravennaschlucht****\\
Freiburg M\"unster****

\section{April 25: Feldberg \& Feldsee}
\label{Feldsee}

Being back home in Germany over the long Easter weekend offers the opportunity to explore the close-by hills of the Black Forest. The largest mountain is the Feldberg, once home to the Feldberg glacier. Down the rock-face is the cute little lake of Feldsee, also start of the Gutach/Wutach which a few km down reaches my home village. Little pine forests around the lake, little waterfalls here and there, the outflow of the river at one of the moraines of the former Feldberg glacier, switchbacks to get up the mountains, all in all a cute afternoon hike at the start of the hiking season.\\

Feldberg: Feldsee****

\section{May 15: Martigny \& Vernayaz}
\label{Martigny2011}

This time my brother and I decided to spend some more time in Valais. Stopping in Martigny we opted for the Roman amphitheatre, clearly a smaller even venue compared to the big arenas everybody knows off. And on to our second stop the village of Vernayaz with the heavily advertised Gorge du Trient. The gorge is a nice slot canyon, also water levels were great. It was just a bit short, about 1 km and that was it. I don't know if it is planned to extend the pathway, the gorge itself is quite a bit longer at least. A bit disappointed we continued to walk along the village's main road until we reached the waterfall of Pissevache. Clearly visible from the train racks all year round, particularly impressive in winter when the water levels are rather large.\\

Martign:y Amphitheatre**\\
Vernayaz: Gorge du Trient***, Pissevache****

\section{May 21: Grottes des Vallorbe}
\label{Vallorbe2011}

My parents are visiting me and my brother. Thus we decided to plan a bit to make their stay worthwhile. We decided to meet halfway at Vallorbe. This border town is not only home to one of the several mountain forts, but it is also home to one of the largest caves in Switzerland. The Jura mountain has several big caves, though mainly on the French side. Several halls can be visited, an underground stream leads through some of the chambers, before it appears as the Orbe overground. The halls are full of stalactites and stalagmites, in a couple of rooms at the end of the way several crystals and gemstones can be admired. Back then photography was forbidden, but that has changed in the meanwhile (status 2019). Crossing the Orbe and walking up the hills there is another smaller cave -- the Grotte de Fees. One can go in at ones own risk with large flashlights, helmets are not a must, since the cave has a decent height, I thought it is still nice to do.\\

Vallorbe: Grottes des Vallorbe*****, Grotte de Fees***

\section{May 22: Lyon}
\label{Lyon2011}

After a night in Geneva we spent the second day of my parents' visit in Lyon. As one of the three largest cities in France (way behind Paris and roughly on the same scale like Marseille I believe), Lyon has a nice old town sitting between the Rhone and Saone rivers. Its roots range back into Roman times, the very well conserved Odeon and Theatre are witnesses of that time. Not much is left of the amphitheatre though. The neo-byzantine basilica of Notre Dame de Fourviere towers on a hill side over old town. You can either walk up, or take a small cable car up the hill. The view of the city from the terrace in front of the church is really nice, on good days even the Mont Blanc can be seen (it was not a good day, but pretty hazy). The church itself is full of mosaics, definitely worth a visit. While the gothic cathedral of St Jean is nice, it pales a bit after a visit of the Fourviere basilica. The astronomic clock is the highlight of the cathedral, also stained glass can be found. Then we had lunch, Lyon is famous for its good food and the mussels we had were very nice indeed. Then we strolled over nice town squares, looking at facades and courtyards of Renaissance and Baroque buildings, like the city hall, and we had a short look into two more churches, although these didn't offer something special.\\

Lyon: Notre Dame de Fourviere*****, Odeon \& Theatre****, Cathedrale St Jean****, St Nizier***, St Georges**

\section{May 29: Grindelwald}
\label{Grindelwald2011}

After our two trips to the Zermatt the year before, my brother and I decided that we should see a bit of Grindelwald as well. We started the day taking the gondola up to First. My brother had gotten himself a Nikkon SLR and taught me how to properly use it, since I did already then consider getting myself a new camera after getting my PhD (it took me still a couple of months to get my own SLR, opting for a Canon EOS600D instead). The gondola ride offers superb views of the Upper Grindelwaldglaciers including both icefalls. Nowdays the second ice-fall has collapsed completely, leaving about 2 km of dead ice detached from the glacier. Once we were up on First we walked the nice and easy path to Bachalpsee. Since it was very windy, we had the nice panorama of the Bernese alps in front of us, albeit without their reflections in the water of the lake. 

Grindelwald: First \& Bachalpsee*****, Pfingstegg*****, Kleine Scheidegg*****, Staubachfall****

\section{June 3: Basel Zoo \& Bad S\"ackingen}
\label{Basel2011}

Basel: Zoo Basel*****\\
Bad S\"ackingen: Wooden Bridge***, Fridolinsm\"unster ****

\section{June 4: Mainau}
\label{Mainau2011}

Mainau: Mainau*****

\section{June 19: Zermatt}
\label{Zermatt2011}

Zermatt: Gornergrat*****, Gandeggh\"utte*****

\section{June 26: Trift- \& Steingletscher}
\label{Gadmen}

Gadmen: Trifgletscher \& Triftbr\"ucke*****\\
Sustenpass: Steingletscher*****

\section{July 3: Aletschgletscher}
\label{Aletsch2011}

Fiesch: Eggishorn*****, Fieschergletscher*****, Aletschgletscher*****

\section{July 31: Scharzbergkopf}
\label{Saasalmagell2011}

Saal-Almagell: Scharzbergkopf*****

\section{August 28: Ludwigsburg}
\label{Ludwigsburg2011}

Ludwigsburg: Ludwigsburg Palace*****, Favorite Palace****

\section{September 3--September 5: F\"ugen}
\label{Tirol2011}

September 3: Hintertux:\\
Hintertux: Hintertuxer Glacier*****, Hintertuxer Falls****\\

September 4: Zell am Ziller:\\
Zell am Ziller: St Maria Rast***, Gold Mine***

September 5: Innbruck:\\
Innsbruck: Dom*****, Hofkirche*****, Hofburg****

\section{September 29-October 1: Franken}
\label{Franken2011}

September 29: W\"urzburg:\\
W\"urzburg: Residenz*****, Neum\"unster****, Festung Marienberg***, K\"apelle*****\\

September 30:\\
Bamberg: Kaiserdom*****, Neue Residenz*****, St Michael****\\
Bad Staffelstein: Vierzehnheiligen*****\\
Banz: Monastery****\\
Coburg: Veste***\\

October 1:\\
Pommersfelden: Schloss Weissenstein*****\\
W\"urzburg: Residenz*****

\section{October 16: Chamonix}
\label{Chamonix2011}

Chamonix: Mer de Glace*****, Glacier de Nantillons****, Glacier des Bossons*****

\section{December 27--December 30: Vienna}
\label{Vienna2011}

Although I had visited many places previously, I never made it to the city which had been the capital and thus the seat of many art movements of the German Empire for several centuries (just think of classical music). In retro-prospect I had no idea that I would call this place my home some years down the line. Clearly many places I visited again in 2014 or 2020, but for some 2011 still remain the only time I saw them, but more about that later on.\\

Since my parents had visited Vienna previously, I relied also partially on their suggestions with a bit of input from me and my brother, and once again we didn't stay over up to New Year's Eve.\\

December 27: Vienna: \\
Our first touristic point was Sch\"onbrunn, the former Imperial Summer Palace. We expected it to rival Versailles in any respect, and with such expectations you can only be disappointed. The palace is nice, but feels far less grandiose. In fact it felt (and still does) that folks try to rush you through quickly, many tour groups clutter in places and fill up rooms, only the rooms exclusive to the Grand Tour feel fine, but those include indeed highlights such as the Vieux-Laque-Room. The Grand Gallery also gives you an Imperial Royal flair. Also a minus point that no photos were allowed. The park is nice, but in winter the fountains are switched of and obviously flowers are not blooming either.\\
Stephansdom is a nice church, but you only see half of it without paying, and thus we only saw a the altars or tombs from a bit far. Only my mum managed to sneak by somehow and get inside the area where you would in theory need a ticket for without paying. I don't know how she managed but there she was roaming around and enjoying the special views.\\

Vienna: Schloss Sch\"onbrunn*****, Stephansdom****\\

December 28: Vienna: \\
Vienna: Upper Belvedere*****, Lower Belvedere*****, Karlskirche***, Albertina****, Hofdepot***\\

December 29: Vienna:\\
Vienna: Hofburg****, National Library*****, Augustinerkirche***, Peterskirche****, Jesuitenkirche****, Dominikanerkirche****, Greek-Orthodox Cathedral***, Maria am Gestade***, Schottenkirche***, Minoritenkirche**, Parliament****, Burgtheater****

December 30: Vienna:\\
Vienna: Hofburg: Neue Burg***, Hofburgkapelle**, Hofburg: Treasury*****, Hofburg: Silverchamber**, City Hall****, Votivkirche***, University**, Kunsthistorisches Museum*****
\chapter{Year 2012}
\label{2012}

\section{February 11--February 26: Edit 2012}
\label{2012:EDIT}

A new year a new challenge: my boss decided that it would be nice for me to get a hands on overview on a multitude of detector technology, something provided by the EDIT school which is a two week long school to get young physicists interested and educated in current and past technologies to detect particles. This 2012 edition it would be hosted at Fermilab. Realising that my passport was close to expiration I exchanged it just shortly before Christmas time, and it did arrive just in time for this visit (reminder to my 2021 self, not to forget to extend my current passport as quickly as possible). We all stayed at one resort with a large bar, where we enjoyed long nice discussions (after eating at one of the many large restaurants in the vicinity). The resort had also two indoor pools which we made heavy use off too.\\

February 12:\\
During the school we had the opportunity to see some of detectors at Fermilab, among those the large neutrino experiment of Minos, and the two multi-purpose detectors of the Tevatron, which just had been decommissioned a couple of months before: CDF and D0. While D0 was stilled closed and taking cosmic data, CDF had already been opened which gave us unique views of its interior.\\

Batavia: Minos****, D0*****, CDF*****\\

February 13:\\
I put that data as random, in fact on all days of the EDIT school we had been at Wilson hall. The hall was inspired by the large French gothic cathedrals, it particularly has a setting where the building gets smaller the higher it gets. Thus it creates the feeling of the high naves of French gothic churches.\\

Wilson Hall****\\

February 18: Chicago:\\
When in Chicago you go to downtown and the loop. This time I started out the day by Field museum with its giant Tyrannosaurus Rex Skeleton and the remains of old Mayan towns. Then Jingtuan, Ina, and I walked to John Hancock Center enjoying the Chicago skyline from the observatory before going over to Chicago's cathedral, going alongside the Chicago river and its art deco skyscrapers. Once we arrived by Willis tower we were told it might take up to 2 hours to get up, but we only had about 2 h time to get back to the Limousine all of us ordered to get us back to the resort. The others didn't want to risk it but I decided to go for it to get some nice night views out. Once I got in to watch the movie in the waiting hall i placed myself by the exit, thus I skipped already a bit. Then we were offered the opportunity to get up earlier at the expense to walk up the last 10 floors. I obviously volunteered and one lift switch later mid tower I was up there walking the last 10 floors up. Instead of 2 hours it took me about 40 minutes to get up after all. Thus enough time to get night photos out of the sea of houses up to the horizon (Greater Chicago is gigantic, maybe not as large as Greater LA, but still). Anyways I got down early enough joined the others who had a large dinner, had just a snack instead but I clearly made it on the limousine too AND got my photos - so win-win after all.\\

Field Museum****, Willis Tower*****, Cathedral***

\section{A new camera: an DSLR Canon EOS 600D}
\label{canon600D}

After all years with my faithful companion Olympus Superzoom I realised its limitations particularly for interior photography and in the white balance settings, which just had become outdated at that point. Whenever inside I needed a stable surface, and tripods were not allowed in many places either. Thus it was time to retire my first digital camera and go for a larger DSLR. Since my brother had already a Nikkon I got the complimentary Canon and the model I thought would be right for myself, and EOS 600D.

\section{May 11--May 13: Prague}
\label{2012Prag}

May 11: Karlstejn\\
Burg Karlstejn****, Fanziskus von Assisi***, Nikolauskirche***, Alst\"adter Rathaus****\\

May 12: Prag:\\
Prague Castle: State Apartments*****, Prague Castle: Old Royal Palace*****, Vitus Cathedral*****, Loreto Monastery****, Strahov Monastery*****

May 13: Prague\\
St Jacob****, Teynchurch***, St Nikolaus (Kleinseite)*****, Palais Waldstein****, Bethlehem Chapel*

\section{May 27: Bernese Lakes}
\label{Brienz2012}

Lake Brienz****, Lake Thun****, Giessbachfalls*****

\section{May 31--June 3: Loire Valley}
\label{2012Loire}

May 31:\\
Chenonceaux: Chateau Chenonceau*****\\

June 1: Blois:\\
Blois: St Vincent \& Paul***, Chateau Blois*****, Cathedrale***, Notre Dame de la Trinite****, St Nicolas***\\
Amboise: Chateau Amboise****\\
Chaumont-sur-Loire: Chateau Chaumont-sur-Loire****\\

June 2:\\
xyz: Chateau Cheverny*******\\
Chambord: Chateau Chambord*****\\\
Orleans: Cathedrale****, Hotel Groslot***\\

June 3:\\
Vezelay: Basilica Maria Magdalene*****
Cluny: Abbey****

\section{June 17:Valais}
\label{Valais2012}

Sion: Valere***, Leukerbad: village**, Saas-Fee: Mittelallalin****

\section{July 22: Leuck \& Leukerbad}
\label{Leuk2012}

Leuck: Church**\\
Leukerbad: Gemmiwall****, Daubensee****, Wildstrubelglacier****

\section{July 24-July 31: Berlin}
\label{Berlin2012}

How this trip happened:\\
Finally I had no obligation to block July holidays due to possible exam correction or exam protocol days. It had been years since I visited Berlin, and I always wanted to see also the ``little'' palaces of the Prussian kings and German emperors, particularly since the UNESCO awarded them World Heritage status too. Thus I purchased the yearly ticket of the palaces of Berlin and Brandenburg. This would also cover the special exhibit on Friedrich II der Grosse of Prussia (unfortunately this also meant no photos in the New Palace of Sanssouci which hosted that exhibit). I also found out that it is in fact possible to visit the presidential palace of Bellevue (a bit hidden on the president's webpage), thus I inquired and was given a couple of dates, which then decided after all when I would visit the city.\\

Co-Travellers:\\
I didn't manage to get people from Geneva on board, so I tried to engage my friends from my physics studies at ETH to join me on this trip. Gianrico decided that Berlin is always worth a short trip and he would join me for a couple of days. He also suggested a Motel One hotel for the stay, which was very central and since I booked a couple of months in advance also not too expensive.\\

July 14:\\
I took a late evening flight. In fact it was supposed to leave at a different time and to land in Berlin-Brandenburg. But then we all know the story by now, just a couple of weeks before inauguration major problems with the fire extinguisher system were announced and the whole airport terminal had to be revised, and only almost a decade later the airport finally was opened. So old GDR style airport Sch\"onefeld it was for me instead. After arriving I just arrived in the hotel and then did a small walk over to KDW and Kaiser-Wilhelms-Ged\"achtniskirche which was hidden under tons of scaffolding.\\

July 25:\\
Originally I feared the mornings would be cold and then it would get too warm in the afternoons, but then already in the morning stepping out in long jeans I felt it was a bit too warm, so changed to shorts right away (right choice!). Anyways off to the very first palace of Charlottenburg. This is the largest palace in Berlin which wasn't completely destroyed in the war (or blown up like Berlin City Palace). Since this palace was used by multiple monarchs, it contains rooms from Baroque, Rokoko, and Classicist style. It used to be home of the Amber room too before it was given to the Russian Emperor as a present. I was particularly impressed by the Porcelain room and the palace chapel. Then there is the new wing which has a flight of rooms from the time of Friedrich II with lots of silver decorations. The park is of French Baroque style with a revival style mausoleum, the Belvedere which exhibits Porcelain nowadays, and the Neue Pavilion, built as leisure palace for the crown-prince of that era by Schinkel.\\
Then I moved over to the eastern part of Berlin, visiting the Schloss Sch\"onhausen. This palace had been the main residence of the wife of Friedrich II. The two didn't really have a harmonic marriage to say the least. Only a couple of rooms remain intact from this era, as the palace had been refurbished as the state guest house of the German Democratic Republic. Although this was done in a communist era, the rooms actual respect the appearance of the overall place, and it fits quite well with the other rooms.\\
And then back to Gendarmenmarkt, one of the most beautiful squares of Berlin with the classic style churches and the classicist Konzerthaus as main focus. The tour of the concert hall was quite informative, it is also a far more classical setting with a large organ in the main hall than the modern style building of the Berlin philharmonics. Thus depending on your taste you have the possibility to witness concerts traditional style or in a more modern environment. \\
I ended the day walking across the Lindenstrasse through the Brandenburg Gate with a short stop by Hedwigskathedrale, the catholic cathedral of Berlin. Largely destroyed in WWII, this baroque building was re-interpreted in a modern manner, but opening up the crypt to the main central rotunda. Now, by the begin of the 2020s a new renovation of the cathedral will change the appearance of the cathedral once again, closing down the crypt.\\

Schloss Charlottenburg*****( Mausoleum***, Belvedere***, New Pavillon****), Schloss Sch\"onhausen****, Franz\"osischer Dom***, Konzerthaus*****, Hedwigskathedrale***, Brandenburger Tor****\\

July 26: Potsdam\\
Schloss Cecilienhof****, Marmorpalais*****, Belvedere Pfingstberg***, Schloss Sanssouci***** (Gem\"aldegalerie*****, Neue Kammern*****, Orangerieschloss****, Charlottenhof***, Neues Palais****)\\

July 27:\\
Schloss Glienicke***, Schloss Pfaueninsel****, Nikolaikirche***, Schloss Babelsberg***, Flatow-Turm***\\

July 28:\\
Schloss Sanssouci***** (see above and Chinese Tea House***, Belvedere***, Norman Tower***), Machine House****\\

July 29:\\
Dom*****, Nikolaikirche***, Graues Kloster***, Marienkirche***, Airport Berlin-Tempelhof***\\

July 30:\\
Schloss Bellevue*****, Bunderpr\"asidialamt***, New Museum*****, Pergamon Museum*****, Rotes Rathaus***, Sophienkirche***, Reichstag*****\\

July 31:
Kaiser-Wilhelms-Gedaechtniskirche****, Old National Gallery***, Bode Museum***, Ephraim-Palais*, New National Gallery****, Olympic Stadium***

\section{August 11: Argentiere}
\label{Argentiere2012}

How this trip started: Nice weekends in August are always great to hike and explore the mountains. Considering one of us had a rental car, choice was to go to Chamonix, close enough to return early enough for the big fireworks of the Fete de Geneve 2012.\\

Co-travellers:\\
Jacob (US-American), UCLA undergrad, Dan and Brant (both US-Americans), North Eastern undergrads, and Jesse (US-American) UC Riverside grad student. All of us worked together on a muon chamber construction facility. Jacob had a car to transfer from his apartment to the Prevessin CERN side.\\

Jacob just had a couple of experience driving stick shift, at some point he decided he had enough of that stop and go traffic jams, and switched with Jesse, who then took the driver's seat. Once we arrived at Argentiere, we made it to the top of Grand Montets, enjoying for once a magnificent cloudless sky. It was even possible to look down on the Mer de Glace, as well as Mont Blanc. Then we went down to the middle station of Lognan. From there it is a pretty easy path over to the icefall of the Glacier d'Argentiere. We had to pass by herds of mountain goats and sheep, though all seemed not to bother too much about hikers or were curious enough to lick hands etc. Once we reached the glacier, Brant and Jacob decided to walk on the ice, just like most of the other tourists did. By that time even at an altitude of about 2000 m it was still around 20 degrees, so thank god we all were in T-shirts or shorts. The hike back was even faster, so we were back in Geneva early enough to still get perfect spots for another great 45 min fireworks over the lake. But only Dan in fact chose to see the fireworks on my spot, whereas the others opted for a spot by Perle de Lac with UN folks. We did meet up later to celebrate further on.\\

Glacier d'Argentiere*****, Glacier des Grands Montets****

\section{August 18: Jungfraujoch \& Grindelwald}
\label{Jsungfrau2012}

How this trip started: Brant and Jacob wanted to do Canyoning in Switzerland. They found a pretty decent offer close to Interlaken and decided to spend the weekend there. We got a newcomer in Muon Chamber land -- Indara -- who also expressed interest to see some mountains. I thought it could be nice to get to Jungfrau again. Indara and I decided to go there, and later meeting up with Jacob and Brant for dinner and a train ride back.\\

Co-travellers:\\
Indara (Mexican), physics grad student, also part of the Muon chambers crew. Having not seen most of the mountains, glaciers, or snow she is very excited to see what mountains have to offer.\\

 We got ourselves day tickets for the demi-tarif on swiss trains and decided to meet at the train station. I arrived there 15 minutes early, and waited, and waited, and waited. No Indara to be seen anywhere. I then realised that we forgot to switch phone numbers. So I thought well i go ahead and let Eric (my grad student) know what i went ahead. After all he was smarter to exchange phone numbers with any newcomer. Indara had the same idea and 20 minutes later we were in contact: the following had happened -- Indara wanted to park her car at CERN, but then the guard gave her car a very thorough check, when entering CERN. Thus she missed the tram, and arrived about 25 minutes later than planned, so she jumped on the next train.\\
 I decided that we would could meet up by the middle station of Kleine Scheidegg, which has great views, where I could also grab stuff to eat before moving on. Switching to the cog rail at Lauterbrunnen, Indara calls again which train to take at Interlaken. I tell her to either take the one to Grindelwald or Lauterbrunnen, in fact it is the same train -- the front part goes to Lauterbrunnen, the back part to Grindelwald -- I prefer to take the one to Lauterbrunnen. That valley is far more breath-taking than the boring ride to Grindelwald. Thus Indara decided she wants to have that view too, but messed up and ended up in the Grindelwald part. Unfortunately she was not fast enough to switch between the parts, thus half an hour more of waiting time. Anyways I don't mind taking more time eating my R\"osti at Kleine Scheidegg, at this point, we would skip the originally planned hike from Pfingsteg to B\"aregg and just stick to Jungfraujoch. Finally we met up, and onwards to the mountain top. Of the stops along the way the Eismeer station offers the best experience, with an overlook of the Grindelwald-Fieschergletscher and the Eismeer in the distance. Up on Jungfraujoch we enjoyed the glacier cave, and the views of the Aletschgletscher. We walked a bit on the trail, and Indara decided to try sledding for the first time in her life. An area was set up for tourists to book sleds, and so she did. Once she sat on it, she wasn't sure anymore if it was a good idea, so I gave her a gentle push. Now 20 mins later she decided she had enough of it, and we took the cog rail down to Grindelwald. \\
There we decided to take the cable car up to First, to enjoy a better view of the Upper and Lower Grindelwald glaciers. And while we were up there we treated ourselves with a giant cup of ice-cream. Back in 2012 the tongue of the Upper Grindelwaldglacier still reached almost down to the valley, but the middle section had already melted quite a bit. In the late 2010s the lower part separated from the upper part of the glacier completely. The Lower Grindelwaldglacier was already on the verge of collapse int he early 2010s, now having reached 2021 it completely disappeared. So my recommendation wherever there's a glacier close to you make the effort to see it. I know many of us try nowadays to keep our emissions low, e.g. walking instead of taking the car for short distances. But unfortunately even if we would stop our carbon emissions right now it will be too late for glaciers to recover in our lifetime again most probably. Thus we have to enjoy them now before it is completely too late. And that's what we did enjoy ice cream and enjoy the view. Then we made our way to the youth hostel of Interlaken, where we met up with Brant and Jacob. There we also met their friends Kate and Stan, who worked at UN organisations in Geneva. in fact we might have overdone ourselves with that gigantic cup, once we were offered fries we only had a handful of those. After having that as snack for dinner, we enjoyed a parade of cows and the sunset views of Jungfrau, before taking the last train out of Interlaken, arriving after midnight in Geneva.\\

Jungfraujoch*****, First*****

\section{August 26: Meersburg}
\label{Meersburg2018}

I was home for a weekend once again, shortly before my first -- and only -- CMS week outside of CERN. Only recently the renovated New Castle of Meersburg was open for tourists, so I convinced my family to join me on that adventure. The large staircase is indeed impressive, the festival hall is nice to see as well, the private apartment of the bishop was smaller than I expected. Back then no photography was allowed and overall the explanations were not up to point shortly after opening. Thus a nice place to see but could do better I think. Then my mum and I visited the old castle, one of the oldest non ruined medieval castles in Germany, albeit with lots of modifications done during the times.\\
Since we were done a lot quicker than though in Meersburg we drove to the neighbouring village of Unteruhldingen. Here a lot of archaeological remains of pile dwellings have been found. Nearby a pile dwelling village has been reconstructed as open air museum, with displays of life back then happening from time to time as well. On our way back we stopped by the beautiful baroque Pilgrimage church.\\
Considering my 2020/2021 situation: IF covid19 wouldn't have happened this would have been a perfect affordable day to spend in my region, but even such things are nowadays a luxurious thing of the past.

Meersburg: Neues Schloss Meersburg****, Altes Schloss Meersburg****\\
Unteruhldingen: Pile Dwellings****\\
Birnau: Pilgrimage church***

\section{August 30--September 9: Portugal}
\label{Portugal2012}

August 30: Lisbon\\
Santo Antonio da Se***, Se Cathedral****, Church of de Craca****, Sao Vincente de Fora****,  Panteon Santa Engracia***, San Martires***, Church da Encarnacao***, San Loreto**, Sao Roque*****\\

August 31:\\
Palacio de Mafra*****, Palacio de Queluz*****, Sintra: National Palace****, Palacio de Monserrate****, Quinta de Regaleira****\\

September 1:\\
Sintra: Palacio Pena****, Castello di Mauro***, Lisbon: Sao Roque*****, Santa Catarina***, Sao Domingos****\\

September 2: Belem: Sunday\\
Co-Traveller for the day: David, US-American, a professor from UCLA and my boss. David decided to come to this year's CMS week as well. In order to get not into the worst jet-lag he arrived a day early and decided I should take him on a tourist program such that he wouldn't get too tempted to sleep in or get to sleep far too early. I clearly took on the challenge to have an interesting day set up for another person, it might have been a tad too much.\\

We started the day in the Palacio de Ajuda. After the palace in Lisbon had been destroyed in the large earth quake, finally a spot in Ajuda in the outskirts had been chosen as site for the new grand palace. But then in 1807 the Royal family had to flee to Brazil. That's also the reason why in Brazil you in fact have real royal palaces where indeed royalty lived and was not only represented by a governer. Only in 1821 the king returned to Portugal and only then the palace became a royal residence. When in 1910 the monarchy was abolished only a small part of the palace had been finished. Indeed until now you can see the unfinished walls of one of the wings. The state apartments which were already finished are decorated mainly in classical style, but also winter gardens with fountains or a Chinese style room can be found there. All in all definitely a palace I recommend you to see. \\
Then we visited the gothic Monasteiro Jeronimo in Belem. Besides the church with its small filigree pillars and elaborate vaults, the monastery is particularly famous for its highly decorated cloister considered to be the highlight of Portuguese gothic architecture together with the monastery of Batalha. Although California is famous for its sun, the climate is far more temperate than the Mediterranean. David already complained the heat was getting unbearable. We still went up on the monument of explorers. Besides the good view of Belems and its garden, and the famous tower of Belem, you also have literally the world at your feet, since a giant world map covers the square in front of the monument. Still completely exhausted and overheated David decided it was long time that we get dinner and particularly tons of water. \\
After having refreshed ourselves we went to the last stop of our tourist day, the Palacio de Belem, once a royal palace which is nowadays the presidential palace. The tour was given in Portuguese, the only word we understood was, you are allowed to take photos. Clearly I was happy about that and started to take photos. But then unlike we were told it was supposed to mean you CANNOT, obviously that one word changes the meaning to the opposite. Still interesting to see in what wonderful richly ornate rooms the President of Portugal can hold his meeting, quite the opposite compared to the cold white of the German presidential palace or the rather modest outfit of the White House. Once in the gardens we could indeed take photos. The garden is a very geometric French style gardens with little Pavilions for more private meetings. At the end of the tour David called it a day and went back to Lisbon, and even I agreed that it was a little bit too hot for my taste as well.\\ 

Lisbon: Palacio de Ajuda*****, Monasteiro Jeronimo*****, Monument of the Explorers****, Palacio de Belem****\\

September 5:\\
This day of the workshop ended with a reception by the technology museum in Lisbon. I was too enthusiastic using one of the pumps to produce big bubbles in a water column that I hurt my wrist that much, that the strain made it hurt for two months later when bent backwards. So be enthusiastic when in museums, but keep your excitement under control. Don't be me!\\

Lisbon: Technology Museum**\\

September 6:\\
Once again a day full of meetings, and the afternoon focused on the heavy ion runs, which wasn't my field of study at all. Thus I decided to take the afternoon off, and by suggestion of a colleague I took the regional train to Tomar. In Tomar you have one of the largest old monasteries in Portugal. The convent was built throughout many centuries. Some parts, such as the old gothic chapter house, are nowadays in ruins, but most of the cloisters survived all troubles. The most beautiful of those multiple courtyards is the large Renaissance one constructed during the times, when Felipe II of Spain was also king of Portugal. The church itself is much older, the Rotunda houses a giant masterpiece of wood and stone carving with multiple coloured statues with impressive details. \\
And back to Lisbon for the workshop dinner in the Palacio do Conde d'Obidos. Not only the food was nice, but also the view of the harbour and the Christ Redeemer Statue on the other side of the river. I did enjoy the baroque library of the palace as well.\\

Tomar: Convent*****\\
Lisbon: Palacio do Conde d'Obidos***\\

September 7:\\
Cascais: Boca do Inferno*****\\

September 8: Porto \& Coimbra\\
Porto: Se Cathedral****, Church dos Grilos***, Palacio do Bolsa*****, Sao Francisco*****, Church dos Clerigos****, Church dos Congregados***, Coimbra: Monastery Santa Cruz****, University****, New Cathedral***, Old Cathedral****\\

September 9: Lisbon\\
Estadio da Luz****, Basilica de Estrela****, Oceanarium****

\section{September 16: Grimsel}
\label{Grimsel2012}

Sidelhorn*****, Unteraargletscher*****, Oberaargletscher*****, Rhonegletscher****

\section{September 28--October 1: Poland}
\label{Poland2012}

I wanted to visit Poland and particularly Warsaw already for quite a couple of year. My family wasn't so keen on a trip to Poland around New Year's (most probably rightfully so). So I finally decided that I should try it late September. My boss agreed that it would be fine to leave my hardware test stand in the hands of my grad student Eric who has shown some good work already. Once I told Eric I would go to Poland and I thought he would be now ready to take over operating the test stand for two days on his own, he asked me if he could rather join. I was obviously fine with it, and so was our boss and that's how I started taking friends and colleagues on my trips.\\

Co-traveller:\\
Eric: Japanese-American, physics grad student at UCLA. Eric had already seen a bit of Europe before during an exchange semester in Budapest. He is clearly up to see more of Europe, so off we go.\\

September 28: Warsaw:\\
We flew to Warsaw via Vienna with Austrian airlines. I checked in a suitcase, Eric preferred to have everything on him. In Vienna Eric got himself a book and once we arrived in Warsaw my suitcase had been lost, in fact it was in Catania. I was promised it would arrive a day later, but then it only made it to Rome and then all traces had been lost from thereon. Besides having to get myself a couple of new underwear my battery charger for my camera had been lost too, so unfortunately I had to make sure that it would last until the end of our trip. And so it did indeed. Thus after about 45 minutes of an delay we finally got out of the airport. We did go to the Lazienkowski Park, but due to our time constraints we didn't see all of the pavilions and little palaces which we wanted to see. We started with the Palace on the Water, a classicist palace, which was rebuilt after it was burnt down by Germans in 1944. The Solomon room, as well as the ballroom, and the bathing room are still impressive, but most of the frescoes as well as the gold leave decorations are lost. The Myslewicki Palace is an nice classical palace with several nicely decorated rooms, it had been used as Poland's guest house during communist times. Our hotel was close-by the Palace of Culture and Science, a ``present'' by Stalin to the Polish people's republic. Unfortunately the roof terrace had closed down, but the Foyer was still nice to see, as well as the colourful lighting of the facades. But then we had drinks by the hotel while waiting to hear back where my bag was, but no success so more alcohol it was.\\

Lazienkowski Park: Palace on the Water*****, Myslewicki Palace****, Palace of Culture and Science****\\

September 29: Krakow\\
We wanted to have all the time we need in Krakow so we booked an early 6 am train. But our train was sad to be 30 minutes late, nope 60 minutes, nope 90 minutes. Finally it arrived and we were told that they had to swap the trains due to an engine failure. Anyways now having been late on arrival by about 2 hours we rushed over to the castle. Clearly the line to the castle tours was long, and we wanted to see both the standard and the private apartment tours. We still managed to get tickets for both. I didn't enjoy the fact that on the whole premise, on both tours and in the cathedral photography is forbidden (didn't change until 2020). Some remarks on tripadvisor state such that people won't see how disappointing the palace is. While the palace is not up there with Versailles, it does have some cute and nice moments, mainly tapestries, coffered ceilings. The cathedral is outstanding with many royal tombs in many different styles surrounding the gothic main nave. The cave below the castle once was occupied by the local dragon. Another highlight of Brick Gothic is the breathtaking Mary's Church. The most beautiful basilica in Poland I have seen up to 2021. Richly decorated with a masterpiece of gothic wood carving as high altar piece. The choir stalls are outstanding, and the blue ceiling with its golden star gives it a nice flair too. Then we had another good large dinner in one of the side roads by the market square.\\

Wawel Castle*****, Cathedral*****, Dragon's Den***, Bernhardine Church****, Mary's Church*****, St Francesco***\\

September 30: Warsaw\\
Early in the morning we took the bus to the Wilanow Palace, the Baroque summer residence of the Polish kings, and one of the few palaces not to be destroyed by the German army. The royal apartments are richly decorated and nice to see, particularly the state bed rooms, as well as many paintings in several galleries. Some paintings of the White Hall have been lost though. The French gardens are nice as well. On our way back we realised that the bus didn't follow the typical paths at all, due to the Marathon taking place on this day. Thus we got off somewhere, but not close to old town where we wanted to get to actually. We had lunch there, and then got on the metro, which was not affected by the ongoing Marathon. The royal castle was completely burnt down by Germans in 1944, reconstructed in 1971-1984 according to photographs done shortly before outbreak of World War II. They did a really nice job with that. Then we walked through old town stopping here and there for many churches, all with detailed reconstruction work going into it. The polish army cathedral had though many modern style additions, but all of them done in a fitting tone. Then we walked over the large river and visited the revival style cathedral on the other side of the river. And then more time for a big dinner and tasting of several Polish vodkas in a bar afterwards.\\ 

Wilanow Palace*****, Royal Palace*****, Jesuit Church***, St John's Cathedral****, Mary's Church**, Holy-Ghost-Church***, Cathedral of the polish Army****, St Anna****, Carmelite Church***, Cathedral St Michael \& St Florian***\\

October 1:\\
Poland was the country most hit by the Wehrmacht in WWII. One very good museum dedicated to the Warsaw Uprising illustrated the efforts normal local citizens undertook to fight the Nazis, albeit with devastating failed results. Since our trip to Lazienkowski Park had been cut short by my lost bag on the first day (which had been found in Catania, and got lost AGAIN on its way to Warsaw in Roma Fiumicino, got reimbursed for it later on), we made a second visit to the park, watching the White House and the Old Orangerie with the cute Baroque Court theatre this time around. And on our flight back we switched at Brussels airport, enjoying a Belgium beer in Belgium while we had the chance. In the end it took me until 2016 before I visited Brussels itself, but then my visits to Brussels stand at 4 all up to 2021.\\

Museum of the Warsaw Uprising****, Lazienkowski Park: White House***, Old Orangerie****

\chapter{Year 2013}
\label{2013}

\section{January 2--January 6: Netherlands}
\label{2013:Netherlands}

January 2: Amsterdam\\
Back in 2013 it was still possible to get as foreigner the yearly Museumskaart of the Netherlands. Although we would only be in Amsterdam for a couple of days our planned sights would already be sufficient to make the yearly card worth while. By now tourists are not allowed to buy this card anymore, but pay EXACTLY the same price for a monthly ticket which gives you entrance to museums than a Dutch person would for a yearly ticket. In that sense just like Poland, Russia, India and Hungary the Netherlands decided to introduce different rules for their own population and people from abroad. Just think of if all countries decide to do that, at least within the EU it would rightfully lead to an outcry. I am obviously aware of the vastly different income of a tourist and local residents in India or Russia where it is all justified to keep the tickets for locals at a level which can be affordable. But this doesn't clearly apply to the Netherlands at all.\\

Anyways we started out our trip in the private residence of Rembrandt.\\

Amsterdam: Rembrandthouse****, Royal Palace*****, Nieuwe Kerk***, Oude Kerk****, He Hua Temple**\\

January 3: Appeldoorn \& Amsterdam\\
And today I started my first trip outside of Amsterdam, taking the train to Appeldoorn. After having a first snack there I made may way to the former royal palace of Het Loo, which has been in use by the royal family from the 18th century to the 1970s. Since 1984 the palace is a state museum showing the interiours with all the original furniture from the Orange-Nassau era. Since it was shortly after Christmas each room was still decorated with Christmas trees, ornaments and candles and thus creating a nice mood. Most of the rooms had been decorated in Baroque style, but some of the later rooms were clearly more modern. In the french style garden the main fountain was being renovated and due to low temperatures non of the other fountains or canals were running, most statues were not hidden in wooden boxes as they often do in Germany though. All in all I can only recommend to visit the palace. \\
Taking the train back I walked along the Grachten to the van Loon House, belonging to one of the noble families of Amsterdam with interiors of the 18th and 19th century. The last place I visited on this day had been the Willet Holthuysen House. Maybe it had been the time, but I was least impressed by that house on this day, y that time it was also pretty dark already, which might have played a role too.\\

Appeldoorn: Het Loo Palace*****\\
Amsterdam: van Loon House****, Willet Holthuysen House***\\

January 4: Den Haag\\
This was the day which we had planned as family day trip from Amsterdam. The first place we visited was the parliament with its medieval gothic Knight's Hall. The only available tour was in Dutch. While Dutch is sufficiently close to German you unfortunately got only 50 \% of what was said, but still sufficiently more compared to tours in Spanish, Italian, French, Czech, or Russian which I would follow in the years to come. While the wooden roof of this large gothic hall is not as elaborate as e.g. Westminster Hall, it is still an amazing medieval style place to be. The actual parliament tour afterwards was alright, though the parliament didn't leave a particular wow effect.\\
Then we visited the former Royal Lange Voorhout Palace. In fact only a couple of rooms are still in their original Baroque appearance, most of the rooms have been adapted to house a large exhibition of Escher's art. Clearly the art of this virtuous genius of perspective paintings and impossible geometric layers alone is worth a visit alone. Then we checked out the inside of the seat of the international court of justice in the neo-gothic Fredenspalace (with a ban on photography). Our day ended with a visit of the local arts museums, the Gemeentemuseum and GEM, which were large though not too impressive, particularly considering that Amsterdam spoiled us with world wide famous art just a couple of days later.\\

The Hague: Binnenhof with Knight's Hall****, Lange Voorhout Palace (Escher Museum)*****, Fredenspalace***, Gemeentemuseum**, GEM***\\

January 5: Soestdijk \& Amsterdam\\
One of the last royal palaces to be opened to the public was the classicist Soestdijk Palace. Sitting in a nice English style garden besides the state halls, another Escher exhibition was displayed in the palace. They didn't allow photography, but I did get a free hot chocolate in the ball room.\\
Once back in Amsterdam I realised a tourist convention took place in the Beurs van Berlage. That was my chance to actually see the inside of the building, which is considered a front runner of industrial convention design with a touch of modernist architecture. Then i stopped in the house of Geelvinck Hinlopen, another example of traditional style Dutch houses by the Grachten. Most of the Rijksmuseum had been closed for renovation, but the main paintings such as Rembrandt's night watch were still on display. I was though more intrigued by the art shown in the Stedelijkmuseum, the local city gallery.\\

Soestdijk Palace****\\
Amsterdam: Beurs van Berlage***, Geelvinck Hinlopen House***, Rijksmuseum***, Stedelijkmuseum*****\\

January 6: Amsterdam\\
On this day we saw the jewish heritage in Amsterdam with the old synagogues and the attached treasures. The NEMO was basically a kids museum. Nothing wrong with getting kids interested in science, but in most other technical museums I had been it was both set up to be enjoyable by kids and the parents as well, that was unfortunately a big let down.\\

Amsterdam: Portuguese Synagogue**, Jewish Museum (Large Synagogue)***, NEMO*

\section{March 21--March 25: Paris I -- sponsored by Eric}
\label{2013:ParisI}

Why Paris: why not, well after all Paris is one of my most favourite places to see (having been there over and over again). Paris is also the closest big city from Geneva and a good place to show to your friends. This time I decided to see the private apartments at Versailles as well as at Fontainebleau, so should be big fun. 

Co-travellers:\\
Andrew: A UCLA undergrad student, who came over to help us testing muon chambers. His first time in Europe, I decided the first thing to see should be Paris, particularly since it is so easily accessible by train.\\

March 21: The train ride\\
As the afternoon passed Eric was still busy down in the cavern and I commented that it might get tight if he does not return immediately. For reasons unknown to me he still continued working and indeed missed the train ride (and wouldn't even join later on, so in a way he sponsored our trip). Once we arrived by the hotel we realised it was in renovation (actually more the hallway and the elevator). No issue to carry our not so big luggage up the flight of stairs though. The door to our suite was a bit small, though not too small for me to go through straight up, and Andrew didn't mind to claim the bed for two for himself alone.\\

March 22: Versailles\\
Having been in Versailles already twice before, I decided that this time around it was the time to try something more: the private apartments of the king and the queen. Most of the state apartments are amazing and full of wonderful history, art, and splendour, but behind the scenes is where the royal family really lived with private cabinets, a gaming saloon, their library or even their actual bedroom. The state bedroom was the place where they officially started their days sitting in bed, but it was sad to be very uncomfortable and windy in that room. Also the private inner cabinet of the king is one of the big masterpieces of French Rococo, similarly the Room of the Golden Plates. Only with a tour you can get in there, another fantastic highlight of the tour is the Royal Opera house. Although used as large private theatre it is similar in size to the Parisian Opera house of that time. Afterwards Andrew and I admired the state apartments. Even in March they were very crowded, but nothing to the sheer mass of people in summer. \\
Then we spent more hours in the park and the garden palaces of Grand \& Petit Trianon as well as the Queen's hamlet. In mild March the sun doesn't burn you, the downside of all was that we went on a Friday, which means less crowds but also fountains which are not running. Clearly the sculptures are still nice and pretty, but the running fountains give a totally different flair to things.\\
On our way to the train station we did a small detour through the Hotel de Ville of Versailles with another beautiful gold leaf decorated festival hall and a city council hall with baroque supra portas. Nothing to amazing but still cute to visit once .\\
Back In Paris we did an afternoon and evening tour of the Louvre, making use of their extended opening time on Fridays. We started out in the Empire style Apartments Napoleon III, but also didn't miss out on the artefacts from old Persia, Assyria, or Michelangelo's sculptures, the Venus de Milo or the Mona Lisa. Even in evenings this painting has a waiting line. The Louvre is one of the most amazing museums, also very large. Even a tour of highlights takes at least two hours, and that is if you walk fast. The splendour of the former use as royal palace can be admired in the Gallery of Apollon, although the main royal palace were the Tuleries, a bit further down the Louvre, but those had been completely destroyed in one of the revolutions, although some of the furniture still exists, like Napoleon's bedroom.\\
After that successful first day I wanted to try out something I didn't know, so I got myself a Monaco which was listed under beers. Little did I know that I got myself beer mixed with grenadine juice and 7up. At least Andrew enjoyed watching me drink it.\\

Versailles: Chateau***** (State Apartments*****, Private Apartments*****, Trianons*****), Hotel de Ville***, Paris: Louvre*****\\

March 23: Paris\\
We started the second day walking over the Place de la Concorde, in my opinion the most beautiful square in Paris, before a guided tour through the French House of Representatives in Palais Bourbon. This had been a private palace of members of the extended royal family once, but was completely remodelled to house the parliament later on. Many ceilings and walls have been decorated by famous French artists such as Alechinsky or Delacroix. The interior is largely classicist, the highlights being the Festival Hall and the Library.\\
 Afterwards we walked over to the Ile de la Cite starting with the medieval royal palace of Paris. The Conciergerie is the hall of the guards which is a giant gothic style vaulted hall. Unfortunately nothing remains of the actual Great Hall of the palace with the same dimensions which was situated just above the guard hall. Still it is one of the largest gothic secular halls remaining on the continent. The masterpiece of the palace was though the Royal Chapel, the Sainte-Chapelle which has walls made out of spectacular stained glass, some of the best medieval glass art in the whole world. Some of it had been damaged in the revolutions and had to be reworked, but a big ratio is still from medieval times.\\
 Then we explored a bit of unknown Paris, getting the metro to the outskirts and walked through the beautiful Park de Chateau Bagatelle with nice artificial grottoes and peacocks all over the park. The chateau was according to the Paris online tourist site open on that weekend for a guided tour, but that information seemed to be outdated. Anyway we did enjoy the park for what it was and took the Metro back to Paris where we wanted to see the Hotel de Soubise, which is the seat of the French National Archive.\\
 Although it was over two hours before closing time and even their own posts at the premise did state this, we were refused entry, having arrived too late. Maybe a special event was happening, but we were not given any information about that statement contradicting the information written just behind the counter. Instead we walked over to the nearby Hotel Carnavalet which is now the Parisian House Museum, containing a lot of rooms of noble houses, villas and hotels, which had been transferred to this place before the original houses had been destroyed. Thus you get a pretty good idea how the fashion of private homes changed throughout the years. Last but not least we walked over to the Renaissance square of Place de Voges. In one of those houses Victor Hugo wrote his world famous books, and his house can be visited even for free. So do it if you're in Paris.\\

Paris: Place de la Concorde*****, Palais Bourbon*****, Conciergerie***, Sainte-Chapelle*****, Notre Dame*****, Park de Chateau Bagatelle***, Hotel Carnavalet****, House of Victor Hugo***\\

March 24: Fontainebleau:\\
Having visited Fontainebleau back in 2010 my parents and I missed out on the tour of the Private Apartments by just about 5 minutes. Not wanting to do the same mistake this time I made sure we ordered tickets just when the ticket office opened. On my request for two tours, the one of the Guest Apartments and the Private Apartments in English - the lady on the counter told me, the tours will be in French I cannot sell you the tickets for them. I told her I don't mind I want to see the rooms even if my French is not deemed worthy enough. The lady told me off rudely without good French no tour. Thus I walked over to the counter next to this rude despicable lady now ordering the tickets in French. My French is indeed basic, but it does exist to a point, thus I can in fact follow at least the big theme and topic of French guided tours. This time we did obtain tickets for both tours. \\
With enough time to spare we visited the Chinese Museum in other wings of the palace before. The Chinese museum contains artefacts of the Empress in fact stolen from Beijing's original summer palace before it was completely erased by French and British troops in the Opium wars. Amazing artefacts but sad to see how they were stolen from China in first place. Afterwards we toured the guest apartments in the upper floors of the castle. Usually these contained a cabinet for work, a living room and a bedroom. On display were original desks with newspaper from the time of the second empire.\\
Then we visited the Museum of Napoleon I, set up by his nephew Napoleon III to commemorate the deeds of his uncle. They contain tents from his military campaigns, clothes, the living room of his son Napoleon II, and other artefacts. And then it was time to admire the amazing state apartments, with several rooms built during the Renaissance times of Francois I of Valois with a large wood carved gallery, a ballroom with multiple frescoes, and two rooms with Renaissance tapestries. These are followed by Bourbon time rooms from Louis XIII or a cabinet of Marie Antoinette. The largest rooms, namely the state bed rooms, dining rooms, and throne room had been completely remodelled by Napoleon I, who used Fontainebleau as his major seat, having converted Versailles to a museum. One wing, the so-called Papal apartments didn't open until the afternoon, when we were having our tour of the private apartments. \\
Thus we had time for a three course lunch close to the castle, before having a short look at the local church of St Louis, and back to the palace for the tour of the private apartments with the same guide who gave the guest apartment tour in the morning. Now the private apartments are mainly from Napoleon's times again, but a salon of Madame Pompadour and Louis XVI can be seen at the start of the tour. The tour includes in addition the flat of the private secretary of Napoleon too, thus you get a view of how lower class people lived, even if they worked for the Imperial family. The inner bedroom of the Empress is the most beautiful room of the inner apartments, although I did like the Gallery of Deers which finished the tour. Now we wanted to see the Papal Apartments but then we were told those were closed again and we were too late, no matter if our tickets were valid for those, thus all in all it is IMPOSSIBLE to see all of the palace in winter even if you pay for it.\\
Anyways instead we jumped on the train got back to Paris and walked up to the second floor of the Eiffel Tower. First of all climbing the tower by the stairs gives you a nice view of the inner structure of the pylons and the waiting time is typically substantially less, unless you arrive at night when things slow down considerably. Thus it is usually the fastest way to get up, and a bit of sports doesn't hurt anyway either. You have nice views already from the second floor, the third floor doesn't give you much in addition. Now the downside is once up the Eiffel tower, you don't see the Eiffel tower itself anymore. In order to achieve that go up on Tour Montparnasse instead. Which is exactly what we did, some even claim another positive aspect: from Tour Montparnasse it is impossible to see Tour Montparnasse itself which some people consider really an ugly building - if you ask me it is totally fine, only not really suited the classic Parisian city centre, but then again such is Tour Eiffel as well. \\

Fontainebleau: Chateau***** (Chinese Museum \& Museum Napoleon****, Guest Apartments***, State Apartments*****, Private Apartments*****)\\
Paris: Eiffel Tower*****, Tour Montparnasse*****\\

March 25: St Denis \& Paris:\\
This day we started on the outskirts of St Denis. The basilica of St Denis is the first gothic style building in the world, the stained glass is though largely remodelled from the 19th century. This is though not the point why you should go there. In St Denis a large majority of French kings are buried, thus the tombs range from ancient medieval to Renaissance and Baroque monuments, really unique to see, only Westminster Abbey can compete in this respect. Then we went back to Paris to visit the neo-baroque amazing beautiful Opera Garnier, the most beautiful theatre I have seen so far. If you shouldn't be into Baroque aesthetics maybe a large ceiling painting done by nobody else but Marc Chagall can sweeten the deal. For shopping mall lovers Galleries Lafayette just next to the Opera offers a unique glimpse of the grandiose splendour of malls of the 19th century, and the view of the Opera house from the roof top terrace is nice too. \\
Napoleon I and his family members aren't buried in St Denis but in the Dome d'Invalides attached to the French Army museum, the church itself is of late Baroque classic style. Another masterpiece of classicism is the Pantheon, originally built as a church, which is now the place where major figures of France, such as Victor Hugo are buried nowadays. And our final destination was Hotel Jacquemart-Andre, which displays the saloons and the art collection of this family. Then we rushed back to the hotel did shop Baguettes, cheese and salmon for dinner. Sometime during the day I had also strained my ankle, which didn't stop me from continuing our program, but feet proofed to be pretty swollen at that point. We took the metro which was supposed to bring us to Gare de Lyon the quickest, but there seemed to be a problem on the line, at least the metro stopped for 10 minutes, thus we rushed to another station, jumped on another metro line and made it to the TGV just 5 minutes before departure. But we still made it, and then we could enjoy our French style dinner on the way back to Geneva.\\

St Denis: Basilica*****\\
Paris: Opera Garnier*****, Galleries Lafayette****, Dome d'Invalides****, Pantheon****, Hotel Jacquemart-Andre****

\section{April 3--April 11: Roma}
\label{Roma2013}

April 3: Rome\\
Flying over the Alps in early April is fun, particularly if you start below the clouds, and just a couple of minutes after the start you finally see the Alps with Mont Blanc appear above a sea of white fluffiness. Everything was covered in snow, naturally Mont Blanc was all white, but everything else was sunlit and bright as well. Landing in Rome i had to wait another 90 minutes, since my mum's plane from Zurich was delayed by over an hour. Arriving at our hotel we had to rush walk quite a bit to make it to the Palazzo Farnese. One of the highlights of Italian Renaissance, this palace is nowadays the French embassy. Naturally our tour was held in French. Although I live in a French speaking place, I am not particularly fluent in French, my mum's French might be even superior, she had quite a couple of years of French at school, though a couple of years ago by now. I got though most of what we were told. We saw the hall where the famous Hercules of Farnese had been standing for many years, before it was transferred over to Naples. And then we saw the Carracci Gallery. Unfortunately only photos in the courtyard and the gardens were allowed. Nowadays the embassy even forbids that.\\

Rome: Palazzo Farnese****, St Andrea della Valle****, Pantheon*****, St Maria Sopra Minerva****, Marc-Aurel-Column****, Spanish Stairs****\\

April 4: Rome\\
After an extensive first day breakfast and having a short peak into the basilica of Santa Maria degli Angeli which is in fact a converted hall of the Baths of Diocletian we took the metro to the Vatican City region.
We didn't want to spent the 4 EUR extra fee to pre-book tickets for the Vatican museum, so we did arrive a couple of minutes earlier to be pretty much front of the line once the museum would open. While queueing I thought the face of a young man in front of me looked familiar. But then how likely is that, I just thought I just imagine that, but then i saw more familiar faces. Seems almost my whole group of first year physics students in my assistant class just happened to visit Rome and the Vatican museum on the very same day, what a coincidence.\\
The museum was very busy as usually, particularly the wonderful Sistine's chapel. I did enjoy the Stanze di Raffaelo the most, just some of the most beautiful painted rooms I have witnessed. I sometimes wonder what the other rooms of the Apostolic palace might look like which only official visitors can see (and not us mortals). Then St Peter's basilica didn't fail to impress not only by its giant size but also the several monuments and arts peaces. Unlike nowadays one could also visit most of the lower church and even still take a couple of photos. Then it was time to see something new again, this time the Palazzo Primoli, once a private place for the Bonaparte family housing a museum of Napoleon. It was alright, but nothing to outstanding. \\
Then once close to the old city we had a couple of churches to visit, climbed up the stairs to Santa Maria in Aracoeli, and then it was time for another fantastic museum (also a first for me), the Capitoline Museum. This museum is home of the famous She-Wolf, remains of the temple of Jupiter, offers great views of the Forum Romanum, has a couple of other paintings and statues around, and beautifully decorated halls. On our visit of that year due to a special exhibit (with pretty ugly art), the main halls were off limits for photography, but since then I visited again, and photography is generally allowed. Last but not least we ended our day with pasta and a visit of the Trevi Fountain.\\

Rome: Santa Maria degli Angeli****, Vatican Museums*****, St Peter's Basilica***** (with lower church****), Palazzo Primoli***, Piazza Navona***** (with Sant'Agnese in Agone****), Il Gesu*****, Santa Maria in Aracoeli*****, Capitoline Museums*****, Santi Apostoli****, Oratorio Santissimo Crocifisso***, Trevi Fountain****\\

April 5:\\
Rome: Santa Maria Maggiore*****, San Prassede****, Santa Pudenzia***, Santa Croce****, San Giovanni in Laterano*****, Palazzo Laterano****, Battistero Laterano****, Villa dei Quinitili****, Via Appia Antica****, San Sebastiano fuori le Mura****, Aurelian Walls***, San Paolo fuori le Mura****\\

April 6: Rome\\
Rome: Piazza Navona*****, Santa Maria in Aquiro***, Nostra Signora del Sacro Cuore**, Palazzo Colonna***** (+Private Apartments*****), Palazzo Doria Pamphilj****, Santi Apostoli****, San Ignazio*****, Pantheon*****, St Maria Sopra Minerva****, Sant'Angelo in Pescheria**, Marcellus Theatre***, Santa Maria in Cosmedin***, Santa Sabina****, Caracalla Baths*****, Santo Stefano Rotondo****, Scala Scanta***, Palazzo Madama****\\

April 7: Rome:\\
Rome: Sant'Andrea al Quirinale****, Palazzo Quirinale*****, Trajan's Column****, Santissimo Nome di Maria al Foro Traiano***, Monumento Vittorio Emanuele II***, Santa Maria di Loreto***, Imperial Fora****, Santi Cosma e Damiano****, Villa Torlonia*****,  Santa Constanza****, Sant'Agnese fuori le Mura****, Palazzo Montecitorio****\\

April 8: Rome:\\
Rome: San Silvestro***, San Claudio**, Pantheon*****, San Andrea della Valle****, Villa Farnesina*****, San Pietro in Montorio \& Tempietto***, Santa Maria in Trastevere****, Santi Quaranta Martiri e San Pasquale Babylon***, San Francesco a Ripa in Trastevere***, Santa Cecilia in Trastevere***, Kolosseum****, Forum Romanum \& Palatine Hill****, Santa Francesca Romana****, Santi Giovanni e Paolo****, San Clemente (all lower levels too)*****, San Pietro in Vincoli***, Trajan's Baths*, San Carlo alle Quattro Fontane****\\

Arpil 9: Tivoli \& Rome:\\
Tivoli: Villa Hadriana*****, Villa d'Este*****, Villa Gregoriana*****\\
Rome: Santa Maria del Popolo****, Santa Maria dei Miracoli***, Spanish Stairs****, Trinita dei Monti***\\\

April 10: Rome \& Ostia Antica:\\
Rome: Galleria Borghese*****, Villa Medici***, Ostia Antica*****\\

April 11: Rome\\
Rome San Lorenzo fuori le Mura****, Baths of Diocletian****

\section{April 14: Grandvaux}
\label{Lavaux2013}

Grandvaux: Lavaux*******

\section{May 1: Le Mole}
\label{Mole2013}

Why the Mole: The Mole is a prominent mountain in the pre-alps. Since it is close to Geneva some call it the mock-up Mont Blanc since it appears to be of about the same size due to its proximity. Considering May 1st is a traditional hiking day, and a holiday, and my boss was in town all over from LA for a CERN workshop we decided it could be a great UCLA group event. Since we had one spare seat we invited Rosana, Indara's room mate on the hiking trip, and she gladly joined too.\\

Co-travellers:\\
Pieter: my fellow UCLA postdoc originating from Belgium. Pieter is a far more experienced hiker than myself, though clearly always up for hiking and the alps.\\
Andrew: UCLA undergrad turned electronic engineer who came to CERN in 2013 to help me with ongoing muon chamber testing.\\
Rosana: physics grad student at the university of Milan, based at CERN as well, originating from Cuba and more than pumped to explore the mountains.\\
David: physics professor at UCLA, having been at CERN as a fellow for a couple of years David is no stranger to mountains and hiking, considering that my program back in Lisbon was a tad full for someone just flying from LA the day previously a nice normal easy hike seems to be a better way to get started.\\
special mention: Line, working for CERN IT, Line welcomed us after the hike for beers and snacks to cool down our heated up skin, but more about that later.\\

The weather forecast was anything but spectacular, clouds and fogs all day long. Since May 1st just happens on that day we didn't shift the hike since it was supposed to stay dry but just covered. Since all was below layers of clouds and fog none of us though about bringing any sun screen. We all jumped into David's car and stopped 1/3 up the mountain to start the hike. The trail was largely snow free, but in some places still snow was covering the trail, but it was more on little side valleys. About 100-200 metres down the mountain top the fog started to get less dense and about 50 metres below the peak we were in bright sunshine. Good for views and good for photos. Which we clearly took plenty of, mountain tops above a sea of fog just looks magical. We had our snacks up there and also enjoyed the snow field which still covered most of the top, unfortunately even 1 hour on the top was enough for most of us, even Rosana did get burnt, the only one who thought ahead was David who brought a sun hat with him. Anyways Andrew had his first tasting of Swiss Landj\"ager (or Gendarms as the Swiss Romands call them). And we all walked back through the fog and a nice breeze. \\
Having arrived down the mountain we all went to Pieter's for more beer and snacks, where Line was a superb host but also commented on the fact that all four of us had forgotten to think about something practical such as sun screen considering we all had backpacks on us - guess the none practical traits of physicists. Still a very fun May 1st, particularly from a May 1st 2021 perspective which is the day I write this entry.\\

Le Mole****

\section{May 25: Lyon}
\label{Lyon2013}

And the evolution of this trip: Having been in Lyon before, I decided to try to see new things: the tourist association of Lyon offers from time to time guided tours of certain buildings in the city. This time, coincidentally both city hall and opera house had tours scheduled. Thus I convinced Andrew and my brother to join me. Having talked to Indara earlier, she didn't impress any particular interest in Lyon. But once she realised I was taking Andrew a full Saturday day away from here, she was quit displeased to say the least (some Buenos and beers might have made her happy again a couple of days later though, oh and Apfelschorle).\\

Co-travellers:\\
Andrew: after our trip to Paris a couple of weeks ago, Andrew wants to see a bit more of France and not only Paris.\\
my younger brother: having been in Lyon a couple of months ago, ready to see the places which are typically not easily accessible\\

Equipped with Baguette and a couple of cheese, salmon and sausages we had all ready before getting on that trip. Having arrived as early as possible we visited the cathedral of Lyon first on our own. A typical gothic style French cathedral, not as amazing as some of those world renown cathedrals of Ile de France, but still nice and interesting to see with stained glass, a nice astronomical clock, unfortunately with a closed off choir area, which was in renovation at that point. And then at 10 am our tour of the Hotel de Ville started. In total we saw the Staircase of Honour and 10 rooms. The city hall is in fact quite large and almost all rooms are from the Baroque to the Empire style era, or even older such as the Hall of Coat of Arms. All full of stuccos, gold leaf decorations, tapestries, paintings, all very much in good French Style. \\
After a lunch break of about an hour the tour of the Opera house started. While the Foyer is still from the previous Opera house and thus very classic ornate in typical 19th century style, the rest of the Opera house has been completely rebuilt. And the architect thought it is a great idea for the flair and the sounds to have everything in dark black. This does include all escalators, all walls of the rehearsal area, chairs, balconies, everything, Even a couple of years later Andrew still comments on the fact that he thinks the main auditorium of Lyon's opera might be one of the ugliest interior's he has ever seen (while I in fact did enjoy it) so I would say with modern buildings it is all a matter of taste after all. I do agree though that the round roof top looks a bit out of place if you consider the outer appearance. After the tour finished we went up to the remains of the Roman amphitheatre, but besides a few rows of seats not much else remains from it, so you can safely skip it. On a hill top Notre Dame de Fourviere, a neo-byzantine revival basilica overlooks old town. Particularly if you are into mosaics you will be blown away by this church. Usually such churches were built with great plans in mind, but the mosaics were never realised (e.g. Westminster Cathedral), or are very slowly installed one after each other (the catholic basilica in Washington DC). But here the full ceiling and the walls are covered in many mosaics, so definitely go there when you can. \\
Next to the Fourviere basilica are the remains of the Roman Theatre and Odeon, still very well conserved, in France second only to the theatre in Orange. In summer (unless it is summer 2020 and maybe 2021) plays and concerts happen here regularly. Then we walked through the Part Dieux quarter in old town, which is famous for its open courtyards and old houses. Don't forget that Lyon is also considered by many as food capital in France, so maybe try to have some Oysters or some Escargot while you can.\\

Lyon: Cathedrale***, Hotel de Ville*****, Opera****, Notre Dame de Fourviere*****, Roman amphitheatre*, Roman Theatre \& Odeon****\\

\section{June 16: Reculet}
\label{Reculet2013}

Jacob was back in the CSC factory for the summer. Since in June there is still snow on the trails in the alps, we decided to hike in the Jura. While I had been hiking there before, for example to a fondue place between Reculet and Crete de la Neige, I so far had not been up to Reculet itself after all those years. Since I had a car around CERN nowadays due to my work in Prevessin I drove up to the Tiocan parking with my younger brother, Jacob, Jesse, and Rosana who we convinced to join us on a hike. The hike itself is pretty nice, I decided to take it a bit slower than Jacob and Jesse which went ahead quickly, taking a couple of photos instead. The hike itself is pretty steep particularly through the forest, while it is in fact more easy at the end close to the top, since the trail is less covered in rocks. From up there we had a nice few, while Jacob decided to climb up the radio mast. We enjoyed he nice breeze up there, took a couple of photos, unfortunately a bit hazy to get good views of the alps, enjoyed the lizard which warmed itself up on my camera bag, and then walked all the way down, where we had ice cream and coffee to finish off the nice beautiful afternoon.\\

Thoiry: Reculet****\\

This was so far the last trip with Jacob, who decided to continue his career outside of high energy physics in medical physics and radiological devices. Rosana remained in the Geneva area for a couple of more years, but we didn't do further hikes with each other, but did enjoy some parties and dinners later on. It was also the last hike I did with Jesse but we also had several more celebrations together, and even after he moved all the way to Canada we met up later in Geneva from time to time and for a last time at ICHEP 2018 in Seoul, where we enjoyed some nice Korean barbecue getting a bit nostalgic about the good times we had back in Geneva (or Ferney for that matter).

\section{July 4: Flegere}
\label{Flegere2013}

How we got everybody to Chamonix: as UCLA postdoc I have many US American friends and the 4th of July is Independence day and a valid reason to take the day off go into the mountains, taking all grad students on a hike, and then enjoying a gigantic barbecue by one of the professors houses. \\

Co-travellers: all young UCLA folks at that point:\\
David: my boss having come over from UCLA once again. In fact it was him who decided nobody working for UCLA should work on that day (no matter if CERN says it is a working day), and we should rather spent another team building event, particularly due to the fact that we had also summer students around now.\\
Pieter: my fellow UCLA postdoc originating from Belgium. Pieter knows many hikes in the area, in fact he suggested we do the Lac Blanc hike which is nice but not too challenging if some people would not have large hiking experience so far.\\
Eve: a starting PhD student at UCLA originating from Greece. Eve had been a summer student at CERN for another friend of mine previously, thus she knew already about CERN itself. In the end Eve decided to focus on other fields for her PhD thesis and enjoyed more time in the LA area from then on.\\
Cameron: UCLA grad student, originating from the US, Kansas City but the part which is in Missouri. Cameron had been in Switzerland for almost a full year previously, working as an exchange student at ETH - my Alma Mater in fact -  or rather at PSI on the CMS silicon pixel detector. Thus no stranger to mountains, Cameron happily joined our hike as well.\\
Andrew: also from UCLA, originating from California and after our hike of the Mole more than ready to get into higher mountains this time.\\
Nick: a Texas A\&M undergrad, supervised by Indara, and he worked on rewriting parts of the code which I used for testing. Since he had arrived at CERN just a bit early on, and stayed at the same place like Andrew, we happily adopted him for this hike.\\

Unfortunately the weather forecast was once again not predicting sun for all the day, but what can you do. Once we arrived in Flegere we took the cable car to the middle station, seems one golfer on the golf course next to the cable car thought it would be fun to hit the cable car, and indeed he hit the target (if not intentionally kudos to Murphy's law). We started the hike in very foggy weather, we couldn't see more than 10 metres far, at time there was still snow on the trail. But nothing was too slippery or dangerous, the trail is also equipped with ropes and ladder on steep territory. Once we got close to the Aigulle du Belvedere and Lac Blanc the fog cleared up, but most of the lake was still covered in ice and a thick layer of snow. I was not as courageous as Eve, Andrew and Pieter to walk on the snow though, but seems it was firm enough to carry multiple people. On the other side of the valley clouds covered the mountain top, thus no view of Mont Blanc for us, the tongue of the glaciers of Argentiere or the Mer de Glace were though still visible, but it being summer they rather appeared grey with gravel than white in snow. On our way back we crossed some little mountain creek, where Andrew and Nick tasted some mountain water, later on the clouds cleared up to reveal the Dent du Geant with the Glacier des Periades. We also passed a large waterfall. Finally the clouds cleared up far enough that we could see Mont Blanc and the Glacier des Bossons, at least partially. At one point a couple of chamoises crossed our trail and were curious for a couple of seconds what we might be up to. And then the hike was finished, we were ready to eat, and thus it was ideal to have a couple of drinks and a lot of grilled food at Bob's BBQ.\\

Chamonix: Lac Blanc*******

\section{July 7: Milano}
\label{Milano2013}

And how we got to Milan this time around: Since I made it my task to show Andrew a bit of Europe, I thought after having done a bit of France it would be nice to see another country too, for example Italy. Indara heard that too, and since she wanted Nick to see a bit of Europe too she suggested he could join our trip to Milan as well. We decided on the date with everybody, Indara didn't want to join on that date but told us to go ahead. What she missed to consider is that we planned to take the earliest train possible. Thus Andrew and Nick stayed over at my place (they didn't like the fact that the road was so loud at night, and the fact that without an open window in summer the studio was not amazing either, thus no good sleep for them). The day before Indara planned her birthday party. Unfortunately she also did an outing to Bern but then messed up taking the right train, thus she arrived 2 hours late to her own party and she wasn't really happy when we left already a bit early, since getting on a train at 5:40 am means one should get a tiny bit of sleep before maybe.\\

Co-travellers on that trip: Andrew and Nick\\

Having failed to get into the Palazzo Reale in Milano the first time I visited, this time I put it as first item on the agenda. This time a photo exhibit was taking place in the former guest apartments, and Andrew enjoyed replicating the poses of the models to our great pleasure. Afterwards we saw the tapestry halls and the hall of lanterns of the state apartments. \\
Afterwards we walked over to close-by churches, and this was the first time I visited the ossuary of San Bernardino alle Ossa. This was the first time I walked into an Ossuary, which is a chapel where the skeletal remains of bodies are kept, if a cemetery is given up. Thus the chapel itself is decorated using bones and skulls instead of stuccos or paintings. I thought it was very fascinating and whenever I visited Milano nowadays I tend to bring people there. \\
There are many other baroque churches I visited on that trip, also beautiful once such as San Fedele, San Giorgio al Palazzo or Sant'Alessandro in Zebedia (many frescos decorate this one). This time was the only time I visited La Scala, Milan's renown Opera house, which had been rebuilt after a devastating fire in its original style. The quality of the opera house is obviously outstanding, the amphitheatre itself is a standard classical opera house, also not allowed to take photos, so maybe skip it if you want to see the theatre of itself. Do instead the Galleria Vittorio Emanuele II instead, an outstanding example of 19th century galleries. \\
The highlight of Milano is its cathedral. The duomo is a gigantic masterpiece of Italian gothic, the largest church in Italy. Its capitals are decorated with statues and sculptures, the windows in the choir are some of the best stained glasses from Renaissance times in Italy. Then we saw the museums of Castello Sforzesco. The former residence of the dukes of Milano is now a museum with old equestrian monuments, Michelangelo's last sculpture (the unfinished Pieta Rondanini), medieval altars, armors, and last but not least the actual halls of the castles itself, some of them painted by none other than Leonardo da Vinci (e.g. the Sala delle Asse). \\
Once again we didn't manage to get tickets for Leonardo da Vinci's last supper, but the aps of the convent church of Santa Maria delle Grazie itself is considered a prime example of Renaissance church architecture and one of Bramante's best works. Should you rather be into more historic things, Roman columns can be found next to San Lorenzo, whose baptistery chapel contains early christian mosaics of the 4th century or the romanesque basilica of Sant'Ambrogio. And then all back on the last train leaving Milan. Since it was in mid of July, we did also get to see the glacier covered peaks of the Swiss alps in sunset.\\

Milan: Santo Stefano Maggiore***, San Bernardino alle Ossa*****, San Fedele****, La Scala***, Galleria Vittorio Emanuele II*****, San Sebastiano***, Duomo***** (no roof this time), Castello Sforzesco*****, Santa Maria delle Grazie****, Santa Maria presso San Celso****, Sant'Ambrogio***, San Lorenzo****, Palazzo Reale****, San Giorgio al Palazzo***,\\ Sant'Alessandro in Zebedia****

\section{July 14--July 28: Sweden \& Denmark}
\label{2013:SwedenDenmark}

July 15: Stockholm:\\
Stockholm: Cathedral****, Royal Palace*****, Riddarholm****, Ridderhaus****, Parliament***, Concert House****\\

July 16: Drottningholm\\
Drottningholm: Drottningholm Palace***** (Chinese Palace****, Theatre****)\\
Stockholm: Ulriksdal Palace****, Hallwyl Palace*****\\

July 17: Mariefred\\
Mariefred: Gripsholm Palace*****\\

July 18: Stockholm:\\
Stockholm: City Hall*****\\

July 20: Skokloster\\
Skokloster Palace****, Stockholm: Royal Palace*****\\

July 21: Roserberg \& Stockholm\\
Rosersberg: Rosersberg Palace*****\\
Stockholm: Rosendal Palace****, Prince Eugen's Waldemarsudde****\\

July 22: Stockholm\\
Stockholm: Stockholm Archipelago*****\\

July 23: Stockholm\\
Stockholm: Hagapark***, Pavillon Gustav's III****, Wasamuseum*****\\

July 25: Hilledord \& Copenhagen\\
Hillerod: Frederiksborg Palace*****\\
Copenhagen: Christiansborg Palace*****, Holmenchurch***, Frederik's Church***, Little Mermaid****, Nikolai Church***\\

July 26: Helsingor:  \& Fredensborg\\
Helsingor: Kronborg****\\
Fredensborg: Fredensborg Palace*****\\
Copenhagen: Tivoli*****\\

July 27: Copenhagen\\
This day was my tourist outing with Susanne. We started out by the palace of Amalienborg, which is in fact a collection of four palaces, two of those occupied by the queen and the queen consort, and the family of the crown prince. In the Palace of Christian VIII the former offices and staterooms (including a ball and a throne room) of the classicist era of Denmark can be admired. The Palace of Christian VII contains several Rococo style rooms which are nowadays used in visits of state guests. Unfortunately photography was not allowed, the tour was crammed so hardly all fit inside the room, with the guide then complaining if you stayed in the doorway and not squeezed in. They had in fact two guides but one had to take the teenage girls to the rest room about 4-5 times on the tour.\\
After Amalienborg we went to Rosenborg. Besides hosting period rooms from Renaissance up to Classicist times the palace houses nowadays the Danish crown jewels as well collections of Porcelain and previous glasses. \\
In the evening Susanne and I had pizza before we watched the fountain laser show. Then we had more ice cream and had a free concert before the grand fireworks by the end of the night. This was unfortunately also the last time I met Susanne in person.\\

Copenhagen: Amalienborg Palace: Palace Christian VIII*** and Palace Christian's VII*****, Rosenborg Palace*****, Tivoli*****\\

July 28: Copenhagen\\
Copenhagen: Glypothek*****, Christiansborg: Parliament****, Cathedral***, Danish National Museum****

\section{September 5--September 8: Venice}
\label{Venice2013}

September 5: Venice\\
Venice: Ca' Rezzonico*****, Palazzo Zenobio****, Palazzo Franchetti***, Ca' Giustinian***\\

September 6: Venice\\
Venice: Ca' d'Oro***, San Giacomo di Rialto***, Palazzo Pisani Moretta****, Casa di Carlo Goldoni*, Scuola Grandi di San Rocco*****, Ca' Foscari**, San Pantalon****, Scuola Grande di San Giovanni Evangelista****, San Giovanni Evangelista***, Scuola Grande dei Carmini*****, Teatro la Fenice****, Palazzo Grimani****, Palaqzzo Querini Stampalia***, Palazzo Reale****, Biblioteca San Marco****,  Basilica di San Marco*****, Palazzo Ducale*****\\

September 7: Venice\\
Venice: San Salvador**, Palazzo Ducale*****, San Giorgio Maggiore****, Palazzo delle Prigione***, San Maurizio*, Santa Maria della Salute****, Santi Apostoli***,  Palazzo Fortuny*\\

September 8: Venice\\
Venice: Galleria Academia****, Palazzo Grassi****, San Giovanni Elemosinario***, Ca Corner***, Ca Pesaro***, Palazzo Pisani a Santa Marina*, San Giovanni e Paolo*****, Palazzo Mora**, Palazzo Michiel dal Brusa***, Palazzo Albrizzi***

\section{September 21: Chamonix}
\label{Chamonix2013}

I wanted to go to the mountains and it was easy to convince Indara to join, as well as James a student from North Eastern who was at CERN for ME11 CSC work. We started the hike from middle station of the Aiguille du Midi cable car walking over to the debris covered glacier of Pelerins. Unfortunately the gravel didn't seem so stable enough to cross the ice safely to get over to the Glacier des Bossons which had been our original plan. Instead we turned back to the cable car station again and purchased tickets for the second part to get up to Aiguille du Midi. There we enjoyed the few had cheese burgers, took photos, and enjoyed later some coffee before driving back to Geneva.\\

Chamonix: Glacier des Pelerins****, Glacier des Bossons*****, Aiguille du Midi*****

\section{November 7--November 11: Paris II}
\label{2013ParisII}

Did I mention that I love going to Paris? Well, as a matter of fact, I do, but this time I wanted to see places, which are less common or frequently visited by first time Paris travellers. But once again I had first time Paris travellers with me - Chris and Chris.\\

Co-travellers:\\
Eric: UCLA grad student after he didn't make it the first planned trip this year, Eric wants to make sure to not miss the train this time around:\\
Chris L: UCLA undergrad: our hardware project needed re-enforcement and we were happy to get Chris on board. As a first trip close to the region I suggested Paris. Since Chris has friends living in Paris, he clearly was up for it.\\
Chris C: US-American, a grad student working my hardware project building muon chambers. Always up to spend a good evening out, Chris thinks it is time for a longer trip. His aunt always told him how amazing Paris is, so he wants to make sure to take photos he can show her once he will be back home.\\

November 7:\\
Taking our usual late Friday night TGV out of Geneva, tolerating the not that amazing snacks on our way, we arrive at the hotel and get out room for four people on the top floor at a hotel close to the Place de la Republique.\\

November 8: Compiegne, Chantilly \& Paris\\
Typically people visit either the palace of Versailles or to lesser extend Fontainebleau. But only about an hour out of Paris is a third palace used by the emperors of the Bonaparte family, the palace of Compiegne. Almost all of the rooms have been decorated in the Empire style, they are in fact very beautiful too. The chateau might miss the old Renaissance or Baroque period rooms as you find them in the other two royal palaces, but what you find is definitely top notch too. It was a very rainy day, thus we didn't spend any time in the vast park. We visited the town church of St Jacques, which was a decent church. Plenty of churches and old houses belong to the buildings inscribed in the UNESCO world heritage as part of the pilgrim paths leading to Santiago de Compostela, and this church was one of them. Then we took a train to Chantilly. We walked through the little town along the forests and the horse riding race track to the castle of Chantilly. This chateau has very exquisite Baroque state rooms, which everybody was impressed by, particularly the large gallery, which has also paintings of the Three Graces by Raffael. Everybody else preferred skipping the French tour of the private apartments over a visit of the Baroque gardens. It wasn't raining as heavily anymore, so that was enjoyable too. And then we took the train back and did a late night visit of the Louvre. During the late night visits certain sections are usually closed off, in our case the Egyptian part of the exhibition. We saw though the Mona Lisa, the gallery of Apollon, the Venus de Milo, the Empire style rooms of the former ministry of finance of Napoleon III, as well as Assyrian and Babylonian artefacts. The Louvre is most probably my second most favourite museum I have seen so far (only beaten by the Vatican). \\

Compiegne: Chateau*****, St Jacques***\\
Chantilly: Chateau*****\\
Paris: Louvre*****\\

November 9: Paris, Malmaison \& Maisons-Lafitte\\
And our day started with a special visit of the Banque de France, or rather the representative rooms of the Hotel de Toulouse. Only a handful of tours are given by the French centre of national monuments, and the rooms can be visited during the European heritage weekend. I realised on the weekend I chose, the headquarters had a tour scheduled, and I successfully inscribed ourselves. Unfortunately at first they only claimed to accept payment by cheque, but my bank told me they wouldn't issue cheques anymore. Faced with that issue, I wrote the organisers an email, that either we would pay the exact amount of cash for all four of us, or we unfortunately would have to cancel our participation. This proposal was fortunately accepted, I handed over the cash to our guide, and we were ready to go. Indeed the tour was very nice, the golden gallery is a very fine and nice room with great frescoes and nice sculptures covered in gold leaf. Other meeting rooms were full with precious wood carving and tapestries. The Hotel de Toulouse is definitely worth the obstacles. After this visit both Chris's took of for the palace of Versailles. Both enjoyed it, took though the gold card to get through the vast park. Eric and I made out way to Malmaison. The castle of Malmaison had been the private home of Josephine de Beauharnais, the first wife of Napoleon I Bonaparte, and the last house Napoleon himself lived in, before being sent to exile on St Helena in 1815. The chateau sits in a nice small garden, with small but nice and finely equipped rooms, including the very beautiful library. Then we both had one grilled chicken from a food truck, and then we wanted to get to Maisons-Lafitte. That proved to be more difficult than we thought. The RER line which was supposed to get us there straight away had been closed for construction on that day, but another line was supposed to be used instead. After arriving there we found out, this line as well was closed for work and it suggested we use the previous one instead. Then we found out that a bus line was supposed to get us there, but we had to find out, indeed it has a final stop exactly at the castle we wanted to reach, but only later in the evening or early in the morning. We then had a coffee at McDonalds in order to get on a wireless network which could tell us what to do instead, and it provided us with a solution: We had to take a tram line up to its final stop and then transfer to another bus. And we finally arrived at Maisons-Lafitte, but over an hour later than originally planned. Now I decided to watch the state rooms. These were illuminated partly in green and red to test out the lightning for a private event which was planned to take place later that evening. While I managed to see all the state rooms, Eric decided to see the private rooms instead. Unfortunately we arrived that late, that we each didn't manage to see all the rooms. But what I saw was very nice, so not a wasted opportunity. Who knows, if I will come there again at some point (at least not until now in 2020). And then it was time to visit the observatory on the roof terrace of Tour Montparnasse. Still a very superb view, but instead of offering an unobstructed view of Paris, now large life-size glass windows had been installed, on some places with tiny gaps to take photos, clearly not improving the visitor experience. Still the view of Paris, particularly of the Eiffel tower is and remains magnificent (but a bit more expensive than walking up the second floor of the Eiffel Tower, but a better view since you see the Eiffel Tower). On our way back to the hotel Eric and I stopped for a short while by Notre Dame, took a view pictures of the cathedral, the river, as well as the Hotel de Ville by night and then had food.\\

Paris: Banque de France (Hotel de Toulouse)*****, Tour Montparnasse*****\\
Malmaison: Chateau****\\
Maisons-Lafitte: Chateau****\\

November 10: Champs-sur-Marne \& Vaux-le-Vicomte\\
Today we all planned to have an excursion day doing nothing in Paris downtown. The first stop was the chateau of Champs-sur-Marne. A wonderful small Baroque palace with several wonderful nice rococo salons, like the Chinese salon, it used to be the French guest house between 1959-1974. The chateau is surrounded by a French style park. After a short lunch we took the RER to Meudon, and then a bus to the large estate of Vaux-le-Vicomte. The Baroque chateau was built for Nicolas Fouquet, the superindent of finances during the times of Louis XIV. The decoration of the halls were done by Charles Le Brun, the park was designed by Andre le Notre, both later worked on the construction and planning for the palace of Versailles. Regarding the castle itself, particularly the state rooms on the lower floor are magnificent with the state bedroom of the King's apartment and the game cabinet. While we were there a chocolate festival was taking place, the palace was quite crowded, but the chocolate was very tasty. The park is quite extensive. It is the oldest Baroque garden in France and its layout was later used as basis for the gardens of Versailles with fountains, sculptures, vases, and a geometric parterre. The bus arrived pretty late and we had to ran to catch the train, Chris C fell, but besides a few bruises nothing drastic happened. The four of us just made the train a few seconds before it took off. Chris L had a long night out with his friends, while the three of us just had a dinner followed by a night cap.\\

Champs-sur-Marne: Chateau****\\
Vaux-le-Vicomte: Chateau*****\\

November 11: St Denis \& Paris\\
On our last day we stayed closed to the city centre. We started in St Denis with a visit of the basilica, enjoying once again the Royal tombs and the royal crypt. I am always impressed by the tombs of Louis XII. and Francois I, which show am impressive detail and skills of French Renaissance artists. The later tombs of the Bourbon kings in the crypt are in fact less sumptuous. The next highlight was a visit of the Opera Garnier, I am always happy to come back to what I consider the most beautiful theatre/opera/concert building I have been to so far. After the obligatorily stop by the Conciergerie, we were all impressed the stained glass of Sainte Chapelle (the renovation was still ongoing), Chris C and Eric bought little pieces of stained glass as memory and presents. And last but not least we stopped once again by Notre-Dame cathedral for a while before getting dinner, this time I opted for duck and fries. Then we had a photo stop by the Arc de Triomphe before getting a couple of snacks and drink for our 3 h long train ride back to Geneva.\\

St Denis: Basilica*****\\
Paris: Opera Garnier*****, Sainte-Chapelle*****, Conciergerie***, Notre-Dame*****

\section{November 30: Prangins}
\label{2013:Prangins}

The Swiss national museum in the French speaking part is situated in the Castle of Prangins. The castle itself still has some of the original Baroque interior conserved in three rooms as well as a baroque style garden. Else historical documents and write ups are displayed, cute but nothing too special.\\

Prangins: Chateau de Prangins****

\section{December 27: Strasbourg}
\label{2013Strasbourg}

Since staying in the local village solely for two weeks seems rather boring, my parents, my brother, and my sister did a short trip to Strasbourg, where we paid a visit to the M\"unster with its famous one tower facade. In fact the minster (and cathedral of the local bishop) had been the tallest building on Earth for a couple of centuries before being topped out by Cologne cathedral. While the choir area and the dome are still in Romanesque style most of the cathedral is a bright gothic style master piece with lots of beautiful stained glass windows as well as a row of tapestries covering the walls.\\
Palais Rohan used to be the residence of the bishop and later has been used by the French kings and emperors on visits to Strasbourg. Most of the palais is nowadays the local arts museum but the main state rooms still show their original Baroque and Empire style. Then we had a short walk through the former German old town as well as the French part of old town, called Petite France situated on a large island between two river arms, before we had lunch at a local restaurant.\\

Strassbourg: M\"unster*****, Palais Rohan*****
\chapter{Year 2014}
\label{2014}

\section{April 18--April 22: Bavaria}
\label{2014:Germany}

Co-travellers: \\
Eric:\\
After Eric was back from his semester in Los Angeles, it was time to go on a trip again. What could be better than showing him one of the beautiful regions of Germany. Little did he know that a year later he would relocate to Garching.\\

Indara:\\
Not only Eric is back, but Indara as well. After we did a couple of trips to France and Switzerland, Indara is ready for a longer trip, provided that we the itinerary includes shopping or coffee time.\\

How this trip came into existence:\\
Bavaria is what most international tourists think of first, when asked about Germany. Since I had been in Bavaria multiple times, I felt confident enough to set up a nice list of places to see. Since this weekend was over Easter holidays, we just had to take 1 day off work for a 5 day little tour. And there is no speed limit on the Autobahn - that's a thrill for all drivers as well. I more or less pressure talked Eric into that trip -- now knowing that his alternative would be visiting his girlfriend in Boston -- only for two nights, but that's better than nothing. Thank god later on I did make up for it, by planning another good trip for the three of us, but she told me she was pretty mad on me for keeping him for my trip weekend.\\

April 18: Neuschwanstein in a snow storm:\\
Starting the day with the earliest flight to Munich. Having two drivers is always a good idea. Since both Indara and Eric were excited to drive in a place without a speed limit, it was clear who would be both drivers. We originally reserved a VW Golf type of car. Since they were out of those, we got a Mercedes B class instead. Eric wasn't displeased. And off we went deep south to Schwangau. Although we only would be in Bavaria for five days, our plan included to visit quite a few palaces, so we got the two week palace tickets by the Bavarian palace administration (one single, and one couple version). It gives you access to almost all important Bavarian palaces and castles (also some monasteries) for only 24?? EUR. We still opted to reserve Neuschwanstein tickets, since without a reservation it can happen that you have to wait more than 4 hours to get inside, or even for the next day (so a well invested 3 EUR reservation fee). Hohenschwangau is still owned by the former Bavarian Royal family, thus you still have to pay for that one. So Hohenschwangau was our first stop (no photos). Rebuilt by Maximilian II of Bavaria as hunting palace, it is built in a neo gothic style with lots of frescoes in most of the rooms. Some of which were restyled by his son Ludwig II later on. Not as grandiose as Neuschwanstein, but a finished interior and still quite cute. After touring Hohenschwangau we went up to Neuschwanstein. I hoped we would have a good view of the castle from the bridge over the P\"ollat gorge, but we ended up in a snow storm. Thank god our tour started soon after and we were in a warm place, and a nice place, not as crowded as German speaking tours a couple of years earlier. Once we were done with Neuschwanstein, we drove over to the Lechfall and the Lechklamm, and checked in. Then we had a really nice dinner at Gasthaus Krone, and afterwards ice cream. Since we shared a room, Indara decided to clean up the mess we made while unpacking.\\

Schwangau: Hohenschwangau**, Neuschwanstein***\\
F\"ussen: Lechfall*, Lechklamm**\\

April 19: Palaces, Churches, and Dirndls\\
We started the day early driving over the snow-covered Autobahn to Kempten. We first had coffee at a market stand, and then started out in the Basilica. A really large, maybe even over dimensioned church for Kempton, built in baroque style. Then we continued with a guided tour of the Residenz. Unfortunately it had to be in German, the leaf lets in English were pretty sparse, and the fellow guests (only two) complained that we would be too loud, or take too many photos (which is allowed). The tour encompasses the seven rooms of the state apartment, with the audience room and the throne room as highlights of late Bavarian baroque. The other rooms are nowadays used for local administration. Then we had to deal with Indara's main point of the day -- Dirndl shopping. Closeby of the Residenz is a Dirndl shop, where modern versions of the traditional Bavarian dresses can be bought (not as overpriced as in some shops in Munich). In the end we had to decide between a more classical blue checkered dress, and a more modern dark green dress. And we decided to go for the more fancy modern version (after revisiting each of the dresses several times again, combining it with appropriate belts and aprons). By that time I realised I didn't have my mobile phone in my pocket. The previous day I had given it to Indara, since her CERN phone died and my phone's battery was still fine. She gave it back to me at the end of the day, but it was gone. Notifying the hotel, they claimed the room was empty, only to find the phone over two months later in one of the wardrobe drawers (and sending me the phone card back to me). Anyways the next stop of the day was Wieskirche -- the highlight of Bavarian Rokoko, standing in the middle of nowhere among green meadows. It was my fourth visit -- and each time i am amazed about its beauty again. The most beautiful of Ludwig's II castles is Schloss Linderhof -- also the only palace which was finished during the King's lifetime. The park is large, the setting in the alps is amazing; even the park houses - in Moroccan and Moorish style. The king loved pea-cocks, thus many statues of those can be found throughout the palace, the main rooms are in blue colors, also the king's favorite. The dining room has a table which can be lowered to the lower floor (similar features can be found in Neuschwanstein too). The king was a huge fan of Richard Wagner, thus a grotto with a Swan Boat and an artificial waterfall was constructed illuminated in multiple colors while playing Wagner's music to entertain the king. The large fountain operated by a natural gradient pressure, pushing it up about every hour on the hour. Quite close to the Schloss is the famous monastery of Ettal, check out the monastery church if you have time. The final stop of the day was Garmisch-Partenkirchen, the host of the Winter Olympics in 1936. The ski jumping hill has been renovated multiple times, the last time in 2007. A few minutes later the entry to the slot canyon of the Partnachklamm starts, one of Germany's most famous Canyons. We were all very impressed, but then we still had to make our way to the hotel in Rosenheim, where we ordered the Bavarian version of sausage salad with Apfelstrudel.\\

Kempten: Residenz**, Basilica St Lorenz**, Steingaden: Wieskirche***, \\
Ettal: Schloss Linderhof*** (Venusgrotte***), Monastery**,\\
 Garmisch-Partenkirchen: Ski Jumping Stadium*, Partnachklamm***\\

April 20: Barbara oh Barbara\\
Today the largest lake of Bavaria is our main destination - the Chiemsee. Getting on a lovely little boat -- Barbara -- we set over to the larger of two islands in the lake -- Herreninsel. On this island Ludwig II built a replica of Versailles together with a vast French garden with many baroque style fountain, including a copy of the Latona fountain of Versailles. The centre piece of Herrenchiemsee is the Hall of Mirrors - slightly larger than the original in Versailles, more in a golden tone than the original blue. The state bed room is in fact the most expensive room built in the 19th century. Since Ludwig II died by committing suicide in a lake, the palace was left unfinished, the second stair case is still a brick shell. The side wings have not been realised as well. Afterwards we walked over to the old palace, in fact an old monastery. At this place Germany's constitution had been formulated back in August 1949. And back to our car, once again with Barbara. Now we continued our trip to Landshut. There we visited first the Residenz, the old part being heavily inspired by Italian Renaissance Palazzi. The inner decoration is heavily reminiscent of Italian stucco ceilings, including the frescoes. The later king Ludwig I studied in Landshut, during this time a couple of rooms were redecorated, including Wallpapers in bright green, the Paris green, in German called poisonous green (``giftgr\"un''), since it contains arsenates. Afterwards we paid a visit to the old gothic brick church of St Martin, and then we dropped the car at Munich airport. Where another passenger recommended us to get on a group ride ticket with him to the city centre, which turned out to be cheaper for all of us.\\

Herrenchiemsee: New Palace***, Old Palace**, \\
Landshut: Residenz*, St Martin*\\

April 21: Palaced out\\
In Munich there are two real nice palaces -- and we wanted to see both of them on this very day. Starting the day in Nymphenburg, the summer palace of Munich, as well as the birthplace of Ludwig II. The main hall (Steinerner Saal) and the adjacent bedrooms and cabinets are sumptuous baroque rooms, The vast garden is of English style with four park palaces. Amalienburg is considered the highlight of secular Bavarian Rokoko. The chapel of Magdalenenklause is decorated with shells and old snail houses. Afterwards we had M\"unchener Schnitzel closeby, which had a filling with leek and spinash. We did stop by Max Krug and obtained Steins for Eric (and Siyi) and Indara. Then we visited the giant palace of the Residenz, the largest palace of Germany. The royal apartments were still in renovation (still after over 6 years, after all they reopened in 2019), but the rooms of the Prince-electro times, and the rich rooms (Reiche Zimmer) - still make up over 80 rooms. After going through all those rooms, Eric and Indara were clearly palaced out, and they had no more attention span for the dozen rooms of the treasury. As usually when in Munich we had Grandma's meatballs (``Omas Fleischpflanzl'') at the N\"urnberger Bratwurst Gl\"ockl am Dom (really amazing once more).\\

Munich: Nymphenburg*** (Badenburg**, Amalienburg**, Pagodenburg**, Magdalenenklause**), Residenz*** (Cuvilies-Theater**, Treasury***), St Michael**\\

April 22:\\
Indara decided to meet up with one of her old friends in Munich (and his new born daughter). She also had to deal with slides for a meeting. Thus Eric and I spent most of the day alone. This time we went to Schleissheim, a village, where the Bavarian rulers built their summer residence.  The original plan of the complex has not been realised, since the war of spanish succession put the planned extension of the main wing on hold. The Corps de Logis is though very impressive. Some of the rooms have been transformed into a gallery of Baroque paintings, but most of the halls are decorated with impressive stuccos, gilded wood carvings, tapestries, and large frescoes. Period furniture is still available, particularly in the state bedrooms. The French garden is nice, at the end is the Schloss Lustheim. The main hall of this pleasure palace was unfortunately in renovation at this point. In Munich we rushed a bit through the three main churches of Munich. The inner city of Munich is not that large, in fact you are able to cross it within at most half an hour. We met up with Indara and her friends by the Chinese Tower as previously planned. And we had lunch and a Mass of beer there. After her second Mass, wanting to taste both Pils and Starkbier, Indara attended her meeting, and then it was already time to say goodbye to Munich once more. On our flight we transferred in Zurich, getting a couple of chocolate bars from Swiss on our way, while Indara lost track of her hat at some point.\\

Schleissheim: Old Palace*, New Palace***, Lustheim Palace**, \\
Munich: St Michael**, Frauenkirche*, Theatinerkirche**, English Garden with Chinese Tower**

\section{May 1: Neuchatel}
\label{Neuchatel2014}

What should you do on a free day -- preferably not work. I didn't plan any large outing beforehand, thus my brother and I decided to get ourselves to Neuchatel in order to visit the castle. The castle is one of the largest in Switzerland, it is nowadays used to host the cantonal parliament of Neuchatel. The interior has been clearly adapted for that purpose, though also keeping a couple of representative rooms. The Collegiate church next to the castle is a medieval church with a couple of statues, portals and a nice decorated roof top. Other than that Neuchatel is a cute little town just by the lake of Neuchatel, in good weather conditions one can see as far as the Bernese alps, dominated by Eiger, M\"onch, and Jungfrau.\\

Castle**, Collegiate**

\section{May 4: Stresa}
\label{Stresa2014}

Stresa: Isola Superiore**, Isola Madre**, Isola Bella***, Baveno: Santi Gervasio e Protasio*, Domodossola:  St. Gervasius \& Protasius*

\section{May 18: Baden-W\"urttemberg}
\label{Donaueschingen2014}

Donaueschingen: F\"urstenberg Museum*, St Johann*, Sigmaringen: Castle**, Beuron: Abbey**

\section{May 29: Gorge du Durnand}
\label{Durnand2014}

Gorge du Durnand***

\section{June 8: Grindelwald}
\label{Grindelwald2014}

Pfingsthorn***

\section{June 22: Meiringen}
\label{Meiringen2014}

Another weekend, another trip to the alps, once again close to the Interlaken region. This time we wanted to admire the famous Reichenbachfalls, immortalised by the final fight between Moriati and Sherlock Holmes. Well we took the cable car up and walked down. But it was a bit of a let down. Having seen many alpine waterfalls, at least in June the Reichenbachfalls didn't appear overly impressive for out tastes. A bit disappointed we continued to hike along the trails to reach the Aareschlucht. Now THAT is an impressive slot canyon, a gorge which definitely deserves a visit. The trail takes about an hour, both ends can be reached with a little train from Meiringen (else it is also less than a km away from the train station, otherwise you can also go through the gorge twice). Then we took the train back to Spiez and got on a boat over to Oberhofen. A while ago we considered getting off at Oberhofen while on a cruise of Lake Thun, but decided against. This time we thought we had enough time for a short visit. The castle is nicely located on a peninsula reaching into Lake Thun. Not only cute from the outside, the inside has carved rooms, more of a neo baroque or classicist interior, as was en vogue in the late 19th or early 20th century. The castle (and the adjacent gardens) for sure deserve a visit.\\

Meiringen: Reichenbachfall*, Aareschlucht***\\
Oberhofen: Castle**

\section{July 24--August 4: South England}
\label{southengland2014}

I enjoy going to London, but by now it had been quite a while since I had been there the last time. This time I planned to do also the ``small places'' and sights in London, maybe also with small trips out of London. My sister enjoys being in London too, albeit spending times with friends in the London greater area as well. Since she wanted to leave late July and Buckingham palace opens up only in August, I extended the trip by a couple of days. Thus for once we all were up for a big family trip again with my parents, my little brother, and my sister. Our planning made the list of places larger and larger, and we decided to get the London Plus Railpass, which would also allow us to see most of South East England. Particularly my mum was very excited about this proposal, since she had wanted to do a whole organised tour of the Southern part of Great Britain for quite a while. Since we wanted to see all former Royal Palaces of London we got ourselves a partner ticket for two including the four old big palaces of London, as well as a discount on Kew Gardens.\\

July 24: London\\
My brother and I took the earliest flight out of Geneva to Gatwick and we met our parents at our hotel, where we all stayed in a four-bed room together. They had already flown in the previous day, seeing some museums with my sister during the afternoon. While my dad and my brother wanted to have a longer lunch, my mum and I decided to go to Syon House instead. Syon House is a large noble house in the suburbs of London with lots of nice classically decorated rooms, Queen Victoria had spent parts of her child hood in this house. The large greenhouse is quite famous as well with nice orchids. Then we took the train to Kew Gardens, where we met up with the rest of the family. Kew gardens is one of the largest botanical gardens in Europe, also one of the oldest. Besides giant greenhouses with palm trees, orchids, cactuses, as well as alpine plants, you can enjoy yourself on a tree top path, stroll across a Japanese Garden with a Japanese style pavilion, and a Chinese style pagoda. Inside the gardens is also the rather modest Kew Palace, once a private getaway for the Royal Family. Some of the rooms are still in their original state, as well as the tiny French style palace garden. Nice and cute, but since it was a private home, don't expect to see any large state rooms. We did enjoy our summer afternoon and evening out in this vast park. And then we had dinner at a fish restaurant.\\

Syon House**, Kew Gardens**\\

July 25: Windsor \& Hatfield\\
My parents and my brother joined me on this trip, but my dad and my brother decided they rather want to see Eton College than Windsor Castle, so it was me and my mum alone for the castle. The core of Windsor castle is from medieval times, the state apartments are from Baroque times onwards. Parts of the castle contain the private and semi-private chambers of the Royal family. While the private rooms are clearly off-limits all around the year, three of the semi-private rooms can be visited in winter time. The state apartments are very impressive, the King's dining room and the audience chamber of the queen are particularly impressive from these times. The Waterloo hall is the most impressive part of the Gregorian times, the Reception Hall, which had to be largely rebuilt after the devastating fire of the 1990s is an amazing room with gilded walls, tapestries and large windows overlooking the river. The second part of the castle is dominated by the large gothic St George's Chapel with nice wood-carved choir stalls, well-sculptured altar pieces and tombs in the side-chapels and stained glass windows. Photography is forbidden inside the castle and the chapel, but the ticket can be converted into a year ticket. We had a little snack by the train station, also taking a view photos of the steam engine ``The Queen''. Then we took trains over to Hatfield. My highschool English lessons book centred around the town of Hatfield and what the kids there were doing. Among those stories was a visit to Hatfield House. Remembering those stories I decided to just enquire what there would be to see. I liked what I found and convinced everybody to join me. The suggestion to visit St Alban's afterwards was voted down though. From the train station it is just a short 15 minute walk to Hatfield House. The Marble Hall is home to one of the rainbow portraits of Queen Elisabath I, the King James room is nice as well. The long gallery has a nice ceiling, wood carved walls, marble fireplaces, it has appeared in historical movies as well. Then we walked through the large French style gardens of the house, also having a short look into the Great Hall of the old palace, a big gothic hall which was being prepared for a Banquet, thus we couldn't walk through it on this day. And then we took the train back to London where we had dinner at the cafe of the Victoria \& Albert Museum, where my mum couldn't deal with the spiciness of the Chicken curry (it was in fact not very spicy). The museum had its late opening day, it does house artefacts from Nepal, India, China, Silverware, paintings from the 19th century, the Devonshire tapestries from the 15th century, as well as copies of famous artefacts, just like Trajan's column or Marc Aurel's column in Rome. I thought the museum was a bit hit and miss, clearly not my favourite in London itself even.\\
  
Windsor Castle***, Hatfield House***, London: Victoria \& Albert Museum*\\
 
July 26: Brighton\\
Seaford: Seven Sisters***, Brighton: Royal Pavillon**, Brighton Pier*\\

July 27: London\\
London: Eltham Palace**, Kensington Palace**, British Museum***\\

July 28: Isle of Wight:\\

Coming back to London I convinced my mum to try a Turkish restaurant. She rather wanted to go to an Italian restaurant, claiming she wouldn't know what could be a good dish from Turkey she'd enjoy. Seems a mixed place of Kebab and Koefte was just down her alley. She enjoyed it that much, that we ate there twice more on this trip.\\

East Cowes: Osborne House**, Totland: Alum Bay with Needles***\\

July 29: London:\\
All Hallows-by-the-Tower*, Tower of London**, St Paul's Cathedral***, Mansion House**, Freemason's Hall*\\

July 30: Oxford \& Blenheim Palace:\\
My mum and I started our day early, since we wanted to participate in a special tour of Bodleian Library which would also include a sneak peak into the Radcliffe Camera Reading rooms. The tour started in the magnificent Divinity School, a large gothic room with lots of sculptures, nice elaborate vaults, before continuing into the old reading rooms of Bodleian Library with its wood carved shelves and paintings. The Radcliffe Camera is more of a classical building. We also walked though the modern extension of the library.Then we met up with my dad and my brother, which had arrived in Oxford on a later train. Altogether we visited Christ Church College with its large hall, which inspired the big hall in the Harry Potter movie series. The cloisters and stair cases played a role in those movies too. The church of the college is in fact the cathedral of the diocese of Oxford as well, albeit a smallish cathedral. Another church which dominates the main road in Oxford is the gothic Virgin Mary church. After a small snack we took the bus over to Woodstock which is home of Blenheim Palace. Blenheim Palace is one of the largest palaces in England, built in Baroque style by the Churchill family. The state rooms can be visited, the private apartments of the dukes on special occasions. The rooms are filled with frescoes, marble statues, paintings and tapestries. The birth room of Winston Churchill is part of the tour, as well as a small exhibition of his private items. The library is one of the largest private classicist library in the country. The gardens are large and vast with a couple of pavilions, fountains, a cascade as well as some little flower gardens. We took the bus back to Oxford, had dinner there in a small basement restaurant/pub having also some local beers, and then took the train back to London.\\
 
Oxford: Bodleian Library with Divinity School***, Radcliffe Camera*, Christ Church College***, Virgin Mary Church*, Woodstock: Blenheim Palace***\\

July 31: London:\\
Houses of Parliament***, Banqueting House**, St Margaret's*, Westminster Abbey***, Jewel Tower, Westminster Cathedral**, Apsley House*, British Museum***\\

August 1: Rochester, Canterbury \& Dover\\
We originally wanted to take the train from Victoria to Rochester. But then the train was delayed indefinitely, thus we took the metro over to Paddington and went to Rochester from there. We started our day with the cathedral of Rochester, a decent cathedral which looked like a romanesque-gothic mixture including modern murals from 2004, but in a rather romanticised style. Next to the cathedral is the old castle, which looks still almost intact from the outside, but is clearly ruined for centuries, it had been used as Robin Hood's castle in the 90s Kevin Costner movie. Quite interesting to climb the upper floors with superb views of the cathedral, the river, and the rusty old submarine of U-475 Black Widdow. And on by train to the second town of the day: Canterbury. The ruined remains of the keep are all that is left from Canterbury Castles, but still just a short detour on our way to the cathedral, which is the seat of the spiritual leader of the anglican church. A really beautiful impressive gothic church, with many tombs, stained glass windows and beautiful screens, a cloister with ruins of a monastery and a chapter house with a nice ceiling. And then it was time to take the train to Dover. We walked up to the castle, where folks tried to replicate the original medieval interior, which falls a bit flat though and rather looks like a Disney replicate. Dover castle had been used as a hospital during the Blitz, we took a guided tour of what remains of the former rooms. Then we walked over to the harbour, where ferries from France were landing and departing. In the distance one could see the start of the cliffs of Dover.\\

Rochester: Cathedral*, Castle**\\
Canterbury: Castle*, Cathedral***\\
 Dover: Castle*\\

August 2: Stonehenge, Salisbury \& Bath\\
This day was supposed to be one of the highlights of our trip, it clearly turned out to be the day with one of the heaviest rainfalls I walked through. We took the train out to Salisbury and got on the bus there, it did rain quite a bit on the ride, but once we switched to the bus which was bringing us from the parking lot closer to the site that is Stonehenge, the rain clearly got worse. Some people preferred to stay on the bus, me, my parents, and my brother decided that we didn't get there just to have one photo and then just turn our backs. So we walked around the circle: clearly it is an impressive sight to see something that old just standing there in front of you. In order to prevent tourists from damaging the site, you don't get any closer than 5 or 10 m, but still impressive. Unfortunately the rain was coming from all sides, even an Umbrella was useless. Thus we all rushed into the museum and warmed ourselves up and tried to dry up by a bit. Unfortunately not with much success, but we got on the bus to Salisbury. The gothic cathedral is one of the famous English cathedrals, completely dominated by its impressive spire. Most probably the cathedral survived the war without any damage based on the fact, that the spire was a useful landmark for the german air force. The interior of the cathedral is one of the finest in England as well. It even turned out to be a sunny afternoon which helped us to further dry up walking around town a bit. We got back to Salisbury, had a coffee break, and wondered if our London plus pass would cover actually the trip from Salisbury to Bath. Bath and Salisbury were clearly mentioned as cities, which could be reached directly from London with the pass, but the direct way to Bath doesn't lead over Salisbury, although the time spent via Salisbury is roughly the same as it turned out later. Once we arrived in Bath we first stepped into the Abbey. While it is a decent church (pales a bit compared to Salisbury), the way the guards tried to force you into donating 2 pounds was appalling. I usually donate, but AFTER I have seen the place, otherwise just charge, I am more than happy to pay a fee, which I know goes into the conservation efforts of the place. At least we got to hear an organ recital which was very nice too. Afterwards we visited the Roman Bath, in fact the Bath has been maintained over many centuries, thus it is still in a functioning superb state. Bath is also famous for its squares, though the Circus and the Royal Crescent failed to impress me at all. OK they are nice shaped squares, and the facades of the houses are not shabby, but that can be said about many squares in many European cities.\\

Amesbury: Stonehenge***\\
Salisbury: Cathedral***\\
Bath: Abbey**, Roman Bath**, The Circus, Royal Crescent\\

August 3: London\\
Hampton Court Palace***, Strawberry Hill House**, Chiswick House*\\

August 4: London\\
And we started our day at Buckingham Palace. Clearly one of the most famous palaces in the world, it is also one of the most beautiful. The state rooms are large classicist halls, a large gallery is part of the state apartments as well, also the large ball room. As in all palaces still used by the royal family, photos are not permitted inside, but you can take some on the vast grounds of the palace, although parts of the gardens not accessible either. Then our parents left for their flight, while me and my brother wondered if we should do Clarence House. Deciding against that we instead saw the arts exhibitions in Somerset House, which was more or less OK. Then we took a metro to the Guildhall, which turned out to be a nice neogothic large hall.\\

Buckingham Palace***, Somerset House*, Guildhall**\\

\section{August 22-August 31: Vienna, Bratislava \& Budapest}
\label{austria2014}

August 22: Vienna:\\
Hofburg: State Apartments**, National Library***, Treasury***, Silver Chamber*, Garden Palace Liechtenstein**, City Palace Liechtenstein***, Stephansdom**\\

August 23: Eisenstadt \& Fert\"od:\\
Eisenstadt: Esterhazy Palace**, Fert\"od: Esterhazy Palace***, Hofburg: Great Redoutensaal**\\

August 24: Bratislava \& Schlosshof:\\
Bratislava: Palais Grassalkovich*, Bratislava Castle*, Primatialpalais**, Palais Apponyi*, Old City Hall*, Martin's Cathedral**, Schlosshof: Schloss Hof***\\

August 25: Vienna:\\
Karlskirche*\\

August 27: Vienna:\\
Peterskirche**, Albertina**\\

August 28: Vienna:\\
Schloss Sch\"onbrunn***, Hermesvilla***, Staatsoper**, Prince Eugen Winterpalais**, Stephansdom**, Schlosstheater Sch\"onbrunn**\\

August 29: Vienna\\
Sezession's Building*, Upper Belvedere***, Lower Belvedere***, Old Danube**\\

August 30: Budapest \& G\"od\"ollo:\\
Budapest: Parliament***, Stephen's Basilica**, State Opera***, Fischerbastei**, G\"od\"ollo: Palace**\\

August 31: Budapest:\\
Castle with National Gallery***, Matthias Church***, Stephen's Basilica**, Great Synagoge \& Jewish Museum**

\section{September 6--September 14: Tuscany}
\label{Tuscany2014}

September 6: Firenze\\
Palazzo Pitti***, Boboli Gardens with Buontalenti Grotta***\\

September 7: Pisa \& Firenze:
Pisa: Duomo***, Battisterio*, Palazzo Blu*, San Michele in Borgo*, Firenze: Orsanmichele*, Palazzo Medici Riccardi***, Galleria Academia***, San Gaetano**\\

September 8: Bologna:\\
Oratorio San Filippo Neri*

September 9: Bologna:\\
Cathedral*

September 10: Bologna:\\
Basilica San Petronio*, Archiginnasio**, Basilica San Stefano**, Palazzo d'Accursio**

September 11: Bologna:\\
Palazzo d'Accursio**, Santa Maria della Vita*, Basilica San Domenico**, Santuario de Corpus Domini**, Palazzo Pepoli Campogrande**, Basilica Santo Stefano**, San Giacomo Maggiore**, Basilica Madonna di San Luca*, Cathedral*\\

September 12: Firenze:\\
San Lorenzo*, Battisterio***,  Palazzo Vecchio***, Medici Chapel***\\

September 13: Firenze \& Siena:\\
Palazzo Davanzati**, Duomo** (Dome**, without it not even one star for the church), Casa Martelli**, San Spirito*, Santa Croce***, Siena: Palazzo Pubblico***, Duomo***\\

September 14: Firenze:\\
Villa della Petraia***, Villa Corsini*, Palazzo del Bargello**, San Lorenzo*, Santa Maria Novella***

\section{September 19--September 21: Paris}
\label{Paris2014}

September 20:\\
Palais d'Elysee***, Hotel de Beauvau*, Hotel de Charost**, Hotel de Monaco***, Hotel du Chatelet**, Hotel de Clermont**, Hotel de Villeroy*, Hotel de Castries*, Hotel de Matignon***\\

September 21:\\
Palais du Luxembourg***, Petit Luxembourg**, Hotel de Ville***, Palais Royal***, Hotel de Bourvallais**, Hotel de Talleyrand**, Hotel de la Marine***, Hotel de Salm**, Hotel d'Estrees**, Hotel de Roquelaure**

\section{October 4: Creux du Van}
\label{Creuxduvan2014}

The Creux du Van is called the Grand Canyon of Switzerland (obviously not as large as the Grand Canyon at all). Located in the Jura Mountains, the walls of the crescent shaped mountain fall straight down on the inner semi-circle by about 200 m. Still very impreszsive isn't it. My first hike in a while without my brother, who moved away to the Chicago area, a full ocean away from me.\\

Co-hikers:\\
Manuel: US-American, physics grad student at UC Riverside. After hearing a lot about trips and travels with me, Manuel thinks it is time to experience this adventure himself.\\

Bing: Chinese, physics grad student at Ohio State University. Bing loves hiking too (typically longer and more difficult hikes than I do though). He enjoys a good panoramic view, so Creux du Van seemed to be ideal.\\

October 4:\\
We started out by CERN just after lunch time, and we had to suffer getting up to the village of Noirague. Manuel's car was an automatic, but pretty old, thus the submission never worked higher than 4th gear, but it wouldn't just stuck to 4th, it tried and continuously failed to switch to 5th gear shortly above 90 km/h. Thus our average cruising speed was about 90 km/h. Anyways arriving a bit more than 2 h later, we started to hike along cow filled meadows, reaching soon the forrest and slowly climbing up the mountain, until we reached the cliffs of the Creux du Van. It was a beautiful sunny early October day, so pretty nice to hike along the cliffs. The way down was a bit more scenic, also since the trees started already slightly to change colour. Compared to the previous times we hiked pretty relaxed, since we knew we could take our time. The hike is assigned a time of about 3 h, so pretty nice and chill. The view of rocky cliffs is always an amazing sight, indeed some of the most impressive in Switzerland\\

Creux du Van***

\section{October 23--October 27: Istanbul}
\label{Istanbul2014}

Co-traveller:\\
Chris M.: US-American from Kansas. Chris hasn't been travelling a lot with me so far (besides one hike to Grindelwald). Considering my claim to fame to do a lot during one trip, he was already a bit on the edge if our way to travel would be compatible. The day programme that is, during the evenings most of our interests line up pretty well. So we decided in case we realise that it gets too much, each of us can do part of the day alone and meet up at pre-defined places after, something which I did for most other trips with people from then on as well. Chris has never been to Turkey or even to Asia, so we plan to have two firsts here. As far as I remember it was also his first time in a Muslim country.\\

October 23 (Thursday):\\
Turkish Airlines provided excellent service (great dinner, nice on-board entertainment). But 20 mins in the person in front of Chris decided to incline their seat to the max. Thus Chris requested if he could have another seat (which was possible). Once we got to the airport and got through immigration (US-Americans need online e-visas, Germans just need to show IDs), we took a taxi to our hotel. This was located in the middle of old town with very nice cafes, restaurants, bars etc around. But it was also not renovated for quite some time. I didn't bother it that much, beds were large and comfy enough, and I didn't plan to spend ages in the shower anyway.\\

October 24 (Friday):\\
We started out having breakfast and coffee by a cafe by the Domabahce Mosque sitting next to the sea. The pier for ferries crossing over to the Asian side was just a couple of metres away. At one point Chris shouted watch out, and I jumped up quickly enough to avoid getting hit by a wave created by one of the large freight ships crossing nearby. Else it would have been a half wet pants day. After that adventure we started the day by Dolmabahce Palace. A giant palace used by the Sultans as new home, after they decided that Topkapi Palace was too old-fashioned as main residence. We decided to get the full ticket, which gave us access to the private Harem quarters and the gardens as well. The state rooms were clearly set up to impress visitors, the staircase alone is decorated with countless crystals. We were also told how proud the Sultan was to be able to show his guest the exotic styles of Europe (kind of just the opposite of what Chinese Saloons or Turkish Salons do in European palaces, whatever is considered exotic is the thing to get). The huge memorial hall is dominated by a giant chandelier, the largest one of its time to be installed, and we were told that the electric light is still the original one installed. The Blue and Pink Halls impressed me most in the Harem quarters. In those quarters one can also visit the office of Kemal Atat\"urk, the founder of Turkey. There is the little downside that photography was not allowed in the whole palace area, including the garden pavilions. I would advice you to visit at least the Crystal Pavilion. Then we planned to visit the Yildiz Palace, walking and enjoying the peaceful Yildiz Park. Once we reached the palace we had to find out that it had been closed for just a couple of day due to a planned renovation (unfortunately none of my online sources had that announced). Anyways we walked down to another small park, where one can find the two Ihlamur Pavilions. Also a perfect stop to have another snack at one of the cafes in the park. And we were shown the inside of both pavilions (should have been a guided tour, but trust me mine and Chris's Turkish is close to non-existent). And then back to see the interior of the Dolmabahce Mosque and back by tram to the other side of the Golden Horn. Clearly I thought we might need more time for what we had seen up to now, but since we had time I decided we should see the Hagia Sophia. Once the largest church of Christianity after Istanbul was conquered by the Ottomans, the church was converted into a Mosque, and all mosaics had been covered. Later transformed into a museum the covers had been largely removed to bring out the mosaics again. Protected by the layer of paintings nowadays they all appear in the glory again. Breathtaking and worth the wait for getting tickets. The main dome is still under renovation (a bit less though than back in 2001). Then we went to the Great Bazaar stopping by Nuruosmaniye Mosque in between. Then we strolled through the many roads and shops of the Great Bazaar, where Chris got his parents some souvenirs. Ending up by the other side of the complex we went to the New Mosque, another mosque built in the typical Ottoman style. And then we had a large dinner in the street in front of our hotel, and later a few drinks in one of the plenty or bars.\\

Dolmabahce Palace*** (Private Quarters***), Ihlamur Pavillons**, Domabahce Mosque*, Hagia Sophia***, Nuruosmaniye Mosque*, Great Bazaar*, New Mosque**\\

October 25 (Saturday):\\
We had chosen a hotel very close to the old town, thus we walked over to the Topkapi Palace, where we didn't have to queue that much being there just before opening time. Unlike last time I decided that we should see the Harem quarters as well. Unlike for European Palaces the rooms are scattered in a vast area around many courts and gardens. We started in the treasury building, the most important piece are old artefacts of Mohammed, including his traditional dress and cape. They also set up a portrait gallery of the sultans, for those whose picture they didn't have they made up portraits (reminded me of what the Venetians did for the Palazzo Ducale, or the Papal Portrait Freeze of St Paul's in Rome). \\

Topkapi Palace*** (Harem***), Hagia Irene*, T\"urbe of Sultans**, Sultan Ahmed Mosque**, Aynalikavak Pavillon**, Theodosian Walls*, Chora Church***\\

October 26 (Sunday):\\
Atik Valide Kui\"olliyesi Mosque*, Beylerbeyi Palace***, K\"uc\"uksu Kiosk**, Yoros Castle*, Rumeli Hisrai***, Anadolu Hisari**\\

October 27 (Monday):\\

\section{November 2: Saas Fee}
\label{SaasFee2014}

Once again Andrew was in town and had the desire to go to the mountains. Since I had been in Grindelwald, Zermatt, and Chamonix pretty often by that point, I thought it could be nice to go to Saas Fee again.\\

Co-traveller:\\
Andrew: Californian, living in Los Angeles, just around for a bit more than a week, thus using the one and only weekend in Europe in a while to see the mountains.\\

While the temperature was pretty cold, there had been almost no snow yet. Thus we decided that we could still go to Saas-Fee and hike quite a bit up. We started very early in the morning and arrived in Saas Fee shortly after sunrise. We started to hike up the ski piste to L\"angfl\"uh. The moraine of the glacier is very impressive, even with a little lake. The Feeglacier starts at the Allalin mountain and splits at L\"angfl\"uh into two parts. We walked up the ridge up to Spielboden, where we stopped for a short snack. Although the temperature was in single digits, the hiking heated up so much that we walked up in t-shirts after a while. Since my last visit in 2010 the glacier had been receded quite substantially, the glacier tongue had detached and a few hundred metres were just dead ice nowadays. The day was very clear in sunny, thus the views of the surrounding scenery and the mountains was very nice. Considering that it starts to get dark already by 4 pm we stopped about 30 metres altitude short of our original planned final stop (we missed basically one corner) and decided to turn around and get to the valley again. There we had coffee and a regional cake before getting back to Geneva, where we had some R\"osti.\\

Feegletscher***\\

\section{December 12-14: London}
\label{London2014}

Co-traveller:\\
Manuel: US-American, first-generation immigrant with roots in Mexico. Manuel wanted to see London, and I love to go to London anyway (in fact I did just a few months before). Thus I still had even valid tickets for some of the places we wanted to see, so didn't take much convincing to see those practically for free once again. At that point our common friend Hossein was in town as well (with another friend of his), thus the ideal weekend to have a bit more fun in London. Also it is one of the weekends where part of the group is on its way back home so not too much to do at home in Geneva.\\

December 12 (Friday):\\
There are multiple flights leaving Geneva for London. On this day we decided to get on a plane to London Gatwick. Unfortunately on this very same day the UK flight control had a major incident where their computing system went down for a while. All flights from and to the UK had a major delay, we talking about delays of over an h here. We still made it out of Geneva, but arriving in Gatwick very shortly before midnight, thus with the transfer over to Paddington it was shortly before 1 am when we got to the (not so much party for tonight) Kings Cross Keystone hostel.\\

December 13 (Saturday):\\
Getting up really early we arrived by Westminster Abbey just as it opened. Clearly less crowded than in the middle of the day in Summer I enjoyed once again seeing all the Royal tombs, or the fantastic chapels, the tomb of Isaac Newton, the choir, transept, just everything. Then our next point was the gigantic venue which is the Royal Albert Hall (even a first time for me). We had an interesting tour (after having a coffee in the cafe of the venue), through the Foyers and the main hall, with a sneak peak on the Queen's seating area. As the Albert Memorial is so close to the Royal Albert Hall, we just had a short detour to take some snap shots. As most museums in London, the Museum of Natural History can be visited for free, once more I was fascinated by the vast collection of dinosaur skeletons (and the gigantic skeleton of whales and sharks as well).  And then it was time to take the British Rail out to the vast complex that is Hampton Court Palace, where I still had the yearly dual ticket for - so yeah for free palace visits. Clearly Hampton Court is one of the most amazing palaces in London, both the huge Gothic Great Hall from the times of Henry VIII., as well as the baroque State Apartments built by Christopher Wren for William III and Mary II. Unlike in summer the eldest part of the palace, Cardinal Wolsey's Cabinet had be reopened. By the time we left the light festival had started. The facades had been illuminated in multiple colours, and we were very eager about what was to come. But then we were told the yearly ticket wasn't valid for this type of event. Thus we rather went back to London, where we went to the Houses of Parliament for some night photos, where I tripped and fell over just in front of the London Eye (without any further consequences). Then we took some more photos of the Tower of London and the Tower Bridge, before we met up with Hossein and his friend for dinner, and early ending pub visit (seems those are rarely open even until midnight).\\

Westminster Abbey***, Museum of Natural History**, Royal Albert Hall*, Albert Memorial*, Hampton Court Palace***\\

December 14 (Sunday):\\
On this Sunday we started the day in Windsor Castle, for which I had a yearly ticket as well. In winter not only the State Apartments are open, but as well the semi-private State Rooms. This includes private dining rooms, and the green and crimson drawing rooms. I am still amazed how the queen's great reception room had been remodelled after the destruction by the big fire in the early 90s. After our visit to Windsor Manuel and I visited the Wallace House with its gigantic private collection of paintings, silver ware, glasses, pottery and porcelain as well as baroque style State Rooms and galleries. Since we still had a bit of time left we spent a bit of time at the British Museum, particularly the mexican, egyptian and assyrian artefacts. At Gatwick we had Mexican food, but they messed up our order and first saved us a different order. Thus both of us got 1 1/2 dinners out after the mistake had been realised and corrected for. At least the flight out was only delayed by about a quarter.\\\

Windsor Castle***, Wallace House**, British Museum***
\chapter{Year 2015}
\label{2015}

\section{January 11: Chillon}
\label{2015:Chillon}

And off to the first trip of the year: Chillon. Time to get out once again this time with a new co-traveller: Siyi\\

let me introduce you first to Siyi:\\
Chinese, grown up in Kunshan close to Suzhou and Shanghai. Moved to Munich after getting her PhD in Los Angeles. Her first trip around the lake Geneva region (and disappointed about the missing authenticity of so-called Chinese food in Geneva. Knows what she wants on her trips - and being very charming she has a talent to talk you into wanting it as well.\\

Chateau Chillon is a castle built on an islet just a few metres of the shore of Lake Geneva a few km away from Montreux. The Bernese representatives resided in the castle up to the 18th century. Afterwards the castle has been used as hospital and prison. Nowadays the Chateau is one of the most visited historical buildings in Switzerland. After our visit in Chillon we walked over to Montreux and had lunch there after a short walk to the Freddy Mercury statue by the harbour. Then we want on a quest for the closest Boba tea shop (and made it in the end).\\

Chateau Chillon*****

\section{March 14-March 15: CERN Physicist Field trip to Milan and Monza}
\label{2015:Milan}

This time on board: A whole bunch of physicists, all first timers on travel with me. How did we all end up going on that trip? I actually don't remember anymore. I remember though that after a year long renovation, the Villa Reale of Monza opened its gate to visitors again, so I was ready for some Milan again. For everybody else,  I guess it was a combination between - we need to go somewhere and me telling everybody how awesome Milan can be. Getting to Milan is only a 4 h train ride, which we can easily to just after work, or really early in the morning. This was the first time I stayed overnight in Milan, in the IBIS hotel by Milano Centrale.\\

Co-travellers:\\
Kennedy, Devin, Aaron, Nate, Laser, Christine (all US-Americans), and Can (Chinese), all physics grad students working at LHC experiments for US institutions (both at CMS and ATLAS). For all of them this was the first trip with me. So far this has been my only trip with most of them, let's see when I have time to change that...\\

April 14: Milano \& Monza\\
Aaron was on the 5:42 train to Milan with myself, while everybody else had left on the last train of the day Friday evening. We met up by the Duomo, one of the largest churches in Italy and Europe, largely built in gothic style. The others had a walk by Castelo Sforzesco previously. Those were the times, when it was easy to enter the cathedral without needing to pay for it, although without the long security checks by the gates (these take sometimes about half a minute per visitor, so incredibly slow). The cathedral is impressive, particularly the little statues and carving on the columns. The choir windows are very precious as well. After our visit of the cathedral we took the train out to Monza. There we had pizza, where everybody else urged me to take the ``German'' W\"urstel pizza (sausages on pizza), which I had rarely seen in Germany though. Then we made our way to the Villa Reale, a former royal palace with vast gardens just outside of Milan. We had booked a guided tour of the state apartments. Most of these were in Empire style, the tour also included the bedroom, where king Umberto I died after being stabbed. Then we walked through the arts exhibitions in the former private quarters of the king. After spending a bit of a chill time in the gardens, we took the train back to Milan, where Devin and Kennedy had a dinner date, while we others had dinner at another place. We had all booked tickets for a Van Gogh exhibition in the Royal Palace. Unfortunately I had to realise that "meeting at the Palazzo Reale" was not really specific. Since I had booked all the tickets, I got them after handing in the voucher, but Devin and Kennedy were nowhere to be seen by the museum entrance up the stairs. They waited outside on the courtyard instead, after 15 minutes or so, they decided that maybe we messed up and they should just go inside, where I waited with the tickets. Clearly they were not pleased (understandably) about us being unspecific of the place where we were supposed to meet. Anyways we all made it to the exhibit in time, and could admire Van Gogh's evolution, looking at his self portraits, hay stakes, and the potato eaters. And we also got some night photos out of the Duomo by night.\\

Milano: Duomo*****, Van Gogh Exhibition****\\
Monza: Villa Reale****\\

April 15: Milano:\\
After we had seen the Villa Reale of Monza the day before and the nice part of the Palazzo Reale, we started the morning with the state apartments of the Palazzo Reale, which housed another free exhibition. Besides enclosing the roof again, the former ballroom is left like it was after the WWII air raids in 1940s. It is still a memory of how destructive wars can be, although what is left is still very impressive. Afterwards I brought people to San Bernardino all Ossa, which is now the resting place of bones, skulls and skeletal remains from a nearby given up cemetery. Then we visited the modern art gallery in the Villa Belgiojoso Bonaparte, the branch of the Italian Gallery in Palazzo Beltrami \& Palazzo Amguissola. The house museum of Palazzo Bagatti Valsecchi is a relict of its time, when it was fashionable to recreate what people thought was amazing about medieval art, so mock up banqueting halls, chapels in neogothic style, armories. Once we arrived by our next stop of Villa Necchi Campiglio, Christine realised she still had the audio guide of the previous Palazzo on her, thus some of us rushed back as quickly as we could, and just made it back in time for the next tour to start. Villa Necchi Campiglio is a more modern place started out in an Art Nouveau core. Some members of the royal family had lived there at some point too, the pool and the garden were nice too.\\

Palazzo Reale*****, San Bernardino all Ossa*****, Villa Belgiojoso Bonaparte****, Palazzo Beltrami \& Palazzo Amguissola (Galleria d'Italia) ****, Palazzo Bagatti Valsecchi***, Villa Necchi Campiglio****

\section{March 27--April 6: Spain -- Madrid and Andalusia}
\label{2015:Spain}

How this trip came into existence:\\
Easter is always a nice time to travel, 2 extra days without taking any additional days off is always a plus. Originally I wanted to see Madrid and the surrounding region once again.  This time maybe seeing the parliament, museums or palaces I hadn't been to previously etc. Asking around for potentially interested co-travellers Madrid alone got a mild response. People seemed to be more pumped about Andalusia. After consulting Renfe, seemed it would be easily feasible to cover both, after all just around 2-3 h in a train isn't too long. But if I wanted to see everything I personally deemed exciting, in total I would cover 10 days (just needing to take 4 days off) -- jackpot for me, but too long for anybody else. So time for a new adventure: Exchange of travel buddies on the way.\\

Co-travellers: \\
Manuel:\\
US-American, but born and raised for some years in Mexico. Thus fluent in Spanish (always helpful when going to Spain). Up to speed with my walking speed, doesn't complain if things on the plan should fold. Hiking speed is faster than mine, so lacking behind quite often there (Sorry). Loved his Spanish Telenovelas, which I didn't understand at all, thus I didn't bother. \\

Eric:\\
Japanese-American, but born and raised in California, so you can guess it, also Spanish speaking. Almost as crazy as myself when travelling alone, thus ideal to go on trips of my caliber. Always in a good mood, and up for adding even more things to do, should the time allow. Others complained about possible snoring at night, but since I am a catastrophe in that respect myself, seems I never found out about that certain aspect. Tends to listen to music when falling asleep. Since I automatically try to guess songs, even with turned down volume, sometimes just by the baseline, that can give me troubles to start falling asleep. \\

March 27\\
Since the house of deputies opens up really early, there was no time to waste, so getting on an Iberia flight on a Friday night. In order to be quick getting the luggage next evening for the bullet train to Cordoba, I booked a hotel pretty close to Atocha station.\\

March 28\\
Palacio de las Cortes - the house of deputies in Spain. The only non-Spanish speaker around in a Spanish only tour -- the only thing I was told in English was ``No Photo''. Having had no Spanish at school the information I understood was limited, but I got some basics out nonetheless. The parliament itself was nice, but (the free tour) was done pretty quickly nonetheless. By the time I got out my travel buddy Manuel made its way from the airport to the parliament. After a short Spanish tour of the Baroque Palacio de Linares (also called Casa de America and ``No Photo'' once again) and a short stop at the Palacio de Cibeles, which is now transformed from main post office -- bullt in modernist style -- to the main government building of the city, we ended the touristy program at the Prado. Most impressive here were Goya's paintings, for the other paintings, after being around for about two hours it got repetitive at some point for my taste -- and wherever you go to in Madrid -- No Photos. After having dinner and wine by the palm garden of Atocha we jumped on the train to Cordoba.\\

Madrid:
Basilica de Jesu de Medinaceli***
Palacio de las Cortes****,
Palacio de Linares***,
Palacio de Cibeles****,
Prado*****\\

March 29 Welcome to Andalusia\\
And our first city in Analusia: Sevilla. I planned we could have a relaxed breakfast at the  Plaza de la Encarnacion. Seems though we were even too early for the cafes? openings. Once the first one finally opened we had just short of 30 mins to spend since we wanted to be at the Palacio de Lebrija just when it opened. This palace is only open on some days of the week, particularly the more private quarters on the upper floor. For this one photos are not allowed and we even managed to get an English tour. The ground floor was in fact more of a highlight, since almost all rooms and halls are decorated with actual roman mosaics ? really amazing. I expected to see a bit more on our next stop, the Archivo General de Indias, but hardly anything was on display from the archive itself. But they had other interesting pieces on display in the library, e.g. early editions of Newton?s write-ups. So on to the Alcazar, which is still used by the Spanish kind as royal residence. The state rooms are on the ground floor, some rooms going back to Moorish times. The palace itself is surrounded by quite extensive gardens. And then we took another tour (no Photos) of the upper floor, which is the floor where the royal family resides when in Sevilla. We even were allowed to have a sneak peak into the living room of the king ? what a gigantic flat screen. Since it was Palm Sunday and thus the begin of easter week the whole city was populated with people wearing Capirotes, the cathedral was off limits for non residents as well. The final stop, after a short D-tour to the Plaza de Espana, was the Casa de Pilatos. A bit off the hustle and bustle of the cathedral square a private residence built in Mujader style (this time with a Spanish tour of the top floor ? and the usual no photos). Seems in Sevilla ground floor halls were always meant for representation and thus to show-off and photos are allowed. The upper floors are the private quarters which nowadays means no photos. And then we participated on the several processions, where each church sends out its preferred statue to a certain square where a religious service is being held then.\\

Sevilla:
Piazza de la Encarnacion****,
Palacio de Lebrija****,
Archivo General de Indias***,
Alcazar***** (Second Floor*****),
Plaza de Espana****,
Casa de Pilatos****.\\

March 30\\
Granada: the former capital of the Nasrid Empire
Now there is no train going to Granada from Cordoba, and the bus takes over 4h, in order to get most out of the day, we thus got on the bus shortly after 5 (together with a couple of other pour souls). The bus was pretty decent and at least I could fold myself enough to get some more sleep. And finally I got to see the cathedral of Granada. Back in 2001 we stood in front of it and were told there was no time (aka we rather had to go for a coffee for 45 minutes, since our tour guide in 2001 couldn't be bothered). The cathedral of Granada is built after Then another little gem was the Madraza of Yusuf I, one of the oldest schools of Granada -- and all of that just for 1 EUR. Then we also paid a visit to the Royal Chapel, which contains the tombs of the catholic kings (you can imagine -- no photos). We had prebooked our tickets for the Nasrid palace of the Alhambra. You cannot print tickets directly, but instead you get a confirmation code, which need to be exchanged for real tickets, but two hours before you actually want to enter the palace, else they just expire (yeah really bad service, but it is even more unlikely to get tickets on sale for the day, and secondary sales go south of 50 \$). Now the tourist information should allow to print tickets, but those machine were out of service. We were told to look for another place a few roads down, and just 5 minutes before the two hour limit. Now a relaxed lunch and then into the Nasrid palace. The royal residence of the Moorish kings, absolutely breathtaking and gorgeous. After we made it to the garden palace of Generalife we ended the day at the private residence and museum of Carmen Rodriguez-Acosta, surrounded by a very extensive garden. And once again due to processions no bus was running back to the main bus station, thus once again a 3 km walk back. And since we were still to early we again got beers, and it took really a lot to convince them we really can handle 0.5 l. And back in Cordoba we witnessed another nightly easter procession, this time with music and torches.\\

Granada: Cathedral*****,
Capilla Real***,
Madraza*****,
Alhambra and Generalife*****,
Carmen Rodriguez-Acosta****.\\

March 31 What the hell - the cathedral is closed today?\\
After Manuel decided to watch Spanish soap operas until late night, we started the day a bit later than planned (aka not at 7 am but 9 am). And here we go -- the Medina Azahara, a moorish ruined palace, once upon a time the capital of Andalusia, like you would imagine it to appear in the stories of 1001 nights. Built around 940 the palace city fell into despair less than 100 years later. Excavation started at the begin of the 20th century, by now around 10 \% of the total area has been excavated and restored. This part includes though the palace of the caliph. By then we were really pumped for the grandiose finale the Mezquita -- which turned into a total disaster. There is one day of the year, when the Mezquita is open to the general public for a whopping 50 minutes, and that day was...today (the original reason why the plan said get up really early).  So there are people guarding all gates to make sure everybody sees the curtains of the door, but not more, indeed you can't even get a sneak peak. So no cathedral for Manuel, absolutely pissed, even pigeons cannot help. I already think how I could combine maybe Nerja and Malaga in a trip later that year, first missing out on the cathedral of Sevilla and then on the cathedral of Cordoba too -- tough luck. After we missed out on the cathedral of Sevilla (really nice to have, but not THE essential building which captures Granada), now messing up the second grandiose cathedral as well -- failed planning I'd say. Instead we made it to the Alcazar (splendid gardens) and the Arabian baths (light gets through star-shaped holes in the ceilings), and after a short stop by a roman temple, we ended up in -- not that grand finale as hoped -- but still cute Palacio de Viana with zillions of courtyards (and sumptuous rooms, but no photos allowed as so often in Spain) ... and all the way back to Madrid (after having the superlarge beers at the Cordoba station -- aka 0.5 l). Still pissed about myself and my failed planning.\\

Cordoba:
Medina Azahara*****,
Arabian Baths****,
Alcazar ****,
Palacio de Viana***,
Roman Temple**. \\

April 1 
Segovia: Done too early - why not adding one more royal summer palace?\\
After spending the past days in 30-35 C warm Andalusia we were back in the centre of Spain. The claim was that it should get close to 23 degrees during lunch time. Thus I decided staying on shorts would be fine. I was so wrong -- missing the bus at Segovia by 10 seconds we had to wait another half an hour (aka breakfast at the train station) -- once we got into downtown standing in front of the Aqueduct I realised immediately below 15 degrees is so cold. The Aqueduct is one of the best conserved roman construction, most probably built already in the 1st century. Impressively it was still in service until mid 19th century. We got into the cathedral so early, that we were in before they started charging the entrance fee, but once again I was impressed by the cathedral (by now photos were allowed). The cathedral is a gothic cathedral, with most of the interior being upgraded in baroque times. I had planned to spend most of the day in Segovia, but then after the Alcazar we were done shortly after 11.  Now we finished a few hours earlier than planned, and I wanted to do the royal palace of La Granja anyway (a few days later on April 4). Thus we decided just to go ahead and do the palace, which also would free almost the full day of April 4 -- WHICH could free up space for either Cordoba or Sevilla. Well anyways off to La Granja. The main halls of the palace (no photos) were cute, but the typical baroque, red, blue, Chinese rooms and a lot of gallery type rooms. I liked the garden rooms of the ground floor a lot more, with statues, frescoes, columns and some with their own fountains - and the gardens. The fountains (all decorated, some with dragons, roman goddesses etc) are only running in certain intervals, which we unfortunately missed, thus we just saw the statues without the water displays, but still impressive. Once back in the hotel I figured out, that indeed I can see cathedrals in Cordoba and Sevilla in one day, leaving early enough in Madrid -- SOLUTION!!, and the holidays are totally saved. \\

Segovia: Aqueduct*****.
Cathedral*****.
Alcazar*****
Town Walls and Gates****\\
La Granja: 
Palacio Real*****\\

April 2 
Madrid \& Toledo: Don't forget your stuff\\
We all know that Spain colonised most of Central and Southern America over the course of the 15th and 16th century. And they brought a lot of artefacts over the Atlantic. All of that is presented in the Museum of the Americas, which sits a bit on the outside of Madrid's old town even beyond Moncloa. I learned quite a bit about the former societies and the history of all different cultures, the contest was also described without glorifying what had actually happened. After our failure to be prepared for the temperature changes the day before, Manuel decided to bring on a sweater and throw it into a backpack in Toledo. As you can imagine -- no photos -- inside the museum and backpack has to go into lockers. After the museum we walked over to the Eremita de San Antonio de la Florida -- the chapel which Goya is buried in. The chapel itself is also full of his paintings, they even removed most of the religious services to a neighbouring chapel, since too many tourists frequent the Eremita (btw no photos). There are actually no guards standing around in the chapel - but don't let this deceive you. Big brother is watching you from the ticket counter, the amounts we heard ``no photos'' over loud speakers was quite astounding. This time I appreciated the chapel more than the first time I had seen it. Maybe it helps if you don't have to walk there for 40 minutes in almost 40 degrees.\\
After lunch -- we had 2 1/2 h to spend - I heard an ``oops ehhhm'' from Manuel. ``The backpack is still in the museum" - which was 45 minutes away by foot. And we still have to get to Atocha to catch our train to Toledo. Internet claimed there was hardly any chance we could make it. We still wanted to give it a shot. Thus up to Moncloa - I told Manuel to run to the museum alone - I would be not up to his speed, and I rather should get the metro tickets for both of us. Said and done, time is ticking while we are on the Metro- finally getting of at Atocha - 5 minutes to go. We run to the high speed train lines -- train is not listed on the display, but shouldn't have left -- I remember Atocha's high speed train terminal has two levels, and we were standing at the wrong one! Tick Tock, Down one floor, rushing through security (yes in Spain for high speed trains you have to go through security like on airports). Clearly in a rush we just yell Toledo -- and the guard yells back the platform number, keeping on running, tickets are checked while hear already the train conductor's whistle, running down the stairs and jumping into the first door in sight, which closes seconds later behind us. Against all chances WE DID IT.\\
Arriving in Toledo we decided to prolong our stay there and exchange tickets for the last train back to Madrid. Manuel also managed to get my new tickets for Cordoba and Sevilla printed. It needed quite some time to convince the man on the counter though, that this is indeed not a mistake, and the tickets are bought for the same day. Our first stop in Toledo was the cathedral. Unlike a few years before, this time photography was allowed. Toledo is another fine gothic cathedral in Spain, once again redecorated in Baroque style. Unlike for the other cathedral of the trip it was though not possible to get close to the fence of the main altar. But that is not the highlight of the church anyway (and nope neither is the sacristy, although this one was painted by El Greco too). The absolute highlight is the astonishing altar of ``El Transparente''. One - if not the most beautiful baroque - altar in Spain. Covering the full height of the main nave, it sits at the back end of the choir, facing the choir chapels. Then we visited the synagogue, which reminded me personally though more of a mosque similar to the Mezquita or how you might find it in Morocco. After a short step in the Jesuit church, climbing the tower to have a nice view of the old town), we ended up in the mosque turned church of Mezquita Del Cristo De La Lu, cute and also close to one of the city gates. Then we finished off our day with dinner and a short sneak peak into the Monasterio San Juan de los Reyes, and all the way down crossing the old bridge over the Tajo by moonlight...and back to Madrid.

Madrid
Museum of the Americas****,
Ermita de San Antonio de la Florida****.\\
Toledo:
Cathedral*****,
Synagogue ***,
Jesuit Church***,
Mezquita Del Cristo De La Luz****,
Monasterio San Juan de los Reyes***\\

April 3
Starting the day early our first Cercanias was late by 15 minutes. Eager to still make the connection at Chamartin falling out of the first train and nope still missed the connection (nothing was hurt). So another wait for roughly 45 minutes (unfortunately train station breakfast instead of a Cafe in El Escorial) and off to the gigantic complex of the Monasterio San Lorenzo de El Escorial. The complex contains the former Habsburg royal palace, as well as the more modern Bourbon royal palace with a large library, a monastery and the royal Pantheon grouped around a gigantic basilica as heart piece. The first time I visited we missed out on the Bourbon apartments, since they were all booked out for that day. Since tickets for that part of the complex cannot be bought online, we got into the queue and were informed that the first english tour would take place only 4 h later, so Spanish it is (and no photos on the whole complex). The apartments are actually quite nice, but unlike everybody around me I just understood hardly anything. Since the palace is still used for some ceremonies, an armed soldier was assigned to supervise each group. Then the visit continued on the Habsburg side, quite a contrast since the goal of those quarters was to show modesty, The opposite of modesty are the library and the church. After the large complex we checked out the private little palace of the Case del Principe (also no photos) with its little garden. Then we planned to have lunch, but after not being served for over 40 minutes (we tried everything), while other newcomers were welcome and served, we left for another sandwich place to have at least something to eat. And back to Madrid for another highlight -- the Museo Reina Sofia with its modern art collection, including Picasso's masterpiece, depicting one of the darkest chapters of the Spanish Civil war -- Guernica (photos are allowed in the museum, but not anywhere close to Guernica).\\

El Escorial:
Monasterio San Lorenzo de El Escorial***** (Bourbon Apartments****, Pantheon**, Basilica \& Library*****, Habsburg Apartments****),
Casa del Principe****\\
Madrid:Museo Reina Sofia*****\\


April 4, 2015 Exchange Day\\
After all that exhausting time travelling to Andalusia and back to Madrid Manuel calls it quits today and will be replaced in the evening by Eric. Thank God I made up for half a day by doing the Palacio de la Granja already a couple of days ago. Since the gives space to Cordoba and Sevilla once again. After saying goodbye to Manuel caught the first train down to Cordoba. And FINALLY made it into the Mezquita once again. Renovated quite a bit since the last time I saw it in 2001. Compared to back then all ropes and nets. were gone. And particularly the cathedral part was very shiny whtie and cleaned. One of the most breathtaking places I have ever been to, even the cathedral which just was put in the middle of the old mosque doesn't destroy the impression of the forrest of columns, and the Mihrab is still conserved. Unfortunately I gave myself hardly more than 60 minutes to see it, and off I was to Sevilla. Sevilla cathedral -- finally -- the largest gothic cathedral of the world containing also the largest Retablo of the world, which happens to be one of the finest as well. Two well-known pieces of the cathedral are the silver altar, which was partially destroyed in a war to pay for its expensive and also the tomb of Columbus. Since Easter threw the plan totally off, in this tight schedule there was no time to make it up the Giralda this time around, so all the way back to Madrid (this time not having a beer alone in the train station though), and finally meeting Eric. And we have that delicious cheese once again (and a bit of Puerta de Sol and Plaza Mayor but oh well).\\

Cordoba:
Mezquita-Cathedral***** the absolute highlight of the trip\\
Sevilla:
Cathedral*****,\\
Madrid:
Plaza Mayor ****\\

April 5 Summer Palace and Spanish understanding of donations
And with Cercanias out to the summer palaces of the Bourbon kings in Aranjuez. We spend about an hour in the garden and then off to the palace (no photos). Certain rooms can only be visited with a guided tour, really little beautiful amazing rooms. Around a certain time the Nasrid quarters of the Alhambra received attention, and the royal family tried to replicate the original colours and decoration in their own state halls. Another beautiful room is the Porcelain room (similar rooms can actually also be found in Southern Italy where the Bourbons had their ``secondary empire"). In the park is the more private Casa del Labrador, which I would recommend people to check out, it's worth it. Then back to Madrid and onwards to the Royal Palace of El Pardo, which was used by the Spanish fascist dictator Franco as his private residence. The remodelling included for example the construction of a cinema in the previous theatre. After a short stroll through the Plaza Espana and the Egyptian temple of Debod (a donation from Egypt for the Spanish help in relocating the Abu Simbel rock temple complex). Then we wanted to visit the crypt of the Almudena - Madrid's cathedral. They asked for a donation of 1 EUR. Typically I give a donation, but after I visited the place. But the elderly Senorita wasn't having that, vehemently demanding at least 2 EUR, or else no entrance - she even slapped us, when we didn't want to force donate. In the end we did - but shame on the cathedral for pulling that off. Either charge, which is fine too, or make it a donation, but forced donation are just bad practise (same for all the US museums which work like that).\\

Aranjuez:
Palacio Real***** (with tour),
San Antonio**,
Casa del Labrador*****\\
El Pardo: Palacio****\\
Madrid:
Plaza Espana***, 
Temple of Debod***,
San Teresa \& San Jose**,
Crypt of Almudena***,
Puerta de Sol***\\

April 6:\\
And we started the day again with the Almudena, this time it was even expected to donate money, although a religious service was taking place. After it finished we strolled through the large nave. The construction of the cathedral started by the end of the 19th century and finished only at 1993. The interior is pretty modern, but also reminiscent of gothic style. I like it. Then we continued with the Palacio Real. The royal palace is one of the largest palaces still in regular use up to today (photos only allowed in the main stair case and the court yard). One of the most beautiful palace - for sure in my personal top 5 up to now (up there with Versailles and Buckingham Palace e.g.). Then we continued with Madrid's old cathedral - San Isidro and the basilica os San Miguel. The last item of this Spain trip was the royal theatre
Madrid. Teatro Real has quite a modern flair (remodelled in the 90s), unlike most of Europe's other royal theatres and operas, but still keeping a classy feeling to it. And then it was time to say good-bye to Madrid and back to work.\\

Madrid:
Almudena****,
Palacio Real*****,
San Isidro****,
San Miguel***,
Teatro Real****\\

\section{May 14--May 18: Campania}
\label{2015:Campania}

burglar attempt, train strike

May 15: \\
Pompei: Roman City*****, Santuario of Our Holy Lady****, Vesuvio*****, Torre Annunziata: Villa Oplontis*****, Napoli: Archeological Museum*****\\

May 16:\\
Pozzuoli: Solfarata*****, Portici: Reggia***, Royal Chapel***, Ercolano: Ruins of Herculaneum*****, Castellammare di Stabia: Hadrian's Villa****, Marcus' Villa****, Gesu Nuovo*****\\

May 17:\\
Caserta: Reggia*****, Napoli: Palazzo Capodimonte****, Palazzo Zevallos (Galleria d'Italia)**, Basilica San Francesco di Paola***\\

May 18:\\
Duomo*****, Capelle Sansevero****, Philippo \& Jacobo***, Basilica San Paolo Maggiore**** (crypt**), Gesu Nuovo*****, Teatro San Carlo****, Palazzo Reale*****, Castel Nuovo****, Santa Maria Incoronate***, San Lorenzo Maggiore****

\section{June 13--June 21: Los Angeles}\
\label{LA2015}

Why LA: UCLA was hosting a theory particle physics conference. The organizers wanted to get speakers from the experimental community for overview talks in their plenary session. After CMS didn't have too many candidates shortly before the deadline, my boss asked me if I would consider applying. Since it was hosted by my home institution the transatlantic flight could be well justified. I applied for the talk and was selected in the end. Preparing a talk for over 45 minutes was quite a challenge, particularly since it included measurements from several working groups, which all needed to sign off on the material shown. \\

June 13:\\
I transferred in London Heathrow for the first time in my life, quite an experience. Waiting for over 3 hours for a flight with having only access to the internet for an hour wasn't that much fun either. But then immigration didn't have long queues this time, in fact my colleague Cameron who hosted me on my first night (Graduation weekend, thus all places close to UCLA had been booked out) was still in traffic once I arrived. Cameron suggested in order not to get into jetlag, to get going to a private party somewhere close to Hollywood. It was indeed quite an interesting crowd, from aspiring actors, directors, screen writers, to people from law \& med school, or people like well me. Still a nice evening out, although I didn't stick around much after midnight, having gotten up early in Geneva that day.\\

June 14:\\
Just after getting up I did a couple of test presentations of my talk, keeping a flow with enough details and then also reading up on the results I wanted to present. An overview plenary talk is really an art on its own. Indara had also gotten the word that I was in LA, thus she organised me Uber to meet her and her friend for a proper brunch. Clearly she knows all about good food places too, having done her Undergrad at UCLA too. Afterwards I spent the afternoon with Cameron and his buddy Matt, who both wrote on a script for a science fiction based series. Then Cameron brought me to the hotel for the rest of the trip. My boss had warned me already before, it was OK but not really up to standards I got used to in the US. The room was spacious enough, also with stable internet connection, but the water was tasting like chlorine flavoured. Anyway still more to train on the talk. On that evening I had Chipotle, since I was told to try it, it was fine but nothing I would go on a trip for.\\

June 15:\\
The first day of the conference and I gave the opening plenary talk. I was told by multiple people afterwards that they liked the talk, also what I presented seems to have been considered interesting. Eric talked about a more special topic which we had worked on for almost a year. After both of us had given our talks and after two more sessions we decided to do a bit of tourism in the late afternoon, getting over to Long Beach. There we visited the USS Iowa, one retired battle ship which is still maintained properly. One of the veterans told us about its past glory fighting against the Japanese (short unsure look to Eric) or Germans (look to me). I mean we do know our heritage and what went wrong, so nothing he needs to feel uneasy about talking to us. Anyway after that interesting trip, we did go to see the RMS Queen Mary, a retired Ocean liner converted into a hotel. Still amazing to see how luxurious some people travelled and how poor the conditions were for the lower class passengers. At the end of the day we celebrated Eric's graduation with my boss of that time, Prof David Saltzberg, who also happens to be an science advisor for shows such as the Big Bang Theory. Food was excellent, the sunset over the pacific beach and hills was amazing to see too. And I was introduced to Riju, but you will hear more about him in many more chapters to come.\\

USS Iowa****, RMS Queen Mary****\\

June 16:\\
Today Cameron was supposed to pick me up, and we were supposed to meet Eric at the California Science Center. But then we received a call that his car had broken down and thus he couldn't join. That was clearly not an option so we picked him up. All his family welcomed us, only to silently disappear a couple of minutes later. Seems even a chilled back person like Cameron or a tiny German such as me can be intimidating. Anyways back to the Science Center it was which houses the Space Shuttle Endeavour. Just amazing to see a vehicle which had been out in space leaving the atmosphere of our planet. Just next to the science center is the LA Memorial Coliseum, which was the main stadium of the 1984 Olympic games, as far as I know it is also supposed to be used in the upcoming Olympics of 2028. A random person there told us about his play he was writing which would make it big on Broadway later on. So far I didn't hear of that show though, but a lot of stuff got derailed nowadays. And then we had food, cheese steak to be precise. Driving to Pasadena afterwards we thought we could checkout the Caltech Campus, but then we were told by a guard it was in fact private property. Then Cameron took me to a place to our first Sushi place.

Space Shuttle Endeavour*****, LA Memorial Coliseum***\\

June 17:\\
Today's lunch was at Sushi, and both Eric and Cameron loved it. I obviously enjoyed it too, but without a baseline who am I to judge it. Then we drove over to Hollywood and walked over the Hollywood Walk of Fame. We stopped by the Chinese Theatre, very evolved setup which reminds one of an Americanised version of Chinese art to watch Jurassic World -- cute movie. Then we took a tour of the Dolby Theatre which hosts the annual Academy Awards. We also got into some of the backstage rooms. And then it was time to try the delicious combination of chicken and waffles. Might seem an unusual combination, but it works (don't forget the maple sirup on your waffles.\\

Chinese Theatre***, Dolby Theatre****\\


%Sushi Go 55, Hide

%pasadena: had chicken and Waffles

June 18:\\
On this day we wanted to go to Downtown but then Barack Obama and Hillary Clinton had two back to back fundraisers in the area, thus all main roads had been blocked off, and thus no LA downtown for me. Instead Cameron took me and Eric to a Ramen place and later on we had Chipotle, since I was supposed to find out why some people did extra trips to have Chipotle in Europe, but I didn't get why people were so crazy for it.\\

%ramen Tsujita Annex

%Chipotle

June 19:\\
If you want to find out how rich people built their own private mansions, visit the Getty Villa, which is a ancient roman revival style villa which houses the ancient greek and roman collection of the Gettys. We had our final touristic outing by the Getty Center featuring another nice (but a tad small) modern art exhibit. Another food event later, this time eating at a Mexican place.\\

Getty Villa****, Getty Center****\\

June 20:\\
Public transport does exist in LA, albeit on a very small scale. But I did take the public bus from UCLA to LAX. And then had the window seat enjoying the Mojave Desert, Horseshoe Canyon, and last but not least Canyonlands National Park.\\

Canyonlands National Park overview*****

\section{July 4: La Tour}
\label{LaTour2015}

Glacier du Tour*****

\section{July 11: La Jonction}
\label{Jonction2014}

How this hike came to be: I am always fond of glaciers, and Manuel likes hiking and glaciers. The weekend was announced to be very sunny and very hot, thus Manuel decided to get a rental car, go to the mountains, climb up high and escape the boiling heat. Clearly I always enjoy hiking too, and I love to return to Chamonix for hiking whenever I can. This time we decided to do the longest non alpine hike in the valley: La Jonction. The trail leads from the hamlet of Bossons up a rocky ridge to about 28xx m, where the glacier parted into the Glacier des Bossons and the Glacier du Taconnaz. Unfortunately the connection to the Glacier du Taconnaz broke down already couple of years ago.\\

July 11:\\
We got our car at the airport as early as we could and drove up to the chair lift station. The first part of the hike is only about 50 minutes one way, also not that steep, but if we can give us about 90 min of a head start, we should definitely do it. Compared to the last time I had been here (3 years ago in 2012) the glacier tongue significantly retreated. In fact both glaciers retreated by more than a km from their largest state, which they had during the little ice-age before the industrial revolution came to full force. We got ourselves a cup of coffee before starting the hike, and then hiked up multiple switchbacks until we saw the full length of the Glacier du Taconnaz: still very impressive ice towers, crevasses \& rifts within the ice, and waterfalls of melting water all over. Getting back to the other side of the ridge, the Glacier des Bossons is a very impressive sight, particularly with the Aiguille du Midi in the background. And about 4 h later we finally reached the end of the path, with direct access to the glacier. Both Manuel and I walked a bit on the glacier, in fact just for getting a photo snapshot. The view is absolutely rewarding: a giant stream of ice right in front of you, coming down the 4000 m peaks of the Mont Blanc Massif. And naturally the way down proved to be rather the issue, particularly on the knees. Although we did a short stop in one of the huts along the way to recover a bit, more than 1000 m down a rocky path is just not a walk in the park. Manuel always told me - just a bit, clearly didn't help that I knew it was still more than 1 h down. Anyways we made it down in time with the chair lift still in operation. And back to Geneva, dropping off the car, and then having pasta. This hike still remains up to now the most challenging I have done, not because it is technically difficult per se, but the length and the rapid change in altitude take their toll both on knees and legs. Pro-tip: always check your camera setting by the end of a hike, but more about that later.\\

Glacier du Taconnaz*****, Glacier des Bossons*****

%\begin{figure}[h]
%  \centering
%  \includegraphics[width=0.7\textwidth]{figures/2015/157110120_Chamonix_Glacier_des_Bossons.JPG}
% \caption{View of the glaciers at the top of the La Jonction hike}
% \label{fig:LaJonctionPic}
%\end{figure}

\section{July 16--19: Versailles \& Mont Saint Michel}
\label{2015:ParisI}

How this trip came into existence: I wanted to see Mont Saint Michel, but seems the best way to do it is starting from Paris. Since we have new students in, I thought it would be a good way to show them Paris or Versailles. After finding out that Versailles offers night illumination followed by fireworks on this particular Saturday, I decided to put that up, and find out who wants to join.\\

Co-travellers:\\
Riju: US-American: just came over by begin of July. After meeting Riju for the first time during my trip to LA, I thought it would be a good way to introduce him to Europe. Having been in Cambridge and Paris before, I thought all would be set up properly. \\

July 16:\\
Riju decided to stay at a cheaper place than me, so he travelled on another train than me, spending already a day in Paris. I took the last train out of Geneva arriving at the Hotel Bel-Air just shortly before midnight.\\

July 17: Versailles:\\
Since Riju stayed at another place, consequently we meet up at Versailles. I gave him my phone number and trusted in good faith that things will work out. Indeed once I arrived in Versailles I realised Riju was in front of the line, being there 1 h before opening, instead of my planned 20 mins. Seems his hostel had an incident, where the owners partner got in an argument and trashed the place, so he took most of his stuff in his backpack. Naturally we were also in pretty quickly, but had to stop at the wardrobe to get Riju's stuff locked up. Then we rushed by the audio guide and the Louis XVI rooms in order to be first at the State Apartments. Seeing the Hall of the Mirrors without any other visitor felt like once in a lifetime. After taking photos in the King's State Bedroom I realised that the white balance was off from when I handed the camera to Manuel!! (damn it). Putting the correct white balance settings on, the queen's state rooms and the battle gallery were all fine. We had quite an intensive day with the Trianons and walking through all of the park, and Marie Antoinette's Hamlet (in renovation at that time). By that time we were got close to the time, when the Palace closed, thus I rushed back, while Riju preferred to spend more time in the park, having realised that his shoes were not best suited to run and walk long on cobblestone. Then I did the newly opened Louis XIV rooms (not THAT impressive, so skip them in case you want to see the main rooms empty). Since it was now close to the closing time, the State Apartments were quite free of people again. The Hall of Mirrors was prepared for a Ballet performance afterwards, but I stayed to be the first and last regular visitor of this particular day. Then I met up with Riju again and after a short stop at Versailles' cathedral we had a long dinner. Afterwards we went back to the illumination of the fountains again. With multi-colored spots and some effects like smoke, it was really out of this world. In fact it was the first and (up to now -- 2019) the last time that I have seen the water switched on for the Bath of Apollo. And then we had the 20 minute long fireworks, with flames shooting out along the way as well (accompanied by music). But then all hell break loose. Instead of being clear and free of clouds like predicted it started to thunderstorm just after the fireworks. Thousands of people rushed to the gates, and some people decided to wait out below the Loggia of the Chateau - to no help. I decided to run, having clarified with Riju that he would know how to got back (after all he got to Versailles from a different place, so shouldn't be an issue). It rained so hard, that i had to clean my glasses from water with my fingers, but I made it on the train later switching to a tram. Around 20 minutes later I am on the tram to the hotel, then I receive a message ``whereabouts are you''. Expecting it to be from Riju, I explain him how I rushed out, and also how I think he should get to one of the Versailles train stations and back hom. getting back an ``Interesting''. Later I figured out, that the mobile phone where Riju saved my number ran out of battery, and on his US phone he only had his advisor in, so I effectively communicated with him by the middle man of our professor. Anyways he got my number and I got him home, on the last train getting out of Versailles to Paris on this day, while I got to my hotel safely too.\\

Versailles: Chateau***** (with Trianons*****, and Fireworks*****), Cathedral***\\

July 18: Mont Saint Michel:\\
This trip I did on my own, getting to Dol-de-Bretagne and via two more buses over to the island of Mont Saint Michel. The dam had been recently replaced by a bride, with the goal to make Mont Saint Michel a well separated Island again. It clearly was low tide when I was there, with typically muddy water soaked soil around. The abbey was really nice to see, also listening to the choir performing in the church for Sunday service. After visiting the abbey I had Moules-Frites, the regional dish of mussels and fries. Having enough time on my hands, I decided to walk over the bridge by the Cousesnon Dam, and then taking a bus and the TGV back to Paris.\\

Mont St Michel: Abbey*****\\

July 19:\\
Taking an early train out of Paris for the first time, I expected to meet Riju in the train restaurant. But he didn't appear, and I figured out later, he missed the train by about 10 minutes\\

The Aftermath: Since Riju got lost in Versailles and missed the train out of Paris, our professor decided the that the new grad students are not fit for travel yet, and forbid them to go on my next planned trip to Zermatt and any other trips until further notice. He acknowledged that previously all grad students and even undergrads seemed to have done fine, but maybe this changed. It took up to March 2016 until Riju dared to join me on a trip again, being in the end the co-traveller who joined me on most trip (taking out family to avoid any bias). After all his professor still remembered this incidence during his PhD defence speech 4 years later, so this trip did make an impression on everybody involved.

\section{July 26: Zermatt}
\label{Zermatt2015}

How this trip came into existence: Can you imagine, after being in Switzerland since 2011 (on and off, but still a long time), Eric has never made it to Zermatt. Surprised I decided to take him to ALL of Zermatt, also offering everybody else to join us on the way. As aftermath of my previous events in Paris, the grad students were forbidden to join.

Co-travellers:\\
Eric: his last trip in Switzerland before moving to Germany, now Dr Eric, his first trip to Zermatt though\\

Manuel: hiking is his jam, so Manuel will join us for the hike, though he doesn't want to join as early as 4 am, same for our co-hikers, including friends from Poland, the US, and Stani (Canadian, with roots in Bulgaria).\\

Starting shortly after 4 am, Eric and I make it to the first cable car we could potentially reach to get all up to Kleinmatterhorn, enjoying the panoramic view and the glacier palace. On the way back we meet other US Americans who tell us how amazing they think Zermatt is. By the Schwarzsee station we meet up with our co-hikers. Taking the lead (as slowest hiker) we make it up to the H\"ornlihut, about 30 mins longer than the posted hiking time. Amazing views of the Furgg glacier, as well as the Theodulglacier, all the way over to the highest mountain of Switzerland in the Monte Rosa Massive. Manuel and Eric have first good-bye shots at the foot of the Matterhorn, while we all have a short lunch break before hiking back. Eric and I continue our do-it-all trip catching the train up to Gornergrat, while everybody else gets on the train home to Geneva. Up on Gornergrat clouds start to cover the mountain tops, but the view on the Grenz- and Gornerglaciers is still amazing. Having cokes and snacks in the restaurant at the former observatories up on Gornergrat we enjoy the still warm inside, while it starts to thunderstorm outside. Eric and I get on one of the last trains back to Geneva, arriving there at 1 am. Our last trip in a while, as Eric says good-bye to Geneva, starting a postdoc position at Max-Planck-Institute in Garching, moving in with his girlfriend Siyi.\\

Kleinmatterhorn*****, H\"ornlih\"utte*****, Gornergrat*****

\section{August 21--August 23: Monaco and Nice}
\label{2015:Monaco}

How this trip came into existence:\\
Fantasy football, and no -- not the real one, rather the American version. For whatever reason it had been decided I should be part of a fantasy football league. At the same time the decision had been taken to have the fantasy draft take place in Monaco. Since Monaco had been on my list of places to see for quite a while, obviously I was on board for this trip. Due to time restrictions, which I will get to later, most of our crew had to leave on Sunday morning, thus I planned to spend Sunday in Nice. \\

Co-travellers: \\
Laser:\\
US-American, his boss decided years ago, that it would be ideal to have her group meetings on Sundays, most probably to ensure here students would work on the weekend. There was no way he could allow missing this meeting, thus all others left already Friday morning, and had to leave the first flight on Sunday. Needless to say, he left physics for industry after getting his PhD. Big supporter of the Gators and Green Bay Packers.

Evan:\\
US-American, only got pulled over to the dark side of physics later on, in fact he started out studying law first. Always up for a good time, very effective in talking me into joining. As we had to find out later unfortunately suffered from food allergies.

AJ:\\
US-American, dared to go already on a second trip with me. Likes to make fun of other teams should they lose out to the New England Patriots once again (can be excused if you are from Boston). He knows how to appreciate good quality food and drinks (things I hardly care and know about), and already prepared for his future beyond physics in finance.\\

August 21:\\
Took a very late Friday flight out to Nice. By that time no buses were leaving for Monaco anymore, so trains it is -- or the next train, since the one I wanted to catch was cancelled. After waiting for 70 minutes finally get on my ride to Monaco. The train station is midway up the mountains, and offers really nice (night)views of the harbour. By that point I realised that my lens was full of dried water drops from my previous trips to the Truemmelbach waterfalls. After a thorough cleaning ready to get to our hotel. Hotels in Monaco are notoriously expensive, thus we decided to stay just across the border in France. Per person that was a difference of roughly 50 EUR. The others had already arrived in the morning, checked out the Casino and had dinner at a three star Michelin restaurant. Unfortunately for Evan they served something he reacted allergic too, so a bad ending for him, although he was fine a couple of hours later.\\

August 22: Following the Formula1 race track\\
Starting out the day in the Monte Carlo Casino: during Morning hours the casino can be visited as well, but rather in terms of sightseeing, since those are off business hours and the games are closed. Instead it is allowed to take photos then. Built in the 19th century the rooms are decorated in neo-baroque style. Then I decided that it would be fun to follow along the Formula 1 Grand Prix racetrack over to old town. On the side of the roads you just have to follow the characteristic red-white curbs. Don't expect amazing views or details about the history of the race, from time to time they installed a plate with the name of the corners (and the tunnel is just an ordinary road tunnel). Passing all the fancy yachts and the swimming pool in the harbour we walked up the stairs to the old town for our second touristic stop -- the Prince's palace. Most of the palace is still the private residence of the royal family, in summer the state apartments and the courtyard can be visited (no Photos). The palace has a couple of nice rooms, considering the size of Monaco, you can't expect to have another huge Buckingham Palace style place here. So it is what you expect, a couple of Baroque style living and bed rooms, saloons and as grand final the throne room. We were lucky that the exchange of guards happened to take place just when we got out of the palace. After lunch we stopped by the cathedral. The Cathedral was finished in 1903 and built in a rather reduced style, with one large mosaic in the apse as only sizeable piece of decoration. In the choir you can find the royal tombs, including the tomb of Grace Kelly, the former Princess of Monaco. The former prince of Monaco, Rainier III, was an avid collector of vintage cars and as well formula1 race cars, e.g. a Kimi Raik\"onnen McLaren. Having drinks by the harbour of Fontvieille, the others decided they had enough tourism for the day, while I continued to climb the roads up to the Jardin Exotique. The exotic garden is the botanical garden of Monaco on a cliffside with plans predominantly originating from the Americas, Africa and the Middle East. On the site of the gardens evidence of prehistoric humans have been found (and places in its own museum). I enjoyed the stroll alongside the little paths and the fantastic views of old town and the Cote d'Azur. The most fascinating part was though the tour of the Grotte de l'Observatoire. A nice cave filled with stalagmites, stalactites, draperies and columns, definitely a highlight of the trip. And then the Fantasy football draft: the ``official'' reason for our trip, with a WiFi which proved not to be that stable as though. Being a green horn to American football anyway, my random hit and miss team didn't play out so badly for the season to come, while everybody else's knowledge didn't lead to any smarter drafting necessarily either though. In the evening we all put on our suits and off we went to the Casino, this time for gambling, sipping on cocktails etc. After a visit to another fancy bar we enjoyed the night views of the harbour from one of the roof terraces.\\

Monaco: Monte Carlo Casino*****, Formula 1 race track***, Prince's Palace****, Cathedral*, Vintage Car Collection****, Jardin Exotique****, Grotte de l'Observatoire*****.\\

August 23: Nice and Villas alongside the Cote d'Azur\\
Everybody else left on the earliest flight, so that Laser could attend his group meeting. I decided to spend the day in Nice and places alongside the Cote d'Azur. Plan was to leave the luggage in the train station of Nice (according to the SCNF webpage they had lockers). Little did I know, that they had major construction going on, and the lockers were closed, tourist information didn't know if there were any other lockers in the city closeby. So carrying around my luggage for the rest of the day. Starting out the day in the baroque cathedral (cute but nothing too special), I continued then with the aristocratic Palais Lascaris, filled with loads of old instruments, paintings and tapestries. After a stroll alongside the Promenade des Anglais, a 7 km long road directly by the beach of Nice, I wanted to see the Villa Massena, but since I still had my (tiny) suitcase with me, they wouldn't let me inside. So skipping this place I jumped on a bus to get to Beaulieu-sur-Mer. This village is the home of the Villa Kerylos, a ancient Greek style villa built by a French archeologist just by the sea. The interior decoration is inspired by findings of Greek noble houses on the island of Delos, but with all the comfort of an early 20th century building. Only a short walk along the coast away in the neighbouring village of Saint-Jean-Cap-Ferrat is the Villa Ephrussi de Rothschild. This luxurious home is surrounded by a vast garden, which is split up into nine sub-gardens, each dedicated to a different theme, e.g. a Japanese Garden, a Rose garden, an exotic garden. And just in front of the house a formal French garden with fountains, statues and water games accompanied by late Baroque classical music. Then jumping on the bus back to Nice, where I had a walk through old town passing the Place Garibaldi, the orthodox cathedral and Place Massena with the Fontaine du Soleil.\\

Nice: Cathedral****, Palais Lascaris****, Promenade d'Anglais***, Orthodox Cathedral***, Place Garibaldi***, Place Massena****\\
Beaulieu-sur-Mer: Villa Kerylos****\\
Saint-Jean-Cap-Ferrat: Villa Ephrussi de Rothschild*****

\section{August 28--September 12: St Petersburg}
\label{2015:StPetersburg}
what a conference dinner

August 28: St Petersburg\\
Kasan Cathedral***, Church of the Savior on Spilled Blood*****, Isaac's Cathedral*******\\

August 29: Lomonossov\\
Lomonossov: Oranienbaum***** (Great Menschikov Palace****, Chinese Palace*****, Palace Peter's III***), St Petersburg: Peter \& Paul Cathedral*****

August 30: Pawlovsk \& Gatchina:\\
Pawlovsk Palace*****, Gatchina Palace*****\\

September 2: Moscow:\\
Kasan Cathedral***, Lenin Mausoleum****, Kremlin***** (Cathedrals***** and Armory*****), Christ the Savior Cathedral****, Basilius Cathedral*****, Cosmonaut Museum*****, Cosmonaut Monument****\\

September 3: Peterhof:\\
Lower Palace Gardens*****\\

September 6: Pushkin:\\
Pushkin: Alexander-Park****, Catherine's Palace***** (Upper Bathhouse****, Lower Bathhouse***, Eremitage*****, Grotto Pavillon****, Agate Rooms*****, Concert Hall****, Turkish Baths****)\\

September 7: St Petersburg:\\
Michael's Castle****, Marble Palace***, Stroganov Palace*****, Michailowski Palace****\\

September 8: St Petersburg:\\
Eremitage \& Winterpalace***\\

And this day marks is the last one Manuel makes an appearance here (up to now by begin of 2020) after three long trips with overnight stays and three additional hikes. Just a couple of weeks later Manuel moved back to California, where he wrote up his thesis and successfully defended by the end of this very same year. Unfortunately still the last time we saw each other (and it is 2021 already)\\

September 9: St Petersburg:\\
Elagin Palace****, Menschikov Palace****, Jussupov Palace*****\\

September 10: Peterhof:\\

Someone who worked quite often with artists in Russia and thus had visited the country a couple of times previously was my artist friend Kendal. Thus I asked him if he know of any special arts museums in the St Petersburg area which an ordinary tourist like me might not have heard about. He told me what I planned to do seemed already extensive enough to add more to the agenda, but by coincidence he would also be over for work from September 10 on. Since we overlapped on that one night, we decided we should meet up for dinner, thus we had dinner at his place of stay after a walk through the Nevsky Prospect. Sometimes also nice coincidences like that lead to a nice evening in foreign countries.\\

Peterhof: Peter- \& Paul Cathedral***, Grand Palace***** (Upper Park***, Lower Park*****, Eremitage***, Catherine's Wing*****, Palace Church****, Monplaisir Palace*****), Colonist Park**** (Olga and Tsar Pavillons), Alexandria-Park**** (Farm Palace****, Gothic Church*, Country House****)\\

September 11: Strelna\\
The last day started with the usual big breakfast, and then i took the train out to Strelna. The Konstantin Palace had been destroyed and fallen even further in despair after World War II, before Putin decided to have the palace and the park rebuilt as presidential residence in St Petersburg, also hosting the 2013 G20 summit there. Several tours are offered, I decided to just choose the standard tour, which was held in Russian only anyway, not that many international travellers seemed to be around. The interior tries to recreate the neoclassical and baroque style of the original palace, some quarters are more modern or contain commodities like pool tables and a modern bar.\\
After the Konstantin Palace I continued my afternoon at Peter's I Summer Palace which was quite modest, clearly not allowing any interiour photography like anywhere else around Peterhof. Since I didn't want to be late for check-in I turned up at the airport 3 1/2 h in advance. Unfortunately the check-in didn't open until 2 h before the flight, neither was I allowed inside any area with restaurants OR wireless connection, thus I had to spent 90 minutes doing well pretty much nothing. Air France gave me the emergency exit seat as well, not allowing me to use my camera as well, thus unfortunately I didn't get any nice panoramic photos of St Petersburg and the bay out again despite sunny weather this time around.\\

Strelna: Konstantin Palace*****, Peter's I Summer Palace***\\

\section{September 18--September 20: Paris II}
\label{2015ParisII}

September 19:\\
Hotel de Lassay*****, Palais Bourbon*****, Hotel du Quai d'Orsay*****, Hotel de Brienne***, Hotel de Breteuil****, Hotel Potocki****, Argentinian Embassy**, Hotel de La Tremoille****, Hotel de Behague*****, Ecole Militaire****, Hotel de Soubise*****\\

September 20:\\
Hotel de Noirmoutier****, Hotel de Villars***, Hotel de Montmorin***, Hotel de Vogue*, Sorbonne*****, MINES ParisTech*, Val de Grace****, St Patrick's Chapel**, Pantheon****, Etienne-du-Mont***, Pavillon Boncourt*, Institut de France****, Bourse de Commerce***, St Eustache****

\section{October 3--October 11: Bavaria \& Austria}
\label{Bavaria2015}

I was invited to give a talk at the ISMD 2015 conference about recent QCD measurements at high energy. Since this conference was taking place at the resort of Wildbad Kreuth (belonging to the Wittelsbach dynasty, the former royal family of Bavaria, but more famous as former annual meeting place for Bavaria's ruling conservative party) attendants were supposed to be picked up at Munich airport and then shuttled down to the Bavarian alps. Since Eric had moved to Siyi and Bavaria recently I suggested we should try to meet up. He instead suggested I should get in a day earlier and just stay a night at his place which I quickly agreed to. For the weekend after the conference I proposed a short weekend trip to nearby Austria. Siyi was quickly convinced to join as well.\\

Co-travellers:\\
Eric: after graduating Eric took a postdoc at the Max-Planck-Institute in Garching and moved to Germany, great to catch up with him again.\\

Siyi: Siyi is an astronomer at the European southern observatory and based in the Munich area. Having visited Eric quite a couple of times in Geneva previously, she was happy to host me as well and to join us on the quick trip to Austria.\\

October 3: Munich\\
I took the first plane out of Geneva to Munich. October 3rd is German national holiday and one of the last day of the Munich Oktoberfest (yes the REAL Oktoberfest). Considering that other folks on the plane were in their Lederhosen and Dirndl outfits I guess I was not the only one ready for it. Once I arrived at Siyi's and Eric's place I was introduced to Siyi's friend who would also spent the weekend with us. And then we were ready to go. Considering it was German national holiday all tents had already filled up and were closed to people (and it wasn't even 11 am yet). We got ourselves a Mass of beer and some bretzels and enjoyed the atmosphere. At some point we had enough and then went on to have lunch at our most favourite place in Munich, the N\"urnberger Bratwurst Gl\"ockl am Dom (and yes you can check that in 2010~\ref{2010:Bavaria}, 2014~\ref{2014:Germany}, 2017~\ref{Munich2017}, and 2018~\ref{Germany2018}), having a quick view of Michaelskirche and Frauenkirche on our way. Clearly Eric, Siyi, and I were very pumped for Omas Fleischpflanzl (Grandma's meatballs), so I ordered 10 pieces, unfortunately the waitress understood my order as 10 portions (aka 20 pieces). We were brought a gigantic pot with very delicious meatballs, and Eric and I just though challenge accepted (after all very delicious things are always great to eat). And we started and they were as excellent as always -- same for the beer going with it from Augustinerbr\"au -- but even experts in big portions and quick eating habits like Eric and myself have to declare defeat at some point. Thus we offered our last two pieces to the gentlemen on the table next to us which they gladly accepted. They (and we actually ourselves) had fun to see if we actually would finish the full pot, I guess we clearly expressed our surprise and excitement once we realised our mistake in our order. One of them commented that once he saw what we tried to achieve he was very impressed on how far we got, considering we were actually not looking as we would tackle gigantic portions on a regular basis. Afterwards we just walked a bit through old town with visits of St Peter, the church of the holy ghost (Heilig-Geist-Kirche), the Asam church and Theatinerkirche before heading back to Eric's and Siyi's place having no space for dinner anyway but playing some Majhong where I turned out to be not so bad at for a starter.\\

Frauenkirche***, Michaelskirche****, Heilig-Geist-Kirche***, St Peter****, Asamkirche****, Theatinerkirche****\\

October 4: Munich:\\
On day two in Munich we considered doing one of the palaces: either Nymphenburg, or the Residenz. Usually on a sunny day I prefer to see Nymphenburg but this time two of us had to get somewhere else: to Heidelberg and to Wildbad Kreuth, thus we decided we should rather stay close to the train station. As I have already described multiple times before, the Residenz is gigantic. Some complain that a lot of the artefacts are shown in other rooms than originally, having been stored away in safe places and later brought back in a new shell after reconstruction. Other rooms have been completely reconstructed. Big halls and courtyards like the Renaissance hall of Antiquarium survived almost undamaged. Still no matter the state of work to make the glory reappear I personally like the result. Still the K\"onigsbau with the royal quarters was still in renovation (after almost a decade). Anyways I got back to the airport and hopped on the shuttle bus which brought dozens of other physicists and myself to the ISMD 2015 workshop where I gave an invited talk concerning recent QCD results from several experiments, including also Tevatron results. They were a bit puzzled when I asked them if I could show their results as well as results from three LHC experiments. If I am asked to give an overview talk it better be an overview and not only my personal experiment.\\

Residenz***\\

October 5--October 9: Wildbad Kreuth\\
The conference took place at Wildbad Kreuth, a venue belonging to the former Bavarian royal family, also famous as former meeting venue of the christian socialists in Bavaria. The place has its own swimming pool, a sauna, as well as its own bar. What rather caused issues was the missing stable internet connection in the meeting room, particularly bad if you want to have last minute edits on your talk. As explanation was given that private conversations would be better than online conversation. Also bad for myself, since I wanted to finish up applications while being on the conference myself. The social program was a cruise of the lake Tegernsee, unfortunately the weather was not that kind to us. But still it was nice while the weather was not rainy, the lake and the hills in fog are quite a sight too. Once it started to rain good coffee and cake sweetened the deal.\\

Wildbad Kreuth***, Tegernsee****\\

October 10: Melk\\
Stift Melk towering on a rock over the Danube river is one of the famous photographs which get people excited about Austria. But the Wachau region is a bit away from major cities (OK if you consider major to be beyond two hours). Nevertheless it is iconic enough that I could easily convince Eric and Siyi to see it on a weekend trip.The day before after my conference had finished, and they had finished work we met up at the airport rental car centre and took off. Two traffic jams later we arrived in Krems, had some sausage salad, and after breakfast we were ready to go. Even the monastery with the Kaisersaal and the large library are impressive, sadly one was not allowed to take photos in the library itself. I am obviously aware that the old books and papers are particularly delicate, but they are all behind real glass and if not photographed with flash, at least the art itself cannot be damaged. The church itself is very ornate late Baroque even with a gilded crown hanging over the high altar. Then we stopped by the garden pavilion for a coffee and cake.\\
 Then we continued our visit to Stift G\"ottweig, another large monastery in the Wachau area. There the stair case is very impressive, but also the guest rooms contain some precious tapestries. After having soup for lunch we had still some time to spare, thus instead of driving directly to Salzburg we did drive to Passau instead.\\
 Since this was at the height of the refugee crises border controls had been tightened and guess who they chose to pick - the rental car. Now in this car we have -- a german who didn't live in German anymore, an American on a normal German visa living in Germany, and a Chinese living in Germany on a very special high skilled researcher visa. As Chinese citizen Siyi needs a visa for almost any travel, and as astronomer she travels quite a bit, thus the poor border police man had to scroll through many many visa pages, and the German working visa is a very special kind. But we did obviously clarify that pretty easily, he was only very curious how in the world we three ended up knowing each other. Of course a good and fun story to tell. And 5 minutes later we were on our way to Passau.\\
 Passau is called the three-river-town, since the Inn, the Danube and the Ilz flow into each other by Passau old town. Unfortunately that leads to heavy flooding even nowadays regularly. The cathedral itself is outstanding, large, with many nice frescoes, a great dome with many stuccos and sculptures, and a large famous organ. And then back to Austria and Salzburg. Having arrived in Salzburg we asked what restaurants they recommended. The pizzeria they suggested had a birthday party going on with lots of Schlager music so we decided to pass and ended up in a rather high class restaurant. Food was very good, a tad expensive but not too much, thus good that we ended up there.

Stift Melk*****, Stift G\"ottweig****, Dom Passau*****\\

October 11: Salzburg\\
Salzburg had been the seat of one of the most important bishops of the holy roman empire. He could afford to hire a personal composer, one of them you might have heard of previously: some person called Wolfgang Amadeus Mozart. Well the Old Residenz is a fantastic series of Baroque rooms with lots of paintings, tapestries, and what not (well obviously no photos allowed as in almost all Austrian palaces). It is directly connected to the beautiful baroque cathedral too, another highlight of Austrian Baroque. But even more outstanding is the old castle of Hohensalzburg: one of the best conserved medieval residences on the continent. Never destroyed, the Golden Halls are some of the best conserved wood carvings from the late medieval era and it still houses old chimneys from that era too. We didn't want to risk getting held up on the border again, thus after having some food on the market square we made it back to Germany without using the highways (although we had a car with those privileges). On that day Siyi was also handed over a kitten -- the lovely and lively Oreo -- thus she only wanted to have dinner in one of the Beer-gardens close to the airport but didn't want to do the Erdinger brewery which is also quite close to the airport. Still an unforgettable trip, although Siyi and Eric might it remember more as the day they got Oreo.\\

Alte Residenz*****, Dom*****, Hohensalzburg****, Kajetanerkirche***

\section{November 8--November 15: Berlin and all of Germany}
\label{2015:Berlin}

Why did I go there: For once I wanted to see Madonna in concert, the choices for Germany were either Cologne or Berlin. My mum wanted to see her too and voted that she would rather prefer to see Berlin again over Cologne. My sister and my dad joined on this trip as well for a few days, so we had another family holiday happening.\\

November 8: Potsdam:\\
My parents and I arrived a bit earlier than my sister, so we decided that we should see the palace of Sanssouci (my sister is not that much into old churches or palaces). Thus we dropped our luggage by the hotel before getting on the train to Potsdam. Having arrived there we first had a guided tour through the New Palace (Neues Palais). Originally set up as palace for guests, the emperors made it their main summer residence, preferring Potsdam over the Berlin city palace. While the Marble Hall and the palace theatre were in renovation, I particularly enjoyed seeing the concert room, the silver chamber, as well as the grotto on the ground floor. Afterwards we walked through the gardens over to the main palace, built in Rokoko style for Frederick the Great of Prussia. Almost all of the room are still in their original state (main changes had been done only for the working and bed room of the king). Unlike in my previous visits, this time we were on the last visit of the day, and due to it being winter we saw the palace at night. The ambience was really special, considering the chandeliers and lights illuminated some parts of the rooms brightly, while others were completely in the dark. Then we met up with my sister, checked in and had some late dinner.\\

Potsdam: Sanssouci***** (Schloss***** \& Neues Palais*****)\\

November 9: Hamburg:\\
Hamburg and Berlin are well-connected by high-speed trains. Having never been in Hamburg, my mum and I decided we all should go there for a day. Considering that Germany's large particle physics centre is in Hamburg, I am still surprised I never made it there for work purposes. Clearly this was not supposed to change on this day, instead we started our day with a tour of the city hall. The city hall was built in a neo-renaissance style, to celebrate Hamburg's important past and to impress honorary guests like the German Emperor. The city hall is one of the largest (if not the largest) of Germany, and the rooms are indeed large and impressive, I enjoyed the Tower Room in particular, as well as the emperor's hall. 

Hamburg: City Hall*****, St Michaelis****, Port*****, Speicherstadt***, Elbtunnel****\\

November 10: Berlin:\\
Zeughaus (Museum of History)****, East Side Gallery*****, Madonna: Rebel Heart Concert*****\\

November 11: Schwerin \& L\"ubeck\\
Schwerin: Palace*****, Dom***\\
 L\"ubeck: Holstentor**,** Marienkirche****, Dom***\\

November 12: Berlin:\\
Schloss Bellevue*****, Bundespr\"asidialamt***, Siegess\"aule****, Brandenburg Gate****, Sovjet Memorial***, Reichstag*****, Sony Center****\\

November 13: Weimar, Erfurt \& Fulda:\\
Weimar: Herzogin-Anna-Amalia-Library****, City Palace****\\
Erfurt: Dom*****, Severikirche****, \\
Fulda: City Palace*****, Dom*****, St Michael****, St Blasius****\\

November 14: Rudolstadt \& Gera:\\
Rudolstadt: Heidecksburg****, Gera: Schloss Friedenstein*****\\

November 15: Berlin:\\
Schloss Charlottenburg***** (New Pavillon****), Museum Berggr\"un*****, Scharf-Gerstenberg Museum****
\chapter{Year 2016}
\label{2016}

\section{January 2: Colmar}
\label{Colmar2016}

Colmar: Unterlindenmuseum*****, Martinsm\"unster***

\section{March 16--March 25: Sicily}
\label{Sicily2016}

March 17: Piazza Armerina, Cefalu \& Monreale:\\
Piazza Armerina: Villa Romana del Casale*****\\
Cefalu: Duomo****\\
Monreale: Duomo*****\\

March 18: Palermo:\\
Palermo: Norman Palace****, Capella Palatina*****, San Giorgio in Kemonia**, San Giovanni degli Eremiti***, Chiesa del Gesu*****, San Cataldo****, La Martorana*****, Fontana Pretoria****, Piazza Vigliena****, Cathedral***, Castello della Zisa****, Castello della Cuba***, Teatro Massimo***, Giuseppe dei Teatini****

March 19: Palermo \& Segesta:\\
Palermo: Giuseppe dei Teatini****, Palazzo Mirto*****, Oratorio dei Bianchi***, Francesco dei Assisi****, Oratorio di San Lorenzo*****, San Matteo***
Segesta: Hera-Temple*****, Theatre****\\
Castellammare del Golfo: Church***, Harbour****\\

March 20: Selinunt, Scala dei Turchi \& Agrigento:\\
Selinunte: greek ruins in Selinunt****,\\
Realmonte: Scala dei Turchi*****\\
Agrigento: San Francesco***, San Lorenzo***\\

March 21: Agrigento:\\
Agrigento: Valley of the Temples*****, Church of S. Nicolas***, Archeological Museum****, San Spirito****\\

March 22: Val del Noto:\\
Comiso: St Blasius***, S. Maria Assunta****, Noto: San Domenico***, Cathedral****, Palazzo Ducezio***, San Carlo al Corso***, Montevergine***, Palazzo Nicolaci di Villadorata****, Santa Chiara****, San Francesco***\\

March 23: Siracusa:\\
Siracusa: Santuario della Madonna delle Lacrime***, Duomo****, Apollo Temple***, Carmelite Church***, Amphitheatre***, Greek Theatre****, Necropolis****, Latomia del Paradiso*****, Catania: Francisco de Borja***, Basilica della Collegiata***, Cathedral***\\

March 24: Etna:\\
My mum went to our hotel to heat herself up, while my dad and I paid a visit to the Roman Theatre and the Odeon. Those were hidden in the fundaments of old houses for centuries, before those were removed by and it was tried to give them back their original appearance, and one short visit to the Johannes-Bosco church.\\

Mount Etna Nationalpark: Mount Etna*****, Grotta dei tre Livelli****
Catania: Roman Theatre \& Odeon****, Johannes-Bosco**\\

March 25: Catania:\\
My mum didn't feel like going into the city so after breakfast she wanted to see the picture gallery inside of the Castello Ursino, my dad and I started with the large complex of the Monastery of San Nicolo l'Arena, nowadays part of the university of Catania. The church of San Francesco is a nice Baroque church too, even more impressive is the church of San Benedetto with a vastly decorated church with many paintings. Since we had still a bit of time left, I made it to the picture gallery of Castello Ursino, also nice there the Great Hall and the palace chapel, and then we took the bus to Catania airport. Now I transferred all photos to my laptop, gave my mum a UBS with all photos. On the flight to Geneva I sat on the side opposite of Mount Etna, I only got a glimpse of its snow covered top above the clouds. But then while continuing to fly north we passed the Aeolian Islands, also born in volcanic eruptions over millennia. Most impressive were the islands of Volcano and Stromboli, where an eruption send up clouds at that time. Once back in Geneva we had one more coffee at CERN, before my parents departed, I went home to prepare for my trip to Barcelona the next day, and stupid me deleted all the photos I had shot on the plane, thus no photos of all the volcano and islands from that flight.\\

Catania: Monastery of San Nicolo l'Arena****, San Francesco***, San Benedetto****, Chiesa della Badia***, Castello Ursino***

\section{March 26--March 29: Barcelona}
\label{Barcelona2016}

How this trip happened: Easter always offers a couple of days to spend somewhere else without the need to take any holidays out of your holiday budget. Not that this was of any matter at the end of a contract. Flights to Barcelona were pretty cheap, thus I decided we should check it out again. My second time since the year 2000, when I was in Barcelona during a high school cultural excursion. Since Barcelona is quite crowded nowadays we got a couple of tickets beforehand already, particularly for Sagrada Familia it is notoriously difficult to get tower tickets on the same day.\\

Co-trallers:\\
Riju: after over half a year since our quite turbulent last weekend trip, Riju feels his ready and more prepared for a long weekend, deciding not to join me on every single adventure, but to take it a bit slower than myself.\\

March 26: Barcelona:\\
After arriving and dropping our stuff in our hotel, we went to Casa Mila straight away, where long queues awaited us. Riju bought tickets using their free WIFI and we were ready to embark on our journey through Art Nouveau and modernist Barcelona. The inside of Casa Mila is not that amazing, but the decorations of the inner courts and particularly the roof are very spectacular. I then visited the Casa Llero Morera, the house with the most interesting inner decoration of the four houses I went on to see on that day, particularly the stain glass windows. The house had one of the first elevators in Barcelona, integrated right from the start. For Casa Battlo the facade and the roof are definitely the highlights, the ceiling are nice toowith their integrated large lights. Then the clear highlight of the whole trip was on our agenda: the Sagrada Familia, also the most reviewed building on tripadvisor. And man, did it change since 2001, when I last visited. Back then almost none of the interior had been finished, the choir walls and the two facades had just been finished by then. Now the interior of the church was complete, illuminated through colored glass, the gates decorated with little animals, like snails or beetles, and leaves, columns reminiscent of palm trees. Once more walking up the towers was very special, particularly seeing the decorations up-close. While Riju called it a day, I continued with the Casa Amtaller. A lot darker than the other houses, it was supposed to have a kind of medieval infused appearance. After a short dinner I went to a night show at Casa Mila, particularly solemn, with a movie projected onto the roof chimneys and towers. And as icing on the cake I got snacks and a glass of Cuvee.\\

Barcelona: Casa Mila*****, Casa Llero Morera****, Casa Batllo****, Sagrada Familia*****, Casa Amatller***\\

March 27: Barcelona:\\
Barcelona can get quite busy, but if you go to the old city very early in the morning, you are pretty much on your own. Even the cathedral is deserted. Mainly open for prayers in the morning, it gives you a really peaceful experience just walking through the nave and sitting below the Dome in solemness. Quite the opposite about what you might experience just a couple of hours later, when you rather walk through the packed hordes of people in a stop-and-go pedestrian jam. The cathedral itself is of gothic style, but with many baroque altars in its side chapels, as well as carved wooden seats in the choir. There are also two big other gothic style churches in Barcelona (Santa Maria del Pi and Santa Maria del Mar), but the cathedral is far more beautiful. On Saturdays the city hall, Casa de la Ciutat, can be visited -- back in 2016 for free. The main assembly meets in a room reminiscent of the time when the building was originally constructed. Many of the paintings in neighbouring rooms have been changed over times, thus you are offered an interesting mix of times. Then we took part in the first tour of the Palau de la Music Catalana of this day. Barcelona is famous for its Art Nouveau buildings, particularly by Antoni Gaudi. Only concentrating on Gaudi's building you might miss out on this beautiful piece of architecture, even a world unesco heritage building on its own account. The stain glass of the roof is absolutely breathtaking, the decoration and statues around the stage are magnificent. During the tour we even were given the opportunity to hear the organ play Bach's Toccata in d-minor. By that time we really were pretty tight on the schedule. Riju thought it would be hopeless to reach, but I ran as fast as I could over to the Palau de la Generalitat de Catalunya, the parliament of the Catalonian region. Originally I was informed that no tour was taking place on this day a couple of weeks before, but then a few days before the trip I received an email, that on short notice a tour would be available. Indeed just a minute after I arrived the doors were closed, Riju who also made it eventually was standing in front of closed doors. There are a lot of depictions of the national saint of Catalonia, St George or St Jordi in Catalonian, all over the place. His deeds and miracles are depicted in the giant hall of St Jordi, which houses also the most impressive fresco in the palace, covering all of the upper walls and the roof. I met up with Riju again to see Palau Guell, one of the big projects Gaudi did for one of his most important clients. I personally think it is the most beautiful house Gaudi built in Barcelona (out of the four I know), particularly the central hall with its built-in organ. Riju had decided to spend the rest of the afternoon in the Picasso Museum, which he was really impressed by. Since I had seen quite a couple of Picasso paintings already, I decided to see the grand theatre of Liceu instead in a cute tour. Then I saw the remains of the medieval royal palace of Palau Reial Major, and then went on a stroll through the Barro Gotico with a short visit of the baroque Basilica de la Merce, and then a walk along the alley up to the Columbus column. The Frederic Mares collectors museum was on my way, but nothing too interesting report from it. \\

Barcelona: Cathedral*****, Basilica Santa Maria del Pi***, Casa de la Ciutat****, Palau de la Musica Catalana*****, Palau de la Generalitat de Catalunya****, Palau Guell*****, Gran Teatre del Liceu****, Basilica Santa Maria del Mar***, Basilica de la Merce***, Palau Reial Major***,  Frederic Mares Museum*\\

March 28: Zaragoza:\\
Barcelona is a very exciting place to see, but it is as interesting to discover new places, this time Zaragoza. Close to the train station is the Aljaferia Palace, a fortified medieval Islamic palace. The Golden Hall and the northern halls reminded me a bit of the Alhambra. While the Alhambra is totally out of this world, the Aljaferia is still a really nice place to see, on the level of palaces in Marrakesh for example, another highlight is the old mosque. Once the islamic times in Spain had gone, the catholic Monarchs constructed another royal wing in the late 15th century. The ceilings are clearly the highlights of this wing, particularly in the new Throne room. In the following centuries the palace fell in despair and only in the 1980s the restoration was finished.
Afterwards we passed over the Renaissance courtyard Patio de la Infanta and the St Engracia with a beautiful facade. The most important buildings in Zaragoza are the cathedral and the Basilica del Pilar. The cathedral had been built throughout many years, the apsis and the choir are of particular interest in Mujedar style, the stuccos and sculptures alongside the chapels are really special, the veto on photography is also very rare for Spanish cathedrals though. The gigantic baroque basilica del Pilar had been build for about 200 years, largely finished in the late 19th century. The four towers and multiple domes clearly dominate the city scape of the old town. The domes of the basilica are partially decorated with ceiling paintings by Goya, the main building surrounds a baroque chapel building. 
On the way back a passenger complained about our ``loud english'', while his friend next to him didn't bother him talking on the phone just a bit later on. The family in front of us was quite lively too with one of the kids having to throw up just when before we arrived in Barcelona again - oh well. And the gentleman who complained about us, also lost some 10 or 20 EUR when getting up from his seat, and Riju had to run after him to give it back to him. Riju told me that a step tracker claimed we had walked about 23 km on this day.\\

Zaragoza: Palacio de la Aljaferia*****, Ibercaja Patio de la Infanta***, Basilica de Santa Engracia**, San Gil Abad***, Cathedral*****, Basilica del Pilar*****,San Juan de los Panetes*\\

March 29: Barcelona:\\
The Park Guell is a vast public park in the north of Barcelona. It had been developed shortly after 1900 by Antoni Gaudi, Nowadays the central part of the park, with multicolored mosaic Salamanders, terraces, and a hall with lots of columns and tile ceilings is so popular, that you have to pay to get in. The surrounding colonnaded foot trails and other terraces are not that popular, and nice to spend some time there. The former Hospital de la Santa Creu i Sant Pau was built by Lluis Domenech i Montaner, the architect of the Palau de la Musica. In 2003 the hospital moved out of the complex, and it has been transformed into a museum and cultural centre. The main entrance and former administration building is the most impressive part, other parts of the complex have been restored to their original appearance on the inside.\\

Barcelona: Park Guell*****, Hospital de la Santa Creu i Sant Pau****\\

And this concludes my travel chapters while working at UCLA: My life was supposed to change quite a bit, starting a new job in a new group and in a new (not yet existing) experiment, thus for at least the first couple of months I decided not to take too many days off.\\

\section{April 10: Gruyeres}
\label{2016Gruyeres}

Gruyeres: Musee HR Giger****, Chateau de Gruyeres****, Broc: Jaunbach Gorge****, Monsalvens***

\section{May 5--May 8: Piemonte}

May 5:\\
Venaria Reale: Reggia*****, Castello della Mandria****, Palazzo Reale*****, San Lorenzo****, Palazzo Carignano****, Palazzo Madama****, Duomo****\\

May 6:\\
Racconigi: Castello****, S. Maria Assunta****, Aglie: S Maria***, Castello Ducale****\\

May 7:\\
Torino: Castello del Valentino*****, Borgo Medievale***, Villa della Regina****, San Filippo Neri***, Egypt Museum*****, Nichelino: Palazzina di Caccia di Stupinigi*****\\

May 8:\\
Torino: Basilica di Superga****, Palazzo Accorsi****, Basilica del Corpus Domini****, San Domenico***, Palazzo Carignano****, Santuario della Consolata*****

\section{May 14--May 16: Liguria}
\label{2016Liguria}

Another long weekend, another trip to Italy: This time I wanted to see the town of Genoa with its many little palaces. I convinced Amin to join me on this trip, Eric had unfortunately no time, I decided to use a bus for once, trains just took too long. \\

Co-traveller:\\
Amin: UC Riverside grad student, Iranian. This is our first visit together, and I hope he will enjoy it just like other people did before. Iranians typically need a visa for almost every other country in the world, thus being in Switzerland on a Schengen visa opens up quite a couple of places to visit for him here.\\

May 14: Genoa\\
Starting our trip just shortly after midnight at the bus station of Geneva in a little bus with two other co-travellers. We crossed the Mont Blanc tunnel had a short stop for snacks in the Aosta valley before we had to exchange buses in Milan. We arrived on time by the harbour of Genoa, dropped our luggage by our hotel and started our day with a breakfast and some espresso. Our hotel was located very close to the train station, also not far off from old town. Amin found out that his pin code for his card was too long for the interface of the Italian ATM, so at least no money by the train station. We had our first tourist stop by the nice baroque basilica dell'Annunziata, before walking along the Rolli palaces to the cathedral of San Lorenzo. The choir was covered in scaffolding, but the side chapels and the nave were pretty nice. We then saw an arts exhibition by the Palazzo Ducale. Unfortunately most of the rooms were covered by panels for the exhibition, the large council hall was closed for the day, only the duke's chapel was free. Genova is full of Baroque churches, every major square has at least one of them, e.g. the Chiesa del Gesu, another superb church, with a nice frescoed main dome and also nice domes on the side naves even. We just needed to walk along the streets, no matter which large church we stepped in, they all were pretty and beautiful. One of the famous palaces to see is the Palazzo Spinola di Pellicceria, the visit covers four floors, all of them have at least one nice room, like e.g. a Gallery of Mirrors on one of the upper floors, but also with nice paintings by Brueghel the younger and Rubens, as well as porcelain plates and vases. After a quick lunch we continued to see a couple of other small churches. In the afternoon we returned to the Palazzo Ducale, where a talk was presented in the Small Council Hall. And then we saw the highlight of the day - the Palazzo Reale, originally a palace of one of the local family which was bought by the Royal family of Savoy, which later became the kings of Italy. The rooms are richly decorated, including a nice gallery of mirrors, a throne hall in red (similar to all other Italian Throne Halls I have been at), a room with several nice tapestries and a roof terrace with a nice view of the harbour with the light house, as well as a superb view of the garden. The Nymphaeum with depictions of horses, peacocks, jaguars, and a weird looking depiction of elephans, was my personal highlight of the gardens. On the ground floor of the royal palace they had an exhibition about Antonio Canova. And then we had dinner and Amin watched the Eurovision Song Contest for the first time, not that we felt the need to vote for any of the songs.\\

Basilica dell'Annunziata*****, Cattedrale di San Lorenzo****, Chiesa del Gesu*****, Basilica di Santa Maria delle Vigne****, Palazzo Spinola di Pellicceria****, Santi Vittore e Carlo****,  San Luca***, San Pietro in Banchi***, Palazzo Ducale****, San Donato***, Palazzo Reale*****\\

May 15: Genoa\\
After a small breakfast we walked over to the Villa del Principe, which had a nice garden, and a nice decorated Loggia as well as other nice rooms with old tapestries depicting sea battles, and the legends of Perseus. Many of the ceilings were covered in nice frescoes as well. The gardens had two nice fountains as well. Then we walked up the hill to the romantic house of Castello d'Albertis. The palace had its own version of a mock-up turkish tent inside one of the rooms, as well as artefacts from the Andes as well as Mayan towns. After a nice stroll through some of the hill side quarters we ended up in old town again by Palazzo Rosso. Nowadays this palace is together with the Palazzo Bianco and the Palazzo Doria Tursi an arts museum, full of old masters. Some state rooms of the Palazzo Rossi are part of this museum as well, I liked the rooms of the four seasons in particular. From the roof terrace one has a nice 360 degree view of the whole city. Amin went back to the hotel to relax. I myself had read the day before that the coastline by Genoa was very nice, and an village with a nice old harbour and a small beach would be somewhere around there. So i just started to walk along the shore. The walk was nice, the weather was too, I got myself an ice-cream and continued to walk. But about over an hour in I got bored (not knowing how far longer it would take) and decided to walk back. Looking on a map later on, i missed a little bit less than 2 km to reach the village. Anyway I met up with Amin again and we had dinner and desert.\\

Villa del Principe****,Castello d'Albertis***, Palazzo Angelo Giovanni Spinola***, Palazzo Rosso****, Palazzo Bianco***, Palazzo Doria Tursi**, Loggia dei Mercanti***,  Basilica di San Siro****\\

May 16: Cinque Terre\\
Starting early in the morning getting to the coastline of Cinque Terre by train. We almost missed our train, as the reception was not occupied, even in a 24 h reception, some natural urges have to be followed up from time to time. Anyway we made it and arrived by Riomaggiore. Not too many people were around before 8 am, we went to the harbour and enjoyed the cliffs. Sadly the trail to Manarola is currently being refurbished (still not opened by begin of 2020). Then we had a short breakfast and took the next train to Manarola. The view of the tiny harbour and the village is really beautiful, for me the nicest view of all the villages. The beach of the village is tiny, but some people were already jumping into the water here too. By the time we got off the train in Corniglia considerably more people started to appear (around 9:30 am). Corniglia is the only village without a ferry stop, thus buses or trains are the only means by which it can be reached. Or by hike from the surrounding villages. And then we took the train all the way to Monterosso al Mare. This village is the only one in Cinque Terre with a large sand beach, also with a couple of smaller and larger rocks, which a few brave people did climb on. There are two little churches in the village, the oratory of the dead is more beautiful one of the two. And then we got on the ferry back along the coastline. The ferry is not large, thus the ride can be a bit bumpy, not that I myself had any issue with that, although almost everybody was fine with it. We saw the fifth village of Vernazza only from the sea side and the harbour.. But it looked nice too, the church is directly by the harbour, and a Castello is just a short walk by the harbour as well. We didn't get really close to Corniglia, although a trail exists there to get to the sea via a couple of stairs. Passing Manarola we once again stopped in Riomaggiore. There we had another lunch with pizza and desert. And then we took the train back to Genoa. Just as we arrived by our hotel I received a call of our bus-driver that they arrived and whenever we would be ready, we could start. Since we didn't have much planned to do anymore we went down to the tavern by the port, had another espresso and we (as only passengers from Genoa) started our way home about 40 minutes early. We did arrive in Milano on time, but unfortunately one of the passengers who was supposed to get on board there was nowhere to be found. Thus we left Milano with an additional delay of 35 minutes. After the usual traffic jam by Courmayeur before crossing the Mont Blanc tunnel we arrived in Geneva with about 45 minutes delay. Other folks remained on the bus on their way to Basel. It was even early enough for Amin to get back all the way to France.\\

Cliffs and coastline of Cinque Terre*****, Monterosso al Mare: John Baptists***, Oratory****

\section{May 28--June 3: Cantabria}
\label{Spain2016}

May 29:\\
Santander: Cathedral**, pre-historic archeological Museum****, Puente Viesgo: El Castillo*****, Las Monedas*****\\

May 30:\\
Burgos: Cathedral*****, San Nicola*****, San Gil****, Santander: Palacio de la Magdalena***\\

June 3: Bilbao\\
Guggenheim Museum*****, San Nicolas de Bari***, Cathedral***

\section{June 11: Lauterbrunnen \& Bern}
\label{LauterbrunnenBern2016}

Why this trip and co-traveller:\\
Andrew, my former co-worker from UCLA, hadn't been over in Europe for more than a year, and clearly once he is back we had to use the opportunity to visit the mountains once again, but this time I put not only nature, but also culture on the agenda. My attempt at putting local food on the agenda was thwarted by too expensive prices.\\

We took the car up to Lauterbrunnen. Usually I start my trip up to Jungfraujoch from there, but this time we had other plans: One of Switzerlands highest waterfalls is the Staubachfalls, but it also possible to hike up the cliffside and get behind the wild waters of the waterfall itself. Clearly one has to be willing and ready to get soaking wet, clearly you are guaranteed to be rained on just standing less than a metre behind a waterfall. Still feeling and hearing the power of the falling water is an experience you don't want to miss out on. Then we drove a bit further into the valley to the Tr\"ummelbach waterfalls. These are in total 10 waterfalls hidden within the mountain, where all water from the Jungfrau mountain massif falls down before ending up in the Weisse L\"utschine river. You can imagine that being fed by five glaciers the waterfalls are pretty strong and powerful. Being hidden within a cave like trail adds to the experience. If you are afraid to get wet or afraid in dark places, it might not be your thing to see. Don't fear of needing to walk up, a lift through the mountain takes you to the top, and you just shouldn't slip on your way down. All in all the waterfalls are spectacular and if you end your hike through the rock face with a valley overview in fog you understand that the Lauterbrunnen valley claims to be one major inspiration for J.R.R. Tolkien's Rivendell. And then it was time for local cake and coffee.\\
And on the way back we stopped in Bern. Bern's old town is located between a horseshoe bend of the Aare river. Thus it is a nice natural setting and the old town is more or less unchanged since the 18th century. The gothic style Berner M\"unster is the largest church in Switzerland with stained glass windows from mid 15th century and a main portal originally constructed in the 17th century. The town itself still has many medieval fountains scattered all over the roads, the gates are still in their original form as well. Then we wanted originally to taste Berner Geschnetzeltes, which is small cut veal served in a tasty sauce going with R\"osti or mashed potatoes. Unfortunately prices were south of 50 Swiss Francs thus we went back to Geneva and had some affordable pasta and pizza there instead.\\\

Lauterbrunnen: Staubachfall****\\
Stechelberg: Tr\"ummelbach Falls*****\\
Bern: M\"unster****, Fountains and Gates of Old Town****\\

\section{June 24--June 26: Ravenna \& Padua}
\label{RavennaPadua}

June 25: Ravenna:\\
Ravenna: Archbishop's Chapel and Diocesan Museum*****, San Francesco**, San Giovanni Evangelista**, Arian Battistero***, Sant'Apollinare Nuovo*****, Theoderic's Palace***, Dante's Tomb***, Neonian Battistero****, San Vitale*****, Galla Placidia Mausoleum*****, Santa Maria Maggiore***, Theoderic's Mausoleum***, Duomo***, San Lorenzo in Cesarea****\\�
Classe: Sant'Apollinare in Calsse****\\

June 26: Padua\\
Padua: Emeritani Museum***, Scrovegni Chapel*****, Eremitani Church***, Oratorio di San Rocco****, Duomo***, Battistero****, Basilica di Sant'Antonio****, Palazzo della Ragione*****, Scuola del Santo*****, Basilica di Santa Giustina***, Maria dei Servi**

\section{July 1-July3: Bordeaux -- Eurocup 2016 Germany:Italy}
\label{2016:Bordeaux}

How did we end up here: The UEFA European Championship 2016 took place in France. Considering that it is just around the corner I created a UEFA fan account and tried to get tickets, but ended up getting none. During play off rounds, last minute tickets are up for sale a couple of days before, a maximum of four tickets per person (though not indicated, where the seats would be precisely. The quarter final match between Germany and Italy was scheduled on a Saturday, thus I asked if I could have the day off before, in case I would indeed get tickets (was granted). Moments before sales opened I kept refreshing and refreshing and ended up in a waiting room for 5 minutes. By that time I already informed myself how I could get there: flights were booked out, train to Paris and fan TGV seemed to be the option of choice; hotel rooms were available as well. So I called my sister if she wanted to join and she called my parents to see if they would be up to. After a quick decision (still 1 minute in the waiting room) everybody was on board, so as soon as i got out of the waiting room, I got myself two pairs of tickets, then my sister booked a large motel room for all of us. We decided my dad would pick my sister and me up from Basel train station and we would go with my dad's car.\\

July 1: Driving through France\\
Getting up before 6 I made it to one of the first trains to Basel, met my sister, had some coffee and then getting into the car. Driving by the extinct volcanoes of the central massive, with a stop for snacks around Clermont-Ferrand we finally arrived in Bordeaux, and got to the stadium in order to exchange our voucher for the real tickets. In the evening we took the tram into old town, where we checked out the night views of the Place de la Bourse with the Miroir d'Eau. The square is filled regularly with water (only 2-5 cm), which mirrors the classical buildings of the square (thus Water Mirror). We then saw the last bits of the match Wales-Belgium on the Fanzone by Place des Quinconce. The city was already full by German and Italian fans, our motel was mainly filled by German fans.\\

Place de la Bourse \& Miroir d'Eau*****, Place des Quinconce***\\

July 2: Match day\\
The old city of Bordeaux used to be an important port, and is of historical and culture significance and even considered UNESCO world heritage, thus enough to see for a day. The large churches had been stops on the pilgrimage to Santiago de Compostela as well. We started the day by the large basilica of St-Michel, a gothic church with a separate clock tower, continuing with the romanesque church of St-Croix. After lunch we walked past the remains of the city wall over to the Place de la Bourse. The water mirror is definitely more impressive at night. After another coffee stop we visited the cathedral. Although not necessarily bad, it is a bit plain particularly if you consider all the other cathedral which France has to offer. After getting a Bordeaux wine bottle in order to prepare for an after match night cap, we got on the bus, which should pass by the stadium. At some point we realised though that the bus was taking another route as typical, thus we got off and walked towards the next tram station. Seems trams and cars were the only means of accommodation to get to the stadium. The stadium was rebuilt in view of the euro cup, with 35 000 seats, one of the smaller stadiums in 2016. We got two seats in the German fan-block, just behind the goal. After 10 minutes of intro-programme, the match started. \"Ozil scored the first goal for Germany, Bonucci scored the 1:1 in a penalty for Italy. After overtime the result was still 1:1, thus penalties started. In between Schweinsteiger could have scored for Germany for the victory, but nope he failed. Thus it went on ... and on ... and on. Zaza tried to decay Neuer - and failed too. In the in total 18th! attempt, Jonas Hector finally scored for Germany for the final result of 6:5 after penalties. One of the rare victories of a German over an Italian team in play-off rounds. Celebrations and chanting all round us (and unintentional beer showers). As the stadium slowly emptied, tram after tram was leaving, each of them completely packed. We arrived at the motel roughly and hour later, and then had our Bordeaux wine.\\

St-Michel****, St-Croix****, Cathedral***, Stade Matmut-Atlantique***\\

July 3: Drive home\\
After sleeping a bit in and hours of driving, we all arrived home. A fun trip, but if I would go out of my way to just see Bordeaux in this region, the answer would rather be no.

\section{July 10: Nid d'Aigle}
\label{Niddaigle2016}

Glacier de Bionassay*****\\

This is so far the last time I went on a hike (out of four in total) with Pieter. A contributing factor was, that once you have two kids it can get tricky squeezing a fifth person into your car.\\

\section{July 16--July 17: Zermatt \&Fiesch}
\label{ZermattFiesch2016}

July 16: Zermatt:\\
Kleinmatterhorn*****, Matterhorn Glacier Hike*****\\

July 17: Fiesch:\\
Aletschgletscher*****, Fieschergletscher*****

\section{July 22--July 24: Brussels \& Aachen}
\label{Brussels2016}

July 22:\\
Grand Place*****\\

July 23:\\
Cathedral****, Palais Royal*****, Maison du Roi**,  Aachen: Dom*****, Cathedral Treasury*****\\

July 24:\\
San Nicolas***, National Basilica****, Stadthuis*****, St Catherine***, Notre-Dame du Sablon****,  Atomium*****, Notre-Dame de Laeken***

\section{August 12--August 29: US National Parks}
\label{US2016}

August 13: Chicago\\
Cloud Gate****, Millenium Park***\\

August 14: Chicago\\
Chicago Art Institute*****\\

August 15: Loosing the wind-safe hat: \\
And the day of our first inner US flight of this trip between Chicago and Las Vegas (United once again), and we had our transfer via a stretch limousine, actually the cheapest mode of transportation, it reminded me of my last visit to Chicago in 2012. This time I had the middle seat, so nothing to report about what route we took, my mum had to sit on a separate place from us. Then we got our gigantic rental car (gigantic at least for Europeans, no way it could have made it through old towns in Italy or France). We stopped shortly before the border from Nevada overlooking the low water levels of Lake Mead. We crossed the state border by the bridge spanning over the Colorado River with impressive views, and then it happened: Heavy winds and my mum's hat was set free, and made a happy trip down before landing on the Hoover Dam. She had asked specifically for a wind-safe hat, but seems windy times in Germany are nothing compared to windy times in the US. At least the hat was definitely gone. The first thing we did after arriving at our hotel was shopping for another sun hat, this time with laces, in order to be indeed wind safe. And yes - no wind was strong enough to let the hat sail away until now in 2019\\

Hoover Dam****\\

August 16: Grand Canyon:\\
Grand Canyon View Points*****

August 17: Grand Canyon:\\
South Kebab Trail*****, View Points*****\\

August 18:\\
Grand Canyon Lipan Point*****, Cameron: Little Colorado River Gorge****, Page: Horseshoe Bend*****, Upper Antelope Canyon*****\\

August 19:\\
Oljato-Monument Valley*****, Gooseneck State Park****, Mesa Verde National Park*****\\

August 20:\\
Mesa Verde National Park*****, including Cliff Palace*****\\

August 21"\\
Arches National Park*****\\

August 22:\\
Colorado River Rock Canyon***, Eagle Canyon****, Red Canyon*****, Bryce Canyon National Park*****\\

August 23:\\
Bryce Canyon National Park*****, Zion National Park*****\\

August 24\\
Zion National Park*****, Bellagio Fountain Show****, The Strip****, Volcano****\\

August 25: Las Vegas\\
Luxor***, Strip at Night****\\

August 26: Washington DC\\
White House***\\

August 27 Washington DC\\
United States Capitol*****,  Library of Congress****, Washington National Cathedral***, Washington Zoo****, Basilica of the National Shrine of the Immaculate Conception*****\\

August 28: Washington DC\\
World War II Memorial****, Lincoln Memorial****, Vietnam Memorial***, Korean War Memorial***, Martin Luther King Jr National Memorial***, Jefferson Memorial****, FDR Memorial***, Free Sackler Gallery***, Museum for African Art***, Hirshhorn Museum****, National Gallery****, Matthew Cathedral***\\

August 29: Washington DC:\\
On this day I met up with Luis, an actual tour and travel guide who has travelled the world far more than I did, and who I got in contact with on travel forums before. Once I told him I will be heading to DC, he suggested that we could meet up, should i have time on my hands. Since my parents weren't too keen on another US Capitol tour, but I still wanted to see the House Wing, I decided to suggest that to Luis. And indeed we met in the US Capitol lobby, and joined the tour of the House Wing. I was particularly impressed by the Old Hall of the House of Representatives, which houses nowadays statues of famous figures from all states, e.g. Rosa Parks. Once we finished the tour, he suggested to have a look into the Senate and House halls, which were up for few, since both the house and the senate were not in session. After this he guided me to the National Archives, which in fact I even didn't put on my own DC bucket list. Thanks to him I got to see the Rotunda for the Charters, where the US Constitution, the Bill of Rights, as well as the US Declaration of Indepence are put on display. After a coffee break we went on to the Botanical Gardens and the National Museum of Natural History (Diamonds and corll reefs on display), and then it was time to make it back home after a metro ride, followed by a seemingly endless bus ride to Dullas International.\\

United States Capitol*****, Botanical Gardens***, National Archive*****, National Museum of Natural History****

\section{September 10: Verbier}
\label{Verbier2018}

Glacier de Corbassiere*****, Mont Fort*****

\section{September 11: Hermence}
\label{GrandeDixence2016}

For reasons still unknown to me -- let's call it my own stupidity -- I managed to upload photos on social networks but somehow also do delete all photos of that trip, even before making a safety copy, the second time of that very same year, but so far last time this happened. Usually photos are a good aid keeping things you experienced in memory, else the danger is that many things fade away with time without constant reminders through photos.\\

Barrage de la Grande Dixence*****, Glacier de Cheillon*****

\section{September 13-- September 18: Arlington}
\label{Arlington2016}

\section{October 8--October 9: Santiago de Compostela}
\label{Santiago2016}

Cathedral*****, Pelagius Church****, Fructuosus Church***, University***, Francesco di Assisi***, Capilla San Roce***

\section{October 14--October 16: Cologne}
\label{Cologne2016}

October 14\\
Cologna: Dom*****, D\"usseldorf: Benrath Palace****, Limburg: Dom*****\\

October 15:\\
Br\"uhl: Schloss Augustusburg*****, Schloss Falkenlust****, Bonn: M\"unster****, Cologne: Dominican Church**

\section{November 25--November 27: Milano, Mantua \& Verona}
\label{Milano2016}

November 26:\\
Milano: St Maria delle Grazi with Last Supper*****, Santuario di San Bernardino alle Ossa*****, Palazzo Reale*****, Duomo*****, Mantua: Palazzo Te*****, Basilica di Sant'Andrea*****, Duomo****\\

November 27:\\
Mantua: Duomo****, Palazzo Ducale*****, Teatro Bibiena*****, Verona: Arena****, Basilica Sant'Anastasia****, Duomo****, Santa Maria Antica**

\section{December 4--December 17: Japan}
\label{Japan2016}

December 4: Sunday\\
This time starting out in Zurich, with a planned transfer in Paris Charles-de-Gaulle. Unfortunately we started out with about an hour delay, after the deicing lasted quite a bit of time. With just 25 minutes to spare for transferring I was told to just start running to switch terminals. Once I started running to the departure gate for my flight to Tokyo, I heard already the last call for my next flight. I still made it through passport control and to the gate in time, even not being the last person to be allowed on board.\\

December 5: Monday\\
I had troubles to stay awake on this 12 h flight, but flying over the Tundra of Russia and a couple of movies helped me to get through. The plane arrived so late, that the train office was closed, so no possibility to exchange a Japanese rail pass voucher on this day. Once I arrived in Japan I almost missed the fact that the train I originally got on only stopped at fast line stops, but a local resident made me aware to switch to the correct train. Anyways I finally arrived at my hotel, already quite late, greeted by Origami birds on the hotel bed.\\

December 6: Tuesday\\
Waking up before 6 am, I got on the train too early to be able to exchange a voucher for a Japanese rail pass, thus full price for the first Shinkansen north. Then checking in at the hotel, which was just a 5 min walk away from the conference centre, and on to the first session of the workshop. It did start to snow on this day, and continued to do so for the remaining days of the week. The food was really great in Morioka, particularly the Sushi.\\

December 8:\\
The conference advertised to see a bit of the surroundings, with a focus on nearby hot springs, the coast line (unfortunately not accessible by train anymore due to earthquakes destroying the connection), or the old temples of Hiraizumi. The most important artefact is the golden altar of Konjiki-do in the Chuson-ji temple complex. After passing by the remains of the Muryoko-in temple, I saw my first nice old Japanese garden with a rock monument in a little lake.\\

Hiraizumi: Chuson-ji Temple (with Golden Hall)****, Muryoko-in***, Motsuji Temple and Garden****\\

December 9: Tokyo\\
Tokyo Skytree*****, Senso-ji Temple***, Tokyo Tower****\\

December 10: Nikko\\
Shinkyo Bridge***, Rinno-ji****, Toshogu Shrine*****, Taiyuin Reibyo*****, Furarasan Shrine*\\

December 11: Kyoto\\
Kyoto: Imperial Palace and Gardens****, To-ji Temple*****, Nishi Hongwanji****, Taiyuin Reibyo****, Kosho-ji Temple***, Higashi Hongwanji***, JR Kyoto Station****, Nijo Castle*****\\

December 12;\\
Kyoto: Kiyomizu-dera  Temple*****, Ginkakuji  Temple*****, Rokuon-ji  Temple*****, Ryoan-ji Temple****, Ninnaji Temple*****, Kamigamo-Shrine, Shimogamo Shrine***, Rishoin, Daigo-Ji*****, Sambo-in***\\

December 13\\
Nara: Todai-ji Temple*****, Kasuga-taisha Shrine****, Shin-yakushi-ji Temple****, Gangoji Temple**, Kohfukuji Temple*****, Toshodaiji Temple*****, Yakusjiji Temple****\\
Ikaruga: Horyuji Temple****, Chuguji Temple***, Horinji Temple****, Hokiji Temple****\\

December 14:\\
Hiroshima: Atom bomb Dome****, Peace Park \& Museum*****, Hiroshima Castle***\\
Himeji: Castle*****, Koko-en Garden****\\

December 15:\\
Hatsukaichi: Miyajima Island - Itsukushima Shrine*****,\\
Osaka: Castle***, Umeda Sky Building*****\\

December 16:\\
Kyoto: Fushimi Inari-Taisha Shrine****, Tenryuji Shrine***\\
Uji: Byodo-in Temple*****, Ujigami Shrine*\\
Tokyo: National Museum for Western Art***, Tokyo Metropolitan Art Museum****, National Museum Tokyo*******\\

December 17: Tokyo\\
Tokyo: Imperial Palace****, Metropolitan Government Building****, Teien Palace****
\chapter{Year 2017}
\label{2017}

\section{January 2-January 5: Dublin}
\label{2017: Ireland}

Co-travellers: \\
Chris:\\
US-American, very eager to get to know different parts of the world, but at a more relaxed pace than myself. Thus perfect to hang out in a place like Ireland, where there are enough sights, but not to an overwhelming degree as e.g. in Rome or Paris. Also always up for a beer or just watching sports events in pubs.\\

January 2:\\
How we ended up in Ireland: Chris was relocating to Switzerland for quite a while and thus tried to find the cheapest one way transaltlantic flight offer around new year. Seems Air Lingus was the airline of choice. Since that means flying over Dublin, why not spending a couple of days there. Since I had been planning to go to Dublin starting all way back in 2013 it was pretty easy to convince me to join. Getting on the first flight out at 6 am flight, we met up in a nice restaurant for some shepherd's pie. Our first sight was Dublin Castle, remodelled in baroque revival style, the castle is still used by Irish president for state visits, I would advice you to check out the chapel as well. Then after a short stop at Christ Church Cathedral we spend our first evening and night in Temple Square.\\

Dublin: Dublin Castle**, Christ Church Cathedral*\\

January 3:\\
Trinity college is famous for its library and the ancient book of of Kelis, which was created in the 9th century.
Then we moved on to the national cathedral of Ireland - St. Patrick's cathedral -- after a very short stop by Dublin's City Hall with its classical rotunda. St Patrick's cathedral is built in gothic style - the largest church of the country. Next we opted for the four glass option at the Whiskeymuseum. And last but not least learnt about all details of brewing at the Guiness Storehouse. Dinner was served at a former church, transformed into a stylish restaurant, including live music and dance performances.\\

Dublin: St Mary's Pro Cathedral, Trinity College***, City Hall, St Patrick's Cathedral**, Whiskey Museum**, Guiness Storehouse*\\

January 4: of we go to Northern Ireland:\\
Today's the day when we hop on a bus taking us all to the Northern tip of the island. Our bus driver told us all about the positive changes achieved after the signing of the Good Friday agreement. The first stop were the Dark Hedges by Ballymoney, which are featured in a couple of TV series nowadays. Then the trip continued with a hike by the cliffs of the Northern Sea up to the Carrick-a-Rede Rode Bridge. Having grown up far away from any coast, I am amazed again and again by the sea. Since we had quite a bit of sunshine! we had even a good few up to Scotland. The centrepiece of the trip was though the Giant's Causeway. Unfairly called one of the most disappointing tourist attractions of Ireland by some surveys, it is actually an scenic natural sight. About 40000 basalt columns lead all up to the coastline, the remains of a volcanic eruption millions of years ago. After a nice quick lunch by the Nook Pub and a photo stop by Dunluce Castle in sunset we had a one hour layover in Belfast with its gigantic illuminated City Hall.\\

Ballymoney: Dark Hedges*\\
Ballycastle: Coastline with Carrick-a-Rede Rope Brige**\\
Bushmills: Giant's Causeway***\\
Belfast: City Hall**\\

January 5: 
Getting up really early in the morning, jumping on the Airlink express and off with Air Lingus back home. A nice trip, a relaxed start in the year, even the weather was pretty nice, so a mixture of leisure, culture, and after all even nature. Maybe not enough of cultural highlights for people who are looking for sights, which blow your mind.\\

Unfortunately this was already the end of my trips with Chris (out of two overnight trips, and one day-trip). They were always fun, including nice evenings or nights out. Chris didn't move back right away like most other co-travellers, sticking around the area until begin of 2019, but another trip or hike just never materialised, but rather afternoons or evenings watching either Premier league or Champions League matches.

\section{January 13--January 15: Carcasonne \& Avignon}
\label{2017:Provence}

January 13:\\
Avignon: City Walls\\

January 14:\\
Carcassonne: Canal Du Midi*, Cite de Carcassonne***, Chateau Comtal***, Basilique Saint-Nazaire**, St Vincent*, Cathedral* (Tower**)\\

January 15:\\
Avignon: Cathedral**, Pope's Palace***, Rocher des Doms**, Pont Saint-Benezet*, Petit Palais*, Basilique St Pierre**, Lapidarium*\\
Villeneuve-les-Avignon: Fort Saint-Andre**\\

And then it was time to retire my CANON EOS 600D which had been my faithful companion since my 2012 trip to Prague \ref{2012Prag}. We travelled through many countries, covering three continents, taking 10'000s of pictures. After all it found a new nice home at another PhD student's place later on. Hopefully you are doing fine. 

\section{February 10--February 12: Belgium}
\label{Belgium2017}

February is the month I take least photos. Now you can say well on average a February has about 28.25 days, clearly less than any other months, so what gives. While that is true, I have a significant dips in trips and travel in February too (most probably after travelling around new year/christmas I don't feel the urge to go somewhere. But then there is the odd chance I get outside of the country from time to time, for example to Brussels, taking Riju and my new camera, a CANON EOS 80D with me.\\

Co-travellers:\\
Riju likes to see more countries, I convinced him that Belgium offers nice towns and nice food and nice drinks. Since he had not been in Brussels before, he decided to spent the first day there, while I would go to Bruges first and then to Ghent, where we would meet up again.\\

February 10:\\
As usually I was on the late evening flight to Brussels, once again I booked at the quirky moroccan styled Hotel Mozart (the irony) which was just a few metres away from Grand Place. Riju and I took a couple of night photos - and yes this camera takes so much better photos at large ISO as you need without a tripod. Then we had a few drinks at the Au Brassuer by the Stadthuis and then it was time for some sleep, particularly for myself, considering I would get up about 2 h earlier than Riju (also missing out on breakfast, which was actually included in the hotel room booking).\\

Brussels: Grand Place***\\

February 11:\\
My parents had visited both Bruges and Ghent previously and both preferred Bruges. I still booked the night in Ghent, but that didn't keep me from visiting Bruges. In fact since they also weren't that fond of the flemish Begijnhofs either, I gave that a miss (OK world UNESCO heritage, but if you have no time, you gotta sacrifice something). After a short stop by the Belfry it actually stated to snow, while I rushed to the town hall to escape it. Event my little backpack was considered too large, and I had to leave it at the info desk. The Golden Room of the town hall was very nice with murals depicting important historical events of the city, as well as golden statues of important figures of the city. Next to the town hall is the Vrije, where the castellany was located in early times. The main room show some excellent wood carvings on the wall and the ceiling. The third important building of the Burg square is the basilica of the Holy Blood in neogothic style, which houses the relic of the precious blood, a cloth soaked with blood of Jesus Christ. After a short stroll to the Grote Markt with the Belfry and the Provinciaal Hof, I walked along the Rozenhoedkaai. In the church of our lady are the tombs of the last Valois Dukes of Burgundy as well as Michelangelo's sculpture of Madonna and Child, the only sculpture leaving Italy while Michelangelo was alive. The cathedral of St Salvator is a nice church, though overshadowed by the church of our lady, the tower was made taller in the 19th century to appear more like a ``real'' cathedral. Then it was time to get on the train back to Ghent. There I first visited the St Bavo's Cathedral, a gothic church with a largely baroque decoration, and one of the most precious gothic altarpieces, painted by Hubert and Jan van Eyck, the so-called Ghent Altarpiece. Next to the cathedral is the 91 m tall belfry, the tallest one in Belgium. Opposite of the cathedral is the gothic Niklaaskerk, which also houses an old 19th century organ. After the churches I went to the tourist information to obtain tickets for the two famous Hotels of Ghent, getting the last available ticket for the day. The tour stated at Hotel Clemmen at the Chinese Salon, and then continued with the Hotel d'Hane-Steenhuyse on the opposite side of the road. The main facade is on the garden size, built in a classic style. This large house had been the refuge of the French king Louis XVIII. during Napoleon I's 100 days. The main hall is the Italian Hall with a precious parquet floor as well as a giant carpet and a large frescoed ceiling, covering both floors of the house. The more important rooms are located in the upper floor of the house, but the king preferred not to climb stairs, thus a ground floor room was made into the King's room by the city. Once I finished the tour, Riju had already arrived and checked into our hotel room. Since I didn't know who would arrive first I had booked the room, but given instructions in case Riju would arrive before me, to make sure he would get the key as well. Once I arrived the lady at the reception said, it was very untypical but very accurate description so she felt she could give him the keys. Once we were ready for food, we did a couple of night photos (town hall, belfry, cathedral), but more interesting were photos of the Graslei quay as well as the Gravensteen castle. We had a couple of fried food at one of the many friteries, and then a couple of beers at the Dulle Griet pub, rumoured to have the largest collection fo Belgian beer in the city.\\

Bruges: City Hall***, Vrije***, Basilica of the Holy Blood**, Great Market**, Church of our Lady***, Cathedral St Salvator**\\
Ghent: Cathedral St Bavo with Alter of Ghent**, Niklaaskerk**, Hotel Clemmen**, Hotel d'Hane-Steenhuyse***\\

February 12:\\
Starting our day at Ghent we took a couple of photos along the Graslei before visiting the Gravensteen castle. One of the most famous castles of Belgium several arms are exhibited there, the halls and rooms are sparse as expected from medieval times, the tower top offers a nice view of old town, particularly dominated by the three towers of the cathedral, the belfry and the Niklaaskerk. And then it was time to take the train to Antwerp. Antwerp Centraal is one of the most beautiful train stations I have ever seen. The main hall was built in the 19th century, by now new platforms have been built underground mainly for the high speed trains. We had a short lunch at the train station, getting some chocolate in the ground floor of the old royal palace, the Palais op de Meir (the main rooms are on the first floor) and then we saw the cathedral. Some of the altarpieces were painted by Peter Paul Rubens, the cathedral has also nice stained glass windows, and a nice fresco in the square tower. The Grote Markt is surrounded by guildhalls, the exhibit about the city's history in the city hall was nice too. At this point Riju got himself another snack by one of the friteries, while we walked over to the old port with the small Hetsteen castle. Since we still had a bit of time, we decided to get into the Rubens House, the residence of the famed Flemish artist, who also collected a couple of paintings from his contemporaries like van Dyck, Brueghel or de Vos. The court facades and the garden were nice too. Then we stopped at the Stadtfestsaal shopping mall before getting back to Brussels airport where we had our dinner before flying home to Geneva again.\\

Ghent: Gravensteen***\\
Antwerp: Antwerp Centraal***, Palais op de Meir*, Cathedral***, Great Market**, Hetsteen**, Rubens House***, Stadtfestsaal*

\section{March 12: Switzerfrance: Lac des Brenets}
\label{2017:LacDesBrenets}

Co-travellers: \\
Sarah:\\
US-American: can you imagine, I also know non-physicists. Sarah is a science writer, who is very good in explaining to a none-science audience about particle physics research, as well as teaching scientists on how to become better communicators themselves. When Sarah needs a break from that pretty exhausting business, Sarah typically relaxes canoeing on the wild rivers around lake Geneva or climbing and hiking in the alps, or in American national parks (and why not crossing glaciers while at it). And sometimes she even manages to get not that much in shape physicists to tag along.\\

Riju:\\
Riju is not only up for city trips (see all previous trips) but also for decent hikes, as long as they don't go too wild by almost crossing altitude differences of 1 km. \\

How it all started: Saturday afternoon, sitting home alone, minding my own business, I get a message from Sarah: ``bored as well, we should do something tomorrow''. Agreeing with Sarah we check out hikes which are not that far from home (aka less than 2 h away), which still seem to be decent to do and not over the snow line (somewhere around 1000-1200 metres at that point). And we find a pretty cute hike in the Jura along side Lac des Brenets (which I did back in 2010, when I missed the boat back in work outing trip, and thus missed lunch). Said and done, now we wonder if we should take someone else on the ride, and Sarah leaves Riju no choice, but just tells him to show up at a certain time and a certain place "It will be fun". And off we go: Lac des Brenets is a small lake in the Jura along the border between France and Switzerland (Canton Neuchatel). The lake is about 3.5 km long and meanders through a narrow gorge. We decided to take the the ``difficult'' path, which is a pretty tight path (around 30-50 cm) going midway through the cliff sides for a while. At the end of the lake, we keep following the river of Doubs until we reach the end of the path by the 27 m high waterfall of Saut du Doubs. In March the water levels are pretty high. In winter the lake surface can freeze completely. Rarely the lake dries up due to the lack of rainfall providing enough water supply. Riju and Sarah dared to go closer to the waterfall edge, I had a bit more respect of the thunderous amount of water falling. And we then went back to the lake, crossing over to France and up the hill for a better panoramic view. On our way back we met a group of people, selling cake and coffee for donations - which we happily made use off. Back by the start of the lake we decided to move on to the city of Neuchatel. In between we stopped by snow covered fields to enjoy the silence for a couple of minutes. In Neuchatel we first went up the castle hill and then down to the harbour to have some Mexican food by the lake Neuchatel, and then back home. A nice hike for a slow start into the hiking season, even suited for not experienced people (but these should use the wide path).\\

Lac des Brenets***,Saut du Doubs**,Chateau de Neuchatel*\\

\section{March 24--March 26: Umbrien}
\label{2017:Umbria}

March 25:\\
Assisi: Cathedral San Rufino**, Basilica di Santa Chiara**, Chiesa Nuova*, Palazzo del Comune*, Tempio di Minerva**, Basilica Papale di San Francesco d'Assisi*** (Lower Church***, Upper Church***) , Oratorio dei Pellegrini**, Basilica Papale di Santa Maria degli Angeli* (Porziuncola**)\\

March 26:\\
Assisi: Cathedral San Rufino**\\
Perugia: Sant'Ercolano**, Basilica di San Domenico*, Basilica di San Pietro***, Chiesa del Gesu**, San Filippo Neri**, Cathedral San Lorenzo**, Capella di San Severo*, Palazzo dei Priori**, Collegio del Cambio**, Rocca Paolina***

\section{April 14--April 17: Kiev}
\label{Kiev2017}

April 15:\\
Kiev: Vladimir Cathedral**, Golden Gate*, Gorodetsky House***, Pechersk Lavra (Cave Monastery)*** (Cathedral of the Dormition***, Gate Church of the Trinity***, Refectory Chambers**, Church of the Saviour at Berestov***), Feodosiyivskyy Monastery**, Sergius of Radonesch Church**, Monastery St Michael**, Sophia's Cathedral***

April 16:\\
Kiev: Nikolai-Prytyska-Church*, Florivsky Monastery**, Mary Ascension Cathedral*, Alexander Church*, Nikolaus Cathedral**, Airplane Museum***

\section{April 28--May2: Frankfurt}
\label{Frankfurt2017}

April 28:\\
Frankfurt: Main Tower Observatory**\\

April 29:\\
Aschaffenburg: Park Sch\"onbusch and Schloss Sch\"onbusch**, Johannisburg Palace*, Pompejanum**, Frauenkirche*, Stiftsbasilika St Peter \& Alexander**\\
Wiesbaden: Stadtschloss***, Kurhaus**\\
Frankfurt: Kaiserdom St Batholom\"aus*** (with Tower***)\\

April 30:\\
Bruchsal: Palace***\\
Mannheim: Palace** (with everything open***), Jesuit Church**\\
Speyer: Dom*** (with Kaisersaal***)\\
Mainz: Dom**\\

May 1:\\
Kassel: Orangerieschloss \& Marmorbad***, Schloss Wilhelmsh\"ohe*** (Weissensteinfl\"ugel**, Bergpark***, L\"owenburg in Renovation*)

\section{May 24--May 28: Madrid}
\label{Madrid2017}

I felt it was time to see a bit more of Spain, once again choosing Madrid as base, and getting to other places from there, I booked most of my tickets well in advance to benefit from the cheap high speed train tickets. I also bought my ticket for the Royal Palace, but then the King had a meeting there on very short notice, so sadly the palace administration cancelled on me just about a week before the trip.\\

May 24:\\
Having arrived in Madrid I was very happy to see how large the room was, also with a magnificent view of the illuminated Cuatro Torres Business Area. The four skyscrapers are between 220 and 249 m high, illuminated in several colours as well. Seems my digestive system was not happy about food I had at the airport, at least I woke up several times rushing for the rest room, thus for the next two days I set myself on a diet with bananas, pretzel sticks, and coke and water. That was not the way I wanted to start my holidays, but you sometimes get what you don't like.

May 25:\\
Salamanca: University**, Cathedral***, Old Cathedral***, Convento de San Esteban***, Palacio de la Salina**, Palacio de Anaya*\\

May 26:\\
Valencia: City Hall**, Basilica of the Virgin of the Foresaken (Virgen de los desamparados)**, Cathedral*** (Holy Grail's Chapel***, Treasury***), Sant Nicolau***, Almudin*, Banos del Almirante*, Palacio del Marques de Dos Aguas**, Thomas \& Philipp Neri Church*, Palau de Cervello*, Mercado Central**, Llotja de la Seda***, Science Museum*** (the actual Museum exhibit*), Hemisferic***, Palau de les Arts Reina Sofia**, Church Virgin Of Monteolivete*\\

May 27:\\
Leon: Cathedral***, Basilica de San Isidro** (Cloister \& Pantheon of the Kings***), San Martin*\\se
Oviedo: La Foncalada*, Cathedral*** (Camara Santa and Treasury***), Balesquida Chapel*\\

May 28:\\
Avila: Convento de San Jose**, City Walls***, Cathedral***, Santo Tome*, San Vicente***, San Juan Bautista*, Capilla de las Nieves*, Real Monasterio de Santo Tomas**

\section{June 3--June 6: Athens}
\label{2017Athens}

Having not been in Greece for almost 20 years, Athens had been on my list of places to revisit for quite a while. \\

Co-travellers:\\
Riju: Since Riju appreciates ancient architecture he was eager to join, moreover since his friend MaryAnn was based in Athens at that time.\\
MaryAnn: Riju's friend, who works on her PhD in archaeology in Greece. Since she is based in Athens the was giving us all expert insights spending most of her weekend with us.\\
Amin: Having spent already a long weekend in Italy, Amin is ready for another trip and wants to see more of Europe.\\

June 3:\\
We booked a hotel close to the National Archaeology Museum a couple of months before our trip. But just about a months before our trip horrendous reviews of our original hotel appeared, linked to an ongoing issue with bedbugs. I didn't want to risk anything and found a good hotel quite close to old town for an affordable price. So I convinced everybody else that we indeed should stay at this other hotel. And indeed this hotel turned out to be just fine. After our long metro ride, getting a 72h public transport ticket, we arrived in Athens downtown, where Riju and MaryAnn set up our meeting by the Acropolis Museum. On our way to the museum we stopped by the Kapnikarea church, and old medieval church with nice frescoes. Unlike in 1999 photos were not allowed anymore. The cathedral of Athens was built after the city was elevated to the capital of Greece. Athens had been a small town around that time but rapidly grew into the metropolis it is nowadays. The cathedral is a large greek-orthodox church with lots of Icons and mosaics. We passed the monument of Lysikrates, Amin got himself a quick snack, and we met MaryAnn who took us to the Acropolis Museum. Many of the original statues have been placed in this museum in order to protect them from air pollution, rain, and wind, including five of the statues of the Erechtheion (the 6th statue was taken by the British and is now exhibited in the British Museum in London). MaryAnn gave us a tour of the museum outlining the evolution of greek ancient style and how certain artefacts can be seen in context of political issues and struggles as well. Even without the Elgin marbles of the Parthenon in place (instead one sees photographs of the missing pieces) the remaining sculptures of the frieze are still amazing and interesting to see. The development of greek sculpturing is explained as well using artefacts found originating from different times. The view of the acropolis from the museum is quite a sight too. Then MaryAnn took us to one of her favourite close-by restaurants before we returned to the Odeon of Herodes Atticus for a performance of Madame Butterfly by the Greek National Opera. Then we ended the day on a roof-terrace by Monastiraki with night views of the town having some cocktails.\\

Athens: Cathedral**, Acropolis Museum***, Madame Butterfly Performance in Odeon of Herodes Atticus***\\

June 4:\\
We met MaryAnn once more in the morning for our tour of the Acropolis, where she outlined the history of the structures and temples, as well as excavation campaigns of the surroundings and the development of the different buildings through time. Since we started the day early, the Acropolis was quite empty, only one hour later substantially more people entered, by that time we left and went down from the Acropolis to the Agora. From medieval time the church of the apostles close to the Agora offers nice frescoes and mosaics, by then Athens was a little bit more than a small village. Besides the museum in the reconstructed stoa, the almost perfectly conserved temple of Hephaistos is the dominating remaining building of the Agore. Then we walked over to the old Roman Agore with the library of Hadrian. During the Roman times the large temple of the Olympic Zeus was constructed, one of the highlights of the corinthian style. Closeby is the Panathinaiko stadium, the site of the first modern olympic games, it had been also used during the 2004 olympics in Athens once more. Then we had time for a long lunch, where MaryAnn ordered a lot of local specialities. We also felt like staying longer, particularly after a large rain storm appeared and flooded the whole road, thankfully the restaurant had plastic planes all over the place, although the feet of the tables and chairs were soon flooded by quite a bit. After 10-15 minutes everything started to clear up though again. Then we took the bus to the Greek National Archaeology Museum. Full of artefacts from ancient greek times, from Mykene, Tiryns, Athens as well, and Thera. Amazing to visit and see it again. Then we had a last big dinner and drinks on the roof terrace once again.\\

Athens: Acropolis***, Agora** (Church of Apostles**, Hephaistos Temple**), Hadrian's Library**, Roman Agora*, Temple of Olympic Zeus**, National Archaeology Museum***\\

June 5:\\
We took the local bus to get to the long distance bus station and bought our tickets for Delphi. It is a ruined city nicely located surrounded by hills. We walked through the upper part of the city with the roman Agora and the temple of Apollo, the theatre and the stadium. Back in 1999 one could still walked through the stadium, nowadays one can observe it from a hill side, but otherwise it is off limits. Then we walked over to the remains of the lower city with the Tholos, a round temple in doric order. In the local museum, friezes of old treasure houses, decorations and statues of the temple of Apollo, and several monuments are exhibited. Then we had a long lunch trying out many things, before taking the bus back to Athens.\\

Delphi: Excavations*** (Temple of Apollo**, Stadium***, Theatre**, Tholos***), Museum***\\

June 6:\\
We took the metro early in the morning, and back in Geneva we even had to go through security, although we moved from Schengen to Schengen they asked many questions, I never found out why.

\section{June 9--June 23: USA: Yellowstone}
\label{US2017}

June 9:\\
Great Salt Lake**\\

June 10:\\
Twin Falls: Shoshone Falls***, Snake River Canyon**\\
Bruneau: Dunes State Park***\\
Bliss: Snake River Canyon*\\
Salmon Springs: Thousand Springs Reserve**\\

June 11:\\
Arco: Craters of the Moon***\\

June 12:\\
Yellowstone National Park: Madison River**, Firehole Canyon**, Fountain Paint Pots**, Upper Geyser Basin with Old Faithful Geyser***\\

June 13:\\
Yellowstone National Park: Gibbon Falls**, Norris Geyser Bassin***, Mammoth Hot Springs***, Beryl Spring*\\

June 14:\\
Yellowstone National Park: Madison River**, Yellowstone Canyon*** (Lower Falls***, Upper Falls***, Cascade Falls**), Tower Falls**, Petrified Tree*\\

June 15:\\
Yellowstone National Park: Mud Volcano Area**, Lake Village*, Natural Bridge**, West Thumb Geyser Basin***, Kepler Cascades**, Upper Geyser Basin with Old Faithful***, Firehole Lake Drive**\\

June 16:\\
Ashton: Upper Mesa Falls***, Lower Mesa Falls**, Warm River Walk**\\

June 17:\\
Grand Teton National Park: Phelps Lake Hike**\\

June 18:\\
Grand Teton National Park: Leigh Lake**, Jenny Lake Hike***, Cascade Creek and Hidden Falls***\\

June 19:\\
Grand Teton National Park: Jackson Lake \& Jackson Lake Dam***, National Museum of Wildlife\\

June 20:\\
Afton: Bridger National Forrest hike to Periodic Spring**\\

June 21:\\
Salt Lake City: Salt Lake City Library**, Cathedral St Madelaine**, Beehive House**, Joseph Smith Memorial Building*, Assembly Hall, Utah State Capitol***\\

June 22:\\
Magna: Great Salt Lake**\\
Salt Lake City: Hogle Zoo***

\section{July 5--July 16: Venice: EPS2017}
\label{Venice2017}

Having be selected for the European Physics Society high energy physics conference in 2017, I decided to spent a couple of more days in Italy, this time around Trieste. Reyer wanted to join for sure, Nate thought he might be able to join, but unfortunately it was Reyer and I alone after all.\\

Co-travellers:\\
Reyer: Not having been around many places so far, Reyer was all game to rent a car to get to Slovenia and Croatia afterwards too.\\

July 6: Venice\\
After the first day of the conference I took a boat along most of the Canal Grande, getting a Spritz and a nice dinner with my colleagues, then we walked along some canals until after sunset and enjoyed some night views of the canals.\\

Venice: Canal Grande***\\

July 7: Venice\\
The conference took place in the Palazzo del Casino on the island of Lido di Venezia. The rooms were decorated with lots of impressive mosaics. The building was constructed in the 1930s, nowadays the Casino moved to a Palazzo by the Canal Grande and the palazzo is used as congress centre.\\

Venice: Lido - Palazzo del Casino**\\

July 8: Venice\\
Venice: Collezione Peggy Guggenheim**, Santa Maria della Salute**, Palazzo Reale**, Biblioteca Nazionale Marciana***, Palazzo Ducale***\\

July 9: Loreto \& Ancona\\
Loreto: Basilica of the Holy House***\\
Ancona: San Francesco delle Scale, Palazzo Ferretti - National Museum of Marche*, Duomo**\\

July 10: Venice\\
Venice: Jesuit Church***, Murano - San Pietro Martire**, Murano - Santa Maria e San Donato***, San Giovanni Crisostomo*, San Silvestro**, Basilica Santa Maria Gloriosa dei Frari***, Ca' Rezzonico***, Basilica di San Marco*** (mit Pala d'Oro***), Lido - Palazzo del Cinema, Lido -Palazzo del Casino**\\

July 11: Venice\\
Venice: Lido - Hotel Excelsior**\\

Up to now unfortunately this was the last time Indara and I had a trip experiences together. Also travelling with Indara is more chaotic than with other people, that does add certain spice and fun too, when things have to be re-arranged and changed on the fly. After all, it always worked out in that sense, that I always got to see and experience what I wanted to, together with a fun, interesting, and charming travel company.\\

July 12: Vicenza \& Stra\\
Vicenza: Duomo**, Villa La Rotonda***, Villa Valmarana ai Nanai***, Basilica di Santa Maria di Monte Berico**, Piazza della Biade**, San Gaetano*, Santa Corona, Teatro Olimpico***, Santa Corona**, Palazzo Barbaran da Porto - Museo Palladio**\\
Stra: Villa Pisani***\\

July 13: Ferrara \& Padua\\
Ferrara: Palazzo Comunale**, Cathedral*, Basilica di San Francesco*, San Girolamo, Palazzina Marfisa d?Este***, Basilica Santa Maria in Vado**, Palazzo Castabili***, Palazzo Schifanoia***, Biblioteca Ariostea*, Castello Estense***\\
Padua: Basilica di Santa Guistina*, Botanical Garden**, Cappella degli Scrovegni***\\

July 14: Skocjan Caves \& Porec\\
Divaca: Skocjan Caves***\\
Porec: Euphrasian Basilica***\\

July 15: Aquileia, Cividale del Friuli \& Trieste\\
Aquileia: Basilica*** (Crypt***) , Battistero*\\
Cividale del Friuli: Cathedral**, Museo Cristiano**, Santa Maria in Valle \& Tempietto Langobardo***, Archeological Museum***\\
Trieste: Miramare Castle**\\

July 16: Grotta Gigante\\
Sgonico: Grotta Gigante***\\
Trieste: Cathedral**

\section{July 21--July 23: Oslo}
\label{Oslo2017}

July 21:\\
Dom***\\

July 22:\\
Royal Palace**, City Hall***, Akershus Fortress**, Opera*, National Gallery**, Vigelandpark***, Dom**\\

July 23:\\
Oskarshall**, Viking Ships***, Fram**, Kon-Tiki Museum*, Norwegian Museum of Cultural History**, Norwegian Museum of Cultural History***

\section{July 28--July 30: Munich}
\label{Munich2017}

A sad day in history: saying good-bye to Eric and Siyi. While I was in Venice, I talked to Siyi about her impending move back to the US. She told me on one hand they are happy since it will mean more life stability for both of them due to her new tenure position in Hawaii, but on the other hand it means leaving all friends behind. She then suggested that I could technically come over for their last weekend in Munich (should I want to). Since I could stay at their place, I only had to check for flights, and since those didn't seem to be outrageously expensive, I booked right away.\\

June 29:\\
Munich: St Peter**, Old City Hall*, New City Hall**, Frauenkirche*, Michaelskirche**, B\"urgersaalkirche**, St Anna**, Theatinerkirche**, St Ludwig*, Lenbachhaus***\\

June 30:\\
Munich: Schloss Nymphenburg*** (without park palaces this time), Villa Stuck**, Pinakothek der Moderne***, Neue Pinakothek**, Alte Pinakothek*, Glyptothek**, Antikensammlung**\\

Up to this day, this marks the last time I saw Eric in person, besides working together for about four years at CMS, we also did five trips together involving overnight stays and three additional day-trips or hikes. Siyi and him moved all the way to the other side of Earth, to the big island of Hawaii, where Siyi continues here Astronomy research and Eric moved on to the industry world.\\

\section{August 26: Glacier 3000}
\label{lesdiablerets2017}

Walking on a glacier sounds amazing, and this is exactly what you can do by the Tsanfleuron glacier (particularly since it doesn't have any huge crevasses). Originally I wanted to do this trip already a week earlier with Riju, but then he wasn't around, told me he overslept. I got my ticket on the day and that one would be valid for over a week, and i just would need to use it a week later. In any case a cancellation of the train by SBB facilitated the decision too, since the next one would arrive about 2 h later. Anyway one week later, August 26, the day I HAD to use this ticket, Riju called in sick, so once again I faced the fact that I had to do this trip on my own -- at least the weather was beautiful. The glacier of Tsanfleuron is a dying glacier, even in summer almost nothing is accumulated, in a couple of decades nothing will be left. Still a nice adventure to walk across all of the glacier. On the way it becomes obvious how huge amounts of ice melt on a normal summer day. Once I arrived at the hut by tour St Martin I did enjoy the views of the surrounding mountain peaks, glaciers and valleys, while having a cheese platter, and then I walked back over the glacier, taking the cable car from the Glacier 3000 station down to Les Diablerets.

Glacier de Tsanfleuron***

\section{August 27--August 28: Split}
\label{split2017}

August 27:\\
Trogir: Cathedral of St Lorenz***, Fortifications**\\
Split: Diocletian's Palace***, Cathedral (Mausoleum)***, Temple of Jupiter**, Franciscan Church, Kastelet**, Ivan Mestrovic Gallery**, Great Papalic Palace*

\section{September 6- September 11: The mother of all trips -- Rome}
\label{2017:Rome}

Co-travellers:\\
Riju:\\
US-American: by now Riju knows what expects him while travelling with me, particularly when going to my most favourite country -- Italy. And Rome is the city with by far largest number of things to do. Even after an warning, Riju is still up for all day walking through Rome. An avid photographer himself, Rome might prove to be a paradise for Riju as well.\\

Amin:\\
Iranian: After two previous travels Amin also knows what he is getting himself into. But Amin was chosen as an instructor at a school in Bari, thus joining us a day later, so we still keep the highlights for the last days.\\

How we ended up here:\\
Rome is ALWAYS worth a trip. In fact this is my third trip to Rome. From the Roman ruins to the huge amount of churches, particularly from Baroque times, Palazzos of Rome aristocracy. The centre of Christianity with the vast Papal collections in the Vatican (as well as in their private palace). Technically it isn't even that long to get from Rome to Florence, Pisa or Naples either, so if you should want to spend two weeks in Italy plan at least 3-4 days in Rome, but it could also serve as starting point of a trip through all of Italy. Last time I visited, the Domus Aurea was closed for renovation. I realised it opened up again for visits, thus one more reason to go to Rome again. Since Riju and Amin didn't see Rome yet, it was easy to convince them to join.\\

September 6:\\
Flying over the alps with the ice-covered Mont Blanc in full sight. Our hotel is just a few steps away from the Spanish Steps: Based on tripadvisor's rating \#1175 out of 1277 hotels in Rome (significantly better on booking where we reserved the room). And indeed it was a fine room, not really modern but comfortable enough, with working wifi and working air conditioning, and a clean bathroom. So what more would you need, but still pretty relieved. Then we went over to see the Trevi Fountain by night - magical (though be advised that even close to midnight, you want be the only one).\\

Trevi Fountain***\\

September 7: The smallest country of the world:\\
One of the advantages of getting early in Rome are free views of typically crowded places: in our case the Pizza di Spagna with the Spanish Stairs. We were pretty much alone at 7 am. Once we arrived at St Peter's we went up to the Dome with really nice views over Rome in early sunshine. And then the overwhelming gigantic huge basilica of St Peter's: so much to see here, dozens of monuments, Michelangelo's Pieta, Bernini's Cathedra Petri and Bernini's Baldachin among others. Since we had timed tickets for the Vatican Museum I pressured Riju into leaving. He complained we hardly had spent any time in St Peter's - exif data showed that we had spent though shortly less than an hour inside. So maybe plan rather around 90 minutes for St Peter's. The Vatican Museums cover several topics: a collection of Roman and Greek artefacts, a vast collection of baroque paintings, cars and carriages of the Pope's, tapestries, collections of books - and last but not least the former apartments of the Popes themselves. The rooms of Raffael are  a suite of reception rooms famous for Raphael frescoes, the most notable is the School of Athens. Always the highlight of a visit to the Vatican together with the Sistine's Chapel. The Sistine's chapel is the place where the cardinal's elect the Pope, the whole ceilings and paintings are fully covered in paintings by Michelangelo, you might have seen excerpts of the ceiling, e.g. the creation of Adam (BTW no photos in the Sistine's Chapel). Continuing with our topic of the day in the Castel Sant'Angelo. Constructed as mausoleum for the Roman emperor Hadrian between 123-139 (in fact you still can see the burial chamber), the structure was converted into the papal residence in the 16th century, the rich apartment was set up to ensure the pope could still live in lavage even in case of a future siege. And then came the let down of this day -- the Ara Pacis Augustae, the Roman altar dedicated to peace, consecrated at 9 BC. Parts of the frieze are now scattered all over the world, but most parts of the altar are still here in Rome. The problem is -- the altar is in a glass building -- AND costs 10.5 EUR (for a 10-15 minute visit). That is just way too expensive for what you get to see, consider that the Colosseum together with the Forum Romanum and the Palatine hill is e.g 12 EUR only, the baths of Caracalla with 8 EUR even cheaper. Rome has so many churches, you easily step inside any random church, and you might be blown away. This has the disadvantage that you just might be overwhelmed at some point and not appreciate what you see anymore. One example of a church which might have deserved a bit more attention: Basilica dei Santi Ambrogio e Carlo al Corso, a fine baroque church. By then we really needed to grab some food, thus we had a mountain of ice cream by Piazza de Popolo. After paying a visit to Santa Maria del Popolo, we walked up the Pincio and through the Villa Borghese to the Casino Nobile of Galleria Borghese. I have been there twice, and on both occasions all tickets for the day were fully booked, even just after opening. Unlike a couple of years ago photography was allowed (without flash). The highlights of the museums are the many sculptures by Bernini, collected by the Borghese family. Then we walked over to the Baths of Diocletian, which Michelangelo transformed into a large church.\\

Spanish Steps**,St Peter's Basilica***,Vatican Museum***,Castel Sant'Angelo***,Ara Pacis*,Santi Ambrogio e Carlo al Corso**,San Giacomo in Augusta*,Chiesa di Gesu e Maria,Santa Maria del Popolo*,Galleria Borghese***,Santa Maria degli Angeli e dei Martiri**, Fountain of Moses*,Santi Ildefonso e Tommaso da Villanova,Santa Trinita dei Monti**\\

September 8: Amin joins us and paying a visit at the President's palace:\\
Riju and I started the day on our own in the Basilica of Santa Maria Maggiore - one of the big four (the other being Laterano, St Peter's, St Paul's), or big 6 if we extend the list by San Sebastiano and San Lorenzo. The church has been modernised in baroque times, particularly two side chapels have been constructed. The Borghese Chapel is covered in gold, while the Sistine's Chapel (familiar names, aren't they) is covered in frescoes and statues. The main nave and aps of the large church are covered in mosaics, a majority of those as old as the 5th century. Closeby is Santa Prassede, the chapel is once again famous for its old mosaics ( this time). Then we stood in front of the closed gates of Santa Pudenzia. This church is dedicated to the sister of Prassede, once again containing a really old mosaic. Riju told me not to be that German and just accept that in Italy some churches sometimes happen to open a couple of minutes late (more than 15 minutes late in the end). After a short stop by Sant'Andrea a Quirinale* and the rooms of the holy Stanislaus Kostka we arrived by the gates of the Palazzo del Quirinale. Quirinale used to be the palace where the Pope's spend most of their time, particularly in summer times. Later Napoleon used the complex as his residence in Rome, then it was the seat of the Italian kings, and nowadays the palace serves as seat of the Italian president. While waiting we saw a couple of cars being escorted out by lots of police. And the palace once again didn't disappoint. Although Quirinale is hardly listed as the sight to see in Rome, I would definitely advice you to put it into your itinerary and book in advance (tickets are very cheap). The webpage of the Italian president allows you to see all rooms in a virtual tour, so check out if it is something for you. This time three of the larger rooms have been closed for renovation, including the large palatine chapel, still we saw 28 rooms on the main tour, and later the attached museum of the constitution of the Italian Republic from 1947. By now Amin had arrived in Rome, thus after a short stop in the tiny cut church of San Carlo alle Quattro Fontane (as the name says, check ouf the four fountains at that corner) we all three met up by the Lateran. Giovanni in Laterano is in fact the cathedral of Rome and thus the seat of the bishop of Rome. The church is built in high Baroque, filled with the statues of Apostles in the main nave. In between we went to the Baptistery, which is in a separate building (as often the case in Italy). Close-by are the holy stairs, which are according to the legend the stairs Jesus went up to face Pilate in his trial. Helena, the mother of emperor Constantine brought them from Jerusalem to Rome. Then we went back to the city centre, where we checked out the church of Sant'Ignazio with its large ceiling fresco, continuing with the Pantheon. The Pantheon was built around the 120s in ancient Rome, one of the best preserved buildings of the ancient Rome. The next church on our list was Santa Maria sopra Minerva, the only gothic church in Rome. Our next booking was the Palazzo Farnese, one of the most important Renaissance palaces existing, nowadays housing the French embassy, where we were supposed to listen to a 45 minute long tour in French. We were told that we would be accompanied by a film crew, which was shooting part of a documentary. Nothing wrong with that, what we didn't expect was that our tour guide would not only talk a bit about the few rooms (e.g. the room of Hercules with a nowadays replica of the Farnese Hercules and the Carracci gallery), instead she talked about 100 minutes non stop to the film crew, while we at some point lost track of what she said. Unlike last time, maybe because of the film shooting, the windows of the Carracci gallery had been opened (still no photos allowed). Sant' Andrea della Valle and the ancient forum of Largo di Torre Argentina were brief stops on our way to Il Gesu, the first Baroque church of the world, containing the tomb of Ignazio di Loyola, the founder of the Jesuit order. At that point Riju and Amin opted out and had a short stop at a Cafe, whereas I continued to see the private apartment of Ignazio di Loyola, the church of San Marco and then the Palazzo Venezia, which had been the seat of Mussolini during the fascist times in Italy. Our last point of this very long day was the Capitoline Museum. It houses the foundations of the temple of Jupiter, the statue of Mark Aurel and fantastic views of the Forum Romanum. And then we had an evening stroll along the imperial forums before having another Pasta close to the Trevi fountain. \\

Santa Maria Maggiore***, Santa Prassede**, Santa Pudenzia*, Sant Andrea al Quirinale*,  Palazzo Quirinale***, San Carlo alle Quattro Fontane*,  San Giovanni in Laterano***,  Battisterio Laterano**, Sancta Sanctorum**,  Sant'Ignazio**,  Pantheon***,  Santa Maria sopra Minerva**,  Palazzo Farnese**,  Sant'Andrea della Valle**,  Largo di Torre Argentina*, Il Gesu***,  Stanze di Ignazio di Loyola*,  San Marco**,  Palazzo Venezia**,  Capitoline Museum***, Imperial Forums** (no need to pay for getting closer),  Trajan's Column***\\

September 9:
Starting the day very very early today, thus checking out some of the many tiny churches of Rome before getting Amin to the Trevi Fountain -- seems early morning gives you emptier views than late evening/night. Riju planned to see the Palatine hill, the Colloseum and Forum Romanum today, thus we decided to meet maybe late afternoon. By now very few people were actually around, thus we stepped into the Pantheon again to see it free of large groups of tourists. Same for the fountains of Piazza Navona and the beautiful church of Sant'Agnese in Agone. Then we paid a visit to Santa Maria dell'Anima, the church of German pilgrims, with the tomb of the last German Pope (before Benedict XVI took office). The church of Santa Maria della Pace contains a couple of frescoes by Raffael. After a short stop on the church on the Tiber Island we crossed over to the other side of Rome, Trastevere. This time I also have seen the crypt of the Basilica di Santa Cecilia, which had been closed last time i had been in the church. After seeing Santa Maria in Trastevere with its 12th and 13th century mosaics Amin preferred to move on to St Peter's and the Vatican museum, thus I spend most of the remaining day by myself. After a short stop by Santa Maria della Scala, I saw the part of the Italian Gallery in Palazzo Corsini. Next I crossed back over the Tiber and saw the famous Galleria Borromi of Palazzo Spada. This corridor is actually pretty short, but perspective makes it appear pretty long. After a short stop at the inner courtyard of Palazzo Mattei with very nice friezes all over. A bit disappointing were the remains of the Stadium of Domitian, hardly anything is left, but could have been skipped. I always like walking up the Basilica Santa Maria in Aracoeli, typically it would have been more natural to do it with the other parts of the Capitoline hill. Now I asked myself what I should do next, churches are closed over lunch time, so I would have some leisure time. Seems I just was done a bit too quickly. Thus I decided that it would be nice to see the Forum Romanum and the Colosseum once more. Don't get me wrong, those are very interesting based on history and also from an artistic point of view. The queues are usually acceptable too, particularly if you don't do the Colosseum first, but the Forum Romanum instead (coming from the Imperial forums, the queue was basically non existent. After the Forum Romanum I went to San Celemente first though, a church which has three different layers of history. The church itself built in the 11th and 12th century with lots of mosaics. One floor lower you can see the early christian church, built in the 4th century, left with early frescoes. The third level are rooms from Roman times (no photos in the whole complex), and then off to the Colosseum. Then we all met to have a walk on the park next to the Colosseum with all the remains of Trajans' Baths. After dinner we enjoyed the night views and illumination of St Peters, the Lateran, Colosseum and the Forum Romanum. Oh and another anecdote - the evening before we were taking the metro home when this group of teenage girls appeared, dancing singing, just behaving unnaturally different. We all thought - maybe just crazy teenagers, but then Riju mentioned he felt one of them might have tried to get into his pockets - or just brushed past. Well guess what on this very late evening, they did appear again, more strangely this time they decided to just sit next to us - in an otherwise totally empty car - strange isn't it. Well this time singing again I felt a hand coming up to my pocket and I just slapped it hard. A loud scream and a station further ahead all of them stormed out. So teenage girls -- never try to pickpocket the same group of tourists twice.\\

Sant'Andrea delle Fratte*,  San Silvestro in Capite*,  Santi Claudio e Andrea dei Borgognoni,  Santa Maria in Via*,  Mark-Aurel-Column**,  Santa Maria in Aquiro,  Sant Eustachio in Campo Marzi,  Piazza Navona***,  San'tAgnese in Agone**,  Santa Maria dell'Anima**,  Santa Maria della Pace*,  San Bartolomeo all'Isola*,  Santa Cecilia in Trastevere**,  Santa Maria in Trastevere***,  Santa Maria della Scala*,  Palazzo Corsini*,  Palazzo Spada**,  Palazzo Mattei**,  Stadio di Domitiano,  Santa Maria in Aracoeli**,  Santi Luca \& Martino*,  Joseph dei Falegnami,  Forum Romanum**,  Santi Cosma \& Damiano*,  Santi Quattro Coronati*,  San Clemente**,  Colosseum***,  Baths of Trajan*\\

September 10: Thunderstorms\\
Summer 2017 was a very dry one, even for Rome. And then on September 10 came along: continuous Thunderstorms and heavy rain for hours. So we put on our biggest shoes, armed with Umbrellas and off we go: once we got out of the Metro by the Colosseum we realised that was not enough. Rome's canalisation isn't suited for that much of rain. Water was standing on the roads already, walking up to Domus Aurea little rivers were flowing down the roads. But we got there and were soon armed with hard hats and ready to go. Domus Aurea are the remains of Nero's famous Golden House. Only a small part of the former palace still exists, and only because the Baths of Trajan were built on top of it. The remains suffer a lot due to water damaging the structure after heavy rain falls. Thus a large restoration campaign is going on to save the remains for future generations. On weekends the palace ruins can be visited, though only the parts were construction has been finished. Even centuries later you can feel the splendour and the lavish life of the Roman emperors. At one point Augmented reality allows you to explore the gardens of the palace. Once we reached the banqueting hall, lights went out and more and more rain water started to flow by. Thus we went back to the entrance, and by now the roads of Rome were flooded, our feet for soaking wet. Buses stopped running, the metro had been closed down. Colosseum and the Forum Romanum were closed for the day as well (seem thunderstorms are something Rome is not prepared for. I decided that whatever I would do, i would get soaking wet instantly anyway, Riju and Amin decided to go back to the hotel and rather wait out until the rain might slow down. I walked though the water on the streets down to Palazzo Doria Pamphilj, which also had been closed until things cleared up. Thus I started to walk up to our hotel, but then I found out that the Palazzo Barberini was still open. Palazzo Barberini is part of the Italian National gallery, the highlight is though the gigantic ceiling fresco of the large saloon, which needed along around six years to be finished. By now the rain calmed down, so I passed by our hotel to change (a good feeling to get out of the wet socks, and of to further adventures, notably Palazzo Altemps, which houses part of the National Roman Museum with a vast collection of ancient sculptures. The Museum of the City of Rome is housed in Palazzo Braschi by Piazza Navona. Although the palace was built by the 19th century only, it has already seen quite a bit of history, it was used as ministry of interior, before Mussolini used it as headquarter of his party. By now the Palazzo Doria Pamphilj had opened up again. The palazzo is still used as residence by the family, but their vast collection as well as the state rooms of the palace can be visited. Since it is a private gallery, the paintings are put up in a less sterile way than in most museums. I also loved the tour of the state rooms. By now photography in the palace is allowed. Since the ticket of Palazzo Altemps covers the entrance for all buildings of the National Roman Museum, I went to the Crypta Balbi, which exhibits coins, sculptures and frescoes and paintings of early medieval times. And last but not least I visited the Capuchin Crypt, which is made up of six chapels, decorated with the bones of diseased monks (no Photos).\\

Domus Aurea***, Palazzo Barberini**, Palazzo Altemps**, Palazzo Braschi*, Palazzo Doria Pamphilj***, Santa Maria in Via Lata, Crypta Balbi*, Capuchin Crypt**\\

September 11: Flight Home\\
Time to say Goodbye. Getting up around 4 am to be on the first flight back home. Lots of clouds along our way, until we reach the alps, where things clear up: Mountains and glaciers over glaciers, passing Mont Blanc another time and back home. And Rome is always mind blowing, it might be even too overwhelming. Since traffic is pretty chaotic, buses slow and metro lines rather at the edge of the old city, most sights have to be done by foot. Expect people to be almost around everywhere. In case you should not appreciate doing many items on one day, then restrict yourself to just a handful instead.


\section{September 15--September 18: London}
\label{London2017}

September 16:\\
Houses of Parliament (Palace of Westminster)***, Westminster Abbey***, Foreign \& Commenwealth Office***, Marlborough House**, The Royal Society**, St Martin-in-the-Fields**, King's College**, National Gallery***, Queen's Theatre: Les Miserables***\\

September 17:\\
Buckingham Palace***, Lancaster House***, Canada House**, National Gallery**, Tate Modern***, The View from the Shard***

\section{September 20: Rolling Stones Concert}
\label{2017RollingStones}

Rolling Stones Concert***

\section{September 30--September 31: Graz}
\label{Graz2017}

Having been in Austria a couple of times, including Vienna, Salzburg, Innsbruck and Bregenz, it is time to visit another State capital, this time Graz, the capital of Steiermark:\\

September 30:\\
Graz: City Church**, Landhaus**, Dom**, Katharinenkirche (Mausoleum of Emperor Ferdinand II)***, Mariahelferkirche**, Schloss Eggenberg***, Basilica Mariatrost***, Murinsel**\\

September 31:\\
Graz: Palais Herberstein**, Franziskanerkirche*, Dreifaltigkeitskirche*, Mariahelferkirche**, Schloss**, Barmherzigenkirche**, Kunsthaus**

\section{October 6--October 9: Estonia}
\label{Estonia2017}

How we ended up here: Rachel decided that she wanted to celebrate her birthday somewhere else, preferably in one Scandinavian country. She suggested Finland, but I convinced here that we could go to Estonia instead, where drinks and food would be cheaper, with large Vegetarian options and also a good selection of sorts of beers. And we can do a trip across the baltic sea over to Finland to see Helsinki too.\\

October 6:\\
Tallinn: City Hall*\\

October 7:\\
Helsinki: Suomenlinna***, Uspenski Cathedral**, Dom**\\

October 8:\\
Olaf's Church* (with Tower**), City Wall**, Kadriorg Palace**, Alexander Nevsky Cathedral*, Dom**, Nikolai Church**

\section{October 15: Zermatt}
\label{Zermatt2017}

Gornergrat***

\section{November 4: St Gallen, Reichenau \& Konstanz}
\label{StGallen2017}

Why this trip:\\
Reyer had never been to Germany, so I wondered what could be done close-by. Both Konstanz and Freiburg im Breisgau came to my mind. St Gallen is also a very nice place to see, also on the Eastern part of Switzerland for once, and then maybe going to the monastery island of Reichenau. But then Reyer had to stay up the night before and unfortunately didn't make it to this trip. Since I had bought the day ticket though, I just did this trip on my own.\\

After Reyer told me he couldn't join, I just sat on the train and napped for almost all of the trip. The cathedral of St Gallen used to be the church of the large abbey. It is one of the largest late baroque abbeys in central europe, built by Peter Thumb. Everything is vastly decorated with stuccos, statues, altars and large veiling frescoes, particularly the one of the Rotunda. The choir stalls are a masterpiece of wood-carving, some parts are gilded in gold. The high altar and most of the choir decoration is already heavily influenced by the classicist era. The library of the former abbey is the last large monastery library built in northern Europe. An amazing masterpiece full of rich decoration, it also houses precious manuscripts, a mummy, as well as a famous globe. No photography is permitted unfortunately. After having a little snack of a St Galler Bratwurst I took the train to Konstanz, from there I continued with another train and bus over to the monastery island of Reichenau. Just after crossing the dam you reach the first church - St Georg. The whole walls are decorated with frescoes from the 11th century. The decoration of the choir was lost unfortunately, the entrance hall and the Westwerk still contain some faded remaining parts of the original decoration. In summer the tourist flow is heavily regulated, only allowing two tours inside the church. In winter season far less people visit the island and you are free to visit when you feel like it. The largest church belongs to the main monastery. A big romanesque church with the typical sparse decoration, besides the choir, which was remodelled in gothic times, with a nice altarpiece and a few frescoes. Unfortunately you only get close to those if you take a tour of the church. Walking for another 20 minutes you reach the last church St Peter \& Paul. The apsis is decorated with an old romanesque mural, while the nave of the church has been vastly redecorated in Baroque times. Nice to see how people believed they need to go with time, and what contrast that might create. Then i rushed to the bus, when i started running and gentleman asked me if he should just take me along for a bit, so I for sure would make that bus, and I accepted that offer being really thankful. Since I made it now to an earlier bus, I had still a bit of time to go to the harbour front with the quite odd Imperia statue and the council hall, and then I went on to see the Minster of Constance, the former cathedral of the once largest diocese in late medieval times. Even the famous council of constance took place in the minster where the catholic church tried to solve the schism which was created after three popes had been elected at the same time (solved by forcing all of them to step down and electing a new pope). The Minster was built in Romanesque style, but the facades had been remodelled during gothic times. The inner decoration is mainly from classicist times, but murals from former times survived in a few of the side chapels, along the choir is a chapel which houses Constance's version of the tomb of Jesus, the so-called medieval Mauritius Rotunda, decorated with several statues.\\

St Gallen: Cathedral***, Library***\\
Reichenau: St Georg***, Monastery**, St Peter \& Paul**\\
Konstanz: M\"unster**

\section{November 10--November 13: Marrakesh}
\label{Marrakesh2017}

Why did we get here:\\
Northern African countries can be reached pretty easily from Geneva with direct connections to Tunisia, Morocco and Egypt. Cameron, Riju, Tony, Rachel and myself thought it would be great to visit it around November. Flights were affordable and typically the weather is still warm and sunny unlike things in central Europe. Same can be clearly said about the other two countries, but Marrakesh seemed more of a relaxed option than visiting the gigantic cultural heritage of Cairo for example, with a good mixture of food and a bit of culture instead though. Another advantage is our existing knowledge of French, thus most of us felt more confident to go to Morocco on our own, while in Cairo French would definitely not much of help. And flights were a lot cheaper than getting to Tunis, so Morocco it was, and from my previous trip to Morocco I preferred Marrakesh to Casablanca which was the other option to get to in Morocco. Now we are there for three full days, thus I checked our options for day two for a getaway. Three possible exiting options were available: A train or bus ride to Casablanca with a visit of the giant mosque, or a hike in one of the Canyons (not offered by many tour providers in November anymore), or a day trip to the Fortress of Ait Ben Haddou. Contacting one of the highly rated day tour (also multi-day) tour providers it became clear they could provide us a ride and guide for Ait Ben Haddou, so we opted for that one.\\

Co-travellers:\\
Riju: clearly my most loyal co-traveller loves to step food on Africa as well.\\
Cameron: another UCLA grad student from the US, he strongly voted to get to Marrakesh over the other options so clearly interest in what Morocco is up to.\\
Tony: a Ohio State grad student from the US, after about a year of being at CERN he dares to go on a trip with me, after having seen the most highly regarded places in Europe, ready to see a more unusual place like Morocco\\
Rachel: well here we go: if everything would have gone according to plan after giving a talk at a workshop in Hamburg the day before, Rachel would take Swiss with a transfer at Zurich to be on our early morning flight. Unfortunately some crazy German man decided it would be fun to run around the runway of Hamburg airport, leading to substantial delay of all flights. Once Rachel arrived in Zurich the flight to Geneva had already departed. Getting from Zurich to Geneva on that night by train or bus didn't seem feasible, and the early morning flight from Zurich to Geneva would arrive after our flight would leave. Thus Rachel was unfortunately not able to join us.\\

November 10:\\
We had booked a hotel quitee close to Djemaa el Fna, tha large market square of Marrakesh in the Medina. Since we arrived so early in Marrakesh (after all in another time zone than central europe), we left our luggage by the reception and had breakfast on the roof terrace. Different types of bread, cheese, humus, their version of yoghurt as desert as well as coffee and tea - what more can you ask for. Then we started to explore the town: walking through a market road we reached the Bahia Palace. Not having seen it the first time I was in Marrakesh I was pleasantly surprised to see beautiful courtyards with gardens and multiple nice rooms with gilded carved ceilings and beautiful tiles on the walls. And then it was time for a nice lunch, choosing one of the other squares as our lunch destination. Unfortunately for the waiter we still can add numbers in our heads, so he clearly tried to get a bit more out of us clearly foreign tourists. The El Badi palace is now in ruins, but impressive ruins giving you an idea of the amazing building it once had been. Just a couple of blocks away one can find the Tombs of the Saadian family, housed in a large courtyard with two halls, the hall of twelve columns is particularly impressive with fancy capitals, precious ceilings, a room as you would imagine it too look like from the 1001 night stories. And then it was time for a long dinner with Tajine among other things, topping it off with a round of drinks by a bar overlooking the square, which is still very busy and lively by night as well.\\

Marrakesh: Bahia Palace***, El Badi Palace**, Saadian Tombs***, Djemaa el Fna**\\

November 11:\\
Telouet: Kasbah***\\
Ait Ben Haddou***\\
Tizi N'Tichka Pass**\\

November 12:\\
Marrakesh: Medersa***, Mnebhi Palace (Marrakesh Museum)**, Almoravidic Kouba**, Jardin Secret**, Djemaa el Fna**

\section{November 18: Fonteney, Arc-et-Senan \& Besancon}
\label{Nov18}

Marmagne: Fonteney Abbey***\\
Arc-et-Senan: Saline Royale**\\
Besancon: Citadel**, Cathedral**

\section{December 9: Lyon - Fete de Lumiere}
\label{LumiereLyon2017}

Fete de Lumiere***

\section{China: December 29, 2017-January 21, 2018}
\label{2018:China}

Oh yeah that's. new one, I never had a trip crossing years. And that's how we got there. I was asked if I want to go either to Hong Kong or Chile, first i though - well Chile, would be my first time in the Southern hemisphere, so definitely very very exciting, but China (OK Hong Kong, still a different system, also visa wise, but close enough) is so close, so why not -- after all a long standing dream to see it once in my life. After all, I found out someone else wanted to give the talk in Chile, so Hong Kong it is. I checked what I could do without getting a visa, so I asked my friend Siyi, what she would suggest to do close to Hong Kong, but then she told me, she would visit her parents around that time, and considering i would only take a couple additional days off, I could make it a 3 week trip to China. Obviously China would need some preparation to visit, but if Siyi volunteers her help, clearly that is an offer to take, so just a couple of days later I booked my flight and now after applying for a visa with help from Siyi to plan my itinerary, by late October it is clear I will spend new year in China (my parents and my younger brothr were a bit sad though, after all my first new year all on my own). \\

Decr 29-30: (Fr/Sa) Flight and first day in Shanghai\\
Now on Dec 29 I get to Dock E at Zurich airport, the A340 has no functioning on-plane entertainment (I wanted to sleep anyway). Still suffering from the ending days of a cold, I finally get to China on Dec 30, find out my change is not enough for getting a ticket on the Shanghai metro - getting a coke to get the sufficient change - and ready to go and ending up in the city centre. Surprisingly enough gmail still works - so I can tell my parents i arrived safely. Then in this rainy very smoggy day I check out the Bund (unfortunately not seeing that far, but OK) - get some chinese food - and just hope the jetlag wouldn't catch me getting to be at 9 pm local time.\\

Shanghai: The Bund***, People's Square**\\

Dec 31: celebrating new year in China:\\
Today is the day when I am supposed to meet Siyi and her friends. Hot-pot is supposed to be really nice, and close to the high speed train station outside of the city centre is where you are supposed to get it. I note the address, not knowing that the shopping centre in question has four buildings. Google maps is not a thing and alibabab works in Mandarin only, so clearly I get lost. Calling Siyi she comes to rescue me, but still it takes over 25 mins to get hold of me. Anyways hot pot is so amazing, her friends are really cool too, she suggests to go to old town later on. Amazing really amazing, rather though old town in a US sense (aka buildings and surroundings are less than 200 years old, but still cute, my first rather traditional Chinese buildings). Siyi suggest to go to a fancy place for new years, we arrive there but seems we might have needed to reserve, and the dress-code also exceeded expectations, so we decide rather to walk to the inner city along the river, having cocktails at our hotel and then later by a cafe by Nanjing road. Overall a great new year's eve, but no fireworks (Chinese new year's falls on a different day than western new year) --- and continuing the trip in 2018.\\

Shanghai: The Bund***, Shanghai Old City**\\

 \chapter{Year 2018}
\label{2018}

\section{China: December 29, 2017-January 21, 2018}
\label{2018:China}

January 1: Shanghai without the Smog:\\
China is infamous for its Smog, so I was very happy after two kind off rainy and very smoggy days to finally see blue skies. And what a nice view it was. Since we did celebrate into New Year the day before (not Chinese New Year, thus no fireworks) we got up late, had Dumplings by Nanjing Road and then walked along the Bund once more. This time we were able to see the top of the Shanghai Tower - really an iconic skyline. Then we went to Shanghai station, where I obtained (with lots of help from Siyi) all my reserved train tickets. The ticket counters are situated in a separate building, and it was all huge. Same for Shanghai train station, there is a check-in where everything has to go through security checks as well. Since we booked so late, we didn't get on the ultra high speed train to Kunshan/Suzhou but we had to use the old high-speed trains. And we had to stand, since no seats were free anymore. Siyi's mum was displeased to pick her up at the old Kunshan station, thus she had to take the bus. I took the metro and walked back to my hotel -- an absolutely amazing hotel that is, turned out to serve the best food I ever had in a hotel so far for the next couple of days. Only then I realised that all metro maps I used from Suzhou were outdated, and the newly opened metro line stopped a mere 50 m from my hotel (could have saved me that 1 km walk with all my luggage). I stayed in the middle of the city, so I went along the Guanquian Street and got some grilled meat and sugar-coated apples and hot tea, and then made it to the Xuanmiao Temple before calling it a day.\\

Shanghai: The Bund****, Suzhou: Xuanmiao Temple**\\

January 2: Hangzhou: Westlake:\\
I managed successfully to get on the correct ultra-high speed train, and since stations are announced in English as well, this went really smooth, coffee and sandwiches were served on the train as well, so I was ready to go (after another metro ride). Unfortunately the day was pretty foggy, so not everything looked as pretty around the West Lake as maybe in summer. I first climbed the little hill with the Baoshi Pagode and a rock garden with many little caves and ancient Buddha statues. There are many little bridges connecting little islands, many boats which operate between the islands on the lake, a nice museum about the Zhejiang province. Then I walked through the neighbouring Zongshan Park, only a few metres away from the temple and the tomb of General Yue Fei. Another boat-ride to get to the very beautiful island of the three lakes (peaceful too, the whole lake side was pretty deserted). And then the final stop: the reconstructed Leifeng Pagode, full of fantastic wood carvings, gold plated sculptures, a ceiling with 1000 little Buddha statues, really nicely done.\\

Hangzhou: West Lake*****, Baoshi Pagode and Rock Garden****, Zongshan Park****, Temple of General Yue Fei***, Zhejiang Provincial Museum***, Island of three lakes*****, Leifeng Pagode*****\\

January 3: Suzhou and meeting the parents:\\
This time it was a very rainy and cold day again. Unfortunate for visiting the famous classical gardens in Suzhou. The largest garden is the Humble Administrator's Garden with several little pavilions, a large lake with several little islands. The Lion Grove Garden has a nice little lake in its centre, with lots of rocks and little houses around it, similarly the Yipu garden, whereas the Mountain villa is dominated by an aritificial rocky mountain. Suzhou is located by the Grand Canal, with little side canals crossing the town in a fine grid. On my way I also went into two pretty modern temples. Then I was picked up by Siyi, her parents, and a family friend and we went to lunch, sitting around a very large round table, which could be easily rotated, to get access to any of the delicious food items - from mush rooms, pork belly, beef, as well as trout and eel, and many more items. Afterwards we went to the water town of Luzhi, with many canals, little market stands, bridges and private residences. In between we listened to a choir performing on the town stage. Afterwards we got a private tour through the textile factory of the family friend. I had dinner with other young family friends, Zhicheng (Siyi's cousin), Siyi and a young teenager, seems all of us either studied or worked for a US university at some point in our lives. The food was Western-Chinese fusion, with the local culinary highlight -- the Chinese mitten crab. After I was told how to actually eat a crab, having never had one before, I really enjoyed it -- very tasty. Nowadays this crab might be found in Europe too, considered as an invasive species. I can only recommend to well try to reduce its numbers by just eating it. Anyways after that I had lots of tea with Siyi and her parents, before heating up in a warm big bed. Since Siyi's parents realized how much I enjoyed the local tea, they offered me some bags as presents, which I gladly accepted.\\

Suzhou: Humble Administrator's Garden*****, Lion Grove Garden*****, Mountain Villa with Embracing Beauty****, Garden of Cultivation (Yipu Garden)*****, Chenghuan Temple**, Xuanmiao Temple***\\
Luzhi Town: Canals*****, Jiangnan Park***, Xiaozai Residence**, Residence of Shen**\\
Kunshan: Bacheng-Yangcheng Lake District***\\

January 4: Nanjing:
After stopping at a noodle restaurant - according to both Eric and Siyi one of the best they have been to -- it was time for Siyi to say goodbye to her mum, and she, Zhicheng, and I took the high speed train to Nanjing. It had snowed continuously throughout the night. I was told that much snow is very unusual for the region, in fact if was the largest amount of snow Nanjing had seen in a decade. Nanjing used to be one ancient capital of China, in the early 20th century it was the main political centre of the Republic of China and the headquarters of the Kuomintang. While Siyi went on to give a seminar talk about her work in Astronomy I went to the Chaotian palace, which houses the city museum of Nanjing by now. This was my first large palace in China - very impressive, everything was covered in snow too and looked very idyllic. Afterwards I went to the presidential palace - which was the seat of the government until the late 40s. If you plan to go inside, then don't forget your passports, otherwise you won't be allowed inside. It was very interesting to see how the Chinese government was fascinated by Western architecture at the begin of the 20th century, modelling most houses after European early 20th century architecture (not what I as European would consider particularly beautiful and elegant). The gardens of the Chinese presidents were very nice and beautiful, particularly in the snow fall. Part of the palace are the former quarters (and the throne hall) of the emperor of the short-lived heavenly Taiping empire. The throne was modelled after the imperial halls in the Forbidden City in Beijing. Afterwards I checked into the hotel. There was a bit of confusion as I had a room booked for two, but at that point I was by myself and I had a bit of trouble to make them understand the second person would show up at a later stage. I checked where we would meet for dinner, this time I also checked which floor and which building of the shopping centre I should go to. I took a picture of the name, that way I could recognise the patterns and make sure to show up at the right place. And once it was time to meet for dinner I did find the restaurant in time --- only to find out I was the only one to arrive. So I know I am at the correct place, I also wrote to Siyi to confirm this is indeed the place where I should be. Unfortunately the traffic was horrible due to the unusual level of snow, and everybody was projected to be at least over half an hour late. Unfortunately nobody at the restaurant knew any word of English, neither do I know to speak Mandarin or any of the local languages. It was a bit awkward, since I could not make them understand that I know I am right where i should be, but I still want to wait for my friends to arrive before ordering. Anyways 45 minutes later everybody else showed up, Siyi, her colleagues from Nanjing, Zhicheng, a professor from the US with family ties to China and his family. All in all a group of over 10 people who were very friendly and eager to hear, what I though so far about my time in China. The food was very good too, the first time I tried pigeon (tastes more like chicken than Ostrich or turkey). It was a fun evening and then it came to go back to the hotel. Zhicheng stayed with me and Siyi tried to check in - unfortunately they couldn't find her reservation thus we first went to a wine restaurant to have some local wine. Afterwards she could check in without any issues. What Siyi forgot to mention (and share) until the next morning: she was allowed to have whatever was in the minibar in her room for free, to make up for the delayed check in. She did make use of it, but unfortunately alone only. Anyways since Zhicheng studies physics too, he wondered what my specification was, and then he asked me if I ever had heard of CERN. And yes I obviously did (having worked there at that point by over 10 years).\\

Nanjing: Chatoian Palace (Nanjing City Museum)****, Presidential Palace in Nanjing*****\\

January 5: Nanjing:\\
And we started out with a local speciality -- duck blood soup, pretty tasty in fact. And then we three made our way to the purple mountain - on this day rather a white snowy mountain. We got our tickets and then we made out way to the imperial tomb of Xiaoling, after getting a hot coffee. The tomb starts with a giant gate, followed by an alley through guards - mythical creatures, e.g. unicorns, as well as soldiers, camels, elephants, and lions. After two more halls the actual tomb appeared - a snow-covered citadel with a wide ramp moving through the inside up to a temple like hall. I was impressed. Closeby is the tomb of Dr Sun Yat-sen, the father of the republic of China. The symbolism of the republic and the sun of the Kuomintang is clearly visible, still referencing Chinese classic buildings. Our last stop on the purple mountain was the Meilling Villa, a present for the first lady from the last president of the Chinese republic who had power over the mainland before fleeing onto the island of Taiwan. The Villa combines western and Chinese elements in a pretty harmonic way. At this point it was I had to say farewell to Siyi and Zhicheng, up to now (2020) the last time we've seen each other. I then made my way to the Jiming Temple, a pretty temple close to the city walls. So close that I decided to actually climb the Wall, and walk for about an hour. It is after the Xi'an city wall the best conserved city fortification in all of China. I looked up what was listed on the metro map as additional sight, and the city wall fortress of Zhonghuamen stuck out, so I took the metro to the closest station and then walked 20 minutes back, passing by the modern day replica of the famous Nanjing porcelain tower. \\

Nanjing: Xiaoling Mausoleum*****, Dr Sun Yat-sen Mausoleum****, Music Hall***, Meiling Villa****, Jiming Temple****, City Wall with Zhonghuamen*****\\

January 6: Nanjing\\
I spent the morning working on my presentation, running the latest version of the reconstruction algorithms. After 3 h of work -- clearly the connection was not the best -- I decided to visit the Nanjing Museum, with pieces of art, sculptures, Terracotta statues, silk clothing, bronze pots etc from prehistoric times up to the 20th century. In another wing contemporary paintings from Chinese artists were exhibited, as well as old steam trains and furniture from an old tea house. Only gates remain from the former imperial Ming palace. And then it was time to take the bullet train to Beijing. Due to the unexpected high levels of snow most trains were delayed by over an hour, just like mine. Once I arrived in Beijing I contacted my old co-worker Selena in order to organise where we should meet up the next day, and what we would try to see.\\

Nanjing: Nanjing Museum****, Ming Palace (Forbidden City)**\\

Up to now this has been the last time I saw Siyi in person, after three trips with overnight stays, and one more day trip.\\

January 7: Beijing\\
Selena suggested I get up early in the morning to watch the flag rising ceremony and so I did. A couple of snacks and coffee later I met up with Selena and her PhD student. We had to hand in all our baggage before going to Mao Zedongs mummified body inside a typical socialist style tomb (reminded me of Lenin's mausoleum). Then we spend two hours in the national museum of China, which houses a variety of art, from modern wood carving to african art, presents from Chinese state guests, as well as artefacts from the 2nd millennium before christ to the modern ara. The next item was the former imperial palace, the forbidden city. Selena bought three tickets for us online, technically those are then uploaded to the national ID though, my passport number was given in the booking, but naturally the ticket couldn't be uploaded. We were told that I could get a paper ticket (though not really indicated whereabouts), but seems they were also fine with me just showing my passport. The palace is gigantic, the main halls follow a rigid scheme, with many carved thrones, gilded halls, little bridges and terrasses. There are multiple courtyards each with their own private palaces. We also booked a visit to the treasury. One of their most valued pieces are western clocks, just like in European palaces Chinese porcelain was appreciated. After we enjoyed a bit longer in the imperial gardens we walked along the walls and moats, and then took the bus to get hot-pot in one of these big shopping streets. It was fantastic food and the personnel was very eager to get photos of me and my friends (and some special dessert for first customers). I was told that they try to make this dish for all new customers, including locals. \\

Beijing: Mao Zedong Mausoleum****, Chinese National Museum****, Forbidden City*****\\

January 8: Badaling\\
I informed myself previously what would be the easiest way to see the Great Wall starting from downtown Beijing. Most sites suggest to take buses, since these are far more frequent, or to use some local travel company to get there. I decided to use the train instead. It seemed pretty easy to do and the train times starting in 2018 had been posted two weeks before. The construction of a preliminary train station had just been finished (the new northern Beijing train station is completely remodelled for the winter olympics 2022). On the train I bought some snacks and water, and about 90 minutes later we arrived in Badaling (final stop, but all stops in between were announced in English as well). Now just a 20 min walk later, the Great Wall of Badaling opens up. There are two sections, I decided to walk over both. You might have seen photos of the Great Wall completely packed with people (typically those are taken over the Chinese New Year weekend so, thus not that representative). But seems in January with chilling wind and -14 C almost nobody wanted to walk along. I had multiple layers of closing on me so it was acceptable, and indeed at times I had the wall almost to myself. So I enjoyed it a lot, and then on the way back I got some dumplings at a local restaurant, where nobody spoke any English, but there is hardly anything you can do wrong with getting dumplings. And then back to Beijing again, where I decided to visit the Yonghe temple (also known as Lama temple). In the Great Hall of Harmony and Peace is the gigantic 18 m tall Maitreya Buddha statue on a 8 m pedestal, which is really breathtaking. Then I had dinner by the gates and towers of Zhengyangmen.\\   

Badaling: The Great Wall*****\\
Beijing: Yonghe Temple*****, Zhengyangmen****\\

January 9: Beijing\\
I started the morning enjoying the sunrise at the temple of heaven, free of any person. I was confused by the fact, that one has to leave the enclosure of the first hall, then enter another enclosure with the second prayer hall, and then one has to go back to the first enclosure. Anyway the whole temple complex includes several little palaces, prayer halls, kitchens, a wide park, meteors, so lots of walking (and then I had a short lunch snack before making my almost 2 h trip to the Summer Palace of Beijing. Getting in through the north gate, the first district is the Suzhou road, a re-enacting of of the grand canal by Suzhou. The canal was completely frozen and hardly any of the shops along the road were open. Then I reached a citadel which covers almost the full mountain leading up to a wooden large Buddha statue, with article rocks, grottoes, little pavilions surrounding the main hall. On the other side of the main hall towers a four floor pagoda, overlooking the lake of the summer palace. There are once again many little halls and palaces, as well as monasteries on the complex, a marble boat is sitting on the lake. Many pieces of art can be found in the Wenchang gallery. Some people were brave enough to go on the ice of the lake. I took instead the bridge over to the island. Once again there was an icy wind blowing over the lake, and it was too chilling to be comfortable. Since the Summer Palace is quite far away from the city centre, it can be combined with a visit of the Olympic park of the 2008 Summer Olympics. By now the night illumination was switched on. The stadium is very impressive, the wind was too strong for me to try the roof-top walk, but the restaurant in the stadium was nice.\

Beijing: Temple of Heaven*****, Summer Palace*****, Olympic Stadium*****\\

January 10:\\
Since I managed to see everything I really wanted to see in Beijing, I tried to see some more unusual sight, starting with the Bell and Drum Tower. Walking through the Hutongs (kind of the remains of the old city districts) I passed by the lakes which marked the beginnings of the Grand Canal, which I found out only a couple of months later. The Prince Gong House had a nice garden with the usual palace buildings. I was positively surprised about the Beihai Park. The nine-dragon-screen was praised everywhere, and it is very nice indeed, but far more impressive is an artificial mountain with thousands of statues and on top a golden buddha statue. The white Pagoda dominates the Jade Flower Island in the middle of the lake. Then i walked over to the Jingshan Park, with several pavilions, the main pavilion is just opposite of the forbidden city and offers a full view of the whole imperial palace. And then I got some early sleep for my next day.\\

Beijing: Bell \& Drum Tower****, Hutongs \& Grand Canal****, Prince Gong House****, Beihai Park*****, Jingshan Park****\\

January 11:\\
Getting up very early to reach Beijing West station, one of the largest stations of the country. I had quite a large breakfast there before taking the bullet train to Luoyang. Once I arrived at the train station I quickly found the bus stop, where two tourist buses were supposed to leave for the Longmen Grottoes (going every 20 minutes at least according to the English schedule, even in winter). After 30 minutes I started to get a bit nervous, taxi drivers around me had there apps talking English to me, telling me the next bus would come only in 3 h. I decided to wait a bit more, another 30 mins later, I was too annoyed and started walking. After all, the grottoes are only 3.5 km away from the train station. Once I reached the next large road (after a 10 min walk), there were plenty of buses going to the grottoes. The grottoes are really worth a trip from Beijing. Hundreds of Buddha statues carved out from caves, some even still painted in several colours. The western grottoes are far more interesting in my point of view, with many more surviving statues. One gets to each of the caves over several flights of ladders. Crossing over the Yi river one reaches the eastern grottoes (with a little museum). At the end of the grottoes is the Xiangshan temple, one of the few temples in China where photography is allowed. A villa of a former Kuomintang Governer is closeby as well, a bit boring, since it is built in Western fashion. Behind the temple is the cemetery and the tomb of the general Baijuyis. And once I arrived at the train station, I even managed to exchange my reserved ticket to two trains earlier. In Xi'an I found out, that I could not yet book the tourist bus from the Xi'an North station to the Terracotta army, but would need to come back the next days. I got almost lost on my way to my Ibis hotel, which had just opened a couple of weeks earlier (the lobby was still being furbished, also the clerk didn't know any English). I also had a huge room, with two sofas, a huge flat screen, free coffee and tea, I was impressed. Anyways I made my way to the city centre, where I walked along the Drum and Bell Towers, having the local dish of ..., and then I admired the Yongningmen gate and the city wall with its canals, and back to the Ibis hotel, where I clarified that the bus from the main bus station would be the quickest way to get to the Mausoleum (which would be the last stop too).\\

Luoyang: Longmen Grottoes*****, Xiangshan Temple****, Baijuyis Cemetery****\\
Xi'an: City Wall*****, Drum \& Bell Tower****\\

January 12:\\
Checking out really early I made my way to the main bus station. As people described online there are a lot of handlers trying to get you on their bus, claiming the public one would take ``too long'', but I was prepared for that and got on the correct bus. Once I was at the Mausoleum of Emperor Qing Shihuang I got my coffee, since I had to wait for another 15 minutes before the cash desks opened. I was the first one to arrive at the pits. It is really interesting to see how archaeologists still excavate, catalogue and clean remains they find there. It still seems many years to come before this project will get to completion. I enjoyed the army, but I agree that people should know what to expect. Particularly pit1 is gigantic, but still the soldiers are of similar type (though each with distinct faces), but I understand why people might ask themselves if it is worth a two day detour . I definitely think it is worth it, particularly if you stop in between in Luoyang or if you see the city of Xi'an as well. The museum is a bit of a let down, but the surrounding park is nice especially in a wintery scenery. Then i got back to the city again, walking around some local parks and temples, before going to the Dac'ien temple with the giant wild goose pagoda, one of two large pagodas in the city. The temple halls are full of woodcarvings, some of them plated in gold, and many golden Buddha statues. The pagoda has seven floors to climb. After walking along the city walls a bit I decided to stroll through the bazaar. Xi'an has quite a large century year old Muslim population, so I also went to the mosques. Considerably different than all the mosques in the mediterranean area, the design is rather reminiscent of Asian temples the prayer hall is of limits to non Muslims. And then i got back to the train station for my 4 1/2 h train ride back to Beijing.

Xi'an: Mausoleum of Emperor Qing Shihuang (Terracotta Army)*****, Daci'en Temple with Giant Wild Goose Pagoda*****, Mosque****, City Wall*****\\

January 13:\\
The Confucius temple and the Imperial Academy or next to each others, just a couple of hundred metres away from the Yonghe temple. Quite cute, but nothing which one has to have seen, although the forrest of in-scripted rocks and pillars is quite something. Then I decided to pass by the cathedral of Beijing where a wedding just ended, before getting to the QuanJuDe restaurant by the Qianmen street, where I had a really delicious half of a Beijing duck with pancakes and soup. I decided rather to finish early before having my flight over to Hong Kong the next day.\\

Beijing: Confucius Temple and Imperial Academy***, Cathedral**, Zhengyangmen****\\

January 14:\\
Getting to the airport lasted quite a bit, the plane was late by over an hour, but it was nice to see all these Air China Boeing 747s taking off and landing. Flying over several mountains, snowed in river planes I finally made it to Hong Kong International airport and got on my bus which brought me to downtown and my hotel (room without a window, well the main room, the bathroom did have a window though, so could get some fresh air in). I went over to the sea to enjoy the Harbour laser show. Unfortunately the lakefront walk had been closed down for construction, thus I had to stand at a non ideal spot. The laser show was a bit of a let down, OK a bit of flashing lasers, but very faint, not synchronous with the music (which was not that interesting either). Anyways the panoramic is fancy enough with all the lights, but still I had expected more.\

January 15:\\
After giving a rehearsal of my CLIC overview talk, I made it over to Hong Kong island and visited the museum of the Hong Kong currency in One International Finance Center. Since this museum is for free and situated quite high up the skyscraper you get a nice panoramic view of the island (and a bit of Kowloon). Then I walked through downtown a bit, and then along the shore line and by the piers. Then I took the metro back over to the peninsula and then visited the International Commerce Centre Observatory. Magnificent views of the sunset and the night panoramas of the Kowloon area as well as the harbour and the straight and Hong Kong Island. It is maybe a bit pricey, but I do enjoy observatories, so it was really fun.\\

Hong Kong: One International Finance Center Observatory****, International Commerce Centre Observatory*****\\

January 16:\\
Having purchased my tickets for the Hong Kong-Macau ferry starting in Kowloon I started walking through the parks over to the Terminal during sunrise, having gotten a window seat was nice for the to roughly one h boat ride to Macau. There I walked over to the old town. Macau had been a Portuguese colony for quite some time. Thus the churches and the cathedral are reminiscent of Portuguese styled baroque churches, though still far shy from the amazingly decorated ones in Coimbra, Porto and Lisbon themselves. The temples and private residences were alright, though I did in fact expect a bit more. I wouldn't call a single building (also none of the parks) in Macau a must see sight. The food I had was very nice (not that I had heard any of the names before, so I just went by their description of the ingredients. Restaurant claimed to be of Cantonese cuisine, and I was the only non south east asian around. Since I was finished far earlier than I thought, but still had quite some time to spend before my ferry would leave, I decided to check out the casinos, which are the reason why most tourists come to Macao. Most impressive among those was the Grand Lisboa. I made my way to the ferry terminal and had a snack break by Fisherman's Wharf, where i was greeted by their version of a ruined Roman theatre. Back in Hong Kong I went to the Botanical gardens, which is also home of a little Zoo with lots of birds, reptiles, monkeys, and turles. I enjoyed the Botanical gardens in dusk and stuck around for some downtown night photos before crossing over to Kowloon for dinner.\\

Macau: Cathedra**l, St Paul's****, Church of Antonius***, Domenican Church***, Sam Kai Vui Kun Temple**, St Lorenz Church**, Madarin's House****, A-Ma Temple**, City Office**, Lou Kau House**, Grand Lisboa***, Fisherman's Wharf**, Camoes Gardens****, Na Tcha Temple**, St Joseph***\\
Hong Kong: Botanical Gardens****, St John's Cathedral***

January 17:\\
The waiting time for the cable car up to Victoria Peak was very short in the morning, unlike in the afternoon and evenings before, when over an hour of waiting time was announced. I didn't bother to visit Madam Tussauds or the observatory of the peak tower, but decided to walk along the hiking paths instead. Indeed the panoramic views are very nice, although it was a bit hazy. Although it was only January it was already very warm and time to put on T-shirt and shorts (compare that to the -15 C in Badaling). Once I was down I walked over to the catholic cathedral, just realising that it was just a few metres away from the exit of the botanical gardens. I was still confused about the many layers of Hong Kong roads, where you can walk along roads on different levels just basically stucked on top of each other. After having a brief lunch in one of these pub type restaurants in downtown I got to the other side of the Peninsula to reach the HKUST Conference lodge. I was informed that unfortunately they were out of single rooms, so they could ``only" offer me a double room with the ocean view. I couldn't believe my luck and gladly accepted. Then I had dinner on the university premise (with dozens of restaurants, mensas, all very very fancy, even beating out what I was used to from Switzerland.

Hong Kong: Victoria Peak***, Catholic Cathedral*\\

January 18 \& 19:\\
The workshop on calorimeters was very interesting with very engaging discussions. It was fun to discuss the content of my two presentations with other experts in the field. In the evenings we had really nice food in local restaurants, also with a bit of tasting of local liquors. I did enjoy all of it.\\

January 20:\\
This was one item my sister recommended me to do. Located on an island close to the airport, I had to take a bus, transfer via two metro lines and then take a cable car over a mountainous island, crossing even a full bay. On the way you have a perfect panoramic view of the Hong Kong airport. It was really fun to see the movement and landings and starts from quite a distance. And then it appeared. A gigantic Tian Tha Buddha statue sitting on top of the hill. I walked up the stairs, walking around and inside the statue and then checking out the closeby Po Lin monastery with the many golden Buddha statue, since the hall was nicknamed hall of thousand buddhas, the number of statues was definitely huge. And then about 2 h to get back and another hour to get to the airport, and then having a long dinner and waiting for the departure of my plane very shortly before midnight on a Lufthansa Boeing 747-800.\\

Hong Kong: Tian Tha Buddha*****, Po Lin Monastery****\\

January 21:\\
Once arriving at Frankfurt airport I transferred back into the Schengen area. My camera was checked for remains of explosives, and tested positive. Naturally I had to have a special interview with the border police, aka where I come from, where I go to, why I go there, and I had to show them photos on the camera display. Seems I was not too suspicious and they let me go. Later I informed myself if that was a very rare occasion, or if that happens at a certain frequency, since you rather want your test to be sensitive than missing out a rel threat. It seems even some hand washing lotions can cause the alarm to be set off. I made it back to Geneva and straight away back to work for my first normal working day of the year.\\

\section{January 28: Lausanne}
\label{Lausanne2018}

Lausanne: Palais de Rumine (Cantonal Museum of Vaud)***, Cathedral****

\section{February 4: Nyon}
\label{Nyon} 

Nyon: Castle**, Roman Museum**, Roman Temple \& Terrace***

\section{March 3--March 4: Toulouse \& Albi}
\label{Toulouse2018}

March 3:\\
Albi: Cathedral*****, Palais de la Berbie****, Collegiale St Salvi***, Hotel Reynes**\\
Toulouse: Carmelite Chapel****, Basilica St-Sernin*****, Cathedral****, Canal du Midi****\\

March 4:\\
Canal du Midi****, Cathedral****, Hoel d'Assezat****, Basilique Notre-Dame de la Daurade**, Capitole*****, Jacobine Convent****, Basilica St-Sernin*****, Notre-Dame du Taur***

\section{March 9--March 11: Amsterdam \& Brussels}
\label{BrusselsAmsterdam2018}

Co-travellers:\\
Mohamed: directly responsible for this trip, but more about that later. Often, before I go on trips or start planning trips, I try to maybe get into contact with locals, sometimes it works out, often I get no reply back but without trying for sure you won't find any success. It has been a year-long dream of me to visit Egypt again, particularly to see more of Cairo or to plan getting inside one of the pyramids, and maybe to see more tombs in Luxor or even get to Alexandria -- well you get the idea. Mohamed works as software engineer in the steel industry, but has visited Germany for an extended time related to his work. And he loves to see places and visits interesting sights as well (being at least trilingual with English and German as additional languages is a big plus too). Thus when i started to talk to him he was quite eager to make it also work in his favour, and he asked me what I would suggest to do after another work related business trip to Western Germany, e.g. going to Amsterdam. Which brings me to the introduction of this trip:\\

How did we end up here this time?\\
Mohamed wanted to see Amsterdam. While Amsterdam has its charm (although a bit more in summer times and not March), I still would always rather visit Belgium and Brussels in particular. Thus I suggested that we could indeed see Amsterdam, but I would suggest another day in Brussels after. Mohamed agreed on that plan and we decided to meet up in Amsterdam, with him directly taking a bus from Germany, while I would get on a bus from Brussels.\\

March 9:\\
A typical late flight our of Geneva with Brussels, so delayed by a couple of minutes but nothing special. This time I stayed at an IBIS by the Place d'Agora, but since I had seen it previously and planned to go there the next day anyway I just decided to sleep early.\\

March 10: Amsterdam:\\
Since Mohamed wanted to take the bus instead of the train, I got to experience my first FlixBus rides. The bus left from Bruxelles Gare du Nord so jumping on an early train from Centraal getting myself some coffee and pastry before waiting for the bus. The bus was more or less on time and a stop by Antwerp Centraal later we were on our way to Amsterdam. The bus did even arrive a bit early, while Mohamed's bus from D\"usseldorf was late by about 25 minutes. We got ourselves day tickets, dropped his luggage by Amsterdam Centraal and off we went to the Royal Palace of Amsterdam. Originally built as city hall in Amsterdam's golden age, the city hall was later converted into a palace during Napoleon's I time. Although the king resides now in the Huis ten Bosch by Den Haag, the palace is still used by the royal family for official receptions and state visits. Typically the palace is open for visits all day long though. Although some rooms have been converted to dining rooms or guest bedrooms, they still carry on their names from the former use as city hall, but the style quite clearly changed to Empire furniture. The central hall, the Burgerzaal, covers three floors, dominated by statues of Atlas and virtues. A large fresco covers the wooden ceiling. Large galleries connect the central hall with the private quarters and rooms, like the throne hall or the Moses Salon where audiences take place nowadays. Walking along Amsterdam's canals and getting some more coffee and a short snack we got ourselves Tickets for the Van Gogh museum. Although it was winter season, we queued for about 20 minutes (tickets for the day were not sold online anymore), and we had to wait 90 minutes for the next free spot. I would recommend you buy tickets for this museum in advance. For how hyped it is, it is actually a tad disappointing. I had seen a couple of paintings already previously on special Van Gogh exhibitions, like the Potato Eaters in Milan. Photography was not allowed due to conservatory reasons. You might now think clearly if they would allow it, all paintings would be clogged with people, but I doubt this would get worse. The most important pieces had been surrounded by guides and their groups anyway, and in all other museums there is typically one Van Gogh painting which everybody wants to get a selfie off, while all other paintings just next to it are unfortunately completely neglected. Anyways I did learn a few details more about his life, although I knew quite a bit about what was told beforehand. So your experience might be different depending on how many of his paintings you saw before, or how many details of his life should be familiar to you. They did have a pretend to be in front of a painting selfie spot which was heavily used to have a copy of a self portrait in the background. Since we had now about an hour to spent we went back to Rijksmuseum which is just a couple of metres away from the Van Gogh museum. The main painting there is Rembrandt's Nightwatch. Still a huge painting even considering that parts had been cut out at some point. We also saw other paintings, period rooms, silver ware, carved statues, porcelain, quite different things which can be found here. After dinner we got on the bus back to Brussels (was about 30 mins late). Back in Brussels by night time, we visited the Galeries Royales St-Hubert and the Grand Place and Mohamed got himself some more food and drink.\\

Amsterdam: Royal Palace*****, Rijksmuseum****, Van-Gogh Museum***\\
Brussels: Grand Place*****\\

March 11:\\
I had bought our tickets for the Stadhuis (city hall) of Brussels, and I became quite nervous if I had explained it properly where and when the tour would start when Mohamed wasn't around five minutes before. He arrived on time anyway and I saw once more the beautiful inside of the building. Only this time photography was allowed, we also saw one more room (the anteroom of the Mayor's office). The first rooms are decorated with lots of tapestries, the later room have paintings on the wall as well as wood carvings. The festival hall is decorated with tapestries which are supposed to look like gothic characters, but since they were woven centuries later the style is rather a mock medieval one. After a short stop by St Nicholas we saw the gothic cathedral, which is one of many European cathedrals in such style with stain glass windows. For the first time I visited the Royal Museum of Fine Arts, concentrating on the old Flemish masters, like the Brughel family or Hieronymus van Bosch (pretty impressive). Walking over to the Palace of Justice we stopped inside the nice Notre-Dame au Sablon. And then we had steak frites and coffee by the Place d'Agora. For me it was then time to get home to Geneva, Mohammed stayed for another day, visiting the city of Namur in the Wallon\\

Brussels: Stadhuis*****, Cathedral****, St Nicholas***, St-Jacques-sur-Coudenberg***, Royal Museum of Fine Arts****, Notre-Dame au Sablon****

\section{March 28--March 31: Iceland}
\label{2018Iceland}

How did we all end up in iceland -- by coincidence. By end of 2011 Rachel mentioned that she would love to see the Northern Lights. So we checked how expensive it would be to get to Norway. And I just wondered if Iceland would be cheaper. Unexpectedly flying into Iceland was a lot cheaper, and considering that we also might want to do something during daylight, we decided that Iceland would be the more intriguing alternative. And it was also the easter weekend.

Co-travellers:\\
Rachel: indirectly responsible for this trip, as always on our trips together, Rachel will evaluate where we should have dinner.\\

Riju: by this point my most experienced co-traveller, it took nothing to convince Riju to tag along.\\

Reyer: secretly viking at heart, Reyer was up as well to find out what it's all about in Iceland.\\

March 28:\\

The flight from Geneva to Iceland takes about 4 hours, followed by a bus ride of about 1 h. The final bus stop was just a couple of metres away from our hotel, so we just dropped our stuff there and continued to explore the town. Iceland is only sparsely populated, Reykjavik is by far the most populous town in the country. As volcanic island geothermal heat is used to produce power, as well as heating up roads and water, unless the water is naturally warm enough. There are hot springs all over. The tab water does have a sulphuric taste to it, not dangerous for health reasons, but still something you first have to get used to. Anyways we walked along the coastline, passing by the sun bark??? monument by Jon Arnason, set up by the coast in 1986. Then we warmed ourselves up in the cafe and the foyer of the Harpa Concert Hall, a very modern hall with an exciting design. After that we had dinner by the harbour. Naturally it was already pretty dark, unfortunately also very cloudy, thus we decided to call it a day, since we had to get up pretty early the next day, when we also booked an excursion to bring us away from any settlement to watch the nightsky.\\

Reykjavik: Harpa Concert Hall****\\

March 29:\\
We were picked up by a small bus, in the end travelling with a small group of about 12 folks on what turned out to be the most beautiful day on the island. We first stopped by the huge waterfall of Seljalandsfoss. Most waterfalls are fed by glaciers covering the volcanic peaks of iceland. This waterfall is special in that sense, that just opposite of it you can walk up a little hill across the waterfall, the hilltop being just about halfway the height of the waterfall. It is also possible to walk behind the waterfall (getting soaked substantially though). A couple of minutes of driving a second large waterfall, the Skogafoss appears, with a slight rainbow hovering of the fall. A couple of ladders bring you up to the begin of the waterfall, and we walked a few hundred metres further along the river, seeing a couple of smaller waterfalls. Then we were brought to the Black Sand Beach. The volcanic remains were also visible in large rock islands just a bit after the beach line. This beach is notoriously dangerous due to waves pulling people inside the water with large power. It is advised to keep clear of these waves. Close to the beach is also a grotto, built up by basalt pillars, similarly to what can be found by the Giant's Causeway. We also had a small snack for lunch by the beach. Then we got to the highlight of this day: the large glacier of Solheimaj\"okull. Originally almost reaching the shore, the tongue of the glacier retreated quite substantially, now ending in a large lake of melting water. Getting equipped with crampons and ice axes we started our hike, quickly crossing a part, where the rock starts to become unstable due to melting permafrost. The glacier is covered substantially in volcanic ash originating from decade long eruptions. We tasted the crystal clear ice water running on the top of the ice, also we could look into crevasses in the distance. The walk last between 90-120 minutes. Then all the way back to Reykjavik, a short dinner, and then on to another excursion getting us to remote areas with almost no light pollution. The sky was not absolutely crystal clear, but anyways we were out of luck, and unfortunately no Northern lights.\\

Seljalandsfoss*****, Skogafoss*****, Vik i Myrdal: Black Sand Beach*****, Solheimaj\"okull*****\\

March 30:\\
The morning started with a thick dense layer of fog. We drove up to the upper plains just a bit above the fog, only to climb down into the dark again, this time into the Hafnarfjlorthur Lava Cave, one of the plenty lava tubes from former eruptions. Great to see the remnants of a tunnel carved out by hot magma. We also saw a couple of icicles and snow at the end of the cave. Back to Reykjavik where we got on a large bus for the golden tour. Starting out in the Thingvellier national park, where the American and the Eurasian plate move apart, opening up the middle Atlantic rift, which basically created the whole island of Iceland. We were given about an hour, to explore the rifts and to go down to the rock, where the ancient parliamentary meetings of the Icelandic vikings, the Thing, took place. Anyways an hour was pretty tight to make all of it, and after all it became a bit longer, waiting for Riju. The next stop was the giant Gullfoss waterfall, where the river drops down via two cascades into a canyon. Unlike close to Reykjavik the plains were still covered in snow. It started to rain, so we fled into the restaurant, getting soup for each of us. After this short lunch snack we did a 10 min drive to the Geysir Geyser basin. This is not a typo, Geysir is an ancient geysir, which named the whole phenomena, but it has been laying dormant for a couple of years by now. The close by Strokkur geysir is very active, erupting for a couple of seconds about each 10 minutes. Not as impressive as Old Faithful, but clearly far more regular. Our tour guide joked that every day hundreds of phones run out of battery trying to get that perfect photo of the eruption. On good Friday almost all shops and restaurants are closed for almost the whole day, and many restaurants open shortly before midnight. We managed to find a place though which opened up late in the evening already, and had some good local cheese and sausages (and beer and met).\\

Hafnarfjlorthur Lava Cave*****, Thingvellir National Park*****, Gullfoss*****, Strokkur Geyser*****\\

March 31:\\
We just had a couple of hours before jumping on the bus to the airport. We used it to see the modern Hallgrims Church, built by mid 20s century, The outside is dominated by the tower, whose decoration gives a nod to the basalt columns of volcanic eruptions. The interior reminded me of a modern interpretation of gothic style. Strolling through the city we also stopped by the catholic cathedral of Christ the King, once again built in gothic revival style. And then it was time to leave, transfer to the airport, where the plane had a delay of about an hour.\\

Reykjavik: Hallgrim's Church****, Christ The King Cathedral**

\section{April 7: Valais}
\label{Valais2018}

Sion: Chateau Tourbillon***, Chateau Valere****, Cathedral***\\
Vernayaz: Pissevache****

\section{April 13-April 15: Denmark}
\label{2018:Denmark}

This trip idea was related to Rachel as well: Rachel decided to get a tattoo depicting an event recorded by the Big European Bubble Chamber. Letting us decide if we would prefer to rather spent a long weekend in Copenhagen or Stuttgart, we voted for Copenhagen. When it came to the room, we had quite a bit of a discussion if we should get two rooms or one room for all four of us, Reyer putting an end to the discussion and deciding it should be one room, but more about that later.\\

April 13:\\
Geneva airport can be quite crowded, but this time it was really horrible. Almost every single flight had a delay of at least 20 minutes. Naturally my flight was delayed too. By the time it was supposed to depart, a flight leaving to Frankfurt just arrived for boarding, making it clear, that it would be tricky to transfer in Munich on time. My flight left with a delay of 50 minutes. Although I tried everything sprinting up and down 5 flight of escalators at Munich airport, with a short Monorail trip between satellite and main terminal, I missed my connection by a couple of minutes. I had to discuss to get on a flight by 7 am, and not at 11 am as was original proposed. Anyways this meant getting up the next day by about 4 am, hopefully without suffering the whole day being deprived of sleep.\\

April 14:\\
Getting on the early morning flight I finally made it to the hotel in Frederiksberg by around 9:30. Seems the room was quite crowded already with two bunk beds within 10 square metres, and the typical shower over the toilet, similarly to what I encountered in Copenhagen the last time around. Anyways we just made it in time for the tour of the city hall by 11 am. I was quite happy to make it this time, having missed out on a tour in 2013 by just 5 minutes. In fact the tour had just started but the guide told us he would wait for us to get tickets. The lady by the ticket counter though told us to wait since while she took over 5 minutes to explain another person how to read the metro map. Anyways we still made it onto the tour, so special thanks to the tour guide. The city hall was nice, many rooms decorated by murals, stuccos or tapestries, all in a style typical for the transition time between the 19th and 20th century. Then we visited the Rosenborg Castle with lots of rooms in Renaissance and Baroque style. The great hall houses two thrones, one made out of whale teeth, tapestries depicting important battles, and silver lions. The decoration of the glass cabinet is the highlight of the castle. In the basement of Rosersberg, the danish crowns, diamonds, diadems, sceptre, and imperial apple are displayed, together with other pieces of art of the royal treasury. It was time for a short snack, having Goat sausage this time, as well as my first try for a vegetarian sausage, which I thought was pretty tasty. The next item was the Royal Palace of Christiansborg. Christiansborg had been the main residence for Danish kings for a couple of centuries. After the Baroque palace burnt down, a classical palace had been erected, but that one burnt largely down by the end of 19th century. The next palace was built in the 20th century in a Neo Baroque revival style, this time only for representation purposes. It also houses the parliament, so both the legislative and executive branch of the country. The residence of the queen moved nowadays to the four palaces of Amalienborg. Parts of the former classical palace still exist, e.g. the royal chapel, as well as parts of the decoration of the Alexander Hall. The tapestries of the Great Hall have been woven in a modern style depicting recent historical events and the current royal couple. Seems Copenhagen was quite busy during this time of the year already, only our third try of a restaurant had an open spot. And then we made it to the amusement park of Tivoli. Unlike in summer no concert were going on, but we had a couple of snacks, and enjoyed the decorations, which are built in the style of a Mosque, or an artificial Matterhorn, lakes, pirate boats etc. Rachel and Reyer had fun on a rollercoaster, afterwards we witnessed the Fountain- \& Light show (still really nice). Then Reyer, RIju and I called it a day, while Rachel decided to stay out for a bit longer. So back to the hotel to distribute this time Riju and I on the upper bunk beds, and Reyer and Rachel to the lower floors.\\

Copenhagen: City Hall*****,  Rosenborg*****, Christiansborg*****, Tivoli*****\\

April 15:\\
After flooding the toilet (impossible to keep it dry, since the fixed shower head is directly over the shower, quite common in Denmark in fact), we wanted to get a longer brunch at one place Rachel found. We had to wait for about half an hour though, which Rachel and Reyer used to go hunting for Pokemons. Well anyways once the place was ready, then we had pancakes or salmon and all varities of coffee. Then we walked over to the central station and took the train to Roskilde. The cathedral of Roskilde houses the tombs of the Danish kings, built in brick gothic style (you find a similar style in other scandinavian countries, as well as in northern Germany). The tombs are in several chapels attached to the main nave, not quite as grandiose as those of French and English kings, but more impressive than e.g. Germany or Sweden, or Spain for that matter. The main altarpiece from 1560 is magnificent and most carved statues are covered in gold leafs. It is also possible to get to the second level, which gives you a perfect view of the tombs in the choir. And then it was time to take the trains back to the airport for myself, while Rachel, Riju \& Reyer had dinner and took another flight home to Geneva.\\

Roskilde: Cathedral*****

\section{April 27-May 1: Paris}
\label{2018:Paris}

Paris is amazing, but it is also useful to use as centre of a northern France trip. The French rail system is clearly focused on Paris, and to get from city to city you typically transfer in Paris anyway, so why not going to Paris right away and then doing train trips from there.\\

Co-travellers:\\
Reyer: After quite a couple of trips, Reyer decided we should do a bit of Paris and France, clearly I was on board. Reyer is one of the few people whose normal walking speed is even a bit faster than mine, clearly helped by his height which exceeds 2 m.\\
Christine: With her love for good restaurants and food France and Paris are always a nice place to visit again. Once Reyer and I told her that our trip is expected include Champagne tasting, she was even more eager to join.\\

Our Paris trip was already derailed by the months long train strike beforehand. We had booked a couple of train rides in order to see the major gothic cathedrals, which Reyer was very eager to see. So we had to reroute a lot of the originally planned itinerary. On the first day we originally planned to see Reims and Amiens, followed by Bourges on the second day. On the third day everything worked out as planned, besides taking the very last train out of Rouen (the one before was cancelled two weeks before). On the last day I wanted to see Chartres and that worked out too, but Reyer and I decided to spent a bit less time in Chartres and instead see the cathedral of Amiens beforehand.\\

April 28: Reims\\
After we found out that our TGV ride to Reims was cancelled due to the strike, we all decided that we should still go there, albeit a bit slower taking the Ouibus. We had a short breakfast and coffee at the Gare de Bercy before the bus ride of about 2 hours. Indeed even no single regional train would take us to the city centre of Reims, thus we got on the tram. Having arrived in old town we admired the many sculptures of the gothic facades and portals of the cathedral. Reims cathedral had been the church where the French kings were coronated, unfortunately it was hit very hard in World War I, and most of the stained glass had been lost forever. The interior of the cathedral is though still a magnificent sight, including the statues around the main portal on the inside of the church. Almost none of the original altars had survived, and the windows were mostly redone, also by renowned artists like Marc Chagall, others are even from the last decade, thus quite a nice experience to witness modern stained glass art.\\
 After a short lunch we took the bus over to the Basilica of St Remi, a large gothic church, where the French crown jewels had been kept for quite some time. From an arts perspective the tomb of St Remi is the clear highlight. And then it was time for our tour of the Champagne House of Taittinger. We were told about the history of the place, situated on a hill side on the remains of an old monastery. Long trails and caves have been put into the chalk ground, well suited to let champagne ripen. And then we got to taste a couple of champagnes, among them the best tasting one I ever had. It was really pricy, thus Reyer, Christine, and I decided to get three bottles, among them the almost 200 EUR expensive bottle and have all three of them later during the year. Since we still had about an hour left before taking the tram out to the train station, we saw the last part of the UNESCO heritage, the Palais du Tau, the former residence of the archbishops of Reinms. Here the tapestries of the Banquet Hall and statues from the cathedral are the things to see. And then once we arrived by the train station we had to sit on the bus for another 90 minutes, before we had a very nice dinner at a small restaurant close to the Place de la Bastille in Paris.\\

Reims: Cathedral*****, Basilica St Remi****, Champagne House of Taittinger*****, Palais du Tau*****\\

April 29: Versailles \& Paris\\
Originally I wanted to get to Bourges on this day, but due to the strike the only alternative would have been 6 h bus rides with Flixbus. Since I didn't want to do that, I wondered what we could do instead. Christine suggested to go to Versailles. Since the RER C train line was affected by the strike as well, we instead opted for a trip by metro and bus, starting about half an hour earlier than using the RER. Naturally we were the first ones in line for the palace. Even just when the doors opened the line was already hundreds of metres long. Operating just two security scanners out of the six available does lead to a speedy operation. Anyways we clearly were also the first ones in the state apartments. Thankfully I made sure that this time my white balance was set up properly unlike in 2015. Unfortunately the Room of Peace as well as the whole Queen's state rooms were in renovation, thus we saw about two thirds of the state apartments, albeit without any crowds. Thus we all got out panoramic shots of the hall of mirrors without anybody else. Christine decided for breakfast we should have pastries and the hot chocolate of Angelina (which we did). By the time we arrived at the exit lines had formed for the toilet, the waiting for the palace was over 1 h and the lines were over a km long (just a bit more than 1 h 30 after the opening of the palace). Next we visited the apartments des Madames (the rooms of the Dauphine were in renovation as well), then we decided to see the fountains in operation. \\
We had to find out that unlike the palace, the park is not for free for people below 25 during the weekends. Thus Reyer had to purchase his ticket, and then we were ready to go. On a normal weekend the fountains are switched on for an hour around lunch time, and again for an hour in the afternoon. During peak season fountains can also be seen illuminated during late evenings. Although I knew how to optimise my paths to see the most impressive fountains, and we started our fountain rush just when they started to operate, even with my walking speed I just made it to see all of the fountains. Clearly without preparation and less quick walking people have a hard time to see both sides of the fountains. A pity when you consider how beautiful both of them are. Clearly if you are not that speedy you need a full day, particularly if you want to see the Trianons too, which we decided to skip this time around. After our fountain rush, we had panini and drinks at one of the garden cafes before we went to the court of honour again going to the other pavilion, where guided tours are starting from.\\
 Only two security scanners were working here as well, but with less people queueing. This was my second time time to see the private apartments of the royal family after 2013. The rooms are smaller, but not any less sumptuous than the state apartments and clearly very impressive as well. A clear highlight is the royal opera house, our guide emphasised that the amphitheatre of Opera Garnier is roughly the same size than this spectacular theatre. \\
 After we decided to skip the Trianons we chose to see a bit of Paris instead, opting for Sainte-Chapelle as the clear highlight of the city. As impressive as always, this was the first time I had seen the upper chapel without scaffolding in a decade. One of the most impressive spaces of French gothic, a must-do when in Paris. The chapel is clearly small but still absolutely breathtaking every single time I step food into it. Unlike Reyer and Christine, I didn't want to miss my chance to see Notre Dame. Shortly before a service, the church was illuminated, and i got some nice night shots of the choir and the chapels. Little did I know that this would be my last visit for what seems to be maybe even for the next 10 years. I met up with Christine and Reyer at the Cafe les Deux Palais, getting myself an Irish coffee (it is great, do it if you have time). And then we had all the beef stew at the cafe xyz.\\

Versailles: Chateau***** (State Apartments*****, Private Apartments*****, Gardens*****)\\
Paris: Sainte-Chapelle*****, Notre Dame*****\\

April 30: Bayeux, Caen \& Rouen\\
After being only the two of us again and having dealt with the two days of trains not running, we finally had our first TGV ride out of Paris. Our destination was Bayeux, the home of the famed tapestry of Bayeux from the 11th century, depicting the conquest of England by the Normans led by William the Conquerer. The tapestry had been housed in the cathedral for the longest time, it is assumed that William's brother, who was the bishop at Bayeux, commissioned the tapestry. The tapestry is an almost 70 m long, 50 cm tall embroidered cloth, housed within a glasses case to keep the temperature constant. The story starts with event surrounding the last years of Edward the Confessor, his death, followed by the reign of Harold II Godwinson as King of England, and the preparation for the Norman invasion and the victory of William the Conqueror in the Battle of Hastings. A funny little detail is a star with a streaming tail, assumed to be Halley's Comet. After a visit of the tapestry we walked over to the cathedral, another nice gothic large church. And then back to the train station, arriving only a short while about 50 mins later in Caen.\\
 By that time it started to rain really heavily. Unfortunately Reyer didn't think of bringing an umbrella on this trip and neither a hat. My jacket had a hat, thus I gave him my umbrella. Due to our height difference, clearly only one of us could use the umbrella. As a result both of us were quite soaked once we arrived by the Abbey of Men. I hadn't had clothes wet like this since my visit of Stonehenge. While the church is large the decoration is more or less standard, but the interesting detail is the tomb of William the conquerer. By the time we got out the rain had slowed down considerably. We walked around the ruined church of St-Etienne-le-Vieux, and proceeded to St-Sauveur another gothic church in flamboyant high gothic style. The castle of Caen is impressive from the outside, since we were short on time, and reviews said the interior would be plain, we walked over to the train station, stopping in the courtyard of the Hotel d'Escovile and the church of St Jean with its unfinished square tower.\\\
  And off by bus to our last stop of the day: Rouen, the capital of Normandy. We shortly walked through the Foyer and Stair of Honour of the city hall, walked around the very impressive church of St-Ouen (unfortunately closed on this day), same for the church of St-Maclou. Our first large stop in Rouen was the large gothic cathedral with a nice baroque high altar, as well as impressive stone-carved tombs in the choir. The facade is considerably broader than usual for French cathedral with uneven towers, you might have seen Claude Monet's painting of it. The large square tower is rather untypical for French cathedrals, this style is more reminiscent of English cathedrals. The flamboyant outside of the high gothic Palais de Justice is one of the most impressive secular medieval buildings I have seen in a while. Then we admired the modern church of St-Jeanne-d'Arc, built on the square where she was burnt, the roof having a shape of a flame. Integrated into the modern church are windows of a former renaissance time church, which was completely destroyed in World War II, while the windows had been put at a safe place. After a short stop in the courtyard of the Hotel de Bourgtheroulde (nowdays a luxury hotel), we had dinner and then took the last train back to Paris.\\

Bayeux: Tapestry of Bayeux*****, Cathedral****\\
Caen: Abbey of Men****, Saint-Sauveur****, Hotel d'Escoville***, Saint Jean**\\
Rouen: Hotel de Ville*, St-Ouen*** (outside), Cathedral*****, St-Maclou** (outside), Palais de Justice*****, Sainte-Jeanne-d'Arc****, Hotel de Bourgtheroulde***\\

May 1: Amiens \& Chartres\\
Reyer and I started the day by dropping our luggage at Gare de Lyon before getting to the Gare du Nord. There our train was delayed by about half an hour, seems no train driver was available, maybe an aftermath of the train strike. Once we arrived in Amiens we walked the couple of minutes to the cathedral. The sculptures of the portals are outstanding, once brightly painted, nowadays the colour faded away almost completely. The original stained glass windows were lost, but the structure is considered the most beautiful high gothic cathedral of the country. The choir screens are very nice, most of the interior altars are from Baroque times. The belfry of Amiens is part of world unesco heritages as well, standalone a bit interesting, but nothing i would consider outstanding. Once we arrived by Gare du Nord we rushed to Gare de l'Est to catch the metro to Montparnasse, since the Gare du Nord metro station was in renovation. Once we arrived in Montparnasse we got a bit confused as which hall our train would leave from, we still got on the train in time just with only a few minutes to spare, but we made the transfer a lot shorter than SCNF or even google would suggest. \\
Chartres is one of the giant early gothic cathedrals in France, it seems far too big for the small town. By now the renovation had been finished, compared to my first visit in 2010 the walls had been cleaned considerably, and the inside had been repainted in beige. The choir screen is nice with tons of sculptures, but the most precious pieces of the cathedral are the absolutely beautiful and outstanding medieval stained glass windows. Nowadays after renovation again in their full shiny glory, detailing multiple stories of the bible, full of depictions of holy figures as well as old bishops, unmatched by most other churches I have been to. If you are a fan of stained glass, either visit Sainte Chapelle or this cathedral. While Sainte Chapelle impresses just by being absolutely dominated by the windows and almost no walls, in Chartres the glass windows are not that dominant, but the church is a lot larger, so there is a large variety of windows and stories to see. Having made it to Chartres earlier than I anticipated we also took an earlier train back and Reyer and I appreciated having another nice steak before getting on our TGV.\\

Amiens: Cathedral*****, Belfry**\\
Chartres: Cathedral****

\section{May 18--May 21: Northern Italy \& San Marino}
\label{2018:Italy}

Why I go to Italy this time: I planned to see the city of Bergamo for quite a while, I also wanted to go to San Marino, particularly after my office mate Rickard told me how much he loved to go there (OK I heard also opinions that it is a tourist trap). But time to find out myself. After previous considerations to go to San Marino in 2014 from Bologna,or later on from Venice never came to a successful conclusion, this time I made sure to have everything set up. Interesting fun fact: Pentecost Monday seems to be unknown in Italy, thus on Monday all buses and train run on a work day schedule.\\

Co-travellers:\\
Amin: Our first trip of this year, and once again to Italy combined with a micro state (Riju joked whenever I think about Italy I either ask him or Amin to join). But at least San Marino's claim to fame is, that it is the first Republic -- so maybe a bit more to it.\\

May 18:\\
We took the last train out of Geneva, once again staying close to the Milano Centrale at Hotel Bernina. This time we also had a quick Kebab before getting some sleep.\\

May 19: Pavia, and Brescia\\
We started the day at the Certosa di Pavia an ancient monastery which was partially in renovation at the time. The facade of its church is richly decorated, both with geometrical patterns, statues, marble, you name it. The interior is just as impressive, with tons of frescoes, murals, mosaics. Unfortunately it is not allowed to take photos, not that it is explained besides a written post next to the church door, which I overlooked. I got a very unfriendly reminder pretty quickly though that this is the case. The museum inside the former cloister is pretty nice too, with sculptures and reliefs from the former cells, as well as parts which had been taken off from the exterior for conservatory reasons. One studiolo of the monastery had been kept in its original form as part of the museum. Then we took the train back to Pavia, where we first saw the huge Duomo. This cathedral is huge and had been in construction for centuries, its size is impressive, but the inside is sparsely decorated, just as for other Italian towns the cathedral is not the highlight of the churches. After stopping for coffee on one Piazza we continued in the basilica of San Teodoro. The church was built from bricks, which is clearly visible on the inside, in the choir section old frescos from medieval times are conserved, and old Roman frescoes are on the floor. San Michele Maggiore is similarly built using bricks, but more of the medieval decoration can be seen, people set up the church for a concert of some kind, thus the crypt was closed. After a short stop by the baroque interior of Santa Maria del Carmine we stopped by the most beautiful church we saw in Pavia -- San Pietro in Ciel d'Oro. The clear highlight of this church is the sarcophagus of the holy Augustine, impressive all over depicting scenes from the bible, the life of the holy man, as well as further rich decoration in several floors. And we jumped on a train taking us to the town of Brescia. After a short lunch snack we had a short look into the baroque church of Santa Maria dei Miracoli. Unlike in most other baroque churches the ceiling, including the dome were left in plain white - I didn't find out if this was done intentionally or if they ran out of money during construction, or if the original decoration had been destroyed. While walking through old town we realised that this was the day of the Mille Miglia vintage car race. The town was full of little stages, and with time more and more vintage cars arrived and they were parked in prominent spots and the fans of the race started to come in as well for the big celebrations. The baroque church of San Maria Della Carita is built in an octagonal shape with an annexed chapel, replicating the interior of the Mary's house in Loreto. Brescia has been an important town already since Roman times. The remains of the capitoline temple and the theatre are still astounding, one of the oldest sanctuaries from the republic of Rome can be visited, going about 3 metres below road level of nowadays. Unlike in many other roman ruins, paintings and remains of the fries, but even the marble decoration is almost intact in two rooms of the sanctuary. By virtual reality you can bring the Capitol alive again as well. The floor in the main hall of the capitoline temple is still the original from Roman times. Even nowadays the square still holds a holy space in the form of the church of San Zeno. Besides the Roman ruins Brescia is famous for the Monastery of Santa Giulia. The monastery complex contains three churches, several cloisters, a really interesting museum with an ancient cross as old as the 8th century, as well as the remains of an old roman house. Afterwards we visited the old and the new cathedral and a modern art exhibit by Palazzo Martinengo. On this day the Mille Miglia race finished, thus the town was full of vintage cars, and little concerts and celebrations here and there, so really nice to spent an evening at.\\

Certosa di Pavia: Certosa*****\\
Pavia: Duomo***, San Teodoro****, San Michele Maggiore****, Santa Maria del Carmine***, San Pietro in Ciel d'Oro*****\\
Brescia: Santa Maria dei Miracoli****, Santa Maria Della Carita***, Capitol****, Roman Theatre****, San Zeno**, Santa Giulia*****, Duomo Vecchio****, Duomo Nuovo****, Palazzo Martinengo***\\

May 20: Bergamo, and Cremona\\
We walked through the lower town and took the cable car up to the upper city where we started with the very beautiful Baroque cathedral before heading to the even more beautiful basilica of Santa Maria Maggiore whose ceiling is filled with many frescoes and hundreds of little sculptures. The third little church on this square is the nice Cappella Colleoni with a large equestrian funerary monument. Next we enjoyed some city views from the tower of the Palazzo della Ragione. A little old Roman water reservoir can be found at the edge of the impressive city walls, which have been built in Venetian times. We walked around the city walls down to the lower city, saw a couple of fountains and churches on our way before having some pasta. A couple of trains later we arrived in Cremona, first visiting the Battistero and then climbing the cathedral tower. While we climbed the tower they had a last rehearsal of Jesus Christ Superstar (in Italian), which I would suppose was being performed later that day. I actually enjoyed the show knowing the Musical in its English version by heart. The Duomo is a very nice cathedral as well, not only the nice facade but also the interior where almost the whole walls are covered in large scale frescoes. And we had some more pasta and large cups of Gelato.\\ 

Bergamo: City Walls****, Battistero**, Duomo***** (lower church and treasury***) , Cappella Colleoni*****, Santa Maria Maggiore*****, San Michele al Pozzo Bianco****, Sant'Andrea***, Palazzo della Ragione****, Fontana del Lantro****, San Lorenzo alla Boccola**, Sant'Alessandro della Croce**, San Marco**, Sant'Alessandro in Colonna***\\
Cremona: Duomo*****, Battistero***\\

May 21: San Marino, and Milano\\
I asked Amin, if he wants to see quite a bit of Milano too, as this would mean getting up incredibly early. Amin didn't mind doing that, so we were getting up before 5 am, and then on a train to Rimini for almost 3 h, before taking a bus for about an hour until we reach the town of San Marino. San Marino is a fortified town on a hill top in the middle of the Republic. After a quick stop by the church of San Francesco we walked up to the Parliament which is located in the Palazzo Pubblico. Just a few metres beyond that is the classical basilica of San Marino. Then we walked along the walls up to the three towers of San Marino. The towers are in fact little castles, housing an armoury and special exhibitions nowadays. After spending some time at the National museum we had lunch by the city walls, before we made the way back to Milano. There we purchased the ticket for the whole Duomo complex. Once again the roof top is amazing and spectacular, as is the interior of the cathedral, unfortunately the middle nave is always off limits, thus one cannot get a close look on the choir, the main altar or the organ, or the little dome. The excavations are nothing special, once we got to the treasury -- we were informed that they just had closed (literally 2 seconds), and we should come back tomorrow (great if you are not in town then anymore). So I took Amin to the ossuary of San Bernardino alle Ossa, morbid but fascinating. And our last sight was the Galleria Vittorio Emanuele II, where we stopped before getting steak at Milano Centrale and getting home by the last train.\\

San Marino: San Francesco**, Palazzo Pubblico****, Basilica di San Marino***, Rocca della Guaita***, Rocca della Cesta****, National Museum**\\
Milano: Duomo***** (Roof Terrasse*****, Excavation***), Galleria Vittorio Emanuele II****, San Bernardino alle Ossa*****\\

This was the last trip I did with Amin before he moved back to California for his thesis write-up, after having done four trips with overnight stays. It was always great to see him enjoy Europe and its culture over and over again, considering he comes unlike most of my other co-travellers from a different background.\\

\section{May 25--May 28: Catalonia}
\label{Catalonia2018}

Once again Rachel had the idea for this long weekend: Rachel was scheduled to attend a workshop in the Pyrenees. The way to get there leads over Barcelona, and Rachel thought it is fun to spend a day in Barcelona, before sitting on the bus for the 3 h transfer. Seems she could only convince me to join her this time around.\\

May 25:\\
Usually I choose hotels based on location, price and rating, but often just normal style hotels. This time I got ourselves a pretty fancy hotel with balcony, roof top terrace, and roof top pool. We arrived late so no swimming in the pool, but we still enjoyed the bar and had snacks overlooking the city.\\

May 26:\\
I got up early, while Rachel decided to sleep in and checking out the pool and old town later on, until we would meet by Sagrada Familia. I took the train to Figueras instead. There Salvador Dali put up a museum to house his art-work, paintings, sculptures, diamonds, jewellery. All of that and plenty of it distributed over several floors. In fact I had seen the Teatre-Museu previously on my school trip to Barcelona, so I knew i would love to see it again.\\
After surrealist art, the next point focused on the ancient Roman era. Tarragona had plenty of those still conserved, an amphitheatre, a theatre, a circus, even the city walls are largely from Roman times. The cathedral was also a nice medieval building with later baroque additions as well. And the beach was also nice to see.\\
I met with Rachel by Sagrada Familia, unlike my previous visits the crypt had no religious service scheduled thus I finally got to see that part of the church and Antoni Gaudi's tomb. And then we were off to a fancy paella place. If you can choose to have a really expensive paella coming with chicken, pork, or beef. Or for a surplus of 7 EUR get the pasta with a full lobster, what would you do. I at least decided we should go for the lobster. Rachel was amused that I didn't really know how to eat it properly though. It is just not a normal dish for us Europeans. \\
For the second time I witnessed the light and music show on top of Casa Mila, where the roof top chimneys are used as projecting screen for nature, volcanoes, art videos (after a nightly tour through the house). Afterwards snacks are served with sparkling wine. Rachel did enjoy the event too so a good choice to go for it once more.\\

Figueras: Teatre-Museu Dali*****, Sant Pere**\\
Tarragona: Amphitheatre****, Roman Circus***, Torre del Pretori***, City Walls****, Cathedral*****, Forum Romanum***, Roman Theatre**\\
Barcelona: Sagrada Familia*****, Casa Mila (at night)*****\\

May 27:\\
Starting out our day with a long brunch in Barcelona Rachel got on here bus, which brought her to the workshop in the Pyrenees. The museums of Palau Nacional were my first stop of the day. Originally built for a world exhibition the palace had back then even a Throne hall for the king. Nowadays the building is used as arts museum. The Dome hall and the Oval hall are still very impressive rooms though. In Catalunya there are valleys with several old churches of the 12th and 13th century, some of those have been declared world unesco heritage by now. Some of the frescoes were deemed so important that they were cut out and transferred to the Palau Nacional. Thus I got to see the originals which clearly doesn't make the museum a UNESCO world heritage site itself.\\
Afterwards I walked over to the Olympic stadiums and the gardens, before visiting the Casa de les Punxes (far more impressive from the outside), and the first house built by Gaudi in Barcelona, the Casa Vincens. In fact since it was far more quiet, I appreciated it a bit more than Casa Mila or Casa Battlo. Since i had a bit of time left, I went back to old town and saw the remaining three columns of the Temple of Augustus.\\
Arriving at the airport my flight to Geneva was shown to arrive as scheduled. I had a some dinner and waited. Then I thought i could check my e-mails, and what did I see - an email which stated that my flight had been cancelled due to the strike of the French flight controllers in Marseille, and the proposed option for the next flight would be Wednesday morning. Clearly contradicting the airport screen stating my flight was supposed to be on time, I went to the Easyjet counter, where I was informed that indeed the flight was cancelled and I just should ignore the screen. So I had to get into the queue of people trying to get rerouted as soon as possible desperately (two people were working). About 2 h later I finally got to the counter, the last person of the line, since everybody behind me had given up anyway, and I managed to tell the actually really nice and polite lady my situation and she rerouted me over London Gatwick, which meant I would only need to take half a day off (which indeed was approved by my boss quickly, a good thing in a research environment is the flexibility of bosses and administration). I took a taxi to the hotel I was supposed to stay at...only to find out that the hotel had just given its last room to the person who just queued in front of me, who told them he would choose a fancier hotel originally, but then took the proposed room in the email anyway. The person on the reception told me to walk for another km to another hotel, where they might have more rooms available. Once I did arrive at that hotel (just 30 mins past midnight by now), I told the receptionist that I would need to get up at about 4:30, he gave me a whole holiday house with two floors, two bedrooms, a fully equipped kitchen and a large TV screen. Too bad I hardly made any use of it. I prepared myself a last tea, and then got myself less than 4 hours of sleep.\\

Barcelona: Palau Nacional*****, Olympic Stadium***, Jardins de Joan Maragall****, Casa Vincens****, Casa de les Punxes**, Temple of Augustus***\\

May 28:\\
Off to reception just before 4 am, getting a coffee and then back to the airport by taxi. This time the flight did leave on time, and then 3 more hours to spent at Gatwick Airport. Had another small snack while trying to work (not that successfully as the connection was not that amazing), on to my second flight of today and finally back in Geneva with over half a day delay.

\section{June 2--June 10: Great Britain}
\label{UK2018}

Many years ago I considered going to Scotland, mainly Edinburgh and maybe a trip from there. Back then I realised that both Durham and York were quite easily reachable from Edinburgh. Some nice estates are reachable from York as well, but then it would be already better to fly out from Manchester or Liverpool. Looking here and there the itinerary grew and grew until a couple of days were extended to a full week. Clearly this could be easily done by train, so getting a British Rail pass seemed the way to go. Since I didn't find friends who I could convince of my idea, my parents decided to join all a bit with the idea to return a day earlier since hotels in Manchester had gone up in price quite a bit (seems a music festival took place in the city on that date, and many Taylor Swift fans flooded the town too).\\

June 1:\\
The day my trip was supposed to start but it was already the second time in the same week that Easyjet didn't get me to my destination, since on the next morning nothing would enable to get me to Edinburgh early enough for something useful I accepted the proposed departure late Saturday. I called my parents to tell them that I wouldn't make it before late evening the next days and they would need to spend another day without me. We had gotten a Scottish heritage pass as well as tickets to Holyrood Palace so they could do that without me. Typically the one day tickets can be extended to a whole year entrance tickets via a stamp on the ticket. I hoped that they would do that at Holyrood even without me being present, and this was indeed the case, so back to my Geneva flat and waiting for another day.\\

June 2:\\
After spending another day in Geneva, participating in the Trivia world championship (not that successfully though), this time Easyjet got me to Edinburg where i met up with my parents, where we discussed our next day. I suggested they should have a quite morning stopping by Linlithgow Palace while I would do Edinburgh in the morning before getting to them later during the day.\\

June 3:\\
And I walked to the Castle old town enjoying the Royal Mile. St Giles cathedral was open, but not really. I got in and walked around and enjoyed it for what it was, but then a man told me it would actually be closed (strange considering about 10 other people walked around just like myself) and we were complimented out. Once again I felt VERY EXTREMELY welcome in Anglican churches. Seems to be a common theme, officially welcoming everybody, but not really - or only after a donation of 10 British pounds or so - but then still respecting of course the spirituality via not allowing photography. Once Holyrood Palace opened I got my ticket for the day. The palace is pretty decent, not as grandiose as Windsor Castle and Buckingham Palace, but nice for what it is. Some nice tapestries and nice ceilings, as well as old renaissance chambers of Mary Stuart times. Clearly the British Royal family doesn't mind getting paid by UK citizens but that doesn't mean you can take photographs of public funded art due to reasons. But that is handled in a similar manner in Spain, so what do I know. One of the rather curious parts of Holyrood Palace is the old ruined part of the abbey church, clearly fitting a stereotype. Edinburgh Castle is quite overrun for what it has to offer: A pretty decent great hall and a handful of rooms, old prison rooms as well as Scottish Crown Jewels which are though a tad underwhelming. It does look impressive on this old volcanic cone though. And then I got my rail pass stamped at the rail station and got on my train to Stirling. I enjoyed the Great Hall there more than the one in Edinburgh, the ceiling is far less decorated, but I just enjoyed the flair more. Just like for Dover Castle the remaining rooms (bar the castle chapel) are decorated in mock-up medieval pieces. After a short stop by the Church of the Holy Rood we had enough time to stop about an hour by Linlithgow. I had a short look at St Michael's before making my way to the palace. Once one of the main royal palaces in Scotland it burned out in 1746, still the walls are very impressive, one can climb up almost all four towers. My parents had seen the palace in the morning, so they had cake and coffee instead, now we took the train to South Queensferry, where we walked down to the village to have a nice view of the Forth Bridge. This Railway bridge was constructed in the late 19th century almost 2.5 km long. On the other side of the village is a large road bridge. And we walked back to the train station. There we did realize that the train coming over the bridge was delayed by about 10 minutes, but that the train to go over the bridge would arrive in just a minute. Thus we rushed over to the opposite platform, got on the train and crossed the bridge. Having arrived in North Queensferry we got off, crossed over to the other side and got on the train back. In Edinburgh I tried Haggis, actually quite tasty and some local beer. A nice dinner to end the day.\\

Edinburgh: Holyrood Palace*****, St Giles Cathedral***, Edinburgh Castle****\\
Stirling: Stirling Castle****, Church of the Holy Rude****\\
Linlithgow: St Michael's Church***, Linlithgow Palace*****\\
Queensferry: Forth Bridge****\\

June 4:\\
Glasgow: Cathedral****, City Chambers****\\
Fallkirk: Fallkirk Wheel****, Antonine Wall****\\
York: City Walls****\\

June 5:\\
We got up early and had a nice breakfast in a cafe, before walking over the city walls to the ruins of St Mary's abbey, which are located in one of the city parks, then we took a view outside photos of the Minster and waited for it to be opened. Already 5 minutes late into the officially stated opening time, the door finally opened and a Gentlemen posted a piece of paper, where the scribble said come back in an hour, training of the guards and the personnel. Firstly - no apology was given he rudely told one tourist that it would not be a big deal, once she asked if this couldn't have been communicated earlier - OR ONLINE, where they state special opening times (NOTHING was announced there), secondly he said - it was an unplanned training. This was a big FAT lie, other people who queued with us an hour later, told us they had been told not to come back too early the next day by guards, so unannounced my ass. With nothing better to do we walked through old town and over to Clifford's Tower. Unfortunately the tower didn't open its doors yet, so we could only see it from the outside. Then we arrived back at the minster, where a long queue had already be formed. 20 minutes after the scribbled one hour (so in fact 80!!!) minutes late the minster finally opened. Many people complained and were all rudely told off. By the way I wanted to know more about two things during going through the cathedral, and none of the guards knew even a hint of an answer, so seems the guide and guard training was useless after all. Anyways rant over -- the minster is a giant and very beautiful church, unfortunately I cannot comment on views from the tower, due to missing out on 80 minutes of time, I skipped that part. The large windows are one of the most beautiful medieval stained glass windows in England, some windows have been redone in a modern style. The choir screen is nice too. Unfortunately our time was cut short by the as usually very tourist unfriendly anglican church personnel. And once more we walked back over the city walls to the train station, getting over to Durham, after having a small snack in the train station. In Durham we walked up the hill to the university and got our tickets for Durham castle. My parents had been in Durham in the 70s and decided back then to rather see the cathedral than the castle, but then always wondered what they might have missed out on. Returning to Durham now over 30 years later, they were happy to finally get the chance to see it.\\

York: York Minster*****, City Walls****, St Mary's Abbey****, Clifford's Tower***\\
Durham: Cathedral*****, Castle****\\

June 6:\\
Knaresborough: Castle****\\
Ripon: Fountains Abbey***** and Studley Royal Water Garden***** (one ticket)\\
Kilburn: White Horse Walk****\\
York: Castle Howard*****\\

June 7:\\
Bakewell: Chatsworth House*****\\
Lincoln: Castle***, Cathedral*****\\

June 8:\\
Conwy: Castle*****, City Walls****\\
Caernarfon: Castle*****\\
Bangor: Cathedral**\\

June 9:\\
Liverpool: Cathedral*****, Metropolitan Cathedral****, St George's Hall****, Harbour Front Buildings***\\
Chester: St John the Baptist**, Cathedral***\\

June 10:\\
Manchester: Museum of Science and Industry****, Castlefield****, John Rydlands Library***, Cathedral***, St Mary's the Hidden Gem**, St Ann's Church*

\section{July 1--July 16: South Korea: ICHEP2018}
\label{Korea2018}

How this trip happened is easily explained, the largest important conference of high energy physics happened to take place in Seoul in 2018, I was among the people selected giving presentations on behalf of our collaboration. Naturally if one flies all the way to South Korea, it is only natural to spent a couple of more days in the country and explore it a big. Since I had a couple of friends and colleagues joining on this trip it was fun to take them to places too. In fact I did a trip within a trip with two other fellow-physicists who I only got to know during the conference.

Co-travellers on the first part:\\
Ulrike: a CERN fellow within the LCD group from Germany as well. \\
Eva: CERN stuff from the DP department but with strong ties to LCD and in fact working for CMS, which obviously holds a special place in my heart too\\

The plans:\\
We all were very excited to go to South Korea and since we have also many activities done outside of pure work, we decided to spend a tourist day before the conference, and clearly do other activities and dinners together. A complication which we faced is that in South Korea a concept of Vegetarian food is not that common, and since Eva is a vegetarian we had sometimes a bit of troubles to find a place which had the option. Most of the times they all then assumed we all would prefer a meat less diet. Not that this was a bad experience, since once a vegetarian option was on the menu, it was in fact tasty.\\

the trip within the trip to Busan:\\
just like I did in China from Beijing I had a side trip pretty much on short notice happening. Stephanie and Steffen, friends of Ulrike from her time at ATLAS, had already thought of getting a bit out of Seoul to the South East of the country to Busan. xyz, a professor who I worked with a couple of years earlier, with roots in Korea suggested I should try to see the ancient town of Gyeongju and its surroundings with many artefacts of the Goryo (xyz) kingdom. Since our high speed train stopped on this city too on our way to Busan, it was easy to get both Steffen and Stephanie on board for this place too.\\

July 1:\\
Oil Wells of Iraq*****\\

July 2:\\
Arriving when a typhoon hits the area can be interesting. I took a bus from Incheon airport to Seoul through heavy winds and large puddle of water and heavy rainfall, but guess that's what you expect in such conditions, but then the walk to the hotel was short, and I just went to bed right away, it was about 9 pm at this point anyway, so good enough to get some sleep. \\


Pakistan: Himalayas and Glaciers*****\\
China: Taklamakan Desert****\\

July 3:\\

The next day we realised that the only other international visitor on that tour through the modern palace besides us three has in fact been a physicist too, who attended the same conference like us.\\
After all of that we had seen enough of palaces thus we walked over to one of the city gates and the market area by Sungnyemun with little shops, little roads, and many food options.\\

Seoul: Changdeokgung Palace***** (Garden*****), Changgyeonggung Palace****, Deoksugung Palace**** (both traditional and modern palace), Sungnyemun City Gate****\\

July 4:\\
Seoul: Unhyeongung Palace**, Jongmyo Shrine***, Gyeongbokgung Palace****, National Palace Museum***, Gyeonghuigung Palace***, City Wall and Heunginjimun Gate****, Seolleung****\\

July 7:\\
Seoul: Lotte World Tower Observatory*****\\

July 8:\\
Paju: Imjingak Park****, Dora Observatory \& DMZ Tunnel****, Dorasan Station**\\
Seoul: Bongeunsa Temple***, Seoul N Tower*****, City Wall****\\

July 12:\\
Suwon: Hwaseong Fortress****\\
Gwangju: Namhansanseong Fortress****\\


July 13:\\
The starting day of our trip within the trip. We all had booked our rooms in the hotel in Busan separately, maybe not the smartest as we all could have had a large room for three too, but who cares, we also had gotten our train tickets three days before to be sure that we would face no bad surprises. The poor lady at the counter had troubles to understand our pronunciation of the train stations we wanted to move between too, but after all we managed to get it done, so early morning we met at the station got a snack and jumped on the Korean high speed train. Once we arrived in Gyeongju we got on the bus to Bulguksa Temple, one of the oldest and largest traditional Buddhist temple complexes in Korea. Very beautiful in a nice setting, also a bit of much needed shade, a lot of nice halls with wood carved statues and traditional architecture.\\
We finished a bit earlier than I thought so we went back to the town centre of Gyeongju to visit a market and getting a lunch snack. Since we had reached the temple an hour earlier than I thought (google maps didn't think we would make the first connection), I added the traditional Yangdong Village on our agenda. This village is one of the very few which is still conserved almost completely in its old traditional form with loads of wooden houses on a hill side. Since it was so hot (beyond 40 degrees and very humid), almost no other tourists made it there, particularly since we had to walk 50 minutes from the bus stop on the main road to get there. Coincidentally the only other tourists were three fellow Germans. We walked around for about an hour, watching some of the traditional houses from the inside and also getting something to drink, and then back by bus to Gyeongju.\\
There we went through parks which had loads of Royal Tombs of the former dynasty scattered all over, and an old observatory. Only at this point Steffen realised that in fact this was his second visit to Gyeongju, which had been the social program point on a workshop he attended previously a couple of years earlier. By that point we clearly needed to cool ourselves down, so we got into a local cafe getting their local cold freezing cold drinks enjoying the air conditioning as well.\\
Back to the train station and onwards to Busan, where our hotel was very close to the train station as well. And we walked over to Busan Tower which had though closed down for the day, but from the hillside we had also quite a view of the town, a park, and the bay. And then we did find a place which still served food and even plenty of vegetarian options for Stephanie, and good food options for myself and Steffen too.\\

Gyeongju: Bulguksa Temple*****, Yangdong Village****, Tombs \& Cheomseongdae***\\
Busan: Busan Tower**\\

July 14:\\

After having a traditional Korean breakfast (that's a no to sweet pastry etc which is what we Europeans typically have, but yes to soup, rice, or other ingredients), we took the bus out to Haedong Yonggung Temple. A temple sitting on cliffs overlooking the sea opening up from an adjacent forest. Indeed a very nice setting, and many locals were around to do exactly the same. Then we walked along the coast through the forest passing another small village temple, and a harbour with a rusty fisher boat. And then we took the bus halfway to the large sandy beach of Haeundae, where vapour clouds had formed within the city, which dissolved though while we walked through the sea water. We continued our walk along the beach up to the Nurimaru APEC House, which was a modern style congress centre set up for an APEC meeting in the early 2000s. Quite interesting to read up on that, and then we walked along the bay and got into a cafe for late lunch snacks and drinks cooling ourselves down in the air-conditioned area.\\
We looked into google which option was suggested to get to the train station: suggestion was to take the bus, since the metro line took quite a detour to reach Busan main station. We also made sure to add an additional 30 minutes on the projection just to be on the safe side. And this was indeed very needed, the traffic jam prediction from google for Busan failed totally on that very same day. With only minutes to spare we ran to our train and got on it just about 1-2 minutes before the projected exit time, and trains in Korea leave on time. And we had a large extensive last dinner together in Gangnam. Clearly this was the first and so far last trip I did both with Steffen and Stephanie. They flew out early morning the next day, while I had my flight scheduled around mid night the next day.\\

Busan: Haedong Yonggung Temple****, Haeundae Beach****, Nurimaru APEC House*****\\

July 15:\\
Seoul: National Museum*****, Myeong-dong Cathedral*, Dongdaemun Design Plaza*****, Olympic Park****\\

July 16:\\
Iraq: Mesopotamian Marshes****

\section{August 5: Diavolezza \& Chur}
\label{Diavolezza2018}

Pontresina: Diavolezza*****, R\"athische Eisenbahn****, Kathedrale Chur****

\section{August 12: Fort l'Ecluse}
\label{FortlEcluse}

Riju and I decided spontaneously that we should do something after an uneventful Saturday morning. Since it needed to be close we opted for the Fort l'Ecluse. This fort guards the only natural entrance from France into Switzerland close to Geneva sitting on a small canyon between the Vuache hills and the Jura mountains. The fort had originally been founded by the Savoys, extended by Vauban but then destroyed by the Austrians back in 1815. Between 1816 and 1828 the fort had been rebuilt. The fort is distributed over two sections. The lower fort is just next to the Rhone with many casemates a museum explaining the history of the fort, and another section dedicated to the bats which live in the rooms and cellars of the fort. Some of them endangered and thus the halls are closed off during some time of the year. Afterwards we started climbing up through stairs within a tunnel up the upper fort. There most of the fort had been closed off since the Fort is currently not safe to be visited, but some of the bastions were open and gave us fantastic views of the Rhone valley, and the Alps with Mont Blanc in the distance. There is also a Via Ferrate which can be used to climb from the lower to the upper fort. I have never done it myself, but several of my friends and colleagues did and they claim it is also suited for not so experienced people.\\

Leaz: Fort l'Ecluse****

\section{August 26: \"Oschinensee \& Bern}
\label{Oeschinen2018}

Kandersteg: \"Oschinensee*****, Bern: M\"unster****, Altstadt with Zytglogge \& Fountains****

\section{August 31--September 9: Germany}
\label{Germany2018}

At the time when we booked this trip, Reyer had never been to Germany in all his years in Geneva. By the time we did this trip Reyer had been on a workshop in Hamburg, but anyway for work purposes. Chris arrived only mid of this year. A newbie to most of Europe he decided to tag along, although he received a warning by Riju in advance. I regularly visit Bavaria. This time I put an itinerary together which was supposed to cover Bavaria as well as parts of Eastern Germany and Berlin, including Saxony-Anhalt which was one of the German states I had never been to.\\

Co-travellers:\\
Reyer and Chris B, both Californians doing their PhDs at UC Davis at the CMS experiment. Reyer has already quite an experience travelling with me, this time bringing his own umbrella along. Chris and Reyer will be the drivers on this trip, both already excited to see how fast our car will be able to go.\\

August 31: Flight to Munich:\\
Chris booked a couple of weeks after Reyer and I, by that time flying Easyjet was much more convenient than using Lufthansa, thus he arrived in Munich a couple of hours ahead, while me and Reyer took one of these delayed Friday flights out of Geneva. He also would stay one day longer in Berlin. Once having arrived in Munich it took almost an hour to get to the city (about a decade ago the mayor promised a fast S-Bahn would be built, but that never came even close to planning), and then we just had Kebabs and that was it.\\

September 1: Munich\\
Starting our day out by the largest palace of Germany, the Munich Residenz. After about a decade of renovation FINALLY the royal apartments had been reopened in 2018, at the expense of the Steinzimmer and the Kaisersaal (Hall of the Emperors) which were now being renovated. I was very happy to finally see these classical halls, planned by Leo von Klenze. The Throne halls of the king and the queen were particularly impressive, held in gold and white. The halls of the Nibelungs had been already built with the intention of being open to the public, romanticised frescoes of the late 19th century. The Baroque and Rokoko ``Rich Rooms'' show of the opulence of Bavarian Baroque which is closer to Italian than to French style. The treasury is full of artefacts spanning several centuries, don't forget to see the Cuvilies theatre as well. After my usual short lunch stop getting some of Grandma's meatballs we had a short look into some of Munich's churches before getting to Nymphenburg palace, the summer palace of the Bavarian rulers. The by far most impressive room of the palace is the Steinerne Saal (Stone Hall) with a large ceiling fresco and many chandeliers etc, some Chinese Cabinets and bedrooms. The heavy rain stopped us from enjoying most of the gardens, but we didn't want to miss out on the four little garden castles, among them the Amalienburg with the silver-blue Hall of Mirrors, or the Magdalenenklause whose chapel is decorated with sea shells, sea stars, and sea snails. Then we got our luggage at our hotel and off to the airport to get our rental car, quite a large SUV. Unfortunately other cars are not really suited for Reyer with his over 2 m. On our way to F\"ussen we got into an hour long traffic jam since a Porsche driver crashed his car (the driver seemed to be only mildly harmed). Once we arrived in F\"ussen we headed for Krone, unfortunately they were completely full so we ended up at the Kornhaus instead, still nice food.\\

Munich: Residenz***** (with Treasury***** and Cuvilies-Theatre****), Theatinerkirche****, Schloss Nymphenburg***** (with Amalienburg*****, Badenburg****, Pagodenburg**** \& Magadelenklause****), St Michael****, Frauenkirche***\\

September 2: Castles and Palaces in the alps and Regensburg.\\
After having an excellent breakfast buffet at the Luitpoldpark Hotel we had to queue about 25 mins to get handed over our reserved tickets for Neuschwanstein. I wonder how long it can take to check for those, but seems 3 mins per reservation is about the norm. Anyways we made our way up, naturally in bad weather, at least not in a snowstorm I thought. This time at least one could see the castle from the Marien bridge, but kinda disguised in fog and mist. Above it all unlike last time where we were told things about each room of the castle this time the guide only bothered to talk about four!!! of the rooms only - the indeed impressive throne hall, the bed room, and the singer's hall. But she couldn't be bothered to have any word about any of the other rooms, the bedroom was soon up for renovation and almost everything was covered in white sheets. Considering that it DID indeed feel like a vastly disappointing tourist trap nowadays. NOTHING really NOTHING about the magic the tour of this place once had. Oh yeah did I forget that photography is not permitted. While in early times the Bayrische Schl\"osserverwaltung (Bavarian palace administration) came up with the unscientific ``but it destroys the art'' excuse of a reason, this time they claim it disrupts the flow of the guided tours. I wonder how every other country manages otherwise, but maybe they have the same reasons like Austria. At least still feels like they pretend phones are at the same state like in the early 90s. Then we stopped by Wieskirche, where Sunday Mass just finished, thus you could still smell the candles and olibanum. The church was a tad fuller as usually, but still very great to see. And on to Linderhof palace. Unfortunately the Venus grotto was in renovation, but this time our guide was really great. I myself always prefer Linderhof to Neuschwanstein, and once again our guide confirmed how much more fun, cozy, and entertaining of a place Linderhof is. Unlike Neuschwanstein Linderhof was finished during Ludwig's II reign. Not as impressively situated like fairytale castle it is rather a summer pleasure palace type. The garden is sizeable with a Moroccan House and a Moorish Kiosk both originally constructed for a world fair in Paris and later bought by the King for his park. The fountain in front of the palace goes of every half an hour for about 1-2 minutes. After getting some sausages we had another short stop by the nice monastery of Ettal. About 90 minutes later we arrived by Regensburg old town. We had a stroll through the nice squares and old houses and stopped by the Gothic cathedral. Some of the stained glass windows of the Dom are from the 13th century and statues are also from medieval times. We crossed the Danube for some nice old town views with the Dom, and then on to our hotel. Back there we opted for an all you can eat buffet at a Chinese-Vietnamese restaurant, which also offer the odd choice of meet from Zebra to Ostrich to Kangaroo and a couple of other Asian cocktails, interesting but tasty selections -- enough to make us happy for sure. And these were the first two UNESCO world heritage sites of this trip.\\

Schwangau: Schloss Neuschwanstein****\\
Steingaden: Wieskirche*****\\
Ettal: Schloss Linderhof*****, Monastery****\\
Regensburg: Dom***** (cathedral), St Johann***, Old City Hall***\\

September 3: Danube Gorge, and Bamberg\\
The Liberation Hall is a classical style Rotunda commemorating the liberation of Germany from French influence among others, celebrating the German states as well as the generals who won crucial battles (among them e.g. Waterloo). It dominates the town and the Danube Gorge. From Kehlheim there is a ferry to reach the Weltenburg Monastery via the gorge in about 50 minutes. On our way we had the Weltenburger Klosterbier, the oldest monastery brewed beer, technically not the longest running still existing brewery, but the one by Tegernsee is not attached to a monastery anymore. The gorge itself is really nice with many rock walls nice forrests etc. Once we arrived at Weltenburg we first stopped by the late Baroque church, built by the Asam brothers, before having lunch at the monastery court. We opted to taste the Weltenburg monastery liquor as well, and we all enjoyed it. Our common friend Leona requested us to get her some J\"agermeister on our trip, but since all three of us agreed this liquor was so much better we got her this one and a bottle of J\"agermeister anyway, only to find out that she absolutely hated this one (our bad I guess). On the way back it started pouring heavily, so we enjoyed some coffee and tea while sitting on the lower deck inside. Then we drove along the Danube for another 20 km before we reached the greek mockup temple of Walhalla. Here the Bavarian king erected a temple to celebrate all significant Bavarian people, later this was extended to celebrate all famous German figures, among them Gauss and Einstein. On the village of Memmelsdorf is the summer palace of the bishops of Bamberg, the Schloss Seehof. Unfortunately the palace was already closed for the day, but the fountains and cascade were running just at this time instead. We might have crashed a couple of photos of a wedding party which took some photos at Seehof then, we did our best to avoid being in their photos at least. Then we dropped our luggage in Bamberg's hotel before walking via the old city hall to the Dom of Bamberg. This cathedral is one of the few medieval imperial cathedrals built in Romanesque style. It houses the tomb of German Emperor Henry II as well as the famous Bamberg rider. Across the cathedral is the archbishop's palace with a vast Rose garden where we spent some time in the evening sun. We found a nice restaurant where we had some good food. But then we wondered what Cold Duck in a Boot (yes that is the correct translation of Kalte Ente im Stiefel) would be. The waitress told us it is a type of Sangria which comes in a pitcher in the shape of a boot. Clearly we had to try, quite enjoying it anyway.\\

Kehlheim: Liberation Hall****, Danube Gorge*****, Weltenburg Monastery****\\
Donaustauf: Walhalla****\\
Memmelsdorf: Schloss Seehof**** (only park with switched on fountains)\\
Bamberg: Dom*****, Rose Garden****, Old City Hall***\\

September 4:\\
W\"urzburg: Residenz***** (with guided tour), Dom****\\
Bayreuth: Margravial Opera House*****, New Palace****, Eremitage*****\\

September 5:\\
Eisenach: Wartburg*****\\
Sch\"onstedt: Nationalpark Hainich****\\
Naumburg: Dom****\\

September 6:\\
Dessau-Rosslau: Bauhaus Masters' Houses***, Schloss Mosigkau****\\
Oranienbaum-W\"orlitz: Tabacco Factory Museum*, Schloss W\"orlitz***** (Guest Wing**** and State Apartments*****), Island Stein with Villa Hamilton*****, Park of W\"orlitz*****\\

September 7:\\
Wittenberg: City Church***, Palace Church****\\
Quedlinburg: Castle \& Collegiate Church****\\
Berlin: TV Tower****, Brandenburg Gate****\\

September 8:\\
Potsdam: Sanssouci***** (Schloss*****, New Palace*****, Orangery Palace*****, New Chambers*****, Picture Gallery*****, Roman Baths****, Charlottenhof Palace****)\\
Berlin: Philharmonie*****\\

September 9:\\
Berlin: New Museum*****, Pergamon Museum*****, Dom****\\

And this was the trip with most world unesco heritage sites so far, with 14 in total, beating even my 3 week trip in China. On top of that this number could have been very easily extended adding the modern housing estates of Berlin in, but I myself doubt I would really enjoy those on my own without an expert giving us the details that we could enjoy its significance in the proper context.\\

This was the last trip out of six trips with overnight stays with Reyer before he moved back all the way to California, watching 32 world unesco heritage sites together (second to Riju, excluding family members, who got a head start of about 20 years). And seems the trips were fun enough, that Reyer does come back to Europe for trips from time to time.\\

\section{September 21--September 23: Belgium}
\label{Belgium2018}

One more trip with Rachel, having been in Brussels three times before, didn't take much to convince me. This time Rachel came straight from a workshop in Germany.\\

September 22: Antwerp, Mechelen \& Brussels\\
Once again I got up early to do a morning adventure on my own, leading me to Antwerp for the printing workshop of Plantin-Moretus. There the old printing machines plus the old prints themselves were all on display. The rooms of the noble family itself are very nice to visit on their own as well. And then I went to see the Baroque church of Carolus Borromaeus which I missed out on during my last visit in 2017 as well. On my way to Antwerp I passed by Mechelen, and the tower of the cathedral looked really nice. Since I finished early in Antwerp I had more time to spare, thus I spontaneously got out by Mechelen train station and walked over to the Market Square and the cathedral. Not only interesting from the outside but also a nice gothic style church on the inside. I later found out the tower is also part of the famous collection of Belfried of Belgium and Northern France. Once I was back in Brussels Rachel informed me that she would be a tad late, thus I got myself waffles for lunch.\\
Then Rachel and I did the tour of the Cantillon Brewery followed by beer tasting, and then we got up on the Brussels Atomium. This time they had a nice visual-audio show running once again, and I am totally there for that. And clearly we had more beers and fried food close the Grand Place for dinner.\\

Antwerp: Museum Plantin-Moretus****, St Carolus Borromaeus****\\
Mechelen: Cathedral****, Grote Markt****\\
Brussels: Brewery Cantillon***, Atomium*****, St Catherine**, Grand Place*****\\

September 23: Leuven\\
Another day, another time to discover rain. It was raining cats and dogs. Originally we had wanted to have a slow start with walks through the Great Begijnhof, but then the rain didn't stop so we thought to change our program. One of Rachel's friends studied in Leuven, thus we met for Brunch in a cafe instead of touring the place until the rain would have faded away. Once it did we saw the medieval St Peter's church, followed by another lunch and beer tasting round in a local brewery. Then we did a guided tour of the city hall. From the outside it is very famous for its elaborate flamboyant gothic style facade with plenty of statues and stone carved decorations. On the inside most of the rooms are nowadays restyled in rich Baroque with lots of frescoes, paintings, and tapestries. And then off to the airport and more Belgium beer.\\

Leuven: Great Begijnhof****, Antonius Church**, St Peter's****, City Hall*****\\

This was the last multi-trip I did with Rachel before she moved back to the US for the write-up of her thesis, but it wasn't the last trip with Rachel, but more about that later on.

\section{November 4: Motiers \& Couvet}
\label{Motiers2018}

Did you know that absinthe was developed in the Jura mountains of Switzerland. After being prohibited for quite a few decades only recently it was allowed to produce and sell it again. Reason enough for Rachel to suggest a trip to the Absinthe Museum of Motiers. Since absinthe tasting was involved as well, it was not a problem to get several people on board which were Rachel, Christine, Grace E, Chris B, Kevin, Sasha and myself, all of us somewhat related to CERN. We all got our tickets beforehand, and decided to meet at Boreal for breakfast and coffee. Everybody arrived well on time, and after touring the absinthe museum we opted all for tasting three types of absinthe. We also got sausage and cheese which we were told would go well with it. It was a very nice experience and a nice welcoming snack going with it. Afterwards we had a short look in to the church of the village with its old wooden ceiling, before we had short stroll to the waterfall close to the river. Already on our way we realised that the river had completely dried up, and the waterfall well had no water running down as well. We used our cell phones to shine a bit of light in the little cave, the Grotte de Cascade, next to the waterfall. Unfortunately in the winery of Prieure St Pierre no tours were offered on the day, just wine for purchase. We were also too early for ordering dinner, seems 6 pm was just out of question. Thus we took the train to the next village of Couvet, where at least three restaurants were listed on google maps. Out of those only two were in fact open, and one indeed offered to serve us dinner, pizza, pasta and Flammkueche that is. Overall a nice relaxed day for an early November trip, and what Rachel and I thought might be our last trip together for a while, thankfully she already returned once again in 2019.\\

Motiers: Absinthe Museum*****, Church**, dried-up Waterfall and Grotte de Cascade****

\section{November 17--November 19: Provence}
\label{Provence2018} 

My most short term planned trip ever: Santona was in Geneva for the Heavy Ion runs. She wanted to make a short trip to somewhere, we originally though of Lisbon or Madrid. Unfortunately by the time we wanted to book the price for flight tickets skyrocketed up, thus we thought why not doing the trip by train. We still discussed about details until midnight, switching out possible destinations and decided to go for the order Arles, Marseille, Nimes, and Orange. Getting all high speed train tickets we still had the hotel to choose, so we went for the affordable option of the Ibis budget. Clearly not luxurious, but when booking on the same day, that's the best option, considering between Nimes and Arles as centre of operation.\\

Co-traveller:\\
Santona, Bangladeshi, but doing her PhD in the US, and working on my former experiment CMS. Whenever in Europe, Santona tries to see different countries and places,  e.g. Albania just a short while ago.\\

November 17:\\
Since Santona wanted to get some coffee before leaving we opted for the second earliest train out of Geneva (aka around 8 am). Two transfers later we arrived in Arles. The amphitheatre is one of the best conserved Roman arenas in the world, so we opted for a tour of that. The theatre is quite close to the road, so safe the couple of euros and just watch it from the fences. It still gives you most of the experience. Then we continued by the old cemetery of Alyscamps. While most of the old town of Arles is part of its own world UNESCO heritage site, the church in Alyscamps, St Honorat is part of the trails from France all the way to Santiago de Compostela. A large old romanesque church. Quite dark inside and not full of decoration, but particular in wintery limelight a certain mystique can be felt.\\
 Then we saw the old cathedral in Arles (nowadays the diocese was merged, and the cathedral is only a basilica minor anymore). The portal is richly decorated, the cloister is nice too, the inside offers a couple of tapestries. Afterwards we walked through the Marie-Desiree Smith exhibition they had on display in the old Bishop's palace. Remains of the old Roman forum can be visited as well, the underground storage area of the cryptoportikus, as well as the Baths of Constantine. Both cute but nothing special should you have seen other Roman remains previously. Similar things can be said about the Musee Reattu, but its modern art  exhibit offers quite a couple of nice drawings and paintings. We had to realise that many restaurants had closed for the season already, and we unfortunately had only a few places to choose from which were decent though. And then we made our way to Nimes after enjoying the illuminated Arena in Arles at night.\\ 

Arles: Amphitheatre*****, Roman Theatre****, Alyscamps with St Honorat*****, St Trophime \& Cloister*****, Bishop's Palace Exhibition***, Cryptoportikus***, Baths of Constantine***, Musee Reattu****\\

November 18:\\
Marseille is not high on many people's list when they consider destinations to see in France. But it does offer quite a few nice spots. We started our day getting large cakes and coffee to prepare us for more adventures to come. What is fascinating for myself, is the setting of the city built just by the sea with many lighthouses, old forts guarding the harbour, it even had an Imperial Palace (although that one looked pretty boring from the outside). For someone coming far away from any coastline the idea of a port, particularly an old idyllic harbour such as in Marseille is appealing. The cathedral is in fact an interesting large neo-byzantine church, not full of mosaics as other churches of that style, but with many columns or stone decorations in several geometrical patterns. The outside with two domed towers on its main facade and five domes around the choir is just spectacular.\\
 The abbey of St Victor has several layers with a large upper church and a lower section with several chapels and funery monuments all dating back to Romanesque medieval times. After having lunch in a small restaurant we walked up to the second neo-byzantine church of Notre-Dame de la Garde standing on the highest mountain overlooking the city. From there you have fantastic views over all of Marseille as well as the islands of the Frioul Archipelago with the Chateau d'If fortress. If you have time a boat ride over to the islands in summer time might be nice to do. In November those tours don't take place anymore. We finished the day by the museums of Palais Longchamp. The Palais itself is very eclectic 19th century with a long row of collonades, a large fountain with many layers, topped out by several groups of statues. The arts museum itself is alright, but far less spectacular than the outside. In fact lovers of brutalist architecture might want to check out Corbusier's Unit� d'Habitation.\\
I know about buildings, sculptures, art, also a bit about nature and hikes. I am definitely not the person to ask when it comes to food. Thankfully most of my travel companions can help out with this aspect. Santona was no exception to that rule having chosen a Michelin star restaurant for dinner in Nimes. Indeed a five course meal does come with its price, but it was still on an affordable scale and it was indeed fantastic. And once we had finished dinner we also did a small night walk to get some good view of the clock tower, the Maison Carree, and the amphitheatre.\\

Marseille: Old Harbour****, Cathedral****, Abbey St Victor*****, Notre-Dame de la Garde****, Palais Longchamp****\\

November 19:\\
Having stayed in Nimes for two nights by now, today we explored the town itself. The amphitheatre of Nimes rivals that of Arles, still in a fantastic state and still in use for performances. The romanesque cathedral is old but nothing to get too excited about. The Maison Carree is one of the best conserved Roman temples in the world, in fact it is almost fully intact. The inside from those times is though all gone, but a virtual tour gives you an idea what the temple looked like in antique times. Then we went to the Jardin de la Fontaine. The large fountain with its canals and statues is very pretty, that another old Roman building (in ruins this time) is just next to the fountain adds another idyllic point. A short hike up the hills behind the park gets you to Tour Magne, another relict from Roman times. Once you are up on the tower you have a superb view of the old town.\\
On our way back to Geneva we stopped in Orange. Orange is famous for its Roman theatre, one of the rare places where the stage of the theatre is still fully conserved. Conservations means also regular renovations, unfortunately for us most of the stage itself was behind scaffolding. It was nice though to walk around the catacombs and the seats of the theatre. After a short visit of the baroque cathedral we walked to the Triumphal Arch of Orange, one of the largest one in France, although not as precious as those in Rome or in Benevento. Since we still had time to spare we first checked out the local museum, getting some large hot chocolate by a local cafe next to the theatre before jumping on the train to Lyon and then to Geneva again. All in all a fantastic trip considering we did plan it just on the evening before going.\\

Nimes: Amphitheatre*****, Cathedral***, Maison Carree****, Jardin de la Fontaine with Temple of Diana \& Tour Magne*****\\
Orange: Theatre*** (in renovation), Cathedral***, Arc de Triomphe****

\section{November 25: Freiburg \& Basel}
\label{Freiburg2018}

We are all ready to see Christmas markets. Since Basel was listed as one of the most beautiful in Switzerland we decided to go there. And I suggested to do a detour to Freiburg im Breisgau, since it is so close to Basel and have a bit of German culture in between. Unfortunately I forgot to read the small print which stated that on this particular Saturday the Christmas market would open up only late in the evening.\\

Co-travellers:\\
Riju and Santona, both as excited as myself to have a good pre-Christmas time.\\

Once we arrived in Freiburg we quickly realised that my plan to have some nice local German food and drinks at the Christmas market wouldn't work out, we weren't the only ones having missed the small-print, since we met other friends of us from Geneva in Freiburg by coincidence. Thus I walked a bit through old town to find a Restaurant which would serve also local options (should Santona and Riju want to try some). And we found a nice place which offered for example Sauerbraten (which I chose myself) and also the Black Forrest Cherry Cake. Afterwards we saw the cathedral, the so-called M\"unster originally built by the citizens in gothic style with the transept in romanesque style. The tower is one of the very few big gothic church towers of Germany which was finished in medieval times. Santona and Riju opted for taking their time to climb the tower while i paid for the Choir chapels and then rushed up the tower (and yes I was pretty much out of breath afterwards), and then we jumped on the train back to Basel. The Christmas market in Basel is spread over a couple of squared and roads, the largest part being on M\"unsterhof. We did walk around the minster (clearly by that time it was closed). And then we started to walk along the stands having some Basler Leckerli as well as Raclette, Gl\"uhwein and/or Punsch as well as other nice things. In between we paid a visit to the big Christmas tree in the court of the city hall, and then about 2 1/2 h back on the train to Geneva.\\

Freiburg: M\"unster***** (with Choir***** and Tower Climb*****)\\
Basel: City Hall****, Christmas Market*****

\section{December 8: Burgundy}
\label{Burgundy2018}

This time we considered doing another UNESCO world heritage which was close and accessible to us: the wine region of Burgundy with Dijon as its main centre.\\

Co-travellers:\\
Santona and Chris B are on board, we also convinced two more Davis grad students, Graham (from the US) and YaoYao (her roots are in China) to join us on this trip.\\

We started the day in the Hotel-Dieu des Hospices Civil de Beaune, the former hospital for the poor in Beaune, one of the best examples of 15th century Burgundian gothic style, particularly famous for its glazed-tile roof. The museum describes the former life and activities of the hospice. While the buildings and ceilings are left in their original state, some of the furniture like tapestries or the main altar of the chapel have been moved into an adjacent museum nowadays. Some of the rooms had been refurbished later on, such as the Salon of Hugo, which had been refurbished with Baroque paintings. \\
We continued our trip to Dijon, due to the ongoing protests of the Yellow Vests our exit by the highway was blocked for over half an hour and we were given an improvised speech by a local activist why the protest was important and why we should be mad on the government instead of them. Once we did arrive in Dijon we visited the cathedral. While the cathedral itself is a normal decent sized gothic church, the romanesque crypt is very interesting with domed rotunda as center-peace and many small side chapels with nicely carved capitals on the sides. Then we had lunch with a nice bottle of white Burgundian vine. We originally wanted to visit the Ducal Palace which even should have offered a guided tour of the state rooms on that day but due to the ongoing Yellow Vest protests all the visits had been cancelled and the museum was closed. Instead we saw the churches of Notre-Dame and St Michel. While the exterior facades and portals of the churches were nice, the interiors were rather plain.\\
Thus we decided we had seen enough of Dijon and opted for impromptu visit of the Chateau du Clos de Vougeot. On the way out of Dijon we were once again stopped by Yellow Vest protestors, but this time only to receive candy. Anyways the chateau sits in the middle of the accompanying vineyards. Inside the castle they had a christmas fair going on, with two large decorated christmas trees and two people showing off their owls. The castle itself was pretty nice, the adjacent buildings around the courtyard had an old vine press in operation and showed how the vine had been processed in early times. A nice end of our touristic program before our ride home back to Geneva.\\

Beaune: Hotel-Dieu des Hospices Civil de Beaune*****\\
Dijon: Cathedral****, Hotel Aubriot***, Notre-Dame***, St Michel***\\
Vougeot: Chateau du Clos de Vougeot**

This was also the last trip I did with Santona. Indeed out trips were indeed only covering two months with two day trips and a multi day trip spanning 3 days, but we did manage to see four UNESCO world heritage sites, not too bad for 5 days of touristic outings.

\section{December 28--December 30: Ladenburg}
\label{2018Ladenburg}

December 28:\\
Maulbronn: Monastery*****\\
Ladenburg: Market Square****, Protestant City Church**, Dr-Carl-Benz-Garage***\\

December 29:\\
V\"olklingen: Ironworks*****\\
Saarbr\"ucken: Ludwig's Church****, Schloss***\\
Lorsch: Monastery****\\

December 30:\\
Since Ladenburg is so close to Mannheim and the weather forecast was nice, I convinced my parents to visit the large palace of Mannheim. Having missed out on the big staircase and the knight's hall a year before, this time every room could be visited. 

Mannheim: Palace*****\\
Rastatt: Alexander Church***, Palace*****, Palace Church****
\chapter{Year 2019}
\label{2019}

\section{January 13: Domodossola \& Orta San Giulio}
\label{2019:Domodossola}

Domodossola: Sacro Monte****, Castello di Mattarella***\\
Orta San Giulia: Sacro Monte*****

\section{February 21: St-Cergue}
\label{StCergue2019}

After having discussed for a couple of weeks about wonderful opportunities to do snowshoeing in the Jura, my co-worker Ulrike and I decided to take matters in our hands, choose a good spot for a hike and dinner following the hike later, and then trying to get people on board, organise the trip and the snowshoes itself. We did choose a spot where we could get to and back via public transport but then we realised we had enough cars to make it there by private transportation. Since snowshoeing is a fun things to do it didn't take much to convince many members of our LCD (Linear Collider Department) group at CERN to join. Thus we had a wide range of people: Dominik A, Morag, Mateus, Marko, Tara, Jakob, Eva, Ulrike, Emilia, Thorben, Erica, Dominik D, Magdalena, and Rickard. All in all we covered nations in a wide range including Poland, the United Kingdom (represented by Scotland), Brasil, Slovenia, Germany, Italy, and Sweden. \\

We arrived by St Cergue shortly after sunset with dark trees and a landscape in white snow below a still pink sky. We put on our snow shoes, headlights and off we went up a little hill along a round trip trail. We decided to start on a long round trip trail which could be cut short in case we decided to cut things short. Snowshoeing is fun, not too difficult to do, going along with the flow and just enjoying the night scenery and the forest and the moonlight snowy landscape. Since we did stop here and there to admire the solitude and the night sky we ended up taking the short cut. In such times I regret not having a tripod on me. Might not have been a smart idea to carry it around on the back while snowshoeing, but the nightsky was so clear and free of light smog, for sure one would have gotten out spectacular photos. Still we did enjoy nature. What we didn't consider though is how fast some night skiers were on the trail thus we had to leave half of the trail open at all times to avoid getting into any unfortunate crash. Still definitely a big yes to snowshoeing, and I hope I can do it again sometime (I don't know though of any good places around my current city of Vienna)\\
Having arrived at the restaurant we did enjoy a nice filling cheese fondue accompanied by wine. The poor people who had to drive opted for some non-alcoholic beverages though. And on the way down we did enjoy listening (and maybe singing) to some Madonna songs.\\

St Cergue: Snowshoeing*****

\section{February 24: Chamonix}
\label{Chamonix2019}

Chamonix: Aiguille du Midi*****

\section{March 2: Z\"urich}
\label{Zurich2019}

Swiss National Museum**, Grossm\"unster***, Cloister Fraum\"unster**, Kunsthaus*****

\section{March 10--March 22: Southern Italy}
\label{SouthItaly2019}

March 10: Sunday\\
Bari: Cathedral San Sabino****, Basilica San Nicola*****,  Castello Svevo***\\

March 11 Monday:\\
Matera: Santa Chiara**, Chiesa del Purgatorio***, San Francesco d'Assisi***, Duomo****, Sant'Agostino****, San Pietro Barisano***, San Nicola Dei Greci \& Madonna Delle Virtu with Dali Exhibition****, San Pietro Caveoso***, Casa Grotta****, Santa Lucia alle Malve****, Santa Maria de Idris****, Convicinio di San Antonio**\\

March 12: Tuesday\\
Molfetta: Duomo di San Corrado****, Cathedral Santa Maria Assunta**\\
Bitonto: Cathedral***** (with excavations**)\\
Andria: Castel del Monte*****\\
Trani: Cathedral****\\

March 13: Wednesday\\
Lecce: Chiesa del Carmine***, Basilica San Giovanni Battista***, Chiesa del Gesu****, Roman Amphitheatre with Palazzo del Seggio**, Santa Maria della Grazia**, Basilica di Santa Croce****, Duomo*****, San Matteo***, Roman Theatre**\\
Bari: San Ferdinando**, Piazza Mercantile**, Piazza del Farnese**\\

March 14: Thursday\\
Unlike the previous day weather was bad and rainy. Good that our first part was underground in the giant cave of Grotte di Castellana. We did get a few small snacks at the train station before jumping on the regional train to Castellana Grotte. Since my parents and I love to go to caves I decided right away to go for the large tour. Several big halls form a long system where about 3 km can be visited on a guided tour on the long tour. The visit starts in the first chamber where the vault collapsed a while ago (they called it the Grave). In sunshine sun beams illuminate the floor, we still got enough natural light but rather little waterfalls instead. The Grave itself has a lot of moss covered stalactites and stalagmites scattered all over the floor and walls as well. It is also the only room where they allowed photography (claiming copyright laws). Also the following halls and pathways are very interesting, then one goes through a rather narrow trail for about 1 km before reaching a couple of other halls and finally the white cave. The cave and particularly the roof at this point is full of multiple white shiny stalactites. Other caves might have similar roof tops but usually 10 or 20 m high, here those are just sitting in front of you, which makes it pretty magical, so do the long tour if you can. During the tour which was in Italian we also realised beside the Italian family all other nine people were Germans. \\
And 7 of us gathered below the tiny plastic roof of the train station in heavy rain trying not to get too wet waiting for the next train. Seems 3 of those were students on a budget touring Italy who arrived in Bari on a night train from Bologna the night before. Another lady instead did a rather laid back holiday in Puglia, and she came and went back to Alberobello, which was our next destination in fact. She asked what we had done and what we planned to do in the following days. I told her what I had planned to do and she commented that this was indeed a pretty packed agenda and then she told me where she would eat in Alberobello should we have the time. In Alberobello we walked over to the Trulli, on the way we had one of these traditional houses open for visits, then we walked up through the main street to see the church of Sant'Antonio di Padova. Although the church itself is pretty modern the architect did a good job orienting the architecture close enough to the surrounding Trulli to fit in pretty nicely. Since it was still pouring rain and even started to wind pretty heavily we all felt cold, thus we decided it was time for a longer lunch in the only open restaurant by the main square (clearly March is pre-season). The food was nice but unfortunately we were told that the restaurant was closing until the evening. Thus we walked to the train station again, where we asked to exchange our ticket for an earlier bus ticket. Once we arrived back in Bari we had cake and coffee at a cafe by the Piazza Aldo Moro opposite of the train station\\\

Castellana Grotte: Grotte di Castellana***** (long Tour)\\
Alberobello: Trulli****, Sant'Antonio di Padova****\\

March 15:\\
Napoli: Santa Caterina a Formiello***, San Giovanni a Carbonara*****, Reggia di Capodimonte****, National Archeology Museum*****, Santa Maria del Rosario alle Pigne*, Duomo*****\\

March 16:\\
Caserta: Reggia***** (with Court Theatre*****)\\
Benevento: Arco di Trajano****, Museo del Sannio**, Santa Sofia (church and cloister)****, Basilica San Bartholomeo**, Duomo****, Roman Ruins \& Theatre**\\

March 17:\\
Ercolano: Excavations of Herculaneum*****, Villa Campolieto***\\
Portici: Reggia \& Botanical Gardens***\\
Napoli: Palazzo Reale*****, Galleria Umberto I****, Chiesa del Gesu Nuovo*****, San Domenica Maggiore****, Cathedal Treasury****, Duomo***** (Battistero of Santa Restituta****)\\

March 18:\\
Paestum: Excavations****\\
Napoli: Villa Pignatelli****, Certosa di San Martino*****, Castel Sant'Elmo**, Santa Chiara*** (church)\\

March 19:\\
Pompei: Excavations*****, Santuario****\\
Torre Annunziata: Villa Oplontis*****\\
Napoli: Santa Chiara**** (cloister), Santo Domenico Maggiore**** (church*** and sacristy***)\\

March 20:\\
Positano: Santa Maria Assunta**, Coastal Views*****\\
Amalfi: Coastal views*****, Duomo***** (cloister****, crypt*****, museum****)\\
Salerno: Duomo****\\

March 21:\\
Pozzuoli: Amphitheatre****, Macellum***, San Raffaelo***, San Giuseppe**\\
Bacoli: Lago del Fusaro \& Casina Vanvitellina***\\
Baia: Roman Baths****\\
Lucrino: Monte Nuove****\\
Napoli: Maria di Montesanto**, Michele Arcangelo a Port'Alba***, San Lorenzo Maggiore****


\section{April 6: Murten \& Avenches}
\label{Birthday2019}

Murten: Old City****, Avenches: Old City and Roman Ruins***

\section{April 13--April 17: South East England}
\label{2019SouthEastEngland}

Workshop time: This time I was invited for a talk about particle flow reconstruction at CLIC within the context of the CEPC, the circular electron-positron collider project of China. The workshop was taking place in Oxford, exactly placed the week after one of the many planned Brexit days. Thus I made sure to book a spot on a plane of a Swiss company, since the UK had at least dealt with Switzerland and flights from there. In the end Brexit did NOT happen on this day, and everything went smooth. I also took the opportunity to visit London once again.

April 13:\\
Unusual for me, I took the earliest plane out to London, put all my luggage by the hotel and jumped on the train to Cambridge. A bit ironic to visit Cambridge for the first time due to a workshop taking place in Oxford, but that's just how I roll. I in fact wanted to focus on King's College, particularly due to its large but beautiful chapel, which I was recommended by my friend Riju to visit, who had studied for an exchange semester in Cambridge previously. On my way to King's College I also visited some of the churches on the way. They all pale compared to the beauty which is the King's College Chapel. Elaborate gothic vaults, beautiful stained glass windows all over the walls. Simply breathtaking. Since I still had time left, I went on to tour St John's College, which also has some nice courtyards, a more modern gothic revival chapel, and the famous bridge over the river.\\
A day wouldn't be complete without a visit of one or even better two norman and gothic style cathedrals. The cathedral in Peterborough has one of the best conserved wooden medieval painted ceilings in the UK. Ely cathedral on the other hand is larger, also with a more dominant tower over the facade and a central dome. The ceiling of Ely is painted very elaborately as well, but the ceiling was completely redone in the 19th century (pretty to see, but just not old). Still Ely is an impressive cathedral and the structure itself is medieval. And back to London and King's Cross Station.\\

Cambridge: St Andrews Street Baptist Church**, Great St Mary's***, King's College*****, St John's College****\\
Peterborough: Cathedral****, Guildhall***\\
Ely: Cathedral*****\\

April 14:\\
During my last couple of visits in London I didn't return to Greenwich. Only recently the renovation of the Painted Hall of the Old Royal Naval College had been finished, thus I decided that it would be nice to see that hall again. Since the former Docklands are pretty and just north of the Thames by Greenwich I first took the underground to Canary Wharf, walking along the canals, old cranes, ships, and skyscrapers through the tunnel to Greenwich. The former Royal Naval College was developed by none other than Christopher Wren on the place of a former Royal Palace. It is a really nicely painted festival hall in previous Baroque style, the chapel in the building opposite is nice too. And I had the place pretty much to myself. Similarly visiting the Queen's House which is a former Renaissance palace situated just behind the Royal Naval College. Not much is left from the furniture, but the Tulip Staircase and the floor of the Great Hall are still nice to see, similarly the painted ceiling of the Presence Chamber of the Queen. And then I walked up to the Royal Observatory to cross the Prime Meridian just like any other tourist around me did too.\\
Clearly it didn't take me a full day to see all of that, thus I finally made it to Kenwood House. During my previous visits this house and its vast park have been always on a backup list of possible things to do, but then I never made it. The house is also a nice stately home, particularly the Green Hall and the library. And then it was time to check out old relicts from Assyrian, Babylonian, and Greek Eras in the British Museum.\\
And then it was time to jump on the train to Oxford, checking in at the hotel, and then having a Kebab for dinner by the bus station.

London: Canary Wharf*****, Greenwich: Old Royal Naval College*****, Greenwich: Queen's House****, Greenwich: Royal Observatory****, Kenwood House****, British Museum*****\\

April 15: \\
On my way to the conference place I passed a couple of churches, unfortunately none of them was particularly pretty.

Oxford: St Mary Magdalen*, St Aloysius Catholic Church**, Blackfriars Church*\\

April 16:\\
Oxford does have plenty of beautiful colleges, well, and then there's St Anne's College, which has modern buildings, not that beautiful to see, but then good for actually holding a workshop. The flowers of the inner courtyard are nice too, but then nothing much more nice to see. Clearly the place of the conference dinner was far more fancy. Brasenose College is situated just next to the Radcliffe Camera (visited that one previously in 2014). The chapel of Brasenose has a pretty nice ceiling, also the hall has a nice baroque style wood panelling with old paintings.\\

Oxford: St Anne's College*, Brasenose College****

April 17:\\
Originally I wanted to leave 1 h early to see a few colleges, but then my watch ran out of battery, and in the end I stayed for the full reports. As result I could only see the very close-by colleges, and most of those had mainly their chapels open, they were alright, but nothing to get too excited about.\\

Oxford: Wadham College**, Balliol College***

\section{April 18-April 22: Paris I - sans Paris and without getting any sleep}
\label{2019:Paris I}

How we ended up in Paris this time: First of all, we talk about the long Easter weekend, second point, Riju and I decided we should see also a bit of countryside France, but the best connection to anywhere in France is guaranteed from Paris (using at least public transport). After a longer break since our last multi-day trip Chris decided it is time again to risk another trip with me.\\

Co-travellers:\\
Riju: US-American: We were both aware that this might be our last longer trip for the next while, since Riju sees the end of his PhD and his time in France coming by summer.\\
Chris B: US-American: Finally recovered after last year's trip Chris is ready for a new adventure, also he plans leaving for California by late summer.\\

April 18: Thursday\\
SCNF trains can be late, this time the TGV started about 30 mins late, by the time we reached Gare de Lyon, we had a delay of over an hour. Then we wanted to transfer to the Hotel, unfortunately by that time the Metro didn't run anymore (due to construction), thus we had to switch to a bus. At least we glimpsed a night-view of the Chatelet and the Tour St Jacques. The bus line was not equipped to take over all passengers of the metro, thus instead of the planned 30 mins it took over an hour. By that time it was past 1 am as we finally checked in.\\

April 19: Friday\\
Today we wanted to do the Abbey of St Savin: The issue is, that the bus schedules only two buses each day, thus we either have to get up really early, or we would have less than an hour at the abbey. We decided to rather have more time, but that meant taking one of the earliest TGVs out of Montparnasse. So we got up around 4:30 and made another visit by the hotel reception just about 3 1/2 h after check-in (clerk was confused to say the least). After switch to the bus (and coffee) in Poitiers, we finally arrived in the small village of St Savin. The abbey church is famous for its old painted ceiling, depicting various biblical stories ( Noah, Moses, tower of Babylon, the garden of Eden, Joseph and his brothers, etc etc) - quite amazing to see this paintings from a bit less than thousand of years ago. The abbey itself is not that exciting, but make sure to walk over the old bridge and the river. Frogs clearly seemed to love that place too. Then we had an actually really nice lunch at a restaurant just opposite of the church, with Kirs only costing about 1 EUR (and that after they apologised that they had increased the prices for those just for the season). \\
Then back to Poitiers, where we saw the Notre-Dame-la-Garde (nice painted pillars), the cathedral (your typical gothic cathedral), the baptistery (the oldest still intact church building in France, I loved it, Riju was not impressed at all), and St Radegonde (quite cute) and St Jean de Montierneuf (a bit forgettable). Since we were already close-by we went up to Tour Montparnasse, which offers two advantages compared to the view of the Eiffel Tower: one can see the Eiffel Tower in its full glory, and the view is not spoiled by Tour Montparnasse. \\

Abbey church of St Savin*****, Abbey of St Savin*, Poitiers: Notre-Dame-la-Garde****, cathedral***, baptistery****, St Radegonde***, St Jean de Montierneuf**, Tour Montparnasse*****\\

April 20: Saturday: Provins \& Fontainebleau\\
Provins is typically reachable by train, but the train had been replaced by buses on this weekend. Unfortunately not running that frequently, we only had two options to get there. Missing the first one by just 2 minutes we made it to the second one, and we even caught the bus with a connecting time of 5 mins only. Once we arrived in Provins we had a rustical breakfast at the main square. Seems nobody else was around when we strolled through the Collegiale church, more people were visiting the Tour Cesar. The town can be fully seen from its top, pay attention not to be too close to the bells when they ring each half an hour (quite deafening I can tell). Then we did a round tour along the city walls before stopping for omelettes etc for lunch. \\
And then we transferred by bus, train, and another bus to the Chateau of Fontainebleau. The second large royal palace (after Versailles and before Compiegne) was the main residence during the Napoleonic empires, in fact the room where Napoleon I abdicated can be visited as well. The oldest part of the palace is from Renaissance times of the Valois family, the main part being the gallery of Francois I and the old ballroom. This was my third time in Fontainebleau, this time we managed to visit the Papal rooms, missing out on the private royal apartments instead. The queue was about half an hour long, nothing compared to the 4-5 h lines of Versailles. I like Empire style decorations. Not as grandiose as Versailles, but still magnificent. Then we walked through the gardens before getting back to Paris, where we first paid a sad visit to Notre-Dame (which had burnt down just a week previously), then had Thai food closeby and then walked alongside the Louvre, before jumping on a bus for a late climb of stairs up to the second floor of the Eiffel Tower. Once again we arrived really late at the hotel past midnight.\\

Provins: Collegiate St Quiriace**, Tour Cesar****, City Walls*****, Fontainebleau: Chateau*****, Chateau Gardens****, Paris: Eiffel Tower*****\\

April 21: Sunday: On my own in Bourges\\
On Sunday Riju and Chris started early to make it all the way to Mont St-Michel (just like I had done in 2015), while I FINALLY made it to Bourges (after my last attempt the year previously failed due to strikes of SCNF). It took close to 4 h until I finally reached the cathedral, only to find out that there was a religious service going on. Thus I first did an outside 360 Tour of the Facades and the portals. Afterwards after a short stop for coffee I could visit the cathedral with its old stain-glass windows. I booked the guided tour of the crypt together with a climb of the tower. The view is really nice (though not as spectacular than at other cathedrals). In the crypt they keep the remains of the tomb of the duke de Berry, the remains of the choir decoration as well as the windows of the former St Chapelle in Bourges. All in all definitely worth paying. Afterwards I visited the Palais Jacques-Coeur, a very nice old medieval palace with an interesting exhibition just below the roof-tops. And then I still had time left to see the private collections at the Hotel Lallemant (tapestries, wood-carving, and other sculptures). And then over 4 h to get back to Paris just arriving shortly before midnight this time.\\

Bourges: Cathedral*****, Cathedral Towers****, Cathedral Crypt*****,  Palais Jacques-Coeur****, Hotel Lallemant***\\

April 22: Monday: Le Havre:\\
In order to not have to rush to the hotel and back again in the evening we dropped our luggage by Gare de Lyon and stopped for a long breakfast by Gare St Lazare, and then we took of to Le Havre. Most of Le Havre was completely destroyed in the World War II, and rebuilt betweeen 1945 and 1964 after a plan by Auguste Perret. Unlike other modern cities it doesn't feel ugly at all, but rather harmonic. Parks and fountains and places are put throughout the roads and boulevards as well. The library resembles a Volcano, built after plans by Oscar Niemeyer. The highlight is though the church of St Joseph. Truly amazing and with a very interesting tower resembling a light house, illuminated by multiple coloured rectangular windows. And after getting burgers by the library we got all back to Geneva, just in time for the last 3 rounds of Trivia nailing all 10 Whitney Houson songs (though shaky on The Greatest Love Of All Remix. All in all another fun trip to Paris, though we all ended up sleep deprived. I am very sad that it is my last multi-day trip with Chris and particularly Riju for a while. And not to forget I'll be back in Paris anyway anytime soon.\\

Le Havre: Library***, St Joseph*****, Place de Hotel de Ville***

\section{May 25: Liechtenstein}
\label{2019Liechtenstein}

I talked to Riju for a couple of months that Liechtenstein is one of the countries I have been to, but never took a photo at myself. Clearly it might in fact not be the most amazing place to visit, but it might be still cute to add another little country to his places he had seen in Europe, particularly since it would take less than a day to do it. So that was supposed to be the day. We took the train to Sargans where we got on the bus to bring us to Vaduz. Shortly after passing the border we already got some nice view of the forest covered mountains of Liechtenstein and the Gutenberg Castle by Balzers. Having arrived in the capital of Liechtenstein, Vaduz, we walked up to one of the view points to get a good overview of the town. Then we continued our short hike to Vaduz Castle, the private seat of the Prince. The castle itself, particularly the keep, has roots as early as the 12th century. In the 16th and 17th century the castle was heavily fortified by the Counts of Sulz, and only in the 18th century the area was taken over by the house of Liechtenstein. Although the castle cannot be visited, it is still a nice hike and a nice sight. The cathedral was made into a seat of a bishop to allow the old bishop of Chur to leave his post there while not losing any real privileges. The church is of moderate size and a standard neogothic building with modern style stained glass windows in the choir. Then we thought it is fun to cross over by foot to Switzerland. On our way we walked past a festival and decided to have our lunch there. To your surprise we were told that the first drink would be offered. Very nice and obviously the wine was nice enough to have a second round of it. Then we walked over the old Rhine bridge to Switzerland and later back again. \\
On our way back we stopped in Zurich and visited the medieval Grossm\"unster with its nice modern stained glass windows. Then we walked to Kunsthaus to admire Rodin's Gates of Hell which stands in front of the museum. And then we had the famed gigantic Jumbo-Jumbo Cordon Bleu in Rheinfelder Bierhalle with R\"osti, something I remembered from the good old days when I was a physics student at ETH, definitely a filling dinner. And then three hours back on the train to Geneva.\\

Vaduz: Castle***, Cathedral St Florin**, Old Rhine Bridge**
Zurich: Grossm\"unster****

\section{May 30--June 3: North West Germany}
\label{2019:Northwestgermany}

May 30:\\
Hannover: New City Hall***, Welfenschloss**, Marktkirche***\\
Oldenburg: Palace****

May 31:\\
H\"oxter: Kloster Corvey****\\
Alfeld: Fagus-Plant***\\
Hildesheim: Dom \& Dommuseum*****, Micheliskirche*****\\
Braunschweig: Domplatz****\\

June 1:\\
Bremen: Dom****, Rathaus*****, B\"ottcherstrasse***, Frauenkirche**, Norddeich: Wattenmeer*****\\

June 2:\\
Goslar: Erzbergwerk Rammelsberg*****, Kaiserpfalz****, Marktkirche***\\
Hannover: Herrenh\"auser G\"arten****, Schloss Herrenhausen*, Galeriegeb\"aude Herrenhausen*****, Berggarten****

\section{June 16: Lucerne \& Rigi}
\label{LucerneRigi2019}

Co-travellers: Chris B. and my faithful companions Riju and Rachel. All US Americans, for Riju this is his last day trip with me before he moved back to the US in September 2019. Chris left around September 2019 as well, he wanted to check a few destinations before his dad would come over for a trip through Switzerland (with detours to Austria, Liechtenstein and Italy, can you imagine there was a time when you just could cross borders as you please unlike now in 2021, when you are not even allowed to cross borders if you just leave 2 km next to one). Back in 2018, Rachel had moved back to the US to write up her thesis and graduated. But back in 2019 work trips were still a thing, so Rachel had been back and clearly I took this opportunity to show her Lucerne and the Vierwaldst\"atter See which many people claim to be the most beautiful town and large lake in Switzerland.\\

Why Lucerne: as already noted the place is known for its beauty, and it is not too far away from Geneva, and none of the three had been there before, so off we went.\\

Lucerne is on the outflow of the Reuss river with two beautiful wooden roof covered bridges. Each of them containing many painted panels below the roof. Kapellebr\"ucke has a tilt at a tower with a chapel. Unfortunately Kapellebr\"ucke had been heavily damaged by a recent fire and thus many panels had to be restored. In Lucerne a sizeable section of the city walls still exist, and you can go up on several towers as well. Don't be as unfortunate as one tourist who slipped and broke the glass on his lens. While the ladders and steps are not steep, they are at times irregular. Next to the Reuss in old town is the beautiful Baroque Jesuit church. A bit outside of old town is the weeping lion monument commemorating fallen soldiers. Since we were finished with our tourist program far earlier than thought, we decided to check out boat rides over the Vierwaldst\"atter See, since our day passes would cover boats too. We either could have gone all across the lake, jumping on a train there, crossing through the world's largest train tunnel and back to Geneva, albeit that's around 5 h in the train. Or we could get off Weggis, jump on a cable car getting us up to Rigi, the so-called Queen of the Mountains in Switzerland, and taking then a light train up to Rigi Kulm. Several modes of transportation but very scenic. And we took off by boat over the lake. Since we started our trip early on Chris decided it was nap time, while Rachel, Riju and I enjoyed the view and the breeze instead. In Weggis we made it to the cable car, which offered also a nice overview of the lake. And then we switched to the cog-rail. Having arrived up on Rigi Kulm we did a short walk over the meadows, watching sheep and cows on the surrounding Albs, enjoying the view of Lake Zug and getting eaten up by upcoming clouds. Then we took the cog-rail back all the way to Arth-Goldau where we switched back on the train. Then we did enjoy dinner and drinks on this train before switching in Zurich to our final train of the day to Geneva. For me the trip stopped here but everybody else had to hop on the tram back to CERN making their way home from there.

Lucerne: Spreuerbr\"ucke****, Kapellebr\"ucke****, City Wall****, Jesuitchurch****, Weeping Lion**\\
Vierwaldst\"atter See*****\\
Rigi*****\\

This marks as well the last time Rachel has been on trip with me with five trips with over night stays and two additional day trips, of course hoping that more might follow at some point later on.\\
It is as well the end of an era, losing my so far most loyal travel companion: This marked the last trip I did with Riju, while he was living in France at least. In total we had 14 (sic) trips with overnight stays, additional 10 day trips, and we saw 34 world unesco heritage sites together (the most I saw if I remove family members, 2 more than I saw with Reyer). 

\section{June 22: Vallorbe}
\label{Vallorbe2019}

Grottes de Vallorbe*****

\section{June 23: Lavaux}
\label{Valaux2019}

Since my younger brother had come over from the US after a while my parents planned to come down to Lake Geneva to join us and my sister and my brother-in-law. Since Geneva itself is familiar to all of us by now, we decided to rather have some walks through the vineyards of Lavaux with nice views of Lake Geneva and the adjacent alps. Since we had two cars we put one by the start and the other one by the end of our planned hike. And then we started walking through the hills stopping here and there for family photos. By the end of the trip we had a longer lunch at a pizzeria with ice-cream, espresso, and of course the local vine. Then we parted ways and my dad drove us back to Geneva.\\

Lavaux*****\\

This marks the last time my mum will make an appearance here. Clearly so far we did most multi-day travels together (since my dad bailed out of the Rome trip in 2013). She, together with my dad and my aunt, ignited my love for seeing other places and discovering the world. She clearly paid attention that we get educated about culture and history, but also not take beautiful nature for granted. I cannot express enough gratitude for what she did and meant to me. Thanks mum.

\section{July 28: Glaciers among clouds}
\label{2019:Fiesch}

The alps are always great, and we had a couple of people who didn't see any glacier yet, so time to show them all the Aletschglacier. Chris can also check if the hike would be suited for his dad's upcoming trip to Switzerland.\\

Co-travellers:\\
Ivan:\\
US-American, Ivan will soon graduate, and it is his last weekend in Switzerland at least for this summer.\\

Ulrike:\\
German, after our adventures in Seoul, Ulrike is also up for a hike in the alps. We both have done this hike before, but it's always fun to see Aletsch again\\

Chris B:\\
US-American, Chris wants to check on this hike, if it could be added to his week long hiking adventure which he plans for his dad in a couple of weeks. Excited that coffee-breaks and lunch are foreseen this time.\\

Francesca:\\
US-American, also a physicist working at a CERN experiment. Francesca's dad was born in Italy, thus she has seen a bit of Europe already, but never the mountains and neither glaciers.\\

Jose:\\
US-American, born in Puerto Rico, also a physicist working at a CERN experiment, as well his last weekend in Switzerland for a while.\\

Although the weather forecast wasn't amazing at all, this was the only chance to do the trip. So we still decided to get day tickets and make it our way. Meeting at the assigned time, Ivan and Jose informed us that their taxi would be late, and thus they wouldn't make it in time. Little did they know that SBB had issues as well, and our actual planned train was cancelled. Anyway once Jose and Ivan arrived we still had to buy Jose's ticket and a couple of coffees later we were ready to go. Passing by Lake Geneva, the Lavaux, the Pissevache, and another coffee and pastries stop in Brig we finally reached Fiesch, and after getting already coffee number 3 at Fiescheralp we finally started our hike. The sight was horrendeous, light rain and fog all around us we could only see about 5-10 metres ahead. We didn't miss out horses standing close to the trail and little waterfalls, and once we crossed the tunnel we finally saw something, the glacier and mountain tops with clouds hovering above. We all managed to get down to the ice, it took a bit to convince Francesca that although her shoes were not full fledged hiking boots, it was safe to get down. For some of us it was the first time seeing a glacier. Thus all of us ended up standing on the glacier ice, and this time a glacier cave had opened up. Without a hardhat too dangerous too step inside, but still what we could already see just standing in front of it was magnificent. Then we hiked up to the lakes again, where we stopped at Gletscherstube getting our favourite version of R\"osti. Once we were back at Fiescheralp it was still early enough to make it up to Eggishorn. Although clouds were still covering one side, we could see pretty much all of the glacier, even up to Gr\"unhorn and the Eismeer. We were all impressed to say the least. Once down in Fiesch we had another coffee break, before making it to Brig, getting on the Eurocity from Milan (thus getting dinner on the train unlike for the normal Interregio, which doesn't have a restaurant car -- at least those who didn't fall asleep during the train ride).\\

Eggishorn*****, Aletschglacier*****, Gletscherstube***** 

\section{August 25: Lost between five lakes}
\label{2019:Zermatt}

Having been in Zermatt so many times, clearly it is one of my favourite destinations in Switzerland. The Matterhorn is clearly THE iconic Swiss mountain, so enough to show it to everybody who wants to see it. This time I took plenty of newcomers to Zermatt, since we wanted to have a lunch break, I thought let's do the five lake hike, which was described as a 2 1/2 h hike, thus we should have plenty of time. I even checked times on the Gornergratbahn or the opening hours for the Gornergorge, just to not be done too early - little did I know what was about to come out of that.\\

Co-travellers:\\
Andres: US-American, born in Puerto Rico, although I know Andres already since 2012, this is the first time I convinced him to join me on a trip.\\
Janina: German, Andres' summer student at CERN. Clearly the Matterhorn is on her bucket list to see when in Switzerland - and it is even more fun to do this hike with a large group. Of course this should not be an issue having a second German on this trip.\\
Chris B: US-American, once again Chris wants to check on this hike, if it could be added to his week long hiking adventure which he plans for his dad in a couple of weeks. As experienced co-traveller Chris knows what he is doing.\\
Grace: US-American, after our first trip in May also up for a hike in summer.\\
Christine: US-American, our first trip together of this year, clearly Christine knows she is the one who has to deal with getting good food on this trip.\\
David: US-American, actually run a long-distance race in Zermatt just the day before, plans to do another hike before joining us in the middle of our adventure.\\
Sam: US-American, just arrived in Geneva a couple of weeks ago, this is also Sam's first trip with me, and his first trip to the Alps of Switzerland Zermatt.\\

We all got our day tickets the day before, so we planned to meet as usually in a cafe at the train station. Seems it needed two simultaneous calls for Andres from Christine and Chris to actually get up, so surprisingly everybody was on time (unlike 4 weekends ago) and we were ready to leave as planned. People were still tired, although 6:30 isn't such an early time to start. Grace, Sam and I were the only ones who seem to be already wide awake.\\
 By the time we reached the Mattertal, everybody felt well enough to enjoy the views over the Bisglacier and the remnants of the rock avalanche by Randa, still amazing as always. Once we got to Zermatt it was time for another round of coffee, before we cross over the bridge over the Matter Vispa with the first view of the Matterhorn, and then we got up to Blauherd.\\
  From there we reached the first lake - Stellisee - after 25 mins with the Matterhorn towering over it in the distance, just 25 mins later we reached the Fluehalphut. Gornergrat, the Matterhorn, the Strahlhorn, amazing snow and glacier covered mountains all around us. Unfortunately we had to wait about 30 minutes to be served, the food was nice (almost all of us got Roesti and a local beer), but then once again we had to wait about 30 mins to get the check, and suddenly over 2 h were gone just for having lunch. Still we had 3 h left, so should be fine to go. \\
   By that time Andres decided that Heavy Metal was more interesting than talking to us. And he thought about letting us know about all the details, e.g. "Wow that a real great riff" and less than 10 seconds later - "and it is already over", or "yes", "nice", "awesome". Janina and I were just busy taking as many pictures as we could. We went down to Grindjisee, which is by the Moranes of the Findelglacier. Unfortunately almost half of the original tongue is gone nowadays, thus we could nowadays cross the valley over to Gruensee. At Gruensee we met up with David. By that time we did realise that it would be pretty tight to get to Sunegga, to catch the last cable car down. Ideed if we believed the guide posts, we would be 5 mins late. Thus we decided to walk straight down to Zermatt and just skipping the last two lakes. Since I doubt we were incredibly slow, it rather seems the advertised time for the five-lake-hike targets a shorter pathway, not passing by every lake, but  just overlooking some of the lakes. It might rather take 4-5 h than $<$3 h to finish that hike. Anyways, all the way down along Findelbach for another 400 m altitude until we finally reached Zermatt (our knees didn't cheer anymore by that time). Then we had dinner (Burgers, Gulasch soup, fries and such), and then about 4 h back to get home to Geneva. Although the plan failed for once, nobody seemed to be too unhappy about it.\\

Zermatt: Stellisee \& Matterhorn View*****, Grindjiisee****, Gr\"unsee***

\section{August 31: Aosta Valley}
\label{2019:Aosta}

Can you imagine: Chris never made it to Italy (OK neither did Justin, and he had about 5 years at CERN). Not allowing myself the failure to not showing them my favourite country again, I decided to take Chris to Italy. Since he didn't want to do an 8 h ride to Milan I suggested to get to Aosta instead. Telling Sasha about my plan he asked if he and his friend Yaroslava could join, and since 5 folks fit in a Honda comfortable we were all ready to go. PS: in the end Chris made it to Italy again just a couple of days later, taking the Bernina train from Switzerland over to Tirano.\\

Chris B:  \\

Aosta: Teatro Romano**, Cathedral***\\
Lillaz: Waterfall*****\\
La Thuile: Cascate del Rutor*****

This was the last of my trips with Chris before he moved back to California for teaching and hopefully writing up his thesis anytime soon with two trips including over night stays and five additional day trips, and surprisingly enough we saw 21 world unesco heritage sites on just these couple of days.\\

\section{September 15: Langgletscher}
\label{2019Langgletscher}

Janina wanted to do another hike in the alps. Nobody else besides me was up for it, thus we brain stormed where we could go without needing a car (since her car decided to have technical issues). We first thought we could go to Moiry, or to Sorebois in the Val d'Annivers instead or go to one of the huts close to Zinal, but then I found out that the Anen hut in the L\"otschental overlooking the Langgletscher was still open.This seemed to be the ideal destination, since we had several buses and trains we could take back, in case our hike should last a bit shorter or longer than planned.\\
Since we had to start out very early, Janina stayed at Grace's place over night, we also had a short breakfast in Brig. Once we arrived at the bus stop by Blatten, the path is very nice going along the rapids of the Lonza river through the forrest, before reaching meadows with lots of cows in front of the Grosshorn and the Jeggi glacier. We climbed up the mountain reaching the Anen hut and having superb views of the valley as well as the Lang glacier with its glacier gate and the dirt covered tongue. After getting lunch inside the hut, we walked down over the ridge of the moraine and then following the river, passing by a couple of small lakes, and back to the train station, where we could see people getting on and off the car train of the old L\"otschberg tunnel. And then our train arrived starting our way back home to Geneva.\\

Langgletscher*****

\section{September 20-September 22: Paris II - this time seeing more of Paris}
\label{2019:Paris II}

Another trip for the European heritage days in Paris - but FINALLY not alone. With Riju and Chris gone, I tried to convince other folks that travelling with me can be fun. And yes they have been briefed by Riju and Chris what to expect. Four co-travellers, three of them on the first trip with me, the fourth one on his first non-day-trip with me, two people on their first Paris trip. Lots of first things this time, so a bit of history. Once again I tried to get a couple of people to join - some forth and back (even that long that we missed out on cheap tickets to get to Paris) I ended up convincing Sam and Vivan to join on the trip. Just two days later Vivan asked if her friend Nhi could join as well. So we were ready to go, booked our room, when Vivan realised another one of her friends would be in Europe and potentially interested to join in as well (potentially ended up as definitely only two days later). So here we are with a gang of five people in total.\\

Co-travellers: \\
Vivan: US-American, in fact travelling quite a bit within Europe herself, after moving here for longer term just a couple of months ago, but typically emphasising on other points than me. Since Vivan has been travelled to Paris before and loves the city, it was pretty easy to convince her to see other things, which are only open on this particular weekend.\\
Sam: US-American, just arrived in Geneva a couple of weeks ago. In fact Sam had been on a trip with me to Zermatt, so he knew what would be coming his way. Being in Europe, even if only for a short while offers you the opportunity to see so many places, but Paris is clearly one of the famous highlight, even acknowledged across the Atlantic. Maybe a pity that the usual stuff (e.g. Louvre) will be off-limits for this particular weekend, still my hopes are high that he will enjoy it.\\
Nhi: US-American, doing her PhD in physical chemistry statesside. In France for a conference in the beautiful city of Chamonix (just read one of my summaries, the last one just from February this very same year). This will be her first trip to Paris, more or less convinced single-handedly by Vivan to join on our adventure, I am not aware if she has been told what she has gotten herself into. I hope she will see the trip as begin of a wonderful love for this continent as well. \\
Michael: US-American, soon finishing law school in Chicago. Michael happened to get his hands on tickets for a Tottenham Hotspurs Premier League game just a few days after this weekend. Vivan as good friend of him suggested first to stop in Paris before visiting her in Geneva. Since Michael has been in Paris before, it wasn't too hard to convince him of seeing it again. \\

September 20: Friday\\
After having a few drinks and a quick dinner in Geneva previously we got to Paris in a TGV which was unexpectedly leaving on time and even more unexpectedly arriving 10 minutes earlier than planned at Gare de Lyon. We all checked in in our room for four, which turned out to be two adjacent rooms with one bathroom each - great for the next two mornings, thus we should be able to get in and out of the shower a lot quicker than anticipated - or so I thought, but more about that later on. Michael had arrived in Paris a couple of hours earlier, so we got out into Paris nightlife to meet him in a cafe, where we found out that cocktails are cheaper than the beers on the menu. The ``surprise cocktail'' was surprisingly nice, and we got back home only at 1 am - maybe not the best way to get sleep on this Paris trip either.\\

September 21: Saturday\\
This time around tickets had to be booked for certain attractions, this time I chose to show up at Palais Royal first, starting at 9 am, so I planned for getting out by 7:30, for a nice relaxed breakfast, getting there and not standing at the end of the line. Easier said than done, while all the boys were done by 7:25, the ladies took a bit longer, though 7:35 is still a good showing off for first timers on my trips. After getting Pain au Chocolates and coffees, teas etc for all of us, we head out. While on the metro M1 we get told that due to demonstrations the line will operate only on some metro stations, and all others will be not served -- for the whole day. Bad if we chose the hotel for being connected through this line, but at least the final stop of this Saturday is exactly the stop where we want to get off. Since we are there a bit earlier than expected, we chose to walk over to see the Arc de Triomphe du Carrousel in golden sunrise light, as well as later the same with Opera Garnier. Once getting inside the information stand of Palais Royal we got ourself ``Journees du Patrimoine European 2019'' bags, which seemed to be very loved by anyone we met at all other points, but only available at Palais Royal. And once we got to the Ministry of Culture, a lady was supposed to tell us all about the history. Too bad that she only repeated what we shown on leaflets on all the rooms itself, so we decided to go rouge and go through the place at our own pace. Nothing much changed at Palais Royal from the last time i had visited in 2014, besides new pieces of modern art or new carpets. In Palais Royal we find nowadays the Ministry of Culture, State Court and Supreme Court, decorated during the second Empire of Napoleon III. \\
Afterwards we checked out the Banque de France. The highlight is the Golden Gallery in Hotel de Toulouse. I had seen this part of the national bank in later 2013. Somehow the gallery seemed a lot brighter than I remembered. I found out later that this is because the pale green background of the wood panelling was changed to a shiny white, definitely an improved appearance. Next item was the Hotel de Ville, the largest city hall in the world. This time the Festival Gallery was illuminated by red and blue spots, changing periodically. Across I saw the progress made in restoring Notre Dame. The roof had been covered by large pieces of wood, some of the glass windows had been removed. I wanted to get a closer look at the facade so we crossed over to the Ile de Cite stopping to grab some Crepes and Quiches. Unfortunately the square in front of Notre Dame was completely closed off, so we went over to the other side of the island, paying a short visit to Conciergerie, which was in renovation as well. Afterwards we queued for Sainte-Chapelle the place with the most beautiful gothic stain glass ever. Always try to get in here when visiting Paris. On this particular weekend it was for free, and we had to queue only for 15 minutes. Once we went out this had increased to 50 minutes, so we just caught a quiet time. \\
And off to the Cafe Les Deux Palais, where Reyer and Christine had stopped just two years earlier. And I convinced Sam and Nhi to try the really excellent Irish Coffee, which we might rather call hot Irish Wiskey with a shot of coffee. By that time we were running late by about 1 h behind schedule, due to including the Bank of France spontaneously. Thus we actually missed the time, when the representative salons of the Monnaie had been open for visits. So off to the Hotel d'Avaray the residence of the ambassador of the Netherlands. A nice cute city palace with a nice cute garden. And off to the Hotel de Matignon, the seat of the french prime minister. Once again I was impressed by the palace, though this time they didn't let us enter the office of the prime minister, unfortunately the vast garden was off limits as well. The final stop of our day (well final before dinner) was the Hotel de Soubise, the french national archive. Beautiful baroque and rococo rooms all over the place, as well as ancient documents, dating back as early as the 8th century.\\
Sam and I decided to head over to our dinner place, while the other folks went on a small shopping tour, thus we had delicious white wine to start the evening. For dinner we had once more the wonderful tasty beef stew, white and red wine, and convincing Nhi and Sam to try (and enjoy) their first snails. Then we decided to take the bus, which was supposed to bring us all the way over to the Eiffel Tower. Once again the demonstration cut our trip short, and we had to walk the remaining 2 km to the Eiffel Tower. There we got up to the second floor and enjoyed the night views of Paris. While Sam, Nhi and I were down a bit before closing time, Michael and Vivan remained up until the very last lift down. At least this metro line worked without any issues to take us home.\\

Palais Royal*****, Banque de France/Hotel de Toulouse*****, Hotel de Ville*****, Conciergerie***, Sainte-Chapelle*****, Hotel d'Avaray****, Hotel de Matignon*****, Hotel de Soubise****, Tour d'Eiffel*****\\

September 22: Sunday\\
Today we planned to leave later, thus I put my alarm 15 minutes later than on Saturday. Unfortunately I was woken up by Vivan's alarm from the room next door. We were all done in time, the ladies claimed they would come down 5 mins later, which turned out to be 25 minutes. Not stopping in a cafe on our way we queued by Palais du Luxembourg, there we saw a cafe just across the road. So we went there in two turns having a bit of coffee, croissants, and omelettes. \\
The Palais du Luxembourg houses the Senate of France. The conference hall is absolutely breath taking (built in Empire style), same for the Salon of the Golden book (one of the salons of the original palace in baroque style). Afterwards we visited the seat of the president of the Senate, the so-called Petit Luxembourg, cute too. Afterwards Sam and Nhi convinced us to spend a couple of minutes in the park of Jardin du Luxembourg and we did visit an art exhibit in the park buildings by chance as well.\\
 After a very short stop at St Sulpice we arrived at the second chamber of the parliament, the house of representatives in the Palais Bourbon, with a lot of painted ceilings and walls by Delacroix. The residence of the president of the national assembly is the Hotel de Lassay, very beautiful rooms with nice paintings and gilded wood. And just when we stopped it started to pour rain heavily. Waiting out for about 10 minutes we decided it was hopeless to wait for the rain to finish completely, and it had gotten weaker thus we went on with our plans for the afternoon. While the others went to do some shopping, I finally made it to the State Rooms of the Monnaie, and afterwards to the School of Fine Arts just next door to the Monnaie. The school is home to the Chapelle des Petits-Augustins which houses copies of many pieces of arts, like the Last Judgment from the Sistine's Chapel in the Vatican, and Michelangelo's Moses in Rome, and the copies of the sacristy below the Capella de Medici in Florence. The main auditorium is also full of nice paintings.\\
 Then I walked through the courts of the Louvre, through the whole Tuileries Gardens up to the Musee de l'Orangerie, where we all met again to see the amazing Water Lily paintings by Claude Monet, most of the other rooms of the museums had been closed for refurbishment though. And after a view of the Place de la Concorde we walked to the other end of the park again, since it was only possible to exit there (due to demonstrations), and afterwards searching for another metro line, since line 1 was -- once again -- closed for the whole day. We still made it quite in time for the TGV again.\\

Palais du Luxembourg*****, Petit-Luxembourg****, Jardins du Luxembourg****, St Sulpice***, Palais Bourbon*****, Hotel de Lassay*****, Monnaie***, School of Fine Arts***, Jardins des Tuileries****, Musee de l'Orangerie*****

\section{October 6: Gruyeres \& Lavaux}
\label{2019:Gruyeres}

Andrew is back at CERN for a couple of days. As usually he loves to see a bit more interesting stuff around Geneva besides B904 at CERN. This time we decided we should have cheese fondue -- and there is no better place to get it than in Gruyeres.\\

Co-travellers:\\
Andrew: loyal co-traveller since 2013, Andrew is coming over to CERN to work on firmware and electronics. In order to recharge his batteries we get out to the canton of Fribourg this time. \\
Will: US-American, another UCLA grad-student which I take for a trip. Since Will is planning to stick around the area for a bit longer, making an alright impression is one of my goals.\\
Xuan: Chinese and UIC grad student, at CERN for a short while, Xuan also loves herself getting cheese and chocolate as advertised before starting the trip.\\

October 6: Sunday\\
As typically Andrew rents his car from the French side of the airport, so we first got our Vignette by the next gas station, everybody arrives on time and we were ready to go. Once we arrived by the Maison du Gruyere we took the tour, where ``Cherry'', the virtual cow told us, what the cheese is all about, we got three samples of cheese as well. \\
Then we are ready for a round of Irish coffee in the Giger bar. Xuan did freak out a bit, once she realised she was sitting in front of a wall of baby-heads, everybody liked the alien-type ambience though. And then we were ready to dive into Giger's arts of Mensch-Maschinen, Xenomorphs et al, while also admiring his Oscar statue (no photos). \\
Then we walked down to the Maison du Gruyere, unfortunately off limits for a local event, so we had our pot of cheese fondue at the station restaurant (still nice to have). Then after a short drive we found out we would need to wait over 90 minutes for the next tour of Maison Cailler, so we rather passed on that and got a bunch of chocolate instead, before driving on to Corseaux.\\
 Here Corbusier built a Villa by the lake for his parents. The lake view was magnificent as usually, the Villa was well - very functional, rectangular, not really colourful; with guest rooms, a long row of windows on the lake side, typical Corbusier and thus world unesco heritage. Consensus was though that it is a tad underwhelming, let me quote ``Can you imagine they put that officially on the same level like the Colosseum in Rome''.\\
  Since we were so close to Lavaux (also UNESCO world heritage ) we continued our trip there and walked through the vineyards trying to find a Cave which might be open. Although the guidepost in front of the village claimed that at least one should be opened on October Sundays, seemed not to be the case this weekend. While we walked back, one of the men working on the harvesting of the grapes took pity on us, let us taste a sample of their white vine as well, gave us some of their grapes, and told us to walk back to the next village, since there their local Cave of St-Saphorin vineyard association should be open. And indeed it was, so we got our sample of snacks and a bottle of vine for the four of us, before continuing to walk along vineyards and the lake, before getting back to Geneva, where we ended our day with pasta.\\

Gruyeres: Maison du Gruyere***, Giger Bar****, H.R. Giger Museum****\\
Broc: Maison Cailler*\\
Corseaux Villa le Lac**\\
St-Saphorin: Lavaux*****

\section{October 13: Gorges du Fier \& Bellegarde}
\label{Bellegarde2019}

Since Andrew is around for another weekend, this time we decided to see a bit more of nature on nearby France, opting for a nice little river slot canyon by Lovagny close to Annecy. The Gorges du Fier is a bit more than 1 km long with a trail pass installed about 25 metres above the river. The cliffs are up to 70 metres deep. At the end of the actual canyon a labyrinth of rocks, the Mer de Rochers follows. Fier river flows in several thin arms deep beneath those sea of rocks. We also checked if the castle was still open, but that one closed just a week earlier for all winter.\\
Instead we drove back to Bellegarde-sur-Valserine. In this town the TGV stops and it is a hub to switch regional trains to Evian or Lyon, also local train lines from Geneva stop there. The town itself is small and less than spectacular but one curiosity of nature. Walking along the quite nice riverwalk along the Valserine, passing a Weir, old ruined former power stations and old decaying mills we arrrived by the Pertes de la Valserine. Similar to what we saw in Lovagny this is a wide range of rocks with small little riverlets passing through, but unlike in Lovagny by the Mer de Rochers the river disappears completely over a length of 200-250 metres. At the beginning of the Pertes the Valserine river drops down the rocks in a large waterfalls and the water almost immediately disappears. In between the rocks sometimes you can still see puddles or a bit of water flow but they are largely filled by sand in late October. Clearly during snow melt or after heavy rains the sink hole fills up, but still a nice cute little place to visit, particularly if you should be in the area.\\

Lovagny: Gorges du Fier*****\\
Bellegarde-sur-Valserine: Pertes de la Valserine****\\

Since Andrew comes over to Switzerland from time to time, we never know if this actually marks the end but so far he stands at one trip with overnight stays and nine day trips or hikes. It's always fun to catch up again. But at least now that I left Switzerland, maybe it has to happen somewhere else, and anyways who knows if world wide travel opens up again in late 2021\\

\section{October 26: Glacier de Moiry}
\label{2019:Moiry}

I wanted to do this hike for a while. The trail starts at the foot of the glacier and leads up to the Cabane de Moiry. Unfortunately the bus-line only serves the glacier stop in summer, and it only runs four times a day. End of September the glacier stop is not operated anymore, taking the trip from the lake bus stop extends the hike by over 2 h. I just never convinced someone to go there with me, or to drive there on other occasions. Thus I was very happy when Sasha asked me if I would be up for doing the hike.\\

Co-travellers:\\
Sasha: Ukrainian, unlike last time Sasha brought up the idea of this hike, Sasha works at CERN as most of my co-travellers, and unlike me he has done a couple of more difficult hikes in the alps already. \\
Artem: Ukrainian, also working at CERN, but with an engineering degree, definitely more experienced hiker than me as well.\\
Juno: Indonesian, also a physicist, and as well far more experienced in hiking, mountaineering and anything else which comes with it.\\�
Elga: Indonesian, but for once someone not related to CERN at all. He is working in linguistic research at the university of Geneva. Just like me not one of the more experienced mountaineering experts.\\

October 6: Saturday\\
Starting out at CERN by 8 after picking up everybody, a short coffee and shopping stop in Martigny, and another photo stop at a Waterfalls just past of Grimentz, we reached the Lac de Moiry. Amazing clear bright blue water with all the mountain ranges around reflecting in the water. I walked over the dam of the reservoir to the other side and back, also providing nice views of the valley. Then we drove through a tunnel over to the second lake by the former end of the glacier. Nowadays the glacier tongue receded by about 1 km, ending in a third melting water lake. We walked up the side moraine across some left over snow fields and then up the rocky part to the Cabane de Moiry. While the first part was easy and pretty scenic, all changed once we crossed the 2600 m line. Ice was on the path which we had to be careful not to slip too much, about 50-100 m below our final destination the rocks were snow covered. That was though less dangerous than the icy patches here and there. It was straightforward just sticking the other hand to the rocks by the path. Once we reached the hut -- which was closed down for winter already -- we had lunch on ice free rocks just above the hut. The icefall of the glacier looked absolutely magnificent and still impressive. Not quite an hour later, now filled with Baguette, sausages, cheese, and cherry tomatoes we started our hike back. By now the sun had melted parts of the ice, so it all was less tricky than I had feared previously. Once back on the moraine one can even see all three lakes after each other - quite a nice sight. On the way back after a coffee break by Lac de Moiry, we stopped shortly after Montreux to see the sunset over Lake Geneva. All in all a great hike, even allowed myself time to show up at the first Halloween party of the year. My only fear: after hiking that long with that altitude differences, what will my legs say the next day for another planned hike.\\

Moiry: Glacier de Moiry*****, Barrage \& Lac de Moiry****\\

\section{October 27: Gorge de l'Areuse}
\label{2019:gorgedelareuse}

I suggested this hike already to Andrew just a couple of weeks ago. Christine originally had the idea to do a hike through the gorge, which she found in a hiking book about hikes and stops in breweries along the way. It was also another nice October weekend, and the last weekend of the SBB action of getting day tickets for two for the price of one. Since I was invited by Sasha to join his hike to Moiry on Saturday I convinced everybody to rather go on Sunday. We wanted to start in Boudry and check if we can make it all the way to Boudry\\

Co-travellers (all US-Americans):\\
Grace, Christine, Sam: all know what to expect going on trips, and going on hikes with me, we all were well prepared with our lunch portions, as well as beer provided by Sam.\\
Vivan: After a first city trip, Vivan is ready to join us for a hike too, little did she know we would take longer than 5 pm to get back to Geneva.\\
Dylan: This is the first of my trips, which Dylan joins, considering it is a hike, I hope he will enjoy it.\\�
Will: Provided the first trip went to his liking, Will joins for a second time.\\

October 27: Sunday:\\
Yes, I knew it could be tricky doing back-to-back hikes. My thighs did indeed hurt already going into this hike, after slipping the day before, the right shoulder had still some complaints as well, but considering that the first part of the hike should have a steady but slight climb, it should be alright. Everybody got up in time, but little did we know that the TPG bus coming from France would break down due to technical problems, and although Christine drove to Cornavin, she just made it 2 mins too late to catch the train. So more coffee, and catching the train an hour later we have short photo stops at the vineyards before getting into the gorge. Which is full of colourful leaves, a foggy October day, but autumn colours brighten up the day. Rock walls left and right, and we in between hovering tens of metres above the river. The gorge opens up a bit later on, but still it is full of little waterfalls and nice colour full trees. On the way we still try to branch a bit off into the Creux du Van forrests, sometimes getting a bit lost on the way. We stop for our lunch break at some point, before climbing up the switch-backs of a road until we reach the actual start of the proper Creux du Van hike. Unfortunately the 1 h delay caught up on us on that point, so we decided rather to call it a day and shift the Creux du Van adventure to another time. Instead we stopped watched the sun set over Lac de Neuchatel. Sam managed with only a couple of minutes to spare to get a little tree, Vivan was unfortunately not served quick enough, so she had to leave a flow pot in despair. And we arrived in dark wintery Geneva after 90 more minutes.\\

Boudry: Gorge de l'Areuse*****\\

\section{December 29, 2019-January 5, 2020: East Germany and Poland}
\label{2019:GermanyPoland}

After Christmas my dad and I decided that we should spend a couple of days elsewhere. We checked options from northern Italy to Malta and Dubrovnik, and then we decided to maybe stick to regions a bit closer. Both my dad and I thought it would be nice to see Dresden again, particularly after the Residence was now far more close to completion than years ago when we visited already. Since Dresden is quite close to the border we considered the Czech republic or Poland. My dad decided he rather preferred seeing a new place, thus we ended up in Poland. Considering this is the first trip of us without our mum in more than a decade it might be end up to be a bittersweet trip.\\

December 29: Leipzig and Dresden\\
Starting out just a bit after 7 am we stopped for a short breakfast around Stuttgart before driving all the way up to Leipzig crossing over four German states -- Baden-W\"urttemberg, Bavaria, Saxony-Anhalt and Thuringia to end up in Leipzig in Saxony. The Monument to the Battle of the Nations (in German: V\"olkerschlachtdenkmal) was erected to commemorate the epic battle between Napoleon's French troops and the coalition of Prussia, Austria, Portugal, Sweden, Spain, Russia, the United Kingdom, and the confederation of the Rhine. It was a crucial loss for Napoleon before his first ousting as French emperor. The monument is gigantic with several layers, a crypt a gallery with giant over life-size statues. The highest platform is a bit more than 60 metres up, so be prepared to walk up quite a bit (there is an elevator which can be used though, but even in January the lines were long). The view of the city is nice, but it is a bit far away from the old town so you only see the highest buildings like the new city hall. Then we drove into old town, where we watch the church of St Nikolai, where the first marches of the Peaceful Revolution of 1989 against the communist party of East Germany started. Besides the historical value, the church is a nice classical-baroque building. On our way to St Thomas we passed by the nice baroque old stock exchange building and the old city hall -- an impressive old Renaissance building, one of the largest in Germany. Thomaskirche is a gothic church, itself not that much of an impressive building, but it was here that Johann Sebastian Bach composed and debuted all his famous pieces of art, and the church choir is still one of the most famous in Europe. After a short further walk we passed by the gigantic new city hall, built during the times of the German Empire and then we saw the new modern interpretation of the old university church. Having arrived in Dresden we went to the Elbe river to enjoy the panoramic view of old town, which still offers a view very similar to those enjoyed and put on canvas by Canaletto centuries ago. We crossed the river on Augustus Bridge (heavily in renovation at this moment), and had dinner by Neumarkt, enjoying the outside views of Frauenkirche.\\

Leipzig: Nikolaikirche****, Thomaskirche***, V\"olkerschlachtdenkmal*****\\
Dresden: Neumarkt*** \& Canaletto-View****\\

December 30: Dresden and Saxon Switzerland:\\

Dresden: Semperoper****, Frauenkirche (with dome)*****\\
Saxon Switzerland: Bastei Bridge*****, Neurathen Castle****, Golden Rider***\\

December 31: Dresden:\\

Dresden: Residenzschloss*****, Zwinger**** (Porcelain Museum*****, Art Gallery****, Mathematical-Physical Museum***), Hofkirche***\\



\chapter{Year 2020}
\label{2020}

\section{December 29, 2019-January 5, 2020: East Germany and Poland}
\label{2020:GermanyPoland}

Co-travellers:\\
This is in fact the continuation of a trip started late December with my dad.\\

January 1: Bad Muskau \& Swidnica\\
Clearly the day started really early witnessing the New Year fireworks in Dresden by the meadows of the Elbe river with a view of the Frauenkirche, the cathedral, and the residence. We had a short breakfast by the Autobahn very close to G\"orlitz, then continuing to Bad Muskau and the F\"urst-P\"uckler-Park. The park spans over the border to Poland on both sides of the Lausitzer Neisse, considered one of the most beautiful English landscape park created during the late 19th century. The new castle had been devastated in the last months of World War II, the park had dropped into despair particularly on the Polish side after the war. After the opening of the iron curtain and the German reunification, the new castle had been rebuilt, and the park had been cleaned up. Nowadays the park is close to its original state, bridges have been rebuilt and both parts of the park have been added to the list of UNESCO world heritages. During winter the museum about the life of F\"urst-P\"uckler is closed, thus we spent an extended walk through the park. The most famous view point of the park is the new castle with the Lucie lake. Artificial little river beds, islands, large forests, and alleyways are scattered throughout the park. My dad and I crossed over the bridge to the Polish side, his first time in Poland. The polish side is less elaborate than the German side, but the area of the park on the Polish side is more extensive than the German side. Crossing back over to Germany we stopped on the Polish side, realising the gas prices dropped substantially. One of the long roads on our way to Wroclaw was in a desperate state, one of the worse roads my dad had driven on. The other lane of the road was already renovated, our lane was also partially renovated, so maybe in a few years the road will be in a better state. \\

Bad Muskau/Leknica: F\"urst-P\"uckler-Park**\\
Swidnica: Peace Church***, Cathedral**\\
Wroclaw: Main Square with City Hall***\\

January 2: Wieliczka \& Krakow:\\
Like in 2012 I decided that Krakow is worth a visit. Since back then our train was heavily delayed we didn't get anywhere close to the Wieliczka salt mine as we originally had hoped for. Thus this time around I decided to go there first. Since the train ride for two people was cheaper than going by car, considering gas and tolls and parking fees, and also more relaxed, my dad and I got on the earliest train out of Wroclaw to Krakow. There are two ways to book online tickets for Polish trains, one way is to buy it from the polish rail webpage, unfortunately this page has so far only a Polish version. There exists an English speaking page, where high speed trains can be booked from, but at 50 \% larger ticket prices. Thankfully I have Polish friends, thus they guided me through the polish booking page (actually they confirmed that the translated google feed of the page was correct). Unfortunately the train from Krakow to Wieliczka had been replaced by a bus, and it took us about 20 minutes to find out where the bus would leave from exactly. But we did make it after all in time for our pre-booked tour. The salt mine of Wieliczka has two parts open to tourists, we took the so-called tourist tour, which leads through parts which had been closed for operations in the late 1980s. Many cavities are carved into the salt rock, many statues and chapels can be found along the trail. The highlight is the Kinga chapel carved out in 1896. They even have their own version of Da Vinci's supper carved out of the salt. There are several layers of the mine which can be visited, going deeper than 100 metres, also salt lakes and canals and grottoes are located on the deeper levels. The highest chambers is the Mikatowice Chamber, where several generations of miners worked on. After our tour we were offered another tour of the salt mine museum, unfortunately we would have needed to wait about 45 minutes before the start of the next English tour, thus we opted out of that tour, getting up in one of the tiny shaft elevator baskets. By now trains were running again, before we had a short snack in front of the train station of Wieliczka. Once we arrived in Krakow we walked from the central station over to the Royal Palace on the Wawel hill. Unlike in late September, there were no large lines for tickets in January, but the private royal apartment tours had been sold out, we decided to see thus only the State Apartments. Before we visited the cathedral instead.\\

Wieliczka: Salt Mine***\\
Krakow: Wawel Cathedral***, Wawel Palace**, Mary's Basilica***, University Church St Anna**, St Barbara*, Jesuit Church with Pantheon**, St Andrew\\

January 3: Wroclaw:\\

Wroclaw: Centennial Hall**, St Cyril \& Methodius*, University Church***, University*** (Aula Leopoldina***, Oratorium Marianum**, Mathematical Tower*), St Matthew*, Cathedral (with Chapels)***, St Magdalene with dancing Craddle**, Synagogue, Royal Palace**\\

January 4: Walbrzych:\\

Walbrzych: Zamek Ksiaz (Schloss F\"urstenstein)***\\

January 5: Sanspareil:\\

Sanpareil: Rock Garden***\\

\section{January 11: Jura \& Bourg-en-Bresse}
\label{2020:BourgEnBresse}

The weather forecast for this weekend was excellent, thus Sasha decided we should consider a short trip. I suggested we could visit the royal monastery of Brou but since this would only take about 1-2 hours, it might not be worth the drive alone. Thus we checked the close-by area. All caves in the Jura seemed to be closed until April, but gorges and waterfalls are always accessible. We found two nice spots to add on our itinerary.\\

Co-travellers:\\
Sasha: our fourth day trip together. Just like previously with Riju and Eric after leaving my working group Sasha is still up for trips. \\

Juno: after getting a new camera from Fuji Juno is eager to check how well the new photos will come out, considering that we all three have different models we will find out which one we prefer after all\\

Starting out at CERN by 8 we crossed over mountain paths to the other side of the Jura mountains. We had quite a big of fog on our way, so I was pretty happy when we arrived by the Gorges de la Langouette and found out that the whole canyon was below the fog. Before the deep canyon starts the river drops by an impressive waterfall. Upstream several little waterfalls pave the path, ending by a small water power station with another large waterfall. The gorge is a bit longer than 1 km long, thus within about an hour a round trip is enough to see most of it. Just about 15 minutes later we reached our second destination, the pertes de l'Ain. After a very short gorge the Ain river drop about 15 metres deep and disappears for about 100 m and reappears in a large waterfall. The rocks of the 15 minute fall is covered by moss. The flow of the Ain is nowadays regulated by a reservoir upstream, at times the water flow can be quite substantial and the level can rise quickly. Nevertheless even with a regulated flow the final waterfall is pretty broad and I was impressed. We had our lunch stop at Champagnole, Sasha and I tried the Morbiflette (very similar to Tartiflette, but with a different cheese) and Juno opted for Tartichevre. I was happy with my choice at least. On the way we passed by the signs for the Cascades du Herrison, but after we realised it would be a detour of about an hour we decided to move on to Bourg-en-Bresse. The royal monastery of Brou is very impressive. The church is large but most of the nave is sparsely decorated. The stain-glass windows of the choir are very nice as well as the choir stalls. The most impressive pieces are though the three royal tombs of Margaret of Austria, Philibert the handsome and Marguerite de Bourbon. Besides the life-size statues of the deceased holy figures, roses, animals are depicted as well. In the chapel of Margaret an impressive stone carved retable can be found as well. The adjacent museum of the monastery housed a couple of nice paintings, statues and old alters and tapestries. And about 90 minutes later we arrived back at CERN by about 6 pm.\\ 

Les Planches-en-Montagne: Gorges de la Langouette***\\
Bourg-de-Sirod: Pertes de l'Ain***\\
Bourg-en-Bresse: Monastere Royal de Brou***\\

\section{February 8: St. Gallen, Reichenau, and Constance}
\label{2020:Reichenau}

Sam and Vivan have never seen Germany. I clearly had to change this. As options I gave them the choices of a trip to Freiburg im Breisgau, or a trip including the monastery island of Reichenau. They preferred the later, and I got Dylan and Grace on board as well.\\

Co-travellers:\\
Sam, Vivan, Grace, and Dylan: all US-American grad students working in the field of particle physics. Grace loves going to Germany so was easy to convince to see it again, Dylan, Sam and Vivan are used to my trips as well, and some training ahead of our Rome trip a couple of weekends later might not be the worst idea either.\\

We all got our day tickets and started on the 6:30 early train over to St Gallen. There we first started with the Rokoko monastery library (I felt old being the only one who didn't get a student discount), still a bit underwhelmed that ``conserving the art'' is misused as excuse for the no photo policy. Nevertheless everybody enjoyed it, as well as the previous cathedral, the former monastery church. Unfortunately this time the choir had not been opened for visitors (sometimes if you are lucky you can enter it on weekends). A baptism was going on while we were there. Then most of us got the St Galler Sch\"ublig as snack, before getting on the train, where I tried to teach Sam and Dylan how to play Jassen. Once we arrived in Constance, I rushed to the ticket machine, and just in time got us the regional tickets for Reichenau station. For reasons unknown to me, Deutsche Bahn is not capable to allow you buying these tickets online, and a 4 minute interval to switch and get tickets on the completely inadequate horribly coded touchscreen interface of the German ticket machines doesn't make things easier as well. But we did make it to the train, then made it to the bus (although the bus station had been shifted by 30 metres, which we almost missed as well), and we got off by Peter \& Paul first. Even the little museum room next to the church was already open. All three churches of the island, St Peter \& Paul, the monastery minster, and St Georg are of Romanesque style, with frescoes from the 10th to the 13th century (as well as baroque modifications and frescoes), including some of the most precious large-scale german paintings of that time.

St. Gallen: Library**, Cathedral**\\
Reichenau: St Peter \& Paul**, Monastery**, St Georg***\\
Constance: Minster**

\section{February 20-March 2: Rome \& Italy}
\label{2020:Rome}

Why Rome or Italy:\\
I always love going to Italy, particularly Rome is always a highlight (having been there already three times). This year is a leap year, and I realised I hardly ever did a larger multi-day trip in February. Rome is very well connected to many regions and places, also many towns I had never seen so far yet, but which seemed to be very interesting to visit. Many of my friends had never been to Rome, not even in Italy. Thus when I proposed to spend a couple of days in Rome, I convinced four of my friends, Dylan, Vivan, Sam, and Bryan to join me. While I was in Italy the Corona outbreak started in Lombardy, a couple of cases had popped up in Rome as well as in Geneva at that point. Thus while I considered to remain in Rome, only Bryan joined me for these five days. \\

Co-travellers: Bryan:\\
US-American and grad-student at CERN, having been in Switzerland only for a short while, this will be Bryan's first trip with me, also his first trip to Italy and Rome.\\

February 20: Rome \& Orvieto:\\%20

Rome: Santa Maria Maggiore*** (Loggia*), Sant'Alfonso De Liguori, Sant'Alfonso De Liguori, San Paolo entra le Mura*, San Carlo alle Quattro Fontane**, Fontana dell'Acqua Felice**, Santa Maria della Vittoria***, Santa Maria degli Angeli e dei Martiri
**, Palazzo Massimo***\\%15
Orvieto: Santa Maria dei Servi, Sant'Angelo, Duomo***, San Bernardino*, Santi Andrea e Bartolomeo, San Giovenale*, Orvieto Underground***\\%5

February 21: Urbino \& Pesaro\\%10

Urbino: Santo Spirito**, Oratorio di San Giuseppe***, Oratorio di San Giovanni Battista***, San Domenico, Santa Chiara*, Oratorio della Santa Croce*, Palazzo Ducale**, Santa Caterina**, San Francesco\\%9
Pesaro: Duomo\\%1

February 22: Castel Gandolfo \& Anagni\\%8
I had booked tickets for the Apostolic Palace and the Villa Barberini in Castel Gandolfo, only to realise later, that the walk through Villa Barberini would need to be done with a guide, which thus could not be done in as short of a time as I had hoped for. Thus I feared I would not be able to actually see the gardens after all. I took the earliest train from Rome to Castel Gandolfo, arriving by the volcanic crater lake of Lago Albano in early morning sunshine. I walked up to the old town of Castel Gandolfo having a short look into Bernini's church of San Tommaso da Villanova, a cute little Baroque domed greek-cross shaped church. Once the ticket office opened, I realised although the walking tours of Villa Barberini starts late in 11 am, bus tours are offered starting at 9 am though. I was able to upgrade my walking tour ticket to a bus tour ticket. The Villa Barberini is a vast park, with many alleyways, French style Baroque parterres, fountains, statues, as well as a farm, and a heliport. Within the gardens are the remains of the Imperial villa of the Roman Emperor Domitian. The tour lasted about 1 h with several short stops where the English audio guide told us more details about the different items. Once the tour was finished I continued with a visit of the Apostolic Palace which had been originally built for Pope Urban VIII by Carlo Maderno. In late 2015 the Pope decided not to use the palace for private stays anymore and the palace had been opened to the public in late 2016. Throughout centuries Castel Gandolfo had been the summer residence of several popes, maybe of the rooms had been regularly refurbished. The first part of the visit leads through the Popes' museum which displays many private items of the popes, chairs, tiaras, portraits, as well as writing sets, uniforms of the Swiss guard, and limousines of the Popes through the times. Most of the state rooms are rather modest, but also decorated with precious clocks, tapestries, as well as paintings. The study of Pope Benedict XVI, his secretary, and his bedroom are on display as well (rather modest I have to say). I was particularly surprised by the rather modern style of the private chapel of the Pope. The Gallery of Alexander VII instead is a traditional Baroque loggia with landscape frescoes on the walls. On the way to Anagni I had a short stop in Ciampino, where I had freshly made pizza in one local store by the piazza of the church of Sacro Cuore di Gesu. \\

And then I got on the train to Anagni, where I had to transfer to the local bus bringing me to old town. No real schedule had been posted online, just a statement that a bus would leave from the station roughly every hour shortly after the train arrives. Indeed it left just about 5 minutes after the train had arrived. Stations were not displayed online, just a tiny sign showed where buses would stop. I just got off at the edge of old town and decided to start exploring the town. There were many nice views, little narrow roads, gates, city walls, little churches here and there. I had a short coffee and made my way to the Duomo. The cathedral was built in early medieval times in Romanesque style, most of the cathedral had been refurbished in gothic style in the 13th century. Most of the original frescoes have been lost. The museum of the Duomo holds many treasures from medieval times, particularly from the 13th century, when Anagni had been the summer residence of the Popes before the popes moved to Avignon in 1309. The real treasure of the church is the crypt. Starting with an oratory of Thomas Becket, with almost completely faded frescoes from 1231-1255, the largest room of the crypt is dedicated to San Magno, the patron of Anagni. The mosaic floor as well as the well-conserved frescoes originate from the 1231-1255 as well. The frescoes are some of the best conserved in Italy and have been recently restored, definitely worth to pay to see them. In fact while the church itself might be nothing special, the crypt (both the frescoes and the floor) are worth a detour to Anagni. The second important building in Anagni is the summer palace of Pope Bonifacio VIII, which holds some original frescoes in four large halls. Here the pope was held prisoner for two days by the Colonna family on request of the French king, before the citizens of Anagni freed the Pope. After all of those events the French king proved to be victorious and the Popes left Rome and Anagni, never to return to Anagni. After seeing one more church I walked along the main road until I finally found one of the tiny signs for the bus. I assumed that the bus schedule would be related to trains leaving from the train station as well and I was there in time, to be sure not to miss it. After a wait of about 30 minutes indeed a bus arrived and brought me back to the train station, and then about an hour later I arrived in Rome where I had some Spaghetti Carbonara.\\

While so far I would have thought catching the bus from Anagni old town to the Fiugi-Anagni train station would be the thing to worry about on this trip (at least transportation wise), the first Covid-19 case in Lombardy appeared, which a couple of days would jeopardise this trip far more than I had imagined only a couple of days later.\\

Castel Gandolfo: Lago Albano**, San Tommaso da Villanova*, Villa Barberini***, Palazzo Pontifico**\\%3
Ciampino: Sacro Cuore di Gesu\\%1
Anagni: Madonna di Loreto*, Duomo*** (church*, crypt***), Palazzo di Bonifacio VIII**, San Giovanni De Duce*\\%4

February 23: Terni, Marmore, and Rome\\%9

Terni: San Francesco*, Big Press Monument*\\%2
Marmore: Cascata della Marmore***\\%1
Rome: Piazza del Popolo**, Santa Maria dei Miracoli*, Porta del Popolo**, Villa Giulia (National Etruscan Museum)**, Fontana delle Conche, Santa Maria del Popolo*\\%6

February 24: Pienza \& Val d'Orcia:\\%4 
The plan for this day was to see Pienza during the morning, and to stay in Montepulciano for a quiet afternoon and having a wine tasting there. Sitting though on the earliest train at 6:12 I got already stuck on the way. In a tunnel between Roma Tiburtina and Orte the train stopped, and we were told that there is a technical issue, which means we have to stop for about 10 minutes. After another 15 minutes of not moving, we were told the problem is tougher, and we would have to wait for longer (by that time I knew I won't make it to Pienza in the morning for sure). The air conditioning was switched off a short while later. Sitting in a non moving train in a tunnel, one could feel the pressure wave of incoming trains about 1 minute before it passed by. Each 10 to 20 seconds another peak arrived shaking the train, even after the trains passed, the pressure wave amplitudes were noticeable for about 30 seconds later, and then silence again. The ticket controller went up and down through the whole train phoning to find possible solutions. After about 45 minutes later the power went off. Light was only available through smart phones anymore. The only light still illuminated were red emergency lines just by the doors. Another 15-20 minutes later the ticket controller went through the train with flashlights, and we were told to move to the front of the train. Once we arrived there we were told what the issue was actually: it seems all doors couldn't be opened anymore and each of them would have to be manually opened by an override switch. A train mechanic had arrived, but seems the issue was more drastic than assumed, and we would move on to the next train station and would need to get off there. And soon after the train finally moved on, we finally saw light again, and 10-15 minutes later we arrived in Orte. By now it was about three hours later, a solution we might have been able to have already 3 hours earlier. We were told the next train along our line would arrive just about 10 minutes later. It was announced that this train would be by now 10 minutes delayed too, our stuck train definitely created quite a bit of chaos. Every 10 minutes the train was more and more delayed, in the end even this train departed with over one hour delay.\\

 Once I arrived at Chiusi-Chianciano Terme I figured out if I would make it to Montepulciano in time for a timely connection to Pienza. Seems that was possible, but just in time, so lunch had to be more or less cancelled and the trip would not be relaxing. The bus to Pienza was also full with school kids - in fact I seemed to be the only person on board who wasn't a school kid, and we even had a ticket controller getting on board midway (and one of the kids had no ticket), anyways I made it to Pienza. There I walked through the city, stopping for a short visit of San Francisco, before visiting the cathedral. The Duomo was alright, a decent Renaissance cathedral but nothing special. Pienza is considered one of the first planned ``ideal'' Renaissance towns in Italy, quite nice but not that I understand why it is on the world UNESCO heritage list. I booked a tour through the Palazzo Piccolomini, which was rather a self guided tour, where we listened to an audioguide. The ``guide'''s job was checking that nobody took photos. The palace itself is pretty decent, particularly the bedchamber and the chamber with leather wall decorations. The bed was very nice and carved. After all of that I decided to walk around the paths through meadows and cypresses of the Val d'Orcia. The nature is really very nice and beautiful to walk through, also with the villages and towns on the hill sides, and Pienza is nice to look at, sitting over the valley as well. The nature is really breathtaking, it was also used as background for many movies, e.g. the Gladiator. Alone the views of this valley are worth a trip to Pienza. I had a couple of small pizza slices before getting on the bus back to Montepulciano, another bus (with a few coffees there) and a train later I arrived back in Rome. All in all a short visit to Tuscany with most of the time sitting in trains (and a dark tunnel ahead of Orte). And in order to make me even more happy, Trenitalia informed me that my train had been cancelled, but I could take another train about 90 minutes later without needing to pay an exchange fee.\\

Pienza: San Francesco*, Palazzo Piccolomini**, Duomo**\\
Val d'Orcia***\\

February 25: Siena \& San Gimignano\\%16
Once again I sat on the earliest train to Chiusi-Chianciano Terme, this time successfully on time. Once I arrived there, I asked the lady at the Trenitalia counter, if I could exchange my ticket for the cancelled train for an earlier than the proposed one. Indeed that was possible, unfortunately she couldn't tell me would happen to my reserved trains the next day, when I had booked the same train back to Rome from Florence. Anyways I hoped on the Trenitalia bus which brought me to the train station of Siena. I had been twice to Siena already, both times I was impressed by the cathedral, thus I also planned to see the Duomo this time. Unlike last time I followed the guide posts from the train station to get to old town instead of google maps. Instead of walking up the hill this meant standing on escalators which brought me up to the hill top just a couple of metres away from the gate of Porta Camollia. On my way to the Duomo I had a short sneak peak into the churches of San Pietro Alla Magione, Santa Maria in Portico a Fontegiusta, Sant'Andrea, and San Cristoforo, past gothic Palazzo Salimbeni, and the Loggia della Mercanzia. Out of these places the church of Palazzo Salimbeni is for sure the most popular photo stop, the church of Santa Maria in Portico a Fontegiusta is the one which I would pay a short visit to again.\\
 Next to the cathedral is the old hospital of Santa Maria della Scala (also home of the ticket office of the Duomo). In order to avoid long lines, I had pre-booked a ticket which would give me access to all sights close to the cathedral. Unfortunately I realised only after I paid, that the hospital closed on Tuesday (only in February). Surprisingly tickets were sold for that day though. Once I arrived to exchange the voucher for a ticket I was informed that the ticket for Santa Maria della Scala would be valid for the next day, clearly nobody gave any thought to the fact, that some people might just not be in town for two days. I stood in line of the Duomo and made it into the cathedral as first person of the day. The cathedral is one of the most beautiful large gothic churches in Italy, in my opinion it even beats the Duomo of Milan. The floor with many stone inlays is the absolute highlight of the cathedral. In summer it is possible to go up close to the roof and then this offers even more superb views of the inlays, which depict many stories of the bible, like the murder of the children of Bethlehem, or the death of Absalom after he lost the battle against the troops of his father, king David. The dome of the church is dominated by six giant gilded statues and a line of frescoes of holy figures. The choir itself is decorated by several baroque frescoes. There are three more parts of the Duomo which can be visited with a special extended ticket: the Piccolomini library (also several precious books are on display there), the crypt of the church, and the baptistery. All three places are definitely worth the extra fee, the frescoes are amazing, as well as the bronze statues and plates of the baptistery. The originals of many statues of the cathedral and the stained glass of the large choir window have been replaced by copies by now, they are now on display of the cathedral museum. From the balcony of the unfinished Duomo Nuovo (part of the cathedral museum nowadays) one has the best top view of the city, the Duomo, and the Piazza del Campo with the Palazzo Pubblico.\\
The Casa Santuario di Santa Caterina is a complex where st catherine of Siena lived at with her parents and 24 siblings. The complex houses two oratories, the Oratorio delle Cucina is really nice, and the church of del Crocifisso. Since my originally visit of Santa Maria della Scala didn't work out, I decided to visit the Palazzo Pubblico instead for a second time. This palazzo is the city hall of Siena, but many of the old rooms and chapels with old murals and frescoes dating as far back to Gothic times can be visited as part of the Museo Civico. Nowadays these rooms are used by the Mayor as well as marriages. The tower can be climbed as well. On my way to the central bus station I got a bit lost and ended up by the basilica of San Domenico (just about 150 metres away from the central bus station though). The terrace in front of San Domenico offers the best views of opposite hill side which is dominated by the Duomo and the tower of Palazzo Pubblico. The interior of San Domenico is rather modest, with modern stained glass windows. I got on the bus to San Gimignano well in time (once again with many school kids).\\
San Gimignano is famous for its many medieval towers, which were a sign of the wealth of the rich merchant families. Other Italian towns had a similar outline in medieval times, e.g. Bologna, where only two towers remain nowadays. Instead, San Gimignano managed to conserve fourteen large towers, the main reason why the walled old town was declared a UNESCO world heritage site. I got myself the San Gimignano pass, which covers the Duomo, the Palazzo Comunale with the Torre Grossa, San Lorenzo in Ponte, and the Museum of Fine Arts, as well as the Museum of Archaeology. I climbed the Torre Rossa first, with 54 m the highest of the town towering next to the Palazzo Comunale. The view of the town and the surrounding hills from the top are pretty nice from up there. The Palazzo Comunale is pretty nice too, particularly the chamber of the Podesta and the Council Hall. Next I viewed the Collegiata di Santa Maria Assunta, the so-called Duomo of San Gimignano. The church walls are decorated by many 14th century frescoes, which depict a Poor Man's bible with Old and New Testament cycles, including stories such as the creation of men, the pharaoh and his army after drowing by the flood, the last supper, and the kiss of Judas, as well as the last judgment. I was pleasantly surprised, in fact the Duomo was an unexpected highlight of San Gimignano. Afterwards I left old town and walked down into the valley to have a nice view of the hillside with old town and all towers. Afterwards I went back and saw the arts gallery, and the old church of San Lorenzo in Ponte with its nice frescoes (unfortunately not quite as complete as those of the Duomo). Then I had a selection of local Tuscan cheese and dried meat for dinner, before getting on another bus, a train which was delayed by another half an hour and finally on the train back to Rome after another hour of wait (having a coffee during the wait and one more Panini).\\

Siena: San Pietro Alla Magione, Santa Maria in Portico a Fontegiusta*,  Sant'Andrea, San Cristoforo, Duomo*** (Crypt**, Biblioteca Piccolomini***, Baptistery**), Cathedral Museum**, Saint Niccolo in Sasso**, Oratorio della Camera, Oratorio delle Cucina**, Chiesa Del Crocifisso*, Palazzo Pubblico***\\%11
San Gimignano: Palazzo Comunale**, Torre Grossa*, Duomo***, San Lorenzo in Ponte***, Galleria d'Arte Moderna\\%5

February 26: Florence\\%16

Florence: Santa Maria Maggiore*, Uffizi Gallery***, Palazzo Vecchio***, Duomo** (Dome***, Crypt*), Biblioteca Medicea Laurenziana**, Battisterio Laterano***, San Michele Arcangelo Visdomini, San Lorenzo**, Cappelle Medicee**, Santa Maria Novella***, Santa Croce***, Boboli Gardens***, Palazzo Pitti***, Santa Felicita*, Santissima Annunziata**, Cathedral Museum**\\%16

February 27: Rome\\%10
Originally we all planned to spend this day on the road, visiting old Etruscan tombs, and the Villa Farnese in Caprarola. Instead Bryan and I improvised a bit and saw many other places in Rome itself. First we dropped our luggage by our hotel, since today I was supposed move with all of them to a hotel just opposite of the opera house. We got one room for each of us, and then after getting a 24h public transport ticket we took the metro to Laterano. There we took some shots of the Roman gate of Porta Asinaria and the Porta San Giovanni, which has been constructed once the Porta Asinaria became too small for traffic. Opposite of the Palazzo Laterano is the Sancta Sanctorum, which houses the Scala Scanta, the holy stairs. According to the legend these are the stairs of the Pitate's palace where Jesus walked up for his trial. They were brought to Rome later on. If people walk up the stairs on their knees praying, the catholic church states that a couple of years are forgiven (and yes, that's what I have done previously). On the side stairs frescoes of old testament legends, like the story or Samson are displayed. At the end of the stairs is the chapel of San Lorenzo in Palatio. This chapel is the only remaining part of the medieval Papal Lateran Palace. The ceiling depicts paintings of the four evangelists, on the walls the frescoes depict Popes, the frescoes close to the windows depict old legends, like the stoning of St Steven's, on the wall is a relict of Jesus' chair of the last supper. We walked past the egypt obelisk Lateranese over to the battistery with its old ancient mosaics, and its paintings and the gilded ceiling. The cathedral of San Giovanni in Laterano was closed until 1 pm, thus we decided to take the metro and a bus to Villa dei Quintili first.\\
The Villa dei Quinitili is one of the best conserved large villas by the Via Appia Antica, also the first large Roman ruins I showed to Bryan. We decided to walk along the Via Appia Antica over to the Mausoleo di Cecilia Metella and the Castrum Caetani, continuing to the Circo di Maxentio (still closed) and the Basilica of San Sebastiano with the marble statue of San Sebastian and the wooden ceilings with carved depiction of San Sebastian and the papal crest. We found out that unfortunately the catacombs had been the first sights to be closed down due to the corona virus outbreak. And we took a bus back to the Lateran for the highlight of the day: San Giovanni in Laterno, the cathedral of Rome. For once the transept and the choir were accessible, thus I saw the backside of the exquisite high altar. The treasury was cute, with many golden crosses, ivory cases, as well as capes, altar cloth etc. The giant marble statues are always breathtaking, as well as the wooden ceiling. The cloister (even a first time for me in Rome) is nice as well with lots of pillars, as well as remains of old Roman ceiling.\\
Then we took the tram to the Villa Torlonia, a late classicist villa, which had been the private residence of Benito Mussolini during the Fascist time in Italy. The private quarters on the upper floor of the Casino Nobile are a mesh-mash of romantic and neo-baroque styled rooms. Rather interesting is the Casina delle Civette, particularly the Art Nouveau style stained glass windows, depicting owls, swans, vineyards, other birds, roses, etc etc. The gardens with artificial ruins, palm trees and obelisks are nice for an evening stroll as well. Just a couple of minutes away on foot (even with a short stop for coffee in between) is the old Roman mausoleo of Santa Constanza with several nice late Roman ceiling mosaics. Here we once again were faced with the trend to disable all lights unless it was paid for. We ended the day in the church of Santa Maria degli Angeli e dei Martiri, which is the transformed tepidatirum of the baths of Diocletian.\\

Rome: Porta Asinaria*, Scala Sancta** (with San Lorenzo in Palatio***), Baptistery**, Villa dei Quintili**, Via Appia Antica*, Mauseleum der Cecilia Metella*, San Sebastiano Fuori le Mura**, San Giovanni in Laterano*** (Cloister**, Treasury**), Villa Torlonia*** (Casino Nobile**, Casina delle Civette***), Santa Constanza**, Sant'Agnese fuori le Mura*, Santa Maria degli Angeli e dei Martiri**\\%11

February 28: Rome\\%20

Rome: St Peter's Basilica***, Santa Maria in Traspontina*, Vatican Museum*** (with Sistine's Chapel***), Castel Sant'Angelo***, San Giovanni Battista dei Fiorentini*, Santa Maria della Pace**, Piazza Navona***, Pantheon***, Santa Marta al Collegio Romano*, Sant'Ignazio**, San Luigi dei Francesi**, Marc-Aurel-Column**, Santi Ambrogio e Carlo al Corso**, Piazza del Popolo**, Galleria Borghese***, Oratorio del Santissimo Sacramento al Tritone*, Santi Vincenzo e Anastasio a Fontana di Trevi*, Fontana di Trevi***, Santa Maria in Trivio*, xyz\\%20

February 29: Rome\\%19

Rome: Forum Romanum**, Palatine Hill*** (Casa di Augusto***, Casa di Livia***, Domus Transitoria**, Loggia Mattei***, Palatine Museum*, Criptoportico Neroniana, Oratorio dei Quaranta Martiri*), Santa Francesca Romana**, Palazzo Quirinale***, Santi Domenico e Sisto*, Imperial Forums** (with Trajan's Column***), San Pietro in Vincoli**, Santa Maria ai Monti*, Santa Caterina da Siena in Magnanapoli**, Santa Maria di Loreto*, Santi XII Apostoli**, Santa Maria in Via Lata*, Il Gesu***, Largo di Torre Argentina**, Santissimo Sudario all'Argentina*, Sant'Andrea della Valle***, Santissime Stimmate di San Francesco*, Palazzo Venezia*, San Marco Evangelista al Campidoglio**\\%19

March 1: Tivoli\\

Tivoli: Villa d'Este***, Villa Adriana***, Temple of the Sibyl*, Temple of Vesta**, Villa Gregoriana***\\%5
Rome: Santa Chiara, Palazzo Santa Chiara*\\%2

March 2: Rome\\
We had breakfast and checked out of our hotel, and then continued to make use of our 24h public transport ticket which we had purchased the evening before, getting out to the basilica of San Paolo fuori le Mura. This basilica had been vastly unchanged from the 5th century until the 19th century, when a large fire ravaged through the basilica and largely destroyed the building. The basilica was rebuilt to resemble the original church as close as possible. Only parts of apsis mosaic of the choir, the high altar, and some of the doors could be saved from the original church. The five naves of the basilica are dominated by rows of columns, the wooden ceiling is gilded in gold. A row of Papal portraits (fictional for the oldest ones) surround the naves of the church. Next we stopped shortly by the Piramide station to see the pyramid of Caius Cestius as well as the Aurelian Walls and the Roman gate of Porta San Paolo. One of the best conserved ruins from ancient Rome are the baths of Caracalla, which represent how large these public baths had been once. Most of the statues and decorations had been removed long ago, the most famous is the Hercules of Farnese, which is in the national museum of archaeology in Naples by now. Only some remains of the mosaic floors are still on the premise. Then Bryan and I walked up to the hill of Celio. Previously I had tried to get into the church of San Gregorio al Celio, but to no avail. This time the main door was close too, I also checked side doors, and I realised a small post which stated that to get inside one should ring the bell, and so we did. Indeed a door opened and a nun told me i just needed to go left to get inside the church. The church is pretty modest, which one nice side chapel. Below the Baroque basilica of Santi Giovanni e Paolo there are remains of a couple of Roman Houses. These are full of well-conserved pretty ancient frescoes. A hidden gem, which I would definitely advice to consider visiting.\\
 I wondered what we could do next; since Bryan had loved the Stanze di Raffaelo I suggested we could see the Villa Farnesina over in Trastevere. Bryan liked the idea and we took the tram from Circus Maximus over to Trastevere. On the way to Villa Farnesina we stopped in Santa Maria in Trastevere. Once again the lights nowadays had to be paid for. And once again I witnessed how many people didn't bother to pay for it, but once I paid they had nothing better to do than standing exactly in front of my camera to take their dozen of selfies. Most of those folks are unfortunately not that nice to pay for a next round in case they didn't get their shots. The mosaics of Santa Maria in Trastevere are from the gothic times, so kind of modern by the standards of Rome. A few minutes later we arrived by the Villa Farnesina. The highlights of this villa are the Loggia of Galatea, and the Loggia of Amor and Psyche on the lower floor, both highlighting the talent of Raffael. Two of the upper rooms of the villas are also full of nice frescoes. It is true, that the entrance fee is a tad high to see a small Palazzo like the Villa Farnesina, but the four rooms are one of the most beautiful in Rome. On our way to the Vatican we had a look int a couple of churches, all of them nothing special, in San Giuseppe alla Lungara we just arrived shortly before a small wedding, just the priest, the couple and two more friends. We crossed the Porta Santo Spirito and the Leonine Walls and had lunch by the small restaurant of xyz. Although this restaurant was less than half a kilometre away from St Peters's basilica, it seemed to me that we were the only non-locals here around lunch-time and the place was pretty full, and the food was pretty good (and affordable).\\
  Once we arrived by St Peter's we had to go through security again, about 15 minute wait this time around. On this day the right side nave was free to visit, so this time I got a much closer view of Michelangelo's Pieta. I was still not let into the Sacrament's chapel with my camera, guess that smart phones can take photos didn't make it to the guards yet though. The catacombs had opened by now, it was though pretty disappointing what one is allowed to see nowadays. My first time around all side chapels of the crypt could be seen, even taking photos were allowed, also one could get a view of St Peter's tomb from the fences of the high altar. Nowadays the guards make sure you see absolutely nothing of the tomb standing in front of the high altar, photography in the crypt is not permitted anymore, oh did I mention the glass walls which have been installed everywhere. Unlike in 2013 and 2017 nowadays all of the side chapels and side floors of the crypt are off limits (unless you pay for it, so thanks for being so open for pilgrims from all around the world). Unlike previously one doesn't get out to the nave by the dome anymore as well, but instead one has to get out of the entrance hall of the basilica. You can guess three times what happens by the treasury: Once again it costs a couple of bucks, and once again no photos are allowed. I did expect more from the treasury of the largest church of Christianity: some of the old papal tombs have been transferred to the treasury, and yes they are amazing, to see the Tiara worn by a couple of Popes is nice too, but nothing special can be seen in the treasury otherwise, tons of ordinary cups, boxes, cases, clothes etc, and a copy of St Peter's Chair (the original is housed within the Cathedra of St Peter, and seems it is close to never up for few to normal mortals. \\
We took the bus to the Chiesa Nuova, spend a few minutes in the church (officially called Santa Maria in Vallicell), then we visited Piazza Navona and the Trevi fountain during daylight. By that time it started to rain a bit, so we rushed past Hadrian's temple over to Piazza Rotonda, where we had another cup of ice-cream (delicious as usual), while it started to rain heavily, and everybody fled either into the open cafes or the Pantheon itself. After things calmed down again we rushed to the bus stop by Largo di Torre Argentina getting on the bus to Santa Maria Maggiore. Personally my second favourite of the Basilica Majors, Bryan disagreed and put it in fourth and last place. I enjoy that the basilica offers the splendour of Baroque mixed with ancient 3rd and 4th century mosaics, which shaped the mosaics in Rome for centuries. We paid a short visit to Santa Pudenzia and Santa Prassede (paid for the apsis mosaic) with the side chapel of San Zeno. We walked back to our hotel, got our luggage, took the train to the airport, had dinner at one of the restaurants.\\

Rome: San Paolo fuori le Mura***, Pyramid of Caius Cestius*, Porta San Paolo*, Aurelian Walls**, Baths of Caracalla***, San Gregorio al Celio, Case Romane del Celio***, Santi Giovanni e Paolo*, Santa Maria in Trastevere**, San Giuseppe alla Lungara, Porta Santo Spirito*, Leonine Wall*, St Peter's Basilica*** (Treasury**, Catacombs*), Santa Maria in Vallicell**, Santa Pudenzia**, Santa Maria Maggiore***, Santa Prassede*** %16

 And the aftermath of it all: once we returned home, we went shopping and put ourselves in self-quarantine for the usual two weeks. Either we didn't catch anything, or both of us were cases without any symptoms. After these two weeks I had just one evening left to clean up the desk in my office and pack up all stuff, before CERN went into home office mode, and days later the curfew in France started, and almost everything was shut down in the city of Geneva as well. Travel within Europe, within the Americas and in and out of Australia (and parts of Africa) seiced to exist for a while.\\
 
 \section{May 17: Rossiniere-Rougemont}
\label{2020:Rossiniere}

Golden Pass Line***, Cascade du Ramacle**

 \section{May 31-June 2: Ebenalp \& Walensee}
\label{2020:PentacostHike}

May 31: Ebenalp:\\

Ebenalp***, Berggasthaus \"Ascher-Wildkirchli**, Wildkirchli**\\

June 1:\\

Sch\"afler*** with S\"antis Panorama, Seealpsee***, Walensee with Churfirsten***\\

June 2:\\
Walensee with Seerenwald**, Ringquelle***, Seerenbachfalls (almost dried up), Linthkanal\\

 \section{June 20: Strasbourg \& Baden-Baden}
\label{2020:Strasbourg}

Strasbourg: Minster***, Petite France**\\
Baden-Baden: Kurpark*\\

 \section{July 4: Caves and Waterfalls}
\label{2020:CavesWaterfalls}

Labalme: Grotte du Cerdon***\\
La Balme-les-Grottes: Grottes de La Balme***\\
Belmont-Luthezieu: Gorges de Thurignin**\\
Vieu: Source de Groin**\\
Champagne-en-Valromey: Puits des Tines**, Pain de Sucre**\\
Surjoux: Pain de Sucre**\\


 \section{July 5: Hike St Cergue-Nyon}
\label{2020:StCergue}

 \section{July 18: Mer de Glace}
\label{2020:Mer de Glace}

Chamonix: Mer de Glace***

\section{August 1: Belalp \& Domodossola}
\label{2020:BelalpDomodossola}

Belalp: Sparrhorn mit Oberaargletscher***, Aletschgletscher***
Domodossola: Sacre Monte***

\section{August 9-August 10: Hike Jungfraujoch-Fiesch}
\label{2020:Aletsch}

Lauterbrunnen: Kleine Scheidegg***, Lauterbrunner Tal***\\
Grindelwald: Eismeer***\\
Fieschertal: Jungfraujoch***, Konkordiaplatz***, Aletschglacier***, Fiescherglacier**\\

\section{August 22: Hike Mont Forchat}
\label{2020:Forchat}

Lullin: Mont Forchat**

\section{August 29: Brugg}
\label{2020:Brugg}

Altstadt**

\chapter{Year 2021}
\label{2021}

\section{January 10: Bad S\"ackingen \& Laufenburg (Baden)}
\label{2021:BadSaeckingen}

After December was a stay at home months, I finally got out with my dad, staying close to home though, first visiting the town of Laufenburg. Their advertisement slogan is one town, two countries. Historically the town of Laufenburg has a centuries old history together before Napoleon decided to redraw borders separating the town in the middle. The larger part of old town with a ruined castle is in Switzerland nowadays. Usually one can just walk across both parts of the city, but nowadays with the government in my German state deciding walking over the border is not allowed without quarantining we didn't do that. Instead we continued our short afternoon outing in S\"ackingen. Home to a nice Rokoko church which used to belong to a monastery, the Fridolinsm\"unster. Usually on a nice beautiful day like this many guests, also from the Swiss side might have enjoyed a walk or cafe, but right now that isn't an option either, we did walk a bit on the Wooden Bridge, in fact the longest wood covered bridge in the Europe beating out Kapellbr\"ucke in Luzerrn by a few metres. Clearly not allowed to cross over we turned around midway as well.\\

Bad S\"ackingen: Wooden Bridge****, Fridolinsm\"unster****\\
Laufenburg (Baden): Old Town***\\

\section{February 4: Aach \& Reichenau}
\label{2021:AachReichenau}

It had been raining heavily and snow was melting too, rivers were really might on that day. Even the little creek of Aitrach looked more like a river and the Danube had crossed over its dams in some spots. The Aachtopf is Germany's largest karst spring, fed by water originating from the Danube sinkhole where the Danube itself disappears for about two thirds of the year (clearly not the case during snow melt). The spring is impressive forming a little lake just by a hill side. Since we were in that area we continued to Reichenau in order to see the first UNESCO world heritage site of 2021 (if you follow the past years you realise it appears pretty often on that list). This time my dad and I started at St Peter and Paul, continued with the Monastery, ending in the highlight of St Georg. I always enjoy seeing all those old churches and their early medieval frescoes. Unfortunately this time around we were not able to stop in any cafe to get some coffee or cake, clearly all of those had been closed too.\\

Aach: Aachtopf**\\
Reichenau: St Peter \& Paul****, Monastery****, St Georg*****\\

\section{February 6: Lake Constance Area}
\label{2021:LakeConstance}

Uhldingen-M\"uhlhofen: Pilgrimage Church Birnau****\\
Lindau: Minster***, St Peter***, Harbour with Lighthouse and Bavarian Lion****\\
Weingarten: Abbey****\\
Meersburg: Old Town****

\section{February 14: Haselbach Falls}
\label{2021:Haselbach}

After about a week of not going outside, I convinced my dad to do one of the quick walks close to the area, thus we drove the few villages up to Indlekofen and walked down the little valley of Haselbach. The trail had been refurbished just in Fall 2020 and we made use of those new fences on the slippery icy patches of the path. The Haselbach creek falls down in several cascades with the largest drop by the Haselbachfall which is 14 metres high. The trail crosses the creek on three occasions. Thus you get a view of the waterfall from above and below. At the time of our visit due to the cold temperatures of the preceding days parts of the waterfall had been frozen. Also a tree had fallen and got stuck in the middle of the fall. Not a bad experience for a 45 minute long walk.\\

Waldshut-Tiengen (Indlekofen): Haselbach Falls***

\section{February 19: Schl\"uchtsee}
\label{2021:Schluechtsee}

What do you do if you want to enjoy a nice afternoon in the region: what about going to one of the multiple little small mountain lakes. The Schl\"uchtsee is a very small reservoir, set up originally as ice lake for the close-by Rothaus Brewery. The lake is about 5 m deep, in mid February the lake was completely frozen, but pretty idyllic walking around it (the trail is not more than 2 km long). The lake is within a forest area, the Schl\"ucht river begins just a bit beyond it flows into the lake. Typically one can have beers or wine in the farm close to the lake or in summer just jump into the water having a short swim. It is a nice place for families with small kids, but not really anything special to be honest.\\

Grafenhausen: Schl\"uchtsee***

\section{February 21: Beuron and Upper Danube Canyon}
\label{2021:Beuron}

Beuron: Monastery****, Upper Danube Canyon*****\\
Sigmaringen: St Johann***

\section{February 23: Roggenbach Castles}
\label{2021:Roggenbach}

Bonndorf: Roggenbach Castles***

\section{February 26: Menzenschwand \& Todtnau}
\label{2021:Todtnau}

St Blasien (Menzenschwand): Albklamm \& Menzenschwand Falls****\\
Todtnau: Todtnau Waterfall*****

\section{February 27: Freiburg \& Black Forest}
\label{2021:Freiburg}

Breitnau: Ravenna Gorge*****\\\
Freiburg: City Gates**, Minster******\\
St Peter: Monastery****\\
St M\"argen: Monastery****\\\
Titisee-Neustadt: Titisee***\\
Lenzkirch: Windgf\"allweiher****\\
Schluchsee: Schluchsee***

\section{February 28: Rickenbach}
\label{2021:Rickenbach}

Originally the plan was to go to the ruined castle of Wieladingen first, and then continue via the Lehnbachfalls along the Murg river with its waterfalls up to the Strahlbrusch waterfall. It was possible to get to the cascades of the Lehnbach creek, but the bridge crossing the river at that point had been damaged by fallen trees a couple of days previously. Thus one had to get to the castle via a different route crossing the creek a bit below. The castle itself had been renovated a couple of years before, from the belfry one can see up to the Swiss Jura (the alps were not visible then), and up the Murg valley on the other side. Unfortunately the way along the Murg had been closed as well. Thus we drove up to another parking lot, and start our second hike from there. This hike led alongside an old abandoned quarry and passing the Seelbach creek down to the Strahlbrusch waterfall. The waterfall is about 12 m high, and in snow melt pretty strong as well. We crossed the Murg via an old bridge of an old postal route which also crossed two small tunnels. Since the trail was closed on this side, we just turned back to the car. All in all a nice afternoon out but for sure the Murg canyon itself could have been a nice hike too, but if it is closed due to snow or rock falls you rather don't risk it.\\

Rickenbach: Castle Ruin Wieladingen***, Strahlbrusch Waterfall****

\section{March 2: G\"orwihl}
\label{2021:Goerwihl}

G\"orwihl: Aybach Waterfall***, H\"ollbach Waterfalls****

\section{March 6: Constance \& Hegau}
\label{2021:Konstanz}

Constance: St Stephan***, Minster*****, Christ Church**, Holy Trinity Church**\\
Hilzingen: Hohenkr\"ahen****\\
M\"uhlhausen: M\"agdeberg****\\
Tengen: Hinterburg**, M\"uhlbach Gorge***

\section{March 9: Waldshut}
\label{2021:Waldshut0309}

In Waldshut the Wutach river finishes its flow by the Rhine river. Recently a lot of work had been going into making the mouth appear more natural and less canal like. Thus islands had been created with a second small arm of the Rhine as well as an area where the Wutach river can overflow in case of high water. Only a months ago snow melt was happening but already in March the second high water arm had retreated leaving only muddy but pretty dry territory.\\

Waldshut: Wutach Mouth**\\

\section{March 25: Triberg \& Villingen}
\label{2021:Triberg}

Triberg: Triberg Waterfalls*****, Schonach Waterfall***\\
Schonach: first world's largest cuckoo's clock**\\
Villingen: Johanniterkirche**, City Walls and City Gates***, Minster***

\section{March 28: Alb gorge}
\label{2021:Albtal}

Dachsberg: Alb with Teufelsk\"uche (Devil's Kitchen)****\\
G\"orwihl: Krai-Woog-Gumpen and Gletscherm\\"uhle***\\
Gutenburg: Teufelskessel waterfall***

\section{March 30: Blumberg, Mundelfingen \& Immendingen}
\label{2021:Blumberg}

Blumberg: Schleifenbach Waterfalls****\\
Mundelingen: Mundelfinger Waterfall \& Aubach Gorge**\\
Immendingen: Donau Sink Hole** (too much water at this time of the year), H\"owenegg with crater lake****\\

\section{April 11: St\"uhlingen \& Grimmelshofen}
\label{2021Stuehlingen}

I had been in St\"uhlingen multiple time on carnival parades, music festivals, or just as final destination of bike rides in my teenage years. It had been a while though since I saw old town with the city church, and the church of the Capuchin monastery. Then we continued our trip to the small village of Grimmelshofen, walking along the Wutach and going down to the Dampfkessel waterfall, not as mighty e.g. as the Lauffen waterfall in my home village, but 2-3 m are not nothing either. On the way back we drove past the Hohenlupfen castle and took a panoramic route to Bonndorf enjoying some views of the snow covered alps in the distance.\\

St\"uhlingen: old town***, panoramic road with view of the Alps***\\
Grimmelshofen: Dampfkessel waterfall***

\section{April 20: Hohentengen}
\label{2021Hohentengen}

Another close place to home is the village of Hohentengen, originally home to three castles. Out of those one has been completely destroyed, on that spot there is now a Swiss army bunker. The second castle of Weisswasserstelz is in ruins nowadays too, a few walls remain though, which give you an idea of the former glory. Rotwasserstelz Castle still guards the bridge over to the Swiss town of Kaiserstuhl. But in April 2021 borders are forbidden to be crossed.\\

Hohentengen: Rotwasserstelz Castle**, Ruin Weisswasserstelz**

\section{April 24: Kaiserstuhl \& Breisach}
\label{2021Kaiserstuhl}

Bickensohl: L\"ossholweg hike****\\
Breisach: oldtown with Stephansm\"unster****

\section{April 28: Moving to Vienna}
\label{moveVienna}

At some point also physicists leaving academia find jobs: I found my first job in Vienna. Since the pandemic was still going on (3rd wave now), it all was followed by a quarantine protocol. Since I received the green light for my flat by the very end of April (26 to be precise) and I wanted to avoid being caught up in delayed administrative procedures due to the pandemic I jumped almost immediately on the next available train with two suitcases while packing up and organising the move of all my personal belonging with a company over a week later. Quarantining alone with a limited data package, no TV, and no radio (but two books) is tiresome. This is the time when you realise how much your interactions with friends mean to you, while also living through the downsides of single life.



\backmatter

\clearpage


\end{document}

%Pont du Gard: one of the old aqueducts in Southern France, still in perfect condition, seen on a pit stop of a school trip going to Barcelona in 2001


%Troy: the famed town destroyed by the greeks after a decade of war. Indeed the acropolis had been an important trading centre for a couple of centuries up until Roman times, seen with the family in 2000.

%The castles of Bellinzona: three castles guard the town of Bellinzona, one of two sites in Ticino, still missing out on the other one, I had seen that place with a couple of other physicists back in 2010 during our free social afternoon from a PhD school.

%The Mujedar art in Aragon: a couple of buildings in Teruel and Zaragozza are part of this item, among those the old palace of Aljaferia which has its origins in Moorish times, before the Spanish kings remodelled parts of it. Seen with Riju in 2016.

%Mont Saint Michel: a fortified island in the wadden sea just by the coast of France dominated by an old gothic abbey, a lot of measures have been undertaken to give the island back its character after a dam almost completely lead to the island attaching itself to the mainland. Seen on my own in 2015.

%Kronborg: the castle of Helsingor, made famous by Shakespeare. In fact one of the largest Renaissance castle including the still existing fortifications in Denmark, the inside is a bit plain. Seen in 2013 on my own.

%The cathedral precinct of Pisa: Although mainly famed for its wrongly constructed campanile, the Duomo is in fact one of the most precious example of Romanesque architecture in Italy, you should rather see the cathedral inside than climbing the tower. Seen in 2002 with the family and once again on my own in 2014.

%Coimbra: This Portuguese town is famous for its university buildings from 12th to 16th centuries, including the beautiful library (no photography allowed), as well as the old and new cathedral and churches with royal tombs. Seen on my own in 2012.

%Coimbra: This Portuguese town is famous for its university buildings from 12th to 16th centuries, including the beautiful library (no photography allowed), as well as the old and new cathedral and churches with royal tombs. Seen on my own in 2012.






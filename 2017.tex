\chapter{Year 2017}
\label{2017}

\section{January 2-January 5: Dublin}
\label{2017: Ireland}

Co-travellers: \\
Chris: US-American, very eager to get to know different parts of the world, but at a more relaxed pace than myself. Thus perfect to hang out in a place like Ireland, where there are enough sights, but not to an overwhelming degree as e.g. in Rome or Paris. Also always up for a beer or just watching sports events in pubs.\\

January 2: Dublin\\
How we ended up in Ireland: Chris was relocating to Switzerland for quite a while and thus tried to find the cheapest one way transaltlantic flight offer around new year. Seems Air Lingus was the airline of choice. Since that means flying over Dublin, why not spending a couple of days there. Since I had been planning to go to Dublin starting all way back in 2013 it was pretty easy to convince me to join. Getting on the first flight out at 6 am flight, we met up in a nice restaurant for some shepherd's pie. Our first sight was Dublin Castle, remodelled in baroque revival style, the castle is still used by Irish president for state visits, I would advice you to check out the chapel as well. Then after a short stop at Christ Church Cathedral we spend our first evening and night in Temple Square.\\

Dublin: Dublin Castle****, Christ Church Cathedral***\\

January 3: Dublin\\
Trinity college is famous for its library and the ancient book of of Kelis, which was created in the 9th century.
Then we moved on to the national cathedral of Ireland - St. Patrick's cathedral -- after a very short stop by Dublin's City Hall with its classical rotunda. St Patrick's cathedral is built in gothic style - the largest church of the country. Next we opted for the four glass option at the Whiskeymuseum. And last but not least learnt about all details of brewing at the Guiness Storehouse. Dinner was served at a former church, transformed into a stylish restaurant, including live music and dance performances.\\

Dublin: St Mary's Pro Cathedral**, Trinity College*****, City Hall**, St Patrick's Cathedral****, Whiskey Museum****, Guiness Storehouse***\\

January 4: and off we go to Northern Ireland:\\
Today's the day when we hop on a bus taking us all to the Northern tip of the island. Our bus driver told us all about the positive changes achieved after the signing of the Good Friday agreement. The first stop were the Dark Hedges by Ballymoney, which are featured in a couple of TV series nowadays. Then the trip continued with a hike by the cliffs of the Northern Sea up to the Carrick-a-Rede Rode Bridge. Having grown up far away from any coast, I am amazed again and again by the sea. Since we had quite a bit of sunshine! we had even a good few up to Scotland. The centrepiece of the trip was though the Giant's Causeway. Unfairly called one of the most disappointing tourist attractions of Ireland by some surveys, it is actually an scenic natural sight. About 40000 basalt columns lead all up to the coastline, the remains of a volcanic eruption millions of years ago. After a nice quick lunch by the Nook Pub and a photo stop by Dunluce Castle in sunset we had a one hour layover in Belfast with its gigantic illuminated City Hall.\\

Ballymoney: Dark Hedges***\\
Ballycastle: Coastline with Carrick-a-Rede Rope Brige****\\
Bushmills: Giant's Causeway*****\\
Belfast: City Hall****\\

January 5: flight back: 
Getting up really early in the morning, jumping on the Airlink express and off with Air Lingus back home. A nice trip, a relaxed start in the year, even the weather was pretty nice, so a mixture of leisure, culture, and after all even nature. Maybe not enough of cultural highlights for people who are looking for sights, which blow your mind.\\

Unfortunately this was already the end of my trips with Chris (out of two overnight trips, and one day-trip). They were always fun, including nice evenings or nights out. Chris didn't move back right away like most other co-travellers, sticking around the area until begin of 2019, but another trip or hike just never materialised, but rather afternoons or evenings watching either Premier league or Champions League matches.

\section{January 13--January 15: Carcasonne \& Avignon}
\label{2017:Provence}

January 13: train ride to Avignon\\
Avignon: City Walls**\\

January 14: Carcassonne\\
Carcassonne: Canal Du Midi***, Cite de Carcassonne*****, Chateau Comtal*****, Basilique Saint-Nazaire****, St Vincent***, Cathedral*** (Tower****)\\

January 15: Avignon \& Villeneuve-les-Avignon\\
Avignon: Cathedral****, Pope's Palace*****, Rocher des Doms****, Pont Saint-Benezet***, Petit Palais***, Basilique St Pierre****, Lapidarium***\\
Villeneuve-les-Avignon: Fort Saint-Andre****\\

And then it was time to retire my CANON EOS 600D which had been my faithful companion since my 2012 trip to Prague \ref{2012Prag}. We travelled through many countries, covering three continents, taking 10'000s of pictures. After all it found a new nice home at another PhD student's place later on. Hopefully you are doing fine. 

\section{February 10--February 12: Belgium}
\label{Belgium2017}

February is the month I take least photos. Now you can say well on average a February has about 28.25 days, clearly less than any other months, so what gives. While that is true, I have a significant dips in trips and travel in February too (most probably after travelling around new year/christmas I don't feel the urge to go somewhere. But then there is the odd chance I get outside of the country from time to time, for example to Brussels, taking Riju and my new camera, a CANON EOS 80D with me.\\

Co-travellers:\\
Riju likes to see more countries, I convinced him that Belgium offers nice towns and nice food and nice drinks. Since he had not been in Brussels before, he decided to spent the first day there, while I would go to Bruges first and then to Ghent, where we would meet up again.\\

February 10: Brussels night out\\
As usually I was on the late evening flight to Brussels, once again I booked at the quirky moroccan styled Hotel Mozart (the irony) which was just a few metres away from Grand Place. Riju and I took a couple of night photos - and yes this camera takes so much better photos at large ISO as you need without a tripod. Then we had a few drinks at the Au Brassuer by the Stadthuis and then it was time for some sleep, particularly for myself, considering I would get up about 2 h earlier than Riju (also missing out on breakfast, which was actually included in the hotel room booking).\\

Brussels: Grand Place*****\\

February 11: Bruges \& Ghent\\
My parents had visited both Bruges and Ghent previously and both preferred Bruges. I still booked the night in Ghent, but that didn't keep me from visiting Bruges. In fact since they also weren't that fond of the flemish Begijnhofs either, I gave that a miss (OK world UNESCO heritage, but if you have no time, you gotta sacrifice something). After a short stop by the Belfry it actually stated to snow, while I rushed to the town hall to escape it. Event my little backpack was considered too large, and I had to leave it at the info desk. The Golden Room of the town hall was very nice with murals depicting important historical events of the city, as well as golden statues of important figures of the city. Next to the town hall is the Vrije, where the castellany was located in early times. The main room show some excellent wood carvings on the wall and the ceiling. The third important building of the Burg square is the basilica of the Holy Blood in neogothic style, which houses the relic of the precious blood, a cloth soaked with blood of Jesus Christ. After a short stroll to the Grote Markt with the Belfry and the Provinciaal Hof, I walked along the Rozenhoedkaai. In the church of our lady are the tombs of the last Valois Dukes of Burgundy as well as Michelangelo's sculpture of Madonna and Child, the only sculpture leaving Italy while Michelangelo was alive. The cathedral of St Salvator is a nice church, though overshadowed by the church of our lady, the tower was made taller in the 19th century to appear more like a ``real'' cathedral.\\
 Then it was time to get on the train back to Ghent. There I first visited the St Bavo's Cathedral, a gothic church with a largely baroque decoration, and one of the most precious gothic altarpieces, painted by Hubert and Jan van Eyck, the so-called Ghent Altarpiece. Next to the cathedral is the 91 m tall belfry, the tallest one in Belgium. Opposite of the cathedral is the gothic Niklaaskerk, which also houses an old 19th century organ. After the churches I went to the tourist information to obtain tickets for the two famous Hotels of Ghent, getting the last available ticket for the day. The tour stated at Hotel Clemmen at the Chinese Salon, and then continued with the Hotel d'Hane-Steenhuyse on the opposite side of the road. The main facade is on the garden size, built in a classic style. This large house had been the refuge of the French king Louis XVIII. during Napoleon I's 100 days. The main hall is the Italian Hall with a precious parquet floor as well as a giant carpet and a large frescoed ceiling, covering both floors of the house. The more important rooms are located in the upper floor of the house, but the king preferred not to climb stairs, thus a ground floor room was made into the King's room by the city.\\
 Once I finished the tour, Riju had already arrived and checked into our hotel room. Since I didn't know who would arrive first I had booked the room, but given instructions in case Riju would arrive before me, to make sure he would get the key as well. Once I arrived the lady at the reception said, it was very untypical but very accurate description so she felt she could give him the keys. Once we were ready for food, we did a couple of night photos (town hall, belfry, cathedral), but more interesting were photos of the Graslei quay as well as the Gravensteen castle. We had a couple of fried food at one of the many friteries, and then a couple of beers at the Dulle Griet pub, rumoured to have the largest collection fo Belgian beer in the city.\\

Bruges: City Hall****, Vrije*****, Basilica of the Holy Blood****, Great Market****, Church of our Lady*****, Cathedral St Salvator****\\
Ghent: Cathedral St Bavo with Alter of Ghent****, Niklaaskerk****, Hotel Clemmen****, Hotel d'Hane-Steenhuyse****\\

February 12: Ghent \& Antwerp\\
Starting our day at Ghent we took a couple of photos along the Graslei before visiting the Gravensteen castle. One of the most famous castles of Belgium several arms are exhibited there, the halls and rooms are sparse as expected from medieval times, the tower top offers a nice view of old town, particularly dominated by the three towers of the cathedral, the belfry and the Niklaaskerk. And then it was time to take the train to Antwerp. \\
Antwerp Centraal is one of the most beautiful train stations I have ever seen. The main hall was built in the 19th century, by now new platforms have been built underground mainly for the high speed trains. We had a short lunch at the train station, getting some chocolate in the ground floor of the old royal palace, the Palais op de Meir (the main rooms are on the first floor) and then we saw the cathedral. Some of the altarpieces were painted by Peter Paul Rubens, the cathedral has also nice stained glass windows, and a nice fresco in the square tower. The Grote Markt is surrounded by guildhalls, the exhibit about the city's history in the city hall was nice too. At this point Riju got himself another snack by one of the friteries, while we walked over to the old port with the small Hetsteen castle. Since we still had a bit of time, we decided to get into the Rubens House, the residence of the famed Flemish artist, who also collected a couple of paintings from his contemporaries like van Dyck, Brueghel or de Vos. The court facades and the garden were nice too. Then we stopped at the Stadtfestsaal shopping mall before getting back to Brussels airport where we had our dinner before flying home to Geneva again.\\

Ghent: Gravensteen***\\
Antwerp: Antwerp Centraal*****, Palais op de Meir***, Cathedral*****, Great Market****, Hetsteen****, Rubens House*****, Stadtfestsaal***

\section{March 12: Switzerfrance: Lac des Brenets}
\label{2017:LacDesBrenets}

Co-travellers: \\
Sarah:
US-American: can you imagine, I also know non-physicists. Sarah is a science writer, who is very good in explaining to a none-science audience about particle physics research, as well as teaching scientists on how to become better communicators themselves. When Sarah needs a break from that pretty exhausting business, Sarah typically relaxes canoeing on the wild rivers around lake Geneva or climbing and hiking in the alps, or in American national parks (and why not crossing glaciers while at it). And sometimes she even manages to get not that much in shape physicists to tag along.\\
Riju:
Riju is not only up for city trips (see all previous trips) but also for decent hikes, as long as they don't go too wild by almost crossing altitude differences of 1 km. \\

How it all started: Saturday afternoon, sitting home alone, minding my own business, I get a message from Sarah: ``bored as well, we should do something tomorrow''. Agreeing with Sarah we check out hikes which are not that far from home (aka less than 2 h away), which still seem to be decent to do and not over the snow line (somewhere around 1000-1200 metres at that point). And we find a pretty cute hike in the Jura along side Lac des Brenets (which I did back in 2010, when I missed the boat back in work outing trip, and thus missed lunch). Said and done, now we wonder if we should take someone else on the ride, and Sarah leaves Riju no choice, but just tells him to show up at a certain time and a certain place "It will be fun". And off we go: Lac des Brenets is a small lake in the Jura along the border between France and Switzerland (Canton Neuchatel). The lake is about 3.5 km long and meanders through a narrow gorge. We decided to take the the ``difficult'' path, which is a pretty tight path (around 30-50 cm) going midway through the cliff sides for a while. At the end of the lake, we keep following the river of Doubs until we reach the end of the path by the 27 m high waterfall of Saut du Doubs. In March the water levels are pretty high. In winter the lake surface can freeze completely. Rarely the lake dries up due to the lack of rainfall providing enough water supply. Riju and Sarah dared to go closer to the waterfall edge, I had a bit more respect of the thunderous amount of water falling. And we then went back to the lake, crossing over to France and up the hill for a better panoramic view. On our way back we met a group of people, selling cake and coffee for donations - which we happily made use off. Back by the start of the lake we decided to move on to the city of Neuchatel. In between we stopped by snow covered fields to enjoy the silence for a couple of minutes. In Neuchatel we first went up the castle hill and then down to the harbour to have some Mexican food by the lake Neuchatel, and then back home. A nice hike for a slow start into the hiking season, even suited for not experienced people (but these should use the wide path).\\

Brenets: Lac des Brenets*****,Saut du Doubs*****\\
Neuchatel: Chateau de Neuchatel***\\

\section{March 24--March 26: Umbrien}
\label{2017:Umbria}

March 24: train and pit stop in Milan:\\

March 25: Assisi\\
Assisi: Cathedral San Rufino****, Basilica di Santa Chiara****, Chiesa Nuova***, Palazzo del Comune***, Tempio di Minerva****, Basilica Papale di San Francesco d'Assisi***** (Lower Church*****, Upper Church*****) , Oratorio dei Pellegrini****, Basilica Papale di Santa Maria degli Angeli*** (Porziuncola****)\\

March 26: Perugia\\
Assisi: Cathedral San Rufino**\\
Perugia: Sant'Ercolano****, Basilica di San Domenico***, Basilica di San Pietro****, Chiesa del Gesu****, San Filippo Neri****, Cathedral San Lorenzo****, Capella di San Severo***, Palazzo dei Priori****, Collegio del Cambio****, Rocca Paolina*****

\section{April 14--April 17: Kiev}
\label{Kiev2017}

April 14: flight to Kiev:\\
no photos at train station:

April 15: Kiev\\
Kiev: Vladimir Cathedral****, Golden Gate***, Gorodetsky House*****, Pechersk Lavra (Cave Monastery)***** (Cathedral of the Dormition*****, Gate Church of the Trinity*****, Refectory Chambers****, Church of the Saviour at Berestov*****), Feodosiyivskyy Monastery***, Sergius of Radonesch Church****, Monastery St Michael****, Sophia's Cathedral*****

April 16: Kiev\\
Kiev: Nikolai-Prytyska-Church***, Florivsky Monastery****, Mary Ascension Cathedral***, Alexander Church***, Nikolaus Cathedral****, Airplane Museum*****\\

April 17: flight home:\\
impatient CERN folks

\section{April 28--May2: Frankfurt}
\label{Frankfurt2017}

April 28: Frankfurt night cityscape\\
Frankfurt: Main Tower Observatory****\\

April 29: Aschaffenburg \& Wiesbaden\\
Aschaffenburg: Park Sch\"onbusch and Schloss Sch\"onbusch****, Johannisburg Palace***, Pompejanum****, Frauenkirche***, Stiftsbasilika St Peter \& Alexander****\\
Wiesbaden: Stadtschloss*****, Kurhaus****\\
Frankfurt: Kaiserdom St Batholom\"aus***** (with Tower*****)\\

April 30: Bruchsal, Mannheim, Speyer \& Mainz\\
Bruchsal: Palace*****\\
Mannheim: Palace**** (with everything open*****), Jesuit Church****\\
Speyer: Dom***** (with Kaisersaal*****)\\
Mainz: Dom****\\

May 1: Kassel\\
Kassel: Orangerieschloss \& Marmorbad*****, Schloss Wilhelmsh\"ohe***** (Weissensteinfl\"ugel****, Bergpark*****, L\"owenburg in Renovation***)\\

May 2: flight back:\\
empty airport etc etc

\section{May 24--May 28: Madrid}
\label{Madrid2017}

I felt it was time to see a bit more of Spain, once again choosing Madrid as base, and getting to other places from there, I booked most of my tickets well in advance to benefit from the cheap high speed train tickets. I also bought my ticket for the Royal Palace, but then the King had a meeting there on very short notice, so sadly the palace administration cancelled on me just about a week before the trip.\\

May 24: Madrid four towers night view\\
Having arrived in Madrid I was very happy to see how large the room was, also with a magnificent view of the illuminated Cuatro Torres Business Area. The four skyscrapers are between 220 and 249 m high, illuminated in several colours as well. Seems my digestive system was not happy about food I had at the airport, at least I woke up several times rushing for the rest room, thus for the next two days I set myself on a diet with bananas, pretzel sticks, and coke and water. That was not the way I wanted to start my holidays, but you sometimes get what you don't like.

May 25: Salamanca\\
Salamanca: University****, Cathedral*****, Old Cathedral*****, Convento de San Esteban*****, Palacio de la Salina****, Palacio de Anaya***\\

May 26: Valencia\\
Valencia: City Hall****, Basilica of the Virgin of the Foresaken (Virgen de los desamparados)****, Cathedral***** (Holy Grail's Chapel*****, Treasury*****), Sant Nicolau*****, Almudin***, Banos del Almirante***, Palacio del Marques de Dos Aguas****, Thomas \& Philipp Neri Church**, Palau de Cervello**, Mercado Central***, Llotja de la Seda*****, Science Museum***** (the actual Museum exhibit***), Hemisferic*****, Palau de les Arts Reina Sofia****, Church Virgin Of Monteolivete**\\

May 27: Leon \& Oviedo\\
movies in a bus???\\

Leon: Cathedral*****, Basilica de San Isidro**** (Cloister \& Pantheon of the Kings*****), San Martin***\\
Oviedo: La Foncalada***, Cathedral***** (Camara Santa and Treasury*****), Balesquida Chapel***\\

May 28: Avila\\
Avila: Convento de San Jose****, City Walls*****, Cathedral*****, Santo Tome***, San Vicente*****, San Juan Bautista***, Capilla de las Nieves***, Real Monasterio de Santo Tomas****

\section{June 3--June 6: Athens}
\label{2017Athens}

Having not been in Greece for almost 20 years, Athens had been on my list of places to revisit for quite a while. \\

Co-travellers:\\
Riju: Since Riju appreciates ancient architecture he was eager to join, moreover since his friend MaryAnn was based in Athens at that time.\\
MaryAnn: Riju's friend, who works on her PhD in archaeology in Greece. Since she is based in Athens the was giving us all expert insights spending most of her weekend with us.\\
Amin: Having spent already a long weekend in Italy, Amin is ready for another trip and wants to see more of Europe.\\

June 3: Athens:\\
We booked a hotel close to the National Archaeology Museum a couple of months before our trip. But just about a months before our trip horrendous reviews of our original hotel appeared, linked to an ongoing issue with bedbugs. I didn't want to risk anything and found a good hotel quite close to old town for an affordable price. So I convinced everybody else that we indeed should stay at this other hotel. And indeed this hotel turned out to be just fine. After our long metro ride, getting a 72h public transport ticket, we arrived in Athens downtown, where Riju and MaryAnn set up our meeting by the Acropolis Museum. On our way to the museum we stopped by the Kapnikarea church, and old medieval church with nice frescoes. Unlike in 1999 photos were not allowed anymore. The cathedral of Athens was built after the city was elevated to the capital of Greece. Athens had been a small town around that time but rapidly grew into the metropolis it is nowadays. The cathedral is a large greek-orthodox church with lots of Icons and mosaics. We passed the monument of Lysikrates, Amin got himself a quick snack, and we met MaryAnn who took us to the Acropolis Museum. Many of the original statues have been placed in this museum in order to protect them from air pollution, rain, and wind, including five of the statues of the Erechtheion (the 6th statue was taken by the British and is now exhibited in the British Museum in London). MaryAnn gave us a tour of the museum outlining the evolution of greek ancient style and how certain artefacts can be seen in context of political issues and struggles as well. Even without the Elgin marbles of the Parthenon in place (instead one sees photographs of the missing pieces) the remaining sculptures of the frieze are still amazing and interesting to see. The development of greek sculpturing is explained as well using artefacts found originating from different times. The view of the acropolis from the museum is quite a sight too. Then MaryAnn took us to one of her favourite close-by restaurants before we returned to the Odeon of Herodes Atticus for a performance of Madame Butterfly by the Greek National Opera. Then we ended the day on a roof-terrace by Monastiraki with night views of the town having some cocktails.\\

Athens: Cathedral****, Acropolis Museum*****, Madame Butterfly Performance in Odeon of Herodes Atticus*****\\

June 4: Athens\\
We met MaryAnn once more in the morning for our tour of the Acropolis, where she outlined the history of the structures and temples, as well as excavation campaigns of the surroundings and the development of the different buildings through time. Since we started the day early, the Acropolis was quite empty, only one hour later substantially more people entered, by that time we left and went down from the Acropolis to the Agora. From medieval time the church of the apostles close to the Agora offers nice frescoes and mosaics, by then Athens was a little bit more than a small village. Besides the museum in the reconstructed stoa, the almost perfectly conserved temple of Hephaistos is the dominating remaining building of the Agore. Then we walked over to the old Roman Agore with the library of Hadrian. During the Roman times the large temple of the Olympic Zeus was constructed, one of the highlights of the corinthian style. Closeby is the Panathinaiko stadium, the site of the first modern olympic games, it had been also used during the 2004 olympics in Athens once more. Then we had time for a long lunch, where MaryAnn ordered a lot of local specialities. We also felt like staying longer, particularly after a large rain storm appeared and flooded the whole road, thankfully the restaurant had plastic planes all over the place, although the feet of the tables and chairs were soon flooded by quite a bit. After 10-15 minutes everything started to clear up though again. Then we took the bus to the Greek National Archaeology Museum. Full of artefacts from ancient greek times, from Mykene, Tiryns, Athens as well, and Thera. Amazing to visit and see it again. Then we had a last big dinner and drinks on the roof terrace once again.\\

Athens: Acropolis*****, Agora**** (Church of Apostles****, Hephaistos Temple****), Hadrian's Library****, Roman Agora***, Temple of Olympic Zeus****, National Archaeology Museum*****\\

June 5: Delphi\\
We took the local bus to get to the long distance bus station and bought our tickets for Delphi. It is a ruined city nicely located surrounded by hills. We walked through the upper part of the city with the roman Agora and the temple of Apollo, the theatre and the stadium. Back in 1999 one could still walked through the stadium, nowadays one can observe it from a hill side, but otherwise it is off limits. Then we walked over to the remains of the lower city with the Tholos, a round temple in doric order. In the local museum, friezes of old treasure houses, decorations and statues of the temple of Apollo, and several monuments are exhibited. Then we had a long lunch trying out many things, before taking the bus back to Athens.\\

Delphi: Excavations***** (Temple of Apollo****, Stadium*****, Theatre****, Tholos*****), Museum*****\\

June 6: flight home\\
We took the metro early in the morning, and back in Geneva we even had to go through security, although we moved from Schengen to Schengen they asked many questions, I never found out why. I mean as if they could refuse me entry as Schengen citizen.

\section{June 9--June 23: USA: Yellowstone}
\label{US2017}

June 9: flight to Salt Lake City\\
Great Salt Lake****\\

June 10: Shoshone Falls \& Dunes State Park\\
Twin Falls: Shoshone Falls*****, Snake River Canyon****\\
Bruneau: Dunes State Park*****\\
Bliss: Snake River Canyon***\\
Salmon Springs: Thousand Springs Reserve****\\

June 11: Craters of the Moon\\
Arco: Craters of the Moon*****\\

June 12: Yellowstone National Park\\
Yellowstone National Park: Madison River****, Firehole Canyon****, Fountain Paint Pots****, Upper Geyser Basin with Old Faithful Geyser*****\\

June 13: Yellowstone National Park\\
Yellowstone National Park: Gibbon Falls****, Norris Geyser Bassin*****, Mammoth Hot Springs*****, Beryl Spring***\\

June 14: Yellowstone National Park\\
Yellowstone National Park: Madison River****, Yellowstone Canyon***** (Lower Falls*****, Upper Falls*****, Cascade Falls****), Tower Falls****, Petrified Tree***\\

June 15: Yellowstone National Park\\
Yellowstone National Park: Mud Volcano Area****, Lake Village***, Natural Bridge****, West Thumb Geyser Basin*****, Kepler Cascades****, Upper Geyser Basin with Old Faithful*****, Firehole Lake Drive****\\

June 16: Mes Falls\\
Ashton: Upper Mesa Falls*****, Lower Mesa Falls****, Warm River Walk****\\

June 17: Grand Teton National Park\\
Grand Teton National Park: Phelps Lake Hike****\\

June 18: Grand Teton National Park\\
Grand Teton National Park: Leigh Lake****, Jenny Lake Hike*****, Cascade Creek and Hidden Falls*****\\

June 19: Grand Teton National Park\\
Grand Teton National Park: Jackson Lake \& Jackson Lake Dam*****, National Museum of Wildlife**\\

June 20: (non-)Periodic Spring\\
Afton: Bridger National Forrest hike to Periodic Spring****\\

June 21: Salt Lake City\\
Salt Lake City: Salt Lake City Library****, Cathedral St Madelaine***, Beehive House****, Joseph Smith Memorial Building***, Assembly Hall**, Utah State Capitol****\\

June 22: Great Salt Lake \& Salt Lake City Zoo\\
Magna: Great Salt Lake****\\
Salt Lake City: Hogle Zoo****

\section{July 5--July 16: Venice: EPS2017}
\label{Venice2017}

Having be selected for the European Physics Society high energy physics conference in 2017, I decided to spent a couple of more days in Italy, this time around Trieste. Reyer wanted to join for sure, Nate thought he might be able to join, but unfortunately it was Reyer and I alone after all.\\

Co-travellers: \\
Reyer: Not having been around many places so far, Reyer was all game to rent a car to get to Slovenia and Croatia afterwards too.\\
Riju: I convinced Riju to try to turn his note into a public document and present it in a poster at EPS, that the conference happened in Venice was of course an added bonus\\
Indara: also around to present her studies, Indara also spent one day in town for touristic adventures (besides enjoying a couple of dinners together otherwise)\\

July 6: Venice\\
After the first day of the conference I took a boat along most of the Canal Grande, getting a Spritz and a nice dinner with my colleagues, then we walked along some canals until after sunset and enjoyed some night views of the canals.\\

Venice: Canal Grande*****\\

July 7: Venice\\
The conference took place in the Palazzo del Casino on the island of Lido di Venezia. The rooms were decorated with lots of impressive mosaics. The building was constructed in the 1930s, nowadays the Casino moved to a Palazzo by the Canal Grande and the palazzo is used as congress centre.\\

Venice: Lido - Palazzo del Casino****\\

July 8: Venice\\
Venice: Collezione Peggy Guggenheim****, Santa Maria della Salute****, Palazzo Reale****, Biblioteca Nazionale Marciana*****, Palazzo Ducale*****\\

July 9: Loreto \& Ancona\\
Loreto: Basilica of the Holy House*****\\
Ancona: San Francesco delle Scale**, Palazzo Ferretti - National Museum of Marche***, Duomo****\\

July 10: Venice\\
Venice: Jesuit Church*****, Murano - San Pietro Martire****, Murano - Santa Maria e San Donato****, San Giovanni Crisostomo***, San Silvestro****, Basilica Santa Maria Gloriosa dei Frari*****, Ca' Rezzonico****, Basilica di San Marco***** (mit Pala d'Oro*****), Lido - Palazzo del Cinema*, Lido -Palazzo del Casino****\\

July 11: Venice\\
Venice: Lido - Hotel Excelsior***\\

Up to now unfortunately this was the last time Indara and I had a trip experiences together. Also travelling with Indara is more chaotic than with other people, that does add certain spice and fun too, when things have to be re-arranged and changed on the fly. After all, it always worked out in that sense, that I always got to see and experience what I wanted to, together with a fun, interesting, and charming travel company.\\

July 12: Vicenza \& Stra\\
Vicenza: Duomo**, Villa La Rotonda***, Villa Valmarana ai Nanai***, Basilica di Santa Maria di Monte Berico**, Piazza della Biade**, San Gaetano*, Santa Corona, Teatro Olimpico***, Santa Corona**, Palazzo Barbaran da Porto - Museo Palladio**\\
Stra: Villa Pisani***\\

July 13: Ferrara \& Padua\\
I planned to spend this day in Ferrara, but had in my back that in case I would finish early as I usually do, to check out more stuff in Padua, particularly the Botanical Gardens, which I decided against a year previously. I started in the Palazzo Comunale, the city hall of Ferrara. Besides the rooms for representation purposes there is an old but beautiful chamber of the duchess hidden among the other rooms, which is one of those little hidden gems. The cathedral was a bit of a let down, the famous beautiful facade was in renovation, a renovation which might have been dire need, inside all the roofs had been hidden by ropes and cloth to protect people from falling stones. Also other places had been closed off for visits, maybe also due to structural decay. Now there are also little nice Renaissance palaces famous for their ceiling frescoes scattered throughout the cities, some former garden palaces, others Palazzi with several floors, housing nowdays art and archeological artefacts. Only in one of those little palaces, the Hall of Twelve Months in the Palazzo Schifanoia the full decoration also of the walls can still be admired. The former seat of the dukes, the Castello Estense is more of a renaissance palace than medieval castle. Also there one saw parts damaged by earth quakes of the last decades, but plenty of halls and ceilings and wood panelling and the whole program.\\
But I still had time for padua, where i checked out the giant basilica of Santa Guistina again, before finally doing the botanical gardens. Most plants are nowadays housed in large green houses, the ancient part of the garden had more of a park flair though showcasing some plants or palm trees too. And last but not least the late gothic Cappella degli Scrovegni with its breathtaking paintings by Giotto. Curiously enough unlike last time I was asked to not use the timer on my camera again, as this is what they would call flash (not that a fading red light is damaging such as ACTUAL blue flash light, and not that I used the camera just a year earlier at the same place with the same setting). Anyways visit the chapel it is incredibly beautiful. And then I went back to Mestre.\\
I met up with Reyer and we got on the train to Trieste, who wondered why we didn't have the car yet. Instead we had late Kebab and a bit of pizza close to Trieste Centrale train station.\\

Ferrara: Palazzo Comunale****, Cathedral**, Basilica di San Francesco***, San Girolamo**, Palazzina Marfisa d'Este*****, Basilica Santa Maria in Vado****, Palazzo Castabili*****, Palazzo Schifanoia*****, Biblioteca Ariostea***, Castello Estense*****\\
Padua: Basilica di Santa Guistina***, Botanical Garden****, Cappella degli Scrovegni*****\\

July 14: Skocjan Caves \& Porec\\
Today we got up early, got on the bus to the airport to get our car from Hertz, opting for a small car, since we wouldn't need to take any luggage with us. And then we drove over to Slovenia. Usually I try to do most of my trips with public transport if possible. But unlike needing less than an hour from Trieste to the caves in Skocjan it would take about 3 hours to get to the closest bus stop, and that one would be 30 minutes away from the caves. And moving over to Croatia later would have not been possible at all according at least to online information. The caves are quite popular, although they are one of the rare caves which completely ban photography (they claimed copyright, most probably rather due to people falling behind or taking too much of time for photos). Anyway the cave is large with many halls with plenty of flow stones. The most interesting part is though the Reka river, which runs through the cave in its own little canyon at times about 20-25 metres deep, including little waterfalls here and there on the way too. The canyon itself is crossed on a bridge so you can get a clear view of it too. A really nice event (sadly without any photos). Close to the end of the tour you also pass a large terrace where a side river used to run along, but we were told it hasn't been flowing for years by now.\\
You exit the cave through a giant hall, and then you can either go back to the parking directly, or walk through another little cave. We decided for the second option. You first pass a little rock gate and other little caves here and there which are all fenced off. The small cave is a couple of hundred metres long, and it feels more like a natural high tunnel for the river with little cascades along the wide trail. Still nice to see, before and after the cave the Reka runs in its own little gorge through the forest. At the end of the trail there is the option to climb one a view point over the cliff and the whole canyon or rather the actual collapsed roof of the doline opening in front of you. I would recommend to do all the optional parts for those fit enough, the 60-90 minutes are well spent. And then we had a nice dinner by the cave entrance opting for local meals.\\
And then we drove over to Porec in Croatia. Seems also others had a similar idea, at least a traffic jam formed by the border, which is one border not within the Schengen space. Porec is a nice town just by the sea, famous for the ancient Euphrasian Basilica with Byzantine mosaics and old columns from Early Christian architecture. After a visit of the ancient church, its baptistery and walking along the sea we had dinner on an old town square before driving back to Trieste via Slovenia. While crossing the Croatia-Slovenian border the border guard once again didn't really know how to deal with Reyer's Swiss Card. This residency card is a special Swiss category for people employed and working at International Organisations, thus it differs quite a bit from typical longterm Schengen visas. While for myself my Schengen passport gets me into Schengen countries without any issues, for folks from countries outside of Europe it is often a hassle when reentering Schengen, as one cannot expect a normal border guard to just have knowledge of such special categories. But a couple of minutes later we were let in (after all it is a valid category, and online this information can be checked too). Having arrived in Trieste, we got ourselves some ice cream and enjoyed the sunset over the Adriatic Sea.\\

Divaca: Skocjan Caves*****\\
Porec: Euphrasian Basilica*****\\

July 15: Aquileia, Cividale del Friuli \& Trieste\\
Aquileia: Basilica***** (Crypt*****) , Battistero***\\
Cividale del Friuli: Cathedral****, Museo Cristiano****, Santa Maria in Valle \& Tempietto Langobardo*****, Archeological Museum*****\\
Trieste: Miramare Castle****\\

July 16: Grotta Gigante \& Trieste\\
Sgonico: Grotta Gigante*****\\
Trieste: Cathedral***

\section{July 21--July 23: Oslo}
\label{Oslo2017}

OK I know, in Norway you usually want to see the nature, and trust me I do want to see the Fjords too, but I also wanted to see Oslo, and also visit my friend Messay who moved there from Geneva a couple of years ago, and who I didn't see for a couple of years by now. So two important reasons to add it onto the agenda. I even found flights which allowed me to go there without the need to take days off, with a risk of changing planes within 35 minutes each, but at Brussels and Zurich which are known for their short switches should you stay within the Schengen area. Still I worried a bit, but in the end I made the connection smoothly without any troubles.\\

July 21: flight\\
After arrived in Oslo a one hour train ride later I arrived in downtown Oslo, checked into the hotel and walked over to the cathedral. There I was particularly intrigued by the painted wooden ceiling, which was so different from the many medieval (or revival type) wooden ceilings I knew. A completely different flair, also rather a modern interpretation which made it quite special. Although it was summer it was already dark enough to get a night flair out.\\

Oslo: Dom****\\

July 22: Oslo\\

And then I met up with Messay, first we did a nice walk along the Oslo Fjord (geologically actually not a Fjord) and then we had dinner and loads of cheese and hand baked bread at his place, enjoying the sunset over the Fjord. And then once he told me it is getting late, only then I realised my watch had stopped working (strangely enough that happened far more often when I am on trips than when I am at home).  Still made it obviously back to my hotel.

Oslo: Royal Palace****, City Hall*****, Akershus Fortress****, Opera***, National Gallery****, Vigelandpark*****, Dom****\\

July 23: Oslo\\
Oslo: Oskarshall****, Viking Ships*****, Fram****, Kon-Tiki Museum***, Norwegian Museum of Cultural History****, Norwegian Museum of Cultural History*****

\section{July 28--July 30: Munich}
\label{Munich2017}

A sad day in history: saying good-bye to Eric and Siyi. While I was in Venice, I talked to Siyi about her impending move back to the US. She told me on one hand they are happy since it will mean more life stability for both of them due to her new tenure position in Hawaii, but on the other hand it means leaving all friends behind. She then suggested that I could technically come over for their last weekend in Munich (should I want to). Since I could stay at their place, I only had to check for flights, and since those didn't seem to be outrageously expensive, I booked right away.\\

June 29: Munich\\
Munich: St Peter****, Old City Hall***, New City Hall****, Frauenkirche***, Michaelskirche****, B\"urgersaalkirche****, St Anna****, Theatinerkirche****, St Ludwig***, Lenbachhaus*****\\

June 30: Munich\\
Munich: Schloss Nymphenburg***** (without park palaces this time), Villa Stuck****, Pinakothek der Moderne*****, Neue Pinakothek****, Alte Pinakothek***, Glyptothek****, Antikensammlung****\\

Up to this day, this marks the last time I saw Eric in person, besides working together for about four years at CMS, we also did five trips together involving overnight stays and three additional day-trips or hikes. Siyi and him moved all the way to the other side of Earth, to the big island of Hawaii, where Siyi continues here Astronomy research and Eric moved on to the industry world.\\

\section{August 26: Glacier 3000}
\label{lesdiablerets2017}

Walking on a glacier sounds amazing, and this is exactly what you can do by the Tsanfleuron glacier (particularly since it doesn't have any huge crevasses). Originally I wanted to do this trip already a week earlier with Riju, but then he wasn't around, told me he overslept. I got my ticket on the day and that one would be valid for over a week, and i just would need to use it a week later. In any case a cancellation of the train by SBB facilitated the decision too, since the next one would arrive about 2 h later. Anyway one week later, August 26, the day I HAD to use this ticket, Riju called in sick, so once again I faced the fact that I had to do this trip on my own -- at least the weather was beautiful. The glacier of Tsanfleuron is a dying glacier, even in summer almost nothing is accumulated, in a couple of decades nothing will be left. Still a nice adventure to walk across all of the glacier. On the way it becomes obvious how huge amounts of ice melt on a normal summer day. Once I arrived at the hut by tour St Martin I did enjoy the views of the surrounding mountain peaks, glaciers and valleys, while having a cheese platter, and then I walked back over the glacier, taking the cable car from the Glacier 3000 station down to Les Diablerets.

Les Diablerets: Glacier de Tsanfleuron*****

\section{August 27--August 28: Split}
\label{split2017}

August 27: Trogir \& Split\\
Trogir: Cathedral of St Lorenz*****, Fortifications****\\
Split: Diocletian's Palace*****, Cathedral (Mausoleum)*****, Temple of Jupiter****, Franciscan Church*, Kastelet****, Ivan Mestrovic Gallery****, Great Papalic Palace***

August 28: flight home:\\
airport in construction

\section{September 6- September 11: The mother of all trips -- Rome}
\label{2017:Rome}

Co-travellers:\\
Riju: US-American: by now Riju knows what expects him while travelling with me, particularly when going to my most favourite country -- Italy. And Rome is the city with by far largest number of things to do. Even after an warning, Riju is still up for all day walking through Rome. An avid photographer himself, Rome might prove to be a paradise for Riju as well.\\
Amin: Iranian: After two previous travels Amin also knows what he is getting himself into. But Amin was chosen as an instructor at a school in Bari, thus joining us a day later, so we still keep the highlights for the last days.\\

How we ended up here:\\
Rome is ALWAYS worth a trip. In fact this is my third trip to Rome. From the Roman ruins to the huge amount of churches, particularly from Baroque times, Palazzos of Rome aristocracy. The centre of Christianity with the vast Papal collections in the Vatican (as well as in their private palace). Technically it isn't even that long to get from Rome to Florence, Pisa or Naples either, so if you should want to spend two weeks in Italy plan at least 3-4 days in Rome, but it could also serve as starting point of a trip through all of Italy. Last time I visited, the Domus Aurea was closed for renovation. I realised it opened up again for visits, thus one more reason to go to Rome again. Since Riju and Amin didn't see Rome yet, it was easy to convince them to join.\\

September 6: Rome\\
Flying over the alps with the ice-covered Mont Blanc in full sight. Our hotel is just a few steps away from the Spanish Steps: Based on tripadvisor's rating \#1175 out of 1277 hotels in Rome (significantly better on booking where we reserved the room). And indeed it was a fine room, not really modern but comfortable enough, with working wifi and working air conditioning, and a clean bathroom. So what more would you need, but still pretty relieved. Then we went over to see the Trevi Fountain by night - magical (though be advised that even close to midnight, you want be the only one).\\

Rome: Trevi Fountain*****\\

September 7: The smallest country of the world: The Vatican City in Rome\\
One of the advantages of getting early in Rome are free views of typically crowded places: in our case the Pizza di Spagna with the Spanish Stairs. We were pretty much alone at 7 am. Once we arrived at St Peter's we went up to the Dome with really nice views over Rome in early sunshine. And then the overwhelming gigantic huge basilica of St Peter's: so much to see here, dozens of monuments, Michelangelo's Pieta, Bernini's Cathedra Petri and Bernini's Baldachin among others. Since we had timed tickets for the Vatican Museum I pressured Riju into leaving. He complained we hardly had spent any time in St Peter's - exif data showed that we had spent though shortly less than an hour inside. So maybe plan rather around 90 minutes for St Peter's. The Vatican Museums cover several topics: a collection of Roman and Greek artefacts, a vast collection of baroque paintings, cars and carriages of the Pope's, tapestries, collections of books - and last but not least the former apartments of the Popes themselves. The rooms of Raffael are  a suite of reception rooms famous for Raphael frescoes, the most notable is the School of Athens. Always the highlight of a visit to the Vatican together with the Sistine's Chapel. The Sistine's chapel is the place where the cardinal's elect the Pope, the whole ceilings and paintings are fully covered in paintings by Michelangelo, you might have seen excerpts of the ceiling, e.g. the creation of Adam (BTW no photos in the Sistine's Chapel). Continuing with our topic of the day in the Castel Sant'Angelo. Constructed as mausoleum for the Roman emperor Hadrian between 123-139 (in fact you still can see the burial chamber), the structure was converted into the papal residence in the 16th century, the rich apartment was set up to ensure the pope could still live in lavage even in case of a future siege. And then came the let down of this day -- the Ara Pacis Augustae, the Roman altar dedicated to peace, consecrated at 9 BC. Parts of the frieze are now scattered all over the world, but most parts of the altar are still here in Rome. The problem is -- the altar is in a glass building -- AND costs 10.5 EUR (for a 10-15 minute visit). That is just way too expensive for what you get to see, consider that the Colosseum together with the Forum Romanum and the Palatine hill is e.g 12 EUR only, the baths of Caracalla with 8 EUR even cheaper. Rome has so many churches, you easily step inside any random church, and you might be blown away. This has the disadvantage that you just might be overwhelmed at some point and not appreciate what you see anymore. One example of a church which might have deserved a bit more attention: Basilica dei Santi Ambrogio e Carlo al Corso, a fine baroque church. By then we really needed to grab some food, thus we had a mountain of ice cream by Piazza de Popolo. After paying a visit to Santa Maria del Popolo, we walked up the Pincio and through the Villa Borghese to the Casino Nobile of Galleria Borghese. I have been there twice, and on both occasions all tickets for the day were fully booked, even just after opening. Unlike a couple of years ago photography was allowed (without flash). The highlights of the museums are the many sculptures by Bernini, collected by the Borghese family. Then we walked over to the Baths of Diocletian, which Michelangelo transformed into a large church.\\

Rome: Spanish Steps****,St Peter's Basilica*****,Vatican Museum*****,Castel Sant'Angelo*****,Ara Pacis**,Santi Ambrogio e Carlo al Corso****,San Giacomo in Augusta***,Chiesa di Gesu e Maria**,Santa Maria del Popolo***,Galleria Borghese*****,Santa Maria degli Angeli e dei Martiri****, Fountain of Moses***,Santi Ildefonso e Tommaso da Villanova**,Santa Trinita dei Monti****\\

September 8: Amin joins us and paying a visit at the President's palace in Rome:\\
Riju and I started the day on our own in the Basilica of Santa Maria Maggiore - one of the big four (the other being Laterano, St Peter's, St Paul's), or big 6 if we extend the list by San Sebastiano and San Lorenzo. The church has been modernised in baroque times, particularly two side chapels have been constructed. The Borghese Chapel is covered in gold, while the Sistine's Chapel (familiar names, aren't they) is covered in frescoes and statues. The main nave and aps of the large church are covered in mosaics, a majority of those as old as the 5th century. Closeby is Santa Prassede, the chapel is once again famous for its old mosaics ( this time). Then we stood in front of the closed gates of Santa Pudenzia. This church is dedicated to the sister of Prassede, once again containing a really old mosaic. Riju told me not to be that German and just accept that in Italy some churches sometimes happen to open a couple of minutes late (more than 15 minutes late in the end). After a short stop by Sant'Andrea a Quirinale* and the rooms of the holy Stanislaus Kostka we arrived by the gates of the Palazzo del Quirinale. Quirinale used to be the palace where the Pope's spend most of their time, particularly in summer times. Later Napoleon used the complex as his residence in Rome, then it was the seat of the Italian kings, and nowadays the palace serves as seat of the Italian president. While waiting we saw a couple of cars being escorted out by lots of police. And the palace once again didn't disappoint. Although Quirinale is hardly listed as the sight to see in Rome, I would definitely advice you to put it into your itinerary and book in advance (tickets are very cheap). The webpage of the Italian president allows you to see all rooms in a virtual tour, so check out if it is something for you. This time three of the larger rooms have been closed for renovation, including the large palatine chapel, still we saw 28 rooms on the main tour, and later the attached museum of the constitution of the Italian Republic from 1947. By now Amin had arrived in Rome, thus after a short stop in the tiny cut church of San Carlo alle Quattro Fontane (as the name says, check ouf the four fountains at that corner) we all three met up by the Lateran. Giovanni in Laterano is in fact the cathedral of Rome and thus the seat of the bishop of Rome. The church is built in high Baroque, filled with the statues of Apostles in the main nave. In between we went to the Baptistery, which is in a separate building (as often the case in Italy). Close-by are the holy stairs, which are according to the legend the stairs Jesus went up to face Pilate in his trial. Helena, the mother of emperor Constantine brought them from Jerusalem to Rome. Then we went back to the city centre, where we checked out the church of Sant'Ignazio with its large ceiling fresco, continuing with the Pantheon. The Pantheon was built around the 120s in ancient Rome, one of the best preserved buildings of the ancient Rome. The next church on our list was Santa Maria sopra Minerva, the only gothic church in Rome. Our next booking was the Palazzo Farnese, one of the most important Renaissance palaces existing, nowadays housing the French embassy, where we were supposed to listen to a 45 minute long tour in French. We were told that we would be accompanied by a film crew, which was shooting part of a documentary. Nothing wrong with that, what we didn't expect was that our tour guide would not only talk a bit about the few rooms (e.g. the room of Hercules with a nowadays replica of the Farnese Hercules and the Carracci gallery), instead she talked about 100 minutes non stop to the film crew, while we at some point lost track of what she said. Unlike last time, maybe because of the film shooting, the windows of the Carracci gallery had been opened (still no photos allowed). Sant' Andrea della Valle and the ancient forum of Largo di Torre Argentina were brief stops on our way to Il Gesu, the first Baroque church of the world, containing the tomb of Ignazio di Loyola, the founder of the Jesuit order. At that point Riju and Amin opted out and had a short stop at a Cafe, whereas I continued to see the private apartment of Ignazio di Loyola, the church of San Marco and then the Palazzo Venezia, which had been the seat of Mussolini during the fascist times in Italy. Our last point of this very long day was the Capitoline Museum. It houses the foundations of the temple of Jupiter, the statue of Mark Aurel and fantastic views of the Forum Romanum. And then we had an evening stroll along the imperial forums before having another Pasta close to the Trevi fountain. \\

Rome: Santa Maria Maggiore*****, Santa Prassede****, Santa Pudenzia***, Sant Andrea al Quirinale***,  Palazzo Quirinale*****, San Carlo alle Quattro Fontane***,  San Giovanni in Laterano*****,  Battisterio Laterano****, Sancta Sanctorum****,  Sant'Ignazio****,  Pantheon*****,  Santa Maria sopra Minerva****,  Palazzo Farnese****,  Sant'Andrea della Valle****,  Largo di Torre Argentina***, Il Gesu*****,  Stanze di Ignazio di Loyola***,  San Marco****,  Palazzo Venezia****,  Capitoline Museum*****, Imperial Forums**** (no need to pay for getting closer),  Trajan's Column*****\\

September 9: Rome\\
Starting the day very very early today, thus checking out some of the many tiny churches of Rome before getting Amin to the Trevi Fountain -- seems early morning gives you emptier views than late evening/night. Riju planned to see the Palatine hill, the Colloseum and Forum Romanum today, thus we decided to meet maybe late afternoon. By now very few people were actually around, thus we stepped into the Pantheon again to see it free of large groups of tourists. Same for the fountains of Piazza Navona and the beautiful church of Sant'Agnese in Agone. Then we paid a visit to Santa Maria dell'Anima, the church of German pilgrims, with the tomb of the last German Pope (before Benedict XVI took office). The church of Santa Maria della Pace contains a couple of frescoes by Raffael. After a short stop on the church on the Tiber Island we crossed over to the other side of Rome, Trastevere. This time I also have seen the crypt of the Basilica di Santa Cecilia, which had been closed last time i had been in the church. After seeing Santa Maria in Trastevere with its 12th and 13th century mosaics Amin preferred to move on to St Peter's and the Vatican museum, thus I spend most of the remaining day by myself. After a short stop by Santa Maria della Scala, I saw the part of the Italian Gallery in Palazzo Corsini. Next I crossed back over the Tiber and saw the famous Galleria Borromi of Palazzo Spada. This corridor is actually pretty short, but perspective makes it appear pretty long. After a short stop at the inner courtyard of Palazzo Mattei with very nice friezes all over. A bit disappointing were the remains of the Stadium of Domitian, hardly anything is left, but could have been skipped. I always like walking up the Basilica Santa Maria in Aracoeli, typically it would have been more natural to do it with the other parts of the Capitoline hill. Now I asked myself what I should do next, churches are closed over lunch time, so I would have some leisure time. Seems I just was done a bit too quickly. Thus I decided that it would be nice to see the Forum Romanum and the Colosseum once more. Don't get me wrong, those are very interesting based on history and also from an artistic point of view. The queues are usually acceptable too, particularly if you don't do the Colosseum first, but the Forum Romanum instead (coming from the Imperial forums, the queue was basically non existent. After the Forum Romanum I went to San Celemente first though, a church which has three different layers of history. The church itself built in the 11th and 12th century with lots of mosaics. One floor lower you can see the early christian church, built in the 4th century, left with early frescoes. The third level are rooms from Roman times (no photos in the whole complex), and then off to the Colosseum. Then we all met to have a walk on the park next to the Colosseum with all the remains of Trajans' Baths. After dinner we enjoyed the night views and illumination of St Peters, the Lateran, Colosseum and the Forum Romanum. Oh and another anecdote - the evening before we were taking the metro home when this group of teenage girls appeared, dancing singing, just behaving unnaturally different. We all thought - maybe just crazy teenagers, but then Riju mentioned he felt one of them might have tried to get into his pockets - or just brushed past. Well guess what on this very late evening, they did appear again, more strangely this time they decided to just sit next to us - in an otherwise totally empty car - strange isn't it. Well this time singing again I felt a hand coming up to my pocket and I just slapped it hard. A loud scream and a station further ahead all of them stormed out. So teenage girls -- never try to pickpocket the same group of tourists twice.\\

Rome: Sant'Andrea delle Fratte***,  San Silvestro in Capite***,  Santi Claudio e Andrea dei Borgognoni**,  Santa Maria in Via***,  Mark-Aurel-Column****,  Santa Maria in Aquiro**,  Sant Eustachio in Campo Marzi**,  Piazza Navona*****,  San'tAgnese in Agone****,  Santa Maria dell'Anima****,  Santa Maria della Pace***,  San Bartolomeo all'Isola***,  Santa Cecilia in Trastevere****,  Santa Maria in Trastevere*****,  Santa Maria della Scala***,  Palazzo Corsini***,  Palazzo Spada****,  Palazzo Mattei****,  Stadio di Domitiano*,  Santa Maria in Aracoeli****,  Santi Luca \& Martino***,  Joseph dei Falegnami*,  Forum Romanum****,  Santi Cosma \& Damiano***,  Santi Quattro Coronati***,  San Clemente****,  Colosseum*****,  Baths of Trajan***\\

September 10: Thunderstorms in Rome:\\
Summer 2017 was a very dry one, even for Rome. And then on September 10 came along: continuous Thunderstorms and heavy rain for hours. So we put on our biggest shoes, armed with Umbrellas and off we go: once we got out of the Metro by the Colosseum we realised that was not enough. Rome's canalisation isn't suited for that much of rain. Water was standing on the roads already, walking up to Domus Aurea little rivers were flowing down the roads. But we got there and were soon armed with hard hats and ready to go. Domus Aurea are the remains of Nero's famous Golden House. Only a small part of the former palace still exists, and only because the Baths of Trajan were built on top of it. The remains suffer a lot due to water damaging the structure after heavy rain falls. Thus a large restoration campaign is going on to save the remains for future generations. On weekends the palace ruins can be visited, though only the parts were construction has been finished. Even centuries later you can feel the splendour and the lavish life of the Roman emperors. At one point Augmented reality allows you to explore the gardens of the palace. Once we reached the banqueting hall, lights went out and more and more rain water started to flow by. Thus we went back to the entrance, and by now the roads of Rome were flooded, our feet for soaking wet. Buses stopped running, the metro had been closed down. Colosseum and the Forum Romanum were closed for the day as well (seem thunderstorms are something Rome is not prepared for. I decided that whatever I would do, i would get soaking wet instantly anyway, Riju and Amin decided to go back to the hotel and rather wait out until the rain might slow down. I walked though the water on the streets down to Palazzo Doria Pamphilj, which also had been closed until things cleared up. Thus I started to walk up to our hotel, but then I found out that the Palazzo Barberini was still open. Palazzo Barberini is part of the Italian National gallery, the highlight is though the gigantic ceiling fresco of the large saloon, which needed along around six years to be finished. By now the rain calmed down, so I passed by our hotel to change (a good feeling to get out of the wet socks, and of to further adventures, notably Palazzo Altemps, which houses part of the National Roman Museum with a vast collection of ancient sculptures. The Museum of the City of Rome is housed in Palazzo Braschi by Piazza Navona. Although the palace was built by the 19th century only, it has already seen quite a bit of history, it was used as ministry of interior, before Mussolini used it as headquarter of his party. By now the Palazzo Doria Pamphilj had opened up again. The palazzo is still used as residence by the family, but their vast collection as well as the state rooms of the palace can be visited. Since it is a private gallery, the paintings are put up in a less sterile way than in most museums. I also loved the tour of the state rooms. By now photography in the palace is allowed. Since the ticket of Palazzo Altemps covers the entrance for all buildings of the National Roman Museum, I went to the Crypta Balbi, which exhibits coins, sculptures and frescoes and paintings of early medieval times. And last but not least I visited the Capuchin Crypt, which is made up of six chapels, decorated with the bones of diseased monks (no Photos).\\

Rome: Domus Aurea*****, Palazzo Barberini****, Palazzo Altemps****, Palazzo Braschi***, Palazzo Doria Pamphilj*****, Santa Maria in Via Lata**, Crypta Balbi***, Capuchin Crypt****\\

September 11: Flight Home\\
Time to say Goodbye. Getting up around 4 am to be on the first flight back home. Lots of clouds along our way, until we reach the alps, where things clear up: Mountains and glaciers over glaciers, passing Mont Blanc another time and back home. And Rome is always mind blowing, it might be even too overwhelming. Since traffic is pretty chaotic, buses slow and metro lines rather at the edge of the old city, most sights have to be done by foot. Expect people to be almost around everywhere. In case you should not appreciate doing many items on one day, then restrict yourself to just a handful instead.


\section{September 15--September 18: London}
\label{London2017}

Why this weekend:\\
Many large cities have hidden gems, buildings and places, which are very beautiful, but more or less inaccessible to ordinary folks such as myself. Think of Royal Palaces which are still in use during state visits, or ministries, presidential palaces, embassies, etc. Many of those places do open their doors on special occasions, particularly during heritage days, which nowadays happen in almost all European countries one two or at least on one weekend day. Particularly amazing is the Journees du Patrimoine in Paris, but London has a similar weekend, the Open House weekend. Not anywhere close to be as open and accessible to many members of the public than Paris, it still gives you a decent chance to see places. While in Paris tickets need to be purchased in the most popular places, most houses you just can walk up to and enter. London is far more restrictive. While thousands can enter the Palais d'Elysee for example, Downing Street 10 allows an abysmal number lower than 100 inside and has less than a handful of tours foreseen. Still I made it to nice places too.\\

Co-travellers: Riju:\\
Just like me Riju has visited London previously during his exchange semester in Cambridge, but he missed out on places such as London Tower, or the Buckingham Palace which I am always eager to see again, plus our special houses which we booked. Unfortunately for some places we were too late and tickets had been gone.\\

September 15: uneventful flight to London\\
Nothing special happened this time before the flight, unlike in 2014. The transfer was pretty rapid too, the hotel of our choice this time around had no windows as we opted for the basement, but it had a pretty good ventilation, so we were happy with our choice.\\

September 16: London\\
We started the day with our guided tour of the Houses of Parliament, by now I know what to expect, and yes I still consider this the most beautiful parliament building I've been too (Palais du Luxembourg is close second). The start of the tour in the medieval gothic Westminster Hall with its centuries old wooden beams is always a highlight. Unfortunately only there and in the St Stephan's gallery photography is allowed, but the neogothic hallways, and the wood carvings in the House of Lord and the paintings, all is just outstanding. Another cute detail is the stained glass window in Westminster Hall, which changes colour during the day. Westminster Abbey is always a highlight with its plenty of tombs, the amazing choir stalls, the richly decorated chapel of Henry VII, the majestic nave, downside is the still ongoing ban on photography, which had been introduced in early 2000s.\\
After historic Westminster we walked over to the Foreign \& Commenwealth Office, a building from the Victorian Era with large halls, stucco ceilings, wide courtyard, more on the classical style as typical in the UK. Then we visited Marlborough House, another noble house only open during this weekend, the London residence of the Churchill family, particularly the main hall was a tad reminiscent in style of Blenheim Palace, which is the family's actual stately home in Oxfordshire. Photography was banned here too. Then we visited the Royal Society on our way from St James Palace to Trafalgar Square, also in classical style. Then we wanted to have lunch, found even an Indian restaurant, but they only offered a very expensive multi course weekend menu, which we didn't really have time for. At this point Riju went off to the Tower of London, while I continued with other stuff.  \\
I went to St Martin-in-the-Fields on Trafalgar Square considered a prime example of classicist architecture, then toured King's College with its chapel and the library, and then spent some time in the National Gallery. Afterwards I met with Riju in Soho, where we had a short dinner before witnessing a performance of Les Miserables in Queen's Theatre. As a fan of musicals Riju and I had a good time.\\

London: Houses of Parliament (Palace of Westminster)*****, Westminster Abbey*****, Foreign \& Commenwealth Office*****, Marlborough House****, The Royal Society***, St Martin-in-the-Fields****, King's College****, National Gallery*****, Queen's Theatre: Les Miserables*****\\

September 17: London\\
Buckingham Palace is one of the most beautiful Royal Residences I have seen with magical rooms over and over again. I particularly like the Blue and the White Rooms. Unfortunately it also comes with a ban on photography in all state halls (you are free to take photographs in the vast gardens).  Once we finished our tour of Buckingham, we had to find out that due to the Changing of the Guards Parade some part of the roads had been blocked off from crossing, other roads and sidewalks were just very full with waiting spectators. Thus getting to Lancaster House took a tad longer than the predicted 15 minutes, we made it in time for our tour anyway. \\
Lancaster House is one of the most beautiful and richly decorated houses in London, which is nowadays used for meetings of Commenwealth councils. In addition movies often use the house as stand-in for Buckingham Palace which is off-limits for almost all productions. The staircase and the large state rooms are some of the best I've seen in the United Kingdom, so take your chance next time it might be possible. After Lancaster House we joined another pre-booked tour, of Canada House just by Trafalgar Square. Only some of the rooms are still conserved in their original state such as the throne room which was used for the King's visit. Most of the rooms have pretty modern decorations with particularly interesting carpets. From the roof terrace one has a really good view of Trafalgar Square too. Since we were so close, we saw the National Gallery again, before walking to the Tate Modern. We were told though that the Millenium Bridge was closed due to an incident, thus we had to detour but made it to Tate Modern anyway. I like the variety of modern and contemporary art pieces, both in the usual exhibit and also the special parts (not to forget the balcony view over the City of London). \\
After grabbing dinner close to London Bridge we got up to The View from the Shard, ironic to see that it still was advertised as the tallest skyscraper in the European Union. The view was really nice, particularly in twilight and blue hour, with a sunset by Westminster and clear skies over by Canary Wharf, and St Paul's sticking out from the pretty uniform grid of houses. In order to celebrate our successful London weekend, Riju and I had sparkling vine while we were at it.\\

London: Buckingham Palace*****, Lancaster House*****, Canada House****, National Gallery****, Tate Modern*****, The View from the Shard*****\\

September 18: flight back to Geneva\\
We had to get up before 4 am to catch a plane at Gatwick airport shortly after 6 am. We were up early enough that we even made it to a train earlier than planned. And thank god we did, as the train we wanted to take was announced as cancelled, thus we either would have needed to spent a lot of more money to transfer or would have missed the plane most probably. Anyway taking THIS train we made it to Gatwick comfortably, had some early breakfast, and arrived at work satisfied but tired.

\section{September 20: Rolling Stones Concert}
\label{2017RollingStones}

How to mess and up still succeed. My dad is a big Rolling Stones fan, having had many of their record, knowing the lyrics of most of their classic stuff and having been on their concerts previously. When we watched one of their concerts on TV I asked him if he would be potentially interested in seeing them again, should they ever return to the area. In 2017 the Stones announced their tour would stop in Zurich. Thus I made sure to get tickets, I checked several options, decided for and against some - and got two tickets or so I thought. Turns out I hadn't deleted one of the early access tickets, so I ended up having four tickets. At the same time I took part in a competition of a local radio station for the very same concert. Turns out my office mate's sister is a giant Rolling Stones fan as well, and he likes them either. Thus our office would go on a field trip with each one family member to see a Rolling Stones concert. \\
Back to the competition, turns out I won, now I got two free tickets plus two tickets I had purchased already. Since my aunt had already sadly told me how she tried to get tickets for the Stones concert in Munich but looked into it too late and it turned out to be sold out by that point, I knew who I would offer one of the free tickets too, and I thought my mum for sure wouldn't mind spending the concert as well keeping my aunt company (aka the two ladies got in for free, since why should I charge them for something I won for free).\\

I took the day off and we first met for dinner (Vietnamese food), and then I got the tickets from the competition and got my mum and aunt to make sure to be pretty much to the front of the fan queue (3 1/2 hours before the venue opened up). My dad and I got to the early entry gate, enjoyed some drinks there, and we clearly were middle front in our category, so I can firmly stand I actually saw Mick Jagger perform live (and not only on the screen). My mum and my aunt were front row for their category too, just more to the side, but also excellent view for free tickets.\\
Clearly the Rolling Stones (but also the pre-band) showed off their skill (or routine for that matter). It was a great experience, they performed 2 songs from their latest album, but naturally the focus of their concerts is on greatest hits and their legacy. Then I met up with my office mate and his sister for our 3 hour ride back to Geneva, clearly guessing and singing to songs and other matters too on the road.\\

Z\"urich at Letzigrundstadion: Rolling Stones Concert*****

\section{September 30--September 31: Graz}
\label{Graz2017}

Having been in Austria a couple of times, including Vienna, Salzburg, Innsbruck and Bregenz, it is time to visit another State capital, this time Graz, the capital of Steiermark:\\

September 30: Graz:\\
Graz: City Church****, Landhaus****, Dom****, Katharinenkirche (Mausoleum of Emperor Ferdinand II)*****, Mariahelferkirche****, Schloss Eggenberg*****, Basilica Mariatrost****, Murinsel****\\

September 31: Graz\\
Graz: Palais Herberstein***, Franziskanerkirche***, Dreifaltigkeitskirche***, Mariahelferkirche****, Schloss****, Barmherzigenkirche****, Kunsthaus****

\section{October 6--October 9: Estonia}
\label{Estonia2017}

How we ended up here: Rachel decided that she wanted to celebrate her birthday somewhere else, preferably in one Scandinavian country. She suggested Finland, but I convinced here that we could go to Estonia instead, where drinks and food would be cheaper, with large Vegetarian options and also a good selection of sorts of beers. And we can do a trip across the baltic sea over to Finland to see Helsinki too.\\

October 6: first evening in Tallin\\
Tallinn: City Hall*\\

October 7: Helsinki\\
Helsinki: Suomenlinna*****, Uspenski Cathedral****, Dom****\\

October 8: Tallin\\
Olaf's Church*** (with Tower****), City Wall****, Kadriorg Palace****, Alexander Nevsky Cathedral***, Dom****, Nikolai Church****\\

October 9: flight home:\\
not impressed by that Ottakringer

\section{October 15: Zermatt}
\label{Zermatt2017}

Zermatt is a hiking paradise and also a fantastic 

Zermatt: Gornergrat***

\section{November 4: St Gallen, Reichenau \& Konstanz}
\label{StGallen2017}

Why this trip:\\
Reyer had never been to Germany, so I wondered what could be done close-by. Both Konstanz and Freiburg im Breisgau came to my mind. St Gallen is also a very nice place to see, also on the Eastern part of Switzerland for once, and then maybe going to the monastery island of Reichenau. But then Reyer had to stay up the night before and unfortunately didn't make it to this trip. Since I had bought the day ticket though, I just did this trip on my own.\\

After Reyer told me he couldn't join, I just sat on the train and napped for almost all of the trip. The cathedral of St Gallen used to be the church of the large abbey. It is one of the largest late baroque abbeys in central europe, built by Peter Thumb. Everything is vastly decorated with stuccos, statues, altars and large veiling frescoes, particularly the one of the Rotunda. The choir stalls are a masterpiece of wood-carving, some parts are gilded in gold. The high altar and most of the choir decoration is already heavily influenced by the classicist era. The library of the former abbey is the last large monastery library built in northern Europe. An amazing masterpiece full of rich decoration, it also houses precious manuscripts, a mummy, as well as a famous globe. No photography is permitted unfortunately. After having a little snack of a St Galler Bratwurst I took the train to Konstanz, from there I continued with another train and bus over to the monastery island of Reichenau. \\
Just after crossing the dam you reach the first church - St Georg. The whole walls are decorated with frescoes from the 11th century. The decoration of the choir was lost unfortunately, the entrance hall and the Westwerk still contain some faded remaining parts of the original decoration. In summer the tourist flow is heavily regulated, only allowing two tours inside the church. In winter season far less people visit the island and you are free to visit when you feel like it. The largest church belongs to the main monastery. A big romanesque church with the typical sparse decoration, besides the choir, which was remodelled in gothic times, with a nice altarpiece and a few frescoes. Unfortunately you only get close to those if you take a tour of the church. Walking for another 20 minutes you reach the last church St Peter \& Paul. The apsis is decorated with an old romanesque mural, while the nave of the church has been vastly redecorated in Baroque times. Nice to see how people believed they need to go with time, and what contrast that might create. Then i rushed to the bus, when i started running and gentleman asked me if he should just take me along for a bit, so I for sure would make that bus, and I accepted that offer being really thankful.\\
 Since I made it now to an earlier bus, I had still a bit of time to go to the harbour front with the quite odd Imperia statue and the council hall, and then I went on to see the Minster of Constance, the former cathedral of the once largest diocese in late medieval times. Even the famous council of constance took place in the minster where the catholic church tried to solve the schism which was created after three popes had been elected at the same time (solved by forcing all of them to step down and electing a new pope). The Minster was built in Romanesque style, but the facades had been remodelled during gothic times. The inner decoration is mainly from classicist times, but murals from former times survived in a few of the side chapels, along the choir is a chapel which houses Constance's version of the tomb of Jesus, the so-called medieval Mauritius Rotunda, decorated with several statues.\\

St Gallen: Cathedral*****, Library*****\\
Reichenau: St Georg*****, Monastery****, St Peter \& Paul****\\
Konstanz: M\"unster****

\section{November 10--November 13: Marrakesh}
\label{Marrakesh2017}

Why did we get here:\\
Northern African countries can be reached pretty easily from Geneva with direct connections to Tunisia, Morocco and Egypt. Cameron, Riju, Tony, Rachel and myself thought it would be great to visit it around November. Flights were affordable and typically the weather is still warm and sunny unlike things in central Europe. Same can be clearly said about the other two countries, but Marrakesh seemed more of a relaxed option than visiting the gigantic cultural heritage of Cairo for example, with a good mixture of food and a bit of culture instead though. Another advantage is our existing knowledge of French, thus most of us felt more confident to go to Morocco on our own, while in Cairo French would definitely not much of help. And flights were a lot cheaper than getting to Tunis, so Morocco it was, and from my previous trip to Morocco I preferred Marrakesh to Casablanca which was the other option to get to in Morocco. Now we are there for three full days, thus I checked our options for day two for a getaway. Three possible exiting options were available: A train or bus ride to Casablanca with a visit of the giant mosque, or a hike in one of the Canyons (not offered by many tour providers in November anymore), or a day trip to the Fortress of Ait Ben Haddou. Contacting one of the highly rated day tour (also multi-day) tour providers it became clear they could provide us a ride and guide for Ait Ben Haddou, so we opted for that one.\\

Co-travellers:\\
Riju: clearly my most loyal co-traveller loves to step food on Africa as well.\\
Cameron: another UCLA grad student from the US, he strongly voted to get to Marrakesh over the other options so clearly interest in what Morocco is up to.\\
Tony: a Ohio State grad student from the US, after about a year of being at CERN he dares to go on a trip with me, after having seen the most highly regarded places in Europe, ready to see a more unusual place like Morocco\\
Rachel: well here we go: if everything would have gone according to plan after giving a talk at a workshop in Hamburg the day before, Rachel would take Swiss with a transfer at Zurich to be on our early morning flight. Unfortunately some crazy German man decided it would be fun to run around the runway of Hamburg airport, leading to substantial delay of all flights. Once Rachel arrived in Zurich the flight to Geneva had already departed. Getting from Zurich to Geneva on that night by train or bus didn't seem feasible, and the early morning flight from Zurich to Geneva would arrive after our flight would leave. Thus Rachel was unfortunately not able to join us.\\

November 10: Marrakesh\\
We had booked a hotel quitee close to Djemaa el Fna, tha large market square of Marrakesh in the Medina. Since we arrived so early in Marrakesh (after all in another time zone than central europe), we left our luggage by the reception and had breakfast on the roof terrace. Different types of bread, cheese, humus, their version of yoghurt as desert as well as coffee and tea - what more can you ask for. Then we started to explore the town: walking through a market road we reached the Bahia Palace. Not having seen it the first time I was in Marrakesh I was pleasantly surprised to see beautiful courtyards with gardens and multiple nice rooms with gilded carved ceilings and beautiful tiles on the walls. And then it was time for a nice lunch, choosing one of the other squares as our lunch destination. Unfortunately for the waiter we still can add numbers in our heads, so he clearly tried to get a bit more out of us clearly foreign tourists. The El Badi palace is now in ruins, but impressive ruins giving you an idea of the amazing building it once had been. Just a couple of blocks away one can find the Tombs of the Saadian family, housed in a large courtyard with two halls, the hall of twelve columns is particularly impressive with fancy capitals, precious ceilings, a room as you would imagine it too look like from the 1001 night stories. And then it was time for a long dinner with Tajine among other things, topping it off with a round of drinks by a bar overlooking the square, which is still very busy and lively by night as well.\\

Marrakesh: Bahia Palace*****, El Badi Palace****, Saadian Tombs*****, Djemaa el Fna****\\

November 11: Telouet \& Ait Ben Haddou\\
Telouet: Kasbah*****\\
Ait Ben Haddou*****\\
Tizi N'Tichka Pass****\\

November 12: Marrakesh\\
Marrakesh: Medersa*****, Mnebhi Palace (Marrakesh Museum)****, Almoravidic Kouba****, Jardin Secret****, Djemaa el Fna****

November 13: flight home:\\
giant gardens, Casablanca with large mosque, hi Cordoba

\section{November 18: Fonteney, Arc-et-Senan \& Besancon}
\label{Nov18}

Another trip with Andrew: Unlike last time it only took a bit more than a year for Andrew to return to CERN, this time I looked up what could be only done by car, but which might still be interesting for him and me to see. I realised that by smartly combining three destination I could manage to show Andrew three UNESCO world heritage sites in close-by France. OK close might be exaggerated a bit, since the first item was over three hours out. But still Andrew was all game to see all three places.\\

I took one really early pre 6 am tram up to CERN to meet up with Andrew, having a short snack at CERN's R1 just after it would open. Since we didn't have a Vignette and didn't want to get one unlike in 2016, this time we decided to take a detour through Bellegarde. We arrived by the old Abbey of Fontenay just by the time it open. It is the one of the oldest Cistercian abbeys in Europe, founded in 1118. Almost the whole complex is still intact in its original Romaneque style. Particularly the abbey church with its open ground floor and the early morning sun shining through looked as something you would built up on a green screen for shows such as Games of Thrones. Definitely an impressive old medieval site.\\
 Then we drove to our next stop of Arc-et-Senan. When we arrived the site was on its lunch break and so we had some local Flammkuche and cake before we saw the exhibits of the former Royal saltworks of Arc-et-Senan where salt was exctracted from stones in a delicate procedure. The site itself is an old planned Baroque style ideal town. I thought it was interesting, but I doubt many would go out of their way to see it and deem it worth the long way.\\
 In Besancon the giant citadel towers over the old town with superb views of the Doubs river. It was constructed by the leading military architect of his time, Sebastien Le Preste de Vauban, back in the 17th century after Besancon had been given to France. The citadel is surrounded by three successive bastions with 15-20 metre high walls, as thick as 6 metres. All in all the citadel occupies 11 hectares on the mountain top. Nowadays the Royal Fort contains a museum, but also a zoo, hosting even a Siberian Tiger, or the largest Insectarium in France. Unfortunately it was too late to see all of the zoo, so we largely restricted ourselves to the Aquarium. On our way back to the car we stopped by the medieval cathedral, which had been largely refurbished in Baroque Style though, and then we drove all the way back to Geneva.\\

Marmagne: Fonteney Abbey*****\\
Arc-et-Senan: Saline Royale***\\
Besancon: Citadel****, Cathedral***

\section{December 9: Lyon - Fete de Lumiere}
\label{LumiereLyon2017}

I love to watch light art, but I had never visited the famous Fete de Lumiere in Lyon. This year a friends' group from CERN decided to visit the festival, and I was eager to join. We started out in a market square, where we share oysters, snails, and mussels and fries as dinner, before taking the metro to old town. The light installations were scattered on multiple spots, with a focus on the large squares. Special sculptures had been installed, another category was videos and frames projected on the facades of houses, or just using different patterns on spotlights to illuminate churches and basilicas in a different manner than typically. It was also nice to enjoy Christmas sweets or mulled vine. The city was absolutely packed, even I got lost once when I stopped too long to take pictures. And at the end of the festival we had to queue since obviously thousands of people wanted to take the metro to get to the suburbs from downtown again. And then Riju drove us for another two hours through the wintery night scenery of the mountains.\\

Lyon: Fete de Lumiere*****\\

\section{China: December 29, 2017-January 21, 2018}
\label{2018:China}

Oh yeah that's. new one, I never had a trip crossing years. And that's how we got there. I was asked if I want to go either to Hong Kong or Chile, first i though - well Chile, would be my first time in the Southern hemisphere, so definitely very very exciting, but China (OK Hong Kong, still a different system, also visa wise, but close enough) is so close, so why not -- after all a long standing dream to see it once in my life. After all, I found out someone else wanted to give the talk in Chile, so Hong Kong it is. I checked what I could do without getting a visa, so I asked my friend Siyi, what she would suggest to do close to Hong Kong, but then she told me, she would visit her parents around that time, and considering i would only take a couple additional days off, I could make it a 3 week trip to China. Obviously China would need some preparation to visit, but if Siyi volunteers her help, clearly that is an offer to take, so just a couple of days later I booked my flight and now after applying for a visa with help from Siyi to plan my itinerary, by late October it is clear I will spend new year in China (my parents and my younger brothr were a bit sad though, after all my first new year all on my own). \\

Decr 29-30: (Fr/Sa) Flight and first day in Shanghai\\
Now on Dec 29 I get to Dock E at Zurich airport, the A340 has no functioning on-plane entertainment (I wanted to sleep anyway). Still suffering from the ending days of a cold, I finally get to China on Dec 30, find out my change is not enough for getting a ticket on the Shanghai metro - getting a coke to get the sufficient change - and ready to go and ending up in the city centre. Surprisingly enough gmail still works - so I can tell my parents i arrived safely. Then in this rainy very smoggy day I check out the Bund (unfortunately not seeing that far, but OK) - get some chinese food - and just hope the jetlag wouldn't catch me getting to be at 9 pm local time.\\

Shanghai: The Bund*****, People's Square****\\

Dec 31: celebrating new year in China in Shanghai\\
Today is the day when I am supposed to meet Siyi and her friends. Hot-pot is supposed to be really nice, and close to the high speed train station outside of the city centre is where you are supposed to get it. I note the address, not knowing that the shopping centre in question has four buildings. Google maps is not a thing and alibaba works in Mandarin only, so clearly I get lost. Calling Siyi she comes to rescue me, but still it takes over 25 mins to get hold of me. Anyways hot pot is so amazing, her friends are really cool too, she suggests to go to old town later on. Amazing really amazing, rather though old town in a US sense (aka buildings and surroundings are less than 200 years old, but still cute, my first rather traditional Chinese buildings). Siyi suggest to go to a fancy place for new years, we arrive there but seems we might have needed to reserve, and the dress-code also exceeded expectations, so we decide rather to walk to the inner city along the river, having cocktails at our hotel and then later by a cafe by Nanjing road. Overall a great new year's eve, but no fireworks (Chinese new year's falls on a different day than western new year) --- and continuing the trip in 2018.\\

Shanghai: The Bund*****, Shanghai Old City****\\

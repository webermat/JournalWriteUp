\chapter{Year 1996}
\label{1996}

For Christmas 1995 I got my first camera, an analogue one, which was then with me on all big and little trips until about 2006. Photos are an excellent way to keep memories alive thus the following chapters will continue to be more and more elaborate (naturally being older helps quite a bit too).

\section{Beatenberg}
\label{1996:Beatenberg}

Every couple of years the orchestra of my village does a little trip. The one of this year was an outing to the Swiss alps by Beatenberg. Starting with a boat ride over lake Thun we arrived about lunch time at the resort. My dad, my sister, my aunt and other people decided to do a bit of hiking. While I enjoyed it my aunt went out of steam shortly before the midstation, thus we decided to call it a day and get down with the chair lift again. Meeting my uncles and cousins we had a couple of drinks and cake in a closeby cafe. On the second day we went to the museum village of Ballenberg. There old houses from everywhere in Switzerland (like old schools, old mills, old bakeries etc) have been transferred and rebuilt to give you a flavour of old traditional Swiss life. \\\

Niederhorn *****, Ballenberg****

\section{August: London}
\label{1996:London}

%Westminster Cathedral & Abbey
%Hampton Court
%Tower
%Westminster Abbey
%Greenwich
%St Paul's
%British Museum

This was the first flight in many many years taking a BAE flight from Zurich to London Stanstead. Plan was to take the Stanstead express to Liverpool station and then distribute ourselves over taxis. But since we were one person too much I had to do my first underground ride on my own not that this was anything scary for me. Since I didn't keep track of dates back then and analogue photos don't include useful EXIF data in this respect I don't know how to distribute certain events to certain dates, the order is correct though. Since the Hotel President is by Russel Square the first place to visit was the British Museum, starting out with the East Asian section. On the second day we visited Westminster Abbey. I was very surprised how white the church was, all pictures I had seen of it showed it in black, but it seemed just on that year they finally cleaned all the stones from the dirt of many decades. Notable difference to nowadays, the front section could be visited free of charge. Then we went to the Westminster Cathedral. Back then I was rather impressed by this church (that changed by now), also with the great fact that you could take the elevator.\\
 Early morning next day we queued up at Madame Tussauds. Even in the mid 90s the queues were really large (keep in mind online tickets hadn't been a thing back then). While we queued another family came up to us, asking if we present to be a group (aka 10 people and more) and just stand in the much shorter tour group queue. And that's exactly what we did and less than 5 mins later we were in. I still don't see why people are so amazed by wax figures, but at least I did it once. We also saw Trafalgar Square with St Martin in the fields and the National Gallery, which didn't leave any impression on me back then. Next morning I had to realise how huge the crowd was to witness the Changing Guards by Buckingham Palace, not that I got why that was supposed to be amazing either. Since we had found out how expensive it was to buy tickets for Musicals back in Germany my parents decided to try to get cheap last minute tickets in London. Our preferred choice of Les Miserables was not available, so we decided to go to Starlight Express instead. And seems we got front row in the upper section. Quite an experience to see people passing you on high speed while on roller skates. \\
Since I was already back then fascinated by palaces I took the whole family over to Hampton Court Palace. I was clearly impressed by both the gothic and the baroque section. My siblings enjoyed at least the kitchen section and the park. In the Tower the crown jewels were amazing, but I had expected more of the rest. Then we went to the London Monument (my sister even went up), and then it was shopping time in Harrods and Selfridges, not my kind of thing necessarily, but my sister got a dress out and I got myself new running shoes which I used for almost 10 years until they broke down. Then my dad and I went to the back side of Westminster Abbey with all the royal tombs, the only part you had to pay for in 96. The second notable difference was that back in 1996 on Wednesday afternoon for 2 GBP photography was allowed in Westminster Abbey. Thus I do have in fact a couple of pictures of the inside, unfortunately 1996 analogue camera quality but better than nothing. On Thursday we took the regional train to Greenwich. There we visited the National Maritime Museum, the kinda boring Queen's House and then walked up to the old Observatory. Standing on the zero meridian was nothing special. I was more interested in the Thames Barrier (and the shopping carts in the Thames close to it). During the afternoons we spend quite a bit in the British Museum, I enjoyed particularly the Egyptian and Assyrian artefacts. It was one of the first years Buckingham Palace opened its doors to the public in Summer. Photography of the interior is even up to now not allowed. but back then you were allowed to have a view of the courtyard, which is not possible anymore. Still I was impressed. The last day we spent in St Paul's where photography was in fact allowed back then. Once again that policy has changed and by now it is not allowed anymore (similarly to Westminster Abbey). First they claimed it on the bullshit excuse of protecting the art, then they went on to say it would disturb the guards, at the moment the official excuse of a reason is that it disturbs the spiritual experience (of course unlike paying the almost 20 GBP entrance fee). Anyways the church itself is great (still second to Westminster Abbey), but the view of the City is the best.\\
Also unrelated fun fact: back then in 1996 a song blew up considerably in the UK, ``Wannabe'' by the Spice Girls which was just about EVERYWHERE, already then it became clear this song and the band would most probably conquer the world.\\

Hampton Court Palace*****, Tower of London****, Greenwich***, Buckingham Palace*****, Star Light Express*****, Westminster Abbey*****, Westminster Cathedral*****, St Paul's Cathedral*****, British Museum*****, Madame Tussauds**, National Gallery***

\section{Hartmannswillerkopf}
\label{1996:Hartmannswillerkopf}

In World War I, the Alsace region had been a major point of operations. At Hartmannswillerkopf is one of the largest soldier cemetery of this particular battle, together with a brutalist church and crypt. The remains of the ditches and fortifications are slowly claimed back by the forests, but it was quite solemn and touching if you think back then what had happened and how many young men lost their lives in a rather pointless war, which might have been avoidable if politics back then would have known a bit more of diplomacy.\\

Hartmannwillerkopf****

\section{K\"ussaburg}
\label{1996:Kuessaburg}

Quite close to my home village is the ruined castle of K\"ussaberg. Overlooking an old road which had been used already in Roman times for trading, the castle dominates the nearby mountains. Destroyed and given up centuries ago, almost everything of the inner decoration is gone, but some of the enclosure is still there, as well as some ruined walls of the palace or the fortified outer towers. One of the former towers can be climbed on, as well as the gate fortifications.\\

K\"ussaburg****

\section{Hohentwiel}
\label{1996:Hohentwiel}

Hohentwiel is an extinct volcano (actually the core of the Volcano is the only part which remains after all those years of erosion) close to lake constance by the town of Singen. Originally once one of the largest fortresses in Europe, it was left in ruins by French soldiers. There is a lower and an upper Fort, the ruins still give you an impression how mighty it originally was. I did enjoy the ruins of the church and the underground layers the most. For sure a good trip for a summer family outing. The ramps between the lower and the upper Fort are a tad steep, maybe also slippery in rain.\\

Hohentwiel*****

\section{Unteruhldingen}
\label{1996:Unteruhldingen}

Along the shores of Lake Constance many remains of old pile dwellings can be found. One of the largest archeological sites is by Unteruhldingen, where the artefacts have been placed in the local museum. To give visitors an idea about the original state and life in pile dwellings a whole village has been reconstructed. I always enjoy to walk along, look into the houses, where locals also set up re-enaction of life back then based on the latest historical evidences.\\

Pile Dwellings****


\section{October: Stromberg}
\label{1996:Stromberg}

For the fall holidays we stayed in Germany, in Stromberg. On our trip to the holiday resort we stopped by the romanesque cathedral of Speyer. Partially destroyed by the French in the 17th century the eastern part is still mainly the original from the 11th century, including the imperial tombs in the crypt of the Salian dynasty. This is one of the three imperial cathedral of the Rhineland which we visited on this trip. Arriving, we did a small hike along a small riverbed ending up by the castle of Stromburg, which had been partially converted into a fancy restaurant. On the second day we went to Bingen, a nice little town with an old castle in the middle of the city. In the Rhine is another little tower and this one marks the start of the Rhine Gorge. In Heidelberg is the old Castle, which was destroyed in the nine year war by the French. The castle was a mixture of Gothic and mainly Renaissance buildings. Most of the towers have been split in two by explosions. My sister tried her best to photobomb my pictures too. The castle is one of the most famous romanticised castles in Germany, only a very little part has been rebuilt. The castle is also home of one of the largest vine tuns with a capacity of 219 thousand litres. On the third day we visited the city of Worms, home to the second imperial cathedral. Unlike in Speyer the cathedral had not seen such turbulent times, the baroque high alter and choir decoration survived all these years. The interior is though a lot darker and feels more like being in a castle. Afterwards we saw the old synagogue and the adjacent rooms depicting the history of the jewish community in Worms. The last of the three imperial cathedrals is the one in Mainz, which we saw the day after. In Mainz Johannes Gutenberg invented the print press, which still can be seen in a specialised museum along with one of the printed bibles. in a separate rooms we kids were allowed to try to print out own text.\\ 

Speyer: Dom*****\\
Stromberg: Stromburg*\\
Bingen: Old Town \& M\"auseturm**\\
Heidelberg: Schloss*****\\
Worms: Dom*****, Synagoge***\\
Mainz: Dom****, Gutenbergmuseum***
\chapter{Year 2021}
\label{2021}

\section{January 10: Bad S\"ackingen \& Laufenburg (Baden)}
\label{2021:BadSaeckingen}

After December was a stay at home months, I finally got out with my dad, staying close to home though, first visiting the town of Laufenburg. Their advertisement slogan is one town, two countries. Historically the town of Laufenburg has a centuries old history together before Napoleon decided to redraw borders separating the town in the middle. The larger part of old town with a ruined castle is in Switzerland nowadays. Usually one can just walk across both parts of the city, but nowadays with the government in my German state deciding walking over the border is not allowed without quarantining we didn't do that. Instead we continued our short afternoon outing in S\"ackingen. Home to a nice Rokoko church which used to belong to a monastery, the Fridolinsm\"unster. Usually on a nice beautiful day like this many guests, also from the Swiss side might have enjoyed a walk or cafe, but right now that isn't an option either, we did walk a bit on the Wooden Bridge, in fact the longest wood covered bridge in the Europe beating out Kapellbr\"ucke in Luzerrn by a few metres. Clearly not allowed to cross over we turned around midway as well.\\

Bad S\"ackingen: Wooden Bridge****, Fridolinsm\"unster****\\
Laufenburg (Baden): Old Town***\\

\section{February 4: Aach \& Reichenau}
\label{2021:AachReichenau}

It had been raining heavily and snow was melting too, rivers were really might on that day. Even the little creek of Aitrach looked more like a river and the Danube had crossed over its dams in some spots. The Aachtopf is Germany's largest karst spring, fed by water originating from the Danube sinkhole where the Danube itself disappears for about two thirds of the year (clearly not the case during snow melt). The spring is impressive forming a little lake just by a hill side. Since we were in that area we continued to Reichenau in order to see the first UNESCO world heritage site of 2021 (if you follow the past years you realise it appears pretty often on that list). This time we started at St Peter and Paul, continued with the Monastery, ending in the highlight of St Georg. I always enjoy seeing all those old churches and their early medieval frescoes. Unfortunately this time around we were not able to stop in any cafe to get some coffee or cake, clearly all of those had been closed too.\\

Aach: Aachtopf**\\
Reichenau: St Peter \& Paul****, Monastery****, St Georg*****\\

\section{February 6: Lake Constance Area}
\label{2021:LakeConstance}

Uhldingen-M\"uhlhofen: Pilgrimage Church Birnau****\\
Lindau: Minster***, St Peter***, Harbour with Lighthouse and Bavarian Lion****\\
Weingarten: Abbey****\\
Meersburg: Old Town****

\section{February 14: Haselbach Falls}
\label{2021:Haselbach}

Waldshut-Tiengen (Indlekofen): Haselbach Falls***

\section{February 19: Schl\"uchtsee}
\label{2021:Schluechtsee}

What do you do if you want to enjoy a nice afternoon in the region: what about going to one of the multiple little small mountain lakes. The Schl\"uchtsee is a very small reservoir, set up originally as ice lake for the close-by Rothaus Brewery. The lake is about 5 m deep, in mid February the lake was completely frozen, but pretty idyllic walking around it (the trail is not more than 2 km long). The lake is within a forest area, the Schl\"ucht river begins just a bit beyond it flows into the lake. Typically one can have beers or wine in the farm close to the lake or in summer just jump into the water having a short swim. It is a nice place for families with small kids, but not really anything special to be honest.\\

Grafenhausen: Schl\"uchtsee***

\section{February 21: Beuron and Upper Danube Canyon}
\label{2021:Beuron}

Beuron: Monastery****, Upper Danube Canyon*****\\
Sigmaringen: St Johann***

\section{February 23: Roggenbach Castles}
\label{2021:Roggenbach}

Bonndorf: Roggenbach Castles***

\section{February 26: Menzenschwand \& Todtnau}
\label{2021:Todtnau}

St Blasien (Menzenschwand): Albklamm \& Menzenschwand Falls****\\
Todtnau: Todtnau Waterfall*****

\section{February 27: Freiburg \& Black Forest}
\label{2021:Freiburg}

Breitnau: Ravenna Gorge*****\\\
Freiburg: City Gates**, Minster******\\
St Peter: Monastery****\\
St M\"argen: Monastery****\\\
Titisee-Neustadt: Titisee***\\
Lenzkirch: Windgf\"allweiher****\\
Schluchsee: Schluchsee***

\section{February 28: Rickenbach}
\label{2021:Rickenbach}

Originally the plan was to go to the ruined castle of Wieladingen first, and then continue via the Lehnbachfalls along the Murg river with its waterfalls up to the Strahlbrusch waterfall. It was possible to get to the cascades of the Lehnbach creek, but the bridge crossing the river at that point had been damaged by fallen trees a couple of days previously. Thus one had to get to the castle via a different route crossing the creek a bit below. The castle itself had been renovated a couple of years before, from the belfry one can see up to the Swiss Jura (the alps were not visible then), and up the Murg valley on the other side. Unfortunately the way along the Murg had been closed as well. Thus we drove up to another parking lot, and start our second hike from there. This hike led alongside an old abandoned quarry and passing the Seelbach creek down to the Strahlbrusch waterfall. The waterfall is about 12 m high, and in snow melt pretty strong as well. We crossed the Murg via an old bridge of an old postal route which also crossed two small tunnels. Since the trail was closed on this side, we just turned back to the car. All in all a nice afternoon out but for sure the Murg canyon itself could have been a nice hike too, but if it is closed due to snow or rock falls you rather don't risk it.\\

Rickenbach: Castle Ruin Wieladingen***, Strahlbrusch Waterfall****

\section{March 2: G\"orwihl}
\label{2021:Goerwihl}

G\"orwihl: Aybach Waterfall***, H\"ollbach Waterfalls****

\section{March 6: Constance \& Hegau}
\label{2021:Konstanz}

Constance: St Stephan***, Minster*****, Christ Church**, Holy Trinity Church**\\
Hilzingen: Hohenkr\"ahen****\\
M\"uhlhausen: M\"agdeberg****\\
Tengen: Hinterburg**, M\"uhlbach Gorge***

\section{March 9: Waldshut}
\label{2021:Waldshut0309}

In Waldshut the Wutach river finishes its flow by the Rhine river. Recently a lot of work had been going into making the mouth appear more natural and less canal like. Thus islands had been created with a second small arm of the Rhine as well as an area where the Wutach river can overflow in case of high water. Only a months ago snow melt was happening but already in March the second high water arm had retreated leaving only muddy but pretty dry territory.\\

Waldshut: Wutach Mouth**\\

\section{March 25: Triberg \& Villingen}
\label{2021:Triberg}

Triberg: Triberg Waterfalls*****, Schonach Waterfall***\\
Schonach: first world's largest cuckoo's clock**\\
Villingen: Johanniterkirche**, City Walls and City Gates***, Minster***

\section{March 28: Alb gorge}
\label{2021:Albtal}

Dachsberg: Alb with Teufelsk\"uche (Devil's Kitchen)****\\
G\"orwihl: Krai-Woog-Gumpen and Gletscherm\\"uhle***\\
Gutenburg: Teufelskessel waterfall***

\section{March 30: Blumberg, Mundelfingen \& Immendingen}
\label{2021:Blumberg}

Blumberg: Schleifenbach Waterfalls****\\
Mundelingen: Mundelfinger Waterfall \& Aubach Gorge**\\
Immendingen: Donau Sink Hole** (too much water at this time of the year), H\"owenegg with crater lake****\\

\section{April 11: St\"uhlingen \& Grimmelshofen}
\label{2021Stuehlingen}

I had been in St\"uhlingen multiple time on carnival parades, music festivals, or just as final destination of bike rides in my teenage years. It had been a while though since I saw old town with the city church, and the church of the Capuchin monastery. Then we continued our trip to the small village of Grimmelshofen, walking along the Wutach and going down to the Dampfkessel waterfall, not as mighty e.g. as the Lauffen waterfall in my home village, but 2-3 m are not nothing either. On the way back we drove past the Hohenlupfen castle and took a panoramic route to Bonndorf enjoying some views of the snow covered alps in the distance.\\

St\"uhlingen: old town***, panoramic road with view of the Alps***\\
Grimmelshofen: Dampfkessel waterfall***

\section{April 20: Hohentengen}
\label{2021Hohentengen}

Another close place to home is the village of Hohentengen, originally home to three castles. Out of those one has been completely destroyed, on that spot there is now a Swiss army bunker. The second castle of Weisswasserstelz is in ruins nowadays too, a few walls remain though, which give you an idea of the former glory. Rotwasserstelz Castle still guards the bridge over to the Swiss town of Kaiserstuhl. But in April 2021 borders are forbidden to be crossed.\\

Hohentengen: Rotwasserstelz Castle**, Ruin Weisswasserstelz**

\section{April 24: Kaiserstuhl \& Breisach}
\label{2021Kaiserstuhl}

Bickensohl: L\"ossholweg hike****\\
Breisach: oldtown with Stephansm\"unster****

\section{April 28: Moving to Vienna}
\label{moveVienna}

At some point also physicists leaving academia find jobs: I found my first job in Vienna. Since the pandemic was still going on (3rd wave now), it all was followed by a quarantine protocol. Since I received the green light for my flat by the very end of April (26 to be precise) and I wanted to avoid being caught up in delayed administrative procedures due to the pandemic I jumped almost immediately on the next available train with two suitcases while packing up and organising the move of all my personal belonging with a company over a week later. Quarantining alone with a limited data package, no TV, and no radio (but two books) is tiresome. This is the time when you realise how much your interactions with friends mean to you, while also living through the downsides of single life.



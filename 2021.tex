\chapter{Year 2021}
\label{2021}

\section{January 10: Bad S\"ackingen \& Laufenburg (Baden)}
\label{2021:BadSaeckingen}

After December was a stay at home months, I finally got out with my dad, staying close to home though, first visiting the town of Laufenburg. Their advertisement slogan is one town, two countries. Historically the town of Laufenburg has a centuries old history together before Napoleon decided to redraw borders separating the town in the middle. The larger part of old town with a ruined castle is in Switzerland nowadays. Usually one can just walk across both parts of the city, but nowadays with the government in my German state deciding walking over the border is not allowed without quarantining we didn't do that. Instead we continued our short afternoon outing in S\"ackingen. Home to a nice Rokoko church which used to belong to a monastery, the Fridolinsm\"unster. Usually on a nice beautiful day like this many guests, also from the Swiss side might have enjoyed a walk or cafe, but right now that isn't an option either, we did walk a bit on the Wooden Bridge, in fact the longest wood covered bridge in the Europe beating out Kapellbr\"ucke in Luzerrn by a few metres. Clearly not allowed to cross over we turned around midway as well.\\

Bad S\"ackingen: Wooden Bridge****, Fridolinsm\"unster****\\
Laufenburg (Baden): Old Town***\\

\section{February 4: Aach \& Reichenau}
\label{2021:AachReichenau}

It had been raining heavily and snow was melting too, rivers were really might on that day. Even the little creek of Aitrach looked more like a river and the Danube had crossed over its dams in some spots. The Aachtopf is Germany's largest karst spring, fed by water originating from the Danube sinkhole where the Danube itself disappears for about two thirds of the year (clearly not the case during snow melt). The spring is impressive forming a little lake just by a hill side. Since we were in that area we continued to Reichenau in order to see the first UNESCO world heritage site of 2021 (if you follow the past years you realise it appears pretty often on that list). This time my dad and I started at St Peter and Paul, continued with the Monastery, ending in the highlight of St Georg. I always enjoy seeing all those old churches and their early medieval frescoes. Unfortunately this time around we were not able to stop in any cafe to get some coffee or cake, clearly all of those had been closed too.\\

Aach: Aachtopf**\\
Reichenau: St Peter \& Paul****, Monastery****, St Georg*****\\

\section{February 6: Lake Constance Area}
\label{2021:LakeConstance}

Uhldingen-M\"uhlhofen: Pilgrimage Church Birnau****\\
Lindau: Minster***, St Peter***, Harbour with Lighthouse and Bavarian Lion****\\
Weingarten: Abbey****\\
Meersburg: Old Town****

\section{February 14: Haselbach Falls}
\label{2021:Haselbach}

After about a week of not going outside, I convinced my dad to do one of the quick walks close to the area, thus we drove the few villages up to Indlekofen and walked down the little valley of Haselbach. The trail had been refurbished just in Fall 2020 and we made use of those new fences on the slippery icy patches of the path. The Haselbach creek falls down in several cascades with the largest drop by the Haselbachfall which is 14 metres high. The trail crosses the creek on three occasions. Thus you get a view of the waterfall from above and below. At the time of our visit due to the cold temperatures of the preceding days parts of the waterfall had been frozen. Also a tree had fallen and got stuck in the middle of the fall. Not a bad experience for a 45 minute long walk.\\

Waldshut-Tiengen (Indlekofen): Haselbach Falls***

\section{February 19: Schl\"uchtsee}
\label{2021:Schluechtsee}

What do you do if you want to enjoy a nice afternoon in the region: what about going to one of the multiple little small mountain lakes. The Schl\"uchtsee is a very small reservoir, set up originally as ice lake for the close-by Rothaus Brewery. The lake is about 5 m deep, in mid February the lake was completely frozen, but pretty idyllic walking around it (the trail is not more than 2 km long). The lake is within a forest area, the Schl\"ucht river begins just a bit beyond it flows into the lake. Typically one can have beers or wine in the farm close to the lake or in summer just jump into the water having a short swim. It is a nice place for families with small kids, but not really anything special to be honest.\\

Grafenhausen: Schl\"uchtsee***

\section{February 21: Beuron and Upper Danube Canyon}
\label{2021:Beuron}

It was only February but it felt like an eternity after the second lockdown which had cut short the end of hiking season mid October. Thus my dad and I felt it was time for a hike, particularly since the weather was supposed to be very beautiful. Clearly a first hike of the season shouldn't be too tough either though. Thus a bit of distance without too much elevation seemed ideal, particularly since trails shouldn't be covered with snow. Thus we opted for the Upper Danube Canyon, starting out by the Beuron Monastery. The monastery itself is still a working arch abbey, with a beautiful baroque church, and a chapel built in the so-called Beuron style, a variant of neo-byzantine merged with neo-romanesque and neo-gothic attributes. \\
Then we walked downstream for about 7 km, on walking up the cliff sides to a couple of view points, where one can see castles, little villages, bridges and rock formations alongside the meandering river. In fact this part of the Danube only delivers part of its water to the main stream of the river later on, since the majority of the water disappears downstream  in an underground cave and reappears by the Aachtopf (see \ref{2021:AachReichenau}) close to Lake Constance later on. Usually more than 50 \% of the days in a year the river falls completely dry. The valley is very nice to look at, the hiking was fine too (my dad claimed it was a bit tough for him). We did also have to climb through and above a couple of trees which blocked the normal pathway. On our way back we chose another trail just next to the river. We also had a short peak into little chapels and viewpoint on little rocks over the river bed itself. All in all the trail was something around 8-9 km long, so at least for me a good nice and quick hike to start the 2021 season.\\
Then we drove through the canyon until we reached Sigmaringen. Sigmaringen itself is still home of a cadet branch of the former German Imperial family of Hohenzollern. In fact the state rooms of the palace can be visited all year around (when there is no lockdown). Clearly not an option in February 2021 we still walked along the Danube, had a short look into the town church of St Johann, and got ice cream before driving back home.  \\ 

Beuron: Monastery****, Upper Danube Canyon*****\\
Sigmaringen: St Johann***

\section{February 23: Roggenbach Castles}
\label{2021:Roggenbach}

One of the little things in the area in the Steinatal, a small river of the black forest with a canyon like appearance. I knew that in this valley there were two ruins close to each other pretty much at the end of the valley, but only a 30 minute drive away from home, thus good enough for a short walk. Although the walk is indeed only 1 h between the two ruined castles, the ascent and descent over about 100 m is happening pretty quickly, thus it is indeed a bit steep. A bit of snow was still on parts of the path through the forest, but nice enough. Seems though the castles were still in renovation, so one could only see the outside of them. On the way back we stopped by the stone quarry of Detzeln to take photos of a waterfall down one of its walls.\\

Bonndorf: Roggenbach Castles***

\section{February 26: Menzenschwand \& Todtnau}
\label{2021:Todtnau}

St Blasien (Menzenschwand): Albklamm \& Menzenschwand Falls****\\
Todtnau: Todtnau Waterfall*****

\section{February 27: Freiburg \& Black Forest}
\label{2021:Freiburg}

Breitnau: Ravenna Gorge*****\\\
Freiburg: City Gates**, Minster******\\
St Peter: Monastery****\\
St M\"argen: Monastery****\\\
Titisee-Neustadt: Titisee***\\
Lenzkirch: Windgf\"allweiher****\\
Schluchsee: Schluchsee***

\section{February 28: Rickenbach}
\label{2021:Rickenbach}

Originally the plan was to go to the ruined castle of Wieladingen first, and then continue via the Lehnbachfalls along the Murg river with its waterfalls up to the Strahlbrusch waterfall. It was possible to get to the cascades of the Lehnbach creek, but the bridge crossing the river at that point had been damaged by fallen trees a couple of days previously. Thus one had to get to the castle via a different route crossing the creek a bit below. The castle itself had been renovated a couple of years before, from the belfry one can see up to the Swiss Jura (the alps were not visible then), and up the Murg valley on the other side. Unfortunately the way along the Murg had been closed as well. Thus we drove up to another parking lot, and start our second hike from there. This hike led alongside an old abandoned quarry and passing the Seelbach creek down to the Strahlbrusch waterfall. The waterfall is about 12 m high, and in snow melt pretty strong as well. We crossed the Murg via an old bridge of an old postal route which also crossed two small tunnels. Since the trail was closed on this side, we just turned back to the car. All in all a nice afternoon out but for sure the Murg canyon itself could have been a nice hike too, but if it is closed due to snow or rock falls you rather don't risk it.\\

Rickenbach: Castle Ruin Wieladingen***, Strahlbrusch Waterfall****

\section{March 2: G\"orwihl}
\label{2021:Goerwihl}

G\"orwihl: Aybach Waterfall***, H\"ollbach Waterfalls****

\section{March 6: Constance \& Hegau}
\label{2021:Konstanz}

Konstanz: St Stephan***, Minster*****, Christ Church**, Holy Trinity Church**\\
Hilzingen: Hohenkr\"ahen****\\
M\"uhlhausen: M\"agdeberg****\\
Tengen: Hinterburg**, M\"uhlbach Gorge***

\section{March 9: Waldshut}
\label{2021:Waldshut0309}

In Waldshut the Wutach river finishes its flow by the Rhine river. Recently a lot of work had been going into making the mouth appear more natural and less canal like. Thus islands had been created with a second small arm of the Rhine as well as an area where the Wutach river can overflow in case of high water. Only a months ago snow melt was happening but already in March the second high water arm had retreated leaving only muddy but pretty dry territory.\\

Waldshut: Wutach Mouth**\\

\section{March 25: Triberg \& Villingen}
\label{2021:Triberg}

Triberg: Triberg Waterfalls*****, Schonach Waterfall***\\
Schonach: first world's largest cuckoo's clock**\\
Villingen: Johanniterkirche**, City Walls and City Gates***, Minster***

\section{March 28: Alb gorge}
\label{2021:Albtal}

Dachsberg: Alb with Teufelsk\"uche (Devil's Kitchen)****\\
G\"orwihl: Krai-Woog-Gumpen and Gletscherm\\"uhle***\\
Gutenburg: Teufelskessel waterfall***

\section{March 30: Blumberg, Mundelfingen \& Immendingen}
\label{2021:Blumberg}

Blumberg: Schleifenbach Waterfalls****\\
Mundelingen: Mundelfinger Waterfall \& Aubach Gorge**\\
Immendingen: Donau Sink Hole** (too much water at this time of the year), H\"owenegg with crater lake****\\

\section{April 11: St\"uhlingen \& Grimmelshofen}
\label{2021Stuehlingen}

I had been in St\"uhlingen multiple time on carnival parades, music festivals, or just as final destination of bike rides in my teenage years. It had been a while though since I saw old town with the city church, and the church of the Capuchin monastery. Then we continued our trip to the small village of Grimmelshofen, walking along the Wutach and going down to the Dampfkessel waterfall, not as mighty e.g. as the Lauffen waterfall in my home village, but 2-3 m are not nothing either. On the way back we drove past the Hohenlupfen castle and took a panoramic route to Bonndorf enjoying some views of the snow covered alps in the distance.\\

St\"uhlingen: old town***, panoramic road with view of the Alps***\\
Grimmelshofen: Dampfkessel waterfall***

\section{April 20: Hohentengen}
\label{2021Hohentengen}

Another close place to home is the village of Hohentengen, originally home to three castles. Out of those one has been completely destroyed, on that spot there is now a Swiss army bunker. The second castle of Weisswasserstelz is in ruins nowadays too, a few walls remain though, which give you an idea of the former glory. Rotwasserstelz Castle still guards the bridge over to the Swiss town of Kaiserstuhl. But in April 2021 borders are forbidden to be crossed.\\

Hohentengen: Rotwasserstelz Castle**, Ruin Weisswasserstelz**

\section{April 24: Kaiserstuhl \& Breisach}
\label{2021Kaiserstuhl}

Bickensohl: L\"ossholweg hike****\\
Breisach: oldtown with Stephansm\"unster****

\section{April 28: Moving to Vienna}
\label{moveVienna}

At some point also physicists leaving academia find jobs: I found my first job in Vienna. Since the pandemic was still going on (3rd wave now), it all was followed by a quarantine protocol. Since I received the green light for my flat by the very end of April (26 to be precise) and I wanted to avoid being caught up in delayed administrative procedures due to the pandemic I jumped almost immediately on the next available train with two suitcases while packing up and organising the move of all my personal belonging with a company over a week later. Quarantining alone with a limited data package, no TV, and no radio (but two books) is tiresome. This is the time when you realise how much your interactions with friends mean to you, while also living through the downsides of single life.

\section{2021: Vienna general}
\label{2021Vienna}

Considering that I live in Vienna now, the city drops out of touristic trips and travel for the time being. Nonetheless it is clearly a place to visit with many spots to see. Instead of now being silent about all of that from the following on, I decided to collect per year all information gathered within each year in order to keep track of new developments or just to describe what there is to do (or on the other hand what you should avoid doing).\\

Now what about nature: what many people are not aware of is that in Vienna you even have parts of an Austrian national park: The Donau Auen. Around Vienna the Danube river was left largely unregulated for many centuries. By now the river has been corrected and regulated. One arm is the canalised Donaukanal, the former main arm has no connection to the Danube anymore but exists as lake called Alte Donau. Other side arms are now fed by ground water and rain without being connected to the main river anymore. A lot of reed is growing in those low water and only partially flowing old arms. Nowadays plans are put up to divert periodically a bit of water from the Danube to keep the eco system alive and full, in some parts the arms had been reconnected by now. Anyways you find many wetland typical plants, animals such as frogs, fish, insects all over the are in a forest scape. A grid of walking trails and bike paths has been set up on both sides of the river.\\
From the churches I consider Stephansdom the most outstanding, even the exterior is special with its coloured roof-tiles and the towers next to the choir, and not flanking the main facade. It is a bit of a pity you have to pay to see anything of the interior, but other churches do that too nowadays. The chancel, the Wiener-Neust\"adter Altarpiece and the tomb of Emperor Frederik III are masterpieces of gothic style wood and stone carving, unfortunately only some stained glass windows survived, or have been put in in modern times, thus the usual wow effect of such gothic art is a bit missing. \\
Second only to the masterful tombs of French Kings in St Denis and next to Roskilde, Kapuzinergruft has an absolutely amazing collection of coffins and funeral monuments to offer, far superior to the Hohenzollerngruft at Berliner Dom for example. Now Hofburg has many wings, many museums to offer: the Kaiserliche Schatzkammer contains some of the best treasures in the whole German speaking world, in fact the former German Imperial crown jewels is the most complete collection of medieval regalia in all of Europe, and absolute highlight of medieval art. There's also the actual Austrian Imperial crown, but as more modern piece it is a bit of lesser value in my opinion. Clearly the treasure does contain a lot of church items too. Another museum in the Hofburg complex is the Albertina, which is in fact a museum with two parts: the first part are state hall of Habsburg princes and their families, mostly from the 18th and 19th century (and unlike in the Imperial palaces, photos are allowed here), the second part is a Modern art museum with pieces from Monet, Renoir, Picasso, Ernst, etc). Both parts on their own are amazing, now imagine getting both of them in one ticket. Now for lovers of more traditional art, predominantly from the former Imperial collection of paintings, go to Kunsthistorisches Museum, the piece you might have heard of is Brueghel's Tower of Babylon. Another arts museum combined with Baroque state halls, and a Baroque Garden (that one is for free though), is the Belvedere Museum, where the Upper Palace contains paintings, such as Gustav Klimt's Kiss, the Lower Palace exhibits sculptures (closed for all of 2021 and maybe 2022). \\
If you think of libraries, most of those are situated in monasteries, but the most beautiful library in Vienna is the former Imperial Library, which is nowadays part of the National Library, a giant hall with lots of columns, statues, golden leaf decoration and large ceiling frescoes, one of the most beautiful libraries I know about. The zoo of Vienna is the oldest in the world, in fact it is situated on the grounds of the former Imperial summer palace of Sch\"onbrunn: besides the classical part, which alone shows old baroque architecture and symmetry, you can find Elephants, big cats such as tigers or lions, ice bears, pandas, koala's, rhinos, penguins, etc etc. Sch\"onbrunn palace suffers a bit of overcrowding, you find a bit less crowds in the final rooms of the Grand Tour usually. The palace rooms span from Baroque period of Maria Theresia to periods of the 19th century such as Franz Joseph. My personal highlights are the Great Gallery (you are allowed to take photos in that one hall), and the Vieux-Laque-Room and the Millionenzimmer (both only accessible on the Grand Tour). I had a special experience this time around as I went just on the second day of reopening after the Corona lockdown, I had the palace basically for myself - magical but most probably not able to be replicated.\\
Now the palace garden is vast and large, and largely free to visit, you can find many locals who just enjoy time there sitting by the ponds or the meadows, or doing their jogging rounds through the hills and alleys: now you have artificial ruins, several fountains, statues scattered throughout the park, but also a view point, side gardens, a palm house (one of the best I have seen in botanical gardens, although the rooftop ways had been closed), a desert house, and a large Maze, where it took me a while to find the way to the lookout stairs, so for sure something fun to do. Close to the palace is also the Wagenburg with some of the carriages (and one car) which were in use by the Imperial family up to the fall of the Monarchy in 1918.\\

As former imperial capital what is just four stars in Vienna might be like five stars in other places, but well you got to work in your own backyard too:\\
Vienna has a couple of Baroque churches, all of which are beautiful orienting themselves more on Roman Baroque than the classical style of France or England: Two domed churches are well-known, particularly Karlskirche with its great setting just next to a large reflecting pool. Karlskirche would be a five star church wouldn't some genius have thought leaving an old construction elevator up to the dome would be a good thing to do. It now obstructs the view of a significant portion of the dome which spoils the atmosphere. It does offer close view of the dome and old city with Stephansdom from the dome though, but a roof terrace view which is possible without that obstruction could have been a better alternative. Add the 8 EUR fee and you get the downgrade. Now Peterskirche is beautiful too with the dome dominating the church space, but a tad smaller than Karlskirche and a bit darker, but unobstructed. The Jesuit Church is bright colourful Baroque with many side chapels, the Franziskanerkirche is a bit darker, but I liked the idea and style a bit more. The Bernardikapelle at Heiligenkreuzerhof is a tiny but beautiful gilded Baroque gem. The stained glass window of the neogothic Votivkirche are an excellent example or the revival of stained glass art in the 19th century.\\
Unfortunately the actual state halls of the Hofburg palace are not open for view, some are used in the context of the OSCE headquarters and congress centre, the former ball rooms had been redecorated after a devastating fire, and serve now as temporary parliament chambers. Nevertheless the private quarters of Franz Joseph and Elisabeth, the last long reigning imperial couple of the mid 19th century, are virtually left unchanged in one wing. While not as sumptuous as Sch\"onbrunn or late Baroque palaces, they still give an interesting view of some of the last emperors of Austria (unfortunately no photos allowed either). Now the Staatsoper of Vienna was built in the mid 19th century. From those times one of the Foyers, the loggia, and the former Imperial Hall (now called Teesalon) survived the destructions of the war, but that still gives a great impressions of the fashion of those time, a tad less beautiful than the opera of Budapest though which was built around the same time. What is now special about Vienna's state opera: the new modern interpretation of the amphitheatre as well as new marble inlays and modern style tapestries in the side Foyers, a nice contrast but still with a fitting harmonic style. I so far cannot comment on the quality of the actual performance.\\
Now let's talk about Museums: Albertina Modern offers nice pieces from Lichtenstein, Baselitz or Warhol, as well as other  changing contemporary exhibitions. The Museum of Natural History offers a gigantic mineral collection, also many fossils, dinosaurs, but not only that, even masterpieces of the ancient Hallstatt culture are on display here, as well as some gemstones from the former Imperial Habsburg family, definitely interesting both for folks like me as well as family with young kids. The classical roman and greek artefacts in the Ephesos museum or the worldwide treasures from Central America, South East Asia, and Africa(including some of the problematic Benin Bronzes) can be found in the Neue Burg wing of Hofburg. \\
The central cemetery is gigantic, with loads of famous people's tombs as well (think e.g. of Beethoven). The church is also a large modernist style basilica, which can be also interesting to visit. Another unusual church is the concrete modern style Wortruba church which has an interesting layout by giant bricks and layers of concrete interleaved with glass. For lovers of modern art (and remains of the original Modernist decoration) you could check out the Ernst Fuchs museum in his former villa, which had once been the personal home of architect Otto Wagner. Speaking of Otto Wagner you can still find some of his old train station pavilions scattered along metro lines 2 and 4, particularly nice is the former Imperial station by Sch\"onbrunn, the Hofpavillon which has been turned into a museum, such as the former station by Karlsplatz. The Schmetterlinghaus is a large butterfly house, and butterflies are just something i enjoy looking at.\\
Now I had visited concert in the Large Redoutensaal, but right now the Redoutensaal and adjacent rooms are the stand in for Austria's parliament. All rooms had been part of the Imperial Winter Palace of Hofburg previously, most of them are thus in Baroque or classicist style, besides the Large Redoutensaal which had completely burnt out in the late 90s, and such the walls and particularly the ceiling have been decorated by modern style frescoes, but still nice to see.\\

The Silberkammer in Hofburg is nice, but just gets repetitive quite quickly, a collection of gold and silver plates, chandeliers etc, each one of them really nice, but it feel more of the same already in the second or third room. There are many ticket options where you can combine it with the Kaiserappartements or Sch\"onbrunn but standalone not up with other things. I am not that amazed by horses (if you should be put that up in the highest category AND visit a show or a training performance), thus I attended the architecture tour of the Spanish Horse Riding School. You see the beautiful festive hall of the winter riding school, as well as the roof construction and the roof top view of Vienna. Seeing that construction was a highlight, and don't worry you still get close to the horses when visiting Stallburg. Now the Secession has one single nice room with the Beethovenfries, whenever I had been there all other rooms had been closed off, and the ticket price is just too much for that one (indeed nice room). Vienna has a furniture museum, partially based on old Imperial furniture from castles which had been used for other purposes, as well as furniture the new emperors might not have liked anymore. Then we got a couple of churches which are each on their own nice, but not special compared to the previous ones in category four, from Baroque (Annakirche, Kirche am Hof, Schottenkirche), to revival style (neoromanesque Franz Assisi with neobyzantine with Kaiserin-Elisabeth-Ged\"achtniskpaelle), or the gothic style Maria am Gestade. The museum of Schottenstift is nice too, the Festsaal and Library are from the 1830s, quite nice too (although the Festsaal is another version of the Albertina festival hall by the same architect), but maybe a bit expensive. Usually the Baroque Hall of Vienna's old city hall is closed off, by chance it had been open when I visited, and I did make use of it. The hall and the adjacent rooms are quite nice to see, but most probably they are closed off should you pass by. Same could be true for the beautiful staircase of Palais Daun-Kinsky, although I read online that the stairs are often open and accessible.\\
Some hidden places give you also an impression of medieval life in Vienna, such as the Virgil chapel with an entrance inside the Stephansplatz metro station, or the Neidhart Festival Hall with medieval murals, it is such hidden that I even passed it twice although I knew the address (the door of the building is closed and you need to ring a bell, announce you want to visit the room, and only then you are allowed to get in).\\
A couple of things can be visited on special occasions such as the Tag des Denkmals. In 2021 I visited with my friend Vivan the Austrian Academy of Science. This building was originally built as University during the time of Maria Theresia. From this time the large Festival Hall (with a partially reconstructed ceiling fresco after a devastating fire in the 1960s) with marble decorations and statues and the perspective ceiling in for Vienna atypical classical French Baroque is amazing sight to see, in another room the baroque decoration has been conserved as well. In the French embassy the ambassador welcomed us and his wife guided us through the state rooms as well as the private dining room where French presidents and Austrian guests have been welcomed throughout the years. Another highlight was the former state parliament of Lower Austria, which was in function for politics until 1997 when the parliament moved to St P\"olten. Most of the rooms have renaissance ceilings and wood-carved doors and walls, or grotesque style ceiling. The main hall was though later restyled with a large Baroque ceiling by request of the emperor.\\

Augustinerkirche in Hofburg is the former imperial church, kind of plain though if you ask me, with a nice altar piece but otherwise a bit underwhelming. Minoritenkirche, a gothic style church, has a couple of nice windows and a copy of Da Vinci's Last Supper, but nothing much more. Next to the Kapuzinergruft, Kapuzinerkirche is just a small shadow of a beautiful church. Palais Lobkowitz is a speciality museum, listing history of theatres, old puppets and marionettes and costumes, masks can be found there, for sure something interesting for theatre lovers. The Eroica Hall in the middle of the museum is nice too, the museum has items from re-known artists such as Rubens, Boticelli or Hieronymus Bosch (his last judgement is the only altar I really enjoyed there though), but just not on the level of other arts museums in Vienna. Similarly for the Globe museum, telling the stories of how to produce maps, globes, as well as globes of zodiacs or the sky, or moon globes, once again rather for lovers of globes. Now at Belvedere 21 they had an exhibition of Beuys, in fact a huge collection and variety of his art pieces. Beuys is considered one of the most interesting contemporary artists, but it is just not my thing. Now protestant churches are always a bit more solemn and ``simple'' compared to catholic churches, in fact in Vienna they are still comparably richly decorated, but just not like many other churches here. The Dommuseum is small and just not on the level of what I would have expected from such a beautiful church like Stephansdom, maybe some pieces were transferred to the Imperial Treasury which does have a spectacular exhibition of religious items too. Having visited many Roman museums before, the one of Vienna is just very small and tiny, but Vienna hasn't been a large town in the Roman era, so maybe expected.\\
The Museum for Applied Arts (MAK) has in fact a nice permanent exhibit with old baroque room furniture (and a complete room) from Baroque times, also many glass and table dishes on display, some paintings of the modernist Vienna era (Klimt etc), and beautiful carpets. Unfortunately they rather displayed too few of those things, and instead decided to have an exhibit considering ongoing challenges. The topics they discussed are valid and for sure need discussion, but the take of the artists was lazy and just uninspired in some cases. As someone working with AI and machine learning trust me, most of my colleagues and myself are also worried about blindly applying it, about our personal bias being bad in decisions, and how to address ethical issues. Similarly when it gets to privacy issued, or implications of wrongly implemented or misusing AI. But then to read artists just claiming that all of us are just evil, unethical, only in for the money, or just not caring enough, being weird far removed nerds, that is just lazy off-the-shelf diagnostics and just not true. I do understand these things are complex and maybe we folks need to explain our efforts to fight wrong decisions or how to prevent AI from being misused, but to visit a museum and be just wrongly accused or just yelled at felt just bad. So MAK put more effort into discussing rightfully and important topics but please not just in an itself biased way either.\\
The university is one of the oldest in the German speaking area, the building itself was though finished back in 1874, and has been partially destroyed and rebuilt in WWII. The festival halls are really nice, with paintings by Klimt and Mack, although some of those had been taken down since they had been too controversial for that time, and then were lost during times unfortunately. The library is OK, but not as exciting as other large libraries (and in Austria you can definitely see many beautiful library, e.g. the National Library in Vienna itself or the library in Schottenstift). Palais Epstein is nowadays used by the parliament for meetings, or just right now as library since the actual parliament building is under renovation. Most of the rooms are in neo-baroque style but many of the interior decoration and furniture has been lost, still the ceilings or marble is still around, and not to forget mirrors, original sliding doors and original electric chandeliers, definitely cute to see, but wouldn't put it on top of my places to see in Vienna.\\
The House of Industry was another building I visited during the Tag des Denkmals. One of the last big buildings of the Ringstrasse to be finished in the 1910s the festival halls had been set up as meeting spaces already back then with projection walls, electric lights, elevators, etc, albeit in halls of neo-baroque style.\\

With so many churches to see there are some, which might just be alright in other places, but they just don't hold up at all next to their neighbours etc: the parish church of Kagran up in the north, actually an old church but just nothing much there to see. The Johann-Nepomuk-Kirche bei Nestroyplatz is just a normal church, a large mural in the choir, more a classicist style otherwise, but well just normal. The Malteserkirche is a tad small, and being just between Stephansdom \& Karlskirche doesn't help to raise its status. Deutschordenskirche is close to that area too, has a couple of shields on the wall, but otherwise a tad empty.\\

A total disappoinment: the museum of contemporary art, mumok. Modern art is clearly in the eye of the beholder, but I typically find something interesting. This time I was just bored though, nothing seemed exciting or interesting, even artists I usually really like such as Max Ernst or Pablo Picasso were kind of underwhelming. Add the large entry fee on top of that, just not my thing, might not have helped that the topic of the ongoing exhibit was the anti-art movement.\\

%five, four, two, one, and zero stars are done july 13

Vienna: \\
Five Star Attractions (12): Stephansdom***** (paying to get closer to the altar pieces though, South Tower****, North Tower***, Catacombs**), Kapuzinergruft*****, Schloss Sch\"onbrunn***** (Palace: Grand Tour***** no crowds), Sch\"onbrunn Palace Gardens***** (Maze*****, Gloriette****, Kronprinzengarten***, Palmenhaus*****, W\"ustenhaus***, Wagenburg****),) Hofburg: National Library*****, Hofburg: Kaiserliche Schatzkammer (Imperial Treasury)*****, Tiergarten Sch\"onbrunn*****, Belvedere***** (Oberes Belvedere*****, Unteres Belvedere*****, Belvedere Gardens****), Nationalpark Donau-Auen (Danube Wetlands)*****, Hofburg: Albertina***** (State Rooms*****, Museum****),  Kunsthistorisches Museum*****, Tiergarten Sch\"onbrunn***** mit Kaiserpavillon****\\

Four Star Attractions (12): Peterskirche****, Jesuit Church****, Votivkirche****, Franziskanerkirche****, Hofburg: Kaiserappartements****, Hofburg: Weltmuseum (World Museum)****, Karlskirche****, Albertina Modern****, Naturhistorisches Museum**** , Hofburg: Ephesos Museum****, Bernardikapelle at Heiligenkreuzerhof****, State Opera****, Hermesvilla \& Lainzer Tierpark****, Prater Stadtwanderweg (City hiking trail) with old arms of the Danube****, Otto-Wagner Hofpavillon Hietzing****, Ernst Fuchs Museum (Otto-Wagner-Villa)****, Zentralfriedhof (Central Cemetery****), Wotrubakirche****, Schmetterlinghaus****, Hofburg: Redoutens\"ale \& Replacement Parliament****, Palais Nieder\"osterreich, Franz\"osische Botschaft, \"Osterreichische Akademie der Wissenschaften****, Bundeskanzleramt****, Konzerthaus****\\ 

Three Star Attractions (13):  Michealskirche***, Schottenstift*** (Schottenkirche***, Festival Hall and Library***, Museum****), Kirche am Hof***, Old City Hall*** (with Barocksaal), Hofburg: Silver Chamber***,  Secession with Beethovenfries***, Stadtpark (City Park)***, Franz von Assisi Church (with Kaiserin-Elisabeth-Ged\"achtniskpaelle)***, Hofburg: Spanish Riding School with Roof Top*** , Annakirche***, M\"obelmuseum***, Maria am Gestade***, Palais Daun-Kinsky: Staircase***, Heeresgeschichtliches Museum***, Otto-Wagner-Pavillon Karlsplatz, Virgilkapelle***, Neidhart-Festsaal***, Mariahilf church***, Greek Church of Holy Trinity***, Stadttempel (Great Synagogue)***, Palais Epstein***, University***, Haus der Industrie***, Bundesaussenministerium***, Bundesministerium f\"ur Bildung***, Hofburg: Ahnensaal***\\

Two Stars (10): Hofburg: Augustinerkirche**, Minoritenkirche**   Deutschordenskirche**, Palais Lobkowitz** (Theatre Museum**, Gallery of Academy of Arts***), Palais Mollard: Globenmuseum**, Belvedere 21 (Beuys Exhibition)**, Protestant Church** (interior), Lutheran Church** (interior), Dommuseum**, Kapuzinerkirche**, R\"omermuseum**, Greek-Catholic church of Barbara**, MAK (Museum for Applied Arts)** (permanent exhibition****, extra exhibition*)\\

One Star (3): Johann-Nepomuk-Kirche*, Kagran St Georg*, Malteserkirche*\\

Zero Star: mumok

\section{May 16: Schloss Hof}
\label{2021SchlossHof}

Engelhartstetten: Schloss Niederweiden*, Schloss Hof**** (Schloss***, Park*****)

\section{May 22: Golling an der Salzach \& Salzburg}
\label{2021GollingSalzburg}

OK the weather of May 2021 wasn't on the bright side with frequent rain showers and cloudy days, but finally after the nightly curfew had been lifted I wanted to get a bit out of Vienna into nature, thus the region of Salzburg was my first place to visit. Being my usual self I got up at 4 am to make it to the train to Salzburg comfortably and then going south into the alps. Once I got out in Golling the clouds were hiding the mountain tops, nice to see, but the rain was not that amazing. Anyway I made it to the waterfalls just a bit after they opened at 9 am, and during my visit (in rain) I was the only one around. In Golling the Schwarzbach drops in two big waterfalls and a couple of cascades by about 79 metres. The trail is alright with lots of stairs and a couple of bridges which can be quite humid, not only from the rain such as in my case but also from the water vapour of the waterfalls. The lower fall is particularly beautiful, since it widens up at the base in several water patterns. The trail leading up offers several perspectives of the fall before it crosses over several cascades. At that point one can see the base of the upper waterfall which hides itself within rocks otherwise, only when one crosses the bridge and climbs up a couple of 10 metres from another bridge one gets a full view of the upper fall. Another flight of stairs later one reaches the spring of the creek, which starts just a couple of metres away from the upper fall. Altogether very impressive, although I did hurry a bit due to the rain, so try to visit it in a bit of sun. The foggy/cloudy setting still looks great on photos, but getting real wet wasn't that much fun.\\
And onwards back north to the southern tip of Salzburg by Schloss Hellbrunn. While the castle itself was still closed due to Covid19 regulations, the park and the water fountains could be visited. While water and trick fountains had been en vogue in Baroque times, once they went out of fashion they were removed from the parks in almost all places. Thus typically the places are quite crowded and popular where they can still be found nowadays. Once again we were told that on a usual day Hellbrunn has about 10 times the visitors of my visit, thus the experience might be vastly different then. I did enjoy the audio tour, also the trick fountains in action as well. I did not only enjoy the fountains but also the grottoes on the ground floor of the castle, as well as the mechanical theatres which are part of the park too. Then I walked through the forest area of the park (which can be visited for free). The Stadtaussicht has a nice view of Hohensalzburg Fortress, the Watzmannblick suffered from too many clouds, thus the Watzmann massif hid on that very day, but the Steintheater, a artificial theatre set up by huge blocks of stone was worth the walk alone. Thus Hellbrunn is definitely a place which can go on your Salzburg itinerary.\\
And then it was finally time to get over to old town, which is another UNESCO world heritage site. But we reach the utter absolute disappointment of Salzburg first: The new residence. This place houses typically exhibitions in beautiful state rooms. For whatever reason they had the whole floor with the state rooms closed off and the same ticket price as ever. The exhibit regarding the Salzburg Festival was very poor too, mainly just panels to read with very few exhibits. I was very disappointed. Then I walk over the squares of old town with short visits of Michaelskirche and Kollegienkirche, having a short snack on the Market Square. I was particularly pleasantly by Franziskanerkirche which has a pretty dark romanesque nave which opens up to a wide bright gothic style choir hall, whose chapels have been refurbished in Baroque style. Another Baroque style gem with lots of stucco, large paintings, etc is the arch abbey of St Peter. Then the highlight of Salzburg: the beautiful Old Residence with the Arch Bishop's State Rooms. Amazing baroque paintings, tapestries, large frescoes, the odd modern art pieces here and there though. Not really fitting the context unfortunately, but unlike in other places I did like the sculptures in fact. Thus maybe they should have been placed in other rooms in their own right. Now part of the itinerary is also the cathedral museum, which is small but with some precious pieces. The oratory of the bishop is usually very nice to see, this time it had been inexplicably obstructed by a giant mirror wall dividing the room into two parts (why just why?). Still onwards to the cathedral: one of the largest Baroque churches north of the alps, unlike many other Baroque churches in Austria, this one is more similar to the classic Baroque of France or England. \\
The most visited place in Salzburg is the Fortress of Hohensalzburg, one of the best conserved large medieval castles of the whole continent. And unlike in other places at least parts of the rooms, the so-called F\"urstenzimmer, are largely conserved in their original state. Since you have to pay extra for these rooms only a couple of people opted for that it seems, at least I had all three rooms to myself for quite a while. The gold leaved wood carvings, ceilings, columns, medieval shields on the beams etc are some of the best from original late medieval gothic period. I know those are only three rooms, so might sounds only little for a steep price, but just consider those rooms are about 600 years old, and only a handful of those secular rooms are still conserved in such a good state elsewhere. The view of old town and the mountains is nice too from up of the castle. \\
I had finished my program much faster than anticipated, thus I had time to improvise. First I visited the close-by cemetery by St Peter with a couple of chapels, also some hidden in the hillside (the so-called catacombs). Then I decided first to try to get a good view of old town from the Capuchin hill. On my way to the view point I passed St Johannes im Berg and the Capuchin Church itself (not that amazing though). The view was pretty nice, maybe you could do a snack stop up there. On my way to the train station I walked past Dreifaltigkeitskirche (Holy Trinity Church) and Mirabellpark - places of typical Baroque, also the stair case of Mirabell was nice, but nothing I would go out of my way for. I still managed to just get on a train an hour earlier than I had planned, thus I just arrived back home just in time for the Eurovision Song Contest 2021 (priorities I know, I know).\\

Golling an der Salzach: Waterfalls*****\\
Salzburg: Schloss Hellbrunn Wasserspiele \& Park with Steintheater*****, Alte Residenz***** (State Halls*****, Museum St Peter****, Wunderkammer***, Residenzgalerie****, Cathedral Museum with Oratories****), Neue Residenz, Michaelskirche**, Kollegienkirche***, Franziskanerkirche****, St Peter****, Dom*****, Festung Hohensalzburg***** (F\"urstenzimmer*****), St Peter Cemetery*** (Catacombs**, Mariazellkapelle***, Margarethenkapelle***), St Johannes im Berg**, Capucin Church**, Holy Trinity Church***, Schloss Mirabell (Donnerstiege only) with Gardens***

\section{May 24: Hallstatt}
\label{May2021Hallstatt}

After a one day intermezzo of visits in Vienna (State Opera \& Horse Riding School) it was time for my first visit to the Austrian State of Upper Austria, jumping on the first train out of Vienna once again. Hallstatt is very popular with South Asian tourists, since it features prominently in TV series (I heard it is a series in South Korea). But due to Corona restrictions none of those were expected to be around, but with a higher influx of locals (such as myself). But none of the buses had been full, but people getting on the train in Upper Austria started already their private drinking and hiking parties in the train. The train ride from Linz is already very nice past the mountain scape by Traunsee. I can easily understand why Franz Joseph and Elisabeth enjoyed their summers in the region of Bad Ischl. And then I reached Steeg and made my way to enjoy the view of Hallst\"atter See by the outflow of the Traun before getting on the first of two buses to bring me to Hallstatt Lahn. There I jumped on the cable car up to Salzberg. The view from the Skywalk View point up their is magical with the Dachstein massif all around the lake (although I will never understand why it is the current fashion to built those ugly pointy over the mountain platforms which add next to nothing to the experience itself. Then I walked the 15 minutes to the Salt Mine, put on the purple overall and joined the tour of the old mine tunnels. The Salt Mine is still in continuous use after over 7000 years of mining. Now the two slides were fun, the multimedia projection over the salt lake was neat too, but was it worth 36 EUR, not really, but at least I've done it now, although which Salt Mine can claim to host the oldest wooden staircase of the continent (if not the planet).\\
 Instead of taking the cable car down I opted for the trail down to the village. Besides the obvious pretty views of the mountain lake itself the great waterfalls of the M\"uhlbach were the cherries on the cake. The creek falls down over several hundreds of metres in a series of several large waterfalls, really great to see (the view of the last big falls is a bit spoilt by the ugly-ass parking lot). Once down in the village I saw the catholic church (nice late gothic wood carved altar pieces), decided to skip the Bone House (at least 45 minute waiting time, and I doubt it would be as impressive as those of Rome or Milan), had a short peak into the protestant church and got myself a snack on the market square.\\
  And then it was time for hike number two down into the Echerntal along the Waldbach. The first large waterfall which one passes on that trail is the Schleierfall further into the valley I passed several small cascades with large rocks until I reached the Waldbachstrub wall, where the Waldbach meets the Lauterbach and one can admire four waterfalls. At the time of my visit all waterfalls were carrying quite a bit of water, that much water that on the upper view point the waterfall showered me with its vapour. I definitely recommend you spent these 60-120 minutes (depending on your walking speed) to enjoy these waterfalls. On the way back I passed by other spots and the Gletschergarten. I still had time left to visit the Welterbemuseum, there one can admire artefacts from the Hallstatt period, both from the Salt Mine and the tombs up on Salzberg.\\
   And last but not least time for a boat ride. I had anticipated I would have absolutely no time for a boat tour, but seems my hiking speed was faster despite the long break of any alpine hikes. The boat ride goes in a round trip to Obertraun and back to Hallstatt offering beautiful views of the Dachstein mountains. Maybe once Covid19 dies down, you can also enjoy drinks on the boat again, else enjoy the view masked up. Still time to go I had some local soup and beer before jumping on a bus and three trains plus a metro ride back to Vienna (it does take that long as it sounds, a bit more than 4 hours, but the trip and the Alps, and the views were definitely worth the hassle).\\

Hallstatt: Salt Mine****, Skywalk View Point*****, Trail Salt Mine down along M\"uhlbachfalls*****, Catholic Church****, Protestant Church**, Echerntal hike with Waldbach, Waldbachstrub waterfalls \& Schleierfall*****, Boat Ride round trip to Obertraun****, World Heritage Museum***

\section{May 30: Klosterneuburg}
\label{Klosterneuburg}

Klosterneuburg is a neighbouring town of Vienna, where an old large monastery is located. While the monastery has roots dating back to middle ages, the church interiour was completely remodelled in Baroque style. Emperor Charles VI of Austria wanted to built his summer palace here in an Austrian version of the El Escorial (in fact the cafe in the monastery is called El Escorial). After his untimely death his daughter Maria Theresia decided to not continue the building campaign and thus less than one fourth of the whole project had been realised. Some halls such as the Sala Terrana are half finished, only a handful of rooms of the Emperor's wing give you an idea what could have become of this place, the emperor himself spent one single night in these quarters. The church itself is the typical baroque style of Austria with tons of stuccos, one large nave with many frescoes and one large nave with side chapels. The largest treasure of the monastery can only be seen with a guide, a medieval seven-branched wooden candelabrum, and a large altarpiece with 52 golden enamel plates which had been the decoration of the former pulpit which was turned into an altar after a large fire in the 14th century.\\

Klosterneuburg: Stift***** (church*****, cloister and altar of Verdun*****, Kaiserzimmer****, Treasury****, Museum****)


\section{June 3: Hochosterwitz}
\label{2021Hochosterwitz}

Hochosterwitz: Burg Hochosterwitz*****

\section{June 12: Lilienfeld, Herzogenburg \& St P\"olten}
\label{2021Lilienfeld}

Lilienfeld: Monastery*** (Church****, Cloister****, Library***, State Halls***)\\
Herzogenburg: Monastery****{State Halls****, Church****, Library***)\\
St P\"olten: Franciscan Church****, Cathedral****, xyz**

\section{June 13: Myra Falls, Steinwandklamm \& Wiener Neustadt}
\label{2021Myrafalls}

I had looked up what other folks suggested in terms of beautiful waterfalls. The closest ones from Vienna which appeared on many lists were the Myrafalls. Once I found out that not that far away was also one of the slot canyons which appeared on several suggestions concerning canyons and gorges, it was clear to me that at one point I would try to combine both of those. Since the area is dominated by karst, outside of the snow melt times or thunderstorms the Steinwaldklamm falls completely dry. The waterfalls are always running, but at times water is diverted to reservoirs and thus they can be on low water, something which was unfortunately the case when I visited.\\
I decided against waiting for the bus and walked the 3 km from the closest train station along the river, forests, and over nice meadows. The waterfalls were quite full, seems many people from the Vienna area had the same idea. Still I didn't need to wait more than 10 minutes to get inside. Although the falls were on low water, it was still nice to see the several cascades chaining up on my way to the mountain top (one way walk right now due to Covid19 rules). Once I got out on the top I continued my hike to the Haustein hill top. From this mountain top one has a good view of the valley and on the other side over to Schneeberg. Between 30-60 minutes later I reached the start of Steinwaldklamm, as expected on very low water as well. It was particularly nice how cool the air was among those rock walls, and I did enjoy walking through the tiny trail in between. There is also a Klettersteig installed, starting in the middle of the trail. It is targeting families, but I didn't do it myself, but instead walked up the normal path. At the end of the path the way leads through a small tiny cave, where according to legend villagers tried to hide themselves from Turkish soldiers.\\
And then it was another hour to walk all the way back to the start of the Myrafalls, and further to get to the train station, where I had a small beer while watching one of the Euro cup games on the terrace. On my way back I stopped in Wiener Neustadt with its old castle, which is now a military academy with its own military cathedral which was closed though. On my way to the old cathedral I stopped by a quite decent Baroque monastery of Neukloster, the bishop moved from Wiener Neustadt to St P\"olten in the 19th century, but the cathedral still shows of its former importance.\\

Muggendorf: Myrafalls (on low water)***, Haustein view point****\\
Furth an der Triesting: Steinwaldklamm***** (low water)\\
Wiener Neustadt: Carmelite Church*, Burg with Military Cathedral**, Dom (former cathedral)***, Stift Neukloster***

\section{June 20: Wachau}
\label{2021JuneWachau}

Emmersdorf an der Donau: Marktkapelle**, St Nikolaus***\\
Spitz an der Donau: Tausendeimerberg Viewpoint****\\
D\"urnstein: D\"urnstein Castle****, Stift D\"urnstein****\\
Krems an der Donau: Dom der Wachau****, Piaristenkirche***, museumkrems in Dominikanerkloster***, B\"urgerspitalkirche**

\section{June 21: Baden bei Wien \& Heiligenkreuz im Wienerwald}
\label{2021Heiligenkreuz}

Baden bei Wien has been famous for its thermal baths for quite a while. Emperors, famous architects, and famous musicians spent their summers there to relax. Nowadays some of these classicist baths have been closed down or refurbished, such as the Women's Bath which is now an art gallery for Arnulf Rainer paintings. The Kurhaus is now a casino, the large park around it is still nice to do. Beethoven spent a couple of summers in town, and the museum claims this is where he came up with crucial ideas for his 9th symphony.\\

A short bus ride away from Baden is Heiligenkreuz im Wienerwald. Besides a nice little sacred hill with stations of the way of the cross on a hill side the village is famous for its large Cistercian monastery with its large Romanesque basilica (with nice baroque choir stalls and sacristy). The cloister is largely medieval, with gothic style windows and fountain house. The capital house has been repainted in the 19th century, but even there the new stained glass window won a price at the Paris world fair back in the day.\\ 

Baden bei Wien: Kurpark***, Stephanskirche**, Beethovenhaus***, Arnulf Rainer Museum in Frauenbad***\\
Heiligenkreuz im Wienerwald: Stift Heiligenkreuz****, Kreuzweg***

\section{June 26: Seitenstetten, St Florian bei Linz \& Linz}
\label{2021SeitenstettenStFlorian}

Seitenstetten: Stift Seitenstetten**** (church***, guided tour through main halls****), Hofgarten****\\
St Florian: Stift St Florian***** (church*****, guided tour through monastery halls***** and Kaiserzimmer*****)\\
Linz: Mariendom****, Ursulinenkirche***\\

\section{July 3: Obertraun}
\label{2021Obertraun}

Obertraun: Koppenbr\"ullerh\"ohle***, Dachstein Rieseneish\"ohle*****, Mammuth\"ohle*****, 5fingers viewpoint****, Dachsteinblick (Dachstein Viewpoint)****

    \begin{figure}[htbp!]
  \includegraphics[width=0.32\linewidth]{pictures/2021/P7030030ObertraunKoppenbruellerhoehleWasserfall.jpeg}
    \includegraphics[width=0.32\linewidth]{pictures/2021/P7030080-ObertraunDachsteinRieseneishoehleEisstalagmit.jpeg}
      \includegraphics[width=0.32\linewidth]{pictures/2021/P7030124-ObertraunMammuthoehle.jpeg}
  \caption{Impressions of the Koppenbr\"ullerh\"ohle, the Dachstein Rieseneish\"ohle, and the Mammuth\"ohle}
  \label{fig:Obertraun}
\end{figure}

\section{July 10: Kitzlochklamm, Liechtensteinklamm \& Bischofshofen}
\label{2021Pongau}

On this day I had another one of those road or bike road hikes, since no single bus connected the train stations and the canyons I wanted to see, thus for each of them I had to walk more than 1 hour each way. At one point one of the bike trails along the river was closed thus the detour led through an old stone quarry, which was typically closed off for transfers, despite google maps actually leading through it as nominally quickest trail. But back to the canyons: they were indeed amazingly beautiful.\\

The first canyon was Kitzlochklamm. On my 1 hour hike to get there I passed an old castle hill by the village with some decent view of the village and the valley itself.

Taxenbach: Kitzlochklamm*****\\
St Johann im Pongau: Liechtensteinklamm*****\\
Bischofshofen: Gainfeldwasserfall****, Bachfall Ruin***

\section{July 17: Bad Gastein \& Klagenfurt}
\label{2021GasteinKlagenfurt}

Having enjoyed the slot canyons last weekend I thought already about visiting the waterfall of Bad Gastein due to its proximity to the train station and to combine that with a visit of Klagenfurt. As usually I checked the weather forecast before booking tickets. Despite the forecasted small rainfall, I thought it should be nice to do the trip.\\

Unfortunately what I failed to realise or be informed off in time was, that the Salzach valley had a standing warning of huge flooding, albeit only starting in the afternoon. Still it was a bit unsettling to take a train through that area with the river levels already high in heavy rain fall. Indeed the rainfalls in Bad Gastein were a bit more than small, but not too worrisome. Also all hiking trails were open, such as the trail by the waterfall.\\

I was a bit worried if the train from Salzburg would still be scheduled, and indeed it arrived on time, later that day flooding did indeed happen along the rail, and thus parts of the trains had to be replaced by buses.\\
Having arrived in Klagenfurt I got a short lunch snack before getting to the Landhaus of K\"arnten. The Landhaus itself is a nice Renaissance building with a couple of nice halls, particularly impressive is the Wappensaal, showcasing coats of arms of local rich citizens. Nowadays the Landhaus contains the state parliament of the Austrian state of K\"arnten. Then I walked through old town passing a couple of squares with fountains and monument, and churches, which were OK but nothing too impressive. Then I had to wait for about half an hour for a wedding to finish in the Dom. The cathedral itself is a baroque style church with a couple of nice painting and frescoes (and obviously elaborate flower decorations for the wedding). The original cathedral of the diocese used to be in Gurk for the longest time, before the seat of the bishop was transferred to Klagenfurt. And with this cathedral I managed to see all cathedrals of Austria, considering I saw the first cathedral in Innsbruck in 93 it took quite a bit of time to achieve that though. And then I sat inside the train for more than four hours again, the track passing the Semmering line once again.\\

Bad Gastein: Gasteiner Wasser***** (Waterfall and waterfall hike), church***\\
Klagenfurt: Landhaus****, Hauptpfarrkirche St Egid***, Kapuzinerkirche**, Heiligengeistkirche***, Dom****, Squares with their monuments*** (Heuplatz with Floriani-Denkmal, Alter Platz with Dreifaltigkeitss\"aule, Neuer Platz with Lindwurmbrunnen)\\

\section{July 23-July 25: Innsbruck, Krimml \& Wattens:}
\label{2021Innsbruck}

It had been really quite a while since I did my last city trip (over a year ago to Rome). The last non day trip happened also in August 2020, thus I felt I should go somewhere and stay for a bit. Clearly that wasn't only the case for me, but for my friends too.
I had also not seen my Geneva friends for about a year, so when I suggested a possible trip to Tirol in Austria a couple of folks were interested, and in the end we were a group of four. I have been to Innsbruck previously, but also about 10 years since the last visit, and from Innsbruck getting to the higher part of the Austrian alps is always a possibility too. And Innsbruck has quite a bit of culture too, so you can combine best of both worlds.

Co-Travellers:\\
Sam: our second full weekend trip after Paris 2019;\\
Grace: Grace joked that most long trips of hers were leading to Germany (although she has visited Italy too), thus a welcoming change to checkout Austria.\\
Janina: after our three previous hiking trips, Janina will experience her first multi-day outing with me.\\

July 23, Friday: Innsbruck:\\
I was coming by train from Vienna, while Grace, Janina, and Sam were coming by car from Geneva. We all planned and timed our departures to meet up shortly before 7 in the evening in Innsbruck, in order to have an evening stroll and then dinner later on for a slow start.
Unfortunately traffic in Switzerland was really horrible with multiple accidents and traffic jams on the way, thus I did the evening stroll on my own, while the other three finally arrived around 8:30 pm when we met up by the Ibis Hotel in Innsbruck. On my evening stroll I passed through the Triumphpforte along the Maria-Theresien Boulevard with Annas\"aule up to the Goldene Dachl. Then I visited the cathedral where I was the one and only visitor just entering 15 minutes before closing time. Since it was a sunny day I still could enjoy the ceiling paintings by the Asam brothers and the Baroque altars. I walked around Hofburg and realised that the Jesuit church was still open past 7:30 pm as well. Once Grace, Janina, and Sam arrived I took them for another stroll to show them how close old town was indeed before we had a long dinner in old town. While my Austrian covid certificate and Janina's German Vaccination ID looked familiar, it took a while before they went over Grace's and Sam's CDC vaccination cards. After we had successfully proven that we all are vaccinated we had a two course dinner. Unfortunately the kitchen closed then, thus we got our ice cream on the way back instead.\\

Innsbruck: Dom****, Jesuitenkirche***, Old Town (with Goldenes Dachl, Triumphpforte, Annas\"aule)****\\

July 24, Saturday: Gerlosalpenstrasse, Krimml Waterfalls \& Innsbruck:\\
The weather forecast claimed a bit of sun in the morning, but then more and more incoming clouds and maybe even thunderstorms. Thus we got up before 7 am to get ready for the long ride after a short coffee stop by Innsbruck Central Station. And it was bright and sunny, while we crossed over the Zillertal, stopping for views on the Gerlos Alpenstrasse panoramic road for views of lakes, mountains, glaciers, and the main falls of the Krimml Waterfalls. We discussed if we should continue further going up the Hohe Tauern Nationalpark to reach Pasterze but decided a more active hike and less driving would be the preferred option. We still found a spot on the free parking of Krimml, and then hiked up the waterfall trail. Even in summer the waterfalls carry lots of water, the river itself is fed by melting water from closeby glaciers. All of the waterfalls are impressive, and the observation decks get you so close that the water sprays will most probably get all of your clothes (and cameras) wet. About two hours later once we finished the hikes dark clouds were all over the sky. But once we reached the Gerlos mountain pass again it showed to be more sunny on this side of the valley. Thus we enjoyed drinks and a short lunch snack with a superb view of the Durlassboden lake surrounded by mountain peaks once more.\\

Once we arrived in Innsbruck we started our sightseeing at Hofkirche, the royal church containing one of the most impressive Renaissance tombs of Europe, where Emperor Maximilian's marble cenotaph is surrounded by 28 life-sized bronze statues of his Habsburg ancestors and relatives, as well as legendary kings and emperors. Next to the main nave of Hofkirche is the Silver chapel, with a silver altarpiece and another Renaissance tomb of another Habsburg duke of Tirol. The church is connected to the Imperial Hofburg palace, a former city castle which had been converted into a late Baroque \& Rokoko residence on the orders of Empress Maria Theresia to celebrate her son's wedding in Innsbruck. From that time the Riesensaal festival hall with giant paintings of the Imperial couple and its children is particularly impressive. The imperial apartment had been redecorated for the Imperial family in the late 19th century for their outings to Tirol in summer months. Clearly Innsbruck's Hofburg is far smaller than Vienna's but it is still worthwhile to visit, particularly since its main state halls can be visited unlike in Vienna, where those are used by the President of Austria or the OECD, unfortunately photography is banned in Innsbruck though as well. And clearly the last item on our old town tour was the cathedral. We had a tour scheduled in Stift Wilten so it was time to get there. But for reasons I still don't understand something went wrong when paying our parking ticket. But instead of just failing and asking us to pay using another method the ticket machine just ate up our parking ticket. We called emergency service which then told us to just make it to the barrier and then we should call again so thus that the barrier could be opened manually. We managed almost to get to Wilten in time (aka we were three minutes late), where the Vice Prior was awaiting us.\\

I had been at Stift Wilten previously, but only seen the church, which let's you watch through a fence of the entrance hall typically. In my early teens I had been an altar boy, and one resident of my local village had become a priest back then in his mid 20s. Once on his visit home he held the service and told me how beautiful the monastery would be, and I should try to visit it sometime, and maybe he would guide us. That unfortunately didn't happen back then, but once I considered to visit Innsbruck once again I checked if Stift Wilten would offer tours nowadays. On the homepage it was stated that there are no regular tours scheduled, but one can always try to get it organised. That's what I did and only two days before our visit we were offered a possible time slot and we chose to go for it. The tour was given in German and Janina and I alternated translating for Grace \& Sam. Indeed there were loads of details to be explained for the church which I was not aware of previously, we saw remnants of the gothic altar pieces and the baroque rooms of the monastery including the festival and the entrance hall, but also the private meeting rooms for guests of the monastery. And once I told the vice prior my story of why I finally decided to see the monastery we found out that our guide was in fact the very same priest from my home village who once told me to pass by and he would guide us. So just a bit more than two decades later we finally managed to achieve it, what a coincidence.\\
We had a short peak into the Basilica Wilten and then it started to train, thus we decided to skip another mountain lake, to rest a bit, and then go into old town again for a longer dinner instead of continuing our tourist program. After dinner Sam and I had another round of drinks while Janina and Grace decided to call it a day.\\

Gerlos: Gerlos Alpenstrasse****\\
Krimml: Krimml Waterfalls*****\\
Innsbruck: Dom****, Hofkirche*****, Hofburg*****, Basilica Wilten****, Stift Wilten****\\

July 25, Sunday: Wattens \& Innsbruck:
Since the weather forecast predicted unstable conditions with frequent possible rain showers here and there we decided to focus on culture and discard natural places such as lakes and canyons. Thus we could also sleep in a bit before our breakfast in the train station. Our first stop was Kristallwelten in Wattens. There Swarovski set up a park and an arts exhibition focused around there various forms of crystals (naturally they also have a major shop installed at the end of the exhibit). I did enjoy it and let rephrase Grace's quote who stated that the exhibition was better than she expected, but not as amazing as it could have been with a few small tweaks here and there. Clearly light installations and reflections and a play of contrasts is what the art pieces focus on inside. The outside open air installations work with wind, sunlight, and reflections on ponds, clearly set up to work in summer as well as in winter with spotlights etc. I obviously enjoyed myself.\\
Then we went back to Innsbruck and visited Schloss Ambras. This castle had been turned into a Renaissance palace back in the day. Even in the 16th century parts of the palace had been set up as exhibition place for armoury, curiosities, and art. Thus it is advertised as the world's oldest museum. The palace gardens have formal parts close to the main palace and a landscape park on the outer parts with nice views of the Nordkette mountain range. The most impressive palace hall is the Spanish hall with a large Renaissance wooden ceiling and several murals. The high palace court is decorated with murals depicting ancient legends. Once we finished our visit of Ambras first thunderstorms reached Innsbruck thus we skipped further items on the agenda and rather had a longer lunch, before Grace, Janina, and Sam got back on the road to Geneva, while I jumped on the train back to Vienna.\\

Wattens: Swarovski Kristallwelten****\\
Innsbruck: Nordkette Views****, Schloss Ambras****

\section{August 1: Carnuntum \& Eckartsau:}
\label{2021CarnuntumEckartsau}

Carnuntum was the province capital back in Roman times. Nowadays a couple of houses have been reconstructed in an archaeological park, and excavations revealed amphitheatres, trading houses, baths etc. In fact the site itself is part of Austria's latest addition to the UNESCO world heritage as part of the extension of the Roman border remains. Enough reason to visit the place, and it can be combined with an Imperial Hunting Palace too.\\

The walks had lasted longer than anticipated, but at least the Roman Museum was worth the detour, in fact only after a visit of the museum I left the site of Carnuntum without feeling having wasted my time. I know we should most probably not just compare Roman sites to those in Italy, but if I think of other Roman ruins in Spain, Germany, northern Africa or Turkey, they are all amazing. But even Avenches in Switzerland has a bit more to offer, thus not the best place to visit for Roman stuff.\\
Now to the second part of the trip. Between Vienna and Bratislava there is only a single bridge to cross the Danube. That came in handy to get by bus over to the other side of the Danube to see the royal hunting lodge in Eckartsau. Since Austria doesn't care about good connections to rural places at all, once again I had to walk 4 km from the closest bus stop to the palace. Eckartsau castle had been used quite a bit in Baroque times, but then had been left unattended, and it was only in the 19th century that crown prince Franz Ferdinand rediscovered it and brought it back to life. This is also the place where Austria's last emperor Karl had been staying last before going into exile. It was here that he signed his de-facto abdication papers for Austria and Hungary. The castle is situated in the Donau Auen as well. The old water arms and moats are almost fully covered in reed, and mosquitos had quite a field day even in the palace. I had pear cake and coffee, had a quick walk around the castle, and then the heavy thunderstorm hit, which I waited out in the Imperial Staircase. In fact even without the tour you can see most of this staircase, as well as the chapel, and the small stair case. The Imperial Staircase is largely original Baroque with a nice large ceiling painting and the typical representation purpose. The small stair case is covered in antlers as well as eagles and other birds. So even if you can't be bothered to join a tour, maybe passing by the Donau Auen for a walk and just having a short visit of those three free places might be worth a little detour.\\
And then I had to walk through the rain to the bus stop, two buses and a bit more than two hours later I finally arrived at home. Crazy to think how much slower normal buses are, considering where trains can get you within two hours.\\

Petronell-Carnuntum: Heidentor***, amphitheatre**, archaeological park***, round chapel St Johannes**, church St Petronilla**, Schloss Petronell**, Nationalpark Donau-Auen (Danube Wetlands)***\\
Bad Deutsch-Altenburg: church Maria Himmelfahrt***, amphitheare**, Roman museum****\\
Eckartsau: Schloss Eckartsau****\\

\section{August 7: St Margarethen im Burgenland, Rust am See \& Eisenstadt:}
\label{2021Eisenstadt}

St Margarethen im Burgenland: Stone Quarry***\\
Rust am See: Seebad \& Neusiedler See**, Toleranzkirche*, Dreifaltigkeitskirche*, Old Town***\\
Eisenstadt: Dom*, Franziskanerkirche***, Antoniuskirche**, Kalvarienberg****, Bergkirche**** (with Haydn Mausoleum**), Schloss Esterhazy**** (Haydn Exhibition**, Wine Cellar***, State Rooms****, Tour with Servant Rooms****)\\

\section{August 8: Laxenburg}
\label{2021Laxenburg}

Laxenburg is the second large summer residence of the Austrian Emperors next to Sch\"onbrunn. I considered for quite a while to visit the park, but since it is so close to Vienna I always pushed it to a later date in favour of other places. After coming home the previous day a headache popped up, and considering that I decided to sleep in. Thus possible destinations had to be close to Vienna, thus I finally made it to Laxenburg. Unfortunately the actual large palace of the Blaue Hof is not publicly accessible, in fact a wedding was going on in the former dining hall on that day. The park is nevertheless open to visitors (although you have to pay a little fee even for that). First I visited the parish church, one of the earliest Baroque churches in Austria not built in their typical rectangular manner, but with some movement (also with a nice fresco in the dome). The park itself is a large landscaping park, but still with some original canals, greek temples, statues, and also a nice baroque garden pavilion with a nice ceiling fresco (the Gr\"une Lusthaus). Unfortunately some parts are fenced off due to safety reasons such as the Grotto, although it does look pretty nice from the outside. For sure there is enough to spend 1-2 hours just walking through the vast landscape..\\
 The main attraction within the park is though the revival castle of Franzensburg on an island within the artifical lake. Here Emperor Franz I of Austria (he also happened to be the last Emperor of the Holy Roman Empire as Franz II) constructed over a range of 35 years his own personal castle museum. Already set up as his private museum the emperor expected monasteries, old city halls, old castles, and others to gift him the furniture or the rooms themselves completely. For example the chapel is a transferred medieval chapel from Stift Klosterneuburg, which had been in a pretty dire state back then before the Franz had it transferred and renovated. Thus although the castle itself is built in the 1800s a large part of the interior is in fact genuinely medieval. The windows are done by some of the best stained glass Austrian artists of the 19th century. A downside is that in all rooms of the museum tour photography was strictly forbidden (well strictly for real cameras that is, but more about that later). While dining rooms and also bedrooms exist, they had hardly ever been used for that purpose but they exist rather as museum pieces already from the get go.\\
 Every proper castle has its belfry and Franzensburg is no exception. The large tower, the rooftop and the top floors can be visited on a second tour, and on this tour photography is allowed, also in the interiour (don't ask me for reaons why it is not allowed during the first tour). The guide of the second tour was the same like for the first museum tour, and he did a great job. Unfortunately a fellow visitor thought she needed to reprimand me to listen to the tour and not to take photos. Little did Madam know that the first 10 minutes of explanations were in fact identical to what I had been told the previous tour. And even IF not ever heard of folks which can take photos and listen to what other people say at the same time. In fact it was rather unpleasant that she and her friends refused to wear masks. Add to the fact that she and her friends held up the tour later on the roof top to do their insta shots. Nothing wrong with that, after all I think if it doesn't get too out of hand waiting for others to take photos is all fine, but then if you yourself love to do that don't yell at other people who just do the very same like you do. Anyways the walking over the roofs was really enjoyable, the halls of the tower were nice too with stained glass windows, and a high gothic-revival style decorated Knight's Hall, as well as a painted dungeon with a 19th century robot prisoner, which can be moved mechanically to scare of visitors. By the end of the tour we were led through two rooms of the museum. While I was reminded by our guide of the ban on photography NONE of the others bothered to listen and at least 10 put out their phone to take a couple of shots. The double standard on photography is thus also prevalent in Austria: no photography means always no photo cameras, but a certain tolerance on phones, which obviously does though now include phones of people who own cameras (god knows why). Also our friendly lady from before was one of those who couldn't be bothered to follow. \\
 Anyways, visit Franzensburg, definitely worth more than visiting the inside of other revival castles such as Hohenzollern or Neuschwanstein. I continued my walk through the park to see the second lake island with the nowadays ruined pleasure house from the lake shore, before getting back to Vienna by bus (the bus ride is less than an hour from the main station). In early times there even had been a train line set up from Vienna to the palaces for the emperor, but that one has been decommissioned for almost a century by now. Also for those who might not be capable of long walks anymore, there exists a light ``rail'' which drives through the park, as well as a little ferry to cross the lake, or one can also rent pedal or rowing boats for the lake.\\

Laxenburg: Kreuzschwesternkloster Laxenburg**, Parish Church****, Schlosspark***** (with Grotte****, Gr\"unes Lusthaus****, Concordiatempel***, Waterfall**, Franzensburg***** (Museum Tour*****, tower and roof top tours*****)

\section{August 14: Stift Altenburg \& Stift G\"ottweig}
\label{2021Altenburg}

Altenburg: Stift Altenburg*****\\
Furth bei G\"ottweig: Stift G\"ottweig****

\section{August 21: Semmeringbahnweg}
\label{2021Semmering}

Considering the weather forecast predicted a lot of incoming clouds by afternoon, but still dry weather I opted for a long hike, but not high up in the mountains. I had looked into a hike around the Semmering area previously, particularly since the scenery was said to be nice from a landscape and nature point of view, as well as offering nice views of the Semmeringbahn, which had been the first successful large scale mountain railway built between 1848 and 1854. In fact until the Semmering base tunnel will finish construction it is still used even for high speed trains between Slovenia/Italy and Vienna. Another point is that this hike can be adopted easily since it passes several regional train stops, thus if you feel you had enough you can just jump on the next train (and wait at most 1 h).\\

I started out by Semmering close to the highest point of the mountain pass, a mountain resort both for hiking and ``fresh air'', as well as a skiing resort (even place of a yearly world cup ski race). From their the trail follows along the train tracks and then moves up to the famous 20-Schilling-Blick, a mountain view point overlooking the Polleros mountain wall with two viaducts and a tunnel, a view which was depicted on the former Austrian 20 Schilling bank note. In fact you have a far wider view of the railroad, with one more viaduct and one more tunnel, as well as the Schneeberg massif in the background. The trail leads usually through forests, at times not that wide, but not through rocky territory. So you typically walk within the shade. The first big viaduct on the trail is the Unterer-Adlitzgraben-Viadukt, leading along meadows as well. Usually about every 15 minute a train crossed the bridges. \\

You can also visit a museum for the leading architect and engineer of the Semmering railways project by the Kalte-Rinne-Viadukt. I was personally most impressed by that giant viadukt. Shortly after crossing the village and the train station of Breitenstein, a little detour is possible to reach two viewpoints. It takes about 45-60 minutes in total to get to both viewpoints and forth and back, but I would recommend you do those: the first view point overlooks Breitenstein village with the two rock faces and the Kalte-Rinne-Viadukt. The second view point is even better, firstly you see the Zauberberg and Semmering, but even more impressive is the view to the other side of the Adlitzgraben canyon with the village Klamm with its ruined castle. The trail continues along the rails with nice views of the canyon here and there, passing two more viaducts before reaching Klamm. The church is not that impressive, the castle appears nice, but it is forbidden to get anywhere close to it, since it is still private property.\\

Then you can choose to either hike to Gloggnitz or to Payerbach, both paths take about 1 h 15 - 1 h 30. The one to Gloggnitz follows the train line before crossing the Schwarza river, while the trail to Payerbach leads through hills and forests first, before reaching the rails again, with a couple of less impressive viaducts. The church is also pretty plain, and last but not least I saw the train to Vienna leave in front of my eyes and had to wait another 58 minutes for the next connection. On the Payerbach-Reichenau train station you also find an old steam train which once was used on the Semmeringbahn, but else there is not that much to do in Payerbach itself.\\

Semmering: Semmeringbahnweg mit 20-Schilling-Blick*****, Unterer-Adlitzgraben-Viadukt****\\
Breitenstein: Semmeringbahnweg mit Kalte-Rinne-Viadukt****, Breitensteinblick****, Semmering- \& Adlitzgraben-Blick*****, Kirche Klamm*, Ruine Klamm***\\
Payerbach: Semmeringbahnweg mit Payerbachgraben-Viadukt**, Church*

\section{August 27-August 29: Prague}
\label{2021Prague}

August 27: Prague\\

Prague: Old Town Square\\

August 28: Prague \& Karlstejn\\

Prague: St Charles Bridge****, Prague Castle***** (Vitus Cathedral*****, Old Palace****, St George's Basilica****, Golden Lane****, Gardens with Belvedere***), Loreto****, St Benedict**, Our Lady of Angels**, Sternberg Palace (National Gallery)***\\
Karlstejn: Karlstejn Castle**** (Chapels*****, Large Tower View***)\\

August 29: Prague\\

Prague: Strahov Monastery***** (Libraries*****, Church****, Monastery and Picture Gallery****), St Francis of Assisi****, Our Lady before Tyn****, Klementinum****, St. Nicholas (Stare Mesto)****, Old Town Hall****, Colloredo-Mansfeld Palace****, St Salvator***,
St. Nicholas Church (Mala Strana)*****, Troja Chateau*****, Waldstein Gardens****, St. Clement****, Jerusalem Synagogue****\\

\section{September 2-September 5: Geiselwind}
\label{2021Geiselwind}

September 2: Train ride to Geiselwind with stop in Passau\\

September 3: Bamberg \& Pommersfelden:\\

Bamberg: Altes Rathaus (Old City Hall)***, St Martin****, Dom*****, St Jakob***\\
Pommersfelden: Schloss Weissenstein*****\\

September 4: W\"urzburg:\\

W\"urzburg: Residenz***** (Palace*****, Gardens****), Neum\"unster****, Old Main Bridge***, Marienkapelle***, Dom*****\\

September 5: Regensburg \& Passau:\\

Regensburg: Kollegiatstift unserer Lieben Frau zur alten Kapelle****, Dompfarrkirche Niedermuenster***, Porta Praetoria****, Kloster St Emmeran****, Schottenkirche***, Herz-Jesu-Kirche**, Spitalkirche St Katharina**, St Mang****, Steinerne Br\"ucke****, Dom***** (Treasury****), St Kassian****\\
Passau: Dom*****, Kloster Mariahilf***, Michaelskirche***, Stadtpfarrkirche St Paul****, Votivkirche***, Matth\"auskirche***, Klosterkirche Niedernburg**

\section{September 18: \"Otschergr\"aben \& Mariazell}
\label{2021Geiselwind}

Annaberg bei Mariazell: Stausee Wienerbruck***, Lassinggraben mit Lassingfall \& Kienbachfall****\\
Mitterbach am Erlaufsee: \"Otschergr\"aben***** (Mirafall****, Schleierfall****), Erlaufstausee****\\
Mariazell: Basilika Mariazell****

\section{September 25: Salzburg}
\label{2021SalzburgII}

Why Salzburg?\\
In Austria there are a lot of natural sights (similar on the German border area around Salzburg). But Michael had strained his ankle a couple of weeks before and didn't recover completely. Thus a big hike was out of question. Both of them didn't want to stay in Vienna all the time, thus close enough places (by train) are Salzburg, Graz, or the monasteries of Wachau. Since Salzburg offers a good mix of museums, baroque buildings and churches, as well as gardens, and last but least a castle it was the place they chose.\\

Co-Travellers:\\
Vivan: I planned to see the hidden treasures of Vienna on Sep 26 on my first Tag des Denkmals heritage event. I convinced Vivan that this would be the perfect day to visit.\\
Michael: Unexpectedly Michael had decided to jump on a plane from the US a week earlier. Since he wanted to pay a visit to Austria anyway, he decided that meeting up with Vivan and myself at the same time would be particularly nice.\\

Since we had used the late Thursday museum openings and watched a ballet performance at the Vienna opera with a late dessert stop afterwards on Friday, Vivan and Michael opted for a late start. Thus we got on a 9:30 train out of Vienna which got us to Salzburg 30 minutes past 12. After a short walk we reached Salzburg where we discovered that they had a city festival going on in old town. This time I decided to walk up to Hohensalzburg Castle to also see the outer fortress gates for once without taking the cable car up and down. We had to show our covid19 certificates three times, fill out another contact form, and had our tickets checked three times. I took them to the F\"urstenzimmer since it is rare to see medieval stoves and woodcarvings of that quality centuries later.\\
Then we had a short sausage snack before visiting the Domquartier. They liked the Baroque state rooms of Alte Residenz, particularly the gobelins, as well as the over 1000 years old artefacts in the cathedral museum. From the organ gallery we witness the wedding vow of a couple. Since the wedding continued we couldn't watch the cathedral as of yet thus we stopped by the Franciscan Church and the abbey of St Peter, and then the cemetery of St Peter. There Vivan told me that the catacombs played a major part as setting for the Sound of Music, but I never watched that movie, so how would I know.\\
Once we returned to the cathedral Vivan stated that she thought the church is so beautiful, if not the most beautiful she has seen. Indeed the stuccos of Salzburg cathedral are outstanding, the paintings in the dome are held in red and blue tones. The crypt is pretty modern with stained glass and stone covered lamps.\\
On our way back we stopped in the Mirabell gardens. By coincidence a concert of Vivaldi's Four Seasons was scheduled this evening, unfortunately too late for us though. But this means that the Marble Hall (Marmorsaal) was open for free views from the door, a nice large hall with coloured marble (or stucco marble, couldn't check it obviously as I wasn't allowed to actually go inside). Since it is for free to watch that hall, as well as the staircase leading to it, I would recommend you just pass by and if it happens to be open, have a look into it yourself, it is worth it.\\
And then we took the train back to Vienna where we had Kebabs to end the day.\\

Salzburg: Alte Residenz***** (State Halls*****, Museum St Peter****, Wunderkammer***, Residenzgalerie****, Cathedral Museum with Oratories****), Franziskanerkirche****, St Peter****, Dom*****, Festung Hohensalzburg***** (F\"urstenzimmer*****), Schloss Mirabell (Donnerstiege \& Marmorsaal) with Gardens****

    \begin{figure}[htbp!]
  \includegraphics[width=0.32\linewidth]{pictures/2021/P9290016-SalzburgErzabteiStPeterFranziskanerkirche.jpeg}
    \includegraphics[width=0.32\linewidth]{pictures/2021/P9290028-SalzburgDom-Krypta.jpeg}
      \includegraphics[width=0.32\linewidth]{pictures/2021/P9290034-SalzburgSchlossMirabell-Marmorsaal.jpeg}
  \caption{Salzburg old town, Dom Krypta, and Mirabell's Marble Hall}
  \label{fig:SalzburgII}
\end{figure}

\section{October 22-November 1: Germany, Luxemburg, France, Belgium \& Austria}
\label{2021EuropeTripFrance}

October 22: N\"urnberg:\\

N\"urnberg: Old Town by night*****\\

October 23: N\"urnberg:\\

N\"urrnberg: Lorenzkirche*****, Sch\"oner Brunnen****, Sebalduskirche****, Frauenkirche***, Kaiserburg N\"urnberg****, Fembohaus***, Ehekarusell***, Egidienkirche**, Elisabethkirche**, Jakobskirche**, St Klara**, Justizpalast with Nuremberg Trials***, Germanisches Nationalmuseum****\\

October 24: Luxemburg \& Trier:\\

Luxemburg: Old town fortification with Bock Casemattes \& Tours Vauban****, Cathedrale Notre-Dame***\\
Trier: Porta Nigra****, Dom***** (treasury****), Liebfrauenkirche****, Museum am Dom*****, Konstantinbasilika***, Rheinisches Landesmuseum****, Kaiserthermen****, Amphitheatre****, Thermen am Viehmarkt***, Barbarathermen**, R\"omerbr\"ucke***, Benedictine Abbey St Matthias with cemetery****, Palastgarten***\\

October 25: Metz \& Nancy:\\

Metz: Cathedral*****, Porte des Allemands****\\
Nancy: Place Stanislas****, City Gates***, Place de la Carriere****, Cathedral***, St-Sebastien**, St-Pierre**\\

October 26: Paris:\\

Paris: St Laurent**, La Madelaine****, Musee d'Orsay*****, Hotel Biron (Musee Rodin)***, Dome d'Invalides****, Cathedrale St-Louis-des-Invalides***, Pont Alexandre III****, Petit Palais****, Pantheon****, Opera Garnier*****, Musee des Arts Decoratifs***, Catacombs****, Arc de Triomphe****\\

October 27: Paris \& St Denis:\\

Paris: Louvre*****, Hotel Camondo****, Sainte-Chapelle*****, Abbey of St-Germain-des-Pres****, Hotel de la Marine***** (Grand Tour), St Eustache****, Centre Pompidou****\\
St Denis: Basilique St-Denis*****\\

October 28: Tournai \& Northern France:\\

Tournai: Belfry***, Cathedral*****\\
Lille: Hotel de Ville with Belfry***, Porte de Paris***\\
Oignies: Fosse 9-9bis****, Terril 110****\\
Arras: St-Jean-Baptiste**, Place des Heros mit Hotel de Ville****, Citadel****\\

October 29: Aachen \& K\"oln:\\

Aachen: Dom***** (Tour of Upper Rank \& Choir*****), Domschatzkammer (cathedral treasury)*****, Rathaus****\\
K\"oln: Dom*****, Domschatzkammer****\\

October 30: Speyer, Worms \& Darmstadt:\\

Speyer: Dom***** (with crypt****), Judenhof****, Dreifaltigkeitskirche****\\
Worms: Dom*****\\
Darmstadt: Mathildenh\"ohe**** with Hochzeitsturm****\\
Frankfurt: xyz terrace view****\\

October 31: Braubach \& Essen:\\

Train ride through Middle Rhine Gorge*****\\
Braubach: Old Town with Market Square and Philippsburg****, Marksburg*****\\
Essen: Dom***, Domschatzkammer****, Zeche Zollverein**** (Tour of Kohlenw\"asche****)\\

November 1: train ride back:\\

Vienna: Schlossgarten Sch\"onbrunn*****

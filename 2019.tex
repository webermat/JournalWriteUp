\chapter{Year 2019}
\label{2019}

\section{January 13: Domodossola \& Orta San Giulio}
\label{2019:Domodossola}

Domodossola: Sacro Monte***, Castello di Mattarella**\\
Orta San Giulia: Sacro Monte***

\section{February 21: St-Cergue}
\label{StCergue2019}

Snowshoeing***

\section{February 24: Chamonix}
\label{Chamonix2019}

Aiguille du Midi***

\section{March 2: Z\"urich}
\label{Zurich2019}

Swiss National Museum*, Grossm\"unster**, Cloister Fraum\"unster*, Kunsthaus***

\section{March 10--March 22: Southern Italy}
\label{SouthItaly2019}

March 10:\\
Bari: Cathedral San Sabino**, Basilica San Nicola***,  Castello Svevo**\\

March 11:\\
Matera: Santa Chiara*, Chiesa del Purgatorio**, San Francesco d'Assisi**, Duomo***, Sant'Agostino**, San Pietro Barisano**, San Nicola Dei Greci \& Madonna Delle Virtu with Dali Exhibition***, San Pietro Caveoso**, Casa Grotta**, Santa Lucia alle Malve**, Santa Maria de Idris**, Convicinio di San Antonio*\\

March 12:\\
Molfetta: Duomo di San Corrado**, Cathedral Santa Maria Assunta*\\
Bitonto: Cathedral*** (with excavations**)\\
Andria: Castel del Monte***\\
Trani: Cathedral**\\

March 13:\\
Lecce: Chiesa del Carmine**, Basilica San Giovanni Battista*, Chiesa del Gesu**, Roman Amphitheatre with Palazzo del Seggio**, Santa Maria della Grazia*, Basilica di Santa Croce***, Duomo***, San Matteo**, Roman Theatre*\\
Bari: San Ferdinando*, Piazza Mercantile*, Piazza del Farnese\\

March 14:\\
Castellana Grotte: Grotte di Castellana*** (long Tour)\\
Alberobello: Trulli**, Sant'Antonio di Padova*\\

March 15:\\
Napoli: Santa Caterina a Formiello**, San Giovanni a Carbonara***, Reggia di Capodimonte**, National Archeology Museum***, Santa Maria del Rosario alle Pigne, Duomo***\\

March 16:\\
Caserta: Reggia*** (with Court Theatre***)\\
Benevento: Arco di Trajano**, Museo del Sannio*, Santa Sofia (church and cloister)**, Basilica San Bartholomeo*, Duomo**, Roman Ruins \& Theatre*\\

March 17:\\
Ercolano: Excavations of Herculaneum***, Villa Campolieto**\\
Portici: Reggia \& Botanical Gardens**\\
Napoli: Palazzo Reale***, Galleria Umberto I**, Chiesa del Gesu Nuovo***, San Domenica Maggiore**, Cathedal Treasury***, Duomo*** (Battistero of Santa Restituta**)\\

March 18:\\
Paestum: Excavations***\\
Napoli: Villa Pignatelli**, Certosa di San Martino***, Castel Sant'Elmo**, Santa Chiara** (church)\\

March 19:\\
Pompei: Excavations***, Santuario**\\
Torre Annunziata: Villa Oplontis***\\
Napoli: Santa Chiara** (cloister), Santo Domenico Maggiore** (church** and sacristy**)\\

March 20:\\
Positano: Santa Maria Assunta*, Coastal Views***\\
Amalfi: Coastal views***, Duomo*** (cloister**, crypt***, museum***)\\
Salerno: Duomo**\\

March 21:\\
Pozzuoli: Amphitheatre**, Macellum**, San Raffaelo**, San Giuseppe\\
Bacoli: Lago del Fusaro \& Casina Vanvitellina**\\
Baia: Roman Baths***\\
Lucrino: Monte Nuove**\\
Napoli: Maria di Montesanto, Michele Arcangelo a Port'Alba*, San Lorenzo Maggiore**


\section{April 6: Murten \& Avenches}
\label{Birthday2019}

Murten: Old City**, Avenches: Old City and Roman Ruins**

\section{April 13--April 17: South East England}
\label{2019SouthEastEngland}

April 13:\\
Cambridge: St Andrews Street Baptist Church*, Great St Mary's*, King's College***, St John's College**\\
Peterborough: Cathedral**, Guildhall*\\
Ely: Cathedral***\\

April 14:\\
London: Canary Wharf***, Greenwich-Old Royal Naval College***, Greenwich-Queen's House**, Greenwich-Royal Observatory*, Kenwood House**, British Museum***\\

April 15: \\
Oxford: St Mary Magdalen, St Aloysius Catholic Church*, Blackfriars Church

April 16:\\
Oxford: St Anne's College, Brasenose College**, Balliol College*

April 17:\\
Oxford: Wadham College*, 

\section{April 18-April 22: Paris I - sans Paris and without getting any sleep}
\label{2019:Paris I}

How we ended up in Paris this time: First of all, we talk about the long Easter weekend, second point, Riju and I decided we should see also a bit of countryside France, but the best connection to anywhere in France is guaranteed from Paris (using at least public transport). After a longer break since our last multi-day trip Chris decided it is time again to risk another trip with me.\\

Co-travellers:\\
Riju:\\
US-American: We were both aware that this might be our last longer trip for the next while, since Riju sees the end of his PhD and his time in France coming by summer.\\

Chris B:\\
US-American: Finally recovered after last year's trip Chris is ready for a new adventure, also he plans leaving for California by late summer.\\

April 18: Thursday\\
SCNF trains can be late, this time the TGV started about 30 mins late, by the time we reached Gare de Lyon, we had a delay of over an hour. Then we wanted to transfer to the Hotel, unfortunately by that time the Metro didn't run anymore (due to construction), thus we had to switch to a bus. At least we glimpsed a night-view of the Chatelet and the Tour St Jacques. The bus line was not equipped to take over all passengers of the metro, thus instead of the planned 30 mins it took over an hour. By that time it was past 1 am as we finally checked in.\\

April 19: Friday\\
Today we wanted to do the Abbey of St Savin: The issue is, that the bus schedules only two buses each day, thus we either have to get up really early, or we would have less than an hour at the abbey. We decided to rather have more time, but that meant taking one of the earliest TGVs out of Montparnasse. So we got up around 4:30 and made another visit by the hotel reception just about 3 1/2 h after check-in (clerk was confused to say the least). After switch to the bus (and coffee) in Poitiers, we finally arrived in the small village of St Savin. The abbey church is famous for its old painted ceiling, depicting various biblical stories ( Noah, Moses, tower of Babylon, the garden of Eden, Joseph and his brothers, etc etc) - quite amazing to see this paintings from a bit less than thousand of years ago. The abbey itself is not that exciting, but make sure to walk over the old bridge and the river. Frogs clearly seemed to love that place too. Then we had an actually really nice lunch at a restaurant just opposite of the church, with Kirs only costing about 1 EUR (and that after they apologised that they had increased the prices for those just for the season). Then back to Poitiers, where we saw the Notre-Dame-la-Garde (nice painted pillars), the cathedral (your typical gothic cathedral), the baptistery (the oldest still intact church building in France, I loved it, Riju was not impressed at all), and St Radegonde (quite cute) and St Jean de Montierneuf (a bit forgettable). Since we were already close-by we went up to Tour Montparnasse, which offers two advantages compared to the view of the Eiffel Tower: one can see the Eiffel Tower in its full glory, and the view is not spoiled by Tour Montparnasse. \\

Abbey church of St Savin***, Abbey of St Savin, Poitiers: Notre-Dame-la-Garde**, cathedral*, baptistery**, St Radegonde*, St Jean de Montierneuf, Tour Montparnasse***\\

April 20: Saturday: Provins \& Fontainebleau\\
Provins is typically reachable by train, but the train had been replaced by buses on this weekend. Unfortunately not running that frequently, we only had two options to get there. Missing the first one by just 2 minutes we made it to the second one, and we even caught the bus with a connecting time of 5 mins only. Once we arrived in Provins we had a rustical breakfast at the main square. Seems nobody else was around when we strolled through the Collegiale church, more people were visiting the Tour Cesar. The town can be fully seen from its top, pay attention not to be too close to the bells when they ring each half an hour (quite deafening I can tell). Then we did a round tour along the city walls before stopping for omelettes etc for lunch. And then we transferred by bus, train, and another bus to the Chateau of Fontainebleau. The second large royal palace (after Versailles and before Compiegne) was the main residence during the Napoleonic empires, in fact the room where Napoleon I abdicated can be visited as well. The oldest part of the palace is from Renaissance times of the Valois family, the main part being the gallery of Francois I and the old ballroom. This was my third time in Fontainebleau, this time we managed to visit the Papal rooms, missing out on the private royal apartments instead. The queue was about half an hour long, nothing compared to the 4-5 h lines of Versailles. I like Empire style decorations. Not as grandiose as Versailles, but still magnificent. Then we walked through the gardens before getting back to Paris, where we first paid a sad visit to Notre-Dame (which had burnt down just a week previously), then had Thai food closeby and then walked alongside the Louvre, before jumping on a bus for a late climb of stairs up to the second floor of the Eiffel Tower. Once again we arrived really late at the hotel past midnight.\\

Provins: Collegiate St Quiriace*, Tour Cesar**, City Walls***, Fontainebleau: Chateau***, Chateau Gardens**, Paris: Eiffel Tower***\\

April 21: Sunday: On my own in Bourges\\
On Sunday Riju and Chris started early to make it all the way to Mont St-Michel (just like I had done in 2015), while I FINALLY made it to Bourges (after my last attempt the year previously failed due to strikes of SCNF). It took close to 4 h until I finally reached the cathedral, only to find out that there was a religious service going on. Thus I first did an outside 360 Tour of the Facades and the portals. Afterwards after a short stop for coffee I could visit the cathedral with its old stain-glass windows. I booked the guided tour of the crypt together with a climb of the tower. The view is really nice (though not as spectacular than at other cathedrals). In the crypt they keep the remains of the tomb of the duke de Berry, the remains of the choir decoration as well as the windows of the former St Chapelle in Bourges. All in all definitely worth paying. Afterwards I visited the Palais Jacques-Coeur, a very nice old medieval palace with an interesting exhibition just below the roof-tops. And then I still had time left to see the private collections at the Hotel Lallemant (tapestries, wood-carving, and other sculptures). And then over 4 h to get back to Paris just arriving shortly before midnight this time.\\

Bourges: Cathedral***, Cathedral Towers**, Cathedral Crypt***,  Palais Jacques-Coeur**, Hotel Lallemant*\\

April 22: Monday: Le Havre:\\
In order to not have to rush to the hotel and back again in the evening we dropped our luggage by Gare de Lyon and stopped for a long breakfast by Gare St Lazare, and then we took of to Le Havre. Most of Le Havre was completely destroyed in the World War II, and rebuilt betweeen 1945 and 1964 after a plan by Auguste Perret. Unlike other modern cities it doesn't feel ugly at all, but rather harmonic. Parks and fountains and places are put throughout the roads and boulevards as well. The library resembles a Volcano, built after plans by Oscar Niemeyer. The highlight is though the church of St Joseph. Truly amazing and with a very interesting tower resembling a light house, illuminated by multiple coloured rectangular windows. And after getting burgers by the library we got all back to Geneva, just in time for the last 3 rounds of Trivia nailing all 10 Whitney Houson songs (though shaky on The Greatest Love Of All Remix. All in all another fun trip to Paris, though we all ended up sleep deprived. I am very sad that it is my last multi-day trip with Chris and particularly Riju for a while. And not to forget I'll be back in Paris anyway anytime soon.\\

Le Havre: Library**, St Joseph***, Place de Hotel de Ville*

\section{May 25: Liechtenstein}
\label{2019Liechtenstein}

Vaduz: Castle**, Cathedral St Florin*, Old Rhine Bridge*
Zurich: Grossm\"unster**

\section{May 30--June 3: North West Germany}
\label{2019:Northwestgermany}

May 30:\\
Hannover: New City Hall**, Welfenschloss*, Marktkirche**\\
Oldenburg: Palace**

May 31:\\
H\"oxter: Kloster Corvey***\\
Alfeld: Fagus-Plant**\\
Hildesheim: Dom \& Dommuseum***, Micheliskirche***\\
Braunschweig: Domplatz**\\

June 1:\\
Bremen: Dom**, Rathaus***, B\"ottcherstrasse*, Frauenkirche*, Norddeich: Wattenmeer***\\

June 2:\\
Goslar: Erzbergwerk Rammelsberg***, Kaiserpfalz**, Marktkirche**\\
Hannover: Herrenh\"auser G\"arten***, Schloss Herrenhausen, Galeriegeb\"aude Herrenhausen***, Berggarten***

\section{June 16: Lucerne \& Rigi}
\label{LucerneRigi2019}

Lucerne: Spreuerbr\"ucke**, Kapellebr\"ucke**, City Wall**, Jesuitchurch**, Weeping Lion*\\
Vierwaldst\"atter See***\\
Rigi***\\

This marks as well the last time Rachel has been on trip with me with five trips with over night stays and two additional day trips, of course hoping that more might follow at some point later on.\\
It is as well the end of an era, losing my so far most loyal travel companion: This marked the last trip I did with Riju, while he was living in France at least. In total we had 14 (sic) trips with overnight stays, additional 10 day trips, and we saw 34 world unesco heritage sites together (the most I saw if I remove family members, 2 more than I saw with Reyer). 

\section{June 22: Vallorbe}
\label{Vallorbe2019}

Grottes de Vallorbe***

\section{June 23: Lavaux}
\label{Valaux2019}

Lavaux***\\

This marks the last time my mum will make an appearance here. Clearly so far we did most multi-day travels together (since my dad bailed out of the Rome trip in 2013). She, together with my dad and my aunt, ignited my love for seeing other places and discovering the world. She clearly paid attention that we get educated about culture and history, but also not take beautiful nature for granted. I cannot express enough gratitude for what she did and meant to me. Thanks mum.

\section{July 28: Glaciers among clouds}
\label{2019:Fiesch}

The alps are always great, and we had a couple of people who didn't see any glacier yet, so time to show them all the Aletschglacier. Chris can also check if the hike would be suited for his dad's upcoming trip to Switzerland.\\

Co-travellers:\\
Ivan:\\
US-American, Ivan will soon graduate, and it is his last weekend in Switzerland at least for this summer.\\

Ulrike:\\
German, after our adventures in Seoul, Ulrike is also up for a hike in the alps. We both have done this hike before, but it's always fun to see Aletsch again\\

Chris B:\\
US-American, Chris wants to check on this hike, if it could be added to his week long hiking adventure which he plans for his dad in a couple of weeks. Excited that coffee-breaks and lunch are foreseen this time.\\

Francesca:\\
US-American, also a physicist working at a CERN experiment. Francesca's dad was born in Italy, thus she has seen a bit of Europe already, but never the mountains and neither glaciers.\\

Jose:\\
US-American, born in Puerto Rico, also a physicist working at a CERN experiment, as well his last weekend in Switzerland for a while.\\

Although the weather forecast wasn't amazing at all, this was the only chance to do the trip. So we still decided to get day tickets and make it our way. Meeting at the assigned time, Ivan and Jose informed us that their taxi would be late, and thus they wouldn't make it in time. Little did they know that SBB had issues as well, and our actual planned train was cancelled. Anyway once Jose and Ivan arrived we still had to buy Jose's ticket and a couple of coffees later we were ready to go. Passing by Lake Geneva, the Lavaux, the Pissevache, and another coffee and pastries stop in Brig we finally reached Fiesch, and after getting already coffee number 3 at Fiescheralp we finally started our hike. The sight was horrendeous, light rain and fog all around us we could only see about 5-10 metres ahead. We didn't miss out horses standing close to the trail and little waterfalls, and once we crossed the tunnel we finally saw something, the glacier and mountain tops with clouds hovering above. We all managed to get down to the ice, it took a bit to convince Francesca that although her shoes were not full fledged hiking boots, it was safe to get down. For some of us it was the first time seeing a glacier. Thus all of us ended up standing on the glacier ice, and this time a glacier cave had opened up. Without a hardhat too dangerous too step inside, but still what we could already see just standing in front of it was magnificent. Then we hiked up to the lakes again, where we stopped at Gletscherstube getting our favourite version of R\"osti. Once we were back at Fiescheralp it was still early enough to make it up to Eggishorn. Although clouds were still covering one side, we could see pretty much all of the glacier, even up to Gr\"unhorn and the Eismeer. We were all impressed to say the least. Once down in Fiesch we had another coffee break, before making it to Brig, getting on the Eurocity from Milan (thus getting dinner on the train unlike for the normal Interregio, which doesn't have a restaurant car -- at least those who didn't fall asleep during the train ride).\\

Eggishorn***, Aletschglacier***, Gletscherstube*** 

\section{August 25: Lost between five lakes}
\label{2019:Zermatt}

Having been in Zermatt so many times, clearly it is one of my favourite destinations in Switzerland. The Matterhorn is clearly THE iconic Swiss mountain, so enough to show it to everybody who wants to see it. This time I took plenty of newcomers to Zermatt, since we wanted to have a lunch break, I thought let's do the five lake hike, which was described as a 2 1/2 h hike, thus we should have plenty of time. I even checked times on the Gornergratbahn or the opening hours for the Gornergorge, just to not be done too early - little did I know what was about to come out of that.\\

Co-travellers:\\
Andres:\\
US-American, born in Puerto Rico, although I know Andres already since 2012, this is the first time I convinced him to join me on a trip.\\

Janina:\\
German, Andres' summer student at CERN. Clearly the Matterhorn is on her bucket list to see when in Switzerland - and it is even more fun to do this hike with a large group. Of course this should not be an issue having a second German on this trip..\\

Chris B:\\
US-American, once again Chris wants to check on this hike, if it could be added to his week long hiking adventure which he plans for his dad in a couple of weeks. As experienced co-traveller Chris knows what he is doing.\\

Grace:\\
US-American, after our first trip in May also up for a hike in summer.\\

Christine:\\
US-American, our first trip together of this year, clearly Christine knows she is the one who has to deal with getting good food on this trip.\\

David:\\
US-American, actually run a long-distance race in Zermatt just the day before, plans to do another hike before joining us in the middle of our adventure.\\

Sam:\\
US-American, just arrived in Geneva a couple of weeks ago, this is also Sam's first trip with me, and his first trip to the Alps of Switzerland Zermatt.\\

We all got our day tickets the day before, so we planned to meet as usually in a cafe at the train station. Seems it needed two simultaneous calls for Andres from Christine and Chris to actually get up, so surprisingly everybody was on time (unlike 4 weekends ago) and we were ready to leave as planned. People are still tired, although 6:30 isn't such an early time to start. Grace, Sam and I are the only ones who seem to be already wide awake. By the time we reach the Mattertal, everybody felt well enough to enjoy the views over the Bisglacier and the remnants of the rock avalanche by Randa, still amazing as always. Once we got to Zermatt it is time for another round of coffee, before we cross over the bridge over the Matter Vispa with the first view of the Matterhorn, and then we get up to Blauherd. From there we reached the first lake - Stellisee - after 25 mins with the Matterhorn towering over it in the distance, just 25 mins later we reach the Fluehalphut. Gornergrat, the Matterhorn, the Strahlhorn, amazing snow and glacier covered mountains all around us. Unfortunately waiting about 30 minutes to be served, the food is nice (almost all of us got Roesti and a local beer), but then once again waiting about 30 mins to get the check, and suddenly over 2 h were gone just for having lunch. Still we had 3 h left, so should be fine to go.  By that time Andres decided that Heavy Metal is more interesting than talking to us. And he thought about letting us know about all the details, e.g. "Wow that a real great riff" and less than 10 seconds later - "and it is already over", or "yes", "nice", "awesome". Janina and I were just busy taking as many pictures as we could. We went down to Grindjisee, which is by the Moranes of the Findelglacier. Unfortunately almost half of the original tongue is gone nowadays, thus we could nowadays cross the valley over to Gruensee. At Gruensee we met up with David. By that time we did realise that it would be pretty tight to get to Sunegga, to catch the last cable car down, indeed if we believed the guide posts, we would be 5 mins late. Thus we decided to walk straight down to Zermatt and just skipping the last two lakes. Since I doubt we were incredibly slow, it rather seems the advertised time for the five-lake-hike does not pass by every lake, but rather follows a shorter route, just overlooking some of the lakes. It might rather take 4-5 h than $<$3 h to finish that hike. Anyways, all the way down along Findelbach for another 400 m altitude until we finally reached Zermatt (our knees didn't cheer anymore by that time). Then we had dinner (Burgers, Gulasch soup, fries and such), and then about 4 h back to get home to Geneva. Although the plan failed for once, nobody seemed to be too unhappy about it.

\section{August 31: Aosta Valley}
\label{2019:Aosta}

Can you imagine: Chris never made it to Italy (OK neither did Justin, and he had about 5 years at CERN). Not allowing myself the failure to not showing them my favourite country again, I decided to take Chris to Italy. Since he didn't want to do an 8 h ride to Milan I suggested to get to Aosta instead. Telling Sasha about my plan he asked if he and his friend Yaroslava could join, and since 5 folks fit in a Honda comfortable we were all ready to go. PS: in the end Chris made it to Italy again just a couple of days later, taking the Bernina train from Switzerland over to Tirano.\\

Chris B:  \\

Aosta: Teatro Romano*, Cathedral**\\
Lillaz: Waterfall***\\
La Thuile: Cascate del Rutor***

This was the last of my trips with Chris before he moved back to California for teaching and hopefully writing up his thesis anytime soon with two trips including over night stays and five additional day trips, and surprisingly enough we saw 21 world unesco heritage sites on just these couple of days.\\

\section{September 15: Langgletscher}
\label{2019Langgletscher}

Janina wanted to do another hike in the alps. Nobody else besides me was up for it, thus we brain stormed where we could go without needing a car (since her car decided to have technical issues). We first thought we could go to Moiry, or to Sorebois in the Val d'Annivers instead or go to one of the huts close to Zinal, but then I found out that the Anen hut in the L\"otschental overlooking the Langgletscher was still open, and it would be rather ideal to go there, since we had several buses and trains we could take back, in case our hike lasts a bit shorter or longer than planned. Since we had to start out very early, Janina stayed at Grace's place over night, we also had a short breakfast in Brig. Once we arrived at the bus stop by Blatten, the path is very nice going along the rapids of the Lonza river through the forrest, before reaching meadows with lots of cows in front of the Grosshorn and the Jeggi glacier. We climbed up the mountain reaching the Anen hut and having superb views of the valley as well as the Lang glacier with its glacier gate and the dirt covered tongue. After getting lunch inside the hut, we walked down over the ridge of the moraine and then following the river, passing my a couple of small lakes, and back to the train station, where we could see people getting on and off the car train of the old L\"otschberg tunnel. And then our train arrived starting our way back home to Geneva.\\

Langgletscher***

\section{September 20-September 22: Paris II - this time seeing more of Paris}
\label{2019:Paris II}

Another trip for the European heritage days in Paris - but FINALLY not alone. With Riju and Chris gone, I tried to convince other folks that travelling with me can be fun. And yes they have been briefed by Riju and Chris what to expect. Four co-travellers, three of them on the first trip with me, the fourth one on his first non-day-trip with me, two people on their first Paris trip. Lots of first things this time, so a bit of history. Once again I tried to get a couple of people to join - some forth and back (even that long that we missed out on cheap tickets to get to Paris) I ended up convincing Sam and Vivan to join on the trip. Just two days later Vivan asked if her friend Nhi could join as well. So we were ready to go, booked our room, when Vivan realised another one of her friends would be in Europe and potentially interested to join in as well (potentially ended up as definitely only two days later). So here we are with a gang of five people in total.\\

Co-travellers: \\
Vivan:\\
US-American, in fact travelling quite a bit within Europe herself, after moving here for longer term just a couple of months ago, but typically emphasising on other points than me. Since Vivan has been travelled to Paris before and loves the city, it was pretty easy to convince her to see other things, which are only open on this particular weekend.\\

Sam:\\
US-American, just arrived in Geneva a couple of weeks ago. In fact Sam had been on a trip with me to Zermatt, so he knew what would be coming his way. Being in Europe, even if only for a short while offers you the opportunity to see so many places, but Paris is clearly one of the famous highlight, even acknowledged across the Atlantic. Maybe a pity that the usual stuff (e.g. Louvre) will be off-limits for this particular weekend, still my hopes are high that he will enjoy it.\\

Nhi:\\
US-American, doing her PhD in physical chemistry statesside. In France for a conference in the beautiful city of Chamonix (just read one of my summaries, the last one just from February this very same year). This will be her first trip to Paris, more or less convinced single-handedly by Vivan to join on our adventure, I am not aware if she has been told what she has gotten herself into. I hope she will see the trip as begin of a wonderful love for this continent as well. \\

Michael:\\
US-American, soon finishing law school in Chicago. Michael happened to get his hands on tickets for a Tottenham Hotspurs Premier League game just a few days after this weekend. Vivan as good friend of him suggested first to stop in Paris before visiting her in Geneva. Since Michael has been in Paris before, it wasn't too hard to convince him of seeing it again. \\

September 20: Friday\\
After having a few drinks and a quick dinner in Geneva previously we got to Paris in a TGV which was unexpectedly leaving on time and even more unexpectedly arriving 10 minutes earlier than planned at Gare de Lyon. We all checked in in our room for four, which turned out to be two adjacent rooms with one bathroom each - great for the next two mornings, thus we should be able to get in and out of the shower a lot quicker than anticipated - or so I thought, but more about that later on. Michael had arrived in Paris a couple of hours earlier, so we got out into Paris nightlife to meet him in a cafe, where we found out that cocktails are cheaper than the beers on the menu. The ``surprise cocktail'' was surprisingly nice, and we got back home only at 1 am - maybe not the best way to get sleep on this Paris trip either.\\

September 21: Saturday\\
This time around tickets had to be booked for certain attractions, this time I chose to show up at Palais Royal first, starting at 9 am, so I planned for getting out by 7:30, for a nice relaxed breakfast, getting there and not standing at the end of the line. Easier said than done, while all the boys were done by 7:25, the ladies took a bit longer, though 7:35 is still a good showing off for first timers on my trips. After getting Pain au Chocolates and coffees, teas etc for all of us, we head out. While on the metro M1 we get told that due to demonstrations the line will operate only on some metro stations, and all others will be not served -- for the whole day. Bad if we chose the hotel for being connected through this line, but at least the final stop of this Saturday is exactly the stop where we want to get off. Since we are there a bit earlier than expected, we chose to walk over to see the Arc de Triomphe du Carrousel in golden sunrise light, as well as later the same with Opera Garnier. Once getting inside the information stand of Palais Royal we got ourself ``Journees du Patrimoine European 2019'' bags, which seemed to be very loved by anyone we met at all other points, but only available at Palais Royal. And once we got to the Ministry of Culture, a lady was supposed to tell us all about the history. Too bad that she only repeated what we shown on leaflets on all the rooms itself, so we decided to go rouge and go through the place at our own pace. Nothing much changed at Palais Royal from the last time i had visited in 2014, besides new pieces of modern art or new carpets. In Palais Royal we find nowadays the Ministry of Culture, State Court and Supreme Court, decorated during the second Empire of Napoleon III. Afterwards we checked out the Banque de France. The highlight is the Golden Gallery in Hotel de Toulouse. I had seen this part of the national bank in later 2013. Somehow the gallery seemed a lot brighter than I remembered. I found out later that this is because the pale green background of the wood panelling was changed to a shiny white, definitely an improved appearance. Next item was the Hotel de Ville, the largest city hall in the world. This time the Festival Gallery was illuminated by red and blue spots, changing periodically. Across I saw the progress made in restoring Notre Dame. The roof had been covered by large pieces of wood, some of the glass windows had been removed. I wanted to get a closer look at the facade so we crossed over to the Ile de Cite stopping to grab some Crepes and Quiches. Unfortunately the square in front of Notre Dame was completely closed off, so we went over to the other side of the island, paying a short visit to Conciergerie, which was in renovation as well. Afterwards we queued for Sainte-Chapelle the place with the most beautiful gothic stain glass ever. Always try to get in here when visiting Paris. On this particular weekend it was for free, and we had to queue only for 15 minutes. Once we went out this had increased to 50 minutes, so we just caught a quiet time. And off to the Cafe Les Deux Palais, where Reyer and Christine had stopped just two years earlier. And I convinced Sam and Nhi to try the really excellent Irish Coffee, which we might rather call hot Irish Wishkey with a shot of coffee. By that time we were running late by about 1 h behind schedule, due to including the Bank of France spontaneously. Thus we actually missed the time, when the representative salons of the Monnaie had been open for visits. So off to the Hotel d'Avaray the residence of the ambassador of the Netherlands. A nice cute city palace with a nice cute garden. And off to the Hotel de Matignon, the seat of the french prime minister. Once again I was impressed by the palace, though this time they didn't let us enter the office of the prime minister, unfortunately the vast garden was off limits as well. The final stop of our day (well final before dinner) was the Hotel de Soubise, the french national archive. Beautiful baroque and rococo rooms all over the place, as well as ancient documents, dating back as early as the 8th century. Having once more the wonderful tasty beef stew, white and red wine, and convincing Nhi and Sam to try (and enjoy) their first snails. Then we decided to take the bus, which was supposed to bring us all the way over to the Eiffel Tower. Once again the demonstration cut our trip short, and we had to walk the remaining 2 km to the Eiffel Tower. There we got up to the second floor and enjoyed the night views of Paris. While Sam, Nhi and me were down a bit before closing time, Michael and Vivan remained up until the very last lift down. At least this metro line worked without any issues.\\

Palais Royal***, Banque de France/Hotel de Toulouse***, Hotel de Ville***, Conciergerie*, Sainte-Chapelle***, Hotel d'Avaray**, Hotel de Matignon***, Hotel de Soubise**, Tour d'Eiffel***\\

September 22: Sunday\\
Today we planned to leave later, thus I put my alarm 15 minutes later than on Saturday. Unfortunately I was woken up by Vivan's alarm from the room next door. We were all done in time, the ladies claimed they would come down 5 mins later, which turned out to be 25 minutes. Not stopping in a cafe on our way we queued by Palais du Luxembourg, there we saw a cafe just across the road, so we went there in two turns having a bit of coffee, croissants, and omelettes. The Palais du Luxembourg houses the Senate of France. The conference hall is absolutely breath taking, same for the Salon of the Golden book. Afterwards we visited the seat of the president of the Senate, the so-called Petit Luxembourg, cute too. Afterwards Sam and Nhi convinced us to spend a couple of minutes in the park of Jardin du Luxembourg. After a very short stop at St Sulpice we arrived at the second chamber of the parliament, the house of representatives in the Palais Bourbon, with a lot of painted ceilings and walls by Delacroix. The residence of the president of the national assembly is the Hotel de Lassay, very beautiful rooms with nice paintings and gilded wood. While the others went to do some shopping, I finally made it to the State Rooms of the Monnaie, and afterwards to the School of Fine Arts just next door to the Monnaie. The school is home to the Chapelle des Petits-Augustins which houses copies of many pieces of arts, like the Last Judgment from the Sistine's Chapel in the Vatican, and Michelangelo's Moses in Rome, and the copies of the sacristy below the Capella de Medici in Florence. The main auditorium is also full of nice paintings. Then I walked through the courts of the Louvre, through the whole Tuileries Gardens up to the Musee de l'Orangerie, where we all met again to see the amazing Water Lily paintings by Claude Monet. And after a view of the Place de la Concorde we walked to the other end of the park again, since it was only possible to exit there (due to demonstrations), and afterwards searching for another metro line, since line 1 was -- once again -- closed for the whole day. We still made it quite in time for the TGV again.\\

Palais du Luxembourg***, Petit-Luxembourg**, Jardins du Luxembourg**, St Sulpice**, Palais Bourbon***, Hotel de Lassay***, Monnaie*, School of Fine Arts**, Jardins des Tuileries**, Musee de l'Orangerie***

\section{October 6: Gruyeres \& Lavaux}
\label{2019:Gruyeres}

Andrew is back at CERN for a couple of days. As usually he loves to see a bit more interesting stuff around Geneva besides B904 at CERN. This time we decided we should have cheese fondue -- and there is no better place to get it than in Gruyeres.\\

Co-travellers:\\
Andrew: loyal co-traveller since 2013, Andrew is coming over to CERN to work on firmware and electronics. In order to recharge his batteries we get out to the canton of Fribourg this time. \\

Will: US-American, another UCLA grad-student which I take for a trip. Since Will is planning to stick around the area for a bit longer, making an alright impression is one of my goals.\\

Xuan: Chinese and UIC grad student, at CERN for a short while, Xuan also loves herself getting cheese and chocolate as advertised before starting the trip.\\

October 6: Sunday\\
As typically Andrew rents his car from the French side of the airport, so we first get our Vignette by the next gas station, everybody arrives on time and we are ready to go. Once we arrive by the Maison du Gruyere we take the tour, where ``Cherry'', the virtual cow tells how what the cheese is all about, we get three samples of cheese as well. Then we are ready for a round of Irish coffee in the Giger bar. Xuan is a bit freaked out once she realises she is sitting in front of a wall of baby-heads, everybody likes the alien-type ambience. And then we ready to dive into Giger's arts of Mensch-Maschinen, Xenomorphs et al, while also admiring his Oscar statue (no photos). Then we walk down to the Maison du Gruyere, unfortunately off limits for a local event, so we have our pot of cheese fondue at the station restaurant. Then after a short drive we find out we have to wait over 90 minutes for the next tour of Maison Cailler, so we rather pass and get a bunch of chocolate instead, before driving on to Corseaux. Here Corbusier built a Villa by the lake for his parents. The lake view is magnificent as usually, the Villa is well - very functional, rectangular, not really colourful. With guest rooms, a long row of windows on the lake side, typical Corbusier and thus world unesco heritage. Consensus was though that it is a tad underwhelming ``Can you imagine they put that officially on the same level like the Colosseum in Rome''. Since we are so close to Lavaux (also world unescoheritage ) we continue our trip and walk through the vineyards trying to find a Cave which might be open. Although the guidepost in front of the village claims that at least one should be opened on October Sundays, seems not this weekend. While we walk back, one of the men working on the harvesting of the grapes takes pity on us, let's us taste a sample of their white vine as well some of their grapes and tells us to walk back to the next village, since there their local Cave of St-Saphorin vineyard association should be open. And indeed it is, so we get our sample of snacks and a bottle of vine for the four of us, before continuing to walk along vineyards and the lake, before getting back to Geneva where we end the day with pasta.\\

Maison du Gruyere**, Giger Bar***, H.R. Giger Museum **, Maison Cailler*, Villa le Lac*, Lavaux***

\section{October 13: Gorges du Fier \& Bellegarde}
\label{Bellegarde2019}

Lovagny: Gorges du Fier***\\
Bellegarde-sur-Valserine: Pertes de la Valserine***\\

Since Andrew comes over to Switzerland from time to time, we never know if this actually marks the end but so far he stands at one trip with overnight stays and nine day trips or hikes. It's always fun to catch up again.\\

\section{October 26: Glacier de Moiry}
\label{2019:Moiry}

I wanted to do this hike for a while. The trail starts at the foot of the glacier and leads up to the Cabane de Moiry. Unfortunately the bus-line only serves the glacier stop in summer, and it only runs four times a day. End of September the glacier stop is not operated anymore, taking the trip from the lake bus stop extends the hike by over 2 h. I just never convinced someone to go there with me, or to drive there on other occasions. Thus I was very happy when Sasha asked me if I would be up for doing the hike.\\

Co-travellers:\\
Sasha: Ukrainian, unlike last time Sasha brought up the idea of this hike, Sasha works at CERN as most of my co-travellers, and unlike me he has done a couple of more difficult hikes in the alps already. \\

Artem: Ukrainian, also working at CERN, but with an engineering degree, definitely more experienced hiker than me as well.\\

Juno: Indonesian, also a physicist, and as well far more experienced in hiking, mountaineering and anything else which comes with it.\\

Elga: Indonesian, but for once someone not related to CERN at all. He is working in linguistic research at the university of Geneva. Just like me not one of the more experienced mountaineering experts.\\

October 6: Saturday\\
Starting out at CERN by 8 after picking up everybody, a short coffee and shopping stop in Martigny, and another photo stop at a Waterfalls just past of Grimentz, we reached the Lac de Moiry. Amazing clear bright blue water with all the mountain ranges around reflecting in the water. I walked over the dam of the reservoir to the other side and back, also providing nice views of the valley. Then we drove through a tunnel over to the second lake by the former end of the glacier. Nowadays the glacier tongue receded by about 1 km, ending in a third melting water lake. We walked up the side moraine across some left over snow fields and then up the rocky part to the Cabane de Moiry. While the first part was easy and pretty scenic, all changed once we crossed the 2600 m line. Ice was on the path which we had to be careful not to slip too much, about 50-100 m below our final destination the rocks were snow covered. That was though less dangerous than the icy patches here and there. It was straightforward just sticking the other hand to the rocks by the path. Once we reached the hut -- which was closed down for winter already -- we had lunch on ice free rocks just above the hut. The icefall of the glacier looked absolutely magnificent and still impressive. Not quite an hour later, now filled with Baguette, sausages, cheese, and cherry tomatoes we started our hike back. By now the sun had melted parts of the ice, so it all was less tricky than I had feared previously. Once back on the moraine one can even see all three lakes after each other - quite a nice sight. On the way back after a coffee break by Lac de Moiry, we stopped shortly after Montreux to see the sunset over Lake Geneva. All in all a great hike, even allowed myself time to show up at the first Halloween party of the year. My only fear: after hiking that long with that altitude differences, what will my legs say the next day for another planned hike.\\

Glacier de Moiry***, Barrage \& Lac de Moiry**\\

\section{October 27: Gorge de l'Areuse}
\label{2019:gorgedelareuse}

I suggested this hike already to Andrew just a couple of weeks ago. Christine originally had the idea to do a hike through the gorge, which she found in a hiking book about hikes and stops in breweries along the way. It was also another nice October weekend, and the last weekend of the SBB action of getting day tickets for two for the price of one. Since I was invited by Sasha to join his hike to Moiry on Saturday I convinced everybody to rather go on Sunday. We wanted to start in Boudry and check if we can make it all the way to Boudry\\

Co-travellers (all US-Americans):\\
Grace, Christine, Sam: all know what to expect going on trips, and going on hikes with me, we all were well prepared with our lunch portions, as well as beer provided by Sam.\\

Vivan: After a first city trip, Vivan is ready to join us for a hike too, little did she know we would take longer than 5 pm to get back to Geneva.\\

Dylan: This is the first of my trips, which Dylan joins, considering it is a hike, I hope he will enjoy it.\\

Will: Provided the first trip went to his liking, Will joins for a second time.\\

October 27: Sunday:\\
Yes, I knew it could be tricky doing back-to-back hikes. My thighs did indeed hurt already going into this hike, after slipping the day before, the right shoulder had still some complaints as well, but considering that the first part of the hike should have a steady but slight climb, it should be alright. Everybody got up in time, but little did we know that the TPG bus coming from France would break down due to technical problems, and although Christine drove to Cornavin, she just made it 2 mins too late to catch the train. So more coffee, and catching the train an hour later we have short photo stops at the vineyards before getting into the gorge. Which is full of colourful leaves, a foggy October day, but autumn colours brighten up the day. Rock walls left and right, and we in between hovering tens of metres above the river. The gorge opens up a bit later on, but still it is full of little waterfalls and nice colour full trees. On the way we still try to branch a bit off into the Creux du Van forrests, sometimes getting a bit lost on the way. We stop for our lunch break at some point, before climbing up the switch-backs of a road until we reach the actual start of the proper Creux du Van hike. Unfortunately the 1 h delay caught up on us on that point, so we decided rather to call it a day and shift the Creux du Van adventure to another time. Instead we stopped watched the sun set over Lac de Neuchatel. Sam managed with only a couple of minutes to spare to get a little tree, Vivan was unfortunately not served quick enough, so she had to leave a flow pot in despair. And we arrived in dark wintery Geneva after 90 more minutes.\\

Gorge de l'Areuse***\\

\section{December 29, 2019-January 5, 2020: East Germany and Poland}
\label{2019:GermanyPoland}

After Christmas my dad and I decided that we should spend a couple of days elsewhere. We checked options from northern Italy to Malta and Dubrovnik, and then we decided to maybe stick to regions a bit closer. Both my dad and I thought it would be nice to see Dresden again, particularly after the Residence was now far more close to completion than years ago when we visited already. Since Dresden is quite close to the border we considered the Czech republic or Poland. My dad decided he rather preferred seeing a new place, thus we ended up in Poland. Considering this is the first trip of us without our mum in more than a decade it might be end up to be a bittersweet trip.\\

December 29: Leipzig and Dresden\\
Starting out just a bit after 7 am we stopped for a short breakfast around Stuttgart before driving all the way up to Leipzig crossing over four German states -- Baden-W\"urttemberg, Bavaria, Saxony-Anhalt and Thuringia to end up in Leipzig in Saxony. The Monument to the Battle of the Nations (in German: V\"olkerschlachtdenkmal) was erected to commemorate the epic battle between Napoleon's French troops and the coalition of Prussia, Austria, Portugal, Sweden, Spain, Russia, the United Kingdom, and the confederation of the Rhine. It was a crucial loss for Napoleon before his first ousting as French emperor. The monument is gigantic with several layers, a crypt a gallery with giant over life-size statues. The highest platform is a bit more than 60 metres up, so be prepared to walk up quite a bit (there is an elevator which can be used though, but even in January the lines were long). The view of the city is nice, but it is a bit far away from the old town so you only see the highest buildings like the new city hall. Then we drove into old town, where we watch the church of St Nikolai, where the first marches of the Peaceful Revolution of 1989 against the communist party of East Germany started. Besides the historical value, the church is a nice classical-baroque building. On our way to St Thomas we passed by the nice baroque old stock exchange building and the old city hall -- an impressive old Renaissance building, one of the largest in Germany. Thomaskirche is a gothic church, itself not that much of an impressive building, but it was here that Johann Sebastian Bach composed and debuted all his famous pieces of art, and the church choir is still one of the most famous in Europe. After a short further walk we passed by the gigantic new city hall, built during the times of the German Empire and then we saw the new modern interpretation of the old university church. Having arrived in Dresden we went to the Elbe river to enjoy the panoramic view of old town, which still offers a view very similar to those enjoyed and put on canvas by Canaletto centuries ago. We crossed the river on Augustus Bridge (heavily in renovation at this moment), and had dinner by Neumarkt, enjoying the outside views of Frauenkirche.\\

Leipzig: Nikolaikirche**, Thomaskirche*, V\"olkerschlachtdenkmal***\\
Dresden: Neumarkt** \& Canaletto-View***\\

December 30: Dresden and Saxon Switzerland:\\

Dresden: Semperoper***, Frauenkirche (with dome)***\\
Saxon Switzerland: Bastei Bridge***, Neurathen Castle**, Golden Rider**\\

December 31: Dresden:\\

Dresden: Residenzschloss***, Zwinger*** (Porcelain Museum***, Art Gallery***, Mathematical-Physical Museum**), Hofkirche**\\



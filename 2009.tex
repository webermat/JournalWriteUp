\chapter{Year 2009}
\label{2009}

\section{January 3: Reichenau \& Konstanz}
\label{2009:Reichenau}

The first trip of 2009 in chilly Germany with my parents and my younger brother lead us first to Constance with its gothic-romanesque minster, which had been back in the day the cathedral of the diocese of Konstanz, one of the most prestigious and largest in the German area in medieval time. The interior, particularly the choir area has been remodelled multiple times, the last time in a classicist manner. Quite close to Konstanz is the monastery island of Reichenau, which is particularly famed for its three Romanesque churches. The minster is a typical large basilica with three naves and two choirs. The apsis of St Paul is still decorated with late medieval murals, while its nave has been redecorated in Baroque style. The highlight of the island is though St Georg, which has well-conserved murals from around and before the year 1000 all over its nave and the original wooden ceiling. The mural from Baroque times on the west work has faded considerably. For reasons of conservation in summer people can only visit the church on particular time slots with guided tours, while the church is open continuously in winter time (and yes even during covid). Due to the low temperatures the shores of the Bodensee had frozen over as well, so one was allowed to walk on the ice a bit.\\

Konstanz: M\"unster****\\
Reichenau: St Paul****, Kloster****, St Georg*****

\section{February 1: Nyon}
\label{2009:Nyon}

Nyon: Chateau***\\
Lausanne: Cathedral****

\section{March 21: Switzerland}
\label{2009:Switzerland}

Basel: M\"unster****, Fribourg: Cathedral***

\section{April 11: Rheinfall \& Rheinau}
\label{2009:Rheinfall}

Rheinfall*****, Rheinau***

\section{April 25--April 30: Madrid DIS2009}
\label{2009:Madrid}

The DIS2009 (Deep-Inelastic Scattering) conference wanted to hear about the prospects of the LHC regarding event-shape measurements at the LHC. Since that was the topic of my PhD thesis and we all hoped that we indeed would get collision data this year, my boss nominated me for that talk. I was selected and got myself a hotel by Madrid Chamartin. The conference itself took place at the congress centre just opposite of the Bernabeu stadium of Real Madrid. Having been in Madrid just the year before for the first time, I didn't stay for any additional days.\\

April 25: Saturday: Madrid\\
Having arrived a bit earlier I decided to go to the city centre, visiting the Almudena cathedral. The cathedral took about 110 years to be finished only in 1993. The outside is of a neobaroque style while the interior leans more towards a gothic outlook. The ceiling is very colourful, the same hold for the square cupola with a large traditional style apse painting. The crypt was free to visit back then, built in a neo-romanesque style. Nowadays you are FORCED to pay the official proposed ``Donation'', else you are refused entry.\\

Madrid: Almudena Cathedral*****\\

April 29: Wednesday: Getafe \& Toledo\\
This was the free afternoon, I decided to go to Toledo. Since that train wouldn't leave until later I chose to do a short trip to the outskirts of Getafe, to see the cathedral there. The retabel was really nice, the church itself nice too, particular considering it had been a normal city church for the longest time. Back by the Atocha station i still had time to walk through the palm gardens which have been set up in the middle of the station hall, pretty unusual and unique. Once I arrived in Toledo after a short train ride, I climbed the stairs up to the Alcazar and then walked along nice little roads to the cathedral, on of the largest of the country. Back then photography was not allowed (that changed in the meanwhile). The decoration is breathtaking, from side altars, a large main altar (though that one can only be viewed through a giant screen) and the fantastic baroque El Transparente altarpiece on the outside of the choir screen, covering the full height of the cathedral with statues literally climbing up the altar walls up to a window. The sacristy is also pretty nice, filled with several paintings by El Greco. Another church which I enjoy visiting in Toledo is the large baroque Jesuit church. There you can climb on the roof with nice views of the old town. On my way back I walked up a little hill top to see the Castillo de San Servando, which overlooks the old down of Toledo on the other side of the Tajo river.\\

Getafe: Cathedral****\\
Madrid: Atocha Palmgarden****\\
Toledo: Cathedral*****, Jesuitchurch****, Castillo de San Servando***\\

During the evenings I had a couple of dinners in the old town of Madrid, for example the Plaza Mayor, and walked through the little parks by the Royal palace. The conference dinner was at a fancy roof top restaurant but other than that I just flew home on the last day of the conference, so no time to see anything else.

\section{June 6: K\"ussaburg}
\label{2009:Kuessaburg}

Once again I went home to see my family again, and since there is only so much variety of very near sights, we walked up to the ruined castle of K\"ussaburg once more. This time of the year all the walls were covered in fully green ivy. The views were as usually nice both of Klettgau and the Rhine valley, particularly since this time dark rainy clouds (before the storm) created a pretty eerie athmosphere. We walked around the inner and outer moat for the complete experience, always nice to return.\\

K\"ussaberg: K\"ussaburg****

\section{June 23--July 8:CTEQ09 \& New York City}
\label{2009:US}

June 23: the flight\\
I didn't want to take a bus from Chicago to Madison, so I decided to fly instead via Paris and Cincinnati to Madison. Having arrived in Cincinnati I had to spend quite boring 6 hours at the airport, and then about 30 minutes before our flight was supposed to depart, Delta cancelled our flight due to a technical problem. We were told we could talk to the call center or queue by the counter. I decided to go to the counter, just seconds later people found out the call center didn't work anyway. I was told I could get on a flight to Chicago O'Hare, then taking the bus from there (exactly what I had wanted to avoid originally), but what can you do. I was told to rush the flight would leave in 20 minutes, I still hoped for my luggage to get on board, at least they were happy to hear an Air France checked bag would look very distinct. Three more guys were from my original flight were on the one to Chicago. Surprisingly my bag did make it on the plane so once I got it, I got on the bus to Madison, but when the driver asked me where I wanted to get off, I just had no idea. Seems there are three stations, and since I didn't plan being on a bus, I didn't inform myself about where bus stations are located at. I told him I would get somewhere close to the University, so I was told to get off at Union station, the last one of this bus. About 3 1/2 h later, by this time I had been awake longer than 24 h, we finally arrived at Union station. I got off, got into a taxi with 2 other folks, told the driver where i wanted to go, he drove for about 45 minutes, I got off, gave a generous tip, and finally checked in (having told them I would arrive about 5 h earlier), and finally got some sleep. Waking up the next day I walked to the auditorium ONLY to realise that the bus station was only 500 m or a 5 minute walk away from the hotel. Since then I have sworn to myself never to get on a taxi again, unless there is absolutely no other possibility.\\

June 24 \& 25: Wednesday \& Thursday: Madison\\
Madison didn't feel like a big city to me, it clearly does have a large amount of citizens, the Madison area has roughly the same inhabitants as the canton of Geneva, but they all seem to be scattered over a much larger surface. Old town is also only a couple of streets and it felt like the giant Wisconsin State Capitol rises out in the middle of like nowhere. it is a neoclassical building erected between1906-1917 with four wings and a central 57 m high dome, with pediments on each of the four front facades.The rotunda has two levels plus the dome on top of it, where the upper level is decorated with several dark pillars with golden leaved capitals and four large mosaics on the vaults. Since no building in the vicinity is allowed to surpass the capitol's heigh there are not many high rises in Madison downtown. I tried to find the cathedral of Madison as well - unsuccessfully and read up back home in the hostel that indeed the cathedral had been burnt down just a year previously and the remains were torn down thus no wonder I only saw green grass where google maps told me to look for it.\\
The following day I returned to the State Capitol for a couple of more photos, this time also going into the Senate assembly hall. I also did enjoy some beers out with my colleague J\"urg, Andrea, Cosmin, and other people I had met at the CTEQ summer school, particularly sitting in sunset by the calming water of Lake Mendota.\\

Madison: Wisconsin State Capitol****\\

June 28, Sunday: Chicago:\\
After having sat on the bus from Chicago to Madison I wasn't scared to do the same bus ride again. Unfortunately everybody else had seen Chicago at that point already, thus I was all on my own. Once I arrived in Chicago I tried to get up on Willis Tower, but I was told the observatory would open late due to heavy winds at that time. Thus I walked across the loop to the Shedd Aquarium, one of the best aquariums I have been to, with turtles, sharks, see horses, artificial coral reefs (yes also with clown fishes), frogs, leguans, crabs, sea stars, lobsters, just everything, and also Beluga Whales.\\
I had a short lunch before taking a ferry across Lake Michigan to reach Navy Pier, where i had another drink in the beergarden walking through the indoor gardens there, ending up on the top of John Hancock Center and enjoying the skyline from up there. I passed by the cathedral, which was though closed for renovation works. I walked a bit along the Chicago river taking snapshots of Trump Tower (who would know back in 2009 what we would be up to in 2016, that name didn't age well).\\
Having arrived by Willis Tower again, the observatory was open, and clearly I had to go up and enjoy the Loop are of Chicago downtown from a different perspective while admiring the wide sea of house beneath me to the west and south. Then I sat in Union Station for another half an hour waiting for the bus.\\

Chicago: Willis Tower*****, John Hancock Center*****, Navy Pier***, Shedd Aquarium*****, Millenium Park**\\

July 3: New York City:\\
At least this flights went without major issues, the flight into Detroit departed and landed on time. On my flight to NYC I was bumped up to business class (well better snack for me that is), but then our landing was delayed by over an hour. A light rail and then public train to Philadelphia Station later soon after I arrived by the Hilton Times Square hotel in a gigantic room (for my taste) with a nice view of Empire State Building. But once in New York you have better things to do then stay in a hotel. Thus I walked straight over to Times Square, iconic, and also a wonderful feeling to be in a busy metropolis. By that time the sun was setting, and I walked over to St Patrick's Cathedral, another neo-gothic large church with nice stained-glass windows. I had already booked tickets for the Empire State Building, but unfortunately I was not really alone on the Eve before Independence Day, thus over an hour wait later I arrived by the platform, and I was blown away. Midtown Manhattan is just breathtaking, a gigantic wonderful shiny city on the bottom of your feet from up there. Also the streets which go like bright shiny veins over the island and the illuminated bridges looked just wonderful. And all below an almost full moon. I did enjoy it a lot, no matter the wait.\\

New York City: St Patrick's Cathedral****, Empire State Building*****, Times Square*****\\

4th of July: New York City:\\
Independence day 2009 in New York City, sounds exiting, isn't it. Since hovering around in NYC all on my own sounded like a bad idea, I already tried beforehand to find someone local to show me around, after all what is the internet there for. And I found a wonderful guide, even a local artist, Kendal, who thought it would be fun to show a nerdy physics PhD student the wonders of the town. First I made my way to Grand Central Terminal admiring the beautiful train station and hub of public transportation in NYC. The Main Hall is indeed very impressive, also from a European perspective curious to see the many large US flags hanging all over the hall, maybe also just due to it being the Independence Day weekend. \\
And this is where I met up with Kendal. As I expressed my interest in skyscrapers before he brought me to the lobby of the Chrysler Building, another one of the iconic NYC skyscrapers which had for a short moment been the tallest tower on Earth before it was surpassed by the neighbouring Empire State Building. While it doesn't have an observatory, the lobby itself was really nice, with art deco style marble walls, murals of the Chrysler building on the ceiling, maybe make that extra mile to see it. Then Kendal and I had breakfast at Bryant Park, and we had to take photos of Empire State Building as well. We stopped by Madison Square Park to get a glimpse of the triangular shaped 87 m high Flatiron building from 1902, which some consider the first skyscraper ever built. After one more snack at Union Square we walked past New York City Hall over to Ground Zero, where construction of One World Trade Center had started. \\
Then it was time to say goodbye to Kendal for now, while I walked across Brooklyn Bridge with nice views of neighbouring Manhattan Bridge and freighters on East River. Passing by St Paul's Chapel (nothing too special), Trinity Church is another well-known neo-gothic church in NYC, but if you know European churches it all feels a bit underwhelming. Walking past Wall Street, Federal Hall, the Standard Oil (nice art deco) building and Fort Clinton I ended up at Battery Park where I enjoyed some moments. I asked people where this year's fireworks would take place, since I had read it does sometimes happen over East river, at other time over Hudson River, I was told it would be by Hudson River, thus I moved already over there, where many folks had already gathered by then.\\
About two hours lighter boats paraded over the river shooting out fountains of water during sunset. About an hour later the fireworks started. Clearly the fireworks were outstanding and gigantic, spanning along the river. I was impressed even after being spoiled by the Fete de Geneve fireworks already the year before (and the following years to come). There wasn't a display of music coming with it but all the formations and patterns and colours were breathtaking, sure up there with the best.\\

New York City: Grand Central Terminal****, New York Public Library***, Brooklyn Bridge*****, Manhatten Bridge****, St Paul's Chapel**, Trinity Church***, New York Stock Exchange**, Battery Park**, Independence Day Fireworks*****\\

July 5: New York City:\\
On my second day I had book the morning sunset ticket of the Rockefeller observatory, the Top of the Rock. Unfortunately unlike for the Empire State Building glass walls surround the terrace nowadays, as high as people size. Clearly if the sun is shining against them or they are not cleaned properly (and who would do that every day), the view is a bit obstructed. On the other hand this is the best view you can get of the Empire State Building, and also the iconic birds-eye view of St Patrick's Cathedral, and not to forget the view of Central park which just gives you a feeling why it is called the green lung of NYC. After a short visiting intermezzo at St Patrick's Cathedral, I took the Metro up north to see St John the Divine. This neogothic anglican cathedral is in fact still unfinished, missing the top of its towers as well as the southern transept. The church is with a length of 183 m and a height of 54 m one of the largest in the world.\\
 Afterwards I visited the American Museum of Natural History withs its Meteors and crystal collection as well the dinosaur collection. I then had a walk through Central park with its forests, ponds, and squirrels, before visiting the Metropolitan Museum of Art. The MET has tons of things to see, the screen of the Spanish cathedral of Valladolid, period rooms all the way from Paris, the egyptian temple of Dendur, islamic art such as the Jain Audience Hall, sculptures from Canova, Mesopotamian gates, and many paintings, such as the Cypress painting by Van Gogh, and I finished the day with a further walk across central park. And then came the second part of the Top of the Rock ticket, the night view to admire the Empire State building in its epic Independence Day outfit of red, blue, and white.\\

New York City: Rockefeller Center*****, St Patrick's Cathedral****, Cathedral St John the Divine****, American Museum of Natural History****, Central Park****, Metropolitan Museum of Art*****\\

July 6: New York City:\\
Featured in several movies the New York Public Library is one of the largest of its kind, with nice coffered and painted ceilings the reading rooms, galleries, and entrance halls. I was looking forward to seeing the Headquarters of the United Nations, bought a ticket which listed 5 different places I could see, only to be informed after getting through security that due to ongoing meetings solely the main assembly hall would be open today. Clearly they either could have given us a ticket at a reduced price or told us to come back maybe a day later. After that disappointment I admired the Museum of Modern Art, a fantastic collection including the who is who each with multiple paintings, a must to visit for any lover of modern paintings, you might have heard of Van Gogh's Starry Night, Warhol's Campbell's Soup Cans, Picasso's Demoiselles d'Avignon or Matisse's Dance (and many many more). Since there wasn't much more to do I decided to see a couple of churches instead, like another neo-gothic church of St Thomas (beautiful sculptured high altar), or the neo romanesque/gothic style mix of St Bartholomew.\\
When in a big city you want to do something special, for me that was watching a movie in IMAX, which didn't yet made it across the ocean yet, the only movie they showed at IMAX at that point was Transformers II. The clerk at the cinema told me if I really want to pay that much money for IMAX, considering the price was only half of what a normal ticket in Geneva was clearly I wanted to do that - also got my giant cup (for sure about 1 l inside) of coke and hot dogs for a good real American cinema experience - well besides the fact that it still was Transformers II, oh well you know what you get by Michael Bay.\\

New York City: New York Public Library***, United Nations**, Museum of Modern Art*****, St Thomas Church****, St Bartholomew****\\

July 7: New York City:\\
When in New York City you might want to visit Liberty Island. Clearly I wasn't the only person with that idea. Consequently all tickets to reach the island from Manhattan had been booked. Nevertheless there were several tickets available to do the same trip from the New Jersey side. Thus I took a ferry from Financial District over to Jersey City. Since I had a bit of time I walked into the Central Railroad of New Jersey Terminal, an abandoned large train station where I got myself breakfast and coffee before taking a ferry to Ellis Island. There the museum elaborated on former times of immigration when people were scanned and their immigration forms processed at Ellis Island Terminal, and then the Ferry took us around Liberty Island. Since I had booked the platform ticket I climbed the State of Liberty monument up to the feet of the statue, from there I had a view up the inner skeleton. The tickets to walk up to the crown had been sold out way in advance though. And back to Jersey City and then to Ground Zero it was. \\
That afternoon I met up with Kendal again and we walked through the Meat Packing District along the old refurbished track which make up High Line Park. A very beautiful way to revitalise the area, besides adding green and refreshing spaces in that part of the city, also art installations are scattered throughout the park. Clearly I had to finish the day taking some night photos of Times Square.\\

Jersey City: Central Railroad of New Jersey Terminal***\\
New York City: Ellis Island****, Statue of Liberty****, High Line Park*****\\

July 8: New York City:\\
Seems I had not enough planned for the last day, thus I just did a curiosity museum which was just next to the hotel and advertised by the hotel itself, in the end it delivered also nice items like ivory carved ships from China, wood-carved ships from Japan, parts of the Berlin Wall, Napoleon's death mask among other things. Having planned nothing more I just went to airport already now where I was informed that it was far too early to receive my bag, thus I had to walk around with that clunky suitcase or take it with me while having dinner. The flight in the Air France Boeing 747 was pretty uneventful besides flying into Charles de Gaulle with the best view of Paris I ever had from a plane, unfortunately stupid me didn't kept the camera out of the hand luggage area.\\

New York City: Curiosity Museum**

\section{July 26: Valais}
\label{2009:Valais}

One more weekend in Switzerland which I decided to spent in the Rhone valley. I started out in St Maurice. In this village you find not only an ancient abbey and one of a couple of mountain forts of the Swiss army, but also a nice little cute cave with no flowstones, but with the largest cave waterfall within Switzerland which had still quite a bit of water even in July. A few stops later I arrived in Sion, an old bishop seat with an old but pretty plain cathedral. In fact I was a bit disappointed, but then a couple of visits later I warmed up to it a bit again, after all the altar piece is alright, but there is hardly any other decoration available. On two hillsides close to old town are two castles, one which is more or less a fortified old basilica with one of the oldest functioning organs in Europe, and a ruined former castle of the bishops of Sion on Tourbillon. This time I managed to get inside unlike last time where doors were shut basically in our face. In St Leonhard the largest underground lake in the Europe can be found. One wall of a cave collapsed and the cave filled up with rain water after a while, still nice to have a cute little boat tour on it. Only at the end of the cabe you have permanent lights installed, but on the boats are a couple of spotlights which did give enough lights for decent pictures, even with my old digital camera which for sure isn't anywhere close to the quality of mobile phone cameras or even an SLR. And last but not least the castle of Stockalperpalast, a private palace, which is in fact the largest Renaissance palace in Switzerland with three big towers and a nice courtyard, with its arcaded walks it reminded me of Italy. Stockalper in fact also made his money on the trade route of the Simplon leading after all to Lombardy, so for sure the inspiration came from places south of the alps. Nowadays the local court and the city hall of Brig are situated here, only on guided tours the general public can see some of the rooms nowadays.\\

St Maurice: Grotte de Fees****\\
Sion: Cathedral**, Tourbillon****, Valere****\\
 St Leonhard: Lac Souterrain****\\
Brig: Stockalperpalast***

\section{August 16: Wutachschlucht}
\label{2009:Wutachschlucht}

One of the longest gorges and canyons in Germany is the Wutachschlucht, where the river dug a deep valley of up to 80 metres. There are plenty of trails which altogether are claimed to be 50 km long. The gorge has several side gorge and multiple entries which can be also done in multiple short hikes. There are plenty of waterfalls around, on one spot the river even disappears and reappears again after about 20-30 metres in hot summer at least. There are also slot canyon type places from time to time (small, but they do exist). In ancient times the river continued its flow to the Danube, but digging so deep the connection to this valley was lost and nowadays the river flows into the Rhine. From the former river valley a creek falls down in several cascades into a little side canyon of the Wutachschlucht, but that side part I've only done as late as 2021.\\

Wutach: Wutachschlucht****

\section{August 23: Zermatt}
\label{2009:Zermatt}



Zermatt: Gornergrat*****

\section{August 30: Aareschlucht \& Tr\"ummelbach Falls}
\label{2009:Aareschlucht}

One of the most beautiful canyons in Switzerland is the Aareschlucht. Coming from Bern the train ride itself along the Thuner See and the Brienzer See is very nice itself alongside mountains with the clear blue waters of lake Thun and the shiny green waters of Lake Brienz. The Aareschlucht itself is a river slot canyon which is than 1 m wide at its tightest spot. Throughout the trail you can touch the opposite walls several times, the trail leads through tunnels as well usually across wooden planks, and the gorge has a small waterfall as well.\\

For the second part that is waterfalls galore. Most of the glacier water from the northern side of Jungfrau, M\"onch, and Eiger leads down to the Tr\"ummelbach Falls. The falls themselves (10 of those) are in fact hidden within the mountain, cascading down in 10 pieces. Most of those are very impressive, you might catch a couple of rainbows with the bits of sunlight illuminating the water vapour. Along your steps down (an elevator inside the mountain brings you up to the top most fall beforehand) you sometimes have an open wide view of the idyllic Lauterbrunnen Valley, which already inspired Tolkien during his holidays there to come up with certain landscapes of Middle Earth.\\

Meiringen: Aareschlucht*****\\
Stechelberg: Tr\"ummelbach Falls*****

\section{September 1--September 5: Innsbruck SPS \& OEPG 09}
\label{2009:Innsbruck}

September 1: Innsbruck\\

Innsbruck: Hofburg****, Hofkirche*****, Dom*****, Jesuitenkirche***, Basilika Wilten****, Stiftskirche Wilten****

September 5: Innsbruck\\

Innsbruck: Schloss Ambras****, Nordkette*****

\section{December 18--December: Karlsruhe et al}

December 18: Karlsruhe\\
Schloss Karlsruhe****\\

December 19: Mainz \& Frankfurt\\
Mainz Dom****, Frankfurt: Bankenviertel****, Dom****, Paulskirche***\\

December 20: Speyer, Mannheim, and Worms:\\
Speyer Dom*****, Mannheim: Schloss****, Jesuitenkirche****, Worms: Dom*****\\

December 21: Heidelberg:\\
Heidelberg: Schloss*****, Heilig-Geist-Kirche****, Jesuitenkirche***
\chapter{Year 2015}
\label{2015}

\section{January 11: Chillon}
\label{2015:Chillon}

And off to the first trip of the year: Chillon. Time to get out once again this time with a new co-traveller: Siyi\\

let me introduce you first to Siyi:\\
Chinese, grown up in Kunshan close to Suzhou and Shanghai. Moved to Munich after getting her PhD in Los Angeles. Her first trip around the lake Geneva region (and disappointed about the missing authenticity of so-called Chinese food in Geneva. Knows what she wants on her trips - and being very charming she has a talent to talk you into wanting it as well.\\

Chateau Chillon is a castle built on an islet just a few metres of the shore of Lake Geneva a few km away from Montreux. The Bernese representatives resided in the castle up to the 18th century. Afterwards the castle has been used as hospital and prison. Nowadays the Chateau is one of the most visited historical buildings in Switzerland. After our visit in Chillon we walked over to Montreux and had lunch there after a short walk to the Freddy Mercury statue by the harbour. Then we want on a quest for the closest Boba tea shop (and made it in the end).\\

Chateau Chillon*****

\section{March 14-March 15: CERN Physicist Field trip to Milan and Monza}
\label{2015:Milan}

This time on board: A whole bunch of physicists, all first timers on travel with me. How did we all end up going on that trip? I actually don't remember anymore. I remember though that after a year long renovation, the Villa Reale of Monza opened its gate to visitors again, so I was ready for some Milan again. For everybody else,  I guess it was a combination between - we need to go somewhere and me telling everybody how awesome Milan can be. Getting to Milan is only a 4 h train ride, which we can easily to just after work, or really early in the morning. This was the first time I stayed overnight in Milan, in the IBIS hotel by Milano Centrale.\\

Co-travellers:\\
Kennedy, Devin, Aaron, Nate, Laser, Christine (all US-Americans), and Can (Chinese), all physics grad students working at LHC experiments for US institutions (both at CMS and ATLAS). For all of them this was the first trip with me. So far this has been my only trip with most of them, let's see when I have time to change that...\\

April 14: Milano \& Monza\\
Aaron was on the 5:42 train to Milan with myself, while everybody else had left on the last train of the day Friday evening. We met up by the Duomo, one of the largest churches in Italy and Europe, largely built in gothic style. The others had a walk by Castelo Sforzesco previously. Those were the times, when it was easy to enter the cathedral without needing to pay for it, although without the long security checks by the gates (these take sometimes about half a minute per visitor, so incredibly slow). The cathedral is impressive, particularly the little statues and carving on the columns. The choir windows are very precious as well. After our visit of the cathedral we took the train out to Monza. There we had pizza, where everybody else urged me to take the ``German'' W\"urstel pizza (sausages on pizza), which I had rarely seen in Germany though. Then we made our way to the Villa Reale, a former royal palace with vast gardens just outside of Milan. We had booked a guided tour of the state apartments. Most of these were in Empire style, the tour also included the bedroom, where king Umberto I died after being stabbed. Then we walked through the arts exhibitions in the former private quarters of the king. After spending a bit of a chill time in the gardens, we took the train back to Milan, where Devin and Kennedy had a dinner date, while we others had dinner at another place. We had all booked tickets for a Van Gogh exhibition in the Royal Palace. Unfortunately I had to realise that "meeting at the Palazzo Reale" was not really specific. Since I had booked all the tickets, I got them after handing in the voucher, but Devin and Kennedy were nowhere to be seen by the museum entrance up the stairs. They waited outside on the courtyard instead, after 15 minutes or so, they decided that maybe we messed up and they should just go inside, where I waited with the tickets. Clearly they were not pleased (understandably) about us being unspecific of the place where we were supposed to meet. Anyways we all made it to the exhibit in time, and could admire Van Gogh's evolution, looking at his self portraits, hay stakes, and the potato eaters. And we also got some night photos out of the Duomo by night.\\

Milano: Duomo*****, Van Gogh Exhibition****\\
Monza: Villa Reale****\\

April 15: Milano:\\
After we had seen the Villa Reale of Monza the day before and the nice part of the Palazzo Reale, we started the morning with the state apartments of the Palazzo Reale, which housed another free exhibition. Besides enclosing the roof again, the former ballroom is left like it was after the WWII air raids in 1940s. It is still a memory of how destructive wars can be, although what is left is still very impressive. Afterwards I brought people to San Bernardino all Ossa, which is now the resting place of bones, skulls and skeletal remains from a nearby given up cemetery. Then we visited the modern art gallery in the Villa Belgiojoso Bonaparte, the branch of the Italian Gallery in Palazzo Beltrami \& Palazzo Amguissola. The house museum of Palazzo Bagatti Valsecchi is a relict of its time, when it was fashionable to recreate what people thought was amazing about medieval art, so mock up banqueting halls, chapels in neogothic style, armories. Once we arrived by our next stop of Villa Necchi Campiglio, Christine realised she still had the audio guide of the previous Palazzo on her, thus some of us rushed back as quickly as we could, and just made it back in time for the next tour to start. Villa Necchi Campiglio is a more modern place started out in an Art Nouveau core. Some members of the royal family had lived there at some point too, the pool and the garden were nice too.\\

Palazzo Reale*****, San Bernardino all Ossa*****, Villa Belgiojoso Bonaparte****, Palazzo Beltrami \& Palazzo Amguissola (Galleria d'Italia) ****, Palazzo Bagatti Valsecchi***, Villa Necchi Campiglio****

\section{March 27--April 6: Spain -- Madrid and Andalusia}
\label{2015:Spain}

How this trip came into existence:\\
Easter is always a nice time to travel, 2 extra days without taking any additional days off is always a plus. Originally I wanted to see Madrid and the surrounding region once again.  This time maybe seeing the parliament, museums or palaces I hadn't been to previously etc. Asking around for potentially interested co-travellers Madrid alone got a mild response. People seemed to be more pumped about Andalusia. After consulting Renfe, seemed it would be easily feasible to cover both, after all just around 2-3 h in a train isn't too long. But if I wanted to see everything I personally deemed exciting, in total I would cover 10 days (just needing to take 4 days off) -- jackpot for me, but too long for anybody else. So time for a new adventure: Exchange of travel buddies on the way.\\

Co-travellers: \\
Manuel:\\
US-American, but born and raised for some years in Mexico. Thus fluent in Spanish (always helpful when going to Spain). Up to speed with my walking speed, doesn't complain if things on the plan should fold. Hiking speed is faster than mine, so lacking behind quite often there (Sorry). Loved his Spanish Telenovelas, which I didn't understand at all, thus I didn't bother. \\

Eric:\\
Japanese-American, but born and raised in California, so you can guess it, also Spanish speaking. Almost as crazy as myself when travelling alone, thus ideal to go on trips of my caliber. Always in a good mood, and up for adding even more things to do, should the time allow. Others complained about possible snoring at night, but since I am a catastrophe in that respect myself, seems I never found out about that certain aspect. Tends to listen to music when falling asleep. Since I automatically try to guess songs, even with turned down volume, sometimes just by the baseline, that can give me troubles to start falling asleep. \\

March 27\\
Since the house of deputies opens up really early, there was no time to waste, so getting on an Iberia flight on a Friday night. In order to be quick getting the luggage next evening for the bullet train to Cordoba, I booked a hotel pretty close to Atocha station.\\

March 28\\
Palacio de las Cortes - the house of deputies in Spain. The only non-Spanish speaker around in a Spanish only tour -- the only thing I was told in English was ``No Photo''. Having had no Spanish at school the information I understood was limited, but I got some basics out nonetheless. The parliament itself was nice, but (the free tour) was done pretty quickly nonetheless. By the time I got out my travel buddy Manuel made its way from the airport to the parliament. After a short Spanish tour of the Baroque Palacio de Linares (also called Casa de America and ``No Photo'' once again) and a short stop at the Palacio de Cibeles, which is now transformed from main post office -- bullt in modernist style -- to the main government building of the city, we ended the touristy program at the Prado. Most impressive here were Goya's paintings, for the other paintings, after being around for about two hours it got repetitive at some point for my taste -- and wherever you go to in Madrid -- No Photos. After having dinner and wine by the palm garden of Atocha we jumped on the train to Cordoba.\\

Madrid:
Basilica de Jesu de Medinaceli***
Palacio de las Cortes****,
Palacio de Linares***,
Palacio de Cibeles****,
Prado*****\\

March 29 Welcome to Andalusia\\
And our first city in Analusia: Sevilla. I planned we could have a relaxed breakfast at the  Plaza de la Encarnacion. Seems though we were even too early for the cafes? openings. Once the first one finally opened we had just short of 30 mins to spend since we wanted to be at the Palacio de Lebrija just when it opened. This palace is only open on some days of the week, particularly the more private quarters on the upper floor. For this one photos are not allowed and we even managed to get an English tour. The ground floor was in fact more of a highlight, since almost all rooms and halls are decorated with actual roman mosaics ? really amazing. I expected to see a bit more on our next stop, the Archivo General de Indias, but hardly anything was on display from the archive itself. But they had other interesting pieces on display in the library, e.g. early editions of Newton?s write-ups. So on to the Alcazar, which is still used by the Spanish kind as royal residence. The state rooms are on the ground floor, some rooms going back to Moorish times. The palace itself is surrounded by quite extensive gardens. And then we took another tour (no Photos) of the upper floor, which is the floor where the royal family resides when in Sevilla. We even were allowed to have a sneak peak into the living room of the king ? what a gigantic flat screen. Since it was Palm Sunday and thus the begin of easter week the whole city was populated with people wearing Capirotes, the cathedral was off limits for non residents as well. The final stop, after a short D-tour to the Plaza de Espana, was the Casa de Pilatos. A bit off the hustle and bustle of the cathedral square a private residence built in Mujader style (this time with a Spanish tour of the top floor ? and the usual no photos). Seems in Sevilla ground floor halls were always meant for representation and thus to show-off and photos are allowed. The upper floors are the private quarters which nowadays means no photos. And then we participated on the several processions, where each church sends out its preferred statue to a certain square where a religious service is being held then.\\

Sevilla:
Piazza de la Encarnacion****,
Palacio de Lebrija****,
Archivo General de Indias***,
Alcazar***** (Second Floor*****),
Plaza de Espana****,
Casa de Pilatos****.\\

March 30\\
Granada: the former capital of the Nasrid Empire
Now there is no train going to Granada from Cordoba, and the bus takes over 4h, in order to get most out of the day, we thus got on the bus shortly after 5 (together with a couple of other pour souls). The bus was pretty decent and at least I could fold myself enough to get some more sleep. And finally I got to see the cathedral of Granada. Back in 2001 we stood in front of it and were told there was no time (aka we rather had to go for a coffee for 45 minutes, since our tour guide in 2001 couldn't be bothered). The cathedral of Granada is built after Then another little gem was the Madraza of Yusuf I, one of the oldest schools of Granada -- and all of that just for 1 EUR. Then we also paid a visit to the Royal Chapel, which contains the tombs of the catholic kings (you can imagine -- no photos). We had prebooked our tickets for the Nasrid palace of the Alhambra. You cannot print tickets directly, but instead you get a confirmation code, which need to be exchanged for real tickets, but two hours before you actually want to enter the palace, else they just expire (yeah really bad service, but it is even more unlikely to get tickets on sale for the day, and secondary sales go south of 50 \$). Now the tourist information should allow to print tickets, but those machine were out of service. We were told to look for another place a few roads down, and just 5 minutes before the two hour limit. Now a relaxed lunch and then into the Nasrid palace. The royal residence of the Moorish kings, absolutely breathtaking and gorgeous. After we made it to the garden palace of Generalife we ended the day at the private residence and museum of Carmen Rodriguez-Acosta, surrounded by a very extensive garden. And once again due to processions no bus was running back to the main bus station, thus once again a 3 km walk back. And since we were still to early we again got beers, and it took really a lot to convince them we really can handle 0.5 l. And back in Cordoba we witnessed another nightly easter procession, this time with music and torches.\\

Granada: Cathedral*****,
Capilla Real***,
Madraza*****,
Alhambra and Generalife*****,
Carmen Rodriguez-Acosta****.\\

March 31 What the hell - the cathedral is closed today?\\
After Manuel decided to watch Spanish soap operas until late night, we started the day a bit later than planned (aka not at 7 am but 9 am). And here we go -- the Medina Azahara, a moorish ruined palace, once upon a time the capital of Andalusia, like you would imagine it to appear in the stories of 1001 nights. Built around 940 the palace city fell into despair less than 100 years later. Excavation started at the begin of the 20th century, by now around 10 \% of the total area has been excavated and restored. This part includes though the palace of the caliph. By then we were really pumped for the grandiose finale the Mezquita -- which turned into a total disaster. There is one day of the year, when the Mezquita is open to the general public for a whopping 50 minutes, and that day was...today (the original reason why the plan said get up really early).  So there are people guarding all gates to make sure everybody sees the curtains of the door, but not more, indeed you can't even get a sneak peak. So no cathedral for Manuel, absolutely pissed, even pigeons cannot help. I already think how I could combine maybe Nerja and Malaga in a trip later that year, first missing out on the cathedral of Sevilla and then on the cathedral of Cordoba too -- tough luck. After we missed out on the cathedral of Sevilla (really nice to have, but not THE essential building which captures Granada), now messing up the second grandiose cathedral as well -- failed planning I'd say. Instead we made it to the Alcazar (splendid gardens) and the Arabian baths (light gets through star-shaped holes in the ceilings), and after a short stop by a roman temple, we ended up in -- not that grand finale as hoped -- but still cute Palacio de Viana with zillions of courtyards (and sumptuous rooms, but no photos allowed as so often in Spain) ... and all the way back to Madrid (after having the superlarge beers at the Cordoba station -- aka 0.5 l). Still pissed about myself and my failed planning.\\

Cordoba:
Medina Azahara*****,
Arabian Baths****,
Alcazar ****,
Palacio de Viana***,
Roman Temple**. \\

April 1 
Segovia: Done too early - why not adding one more royal summer palace?\\
After spending the past days in 30-35 C warm Andalusia we were back in the centre of Spain. The claim was that it should get close to 23 degrees during lunch time. Thus I decided staying on shorts would be fine. I was so wrong -- missing the bus at Segovia by 10 seconds we had to wait another half an hour (aka breakfast at the train station) -- once we got into downtown standing in front of the Aqueduct I realised immediately below 15 degrees is so cold. The Aqueduct is one of the best conserved roman construction, most probably built already in the 1st century. Impressively it was still in service until mid 19th century. We got into the cathedral so early, that we were in before they started charging the entrance fee, but once again I was impressed by the cathedral (by now photos were allowed). The cathedral is a gothic cathedral, with most of the interior being upgraded in baroque times. I had planned to spend most of the day in Segovia, but then after the Alcazar we were done shortly after 11.  Now we finished a few hours earlier than planned, and I wanted to do the royal palace of La Granja anyway (a few days later on April 4). Thus we decided just to go ahead and do the palace, which also would free almost the full day of April 4 -- WHICH could free up space for either Cordoba or Sevilla. Well anyways off to La Granja. The main halls of the palace (no photos) were cute, but the typical baroque, red, blue, Chinese rooms and a lot of gallery type rooms. I liked the garden rooms of the ground floor a lot more, with statues, frescoes, columns and some with their own fountains - and the gardens. The fountains (all decorated, some with dragons, roman goddesses etc) are only running in certain intervals, which we unfortunately missed, thus we just saw the statues without the water displays, but still impressive. Once back in the hotel I figured out, that indeed I can see cathedrals in Cordoba and Sevilla in one day, leaving early enough in Madrid -- SOLUTION!!, and the holidays are totally saved. \\

Segovia: Aqueduct*****.
Cathedral*****.
Alcazar*****
Town Walls and Gates****\\
La Granja: 
Palacio Real*****\\

April 2 
Madrid \& Toledo: Don't forget your stuff\\
We all know that Spain colonised most of Central and Southern America over the course of the 15th and 16th century. And they brought a lot of artefacts over the Atlantic. All of that is presented in the Museum of the Americas, which sits a bit on the outside of Madrid's old town even beyond Moncloa. I learned quite a bit about the former societies and the history of all different cultures, the contest was also described without glorifying what had actually happened. After our failure to be prepared for the temperature changes the day before, Manuel decided to bring on a sweater and throw it into a backpack in Toledo. As you can imagine -- no photos -- inside the museum and backpack has to go into lockers. After the museum we walked over to the Eremita de San Antonio de la Florida -- the chapel which Goya is buried in. The chapel itself is also full of his paintings, they even removed most of the religious services to a neighbouring chapel, since too many tourists frequent the Eremita (btw no photos). There are actually no guards standing around in the chapel - but don't let this deceive you. Big brother is watching you from the ticket counter, the amounts we heard ``no photos'' over loud speakers was quite astounding. This time I appreciated the chapel more than the first time I had seen it. Maybe it helps if you don't have to walk there for 40 minutes in almost 40 degrees.\\
After lunch -- we had 2 1/2 h to spend - I heard an ``oops ehhhm'' from Manuel. ``The backpack is still in the museum" - which was 45 minutes away by foot. And we still have to get to Atocha to catch our train to Toledo. Internet claimed there was hardly any chance we could make it. We still wanted to give it a shot. Thus up to Moncloa - I told Manuel to run to the museum alone - I would be not up to his speed, and I rather should get the metro tickets for both of us. Said and done, time is ticking while we are on the Metro- finally getting of at Atocha - 5 minutes to go. We run to the high speed train lines -- train is not listed on the display, but shouldn't have left -- I remember Atocha's high speed train terminal has two levels, and we were standing at the wrong one! Tick Tock, Down one floor, rushing through security (yes in Spain for high speed trains you have to go through security like on airports). Clearly in a rush we just yell Toledo -- and the guard yells back the platform number, keeping on running, tickets are checked while hear already the train conductor's whistle, running down the stairs and jumping into the first door in sight, which closes seconds later behind us. Against all chances WE DID IT.\\
Arriving in Toledo we decided to prolong our stay there and exchange tickets for the last train back to Madrid. Manuel also managed to get my new tickets for Cordoba and Sevilla printed. It needed quite some time to convince the man on the counter though, that this is indeed not a mistake, and the tickets are bought for the same day. Our first stop in Toledo was the cathedral. Unlike a few years before, this time photography was allowed. Toledo is another fine gothic cathedral in Spain, once again redecorated in Baroque style. Unlike for the other cathedral of the trip it was though not possible to get close to the fence of the main altar. But that is not the highlight of the church anyway (and nope neither is the sacristy, although this one was painted by El Greco too). The absolute highlight is the astonishing altar of ``El Transparente''. One - if not the most beautiful baroque - altar in Spain. Covering the full height of the main nave, it sits at the back end of the choir, facing the choir chapels. Then we visited the synagogue, which reminded me personally though more of a mosque similar to the Mezquita or how you might find it in Morocco. After a short step in the Jesuit church, climbing the tower to have a nice view of the old town), we ended up in the mosque turned church of Mezquita Del Cristo De La Lu, cute and also close to one of the city gates. Then we finished off our day with dinner and a short sneak peak into the Monasterio San Juan de los Reyes, and all the way down crossing the old bridge over the Tajo by moonlight...and back to Madrid.

Madrid
Museum of the Americas****,
Ermita de San Antonio de la Florida****.\\
Toledo:
Cathedral*****,
Synagogue ***,
Jesuit Church***,
Mezquita Del Cristo De La Luz****,
Monasterio San Juan de los Reyes***\\

April 3
Starting the day early our first Cercanias was late by 15 minutes. Eager to still make the connection at Chamartin falling out of the first train and nope still missed the connection (nothing was hurt). So another wait for roughly 45 minutes (unfortunately train station breakfast instead of a Cafe in El Escorial) and off to the gigantic complex of the Monasterio San Lorenzo de El Escorial. The complex contains the former Habsburg royal palace, as well as the more modern Bourbon royal palace with a large library, a monastery and the royal Pantheon grouped around a gigantic basilica as heart piece. The first time I visited we missed out on the Bourbon apartments, since they were all booked out for that day. Since tickets for that part of the complex cannot be bought online, we got into the queue and were informed that the first english tour would take place only 4 h later, so Spanish it is (and no photos on the whole complex). The apartments are actually quite nice, but unlike everybody around me I just understood hardly anything. Since the palace is still used for some ceremonies, an armed soldier was assigned to supervise each group. Then the visit continued on the Habsburg side, quite a contrast since the goal of those quarters was to show modesty, The opposite of modesty are the library and the church. After the large complex we checked out the private little palace of the Case del Principe (also no photos) with its little garden. Then we planned to have lunch, but after not being served for over 40 minutes (we tried everything), while other newcomers were welcome and served, we left for another sandwich place to have at least something to eat. And back to Madrid for another highlight -- the Museo Reina Sofia with its modern art collection, including Picasso's masterpiece, depicting one of the darkest chapters of the Spanish Civil war -- Guernica (photos are allowed in the museum, but not anywhere close to Guernica).\\

El Escorial:
Monasterio San Lorenzo de El Escorial***** (Bourbon Apartments****, Pantheon**, Basilica \& Library*****, Habsburg Apartments****),
Casa del Principe****\\
Madrid:Museo Reina Sofia*****\\


April 4, 2015 Exchange Day\\
After all that exhausting time travelling to Andalusia and back to Madrid Manuel calls it quits today and will be replaced in the evening by Eric. Thank God I made up for half a day by doing the Palacio de la Granja already a couple of days ago. Since the gives space to Cordoba and Sevilla once again. After saying goodbye to Manuel caught the first train down to Cordoba. And FINALLY made it into the Mezquita once again. Renovated quite a bit since the last time I saw it in 2001. Compared to back then all ropes and nets. were gone. And particularly the cathedral part was very shiny whtie and cleaned. One of the most breathtaking places I have ever been to, even the cathedral which just was put in the middle of the old mosque doesn't destroy the impression of the forrest of columns, and the Mihrab is still conserved. Unfortunately I gave myself hardly more than 60 minutes to see it, and off I was to Sevilla. Sevilla cathedral -- finally -- the largest gothic cathedral of the world containing also the largest Retablo of the world, which happens to be one of the finest as well. Two well-known pieces of the cathedral are the silver altar, which was partially destroyed in a war to pay for its expensive and also the tomb of Columbus. Since Easter threw the plan totally off, in this tight schedule there was no time to make it up the Giralda this time around, so all the way back to Madrid (this time not having a beer alone in the train station though), and finally meeting Eric. And we have that delicious cheese once again (and a bit of Puerta de Sol and Plaza Mayor but oh well).\\

Cordoba:
Mezquita-Cathedral***** the absolute highlight of the trip\\
Sevilla:
Cathedral*****,\\
Madrid:
Plaza Mayor ****\\

April 5 Summer Palace and Spanish understanding of donations
And with Cercanias out to the summer palaces of the Bourbon kings in Aranjuez. We spend about an hour in the garden and then off to the palace (no photos). Certain rooms can only be visited with a guided tour, really little beautiful amazing rooms. Around a certain time the Nasrid quarters of the Alhambra received attention, and the royal family tried to replicate the original colours and decoration in their own state halls. Another beautiful room is the Porcelain room (similar rooms can actually also be found in Southern Italy where the Bourbons had their ``secondary empire"). In the park is the more private Casa del Labrador, which I would recommend people to check out, it's worth it. Then back to Madrid and onwards to the Royal Palace of El Pardo, which was used by the Spanish fascist dictator Franco as his private residence. The remodelling included for example the construction of a cinema in the previous theatre. After a short stroll through the Plaza Espana and the Egyptian temple of Debod (a donation from Egypt for the Spanish help in relocating the Abu Simbel rock temple complex). Then we wanted to visit the crypt of the Almudena - Madrid's cathedral. They asked for a donation of 1 EUR. Typically I give a donation, but after I visited the place. But the elderly Senorita wasn't having that, vehemently demanding at least 2 EUR, or else no entrance - she even slapped us, when we didn't want to force donate. In the end we did - but shame on the cathedral for pulling that off. Either charge, which is fine too, or make it a donation, but forced donation are just bad practise (same for all the US museums which work like that).\\

Aranjuez:
Palacio Real***** (with tour),
San Antonio**,
Casa del Labrador*****\\
El Pardo: Palacio****\\
Madrid:
Plaza Espana***, 
Temple of Debod***,
San Teresa \& San Jose**,
Crypt of Almudena***,
Puerta de Sol***\\

April 6:\\
And we started the day again with the Almudena, this time it was even expected to donate money, although a religious service was taking place. After it finished we strolled through the large nave. The construction of the cathedral started by the end of the 19th century and finished only at 1993. The interior is pretty modern, but also reminiscent of gothic style. I like it. Then we continued with the Palacio Real. The royal palace is one of the largest palaces still in regular use up to today (photos only allowed in the main stair case and the court yard). One of the most beautiful palace - for sure in my personal top 5 up to now (up there with Versailles and Buckingham Palace e.g.). Then we continued with Madrid's old cathedral - San Isidro and the basilica os San Miguel. The last item of this Spain trip was the royal theatre
Madrid. Teatro Real has quite a modern flair (remodelled in the 90s), unlike most of Europe's other royal theatres and operas, but still keeping a classy feeling to it. And then it was time to say good-bye to Madrid and back to work.\\

Madrid:
Almudena****,
Palacio Real*****,
San Isidro****,
San Miguel***,
Teatro Real****\\

\section{May 14--May 18: Campania}
\label{2015:Campania}

burglar attempt, train strike

May 15: \\
Pompei: Roman City*****, Santuario of Our Holy Lady****, Vesuvio*****, Torre Annunziata: Villa Oplontis*****, Napoli: Archeological Museum*****\\

May 16:\\
Pozzuoli: Solfarata*****, Portici: Reggia***, Royal Chapel***, Ercolano: Ruins of Herculaneum*****, Castellammare di Stabia: Hadrian's Villa****, Marcus' Villa****, Gesu Nuovo*****\\

May 17:\\
Caserta: Reggia*****, Napoli: Palazzo Capodimonte****, Palazzo Zevallos (Galleria d'Italia)**, Basilica San Francesco di Paola***\\

May 18:\\
Duomo*****, Capelle Sansevero****, Philippo \& Jacobo***, Basilica San Paolo Maggiore**** (crypt**), Gesu Nuovo*****, Teatro San Carlo****, Palazzo Reale*****, Castel Nuovo****, Santa Maria Incoronate***, San Lorenzo Maggiore****

\section{June 13--June 21: Los Angeles}\
\label{LA2015}

Why LA: UCLA was hosting a theory particle physics conference. The organizers wanted to get speakers from the experimental community for overview talks in their plenary session. After CMS didn't have too many candidates shortly before the deadline, my boss asked me if I would consider applying. Since it was hosted by my home institution the transatlantic flight could be well justified. I applied for the talk and was selected in the end. Preparing a talk for over 45 minutes was quite a challenge, particularly since it included measurements from several working groups, which all needed to sign off on the material shown. \\

June 13:\\
I transferred in London Heathrow for the first time in my life, quite an experience. Waiting for over 3 hours for a flight with having only access to the internet for an hour wasn't that much fun either. But then immigration didn't have long queues this time, in fact my colleague Cameron who hosted me on my first night (Graduation weekend, thus all places close to UCLA had been booked out) was still in traffic once I arrived. Cameron suggested in order not to get into jetlag, to get going to a private party somewhere close to Hollywood. It was indeed quite an interesting crowd, from aspiring actors, directors, screen writers, to people from law \& med school, or people like well me. Still a nice evening out, although I didn't stick around much after midnight, having gotten up early in Geneva that day.\\

June 14:\\
Just after getting up I did a couple of test presentations of my talk, keeping a flow with enough details and then also reading up on the results I wanted to present. An overview plenary talk is really an art on its own. Indara had also gotten the word that I was in LA, thus she organised me Uber to meet her and her friend for a proper brunch. Clearly she knows all about good food places too, having done her Undergrad at UCLA too. Afterwards I spent the afternoon with Cameron and his buddy Matt, who both wrote on a script for a science fiction based series. Then Cameron brought me to the hotel for the rest of the trip. My boss had warned me already before, it was OK but not really up to standards I got used to in the US. The room was spacious enough, also with stable internet connection, but the water was tasting like chlorine flavoured. Anyway still more to train on the talk. On that evening I had Chipotle, since I was told to try it, it was fine but nothing I would go on a trip for.\\

June 15:\\
The first day of the conference and I gave the opening plenary talk. I was told by multiple people afterwards that they liked the talk, also what I presented seems to have been considered interesting. Eric talked about a more special topic which we had worked on for almost a year. After both of us had given our talks and after two more sessions we decided to do a bit of tourism in the late afternoon, getting over to Long Beach. There we visited the USS Iowa, one retired battle ship which is still maintained properly. One of the veterans told us about its past glory fighting against the Japanese (short unsure look to Eric) or Germans (look to me). I mean we do know our heritage and what went wrong, so nothing he needs to feel uneasy about talking to us. Anyway after that interesting trip, we did go to see the RMS Queen Mary, a retired Ocean liner converted into a hotel. Still amazing to see how luxurious some people travelled and how poor the conditions were for the lower class passengers. At the end of the day we celebrated Eric's graduation with my boss of that time, Prof David Saltzberg, who also happens to be an science advisor for shows such as the Big Bang Theory. Food was excellent, the sunset over the pacific beach and hills was amazing to see too. And I was introduced to Riju, but you will hear more about him in many more chapters to come.\\

USS Iowa****, RMS Queen Mary****\\

June 16:\\
Today Cameron was supposed to pick me up, and we were supposed to meet Eric at the California Science Center. But then we received a call that his car had broken down and thus he couldn't join. That was clearly not an option so we picked him up. All his family welcomed us, only to silently disappear a couple of minutes later. Seems even a chilled back person like Cameron or a tiny German such as me can be intimidating. Anyways back to the Science Center it was which houses the Space Shuttle Endeavour. Just amazing to see a vehicle which had been out in space leaving the atmosphere of our planet. Just next to the science center is the LA Memorial Coliseum, which was the main stadium of the 1984 Olympic games, as far as I know it is also supposed to be used in the upcoming Olympics of 2028. A random person there told us about his play he was writing which would make it big on Broadway later on. So far I didn't hear of that show though, but a lot of stuff got derailed nowadays. And then we had food, cheese steak to be precise. Driving to Pasadena afterwards we thought we could checkout the Caltech Campus, but then we were told by a guard it was in fact private property. Then Cameron took me to a place to our first Sushi place.

Space Shuttle Endeavour*****, LA Memorial Coliseum***\\

June 17:\\
Today's lunch was at Sushi, and both Eric and Cameron loved it. I obviously enjoyed it too, but without a baseline who am I to judge it. Then we drove over to Hollywood and walked over the Hollywood Walk of Fame. We stopped by the Chinese Theatre, very evolved setup which reminds one of an Americanised version of Chinese art to watch Jurassic World -- cute movie. Then we took a tour of the Dolby Theatre which hosts the annual Academy Awards. We also got into some of the backstage rooms. And then it was time to try the delicious combination of chicken and waffles. Might seem an unusual combination, but it works (don't forget the maple sirup on your waffles.\\

Chinese Theatre***, Dolby Theatre****\\


%Sushi Go 55, Hide

%pasadena: had chicken and Waffles

June 18:\\
On this day we wanted to go to Downtown but then Barack Obama and Hillary Clinton had two back to back fundraisers in the area, thus all main roads had been blocked off, and thus no LA downtown for me. Instead Cameron took me and Eric to a Ramen place and later on we had Chipotle, since I was supposed to find out why some people did extra trips to have Chipotle in Europe, but I didn't get why people were so crazy for it.\\

%ramen Tsujita Annex

%Chipotle

June 19:\\
If you want to find out how rich people built their own private mansions, visit the Getty Villa, which is a ancient roman revival style villa which houses the ancient greek and roman collection of the Gettys. We had our final touristic outing by the Getty Center featuring another nice (but a tad small) modern art exhibit. Another food event later, this time eating at a Mexican place.\\

Getty Villa****, Getty Center****\\

June 20:\\
Public transport does exist in LA, albeit on a very small scale. But I did take the public bus from UCLA to LAX. And then had the window seat enjoying the Mojave Desert, Horseshoe Canyon, and last but not least Canyonlands National Park.\\

Canyonlands National Park overview*****

\section{July 4: La Tour}
\label{LaTour2015}

Glacier du Tour*****

\section{July 11: La Jonction}
\label{Jonction2014}

How this hike came to be: I am always fond of glaciers, and Manuel likes hiking and glaciers. The weekend was announced to be very sunny and very hot, thus Manuel decided to get a rental car, go to the mountains, climb up high and escape the boiling heat. Clearly I always enjoy hiking too, and I love to return to Chamonix for hiking whenever I can. This time we decided to do the longest non alpine hike in the valley: La Jonction. The trail leads from the hamlet of Bossons up a rocky ridge to about 28xx m, where the glacier parted into the Glacier des Bossons and the Glacier du Taconnaz. Unfortunately the connection to the Glacier du Taconnaz broke down already couple of years ago.\\

July 11:\\
We got our car at the airport as early as we could and drove up to the chair lift station. The first part of the hike is only about 50 minutes one way, also not that steep, but if we can give us about 90 min of a head start, we should definitely do it. Compared to the last time I had been here (3 years ago in 2012) the glacier tongue significantly retreated. In fact both glaciers retreated by more than a km from their largest state, which they had during the little ice-age before the industrial revolution came to full force. We got ourselves a cup of coffee before starting the hike, and then hiked up multiple switchbacks until we saw the full length of the Glacier du Taconnaz: still very impressive ice towers, crevasses \& rifts within the ice, and waterfalls of melting water all over. Getting back to the other side of the ridge, the Glacier des Bossons is a very impressive sight, particularly with the Aiguille du Midi in the background. And about 4 h later we finally reached the end of the path, with direct access to the glacier. Both Manuel and I walked a bit on the glacier, in fact just for getting a photo snapshot. The view is absolutely rewarding: a giant stream of ice right in front of you, coming down the 4000 m peaks of the Mont Blanc Massif. And naturally the way down proved to be rather the issue, particularly on the knees. Although we did a short stop in one of the huts along the way to recover a bit, more than 1000 m down a rocky path is just not a walk in the park. Manuel always told me - just a bit, clearly didn't help that I knew it was still more than 1 h down. Anyways we made it down in time with the chair lift still in operation. And back to Geneva, dropping off the car, and then having pasta. This hike still remains up to now the most challenging I have done, not because it is technically difficult per se, but the length and the rapid change in altitude take their toll both on knees and legs. Pro-tip: always check your camera setting by the end of a hike, but more about that later.\\

Glacier du Taconnaz*****, Glacier des Bossons*****

%\begin{figure}[h]
%  \centering
%  \includegraphics[width=0.7\textwidth]{figures/2015/157110120_Chamonix_Glacier_des_Bossons.JPG}
% \caption{View of the glaciers at the top of the La Jonction hike}
% \label{fig:LaJonctionPic}
%\end{figure}

\section{July 16--19: Versailles \& Mont Saint Michel}
\label{2015:ParisI}

How this trip came into existence: I wanted to see Mont Saint Michel, but seems the best way to do it is starting from Paris. Since we have new students in, I thought it would be a good way to show them Paris or Versailles. After finding out that Versailles offers night illumination followed by fireworks on this particular Saturday, I decided to put that up, and find out who wants to join.\\

Co-travellers:\\
Riju: US-American: just came over by begin of July. After meeting Riju for the first time during my trip to LA, I thought it would be a good way to introduce him to Europe. Having been in Cambridge and Paris before, I thought all would be set up properly. \\

July 16:\\
Riju decided to stay at a cheaper place than me, so he travelled on another train than me, spending already a day in Paris. I took the last train out of Geneva arriving at the Hotel Bel-Air just shortly before midnight.\\

July 17: Versailles:\\
Since Riju stayed at another place, consequently we meet up at Versailles. I gave him my phone number and trusted in good faith that things will work out. Indeed once I arrived in Versailles I realised Riju was in front of the line, being there 1 h before opening, instead of my planned 20 mins. Seems his hostel had an incident, where the owners partner got in an argument and trashed the place, so he took most of his stuff in his backpack. Naturally we were also in pretty quickly, but had to stop at the wardrobe to get Riju's stuff locked up. Then we rushed by the audio guide and the Louis XVI rooms in order to be first at the State Apartments. Seeing the Hall of the Mirrors without any other visitor felt like once in a lifetime. After taking photos in the King's State Bedroom I realised that the white balance was off from when I handed the camera to Manuel!! (damn it). Putting the correct white balance settings on, the queen's state rooms and the battle gallery were all fine. We had quite an intensive day with the Trianons and walking through all of the park, and Marie Antoinette's Hamlet (in renovation at that time). By that time we were got close to the time, when the Palace closed, thus I rushed back, while Riju preferred to spend more time in the park, having realised that his shoes were not best suited to run and walk long on cobblestone. Then I did the newly opened Louis XIV rooms (not THAT impressive, so skip them in case you want to see the main rooms empty). Since it was now close to the closing time, the State Apartments were quite free of people again. The Hall of Mirrors was prepared for a Ballet performance afterwards, but I stayed to be the first and last regular visitor of this particular day. Then I met up with Riju again and after a short stop at Versailles' cathedral we had a long dinner. Afterwards we went back to the illumination of the fountains again. With multi-colored spots and some effects like smoke, it was really out of this world. In fact it was the first and (up to now -- 2019) the last time that I have seen the water switched on for the Bath of Apollo. And then we had the 20 minute long fireworks, with flames shooting out along the way as well (accompanied by music). But then all hell break loose. Instead of being clear and free of clouds like predicted it started to thunderstorm just after the fireworks. Thousands of people rushed to the gates, and some people decided to wait out below the Loggia of the Chateau - to no help. I decided to run, having clarified with Riju that he would know how to got back (after all he got to Versailles from a different place, so shouldn't be an issue). It rained so hard, that i had to clean my glasses from water with my fingers, but I made it on the train later switching to a tram. Around 20 minutes later I am on the tram to the hotel, then I receive a message ``whereabouts are you''. Expecting it to be from Riju, I explain him how I rushed out, and also how I think he should get to one of the Versailles train stations and back hom. getting back an ``Interesting''. Later I figured out, that the mobile phone where Riju saved my number ran out of battery, and on his US phone he only had his advisor in, so I effectively communicated with him by the middle man of our professor. Anyways he got my number and I got him home, on the last train getting out of Versailles to Paris on this day, while I got to my hotel safely too.\\

Versailles: Chateau***** (with Trianons*****, and Fireworks*****), Cathedral***\\

July 18: Mont Saint Michel:\\
This trip I did on my own, getting to Dol-de-Bretagne and via two more buses over to the island of Mont Saint Michel. The dam had been recently replaced by a bride, with the goal to make Mont Saint Michel a well separated Island again. It clearly was low tide when I was there, with typically muddy water soaked soil around. The abbey was really nice to see, also listening to the choir performing in the church for Sunday service. After visiting the abbey I had Moules-Frites, the regional dish of mussels and fries. Having enough time on my hands, I decided to walk over the bridge by the Cousesnon Dam, and then taking a bus and the TGV back to Paris.\\

Mont St Michel: Abbey*****\\

July 19:\\
Taking an early train out of Paris for the first time, I expected to meet Riju in the train restaurant. But he didn't appear, and I figured out later, he missed the train by about 10 minutes\\

The Aftermath: Since Riju got lost in Versailles and missed the train out of Paris, our professor decided the that the new grad students are not fit for travel yet, and forbid them to go on my next planned trip to Zermatt and any other trips until further notice. He acknowledged that previously all grad students and even undergrads seemed to have done fine, but maybe this changed. It took up to March 2016 until Riju dared to join me on a trip again, being in the end the co-traveller who joined me on most trip (taking out family to avoid any bias). After all his professor still remembered this incidence during his PhD defence speech 4 years later, so this trip did make an impression on everybody involved.

\section{July 26: Zermatt}
\label{Zermatt2015}

How this trip came into existence: Can you imagine, after being in Switzerland since 2011 (on and off, but still a long time), Eric has never made it to Zermatt. Surprised I decided to take him to ALL of Zermatt, also offering everybody else to join us on the way. As aftermath of my previous events in Paris, the grad students were forbidden to join.

Co-travellers:\\
Eric: his last trip in Switzerland before moving to Germany, now Dr Eric, his first trip to Zermatt though\\

Manuel: hiking is his jam, so Manuel will join us for the hike, though he doesn't want to join as early as 4 am, same for our co-hikers, including friends from Poland, the US, and Stani (Canadian, with roots in Bulgaria).\\

Starting shortly after 4 am, Eric and I make it to the first cable car we could potentially reach to get all up to Kleinmatterhorn, enjoying the panoramic view and the glacier palace. On the way back we meet other US Americans who tell us how amazing they think Zermatt is. By the Schwarzsee station we meet up with our co-hikers. Taking the lead (as slowest hiker) we make it up to the H\"ornlihut, about 30 mins longer than the posted hiking time. Amazing views of the Furgg glacier, as well as the Theodulglacier, all the way over to the highest mountain of Switzerland in the Monte Rosa Massive. Manuel and Eric have first good-bye shots at the foot of the Matterhorn, while we all have a short lunch break before hiking back. Eric and I continue our do-it-all trip catching the train up to Gornergrat, while everybody else gets on the train home to Geneva. Up on Gornergrat clouds start to cover the mountain tops, but the view on the Grenz- and Gornerglaciers is still amazing. Having cokes and snacks in the restaurant at the former observatories up on Gornergrat we enjoy the still warm inside, while it starts to thunderstorm outside. Eric and I get on one of the last trains back to Geneva, arriving there at 1 am. Our last trip in a while, as Eric says good-bye to Geneva, starting a postdoc position at Max-Planck-Institute in Garching, moving in with his girlfriend Siyi.\\

Kleinmatterhorn*****, H\"ornlih\"utte*****, Gornergrat*****

\section{August 21--August 23: Monaco and Nice}
\label{2015:Monaco}

How this trip came into existence:\\
Fantasy football, and no -- not the real one, rather the American version. For whatever reason it had been decided I should be part of a fantasy football league. At the same time the decision had been taken to have the fantasy draft take place in Monaco. Since Monaco had been on my list of places to see for quite a while, obviously I was on board for this trip. Due to time restrictions, which I will get to later, most of our crew had to leave on Sunday morning, thus I planned to spend Sunday in Nice. \\

Co-travellers: \\
Laser:\\
US-American, his boss decided years ago, that it would be ideal to have her group meetings on Sundays, most probably to ensure here students would work on the weekend. There was no way he could allow missing this meeting, thus all others left already Friday morning, and had to leave the first flight on Sunday. Needless to say, he left physics for industry after getting his PhD. Big supporter of the Gators and Green Bay Packers.

Evan:\\
US-American, only got pulled over to the dark side of physics later on, in fact he started out studying law first. Always up for a good time, very effective in talking me into joining. As we had to find out later unfortunately suffered from food allergies.

AJ:\\
US-American, dared to go already on a second trip with me. Likes to make fun of other teams should they lose out to the New England Patriots once again (can be excused if you are from Boston). He knows how to appreciate good quality food and drinks (things I hardly care and know about), and already prepared for his future beyond physics in finance.\\

August 21:\\
Took a very late Friday flight out to Nice. By that time no buses were leaving for Monaco anymore, so trains it is -- or the next train, since the one I wanted to catch was cancelled. After waiting for 70 minutes finally get on my ride to Monaco. The train station is midway up the mountains, and offers really nice (night)views of the harbour. By that point I realised that my lens was full of dried water drops from my previous trips to the Truemmelbach waterfalls. After a thorough cleaning ready to get to our hotel. Hotels in Monaco are notoriously expensive, thus we decided to stay just across the border in France. Per person that was a difference of roughly 50 EUR. The others had already arrived in the morning, checked out the Casino and had dinner at a three star Michelin restaurant. Unfortunately for Evan they served something he reacted allergic too, so a bad ending for him, although he was fine a couple of hours later.\\

August 22: Following the Formula1 race track\\
Starting out the day in the Monte Carlo Casino: during Morning hours the casino can be visited as well, but rather in terms of sightseeing, since those are off business hours and the games are closed. Instead it is allowed to take photos then. Built in the 19th century the rooms are decorated in neo-baroque style. Then I decided that it would be fun to follow along the Formula 1 Grand Prix racetrack over to old town. On the side of the roads you just have to follow the characteristic red-white curbs. Don't expect amazing views or details about the history of the race, from time to time they installed a plate with the name of the corners (and the tunnel is just an ordinary road tunnel). Passing all the fancy yachts and the swimming pool in the harbour we walked up the stairs to the old town for our second touristic stop -- the Prince's palace. Most of the palace is still the private residence of the royal family, in summer the state apartments and the courtyard can be visited (no Photos). The palace has a couple of nice rooms, considering the size of Monaco, you can't expect to have another huge Buckingham Palace style place here. So it is what you expect, a couple of Baroque style living and bed rooms, saloons and as grand final the throne room. We were lucky that the exchange of guards happened to take place just when we got out of the palace. After lunch we stopped by the cathedral. The Cathedral was finished in 1903 and built in a rather reduced style, with one large mosaic in the apse as only sizeable piece of decoration. In the choir you can find the royal tombs, including the tomb of Grace Kelly, the former Princess of Monaco. The former prince of Monaco, Rainier III, was an avid collector of vintage cars and as well formula1 race cars, e.g. a Kimi Raik\"onnen McLaren. Having drinks by the harbour of Fontvieille, the others decided they had enough tourism for the day, while I continued to climb the roads up to the Jardin Exotique. The exotic garden is the botanical garden of Monaco on a cliffside with plans predominantly originating from the Americas, Africa and the Middle East. On the site of the gardens evidence of prehistoric humans have been found (and places in its own museum). I enjoyed the stroll alongside the little paths and the fantastic views of old town and the Cote d'Azur. The most fascinating part was though the tour of the Grotte de l'Observatoire. A nice cave filled with stalagmites, stalactites, draperies and columns, definitely a highlight of the trip. And then the Fantasy football draft: the ``official'' reason for our trip, with a WiFi which proved not to be that stable as though. Being a green horn to American football anyway, my random hit and miss team didn't play out so badly for the season to come, while everybody else's knowledge didn't lead to any smarter drafting necessarily either though. In the evening we all put on our suits and off we went to the Casino, this time for gambling, sipping on cocktails etc. After a visit to another fancy bar we enjoyed the night views of the harbour from one of the roof terraces.\\

Monaco: Monte Carlo Casino*****, Formula 1 race track***, Prince's Palace****, Cathedral*, Vintage Car Collection****, Jardin Exotique****, Grotte de l'Observatoire*****.\\

August 23: Nice and Villas alongside the Cote d'Azur\\
Everybody else left on the earliest flight, so that Laser could attend his group meeting. I decided to spend the day in Nice and places alongside the Cote d'Azur. Plan was to leave the luggage in the train station of Nice (according to the SCNF webpage they had lockers). Little did I know, that they had major construction going on, and the lockers were closed, tourist information didn't know if there were any other lockers in the city closeby. So carrying around my luggage for the rest of the day. Starting out the day in the baroque cathedral (cute but nothing too special), I continued then with the aristocratic Palais Lascaris, filled with loads of old instruments, paintings and tapestries. After a stroll alongside the Promenade des Anglais, a 7 km long road directly by the beach of Nice, I wanted to see the Villa Massena, but since I still had my (tiny) suitcase with me, they wouldn't let me inside. So skipping this place I jumped on a bus to get to Beaulieu-sur-Mer. This village is the home of the Villa Kerylos, a ancient Greek style villa built by a French archeologist just by the sea. The interior decoration is inspired by findings of Greek noble houses on the island of Delos, but with all the comfort of an early 20th century building. Only a short walk along the coast away in the neighbouring village of Saint-Jean-Cap-Ferrat is the Villa Ephrussi de Rothschild. This luxurious home is surrounded by a vast garden, which is split up into nine sub-gardens, each dedicated to a different theme, e.g. a Japanese Garden, a Rose garden, an exotic garden. And just in front of the house a formal French garden with fountains, statues and water games accompanied by late Baroque classical music. Then jumping on the bus back to Nice, where I had a walk through old town passing the Place Garibaldi, the orthodox cathedral and Place Massena with the Fontaine du Soleil.\\

Nice: Cathedral****, Palais Lascaris****, Promenade d'Anglais***, Orthodox Cathedral***, Place Garibaldi***, Place Massena****\\
Beaulieu-sur-Mer: Villa Kerylos****\\
Saint-Jean-Cap-Ferrat: Villa Ephrussi de Rothschild*****

\section{August 28--September 12: St Petersburg}
\label{2015:StPetersburg}
what a conference dinner

August 28: St Petersburg\\
Kasan Cathedral***, Church of the Savior on Spilled Blood*****, Isaac's Cathedral*******\\

August 29: Lomonossov\\
Lomonossov: Oranienbaum***** (Great Menschikov Palace****, Chinese Palace*****, Palace Peter's III***), St Petersburg: Peter \& Paul Cathedral*****

August 30: Pawlovsk \& Gatchina:\\
Pawlovsk Palace*****, Gatchina Palace*****\\

September 2: Moscow:\\
Kasan Cathedral***, Lenin Mausoleum****, Kremlin***** (Cathedrals***** and Armory*****), Christ the Savior Cathedral****, Basilius Cathedral*****, Cosmonaut Museum*****, Cosmonaut Monument****\\

September 3: Peterhof:\\
Lower Palace Gardens*****\\

September 6: Pushkin:\\
Pushkin: Alexander-Park****, Catherine's Palace***** (Upper Bathhouse****, Lower Bathhouse***, Eremitage*****, Grotto Pavillon****, Agate Rooms*****, Concert Hall****, Turkish Baths****)\\

September 7: St Petersburg:\\
Michael's Castle****, Marble Palace***, Stroganov Palace*****, Michailowski Palace****\\

September 8: St Petersburg:\\
Eremitage \& Winterpalace***\\

And this day marks is the last one Manuel makes an appearance here (up to now by begin of 2020) after three long trips with overnight stays and three additional hikes. Just a couple of weeks later Manuel moved back to California, where he wrote up his thesis and successfully defended by the end of this very same year. Unfortunately still the last time we saw each other (and it is 2021 already)\\

September 9: St Petersburg:\\
Elagin Palace****, Menschikov Palace****, Jussupov Palace*****\\

September 10: Peterhof:\\

Someone who worked quite often with artists in Russia and thus had visited the country a couple of times previously was my artist friend Kendal. Thus I asked him if he know of any special arts museums in the St Petersburg area which an ordinary tourist like me might not have heard about. He told me what I planned to do seemed already extensive enough to add more to the agenda, but by coincidence he would also be over for work from September 10 on. Since we overlapped on that one night, we decided we should meet up for dinner, thus we had dinner at his place of stay after a walk through the Nevsky Prospect. Sometimes also nice coincidences like that lead to a nice evening in foreign countries.\\

Peterhof: Peter- \& Paul Cathedral***, Grand Palace***** (Upper Park***, Lower Park*****, Eremitage***, Catherine's Wing*****, Palace Church****, Monplaisir Palace*****), Colonist Park**** (Olga and Tsar Pavillons), Alexandria-Park**** (Farm Palace****, Gothic Church*, Country House****)\\

September 11: Strelna\\
The last day started with the usual big breakfast, and then i took the train out to Strelna. The Konstantin Palace had been destroyed and fallen even further in despair after World War II, before Putin decided to have the palace and the park rebuilt as presidential residence in St Petersburg, also hosting the 2013 G20 summit there. Several tours are offered, I decided to just choose the standard tour, which was held in Russian only anyway, not that many international travellers seemed to be around. The interior tries to recreate the neoclassical and baroque style of the original palace, some quarters are more modern or contain commodities like pool tables and a modern bar.\\
After the Konstantin Palace I continued my afternoon at Peter's I Summer Palace which was quite modest, clearly not allowing any interiour photography like anywhere else around Peterhof. Since I didn't want to be late for check-in I turned up at the airport 3 1/2 h in advance. Unfortunately the check-in didn't open until 2 h before the flight, neither was I allowed inside any area with restaurants OR wireless connection, thus I had to spent 90 minutes doing well pretty much nothing. Air France gave me the emergency exit seat as well, not allowing me to use my camera as well, thus unfortunately I didn't get any nice panoramic photos of St Petersburg and the bay out again despite sunny weather this time around.\\

Strelna: Konstantin Palace*****, Peter's I Summer Palace***\\

\section{September 18--September 20: Paris II}
\label{2015ParisII}

September 19:\\
Hotel de Lassay*****, Palais Bourbon*****, Hotel du Quai d'Orsay*****, Hotel de Brienne***, Hotel de Breteuil****, Hotel Potocki****, Argentinian Embassy**, Hotel de La Tremoille****, Hotel de Behague*****, Ecole Militaire****, Hotel de Soubise*****\\

September 20:\\
Hotel de Noirmoutier****, Hotel de Villars***, Hotel de Montmorin***, Hotel de Vogue*, Sorbonne*****, MINES ParisTech*, Val de Grace****, St Patrick's Chapel**, Pantheon****, Etienne-du-Mont***, Pavillon Boncourt*, Institut de France****, Bourse de Commerce***, St Eustache****

\section{October 3--October 11: Bavaria \& Austria}
\label{Bavaria2015}

I was invited to give a talk at the ISMD 2015 conference about recent QCD measurements at high energy. Since this conference was taking place at the resort of Wildbad Kreuth (belonging to the Wittelsbach dynasty, the former royal family of Bavaria, but more famous as former annual meeting place for Bavaria's ruling conservative party) attendants were supposed to be picked up at Munich airport and then shuttled down to the Bavarian alps. Since Eric had moved to Siyi and Bavaria recently I suggested we should try to meet up. He instead suggested I should get in a day earlier and just stay a night at his place which I quickly agreed to. For the weekend after the conference I proposed a short weekend trip to nearby Austria. Siyi was quickly convinced to join as well.\\

Co-travellers:\\
Eric: after graduating Eric took a postdoc at the Max-Planck-Institute in Garching and moved to Germany, great to catch up with him again.\\

Siyi: Siyi is an astronomer at the European southern observatory and based in the Munich area. Having visited Eric quite a couple of times in Geneva previously, she was happy to host me as well and to join us on the quick trip to Austria.\\

October 3: Munich\\
I took the first plane out of Geneva to Munich. October 3rd is German national holiday and one of the last day of the Munich Oktoberfest (yes the REAL Oktoberfest). Considering that other folks on the plane were in their Lederhosen and Dirndl outfits I guess I was not the only one ready for it. Once I arrived at Siyi's and Eric's place I was introduced to Siyi's friend who would also spent the weekend with us. And then we were ready to go. Considering it was German national holiday all tents had already filled up and were closed to people (and it wasn't even 11 am yet). We got ourselves a Mass of beer and some bretzels and enjoyed the atmosphere. At some point we had enough and then went on to have lunch at our most favourite place in Munich, the N\"urnberger Bratwurst Gl\"ockl am Dom (and yes you can check that in 2010~\ref{2010:Bavaria}, 2014~\ref{2014:Germany}, 2017~\ref{Munich2017}, and 2018~\ref{Germany2018}), having a quick view of Michaelskirche and Frauenkirche on our way. Clearly Eric, Siyi, and I were very pumped for Omas Fleischpflanzl (Grandma's meatballs), so I ordered 10 pieces, unfortunately the waitress understood my order as 10 portions (aka 20 pieces). We were brought a gigantic pot with very delicious meatballs, and Eric and I just though challenge accepted (after all very delicious things are always great to eat). And we started and they were as excellent as always -- same for the beer going with it from Augustinerbr\"au -- but even experts in big portions and quick eating habits like Eric and myself have to declare defeat at some point. Thus we offered our last two pieces to the gentlemen on the table next to us which they gladly accepted. They (and we actually ourselves) had fun to see if we actually would finish the full pot, I guess we clearly expressed our surprise and excitement once we realised our mistake in our order. One of them commented that once he saw what we tried to achieve he was very impressed on how far we got, considering we were actually not looking as we would tackle gigantic portions on a regular basis. Afterwards we just walked a bit through old town with visits of St Peter, the church of the holy ghost (Heilig-Geist-Kirche), the Asam church and Theatinerkirche before heading back to Eric's and Siyi's place having no space for dinner anyway but playing some Majhong where I turned out to be not so bad at for a starter.\\

Frauenkirche***, Michaelskirche****, Heilig-Geist-Kirche***, St Peter****, Asamkirche****, Theatinerkirche****\\

October 4: Munich:\\
On day two in Munich we considered doing one of the palaces: either Nymphenburg, or the Residenz. Usually on a sunny day I prefer to see Nymphenburg but this time two of us had to get somewhere else: to Heidelberg and to Wildbad Kreuth, thus we decided we should rather stay close to the train station. As I have already described multiple times before, the Residenz is gigantic. Some complain that a lot of the artefacts are shown in other rooms than originally, having been stored away in safe places and later brought back in a new shell after reconstruction. Other rooms have been completely reconstructed. Big halls and courtyards like the Renaissance hall of Antiquarium survived almost undamaged. Still no matter the state of work to make the glory reappear I personally like the result. Still the K\"onigsbau with the royal quarters was still in renovation (after almost a decade). Anyways I got back to the airport and hopped on the shuttle bus which brought dozens of other physicists and myself to the ISMD 2015 workshop where I gave an invited talk concerning recent QCD results from several experiments, including also Tevatron results. They were a bit puzzled when I asked them if I could show their results as well as results from three LHC experiments. If I am asked to give an overview talk it better be an overview and not only my personal experiment.\\

Residenz***\\

October 5--October 9: Wildbad Kreuth\\
The conference took place at Wildbad Kreuth, a venue belonging to the former Bavarian royal family, also famous as former meeting venue of the christian socialists in Bavaria. The place has its own swimming pool, a sauna, as well as its own bar. What rather caused issues was the missing stable internet connection in the meeting room, particularly bad if you want to have last minute edits on your talk. As explanation was given that private conversations would be better than online conversation. Also bad for myself, since I wanted to finish up applications while being on the conference myself. The social program was a cruise of the lake Tegernsee, unfortunately the weather was not that kind to us. But still it was nice while the weather was not rainy, the lake and the hills in fog are quite a sight too. Once it started to rain good coffee and cake sweetened the deal.\\

Wildbad Kreuth***, Tegernsee****\\

October 10: Melk\\
Stift Melk towering on a rock over the Danube river is one of the famous photographs which get people excited about Austria. But the Wachau region is a bit away from major cities (OK if you consider major to be beyond two hours). Nevertheless it is iconic enough that I could easily convince Eric and Siyi to see it on a weekend trip.The day before after my conference had finished, and they had finished work we met up at the airport rental car centre and took off. Two traffic jams later we arrived in Krems, had some sausage salad, and after breakfast we were ready to go. Even the monastery with the Kaisersaal and the large library are impressive, sadly one was not allowed to take photos in the library itself. I am obviously aware that the old books and papers are particularly delicate, but they are all behind real glass and if not photographed with flash, at least the art itself cannot be damaged. The church itself is very ornate late Baroque even with a gilded crown hanging over the high altar. Then we stopped by the garden pavilion for a coffee and cake.\\
 Then we continued our visit to Stift G\"ottweig, another large monastery in the Wachau area. There the stair case is very impressive, but also the guest rooms contain some precious tapestries. After having soup for lunch we had still some time to spare, thus instead of driving directly to Salzburg we did drive to Passau instead.\\
 Since this was at the height of the refugee crises border controls had been tightened and guess who they chose to pick - the rental car. Now in this car we have -- a german who didn't live in German anymore, an American on a normal German visa living in Germany, and a Chinese living in Germany on a very special high skilled researcher visa. As Chinese citizen Siyi needs a visa for almost any travel, and as astronomer she travels quite a bit, thus the poor border police man had to scroll through many many visa pages, and the German working visa is a very special kind. But we did obviously clarify that pretty easily, he was only very curious how in the world we three ended up knowing each other. Of course a good and fun story to tell. And 5 minutes later we were on our way to Passau.\\
 Passau is called the three-river-town, since the Inn, the Danube and the Ilz flow into each other by Passau old town. Unfortunately that leads to heavy flooding even nowadays regularly. The cathedral itself is outstanding, large, with many nice frescoes, a great dome with many stuccos and sculptures, and a large famous organ. And then back to Austria and Salzburg. Having arrived in Salzburg we asked what restaurants they recommended. The pizzeria they suggested had a birthday party going on with lots of Schlager music so we decided to pass and ended up in a rather high class restaurant. Food was very good, a tad expensive but not too much, thus good that we ended up there.

Stift Melk*****, Stift G\"ottweig****, Dom Passau*****\\

October 11: Salzburg\\
Salzburg had been the seat of one of the most important bishops of the holy roman empire. He could afford to hire a personal composer, one of them you might have heard of previously: some person called Wolfgang Amadeus Mozart. Well the Old Residenz is a fantastic series of Baroque rooms with lots of paintings, tapestries, and what not (well obviously no photos allowed as in almost all Austrian palaces). It is directly connected to the beautiful baroque cathedral too, another highlight of Austrian Baroque. But even more outstanding is the old castle of Hohensalzburg: one of the best conserved medieval residences on the continent. Never destroyed, the Golden Halls are some of the best conserved wood carvings from the late medieval era and it still houses old chimneys from that era too. We didn't want to risk getting held up on the border again, thus after having some food on the market square we made it back to Germany without using the highways (although we had a car with those privileges). On that day Siyi was also handed over a kitten -- the lovely and lively Oreo -- thus she only wanted to have dinner in one of the Beer-gardens close to the airport but didn't want to do the Erdinger brewery which is also quite close to the airport. Still an unforgettable trip, although Siyi and Eric might it remember more as the day they got Oreo.\\

Alte Residenz*****, Dom*****, Hohensalzburg****, Kajetanerkirche***

\section{November 8--November 15: Berlin and all of Germany}
\label{2015:Berlin}

Why did I go there: For once I wanted to see Madonna in concert, the choices for Germany were either Cologne or Berlin. My mum wanted to see her too and voted that she would rather prefer to see Berlin again over Cologne. My sister and my dad joined on this trip as well for a few days, so we had another family holiday happening.\\

November 8: Potsdam:\\
My parents and I arrived a bit earlier than my sister, so we decided that we should see the palace of Sanssouci (my sister is not that much into old churches or palaces). Thus we dropped our luggage by the hotel before getting on the train to Potsdam. Having arrived there we first had a guided tour through the New Palace (Neues Palais). Originally set up as palace for guests, the emperors made it their main summer residence, preferring Potsdam over the Berlin city palace. While the Marble Hall and the palace theatre were in renovation, I particularly enjoyed seeing the concert room, the silver chamber, as well as the grotto on the ground floor. Afterwards we walked through the gardens over to the main palace, built in Rokoko style for Frederick the Great of Prussia. Almost all of the room are still in their original state (main changes had been done only for the working and bed room of the king). Unlike in my previous visits, this time we were on the last visit of the day, and due to it being winter we saw the palace at night. The ambience was really special, considering the chandeliers and lights illuminated some parts of the rooms brightly, while others were completely in the dark. Then we met up with my sister, checked in and had some late dinner.\\

Potsdam: Sanssouci***** (Schloss***** \& Neues Palais*****)\\

November 9: Hamburg:\\
Hamburg and Berlin are well-connected by high-speed trains. Having never been in Hamburg, my mum and I decided we all should go there for a day. Considering that Germany's large particle physics centre is in Hamburg, I am still surprised I never made it there for work purposes. Clearly this was not supposed to change on this day, instead we started our day with a tour of the city hall. The city hall was built in a neo-renaissance style, to celebrate Hamburg's important past and to impress honorary guests like the German Emperor. The city hall is one of the largest (if not the largest) of Germany, and the rooms are indeed large and impressive, I enjoyed the Tower Room in particular, as well as the emperor's hall. 

Hamburg: City Hall*****, St Michaelis****, Port*****, Speicherstadt***, Elbtunnel****\\

November 10: Berlin:\\
Zeughaus (Museum of History)****, East Side Gallery*****, Madonna: Rebel Heart Concert*****\\

November 11: Schwerin \& L\"ubeck\\
Schwerin: Palace*****, Dom***\\
 L\"ubeck: Holstentor**,** Marienkirche****, Dom***\\

November 12: Berlin:\\
Schloss Bellevue*****, Bundespr\"asidialamt***, Siegess\"aule****, Brandenburg Gate****, Sovjet Memorial***, Reichstag*****, Sony Center****\\

November 13: Weimar, Erfurt \& Fulda:\\
Weimar: Herzogin-Anna-Amalia-Library****, City Palace****\\
Erfurt: Dom*****, Severikirche****, \\
Fulda: City Palace*****, Dom*****, St Michael****, St Blasius****\\

November 14: Rudolstadt \& Gera:\\
Rudolstadt: Heidecksburg****, Gera: Schloss Friedenstein*****\\

November 15: Berlin:\\
Schloss Charlottenburg***** (New Pavillon****), Museum Berggr\"un*****, Scharf-Gerstenberg Museum****
\chapter{Year 2006}
\label{2006}


\section{July 11-July 18: London}
\label{2006:London}

I like London, so does my family. My brother had just graduated high school and my last semester had just finished. Since exam season starts in late August, I rather preferred to have a trip right away, before starting studying. Thus my parents, my brother, and I decided to go to London once again. We had all been in London previously, thus it should be easy for us to get around.\\

July 11:\\
We started our tourist program of this London trip in Westminster Cathedral, the large catholic cathedral built in neo-Byzantine style, since the bishop didn't appreciate the gothic revival anymore. The side chapels are full of mosaics, on the main body of the church only one mosaic has been installed. If ever others had been planned, none of those have been realised, and it doesn't look like that would change anytime soon.\\

London: Westminster Cathedral*****\\

July 12:\\
Last time we had seen Hampton Court Palace, thus we opted to visit Windsor Castle instead this time around. Windsor Castle is still inhabited by the royal family, but the state apartments can be visited all year round. The medieval keep still remind the visitor of the old roots of the castle, so does the high gothic St George's Chapel as well. In the afternoon we admired the royal tombs in Westminster Abbey. By this time they had already forbidden photography, abolished the free Wednesday afternoon, and increased the ticket from 2.50 pounds to over 15, but at least the nave could be visited for free. Already in the 2010s that has been abolished, the ticket price has gone to almost 20 pounds, photography is all forbidden - all claimed to be a sign for spirituality. Seems to be the norm for anglican cathedrals, just saying... Then we wanted to walk around the parliament. Once we passed Westminster Hall, my mum saw, that we could witness a parliamentarian debate. Only having passports on us was suited enough to let us in, I cannot imagine that happening anymore. We also had time to admire the great ness of the medieval Westminster Hall. Just like for all parliaments in the world, we had to lock in our phones and cameras, sit quietly on the visitor benches and listen to the debates for about 30-40 minutes. This time the debate happened in the House of Lords. While this chamber is the less powerful of the bicameral UK system, the meeting hall is clearly the more beautiful, the original of the House of Commons had been lost in the 40s, when German bombs hit Westminster Palace.\\

Windosr: Windsor Castle*****\\
London: Westminster Abbey*****, Houses of Parliament*****\\

July 13:\\
And we decided to have a quiet day in between, spending it in the parks of London, Hyde Park in particular, starting out by speaker's corner and the Marble Arch. The park is really large, with lots of squirrels running around, little lakes, large lawns. The Diana fountain is a bit of a let down (or not as beautiful as I hoped it would be). We also walked over to the Albert Memorial and Kensington Gardens, and then we finished our walk by the Wellington Arch. Meeting up with my friend Clive the evening for drinks and dinner.\\

London: Hyde Park***\\

July 14:\\
After having been in the Tower of London already twice before, I decided to skip it this time, handed over my camera to my younger brother, and decided to stroll a bit through the City of London instead. Afterwards we all met up and walk across to the city hall, visiting with the Southwark cathedral another large gothic church, before we ended up by the Tate modern gallery, which is a transformed former power station. I do enjoy spending time in this modern art museum quite a bit, whenever I can in London.\\

London: Southwark Cathedral****,  Tate Modern****\\

July 15:\\
One of the most important classical baroque ensembles in England is the so-called Maritime Greenwich ensemble, comprising of the former Marine hospital and Queen's House. Both the chapel and the Painted Hall had been open for the open days of the university/college of Greenwich. My mum assumed that you could only go inside should you be a guest of the open days and announced her desire to study either history or politics. Since german and english A-levels are treated equivalently in theory she could have gone for it (didn't show much more enthusiasm for starting late studies shortly after though). It was my first time in the Painted Hall, indeed one very fine hall with giant frescoes. Nowadays (2019), the hall is always open for visits. The chapel was alright, I still don't think Queen's House is something really special, most of the interior is long gone anyway. And then we walked over to the observatory, standing on the Greenwich Meridian (not that I get the craze for that one, unlike the equator there is no physical reason behind it being in that particular spot).\\

London: Greenwich (including the Painted Hall, the Chapel, and the Royal Observatory)****\\

July 16:\\
The British Museum is always a highlight of a London trip. No matter if you like old Egyptian, Greek, Mesopotamian, Roman, Buddhist, Islamic, or Persian art, the British museum has artefacts for you. Also rooms full of marbles from the Parthenon, which Athens would love to get back to put them into the acropolis museum though. And on top of it all, you don't need to worry about ticket prices, since the museum is all for free (just like the large natural science museum, or also the national gallery).\\

London: British Museum*****\\

July 17:\\
The second largest dome of a church in Europe, offering the most superb views of the City of London. St Paul's used to dominate the London cityscape for centuries. Now overshadowed by many skyscrapers, the interior is still very nice to see (albeit a bit pricey, after all it is an Anglican cathedral). Back then photos could be taken, that has changed as well nowadays. Anyways you can also find Wellington's and Nelson's tombs in the large crypt.\\

London: St Paul's Cathedral*****

\section{October xyz: move to Meyrin and the CERN hostel}
\label{moveMeyrin}

And this marks the first move to a different place in my life: For my master thesis I stayed at the CERN hostel in Meyrin in order to be close to the group I would be working with over the course of the next 5-6 months. I had my single room, with a shared kitchen, no TV but internet (the usual a physicist needs). Watching youtube got me a warning since the signature at least back then was similar to p2p file share usage, and since that was mainly used to illegally share music or movies it was clearly not allowed. It felt like starting a job and all of that abroad.

\section{October 28: Geneva}
\label{2006:Geneva}

Having arrived in Meyrin just a few days before, the closest big destination is Geneva. The harbour of the city is characterised by it large 150 m high water fountain, the so called Jet d'Eau. It is usually on from 6 am to midnight, in winter or in heavy winds it is usually switched off. In summer time the sun often creates a rainbow. The cathedral is one of the largest churches in Switzerland, particularly the early Romanesque stone masonry and capitals are of value. Beautiful neogothic windows can be found in a side chapel, the towers can be climbed too with nice views of the harbour.\\

Geneva: cathedral****, Jet d'Eau*****

\section{October 29: Geneva}
\label{2006:GenevaII}

Geneva has many parks with former Villas, one of those is the Villa Ariana, containing the local porcelain and ceramic's museum. More interesting if the Palais de Nations constructed as headquarters of the League of Nations it is still the second largest seat of its successor the United Nations. Many meeting rooms can be visited, the old former main assembly of the league, as well as the current general assembly hall. I enjoyed the modern conference hall with its beautiful painted and sculptured ceiling,\\

Geneva: Villa Ariana***, Palais de Nations*****

\section{November 10: Geneva}
\label{2006:GenevaIII}

Geneva's former city hall is now used by the Great Council of the canton. Several meeting rooms in Baroque or Baroque revival style exists and can be visited on tours during the local Escalade city festival. Otherwise no tours of the Hotel de Ville are offered, but the nice courtyard is always accessible. The large neogothic church of Notre Dame is the largest catholic church in town. Although once famed as protestant Rome nowadays catholics are in the majority, and there are even efforts ongoing to create a bishop seat in Geneva again.\\

Geneva: Hotel de Ville****, Basilique Notre Dame***

\section{November 25: Chamonix - Mer de Glace}
\label{2006:Chamonix}

After having started to work at CERN only a couple of weeks early, my parents and my younger brother visited me once more, this time they had a visit of Chamonix in mind. I'll never forget the first up-close view of the snow covered Mont Blanc massive just after getting through the tunnel by Les Houches dominating the full view just ahead of you. The Aiguille du Midi cable car was closed for its yearly inspection, thus we opted to take the train up to Montenvers by the Mer de Glace. The view of the glacier looked absolutely magnificent and the glacier ice was still shiny blue (far shy from the dirt and grey gravel you see nowadays). As the ice cave had been closed down as well, we all decided to walk along the glacier edges and to take the ladders down as far as we could (only to climb them up again and then getting back into Geneva). All in all a really great introductory visit to the valley of Chamonix.\\

Chamonix: Mer de Glace*****\\
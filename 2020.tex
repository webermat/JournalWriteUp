\chapter{Year 2020}
\label{2020}

\section{December 29, 2019-January 5, 2020: East Germany and Poland}
\label{2020:GermanyPoland}

Co-travellers:\\
This is in fact the continuation of a trip started late December with my dad.\\

January 1: Bad Muskau \& Swidnica\\
Clearly the day started really early witnessing the New Year fireworks in Dresden by the meadows of the Elbe river with a view of the Frauenkirche, the cathedral, and the residence. We had a short breakfast by the Autobahn very close to G\"orlitz, then continuing to Bad Muskau and the F\"urst-P\"uckler-Park. The park spans over the border to Poland on both sides of the Lausitzer Neisse, considered one of the most beautiful English landscape park created during the late 19th century. The new castle had been devastated in the last months of World War II, the park had dropped into despair particularly on the Polish side after the war. After the opening of the iron curtain and the German reunification, the new castle had been rebuilt, and the park had been cleaned up. Nowadays the park is close to its original state, bridges have been rebuilt and both parts of the park have been added to the list of UNESCO world heritages. During winter the museum about the life of F\"urst-P\"uckler is closed, thus we spent an extended walk through the park. The most famous view point of the park is the new castle with the Lucie lake. Artificial little river beds, islands, large forests, and alleyways are scattered throughout the park. My dad and I crossed over the bridge to the Polish side, his first time in Poland. The polish side is less elaborate than the German side, but the area of the park on the Polish side is more extensive than the German side. Crossing back over to Germany we stopped on the Polish side, realising the gas prices dropped substantially. One of the long roads on our way to Wroclaw was in a desperate state, one of the worse roads my dad had driven on. The other lane of the road was already renovated, our lane was also partially renovated, so maybe in a few years the road will be in a better state. \\

Bad Muskau/Leknica: F\"urst-P\"uckler-Park****\\
Swidnica: Peace Church*****, Cathedral****\\
Wroclaw: Main Square with City Hall*****\\

January 2: Wieliczka \& Krakow:\\
Like in 2012 I decided that Krakow is worth a visit. Since back then our train was heavily delayed we didn't get anywhere close to the Wieliczka salt mine as we originally had hoped for. Thus this time around I decided to go there first. Since the train ride for two people was cheaper than going by car, considering gas and tolls and parking fees, and also more relaxed, my dad and I got on the earliest train out of Wroclaw to Krakow. There are two ways to book online tickets for Polish trains, one way is to buy it from the polish rail webpage, unfortunately this page has so far only a Polish version. There exists an English speaking page, where high speed trains can be booked from, but at 50 \% larger ticket prices. Thankfully I have Polish friends, thus they guided me through the polish booking page (actually they confirmed that the translated google feed of the page was correct). Unfortunately the train from Krakow to Wieliczka had been replaced by a bus, and it took us about 20 minutes to find out where the bus would leave from exactly. But we did make it after all in time for our pre-booked tour. The salt mine of Wieliczka has two parts open to tourists, we took the so-called tourist tour, which leads through parts which had been closed for operations in the late 1980s. Many cavities are carved into the salt rock, many statues and chapels can be found along the trail. The highlight is the Kinga chapel carved out in 1896. They even have their own version of Da Vinci's supper carved out of the salt. There are several layers of the mine which can be visited, going deeper than 100 metres, also salt lakes and canals and grottoes are located on the deeper levels. The highest chambers is the Mikatowice Chamber, where several generations of miners worked on. After our tour we were offered another tour of the salt mine museum, unfortunately we would have needed to wait about 45 minutes before the start of the next English tour, thus we opted out of that tour, getting up in one of the tiny shaft elevator baskets. By now trains were running again, before we had a short snack in front of the train station of Wieliczka. Once we arrived in Krakow we walked from the central station over to the Royal Palace on the Wawel hill. Unlike in late September, there were no large lines for tickets in January, but the private royal apartment tours had been sold out, we decided to see thus only the State Apartments. Before we visited the cathedral instead.\\

Wieliczka: Salt Mine*****\\
Krakow: Wawel Cathedral*****, Wawel Palace****, Mary's Basilica*****, University Church St Anna****, St Barbara***, Jesuit Church with Pantheon****, St Andrew**\\

January 3: Wroclaw:\\

Wroclaw: Centennial Hall****, St Cyril \& Methodius***, University Church*****, University***** (Aula Leopoldina*****, Oratorium Marianum****, Mathematical Tower***), St Matthew***, Cathedral (with Chapels)*****, St Magdalene with dancing Craddle****, Synagogue**, Royal Palace****\\

January 4: Walbrzych:\\

Walbrzych: Zamek Ksiaz (Schloss F\"urstenstein)*****\\

January 5: Sanspareil:\\

Sanpareil: Rock Garden*****\\

\section{January 11: Jura \& Bourg-en-Bresse}
\label{2020:BourgEnBresse}

The weather forecast for this weekend was excellent, thus Sasha decided we should consider a short trip. I suggested we could visit the royal monastery of Brou but since this would only take about 1-2 hours, it might not be worth the drive alone. Thus we checked the close-by area. All caves in the Jura seemed to be closed until April, but gorges and waterfalls are always accessible. We found two nice spots to add on our itinerary.\\

Co-travellers:\\
Sasha: our fourth day trip together. Just like previously with Riju and Eric after leaving my working group Sasha is still up for trips. \\
Juno: after getting a new camera from Fuji Juno is eager to check how well the new photos will come out, considering that we all three have different models we will find out which one we prefer after all\\

Starting out at CERN by 8 we crossed over mountain paths to the other side of the Jura mountains. We had quite a big of fog on our way, so I was pretty happy when we arrived by the Gorges de la Langouette and found out that the whole canyon was below the fog. Before the deep canyon starts the river drops by an impressive waterfall. Upstream several little waterfalls pave the path, ending by a small water power station with another large waterfall. The gorge is a bit longer than 1 km long, thus within about an hour a round trip is enough to see most of it. Just about 15 minutes later we reached our second destination, the pertes de l'Ain. After a very short gorge the Ain river drop about 15 metres deep and disappears for about 100 m and reappears in a large waterfall. Most rocks of the canyons are covered by moss. The flow of the Ain is nowadays regulated by a reservoir upstream, at times the water flow can be quite substantial and the level can rise quickly. Nevertheless even with a regulated flow the final waterfall is pretty broad and I was impressed. We had our lunch stop at Champagnole, Sasha and I tried the Morbiflette (very similar to Tartiflette, but with a different cheese) and Juno opted for Tartichevre. I was happy with my choice at least.\\
 On the way to Bourg-en-Bresse we passed by the signs for the Cascades du Herrison, but after we realised it would be a detour of about an hour we decided to move on to Bourg-en-Bresse. The royal monastery of Brou is very impressive. The church is large but most of the nave is sparsely decorated. The stain-glass windows of the choir are very nice as well as the choir stalls. The most impressive pieces are though the three royal tombs of Margaret of Austria, Philibert the handsome and Marguerite de Bourbon. Besides the life-size statues of the deceased holy figures, roses, animals are depicted as well. In the chapel of Margaret an impressive stone carved retable can be found as well. The adjacent museum of the monastery housed a couple of nice paintings, statues and old alters and tapestries. And about 90 minutes later we arrived back at CERN by about 6 pm.\\ 

Les Planches-en-Montagne: Gorges de la Langouette****\\
Bourg-de-Sirod: Pertes de l'Ain****\\
Bourg-en-Bresse: Monastere Royal de Brou*****\\

\section{February 8: St. Gallen, Reichenau, and Constance}
\label{2020:Reichenau}

Sam and Vivan have never seen Germany. I clearly had to change this. As options I gave them the choices of a trip to Freiburg im Breisgau, or a trip including the monastery island of Reichenau. They preferred the later, and I got Dylan and Grace on board as well.\\

Co-travellers:\\
Sam, Vivan, Grace, and Dylan: all US-American grad students working in the field of particle physics. Grace loves going to Germany so was easy to convince to see it again, Dylan, Sam and Vivan are used to my trips as well, and some training ahead of our Rome trip a couple of weekends later might not be the worst idea either.\\

We all got our day tickets and started on the 6:30 early train over to St Gallen. There we first started with the Rokoko monastery library (I felt old being the only one who didn't get a student discount), still a bit underwhelmed that ``conserving the art'' is misused as excuse for the no photo policy. Nevertheless everybody enjoyed it, as well as the previous cathedral, the former monastery church. Unfortunately this time the choir had not been opened for visitors (sometimes if you are lucky you can enter it on weekends). A baptism was going on while we were there. Then most of us got the St Galler Sch\"ublig as snack, before getting on the train, where I tried to teach Sam and Dylan how to play Jassen. Once we arrived in Constance, I rushed to the ticket machine, and just in time got us the regional tickets for Reichenau station. For reasons unknown to me, Deutsche Bahn is not capable to allow you buying these tickets online, and a 4 minute interval to switch and get tickets on the completely inadequate horribly coded touchscreen interface of the German ticket machines doesn't make things easier as well. But we did make it to the train, then made it to the bus (although the bus station had been shifted by 30 metres, which we almost missed as well), and we got off by Peter \& Paul first. Even the little museum room next to the church was already open. All three churches of the island, St Peter \& Paul, the monastery minster, and St Georg are of Romanesque style, with frescoes from the 10th to the 13th century (as well as baroque modifications and frescoes), including some of the most precious large-scale german paintings of that time.

St. Gallen: Library****, Cathedral****\\
Reichenau: St Peter \& Paul****, Monastery****, St Georg*****\\
Constance: Minster****

\section{February 20-March 2: Rome \& Italy}
\label{2020:Rome}

Why Rome or Italy:\\
I always love going to Italy, particularly Rome is always a highlight (having been there already three times). This year is a leap year, and I realised I hardly ever did a larger multi-day trip in February. Rome is very well connected to many regions and places, also many towns I had never seen so far yet, but which seemed to be very interesting to visit. Many of my friends had never been to Rome, not even in Italy. Thus when I proposed to spend a couple of days in Rome, I convinced four of my friends, Dylan, Vivan, Sam, and Bryan to join me. While I was in Italy the Corona outbreak started in Lombardy, a couple of cases had popped up in Rome as well as in Geneva at that point. Thus while I considered to remain in Rome, only Bryan joined me for these five days. \\

Co-travellers: Bryan:\\
US-American and grad-student at CERN, having been in Switzerland only for a short while, this will be Bryan's first trip with me, also his first trip to Italy and Rome.\\

February 20: Rome \& Orvieto:\\%20

Rome: Santa Maria Maggiore***** (Loggia***), Sant'Alfonso De Liguori**, San Paolo entra le Mura***, San Carlo alle Quattro Fontane****, Fontana dell'Acqua Felice****, Santa Maria della Vittoria*****, Santa Maria degli Angeli e dei Martiri****, Palazzo Massimo*****\\%15
Orvieto: Santa Maria dei Servi**, Sant'Angelo**, Duomo*****, San Bernardino***, Santi Andrea e Bartolomeo**, San Giovenale***, Orvieto Underground*****\\%5

February 21: Urbino \& Pesaro\\%10

Urbino: Santo Spirito****, Oratorio di San Giuseppe*****, Oratorio di San Giovanni Battista****, San Domenico**, Santa Chiara***, Oratorio della Santa Croce***, Palazzo Ducale*****, Santa Caterina***, San Francesco**\\%9
Pesaro: Duomo\\%1

February 22: Castel Gandolfo \& Anagni\\%8
I had booked tickets for the Apostolic Palace and the Villa Barberini in Castel Gandolfo, only to realise later, that the walk through Villa Barberini would need to be done with a guide, which thus could not be done in as short of a time as I had hoped for. Thus I feared I would not be able to actually see the gardens after all. I took the earliest train from Rome to Castel Gandolfo, arriving by the volcanic crater lake of Lago Albano in early morning sunshine. I walked up to the old town of Castel Gandolfo having a short look into Bernini's church of San Tommaso da Villanova, a cute little Baroque domed greek-cross shaped church. Once the ticket office opened, I realised although the walking tours of Villa Barberini starts late in 11 am, bus tours are offered starting at 9 am though. I was able to upgrade my walking tour ticket to a bus tour ticket. The Villa Barberini is a vast park, with many alleyways, French style Baroque parterres, fountains, statues, as well as a farm, and a heliport. Within the gardens are the remains of the Imperial villa of the Roman Emperor Domitian. The tour lasted about 1 h with several short stops where the English audio guide told us more details about the different items. Once the tour was finished I continued with a visit of the Apostolic Palace which had been originally built for Pope Urban VIII by Carlo Maderno. In late 2015 the Pope decided not to use the palace for private stays anymore and the palace had been opened to the public in late 2016. Throughout centuries Castel Gandolfo had been the summer residence of several popes, maybe of the rooms had been regularly refurbished. The first part of the visit leads through the Popes' museum which displays many private items of the popes, chairs, tiaras, portraits, as well as writing sets, uniforms of the Swiss guard, and limousines of the Popes through the times. Most of the state rooms are rather modest, but also decorated with precious clocks, tapestries, as well as paintings. The study of Pope Benedict XVI, his secretary, and his bedroom are on display as well (rather modest I have to say). I was particularly surprised by the rather modern style of the private chapel of the Pope. The Gallery of Alexander VII instead is a traditional Baroque loggia with landscape frescoes on the walls. On the way to Anagni I had a short stop in Ciampino, where I had freshly made pizza in one local store by the piazza of the church of Sacro Cuore di Gesu. \\

And then I got on the train to Anagni, where I had to transfer to the local bus bringing me to old town. No real schedule had been posted online, just a statement that a bus would leave from the station roughly every hour shortly after the train arrives. Indeed it left just about 5 minutes after the train had arrived. Stations were not displayed online, just a tiny sign showed where buses would stop. I just got off at the edge of old town and decided to start exploring the town. There were many nice views, little narrow roads, gates, city walls, little churches here and there. I had a short coffee and made my way to the Duomo. The cathedral was built in early medieval times in Romanesque style, most of the cathedral had been refurbished in gothic style in the 13th century. Most of the original frescoes have been lost. The museum of the Duomo holds many treasures from medieval times, particularly from the 13th century, when Anagni had been the summer residence of the Popes before the popes moved to Avignon in 1309. The real treasure of the church is the crypt. Starting with an oratory of Thomas Becket, with almost completely faded frescoes from 1231-1255, the largest room of the crypt is dedicated to San Magno, the patron of Anagni. The mosaic floor as well as the well-conserved frescoes originate from the 1231-1255 as well. The frescoes are some of the best conserved in Italy and have been recently restored, definitely worth to pay to see them. In fact while the church itself might be nothing special, the crypt (both the frescoes and the floor) are worth a detour to Anagni. The second important building in Anagni is the summer palace of Pope Bonifacio VIII, which holds some original frescoes in four large halls. Here the pope was held prisoner for two days by the Colonna family on request of the French king, before the citizens of Anagni freed the Pope. After all of those events the French king proved to be victorious and the Popes left Rome and Anagni, never to return to Anagni. After seeing one more church I walked along the main road until I finally found one of the tiny signs for the bus. I assumed that the bus schedule would be related to trains leaving from the train station as well and I was there in time, to be sure not to miss it. After a wait of about 30 minutes indeed a bus arrived and brought me back to the train station, and then about an hour later I arrived in Rome where I had some Spaghetti Carbonara.\\

While so far I would have thought catching the bus from Anagni old town to the Fiugi-Anagni train station would be the thing to worry about on this trip (at least transportation wise), the first Covid-19 case in Lombardy appeared, which a couple of days would jeopardise this trip far more than I had imagined only a couple of days later.\\

Castel Gandolfo: Lago Albano****, San Tommaso da Villanova***, Villa Barberini*****, Palazzo Pontifico****\\%3
Ciampino: Sacro Cuore di Gesu*\\%1
Anagni: Madonna di Loreto***, Duomo***** (church***, crypt*****), Palazzo di Bonifacio VIII****, San Giovanni De Duce***\\%4

February 23: Terni, Marmore, and Rome\\%9

Terni: San Francesco***, Big Press Monument***\\%2
Marmore: Cascata della Marmore*****\\%1
Rome: Piazza del Popolo****, Santa Maria dei Miracoli***, Porta del Popolo****, Villa Giulia (National Etruscan Museum)****, Fontana delle Conche**, Santa Maria del Popolo***\\%6

February 24: Pienza \& Val d'Orcia:\\%4 
The plan for this day was to see Pienza during the morning, and to stay in Montepulciano for a quiet afternoon and having a wine tasting there. Sitting though on the earliest train at 6:12 I got already stuck on the way. In a tunnel between Roma Tiburtina and Orte the train stopped, and we were told that there is a technical issue, which means we have to stop for about 10 minutes. After another 15 minutes of not moving, we were told the problem is tougher, and we would have to wait for longer (by that time I knew I won't make it to Pienza in the morning for sure). The air conditioning was switched off a short while later. Sitting in a non moving train in a tunnel, one could feel the pressure wave of incoming trains about 1 minute before it passed by. Each 10 to 20 seconds another peak arrived shaking the train, even after the trains passed, the pressure wave amplitudes were noticeable for about 30 seconds later, and then silence again. The ticket controller went up and down through the whole train phoning to find possible solutions. After about 45 minutes later the power went off. Light was only available through smart phones anymore. The only light still illuminated were red emergency lines just by the doors. Another 15-20 minutes later the ticket controller went through the train with flashlights, and we were told to move to the front of the train. Once we arrived there we were told what the issue was actually: it seems all doors couldn't be opened anymore and each of them would have to be manually opened by an override switch. A train mechanic had arrived, but seems the issue was more drastic than assumed, and we would move on to the next train station and would need to get off there. And soon after the train finally moved on, we finally saw light again, and 10-15 minutes later we arrived in Orte. By now it was about three hours later, a solution we might have been able to have already 3 hours earlier. We were told the next train along our line would arrive just about 10 minutes later. It was announced that this train would be by now 10 minutes delayed too, our stuck train definitely created quite a bit of chaos. Every 10 minutes the train was more and more delayed, in the end even this train departed with over one hour delay.\\

 Once I arrived at Chiusi-Chianciano Terme I figured out if I would make it to Montepulciano in time for a timely connection to Pienza. Seems that was possible, but just in time, so lunch had to be more or less cancelled and the trip would not be relaxing. The bus to Pienza was also full with school kids - in fact I seemed to be the only person on board who wasn't a school kid, and we even had a ticket controller getting on board midway (and one of the kids had no ticket), anyways I made it to Pienza. There I walked through the city, stopping for a short visit of San Francisco, before visiting the cathedral. The Duomo was alright, a decent Renaissance cathedral but nothing special. Pienza is considered one of the first planned ``ideal'' Renaissance towns in Italy, quite nice but not that I understand why it is on the world UNESCO heritage list. I booked a tour through the Palazzo Piccolomini, which was rather a self guided tour, where we listened to an audioguide. The ``guide'''s job was checking that nobody took photos. The palace itself is pretty decent, particularly the bedchamber and the chamber with leather wall decorations. The bed was very nice and carved. After all of that I decided to walk around the paths through meadows and cypresses of the Val d'Orcia. The nature is really very nice and beautiful to walk through, also with the villages and towns on the hill sides, and Pienza is nice to look at, sitting over the valley as well. The nature is really breathtaking, it was also used as background for many movies, e.g. the Gladiator. Alone the views of this valley are worth a trip to Pienza. I had a couple of small pizza slices before getting on the bus back to Montepulciano, another bus (with a few coffees there) and a train later I arrived back in Rome. All in all a short visit to Tuscany with most of the time sitting in trains (and a dark tunnel ahead of Orte). And in order to make me even more happy, Trenitalia informed me that my train had been cancelled, but I could take another train about 90 minutes later without needing to pay an exchange fee.\\

Pienza: San Francesco***, Palazzo Piccolomini****, Duomo****\\
Val d'Orcia*****\\

February 25: Siena \& San Gimignano\\%16
Once again I sat on the earliest train to Chiusi-Chianciano Terme, this time successfully on time. Once I arrived there, I asked the lady at the Trenitalia counter, if I could exchange my ticket for the cancelled train for an earlier than the proposed one. Indeed that was possible, unfortunately she couldn't tell me would happen to my reserved trains the next day, when I had booked the same train back to Rome from Florence. Anyways I hoped on the Trenitalia bus which brought me to the train station of Siena. I had been twice to Siena already, both times I was impressed by the cathedral, thus I also planned to see the Duomo this time. Unlike last time I followed the guide posts from the train station to get to old town instead of google maps. Instead of walking up the hill this meant standing on escalators which brought me up to the hill top just a couple of metres away from the gate of Porta Camollia. On my way to the Duomo I had a short sneak peak into the churches of San Pietro Alla Magione, Santa Maria in Portico a Fontegiusta, Sant'Andrea, and San Cristoforo, past gothic Palazzo Salimbeni, and the Loggia della Mercanzia. Out of these places the church of Palazzo Salimbeni is for sure the most popular photo stop, the church of Santa Maria in Portico a Fontegiusta is the one which I would pay a short visit to again.\\
 Next to the cathedral is the old hospital of Santa Maria della Scala (also home of the ticket office of the Duomo). In order to avoid long lines, I had pre-booked a ticket which would give me access to all sights close to the cathedral. Unfortunately I realised only after I paid, that the hospital closed on Tuesday (only in February). Surprisingly tickets were sold for that day though. Once I arrived to exchange the voucher for a ticket I was informed that the ticket for Santa Maria della Scala would be valid for the next day, clearly nobody gave any thought to the fact, that some people might just not be in town for two days. I stood in line of the Duomo and made it into the cathedral as first person of the day. The cathedral is one of the most beautiful large gothic churches in Italy, in my opinion it even beats the Duomo of Milan. The floor with many stone inlays is the absolute highlight of the cathedral. In summer it is possible to go up close to the roof and then this offers even more superb views of the inlays, which depict many stories of the bible, like the murder of the children of Bethlehem, or the death of Absalom after he lost the battle against the troops of his father, king David. The dome of the church is dominated by six giant gilded statues and a line of frescoes of holy figures. The choir itself is decorated by several baroque frescoes. There are three more parts of the Duomo which can be visited with a special extended ticket: the Piccolomini library (also several precious books are on display there), the crypt of the church, and the baptistery. All three places are definitely worth the extra fee, the frescoes are amazing, as well as the bronze statues and plates of the baptistery. The originals of many statues of the cathedral and the stained glass of the large choir window have been replaced by copies by now, they are now on display of the cathedral museum. From the balcony of the unfinished Duomo Nuovo (part of the cathedral museum nowadays) one has the best top view of the city, the Duomo, and the Piazza del Campo with the Palazzo Pubblico.\\
The Casa Santuario di Santa Caterina is a complex where st catherine of Siena lived at with her parents and 24 siblings. The complex houses two oratories, the Oratorio delle Cucina is really nice, and the church of del Crocifisso. Since my originally visit of Santa Maria della Scala didn't work out, I decided to visit the Palazzo Pubblico instead for a second time. This palazzo is the city hall of Siena, but many of the old rooms and chapels with old murals and frescoes dating as far back to Gothic times can be visited as part of the Museo Civico. Nowadays these rooms are used by the Mayor as well as marriages. The tower can be climbed as well. On my way to the central bus station I got a bit lost and ended up by the basilica of San Domenico (just about 150 metres away from the central bus station though). The terrace in front of San Domenico offers the best views of opposite hill side which is dominated by the Duomo and the tower of Palazzo Pubblico. The interior of San Domenico is rather modest, with modern stained glass windows. I got on the bus to San Gimignano well in time (once again with many school kids).\\
San Gimignano is famous for its many medieval towers, which were a sign of the wealth of the rich merchant families. Other Italian towns had a similar outline in medieval times, e.g. Bologna, where only two towers remain nowadays. Instead, San Gimignano managed to conserve fourteen large towers, the main reason why the walled old town was declared a UNESCO world heritage site. I got myself the San Gimignano pass, which covers the Duomo, the Palazzo Comunale with the Torre Grossa, San Lorenzo in Ponte, and the Museum of Fine Arts, as well as the Museum of Archaeology. I climbed the Torre Rossa first, with 54 m the highest of the town towering next to the Palazzo Comunale. The view of the town and the surrounding hills from the top are pretty nice from up there. The Palazzo Comunale is pretty nice too, particularly the chamber of the Podesta and the Council Hall. Next I viewed the Collegiata di Santa Maria Assunta, the so-called Duomo of San Gimignano. The church walls are decorated by many 14th century frescoes, which depict a Poor Man's bible with Old and New Testament cycles, including stories such as the creation of men, the pharaoh and his army after drowing by the flood, the last supper, and the kiss of Judas, as well as the last judgment. I was pleasantly surprised, in fact the Duomo was an unexpected highlight of San Gimignano. Afterwards I left old town and walked down into the valley to have a nice view of the hillside with old town and all towers. Afterwards I went back and saw the arts gallery, and the old church of San Lorenzo in Ponte with its nice frescoes (unfortunately not quite as complete as those of the Duomo). Then I had a selection of local Tuscan cheese and dried meat for dinner, before getting on another bus, a train which was delayed by another half an hour and finally on the train back to Rome after another hour of wait (having a coffee during the wait and one more Panini).\\

Siena: San Pietro Alla Magione**, Santa Maria in Portico a Fontegiusta***,  Sant'Andrea**, San Cristoforo, Duomo***** (Crypt****, Biblioteca Piccolomini*****, Baptistery****), Cathedral Museum****, Saint Niccolo in Sasso****, Oratorio della Camera*, Oratorio delle Cucina****, Chiesa Del Crocifisso***, Palazzo Pubblico*****\\%11
San Gimignano: Palazzo Comunale****, Torre Grossa***, Duomo*****, San Lorenzo in Ponte****, Galleria d'Arte Moderna**\\%5

February 26: Florence\\%16

Florence: Santa Maria Maggiore***, Uffizi Gallery*****, Palazzo Vecchio*****, Duomo**** (Dome*****, Crypt***), Biblioteca Medicea Laurenziana****, Battisterio Laterano*****, San Michele Arcangelo Visdomini*, San Lorenzo****, Cappelle Medicee****, Santa Maria Novella*****, Santa Croce*****, Boboli Gardens*****, Palazzo Pitti*****, Santa Felicita***, Santissima Annunziata****, Cathedral Museum****\\%16

February 27: Rome\\%10
Originally we all planned to spend this day on the road, visiting old Etruscan tombs, and the Villa Farnese in Caprarola. Instead Bryan and I improvised a bit and saw many other places in Rome itself. First we dropped our luggage by our hotel, since today I was supposed move with all of them to a hotel just opposite of the opera house. We got one room for each of us, and then after getting a 24h public transport ticket we took the metro to Laterano. There we took some shots of the Roman gate of Porta Asinaria and the Porta San Giovanni, which has been constructed once the Porta Asinaria became too small for traffic. Opposite of the Palazzo Laterano is the Sancta Sanctorum, which houses the Scala Scanta, the holy stairs. According to the legend these are the stairs of the Pitate's palace where Jesus walked up for his trial. They were brought to Rome later on. If people walk up the stairs on their knees praying, the catholic church states that a couple of years are forgiven (and yes, that's what I have done previously). On the side stairs frescoes of old testament legends, like the story or Samson are displayed. At the end of the stairs is the chapel of San Lorenzo in Palatio. This chapel is the only remaining part of the medieval Papal Lateran Palace. The ceiling depicts paintings of the four evangelists, on the walls the frescoes depict Popes, the frescoes close to the windows depict old legends, like the stoning of St Steven's, on the wall is a relict of Jesus' chair of the last supper. We walked past the egypt obelisk Lateranese over to the battistery with its old ancient mosaics, and its paintings and the gilded ceiling. The cathedral of San Giovanni in Laterano was closed until 1 pm, thus we decided to take the metro and a bus to Villa dei Quintili first.\\
The Villa dei Quinitili is one of the best conserved large villas by the Via Appia Antica, also the first large Roman ruins I showed to Bryan. We decided to walk along the Via Appia Antica over to the Mausoleo di Cecilia Metella and the Castrum Caetani, continuing to the Circo di Maxentio (still closed) and the Basilica of San Sebastiano with the marble statue of San Sebastian and the wooden ceilings with carved depiction of San Sebastian and the papal crest. We found out that unfortunately the catacombs had been the first sights to be closed down due to the corona virus outbreak. And we took a bus back to the Lateran for the highlight of the day: San Giovanni in Laterno, the cathedral of Rome. For once the transept and the choir were accessible, thus I saw the backside of the exquisite high altar. The treasury was cute, with many golden crosses, ivory cases, as well as capes, altar cloth etc. The giant marble statues are always breathtaking, as well as the wooden ceiling. The cloister (even a first time for me in Rome) is nice as well with lots of pillars, as well as remains of old Roman ceiling.\\
Then we took the tram to the Villa Torlonia, a late classicist villa, which had been the private residence of Benito Mussolini during the Fascist time in Italy. The private quarters on the upper floor of the Casino Nobile are a mesh-mash of romantic and neo-baroque styled rooms. Rather interesting is the Casina delle Civette, particularly the Art Nouveau style stained glass windows, depicting owls, swans, vineyards, other birds, roses, etc etc. The gardens with artificial ruins, palm trees and obelisks are nice for an evening stroll as well. Just a couple of minutes away on foot (even with a short stop for coffee in between) is the old Roman mausoleo of Santa Constanza with several nice late Roman ceiling mosaics. Here we once again were faced with the trend to disable all lights unless it was paid for. We ended the day in the church of Santa Maria degli Angeli e dei Martiri, which is the transformed tepidatirum of the baths of Diocletian.\\

Rome: Porta Asinaria***, Scala Sancta**** (with San Lorenzo in Palatio*****), Baptistery****, Villa dei Quintili****, Via Appia Antica***, Mauseleum der Cecilia Metella***, San Sebastiano Fuori le Mura****, San Giovanni in Laterano***** (Cloister****, Treasury****), Villa Torlonia***** (Casino Nobile****, Casina delle Civette*****), Santa Constanza****, Sant'Agnese fuori le Mura***, Santa Maria degli Angeli e dei Martiri****\\%11

February 28: Rome\\%20

Rome: St Peter's Basilica*****, Santa Maria in Traspontina**, Vatican Museum***** (with Sistine's Chapel*****), Castel Sant'Angelo*****, San Giovanni Battista dei Fiorentini***, Santa Maria della Pace****, Piazza Navona*****, Pantheon*****, Santa Marta al Collegio Romano***, Sant'Ignazio****, San Luigi dei Francesi****, Marc-Aurel-Column****, Santi Ambrogio e Carlo al Corso****, Piazza del Popolo****, Galleria Borghese*****, Oratorio del Santissimo Sacramento al Tritone***, Santi Vincenzo e Anastasio a Fontana di Trevi***, Fontana di Trevi*****, Santa Maria in Trivio***, xyz**\\%20

February 29: Rome\\%19

Rome: Forum Romanum****, Palatine Hill***** (Casa di Augusto*****, Casa di Livia*****, Domus Transitoria****, Loggia Mattei*****, Palatine Museum***, Criptoportico Neroniana*, Oratorio dei Quaranta Martiri***), Santa Francesca Romana****, Palazzo Quirinale*****, Santi Domenico e Sisto***, Imperial Forums*** (with Trajan's Column*****), San Pietro in Vincoli****, Santa Maria ai Monti***, Santa Caterina da Siena in Magnanapoli****, Santa Maria di Loreto***, Santi XII Apostoli****, Santa Maria in Via Lata***, Il Gesu*****, Largo di Torre Argentina****, Santissimo Sudario all'Argentina***, Sant'Andrea della Valle*****, Santissime Stimmate di San Francesco***, Palazzo Venezia***, San Marco Evangelista al Campidoglio****\\%19

March 1: Tivoli\\

Tivoli: Villa d'Este*****, Villa Adriana*****, Temple of the Sibyl***, Temple of Vesta****, Villa Gregoriana*****\\%5
Rome: Santa Chiara**, Palazzo Santa Chiara***\\%2

March 2: Rome\\
We had breakfast and checked out of our hotel, and then continued to make use of our 24h public transport ticket which we had purchased the evening before, getting out to the basilica of San Paolo fuori le Mura. This basilica had been vastly unchanged from the 5th century until the 19th century, when a large fire ravaged through the basilica and largely destroyed the building. The basilica was rebuilt to resemble the original church as close as possible. Only parts of apsis mosaic of the choir, the high altar, and some of the doors could be saved from the original church. The five naves of the basilica are dominated by rows of columns, the wooden ceiling is gilded in gold. A row of Papal portraits (fictional for the oldest ones) surround the naves of the church. Next we stopped shortly by the Piramide station to see the pyramid of Caius Cestius as well as the Aurelian Walls and the Roman gate of Porta San Paolo. One of the best conserved ruins from ancient Rome are the baths of Caracalla, which represent how large these public baths had been once. Most of the statues and decorations had been removed long ago, the most famous is the Hercules of Farnese, which is in the national museum of archaeology in Naples by now. Only some remains of the mosaic floors are still on the premise. Then Bryan and I walked up to the hill of Celio. Previously I had tried to get into the church of San Gregorio al Celio, but to no avail. This time the main door was close too, I also checked side doors, and I realised a small post which stated that to get inside one should ring the bell, and so we did. Indeed a door opened and a nun told me i just needed to go left to get inside the church. The church is pretty modest, which one nice side chapel. Below the Baroque basilica of Santi Giovanni e Paolo there are remains of a couple of Roman Houses. These are full of well-conserved pretty ancient frescoes. A hidden gem, which I would definitely advice to consider visiting.\\
I wondered what we could do next; since Bryan had loved the Stanze di Raffaelo I suggested we could see the Villa Farnesina over in Trastevere. Bryan liked the idea and we took the tram from Circus Maximus over to Trastevere. On the way to Villa Farnesina we stopped in Santa Maria in Trastevere. Once again the lights nowadays had to be paid for. And once again I witnessed how many people didn't bother to pay for it, but once I paid they had nothing better to do than standing exactly in front of my camera to take their dozen of selfies. Most of those folks are unfortunately not that nice to pay for a next round in case they didn't get their shots. The mosaics of Santa Maria in Trastevere are from the gothic times, so kind of modern by the standards of Rome. A few minutes later we arrived by the Villa Farnesina. The highlights of this villa are the Loggia of Galatea, and the Loggia of Amor and Psyche on the lower floor, both highlighting the talent of Raffael. Two of the upper rooms of the villas are also full of nice frescoes. It is true, that the entrance fee is a tad high to see a small Palazzo like the Villa Farnesina, but the four rooms are one of the most beautiful in Rome. On our way to the Vatican we had a look int a couple of churches, all of them nothing special, in San Giuseppe alla Lungara we just arrived shortly before a small wedding, just the priest, the couple and two more friends. We crossed the Porta Santo Spirito and the Leonine Walls and had lunch by the small restaurant of xyz. Although this restaurant was less than half a kilometre away from St Peters's basilica, it seemed to me that we were the only non-locals here around lunch-time and the place was pretty full, and the food was pretty good (and affordable).\\
  Once we arrived by St Peter's we had to go through security again, about 15 minute wait this time around. On this day the right side nave was free to visit, so this time I got a much closer view of Michelangelo's Pieta. I was still not let into the Sacrament's chapel with my camera, guess that smart phones can take photos didn't make it to the guards yet though. The catacombs had opened by now, it was though pretty disappointing what one is allowed to see nowadays. My first time around all side chapels of the crypt could be seen, even taking photos were allowed, also one could get a view of St Peter's tomb from the fences of the high altar. Nowadays the guards make sure you see absolutely nothing of the tomb standing in front of the high altar, photography in the crypt is not permitted anymore, oh did I mention the glass walls which have been installed everywhere. Unlike in 2013 and 2017 nowadays all of the side chapels and side floors of the crypt are off limits (unless you pay for it, so thanks for being so open for pilgrims from all around the world). Unlike previously one doesn't get out to the nave by the dome anymore as well, but instead one has to get out of the entrance hall of the basilica. You can guess three times what happens by the treasury: Once again it costs a couple of bucks, and once again no photos are allowed. I did expect more from the treasury of the largest church of Christianity: some of the old papal tombs have been transferred to the treasury, and yes they are amazing, to see the Tiara worn by a couple of Popes is nice too, but nothing special can be seen in the treasury otherwise, tons of ordinary cups, boxes, cases, clothes etc, and a copy of St Peter's Chair (the original is housed within the Cathedra of St Peter, and seems it is close to never up for few to normal mortals. \\
We took the bus to the Chiesa Nuova, spend a few minutes in the church (officially called Santa Maria in Vallicell), then we visited Piazza Navona and the Trevi fountain during daylight. By that time it started to rain a bit, so we rushed past Hadrian's temple over to Piazza Rotonda, where we had another cup of ice-cream (delicious as usual), while it started to rain heavily, and everybody fled either into the open cafes or the Pantheon itself. After things calmed down again we rushed to the bus stop by Largo di Torre Argentina getting on the bus to Santa Maria Maggiore. Personally my second favourite of the Basilica Majors, Bryan disagreed and put it in fourth and last place. I enjoy that the basilica offers the splendour of Baroque mixed with ancient 3rd and 4th century mosaics, which shaped the mosaics in Rome for centuries. We paid a short visit to Santa Pudenzia and Santa Prassede (paid for the apsis mosaic) with the side chapel of San Zeno. We walked back to our hotel, got our luggage, took the train to the airport, had dinner at one of the restaurants.\\

Rome: San Paolo fuori le Mura*****, Pyramid of Caius Cestius***, Porta San Paolo***, Aurelian Walls****, Baths of Caracalla*****, San Gregorio al Celio**, Case Romane del Celio*****, Santi Giovanni e Paolo***, Santa Maria in Trastevere****, San Giuseppe alla Lungara*, Porta Santo Spirito***, Leonine Wall***, St Peter's Basilica***** (Treasury****, Catacombs***), Santa Maria in Vallicell****, Santa Pudenzia****, Santa Maria Maggiore*****, Santa Prassede***** %16

 And the aftermath of it all: once we returned home, we went shopping and put ourselves in self-quarantine for the usual two weeks. Either we didn't catch anything, or both of us were cases without any symptoms. After these two weeks I had just one evening left to clean up the desk in my office and pack up all stuff, before CERN went into home office mode, and days later the curfew in France started, and almost everything was shut down in the city of Geneva as well. Travel within Europe, within the Americas and in and out of Australia (and parts of Africa) seized to exist for a while.\\
 
 \section{May 17: Rossiniere-Rougemont}
\label{2020:Rossiniere}

Co-traveller: Juno\\
Since it was not advised to travel in big groups, some of my friends on the Swiss side didn't want to go on trips for the time being, nobody from the French side was allowed to cross over borders anyway. In fact some of my friends in France had to leave without most of us being able to tell them goodbye in person. Other people were called back to their home institutes as well, so tough times for everybody. As PhD student at the University of Geneva Juno did remain in the area and having not been outside and isolated for many weeks, we decided that it would be safe for both of us to get on a trip.\\

Now we quickly realised we were the only ones taking things still seriously, almost nobody besides had masks on, people were still sitting in crowds (while we tried to keep us isolated on our seats from others). Anyways most folks got off shortly after Montreux while we continued on the Golden Pass Line to Rossiniere. There we followed the Sarine valley for a couple of kilometers, enjoying the view of the surrounding pre-alps. Clearly an easy hike with a bit of length, but not high elevation changes, suitable for a start of the season. In between we stopped by the wild water of the river enjoying the Cascade du Ramacle next to us close to the village of Chateau d'Oex. Typically we could have continued the hike up in the mountains if the cable cars would have gone up by our final destination of Rougemont. But at this time of 2020 no cable cars were allowed to go. Still a nice hike through green forests, meadows, cows, rocky mountain walls, and just enjoying talking to another person live instead of video chats like for the previous past 9 weeks.\\

Golden Pass Line*****, Cascade du Ramacle****

 \section{May 31-June 2: Ebenalp \& Walensee}
\label{2020:PentecostHike}

How this trip was planned:\\
Two weeks ago we had the first hike of the season, and we were ready to go on another trip. Still no cable cars were running, but Pentecost gives us three days to enjoy. Since even in late May very high up snow might still be around we didn't want to go too high, but also not too easy. Thus we decided agains the Rhine Canyon of Switzerland and decided to hike around the Alpstein with the famous \"Ascher mountain restaurant. We checked several options, like staying on a farm sleeping on hay beds, or up on one of the alps around Alpstein, and then decided to stay at Berggasthaus Ebenalp for the night, since then on the first day we would need to go only up and not down. The second night we planned around the Walensee area in order to see Switzerlands highest waterfall.\\

May 31: Ebenalp:\\
Taking the train all the way from Geneva to Wasserauen took already quite some time. We opted to go walk up the less popular way not going to the Seealpsee yet in order to avoid the crowds. By Berggasthaus \"Ascher-Wildkirchli we did enjoy some cake and Irish coffee, before we went through the Wildkirchli caves. These caves had been one of the earliest prepared tourist caves in Switzerland even going as far to use dynamite to enlarge the upper entrance, thus that the ladies' clothes wouldn't get dirty (that's at least according to the description). Once we arrived at Ebenalp we were given a room to ourselves. Since the cable car wasn't running yet only a few tourists did go all the way up it seems. We did enjoy very beautiful sunset views over the alpine scenery and looking as far as Bodensee. \\

Ebenalp*****, Berggasthaus \"Ascher-Wildkirchli****, Wildkirchli****\\

June 1:\\
The next morning we woke up shortly before sunset and took as many photos from inside as we could. The guest house didn't open its doors that early. After a large breakfast we continued our way up to the Sch\"afler mountain peak with a beautiful view of the S\"antis group. We didn't stop on that guest house though but continued our trail alongside ropes etc down a pretty steep path to Seealpsee. There cows were very curious about all the people, we did have our lunch break by the lake before continuing our way down alongside a creek with several cascades and waterfalls. Once we arrived at the hotel in Murg we had dinner on the lake terrace, and then walked to the lake shore enjoying the sunset and the blue hour on the sandy beaches with a perfect view of the Churfirsten mountain range on the opposite side of the Walensee lake.\\

Sch\"afler***** with S\"antis Panorama, Seealpsee*****, Walensee with Churfirsten*****\\

June 2:\\
On the next day we took the boat across the lake. Clearly we were the only tourists on board, everybody else was crossing over for work. In fact the far more popular boat route was not running due to Covid19 restrictions. We walked through the forest high up over the lake, going up and down several times. In between some locals offered their home brewed Most, clearly we did enjoy some of that on a short snack break. Once we arrived by Betlis we did enjoy the cascades and also the Ringquelle, on average the largest karst spring in Switzerland, which just sits in the middle of a mountain face, and drops down a couple of 10 metres. Next to it appears typically the Seerenbachfalls, but unfortunately at time it almost completely stops the flow in summer. On that day it was such a day, so we saw a droplet and humid walls instead of a real impressive waterfall. Once we arrived in Weesen, in normal times we could have taken the bus. But as restrictions of the canton of St Gallen said only locals could use the bus. Thus we had to walk another couple of kilometres alongside the dead boring flat Linthkanal. Other people used floating boats next to us on the canal. We were not much slower but walking a flat straight trail after having hiked for 3 full days still goes into your legs and feet. And clearly a boring scenery isn't fun to watch either. Having finally arrived in Ziegelbr\"ucke we got ourselves ice cream and talked on the way back about how we should hike next time on glaciers potentially crossing the Aletschgletscher this summer too.\\

Walensee with Seerenwald****, Ringquelle*****, Seerenbachfalls (almost dried up)*, Linthkanal*\\

 \section{June 20: Strasbourg \& Baden-Baden}
\label{2020:Strasbourg}

Strasbourg: Minster*****, Petite France****\\
Baden-Baden: Kurpark***\\

 \section{July 4: Caves and Waterfalls}
\label{2020:CavesWaterfalls}

Labalme: Grotte du Cerdon*****\\
La Balme-les-Grottes: Grottes de La Balme*****\\
Belmont-Luthezieu: Gorges de Thurignin****\\
Vieu: Source de Groin****\\
Champagne-en-Valromey: Puits des Tines****, Pain de Sucre****\\
Surjoux: Pain de Sucre****\\


 \section{July 5: Hike St Cergue-Nyon}
\label{2020:StCergue}

A longer story concerning this trip: The travel ban had heavily derailed Janina's original plan to do write and perform her master thesis at CERN. But once the travel ban between Germany and Switzerland was lifted Janina visited Geneva again to see her friends in the area again among other things. Once you are in such an area it is only natural to plan a short hike. Since we all didn't want to spend ages in trains and tons of money on cable cars, we opted for a hike in the Jura instead.\\

Co-travellers:\\
Dylan, Sam, and Grace, all US-Americans and PhD students for American universities at CERN. Janina, now a master student at the University of Dortmund, and Kevin, a PhD student, both Germans joining in for the fun as well.\\

We chose a decently challenging hike walking along an old Roman road from a mountain pass in the Jura by St Cergue down to Nyon by Lake Geneva. In total we would cover roughly 800 metres in altitude over about 8 km so quite nice for a bit of sports indeed. Unlike the hikes around the Reculet in France, the trail here was less rocky, also leading mostly through forest and meadows. On our trail we saw a lot of wild flowers and beetles and butterflies enjoying the summer afternoon. Once we reached the valley we walked alongside a couple of wheat fields and apple trees. Once we reached Geneva we all met again by Lake Geneva for some swimming and dinner afterwards. A quick enjoyable early summer hike.\\

St Cergue/Nyon: Ancient Roman Road Trail from St Cergue to Nyon****\\

Little did we know that further lockdowns would follow later in Fall 2020, thus all future plans we had to meet or hike later in Fall or meet up for winter trips, or even spring trips were all thwarted, thus this trip meant a lot of last trips as it stand now:\\
Dylan: with Dylan I did a total of three day trips, our planned multi-day trip to Rome was unfortunately the first victim of the covid19 pandemic. We still managed to see three UNESCO world heritage sites on those trips.\\
Grace: with Grace I did a total of four day trip, and another multi day trip spending four days in northwest Germany. We managed to see a total of nine UNESCO world heritage sites.\\
Janina: although Janina was around for basically one summer and another week, we still managed to do three day trips together, but still managed to see two UNESCO world heritages.\\

 \section{July 18: Mer de Glace}
\label{2020:Mer de Glace}

Chamonix: Mer de Glace*****

\section{August 1: Belalp \& Domodossola}
\label{2020:BelalpDomodossola}

Why you should go there: Belalp offers a front of the tongue of the Aletschgletscher one of the few perspectives I still missed (beside the bird's eye view). In addition you can hike up to viewpoints overlooking the Oberaletschgletscher, another glacier I didn't get a good view of at this point. \\

Co-Travellers:\\
Sam: US-American, after our hikes in the Jura, a trip to Paris, a trip to Reichenau, and our failed attempt to see Rome together, this time we will explore the Alpine mountains for the first time. While I was planning to cross the Aletschglacier the following weekend, Sam wanted to join our friend Bing on some of his hikes in the Chamonix valley, so an ideal outing for him to test his newly purchased hiking equipment.\\

Taking the train from Geneva to Brig, followed by a bus to Blatten, and then a cable car to Belalp. Up there I realised I forgot the inner cup of my thermos jug at home sitting on my kitchen, and tea had leaked all over my backpack. Since I had the jug outside not too much was damaged or wet, but still annoying since I only had half of my planned water on me anymore. A short 500 m walk ahead we reached the chair lift bringing us up to the top. From there we started our ascent to the top of Sparrhorn. Although the weather forecast had foreseen sunny weather with a bit of clouds, that was definitely not the case. Once we arrived on the top of the mountains grey dark clouds were hanging over the next valley, and getting into large rain is not advised on any mountain territory. We snapped a couple of photos and enjoyed the view of Oberaletschglacier, whose ice had been covered by a lot of gravel over the years before we started to rush down.\\
We started to hear thunder and knew it definitely was the right call to get down quickly, halfway down we felt a few rain drops but only shortly before we reached the upper station it started to really rain. As soon as we found shelter also a herd of sheep started to run down out of the rain. The chairlift had stopped operation due to heavy winds coming with the rain and we had to wait about 5-10 minutes before it all had rained down and the sky cleared up again. Once we were down again we walked over to Hotel Belalp and enjoying the best view of the Aletschgletscher directly in front of us over the valley. We were not done far earlier than originally planned, so I checked for possible extensions of our trip and found out we could in theory easily get over to Domodossola, the trains to this Italian town are covered by the Swiss rail day pass. Thus I would still manage to get Sam to Italy for the first time, and Sam clearly liked the idea too. We still had time for lunch, so we ordered at the Hotel Belalp restaurant two Wine Cream soups for us, and they (and the white bread coming with it) were absolutely delicious.\\
And then one cable car, one bus ride, and one more train ride (through the Simplon tunnel) later we did finally reach the Italian town of Domodossola. I had been to Domodossola twice before at that point, the town itself is a cute nice little Italian town with nice Piazzas, a couple of churches and old buildings. The highlight is though the Sacred Mountain a bit to the South of the town (but only about 15 minutes away from the train station). There 12 chapels are scattered around a pathway leading up to a hilltop, each one of the a station of the cross. The woodcarving is outstanding, the statues and chapels were built between the 17th and 19th century. The last wooden statues had been added in 1957. Thus the chapels show quite a variety of styles, and frescos from Baroque to modern times. On the mountain top is a pilgrimage baroque church of SS Crocifisso, and the chapel of paradise showing the resurrection. It was very humid on this day and when it started to rain it was rather a refreshing wall of tiny water droplets almost like transparent fog surrounding us. At that point we rested on top of the mountain next to the ruins of an old castle. Once we were in town again we had to do a full program to celebrate Sam's trip to Italy, getting a big cup of Gelato. Then we got ourselves Espresso and Aperol Spritz, a selection of Italian sausages and cheese were provided as well (and other snacks). Clearly it was obvious at that point that Italy took Covid19 precautions much more serious than Switzerland, with people masked up and trying to keep distance instead of Switzerland where nobody seemed to be bothered. I still think it was a nice first Italian experience, before we got on trains again which brought us back to Geneva roughly 3 and half hours later.\\

Belalp: Sparrhorn mit Oberaletschgletscher*****, Aletschgletscher*****\\
Domodossola: Sacre Monte*****\\\

Unfortunately this marks to this day the last trip I did with Sam. We originally had hoped to meet up for more hikes in October or even November for some sledding in the Black Forest, but then the second wave hit and even when Switzerland did open up for hikes again, Germany banned all none work related border crossings to any neighbouring countries already in the Fall of 2020 and still didn't lift that embargo even until June 2021 (clearly breaking the promise politicians gave of never closing the borders again). To conclude I took Sam on four day hiking day trips and one multi-day trip to Paris (and a failed attempt to Rome which we unfortunately cannot count). We saw a total of seven UNESCO world heritage sites on those trips (plus another local one in Geneva). Since Sam is still in Europe, I do hope that he might be able to visit Vienna, and then we can add that too.

\section{August 9-August 10: Hike Jungfraujoch-Fiesch}
\label{2020:Aletsch}

Lauterbrunnen: Kleine Scheidegg*****, Lauterbrunner Tal*****\\
Grindelwald: Eismeer*****\\
Fieschertal: Jungfraujoch*****, Konkordiaplatz*****, Aletschglacier*****, Fiescherglacier****\\

\section{August 22: Hike Mont Forchat}
\label{2020:Forchat}

Lullin: Mont Forchat****

\section{August 28: moving back to Lauchringen for (not so) short Intermezzo}
\label{moveGeneva}

Physics jobs are hard to get, I tried for a period of 2 years, but nothing successful came out of that. In parallel I was starting to apply for jobs in industry with all the pain that goes with it. Unfortunately during the process the Covid19 pandemic hit as well, which lead to a hiring freeze in many companies, and overall a tougher climate. I decided to join a Data Science bootcamp in Zurich to get myself up to speed and to make full use of my time, so I decided it was finally time to leave CERN (although I kept my affiliation as a CERN external user for half a year longer). Since my home village is close to Zurich I decided to spent my time there and commute every day. At least in the beginning the bootcamp was partially on site, before the second wave hit and all had to go into full remote mode.

\section{August 29: Brugg}
\label{2020:Brugg}

Altstadt***

\section{September 5: Rhine Falls}
\label{2020:Rheinfall}

Rheinau: Monastery****, Hydro Power Station***\\
Laufen: Castle***\\
Neuhausen: Rhine Falls*****

\section{September 6: K\"ussaburg}
\label{2020:Kuessaburg}

Having moved back to Germany gave me also the opportunity to meet other family members again. Thus my dad and I met with my older brother, my sister in law and their children for lunch on a nice rural restaurant close to an old castle ruin. After lunch we walked up to the ruins. One of the most famous and interesting sights in my home region, the K\"ussaburg had been in use up to the 30 year long war, when it was destroyed on purpose to prevent it falling into enemy hands in working conditions. Over the years it had been used as stone quarry. Only in the last hundred years people started to appreciate the ruins again. Thus the remains had been excavated. It is by far the largest castle in the region and still an impressive sight with several fortifications and a nice view of the surrounding landscape, on a good day up to the alps.\\

K\"ussaburg****

\section{September 20: H\"ori}
\label{2020:Hoeri}

\"Ohningen: Klingenbachschlucht****, St Hippolyt \& Verena***
Kattenhorn: St Peter****

\section{October 3: Lauchringen}
\label{2020:Lauchringen}

Steina Canyon****

\section{October 4: Salem}
\label{2020:Salem}

Salem: Monastery****\\
Bermatingen: St Georg***\\
\"Uberlingen: Nicolas Minster****, Franciscan Church****

\section{October 9: Baden}
\label{2020:Baden}

Old Town***

\section{October 11: Baar}
\label{2020Baar}

After six week of data science bootcamp it was my first time taking a new friend I made there on a trip. A short afternoon trip that is. We can obviously discuss if the Baden outing the evening before qualifies as trip, but the hike of this day definitely does. Originally we had planned already two weeks before to go to the Interlaken area and visit caves and waterfalls there. Unfortunately the weather was horrible, even with flooding involved. Thus we opted for a closer place this time.\\

Co-travellers:\\
Namrata: The time at CERN is a blas of the past, but this doesn't stop me from getting to know other physicists, even if it is through a data science bootcamp. Anyways we love to check out caves, and unlike others we need to have time far away from work, thus Namrata was clearly up for trying a first short trip with me. I did explain to her what travel with me can be like, but she was up for the challenge.\\

Having arrived in Zug we took the bus up to the H\"ollgrotten stop. After taking a couple of snaps of Lake Zug, we continued to descent down to the Lorze valley, crossing the dam by a former factory, continuing for about 20 minutes until we reached the caves of H\"ollgrotten. This system consists of an upper and a lower cave. Originally fed by a lot of water, the underground stream was rerouted in the mid 19th century, making the cavities and halls of the caves accessible to tourists. The caves themselves are only thousands of years old, so exceptionally young in geological age. The flowstone cave has plenty of stalagmites and stalactites covering the walls, illuminated in several colours. The water levels dropped substantially, but still a few ponds and lakes can be seen along the way.\\
Once we finished our cave visit Namrata and I got the local H\"ollbier before continuing our hike towards the Wildenburg Castle. Thus this can count as the very first beer hike with a stop for beers in the middle of it. While we had plenty of ideas to do a couple of those for years already, none of them actually happened so far, so I guess we can call that a success. On way to Wildenburg we crossed the Schwarzbach while having a superb view of the cascades and the beautiful Schwarzenbach falls. The castle itself is in ruins, the mountain became unstable, and a substantial part of the cave has been fallen down into the canyon by now. Faced with the decision to return the same way we came or to continue our hike towards Zug, Namrata decided we should keep going. Thus we continued through the forest, across meadows back into the direction of Zug. On our way to Zug we missed barely two busses each coming less than 30 seconds late, and even having arrived in Zug we barely missed the train. Having had a generous time budget that didn't matter in the end, and it was a fun first trip together, maybe with a couple more to follow.\\

Baar: H\"ollgrotten*****, Schwarzenbach Falls****, Wildenburg Castle***
 
\section{November 7: H\"ochenschwand}
\label{2021HoechenschwandI}

H\"ochenschwand: Black Forest Hike***

\section{November 8: Waldshut \& Tiengen}
\label{2021WaldshutTiengen}

Waldshut: Old Town***\\
Tiengen: Old Town with Storchenturm \& Church***\\

\section{November 22: Todtmoos, St Blasien \& H\"ochenschwand}\
\label{2021:Todtmoos}

Todtmoos: Waterfall****\\
St Blasien: Monastery \& Dom****\\
H\"ochenschwand: Alps Panorama during Sunset*****\\


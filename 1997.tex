\chapter{Year 1997}
\label{1997}

\section{May 21--May 25: M\"unchen}
\label{1997:Muenchen}

%22 Glockenspiel, Asamkirche, Frauenkirc he
%23: Nymphenburg, Olympiapark
%24: Deutsches Museum
%25 Dachau

Nymphenburg with Park Palaces***,Theatinerkiche**, Michaelskirche** ,Asamkirche** , Concentration Camp Dachau** , Olympiapark with Olympia Tower**, Deutsches Museum***

\section{August 24--August 31: Rom}
\label{1997:Rom}

After we did a big family trip to London the year before, my mum decided it was time to take on Rome. Considering my interest for anything in old architecture my aunt got me a book about Rome as present. I had a very old second book which originally belonged to my grandmother as second source of information. Once we arrived in Rome we took a taxi to our hotel which was pretty close to Termini. The is one thing I remember from the hotel: it had a pretty old elevator, which you could stop during the ride by slightly opening the doors, and getting it to continue via closing it again (besides the one time, when it got stuck - well teenage years I guess). \\

August 24: Saturday:\\
The first church I ever had seen in Rome was Santa Maria Maggiore, one of the big four churches. The nave is decorated with ancient mosaics, the Sistine and the Borghese Chapels are and will certainly remain my favourites. After a short stop by the Trevi fountain the day already came to an end in the church of San Ignacio, which is home to one of the largest ceiling frescoes in the world.\\

Santa Maria Maggiore*** , Trevi Fountain**, Pantheon**, San Ignazio***\\

August 25: Sunday:\\
On the last Sunday of the months the Forum Romanum and the Palatine hill can be visited for free. I had a board game about the Forum Romanum at home, but clearly the forum had the best times behind it. The palatine hill was nice too. St Paul's is another one of the four big churches, though in my opinion the least impressive. Most of the church burnt down in the 19th century.\\

Collosseum***, Forum Romanum \& Palatine***, San Paolo fuori le Mura**, San Pietro in Vincoli**\\

%26: Il Gesu, Santa Maria Sopra Minerva, Pantheon, Piazza Navona, Diokletians, Santa Maria d'Aracoeli, Himmelsleiter

August 27: St Peter's Basilica:\\
In the morning I remember that big news hit the eternal city that Princess Diana had passed away in a car accident. People were discussing it all around, while we were queuing for the Vatican Museum (free entrance day). I was very impressed by the rooms and halls, particularly the Stanze di Raffaelo and the Sistine Chapel. I also enjoyed the large library. Nowadays only the entrance of the library can be visited, the rest of the hall is off limits for the everyday tourist. Other than that the Apollo del Belvedere, the Laokoon as well as the Pinacotheca never fail to be exciting. Although it was a very rainy day we climbed the dome of St Peter's (not really worth it due to the bad conditions), and then we visited the interior of the gigantic basilica. Nowadays most of the original paintings of all the altars have been replaced by mosaics. The many monuments and tombs of the popes are clear highlights, as well as Michelangelo's Pieta.\\  

Petersdom***\\

%28 Laterno, San CLemente

%29 Ostia Antica

%30 Catacombs

%31 Vatican Museum, Sapnish Stairs

Ostia Antica*** , Baths of Caracalla***, Basilica Laterano*** , San Sebastian Catacombs*, Forum Romanum \& Palatine***, , Il Gesu***, Pantheon***, Santa Maria degli Angeli**, San Andrea al Quirinale**
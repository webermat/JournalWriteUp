\chapter{Year 1997}
\label{1997}

\section{May 21--May 25: M\"unchen}
\label{1997:Muenchen}


Munich is the closest city with over a million inhabitants to my home. Thus at some point it was inevitable we would want to see it, so time for another family trip\\

May 21: Munich\\
Having arrive late in the afternoon we had a late walk through old town with the Sendlinger Tor and Karlsplatz/Stachus.\\

Munich: Sendlinger Tor**, Karlsplatz/Stachus***\\

May 22: Munich\\
This day focused on Munich old town: we saw Marienplatz where we waited to see the a bit underwhelming Glockenspiel. When the bell rings the full hour, a series of statues turn around and perform a mechanical dance. We also visited the Rokoko-church of Asamkirche, which is pretty small but with a very elaborate decoration, done by the Asam brothers who lived close-by. The Frauenkirche is a bright gothic hall church, but a bit plain compared to other gothic cathedrals. Michaelskirche is one of the rare Renaissance churches in Germany with a large decorated facade and a bit more decoration than e.g. Italian counterparts. Theatinerkirche is a baroque church, untypical for Germany is the large stucco decoration and less focus on frescoes or paintings. All in all Munich old town is rather small, so you won't need too much time to see most of it.\\

Munich: Glockenspiel***, Frauenkirche***, Theatinerkiche****, Michaelskirche**** ,Asamkirche**** \\

May 23: Munich\\
Munich has been the residential city for centuries, thus the summer palace of the late Bavarian kings is quite elaborate in a large English Style garden. Particularly the Festival Hall in the middle pavilion of the palace is some of the most beautiful Baroque halls in Germany. Most of the other rooms are more modest from Baroque and classicist times. The park palaces are very nice and if you have time you should definitely visit them too. Badenburg is a baroque bath palace with an adjacent ball room, and that although taking a bath had not been fashionable. The Amalienburg is the most beautiful Rokoko palace in Southern Germany, particularly with its silver-blue Hall of Mirrors. We walked through the park up to the cascade.\\
Afterwards we went to Olympiapark with the famous roof of the Olympic stadium, and then we enjoyed the view from the Olympia Tower.\\

Munich: Nymphenburg with Park Palaces*****Olympiapark with Olympia Tower****\\

May 24: Munich\\
One of largest technical museum of the world is Munich's Deutsches Museum. Here you find old airplanes, submarines, experiments with high voltage, high currents, a demonstration of Faraday cages, old rockets, even a replication of an old underground mine. Suited for adults as well children it is a very interesting to visit for the whole family.\\

Munich: Deutsches Museum*****\\

May 25: Dachau:\\
Germany has the dark history chapter of the Third Reich. In order to avoid such a despicable time to repeat itself, in school this time is discussed heavily, and both the buildings of the Nazi era and concentration camps, as well as houses of the victims of the Nazi dictatorships are conserved in order to keep the memory alive, put things in context and to just not hide the atrocities. One of those sites if the Concentration Camp of Dachau in the suburbs of Munich where barracks and old office buildings have been conserved together with newly built religious spaces of remembrance and a vast documentation centre. Although clearly visiting such a place will fill you with disbelief and sadness it is in my opinion needed to really see how low and horrible things can get if one allows racism and too nationalist feelings go wild and out of control.\\

Dachau: Concentration Camp Dachau**** 

\section{August 24--August 31: Rom}
\label{1997:Rom}

After we did a big family trip to London the year before, my mum decided it was time to take on Rome. Considering my interest for anything in old architecture my aunt got me a book about Rome as present. I had a very old second book which originally belonged to my grandmother as second source of information. Once we arrived in Rome we took a taxi to our hotel which was pretty close to Termini. The is one thing I remember from the hotel: it had a pretty old elevator, which you could stop during the ride by slightly opening the doors, and getting it to continue via closing it again (besides the one time, when it got stuck - well teenage years I guess). \\

August 24: Saturday: Rome\\
The first church I ever had seen in Rome was Santa Maria Maggiore, one of the big four churches. The nave is decorated with ancient mosaics, the Sistine and the Borghese Chapels are and will certainly remain my favourites. After a short stop by the Trevi fountain the day already came to an end in the church of San Ignacio, which is home to one of the largest ceiling frescoes in the world.\\

Rome: Santa Maria Maggiore***** , Trevi Fountain****, San Ignazio*****\\

August 25: Sunday:\\
On the last Sunday of the months the Forum Romanum and the Palatine hill can be visited for free. I had a board game about the Forum Romanum at home, but clearly the forum had the best times behind it. The palatine hill was nice too. We only saw the outside of the Colosseum. St Paul's is another one of the four big churches, though in my opinion the least impressive. Most of the church burnt down in the 19th century.\\

Rome: Colosseum***, Forum Romanum \& Palatine****, San Paolo fuori le Mura****, San Pietro in Vincoli****\\

%26: Il Gesu, Santa Maria Sopra Minerva, Pantheon, Piazza Navona, Diokletians, Santa Maria d'Aracoeli, Himmelsleiter

August 27: Rome\\
In the morning I remember that big news hit the eternal city that Princess Diana had passed away in a car accident. People were discussing it all around, while we were queuing for the Vatican Museum (free entrance day). I was very impressed by the rooms and halls, particularly the Stanze di Raffaelo and the Sistine Chapel. I also enjoyed the large library. Nowadays only the entrance of the library can be visited, the rest of the hall is off limits for the everyday tourist. Other than that the Apollo del Belvedere, the Laokoon as well as the Pinacotheca never fail to be exciting. Although it was a very rainy day we climbed the dome of St Peter's (not really worth it due to the bad conditions), and then we visited the interior of the gigantic basilica. Nowadays most of the original paintings of all the altars have been replaced by mosaics. The many monuments and tombs of the popes are clear highlights, as well as Michelangelo's Pieta.\\  

Rome: Petersdom*****\\

%28 Laterno, San CLemente

%29 Ostia Antica

%30 Catacombs

%31 Vatican Museum, Sapnish Stairs

Ostia Antica**** , Baths of Caracalla*****, Basilica Laterano***** , San Sebastian Catacombs***,  Il Gesu*****, Pantheon*****, Santa Maria degli Angeli****, San Andrea al Quirinale****
\chapter{Year 2002}
\label{2002}

\section{May: Tuscany}
\label{2002:Tuscany}

%San Gimignano
%Siena
%Florenz: San Lorenzo, Duomo, Santa Croce, Ponte Vecchio, Palazzo Pitti, Santo Spirito
%Pisa
%Volterra
%Montaione-San Vivaldo (place where we stayed)

San Gimignano: Old Town****,\\
Siena: Duomo*****\\
Pisa: Duomo*****\\
Firenze: Duomo****, Palazzo Pitti*****, San Lorenzo****, Santa Croce****

\section{August 20-September 3: Morocco}
\label{2002:Morocco}

%20: Agadir
%21: Marrakesh: Saadian Tombs, Fnar el, El Badi, Menara Gardens
%22: Ouaoumana Dam, Atlas
%23 Fes Mdina
%24 Fes, Medersa Attarine, Gerberviertel Meknes: Moulay Ismail, Volubilis
%25 Rabat: Mausoleum, Hasan Mosque, Casablanca
%26 Essauira

After our good experiences in Greece and Turkey my parents decided that we should see a bit of northern Africa, deciding to visit Morocco.\\

August 20: Agadir\\
We flew into Agadir, where we were in a hotel with a nice pool, other than that we got to know out tour group. Our family was put together with a French woman and her 12 year old son. My mum does understand French but is typically missing words when trying to talk. The French lady had the opposite issue, she understood German pretty well, but speaking was more of a problem. Thus they did have some French-German conversations with each other. Having had latin at school, I was of no help anyway.\\

August 21: Marrakech:\\
Starting out very early, once we arrived in Marrakesh we saw the ruined palace of El Badi as well as the Saadian tombs, which are very nice. Then we drove over to the Menara Gardens where we saw the former Royal pavilion. Then we had dinner on a roof top restaurant overlooking the bazaar of the Djemaa el Fna square.\\

Marrakesch: Saadian Tomb*****, El Badi Palace****, Menara Gardens***\\

August 22: Atlas:\\
From Marrakech we drove through the Atlas mountains, passing the Ouaoumana dam on our way, having arrived in Fez we stopped to have a look at the large town on the food of the hills. Fez had been one of the largest old towns of Africa, actually two old towns - the Fes el-Bali from the 9th century and the Fez el-Jdid in the 13th century. Then the new city was founded by the French. The old town is part of world UNESCO heritage and even nowadays the old part of the city has over 100'000 citizens alone. \\

Fez: Gates*****, Chouara Tannery****, Al-Atarine Madrasa*****, Mausoleum Moulay Idris II****\\
Meknes: Moulay Ismail Mausoleum****, Volubilis Excavations*****\\
Rabat: Mausoleum Mohammed V*****\\
Casablanca: Mosque Hassan II****\\
Essauira: City Walls***

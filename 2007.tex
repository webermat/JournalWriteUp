\chapter{Year 2007}
\label{2007}

\section{February 13: Cessy}
\label{2007:Cessy}

Having worked on future LHC data, my supervisor Filip thought it would be time to see the detector itself. Since he was scheduled to give a tour for students from Heidelberg he invited me to tag along. In those days the mid section with the big magnet was still up on the surface while other parts of the detector were and had been already lowered to the detector cavern. The endcaps were still up on the surface as well. We were also allowed to go down to the cavern to witness how the HCAL was being lowered on that day. The detector is gigantic even if split into the several pieces you get a flavour for the sheer size, notable it is also heavier than the Eiffel Tower.\\

Cessy: CMS*****

\section{March xyz: move back to Lauchringen}
\label{moveLauchringen2006}

While it had been decided that I am a suitable fit for the group to start a PhD at ETH based at CERN, for finishing the thesis write up in Zurich I moved back to Lauchringen temporarily to deal with administrative stuff as well as looking for flats. Clearly the housing market is very competitive in the Geneva area and most of the flats I applied for I never heard back from, finally I got a positive reply back allowing me to live in Geneva from May 16 on. 


\section{March 9: Z\"urich}
\label{2007:Zuerich}

Zurich is Switzerland's largest city, also in a nice setting at the end of a lake, where you can enjoy some late evenings or swim in some of the pools in the river (or the lake if you prefer that). The town itself has some old medieval churches, also with St Peter the church with the largest clock. Of the medieval churches I would nowadays recommend to see Grossm\"unster which has interesting modern style stained glass windows in a romanesque setting. While the choir windows of Chagall in Fraum\"unster is quite a nice sight, the opening hours are dismal, I wanted to visit on four different Saturdays and each time the church was closed for special events (seems to be rather regularly). It is also closed on Sunday mornings, and you have to pay a steep fee to get inside, and photography is forbidden nowadays too - in order to fight "rude, insensitive Asian tourists" - this was at least an official statement given by the local responsible when asked why suddenly these restrictive measures were put in place, so don't support that.\\

Z\"urich: Kreuzkirche***, Peterskirche***, Fraum\"unster****, Grossm\"unster***

\section{April 9: Zurzach}
\label{2007:Zurzach}

The Swiss town of Zurzach had been a place where people crossed over the Rhine already in Roman times. The bridge of that time had been guarded by two castels. The castel on the Swiss side is still in quite good condition, having been in use also during later times as guard post. The Verenam\"unster is a medieval basilica built to honour St Verena and used to be part of a monastery. In the 18th century the church was outfitted with baroque stuccos and a new choir, the tomb was moved into the crypt at that point. I wouldn't go out of my way to see Zurzach but it offers some nice spots if in the area.\\

Zurzach: Verenam\"unster****, Roman Castel***

\section{April 14: Inner Switzerland}
\label{2007:Switzerland}

I had finished the write-up of my Master thesis, thus it was time for another trip to Switzerland. This time to one of the largest monasteries in Switzerland by Einsiedeln run by Benedictine Monchs. Clearly they had a no photography policy with several guards on parole. The Monastery itself is very nice, the church is richly decorated with late Baroque frescoes, stuccos etc. The church surrounds an older chapel which displays a black Madonna. In Einsiedeln are also two ski jumps of the Swiss national team. From Einsiedeln we drove to the canton capital of Schwyz walking through the old town with its Baroque church and painted town hall.\\

Einsiedeln: Stift Einsiedeln****\\
Schwyz: Old City**

\section{April 21: Meersburg \& Birnau}
\label{2007:Meersburg}

Time to see some sights along the Bodensee such as the Rococo pligrimage church of Birnau, one of the several churches of Peter Tumb this church is full of nice frescoes, stucco marble pillars and rich altarpieces. Then we went to the nearby town of Meersburg with its lower and upper old towns, a large castle and another beautiful baroque church of the New Castle.\\

Meersburg: Schlosskirche***\\
Birnau: Pilgrimage Church****

\section{May 16: move back to Geneva}
\label{moveGeneva}

Coinciding with the start of my PhD years as doctoral student I moved to a tiny studio in Geneva. Basically consisting of just one room with a kitchen corner and a bathroom the flat was very small, but provided all necessities I needed. It was also close to Lake Geneva. I tried after defending to find another flat. But this all still proved to be a hassle, thus I in fact stayed in that tiny place for over 13 years.

\section{June 3: Saleve}
\label{2007:Saleve}

My parents came over to visit me again after the move, having read how good the views are up on Saleve we took the cable car up. Although it was a very sunny day the view was a bit hazy, thus no alps for us. We did still enjoy walking through the forest, watching a lot of butterflies and nice flowers, as well as enjoying the views of Geneva and the lake. Once we were down in Geneva again we had Flammkuche before my parents drove back to Germany.\\

Geneva area (in fact in France): Saleve****

\section{September 9: Chateau Chillon \& Lausanne}
\label{2007:Chillon}

The most visited building of Switzerland is Chateau Chillon, The castle with roots in early middle ages sits on a small island in lake Geneva. The area around Montreux is also known for its natural beauty of the lake within the mountains on one side and vineyards on the other side. Chillon has several gothic halls, also many old frescoes, particularly precious ones in the chapel. The castle had never been destroyed and in use as seat of bailiffs until the 18th century. On the way back to Geneva i transferred in Lausanne. I took the time to walk up to the cathedral. Built between 1170 and 1275 it is an early gothic church. The church has several rose windows, modern stained glass windows, and an unusual four spires. I climbed the main tower over the entrance portal for good views of Lake Geneva, the Chablais Alps, and the town of Lausanne.\\

Veytaux: Chateau Chillon*****\\
Lausanne: Cathedral****, Cathedral Tower****

\section{December 16: Zermatt}
\label{2007:Zermatt}

After having been in Geneva for over half a year but without doing barely anything I decided it was time to take on Swiss mountains, this time going to Zermatt on the last weekend before Christmas. On the way I thought that Sion looks so nice with its castles, that i should maybe get off and pay a visit, but then I decided to stick to the original plan. I thought the Santa Claus church tower of St Niklas looked very cute. I enjoyed a first view of the Bisgletscher over Randa before finally arriving in Zermatt where I was blown away by seeing the Matterhorn for the first time. Clearly I had made my choice early that the Kleinmatterhorn was the peak to take the cable car up to, and indeed it didn't disappoint. I visited the glacier palace, did enjoy the view from the platform and had lunch in the cafeteria. Once I was down in the village again i had a quick view of Zermatt's church (nothing interesting to see there though).\\

Zermatt: Kleinmatterhorn*****,  church*
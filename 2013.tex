\chapter{Year 2013}
\label{2013}

\section{January 2--January 6: Netherlands}
\label{2013:Netherlands}

January 2:\\
Amsterdam: Rembrandthouse****, Royal Palace*****, Nieuwe Kerk***, Oude Kerk****, He Hua Temple**\\

January 3:\\
And today I started my first trip outside of Amsterdam, taking the train to Appeldoorn. After having a first snack there I made may way to the former royal palace of Het Loo, which has been in use by the royal family from the 18th century to the 1970s. Since 1984 the palace is a state museum showing the interiours with all the original furniture from the Orange-Nassau era. Since it was shortly after Christmas each room was still decorated with Christmas trees, ornaments and candles and thus creating a nice mood. Most of the rooms had been decorated in Baroque style, but some of the later rooms were clearly more modern. In the french style garden the main fountain was being renovated and due to low temperatures non of the other fountains or canals were running, most statues were not hidden in wooden boxes as they often do in Germany though. All in all I can only recommend to visit the palace. \\
Taking the train back I walked along the Grachten to the van Loon House, belonging to one of the noble families of Amsterdam with interiors of the 18th and 19th century. The last place I visited on this day had been the Willet Holthuysen House. Maybe it had been the time, but I was least impressed by that house on this day, y that time it was also pretty dark already, which might have played a role too.\

Appeldoorn: Het Loo Palace*****\\
Amsterdam: van Loon House****, Willet Holthuysen House***\\

January 4:\\
The Hague: Binnenhof with Knight's Hall****, Lange Voorhout Palace (Escher Museum)*****, Fredenspalace***, Gemeentemuseum**, GEM***\\

January 5:\\
Soestdijk Palace****, Amsterdam: Beurs van Berlage***, Geelvinck Hinlopen House***, Rijksmuseum***, Stedelijkmuseum*****\\

January 6:\\
Amsterdam: Portuguese Synagogue**, Jewish Museum (Large Synagogue)***, NEMO*

\section{March 21--March 25: Paris I -- sponsored by Eric}
\label{2013:ParisI}

Why Paris: why not, well after all Paris is one of my most favourite places to see (having been there over and over again). Paris is also the closest big city from Geneva and a good place to show to your friends. This time I decided to see the private apartments at Versailles as well as at Fontainebleau, so should be big fun. 

Co-travellers:\\
Andrew: A UCLA undergrad student, who came over to help us testing muon chambers. His first time in Europe, I decided the first thing to see should be Paris, particularly since it is so easily accessible by train.\\

March 21: The train ride\\
As the afternoon passed Eric was still busy down in the cavern and I commented that it might get tight if he does not return immediately. For reasons unknown to me he still continued working and indeed missed the train ride (and wouldn't even join later on, so in a way he sponsored our trip). Once we arrived by the hotel we realised it was in renovation (actually more the hallway and the elevator). No issue to carry our not so big luggage up the flight of stairs though. The door to our suite was a bit small, though not too small for me to go through straight up, and Andrew didn't mind to claim the bed for two for himself alone.\\

March 22: Versailles\\
Having been in Versailles already twice before, I decided that this time around it was the time to try something more: the private apartments of the king and the queen. Most of the state apartments are amazing and full of wonderful history, art, and splendour, but behind the scenes is where the royal family really lived with private cabinets, a gaming saloon, their library or even their actual bedroom. The state bedroom was the place where they officially started their days sitting in bed, but it was sad to be very uncomfortable and windy in that room. Also the private inner cabinet of the king is one of the big masterpieces of French Rococo, similarly the Room of the Golden Plates. Only with a tour you can get in there, another fantastic highlight of the tour is the Royal Opera house. Although used as large private theatre it is similar in size to the Parisian Opera house of that time. Afterwards Andrew and I admired the state apartments. Even in March they were very crowded, but nothing to the sheer mass of people in summer. \\
Then we spent more hours in the park and the garden palaces of Grand \& Petit Trianon as well as the Queen's hamlet. In mild March the sun doesn't burn you, the downside of all was that we went on a Friday, which means less crowds but also fountains which are not running. Clearly the sculptures are still nice and pretty, but the running fountains give a totally different flair to things.\\
On our way to the train station we did a small detour through the Hotel de Ville of Versailles with another beautiful gold leaf decorated festival hall and a city council hall with baroque supra portas. Nothing to amazing but still cute to visit once .\\
Back In Paris we did an afternoon and evening tour of the Louvre, making use of their extended opening time on Fridays. We started out in the Empire style Apartments Napoleon III, but also didn't miss out on the artefacts from old Persia, Assyria, or Michelangelo's sculptures, the Venus de Milo or the Mona Lisa. Even in evenings this painting has a waiting line. The Louvre is one of the most amazing museums, also very large. Even a tour of highlights takes at least two hours, and that is if you walk fast. The splendour of the former use as royal palace can be admired in the Gallery of Apollon, although the main royal palace were the Tuleries, a bit further down the Louvre, but those had been completely destroyed in one of the revolutions, although some of the furniture still exists, like Napoleon's bedroom.\\
After that successful first day I wanted to try out something I didn't know, so I got myself a Monaco which was listed under beers. Little did I know that I got myself beer mixed with grenadine juice and 7up. At least Andrew enjoyed watching me drink it.\\

Versailles: Chateau***** (State Apartments*****, Private Apartments*****, Trianons*****), Hotel de Ville***, Paris: Louvre*****\\

March 23: Paris\\
We started the second day walking over the Place de la Concorde, in my opinion the most beautiful square in Paris, before a guided tour through the French House of Representatives in Palais Bourbon. This had been a private palace of members of the extended royal family once, but was completely remodelled to house the parliament later on. Many ceilings and walls have been decorated by famous French artists such as Alechinsky or Delacroix. The interior is largely classicist, the highlights being the Festival Hall and the Library.\\
 Afterwards we walked over to the Ile de la Cite starting with the medieval royal palace of Paris. The Conciergerie is the hall of the guards which is a giant gothic style vaulted hall. Unfortunately nothing remains of the actual Great Hall of the palace with the same dimensions which was situated just above the guard hall. Still it is one of the largest gothic secular halls remaining on the continent. The masterpiece of the palace was though the Royal Chapel, the Sainte-Chapelle which has walls made out of spectacular stained glass, some of the best medieval glass art in the whole world. Some of it had been damaged in the revolutions and had to be reworked, but a big ratio is still from medieval times.\\
 Then we explored a bit of unknown Paris, getting the metro to the outskirts and walked through the beautiful Park de Chateau Bagatelle with nice artificial grottoes and peacocks all over the park. The chateau was according to the Paris online tourist site open on that weekend for a guided tour, but that information seemed to be outdated. Anyway we did enjoy the park for what it was and took the Metro back to Paris where we wanted to see the Hotel de Soubise, which is the seat of the French National Archive.\\
 Although it was over two hours before closing time and even their own posts at the premise did state this, we were refused entry, having arrived too late. Maybe a special event was happening, but we were not given any information about that statement contradicting the information written just behind the counter. Instead we walked over to the nearby Hotel Carnavalet which is now the Parisian House Museum, containing a lot of rooms of noble houses, villas and hotels, which had been transferred to this place before the original houses had been destroyed. Thus you get a pretty good idea how the fashion of private homes changed throughout the years. Last but not least we walked over to the Renaissance square of Place de Voges. In one of those houses Victor Hugo wrote his world famous books, and his house can be visited even for free. So do it if you're in Paris.\\

Paris: Place de la Concorde*****, Palais Bourbon*****, Conciergerie***, Sainte-Chapelle*****, Notre Dame*****, Park de Chateau Bagatelle***, Hotel Carnavalet****, House of Victor Hugo***\\

March 24: Fontainebleau:\\
Having visited Fontainebleau back in 2010 my parents and I missed out on the tour of the Private Apartments by just about 5 minutes. Not wanting to do the same mistake this time I made sure we ordered tickets just when the ticket office opened. On my request for two tours, the one of the Guest Apartments and the Private Apartments in English - the lady on the counter told me, the tours will be in French I cannot sell you the tickets for them. I told her I don't mind I want to see the rooms even if my French is not deemed worthy enough. The lady told me off rudely without good French no tour. Thus I walked over to the counter next to this rude despicable lady now ordering the tickets in French. My French is indeed basic, but it does exist to a point, thus I can in fact follow at least the big theme and topic of French guided tours. This time we did obtain tickets for both tours. \\
With enough time to spare we visited the Chinese Museum in other wings of the palace before. The Chinese museum contains artefacts of the Empress in fact stolen from Beijing's original summer palace before it was completely erased by French and British troops in the Opium wars. Amazing artefacts but sad to see how they were stolen from China in first place. Afterwards we toured the guest apartments in the upper floors of the castle. Usually these contained a cabinet for work, a living room and a bedroom. On display were original desks with newspaper from the time of the second empire.\\
Then we visited the Museum of Napoleon I, set up by his nephew Napoleon III to commemorate the deeds of his uncle. They contain tents from his military campaigns, clothes, the living room of his son Napoleon II, and other artefacts. And then it was time to admire the amazing state apartments, with several rooms built during the Renaissance times of Francois I of Valois with a large wood carved gallery, a ballroom with multiple frescoes, and two rooms with Renaissance tapestries. These are followed by Bourbon time rooms from Louis XIII or a cabinet of Marie Antoinette. The largest rooms, namely the state bed rooms, dining rooms, and throne room had been completely remodelled by Napoleon I, who used Fontainebleau as his major seat, having converted Versailles to a museum. One wing, the so-called Papal apartments didn't open until the afternoon, when we were having our tour of the private apartments. \\
Thus we had time for a three course lunch close to the castle, before having a short look at the local church of St Louis, and back to the palace for the tour of the private apartments with the same guide who gave the guest apartment tour in the morning. Now the private apartments are mainly from Napoleon's times again, but a salon of Madame Pompadour and Louis XVI can be seen at the start of the tour. The tour includes in addition the flat of the private secretary of Napoleon too, thus you get a view of how lower class people lived, even if they worked for the Imperial family. The inner bedroom of the Empress is the most beautiful room of the inner apartments, although I did like the Gallery of Deers which finished the tour. Now we wanted to see the Papal Apartments but then we were told those were closed again and we were too late, no matter if our tickets were valid for those, thus all in all it is IMPOSSIBLE to see all of the palace in winter even if you pay for it.\\
Anyways instead we jumped on the train got back to Paris and walked up to the second floor of the Eiffel Tower. First of all climbing the tower by the stairs gives you a nice view of the inner structure of the pylons and the waiting time is typically substantially less, unless you arrive at night when things slow down considerably. Thus it is usually the fastest way to get up, and a bit of sports doesn't hurt anyway either. You have nice views already from the second floor, the third floor doesn't give you much in addition. Now the downside is once up the Eiffel tower, you don't see the Eiffel tower itself anymore. In order to achieve that go up on Tour Montparnasse instead. Which is exactly what we did, some even claim another positive aspect: from Tour Montparnasse it is impossible to see Tour Montparnasse itself which some people consider really an ugly building - if you ask me it is totally fine, only not really suited the classic Parisian city centre, but then again such is Tour Eiffel as well. \\

Fontainebleau: Chateau***** (Chinese Museum \& Museum Napoleon****, Guest Apartments***, State Apartments*****, Private Apartments*****), Paris: Eiffel Tower*****, Tour Montparnasse*****\\

March 25: Paris:\\
This day we started on the outskirts of St Denis. The basilica of St Denis is the first gothic style building in the world, the stained glass is though largely remodelled from the 19th century. This is though not the point why you should go there. In St Denis a large majority of French kings are buried, thus the tombs range from ancient medieval to Renaissance and Baroque monuments, really unique to see, only Westminster Abbey can compete in this respect. Then we went back to Paris to visit the neo-baroque amazing beautiful Opera Garnier, the most beautiful theatre I have seen so far. If you shouldn't be into Baroque aesthetics maybe a large ceiling painting done by nobody else but Marc Chagall can sweeten the deal. For shopping mall lovers Galleries Lafayette just next to the Opera offers a unique glimpse of the grandiose splendour of malls of the 19th century, and the view of the Opera house from the roof top terrace is nice too. \\
Napoleon I and his family members aren't buried in St Denis but in the Dome d'Invalides attached to the French Army museum, the church itself is of late Baroque classic style. Another masterpiece of classicism is the Pantheon, originally built as a church, which is now the place where major figures of France, such as Victor Hugo are buried nowadays. And our final destination was Hotel Jacquemart-Andre, which displays the saloons and the art collection of this family. Then we rushed back to the hotel did shop Baguettes, cheese and salmon for dinner. Sometime during the day I had also strained my ankle, which didn't stop me from continuing our program, but feet proofed to be pretty swollen at that point. We took the metro which was supposed to bring us to Gare de Lyon the quickest, but there seemed to be a problem on the line, at least the metro stopped for 10 minutes, thus we rushed to another station, jumped on another metro line and made it to the TGV just 5 minutes before departure. But we still made it, and then we could enjoy our French style dinner on the way back to Geneva.\\

St Denis: Basilica*****\\
Paris: Opera Garnier*****, Galleries Lafayette****, Dome d'Invalides****, Pantheon****, Hotel Jacquemart-Andre****

\section{April 3--April 11: Roma}
\label{Roma2013}

April 3:\\
Flying over the Alps in early April is fun, particularly if you start below the clouds, and just a couple of minutes after the start you finally see the Alps with Mont Blanc appear above a sea of white fluffiness. Everything was covered in snow, naturally Mont Blanc was all white, but everything else was sunlit and bright as well. Landing in Rome i had to wait another 90 minutes, since my mum's plane from Zurich was delayed by over an hour. Arriving at our hotel we had to rush walk quite a bit to make it to the Palazzo Farnese. One of the highlights of Italian Renaissance, this palace is nowadays the French embassy. Naturally our tour was held in French. Although I live in a French speaking place, I am not particularly fluent in French, my mum's French might be even superior, she had quite a couple of years of French at school, though a couple of years ago by now. I got though most of what we were told. We saw the hall where the famous Hercules of Farnese had been standing for many years, before it was transferred over to Naples. And then we saw the Carracci Gallery. Unfortunately only photos in the courtyard and the gardens were allowed. Nowadays the embassy even forbids that.\\

Palazzo Farnese****, St Andrea della Valle****, Pantheon*****, St Maria Sopra Minerva****, Marc-Aurel-Column****, Spanish Stairs****\\

April 4:\\
Santa Maria degli Angeli****, Vatican Museums*****, St Peter's Basilica***** (with Dome*****), Palazzo Primoli***, Piazza Navona***** (with Sant'Agnese in Agone****), Il Gesu*****, Santa Maria in Aracoeli*****, Capitoline Museums*****, Santi Apostoli****, Oratorio Santissimo Crocifisso***, Trevi Fountain****\\

April 5:\\
Santa Maria Maggiore*****, San Prassede****, Santa Pudenzia***, Santa Croce****, San Giovanni in Laterano*****, Palazzo Laterano****, Battistero Laterano****, Villa dei Quinitili****, Via Appia Antica****, San Sebastiano fuori le Mura****, Aurelian Walls***, San Paolo fuori le Mura****\\

April 6:\\
Piazza Navona*****, Santa Maria in Aquiro***, Nostra Signora del Sacro Cuore**, Palazzo Colonna***** (+Private Apartments*****), Palazzo Doria Pamphilj****, Santi Apostoli****, San Ignazio*****, Pantheon*****, St Maria Sopra Minerva****, Sant'Angelo in Pescheria**, Marcellus Theatre***, Santa Maria in Cosmedin***, Santa Sabina****, Caracalla Baths*****, Santo Stefano Rotondo****, Scala Scanta***, Palazzo Madama****\\

April 7:\\
Sant'Andrea al Quirinale****, Palazzo Quirinale*****, Trajan's Column****, Santissimo Nome di Maria al Foro Traiano***, Monumento Vittorio Emanuele II***, Santa Maria di Loreto***, Imperial Fora****, Santi Cosma e Damiano****, Villa Torlonia*****,  Santa Constanza****, Sant'Agnese fuori le Mura****, Palazzo Montecitorio****\\

April 8:\\
San Silvestro***, San Claudio**, Pantheon*****, San Andrea della Valle****, Villa Farnesina*****, San Pietro in Montorio \& Tempietto***, Santa Maria in Trastevere****, Santi Quaranta Martiri e San Pasquale Babylon***, San Francesco a Ripa in Trastevere***, Santa Cecilia in Trastevere***, Kolosseum****, Forum Romanum \& Palatine Hill****, Santa Francesca Romana****, Santi Giovanni e Paolo****, San Clemente (all lower levels too)*****, San Pietro in Vincoli***, Trajan's Baths*, San Carlo alle Quattro Fontane****\\

Arpil 9: Tivoli:\\
Tivoli: Villa Hadriana*****, Villa d'Este*****, Villa Gregoriana*****, Rom: Santa Maria del Popolo****, Santa Maria dei Miracoli***, Spanish Stairs****, Trinita dei Monti***\\\

April 10:\\
Galleria Borghese*****, Villa Medici***, Ostia Antica*****\\

April 11:\\
San Lorenzo fuori le Mura****, Baths of Diocletian****

\section{April 14: Grandvaux}
\label{Lavaux2013}

Lavaux*******

\section{May 1: Le Mole}
\label{Mole2013}

Why the Mole: The Mole is a prominent mountain in the pre-alps. Since it is close to Geneva some call it the mock-up Mont Blanc since it appears to be of about the same size due to its proximity. Considering May 1st is a traditional hiking day, and a holiday, and my boss was in town all over from LA for a CERN workshop we decided it could be a great UCLA group event. Since we had one spare seat we invited Rosana, Indara's room mate on the hiking trip, and she gladly joined too.\\

Co-travellers:\\
Pieter: my fellow UCLA postdoc originating from Belgium. Pieter is a far more experienced hiker than myself, though clearly always up for hiking and the alps.\\
Andrew: UCLA undergrad turned electronic engineer who came to CERN in 2013 to help me with ongoing muon chamber testing.\\
Rosana: physics grad student at the university of Milan, based at CERN as well, originating from Cuba and more than pumped to explore the mountains.\\
David: physics professor at UCLA, having been at CERN as a fellow for a couple of years David is no stranger to mountains and hiking, considering that my program back in Lisbon was a tad full for someone just flying from LA the day previously a nice normal easy hike seems to be a better way to get started.\\
special mention: Line, working for CERN IT, Line welcomed us after the hike for beers and snacks to cool down our heated up skin, but more about that later.\\

The weather forecast was anything but spectacular, clouds and fogs all day long. Since May 1st just happens on that day we didn't shift the hike since it was supposed to stay dry but just covered. Since all was below layers of clouds and fog none of us though about bringing any sun screen. We all jumped into David's car and stopped 1/3 up the mountain to start the hike. The trail was largely snow free, but in some places still snow was covering the trail, but it was more on little side valleys. About 100-200 metres down the mountain top the fog started to get less dense and about 50 metres below the peak we were in bright sunshine. Good for views and good for photos. Which we clearly took plenty of, mountain tops above a sea of fog just looks magical. We had our snacks up there and also enjoyed the snow field which still covered most of the top, unfortunately even 1 hour on the top was enough for most of us, even Rosana did get burnt, the only one who thought ahead was David who brought a sun hat with him. Anyways Andrew had his first tasting of Swiss Landj\"ager (or Gendarms as the Swiss Romands call them). And we all walked back through the fog and a nice breeze. \\
Having arrived down the mountain we all went to Pieter's for more beer and snacks, where Line was a superb host but also commented on the fact that all four of us had forgotten to think about something practical such as sun screen considering we all had backpacks on us - guess the none practical traits of physicists. Still a very fun May 1st, particularly from a May 1st 2021 perspective which is the day I write this entry.\\

Le Mole****

\section{May 25: Lyon}
\label{Lyon2013}

And the evolution of this trip: Having been in Lyon before, I decided to try to see new things: the tourist association of Lyon offers from time to time guided tours of certain buildings in the city. This time, coincidentally both city hall and opera house had tours scheduled. Thus I convinced Andrew and my brother to join me. Having talked to Indara earlier, she didn't impress any particular interest in Lyon. But once she realised I was taking Andrew a full Saturday day away from here, she was quit displeased to say the least (some Buenos and beers might have made her happy again a couple of days later though, oh and Apfelschorle).\\

Co-travellers:\\
Andrew: after our trip to Paris a couple of weeks ago, Andrew wants to see a bit more of France and not only Paris.\\
my younger brother: having been in Lyon a couple of months ago, ready to see the places which are typically not easily accessible\\

Equipped with Baguette and a couple of cheese, salmon and sausages we had all ready before getting on that trip. Having arrived as early as possible we visited the cathedral of Lyon first on our own. A typical gothic style French cathedral, not as amazing as some of those world renown cathedrals of Ile de France, but still nice and interesting to see with stained glass, a nice astronomical clock, unfortunately with a closed off choir area, which was in renovation at that point. And then at 10 am our tour of the Hotel de Ville started. In total we saw the Staircase of Honour and 10 rooms. The city hall is in fact quite large and almost all rooms are from the Baroque to the Empire style era, or even older such as the Hall of Coat of Arms. All full of stuccos, gold leaf decorations, tapestries, paintings, all very much in good French Style. \\
After a lunch break of about an hour the tour of the Opera house started. While the Foyer is still from the previous Opera house and thus very classic ornate in typical 19th century style, the rest of the Opera house has been completely rebuilt. And the architect thought it is a great idea for the flair and the sounds to have everything in dark black. This does include all escalators, all walls of the rehearsal area, chairs, balconies, everything, Even a couple of years later Andrew still comments on the fact that he thinks the main auditorium of Lyon's opera might be one of the ugliest interior's he has ever seen (while I in fact did enjoy it) so I would say with modern buildings it is all a matter of taste after all. I do agree though that the round roof top looks a bit out of place if you consider the outer appearance. After the tour finished we went up to the remains of the Roman amphitheatre, but besides a few rows of seats not much else remains from it, so you can safely skip it. On a hill top Notre Dame de Fourviere, a neo-byzantine revival basilica overlooks old town. Particularly if you are into mosaics you will be blown away by this church. Usually such churches were built with great plans in mind, but the mosaics were never realised (e.g. Westminster Cathedral), or are very slowly installed one after each other (the catholic basilica in Washington DC). But here the full ceiling and the walls are covered in many mosaics, so definitely go there when you can. \\
Next to the Fourviere basilica are the remains of the Roman Theatre and Odeon, still very well conserved, in France second only to the theatre in Orange. In summer (unless it is summer 2020 and maybe 2021) plays and concerts happen here regularly. Then we walked through the Part Dieux quarter in old town, which is famous for its open courtyards and old houses. Don't forget that Lyon is also considered by many as food capital in France, so maybe try to have some Oysters or some Escargot while you can.\\

Lyon:Cathedrale***, Hotel de Ville*****, Opera****, Notre Dame de Fourviere*****, Roman amphitheatre*, Roman Theatre \& Odeon****\\

\section{June 16: Reculet}
\label{Reculet2013}

Jacob was back in the CSC factory for the summer. Since in June there is still snow on the trails in the alps, we decided to hike in the Jura. While I had been hiking there before, for example to a fondue place between Reculet and Crete de la Neige, I so far had not been up to Reculet itself after all those years. Since I had a car around CERN nowadays due to my work in Prevessin I drove up to the Tiocan parking with my younger brother, Jacob, Jesse, and Rosana who we convinced to join us on a hike. The hike itself is pretty nice, I decided to take it a bit slower than Jacob and Jesse which went ahead quickly, taking a couple of photos instead. The hike itself is pretty steep particularly through the forest, while it is in fact more easy at the end close to the top, since the trail is less covered in rocks. From up there we had a nice few, while Jacob decided to climb up the radio mast. We enjoyed he nice breeze up there, took a couple of photos, unfortunately a bit hazy to get good views of the alps, enjoyed the lizard which warmed itself up on my camera bag, and then walked all the way down, where we had ice cream and coffee to finish off the nice beautiful afternoon.\\

Reculet****\\

This was so far the last trip with Jacob, who decided to continue his career outside of high energy physics in medical physics and radiological devices. Rosana remained in the Geneva area for a couple of more years, but we didn't do further hikes with each other, but did enjoy some parties and dinners later on. It was also the last hike I did with Jesse but we also had several more celebrations together, and even after he moved all the way to Canada we met up later in Geneva from time to time and for a last time at ICHEP 2018 in Seoul, where we enjoyed some nice Korean barbecue getting a bit nostalgic about the good times we had back in Geneva (or Ferney for that matter).

\section{July 4: Flegere}
\label{Flegere2013}

How we got everybody to Chamonix: as UCLA postdoc I have many US American friends and the 4th of July is Independence day and a valid reason to take the day off go into the mountains, taking all grad students on a hike, and then enjoying a gigantic barbecue by one of the professors houses. \\

Co-travellers: all young UCLA folks at that point:\\
David: my boss having come over from UCLA once again. In fact it was him who decided nobody working for UCLA should work on that day (no matter if CERN says it is a working day), and we should rather spent another team building event, particularly due to the fact that we had also summer students around now.\\
Pieter: my fellow UCLA postdoc originating from Belgium. Pieter knows many hikes in the area, in fact he suggested we do the Lac Blanc hike which is nice but not too challenging if some people would not have large hiking experience so far.\\
Eve: a starting PhD student at UCLA originating from Greece. Eve had been a summer student at CERN for another friend of mine previously, thus she knew already about CERN itself. In the end Eve decided to focus on other fields for her PhD thesis and enjoyed more time in the LA area from then on.\\
Cameron: UCLA grad student, originating from the US, Kansas City but the part which is in Missouri. Cameron had been in Switzerland for almost a full year previously, working as an exchange student at ETH - my Alma Mater in fact -  or rather at PSI on the CMS silicon pixel detector. Thus no stranger to mountains, Cameron happily joined our hike as well.\\
Andrew: also from UCLA, originating from California and after our hike of the Mole more than ready to get into higher mountains this time.\\
Nick: a Texas A\&M undergrad, supervised by Indara, and he worked on rewriting parts of the code which I used for testing. Since he had arrived at CERN just a bit early on, and stayed at the same place like Andrew, we happily adopted him for this hike.\\

Unfortunately the weather forecast was once again not predicting sun for all the day, but what can you do. Once we arrived in Flegere we took the cable car to the middle station, seems one golfer on the golf course next to the cable car thought it would be fun to hit the cable car, and indeed he hit the target (if not intentionally kudos to Murphy's law). We started the hike in very foggy weather, we couldn't see more than 10 metres far, at time there was still snow on the trail. But nothing was too slippery or dangerous, the trail is also equipped with ropes and ladder on steep territory. Once we got close to the Aigulle du Belvedere and Lac Blanc the fog cleared up, but most of the lake was still covered in ice and a thick layer of snow. I was not as courageous as Eve, Andrew and Pieter to walk on the snow though, but seems it was firm enough to carry multiple people. On the other side of the valley clouds covered the mountain top, thus no view of Mont Blanc for us, the tongue of the glaciers of Argentiere or the Mer de Glace were though still visible, but it being summer they rather appeared grey with gravel than white in snow. On our way back we crossed some little mountain creek, where Andrew and Nick tasted some mountain water, later on the clouds cleared up to reveal the Dent du Geant with the Glacier des Periades. We also passed a large waterfall. Finally the clouds cleared up far enough that we could see Mont Blanc and the Glacier des Bossons, at least partially. At one point a couple of chamoises crossed our trail and were curious for a couple of seconds what we might be up to. And then the hike was finished, we were ready to eat, and thus it was ideal to have a couple of drinks and a lot of grilled food at Bob's BBQ.\\

Chamonix: Lac Blanc*******

\section{July 7: Milano}
\label{Milano2013}

And how we got to Milan this time around: Since I made it my task to show Andrew a bit of Europe, I thought after having done a bit of France it would be nice to see another country too, for example Italy. Indara heard that too, and since she wanted Nick to see a bit of Europe too she suggested he could join our trip to Milan as well. We decided on the date with everybody, Indara didn't want to join on that date but told us to go ahead. What she missed to consider is that we planned to take the earliest train possible. Thus Andrew and Nick stayed over at my place (they didn't like the fact that the road was so loud at night, and the fact that without an open window in summer the studio was not amazing either, thus no good sleep for them). The day before Indara planned her birthday party. Unfortunately she also did an outing to Bern but then messed up taking the right train, thus she arrived 2 hours late to her own party and she wasn't really happy when we left already a bit early, since getting on a train at 5:40 am means one should get a tiny bit of sleep before maybe.\\

Co-travellers on that trip: Andrew and Nick\\

Having failed to get into the Palazzo Reale in Milano the first time I visited, this time I put it as first item on the agenda. This time a photo exhibit was taking place in the former guest apartments, and Andrew enjoyed replicating the poses of the models to our great pleasure. Afterwards we saw the tapestry halls and the hall of lanterns of the state apartments. \\
Afterwards we walked over to close-by churches, and this was the first time I visited the ossuary of San Bernardino alle Ossa. This was the first time I walked into an Ossuary, which is a chapel where the skeletal remains of bodies are kept, if a cemetery is given up. Thus the chapel itself is decorated using bones and skulls instead of stuccos or paintings. I thought it was very fascinating and whenever I visited Milano nowadays I tend to bring people there. \\
There are many other baroque churches I visited on that trip, also beautiful once such as San Fedele, San Giorgio al Palazzo or Sant'Alessandro in Zebedia (many frescos decorate this one). This time was the only time I visited La Scala, Milan's renown Opera house, which had been rebuilt after a devastating fire in its original style. The quality of the opera house is obviously outstanding, the amphitheatre itself is a standard classical opera house, also not allowed to take photos, so maybe skip it if you want to see the theatre of itself. Do instead the Galleria Vittorio Emanuele II instead, an outstanding example of 19th century galleries. \\
The highlight of Milano is its cathedral. The duomo is a gigantic masterpiece of Italian gothic, the largest church in Italy. Its capitals are decorated with statues and sculptures, the windows in the choir are some of the best stained glasses from Renaissance times in Italy. Then we saw the museums of Castello Sforzesco. The former residence of the dukes of Milano is now a museum with old equestrian monuments, Michelangelo's last sculpture (the unfinished Pieta Rondanini), medieval altars, armors, and last but not least the actual halls of the castles itself, some of them painted by none other than Leonardo da Vinci (e.g. the Sala delle Asse). \\
Once again we didn't manage to get tickets for Leonardo da Vinci's last supper, but the aps of the convent church of Santa Maria delle Grazie itself is considered a prime example of Renaissance church architecture and one of Bramante's best works. Should you rather be into more historic things, Roman columns can be found next to San Lorenzo, whose baptistery chapel contains early christian mosaics of the 4th century or the romanesque basilica of Sant'Ambrogio. And then all back on the last train leaving Milan. Since it was in mid of July, we did also get to see the glacier covered peaks of the Swiss alps in sunset.\\

Milan: Santo Stefano Maggiore***, San Bernardino alle Ossa*****, San Fedele****, La Scala***, Galleria Vittorio Emanuele II*****, San Sebastiano***, Duomo***** (no roof this time), Castello Sforzesco*****, Santa Maria delle Grazie****, Santa Maria presso San Celso****, Sant'Ambrogio***, San Lorenzo****, Palazzo Reale****, San Giorgio al Palazzo***,\\ Sant'Alessandro in Zebedia****

\section{July 14--July 28: Sweden \& Denmark}
\label{2013:SwedenDenmark}

July 15: Stockholm:\\
Cathedral****, Royal Palace*****, Riddarholm****, Ridderhaus****, Parliament***, Concert House****\\

July 16: Drottningholm\\
Drottningholm Palace***** (Chinese Palace****, Theatre****), Ulriksdal Palace****, Stockholm: Hallwyl Palace*****\\

July 17: Mariefred\\
Gripsholm Palace*****\\

July 18: Stockholm:\\
City Hall*****\\

July 20: Skokloster\\
Skokloster Palace****, Stockholm: Royal Palace*****\\

July 21: Roserberg\\
Rosersberg Palace*****, Stockholm: Rosendal Palace****, Prince Eugen's Waldemarsudde****\\

July 22:\\
Stockholm Archipelago*****\\

July 23:\\
Hagapark***, Pavillon Gustav's III****, Wasamuseum*****\\

July 25:\\
Hillerod: Frederiksborg Palace*****, Copenhagen: Christiansborg Palace*****, Holmenchurch***, Frederik's Church***, Little Mermaid****, Nikolai Church***\\

July 26:\\
Helsingor: Kronborg****, Fredensborg Palace*****, Copenhagen: Tivoli*****\\

July 27:\\
Copenhagen: Amalienborg Palace: Palace Christian VIII*** and Palace Christian's VII*****, Rosenborg Palace*****, Tivoli*****\\

July 28:\\
Glypothek*****, Christiansborg: Parliament****, Cathedral***, Danish National Museum****

\section{September 5--September 8: Venice}
\label{Venice2013}

September 5:\\
Ca' Rezzonico*****, Palazzo Zenobio****, Palazzo Franchetti***, Ca' Giustinian***\\

September 6:\\
Ca' d'Oro***, San Giacomo di Rialto***, Palazzo Pisani Moretta****, Casa di Carlo Goldoni*, Scuola Grandi di San Rocco*****, Ca' Foscari**, San Pantalon****, Scuola Grande di San Giovanni Evangelista****, San Giovanni Evangelista***, Scuola Grande dei Carmini*****, Teatro la Fenice****, Palazzo Grimani****, Palaqzzo Querini Stampalia***, Palazzo Reale****, Biblioteca San Marco****,  Basilica di San Marco*****, Palazzo Ducale*****\\

September 7:\\
San Salvador**, Palazzo Ducale*****, San Giorgio Maggiore****, Palazzo delle Prigione***, San Maurizio*, Santa Maria della Salute****, Santi Apostoli***,  Palazzo Fortuny*\\

September 8:\\
Galleria Academia****, Palazzo Grassi****, San Giovanni Elemosinario***, Ca Corner***, Ca Pesaro***, Palazzo Pisani a Santa Marina*, San Giovanni e Paolo*****, Palazzo Mora**, Palazzo Michiel dal Brusa***, Palazzo Albrizzi***

\section{September 21: Chamonix}
\label{Chamonix2013}

Glacier des Pelerins****, Glacier des Bossons*****, Aiguille du Midi*****

\section{November 7--November 11: Paris II}
\label{2013ParisII}

Did I mention that I love going to Paris? Well, as a matter of fact, I do, but this time I wanted to see places, which are less common or frequently visited by first time Paris travellers. But once again I had first time Paris travellers with me - Chris and Chris.\\

Co-travellers:\\
Eric: UCLA grad student after he didn't make it the first planned trip this year, Eric wants to make sure to not miss the train this time around:\\
Chris L: UCLA undergrad: our hardware project needed re-enforcement and we were happy to get Chris on board. As a first trip close to the region I suggested Paris. Since Chris has friends living in Paris, he clearly was up for it.\\
Chris C: US-American, a grad student working my hardware project building muon chambers. Always up to spend a good evening out, Chris thinks it is time for a longer trip. His aunt always told him how amazing Paris is, so he wants to make sure to take photos he can show her once he will be back home.\\

November 7:\\
Taking our usual late Friday night TGV out of Geneva, tolerating the not that amazing snacks on our way, we arrive at the hotel and get out room for four people on the top floor at a hotel close to the Place de la Republique.\\

November 8:\\
Typically people visit either the palace of Versailles or to lesser extend Fontainebleau. But only about an hour out of Paris is a third palace used by the emperors of the Bonaparte family, the palace of Compiegne. Almost all of the rooms have been decorated in the Empire style, they are in fact very beautiful too. The chateau might miss the old Renaissance or Baroque period rooms as you find them in the other two royal palaces, but what you find is definitely top notch too. It was a very rainy day, thus we didn't spend any time in the vast park. We visited the town church of St Jacques, which was a decent church. Plenty of churches and old houses belong to the buildings inscribed in the UNESCO world heritage as part of the pilgrim paths leading to Santiago de Compostela, and this church was one of them. Then we took a train to Chantilly. We walked through the little town along the forests and the horse riding race track to the castle of Chantilly. This chateau has very exquisite Baroque state rooms, which everybody was impressed by, particularly the large gallery, which has also paintings of the Three Graces by Raffael. Everybody else preferred skipping the French tour of the private apartments over a visit of the Baroque gardens. It wasn't raining as heavily anymore, so that was enjoyable too. And then we took the train back and did a late night visit of the Louvre. During the late night visits certain sections are usually closed off, in our case the Egyptian part of the exhibition. We saw though the Mona Lisa, the gallery of Apollon, the Venus de Milo, the Empire style rooms of the former ministry of finance of Napoleon III, as well as Assyrian and Babylonian artefacts. The Louvre is most probably my second most favourite museum I have seen so far (only beaten by the Vatican). \\

Compiegne: Chateau*****, St Jacques***\\
Chantilly: Chateau*****\\
Paris: Louvre*****\\

November 9:\\
And our day started with a special visit of the Banque de France, or rather the representative rooms of the Hotel de Toulouse. Only a handful of tours are given by the French centre of national monuments, and the rooms can be visited during the European heritage weekend. I realised on the weekend I chose, the headquarters had a tour scheduled, and I successfully inscribed ourselves. Unfortunately at first they only claimed to accept payment by cheque, but my bank told me they wouldn't issue cheques anymore. Faced with that issue, I wrote the organisers an email, that either we would pay the exact amount of cash for all four of us, or we unfortunately would have to cancel our participation. This proposal was fortunately accepted, I handed over the cash to our guide, and we were ready to go. Indeed the tour was very nice, the golden gallery is a very fine and nice room with great frescoes and nice sculptures covered in gold leaf. Other meeting rooms were full with precious wood carving and tapestries. The Hotel de Toulouse is definitely worth the obstacles. After this visit both Chris's took of for the palace of Versailles. Both enjoyed it, took though the gold card to get through the vast park. Eric and I made out way to Malmaison. The castle of Malmaison had been the private home of Josephine de Beauharnais, the first wife of Napoleon I Bonaparte, and the last house Napoleon himself lived in, before being sent to exile on St Helena in 1815. The chateau sits in a nice small garden, with small but nice and finely equipped rooms, including the very beautiful library. Then we both had one grilled chicken from a food truck, and then we wanted to get to Maisons-Lafitte. That proved to be more difficult than we thought. The RER line which was supposed to get us there straight away had been closed for construction on that day, but another line was supposed to be used instead. After arriving there we found out, this line as well was closed for work and it suggested we use the previous one instead. Then we found out that a bus line was supposed to get us there, but we had to find out, indeed it has a final stop exactly at the castle we wanted to reach, but only later in the evening or early in the morning. We then had a coffee at McDonalds in order to get on a wireless network which could tell us what to do instead, and it provided us with a solution: We had to take a tram line up to its final stop and then transfer to another bus. And we finally arrived at Maisons-Lafitte, but over an hour later than originally planned. Now I decided to watch the state rooms. These were illuminated partly in green and red to test out the lightning for a private event which was planned to take place later that evening. While I managed to see all the state rooms, Eric decided to see the private rooms instead. Unfortunately we arrived that late, that we each didn't manage to see all the rooms. But what I saw was very nice, so not a wasted opportunity. Who knows, if I will come there again at some point (at least not until now in 2020). And then it was time to visit the observatory on the roof terrace of Tour Montparnasse. Still a very superb view, but instead of offering an unobstructed view of Paris, now large life-size glass windows had been installed, on some places with tiny gaps to take photos, clearly not improving the visitor experience. Still the view of Paris, particularly of the Eiffel tower is and remains magnificent (but a bit more expensive than walking up the second floor of the Eiffel Tower, but a better view since you see the Eiffel Tower). On our way back to the hotel Eric and I stopped for a short while by Notre Dame, took a view pictures of the cathedral, the river, as well as the Hotel de Ville by night and then had food.\\

Paris: Banque de France (Hotel de Toulouse)*****, Tour Montparnasse*****\\
Malmaison: Chateau****\\
Maisons-Lafitte: Chateau****\\

November 10:\\
Today we all planned to have an excursion day doing nothing in Paris downtown. The first stop was the chateau of Champs-sur-Marne. A wonderful small Baroque palace with several wonderful nice rococo salons, like the Chinese salon, it used to be the French guest house between 1959-1974. The chateau is surrounded by a French style park. After a short lunch we took the RER to Meudon, and then a bus to the large estate of Vaux-le-Vicomte. The Baroque chateau was built for Nicolas Fouquet, the superindent of finances during the times of Louis XIV. The decoration of the halls were done by Charles Le Brun, the park was designed by Andre le Notre, both later worked on the construction and planning for the palace of Versailles. Regarding the castle itself, particularly the state rooms on the lower floor are magnificent with the state bedroom of the King's apartment and the game cabinet. While we were there a chocolate festival was taking place, the palace was quite crowded, but the chocolate was very tasty. The park is quite extensive. It is the oldest Baroque garden in France and its layout was later used as basis for the gardens of Versailles with fountains, sculptures, vases, and a geometric parterre. The bus arrived pretty late and we had to ran to catch the train, Chris C fell, but besides a few bruises nothing drastic happened. The four of us just made the train a few seconds before it took off. Chris L had a long night out with his friends, while the three of us just had a dinner followed by a night cap.\\

Champs-sur-Marne: Chateau****\\
Vaux-le-Vicomte: Chateau*****\\

November 11:\\
On our last day we stayed closed to the city centre. We started in St Denis with a visit of the basilica, enjoying once again the Royal tombs and the royal crypt. I am always impressed by the tombs of Louis XII. and Francois I, which show am impressive detail and skills of French Renaissance artists. The later tombs of the Bourbon kings in the crypt are in fact less sumptuous. The next highlight was a visit of the Opera Garnier, I am always happy to come back to what I consider the most beautiful theatre/opera/concert building I have been to so far. After the obligatorily stop by the Conciergerie, we were all impressed the stained glass of Sainte Chapelle (the renovation was still ongoing), Chris C and Eric bought little pieces of stained glass as memory and presents. And last but not least we stopped once again by Notre-Dame cathedral for a while before getting dinner, this time I opted for duck and fries. Then we had a photo stop by the Arc de Triomphe before getting a couple of snacks and drink for our 3 h long train ride back to Geneva.\\

St Denis: Basilica*****\\
Paris: Opera Garnier*****, Sainte-Chapelle*****, Conciergerie***, Notre-Dame*****

\section{November 30: Prangins}
\label{2013:Prangins}

Prangins: Chateau de Prangins****

\section{December 27: Strassbourg}
\label{2013Strassbourg}

Strassbourg: M\"unster*****, Palais Rohan*****
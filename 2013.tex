\chapter{Year 2013}
\label{2013}

\section{January 2--January 6: Netherlands}
\label{2013:Netherlands}

January 2:\\
Amsterdam: Rembrandthouse**, Royal Palace***, Nieuwe Kerk*, Oude Kerk**, He Hua Temple*\\

January 3:\\
Amsterdam: van Loon House**, Willet Holthuysen House, Appeldoorn: Het Loo Palace***\\

January 4:\\
The Hague: Binnenhof with Knight's Hall**, Lange Voorhout Palace (Escher Museum)***, Fredenspalace*, Gemeentemuseum*, GEM**\\

January 5:\\
Soestdijk Palace**, Amsterdam: Beurs van Berlage*, Geelvinck Hinlopen House**, Rijksmuseum**, Stedelijkmuseum***\\

January 6:\\
Amsterdam: Portuguese Synagogue*, Jewish Museum (Large Synagogue)**, NEMO

\section{March 21--March 25: Paris I -- sponsored by Eric}
\label{2013:ParisI}

Why Paris: why not, well after all Paris is one of my most favourite places to see (having been there over and over again). Paris is also the closest big city from Geneva and a good place to show to your friends. This time I decided to see the private apartments at Versailles as well as at Fontainebleau, so should be big fun. 

Co-travellers:\\
Andrew: A UCLA undergrad student, who came over to help us testing muon chambers. His first time in Europe, I decided the first thing to see should be Paris, particularly since it is so easily accessible by train.\\

March 21: The train ride\\
As the afternoon passed Eric was still busy down in the cavern and I commented that it might get tight if he does not return immediately. For reasons unknown to me he still continued working and indeed missed the train ride (and wouldn't even join later on, so in a way he sponsored our trip. Once we arrived by the hotel we realised it was in renovation (actually more the hallway and the elevator). No issue to carry our not so big luggage up the flight of stairs though. The door to our suite was a bit small, though not too small for me to go through straight up, and Andrew didn't mind to claim the bed for two for himself alone.\\

March 22: Versailles\\
After that successful first day I wanted to try out something I didn't know, so I got myself a Monaco which was listed under beers. Little did I know that I got myself beer mixed with grenadine juice and 7up. At least Andrew enjoyed watching me drink it.\\

Versailles: Chateau*** (State Apartments***, Private Apartments***, Trianons***), Hotel de Ville*, Paris: Louvre***\\

March 23: Paris\\
Paris: Place de la Concorde***, Palais Bourbon***, Conciergerie*, Sainte-Chapelle***, Notre Dame***, Park de Chateau Bagatelle**, Hotel Carnavalet**, House of Victor Hugo*\\

March 24: Fontainebleau:\\
Fontainebleau: Chateau*** (Chinese Museum \& Museum Napoleon***, Guest Apartments*, State Apartments***, Private Apartments***), Paris: Eiffel Tower***, Tour Montparnasse***\\

March 25: Paris:\\
St Denis: Basilica***, Paris: Opera Garnier***, Dome d'Invalides**, Pantheon**, Hotel Jacquemart-Andre**

\section{April 3--April 11: Roma}
\label{Roma2013}

April 3:\\
Flying over the Alps in early April is fun, particularly if you start below the clouds, and just a couple of minutes after the start you finally see the Alps with Mont Blanc appear above a sea of white fluffiness. Everything was covered in snow, naturally Mont Blanc was all white, but everything else was sunlit and bright as well. Landing in Rome i had to wait another 90 minutes, since my mum's plane from Zurich was delayed by over an hour. Arriving at our hotel we had to rush walk quite a bit to make it to the Palazzo Farnese. One of the highlights of Italian Renaissance, this palace is nowadays the French embassy. Naturally our tour was held in French. Although I live in a French speaking place, I am not particularly fluent in French, my mum's French might be even superior, she had quite a couple of years of French at school, though a couple of years ago by now. I got though most of what we were told. We saw the hall where the famous Hercules of Farnese had been standing for many years, before it was transferred over to Naples. And then we saw the . Unfortunately only photos in the courtyard and the gardens were allowed. Nowadays the embassy even forbids that.\\

Palazzo Farnese**, St Andrea della Valle**, Pantheon***, St Maria Sopra Minerva**, Marc-Aurel-Column**, Spanish Stairs**\\

April 4:\\
Santa Maria degli Angeli**, Vatican Museums***, St Peter's Basilica*** (with Dome***), Palazzo Primoli*, Piazza Navona*** (with Sant'Agnese in Agone**), Il Gesu***, Santa Maria in Aracoeli***, Capitoline Museums***, Santi Apostoli**, Oratorio Santissimo Crocifisso*, Trevi Fountain***\\

April 5:\\
Santa Maria Maggiore***, San Prassede**, Santa Pudenzia*, Santa Croce**, San Giovanni in Laterano***, Palazzo Laterano**, Battistero Laterano**, Villa dei Quinitili**, Via Appia Antica**, San Sebastiano fuori le Mura**, Aurelian Walls*, San Paolo fuori le Mura**\\

April 6:\\
Piazza Navona***, Santa Maria in Aquiro*, Nostra Signora del Sacro Cuore, Palazzo Colonna*** (+Private Apartments***), Palazzo Doria Pamphilj**, Santi Apostoli**, San Ignazio***, Pantheon***, St Maria Sopra Minerva**, Sant'Angelo in Pescheria*, Marcellus Theatre*, Santa Maria in Cosmedin*, Santa Sabina**, Caracalla Baths***, Santo Stefano Rotondo**, Scala Scanta*, Palazzo Madama**\\

April 7:\\
Sant'Andrea al Quirinale**, Palazzo Quirinale***, Trajan's Column**, Santissimo Nome di Maria al Foro Traiano*, Monumento Vittorio Emanuele II*, Santa Maria di Loreto*, Imperial Fora**, Santi Cosma e Damiano**, Villa Torlonia***,  Santa Constanza**, Sant'Agnese fuori le Mura**, Palazzo Montecitorio**\\

April 8:\\
San Silvestro*, San Claudio, Pantheon***, San Andrea della Valle**, Villa Farnesina***, San Pietro in Montorio \& Tempietto*, Santa Maria in Trastevere**, Santi Quaranta Martiri e San Pasquale Babylon*, San Francesco a Ripa in Trastevere*, Santa Cecilia in Trastevere*, Kolosseum**, Forum Romanum \& Palatine Hill**, Santa Francesca Romana**, Santi Giovanni e Paolo**, San Clemente (all lower levels too)***, San Pietro in Vincoli*, Trajan's Baths*, San Carlo alle Quattro Fontane**\\

Arpil 9: Tivoli:\\
Tivoli: Villa Hadriana***, Villa d'Este***, Villa Gregoriana***, Rom: Santa Maria del Popolo**, Santa Maria dei Miracoli*, Spanish Stairs**, Trinita dei Monti*\\\

April 10:\\
Galleria Borghese***, Villa Medici*, Ostia Antica***\\

April 11:\\
San Lorenzo fuori le Mura**, Baths of Diocletian**

\section{April 14: Grandvaux}
\label{Lavaux2013}

Lavaux***

\section{May 1: Le Mole}
\label{Mole2013}

Le Mole**

\section{May 25: Lyon}
\label{Lyon2013}

And the evolution of this trip: Having been in Lyon before, I decided to try to see new things: the tourist association of Lyon offers from time to time guided tours of certain buildings in the city. This time, coincidentally both city hall and opera house had tours scheduled. Thus I convinced Andrew and my brother to join me. Having talked to Indara earlier, she didn't impress any particular interest in Lyon, once she realised I was taking Andrew a full Saturday day away from here, she was quit displeased to say the least.\\

Co-travellers:\\
Andrew: after our trip to Paris a couple of weeks ago, Andrew wants to see a bit more of France and not only Paris.\\

my younger brother: having been in Lyon a couple of months ago, ready to see the places which are typically not easily accessible\\

Even a couple of years later Andrew still comments on the fact that he thinks the main auditorium of Lyon's opera might be one of the ugliest interior's he has ever seen (while I in fact did enjoy it) so I would say with modern buildings it is all a matter of taste after all. I do agree though that the round roof top looks a bit out of place if you consider the outer appearance.

Cathedrale*, Hotel de Ville***, Opera**, Notre Dame de Fourviere***, Roman Theatre \& Odeon**\\

\section{June 16: Reculet}
\label{Reculet2013}

Reculet**

\section{July4: Flegere}
\label{Flegere2013}

Lac Blanc***

\section{July 7: Milano}
\label{Milano2013}

Santo Stefano Maggiore*, San Bernardino alle Ossa***, San Fedele*, La Scala*, Galleria Vittorio Emanuele II**, San Sebastiano*, Duomo*** (no roof this time), Castello Sforzesco***, Santa Maria delle Grazie**, Santa Maria presso San Celso**, Sant?Ambrogio*, San Lorenzo**, Palazzo Reale**, San Giorgio al Palazzo*, Sant'Alessandro in Zebedia**

\section{July 14--July 28: Sweden \& Denmark}
\label{2013:SwedenDenmark}

July 15: Stockholm:\\
Cathedral**, Royal Palace***, Riddarholm**, Ridderhaus**, Parliament*, Concert House**\\

July 16: Drottningholm\\
Drottningholm Palace*** (Chinese Palace**, Theatre**), Ulriksdal Palace**, Stockholm: Hallwyl Palace***\\

July 17: Mariefred\\
Gripsholm Palace***\\

July 18: Stockholm:\\
City Hall***\\

July 20: Skokloster\\
Skokloster Palace**, Stockholm: Royal Palace***\\

July 21: Roserberg\\
Rosersberg Palace***, Stockholm: Rosendal Palace**, Prince Eugen's Waldemarsudde**\\

July 22:\\
Stockholm Archipelago***\\

July 23:\\
Hagapark*, Pavillon Gustav's III**, Wasamuseum***\\

July 25:\\
Hillerod: Frederiksborg Palace***, Copenhagen: Christiansborg Palace***, Holmenchurch*, Frederik's Church*, Little Mermaid**, Nikolai Church*\\

July 26:\\
Helsingor: Kronborg**, Fredensborg Palace***, Copenhagen: Tivoli***\\

July 27:\\
Copenhagen: Amalienborg Palace: Palace Christian VIII* and Palace Christian's VII***, Rosenborg Palace***, Tivoli***\\

July 28:\\
Glypothek***, Christiansborg: Parliament**, Cathedral*, Danish National Museum**

\section{September 5--September 8: Venice}
\label{Venice2013}

September 5:\\
Ca' Rezzonico***, Palazzo Zenobio**, Palazzo Franchetti*, Ca' Giustinian*\\

September 6:\\
Ca' d'Oro*, San Giacomo di Rialto*, Palazzo Pisani Moretta**, Casa di Carlo Goldoni, Scuola Grandi di San Rocco***, Ca' Foscari, San Pantalon**, Scuola Grande di San Giovanni Evangelista**, San Giovanni Evangelista*, Scuola Grande dei Carmini***, Teatro la Fenice**, Palazzo Grimani**, Palaqzzo Querini Stampalia*, Palazzo Reale**, Biblioteca San Marco**,  Basilica di San Marco***, Palazzo Ducale***\\

September 7:\\
San Sebastiano, Palazzo Ducale***, San Giorgio Maggiore**, Palazzo delle Prigione*, San Maurizio, Santa Maria della Salute**, Santi Apostoli*,  Palazzo Fortuny*\\

September 8:\\
Galleria Academia**, Palazzo Grassi**, San Giovanni Elemosinario*, Ca Corner*, Ca Pesaro*, Palazzo Pisani a Santa Marina, San Giovanni e Paolo***, Palazzo Mora*, Palazzo Michiel dal Brusa*, Palazzo Albrizzi*

\section{September 21: Chamonix}
\label{Chamonix2013}

Glacier des Pelerins**, Glacier des Bossons***, Aiguille du Midi***

\section{November 7--November 11: Paris II}
\label{2013ParisII}

Did I mention that I love going to Paris? Well, as a matter of fact, I do, but this time I wanted to see places, which are less common or frequently visited by first time Paris travellers. But once again I had first time Paris travellers with me - Chris and Chris.\\

Co-travellers:\\
Eric: UCLA grad student after he didn't make it the first planned trip this year, Eric wants to make sure to not miss the train this time around:\\
Chris L: UCLA undergrad: our hardware project needed re-enforcement and we were happy to get Chris on board. As a first trip close to the region I suggested Paris. Since Chris has friends living in Paris, he clearly was up for it.\\
Chris C: US-American, a grad student working my hardware project building muon chambers. Always up to spend a good evening out, Chris thinks it is time for a longer trip. His aunt always told him how amazing Paris is, so he wants to make sure to take photos he can show her once he will be back home.\\

November 7:\\
Taking our usual late Friday night TGV out of Geneva, tolerating the not that amazing snacks on our way, we arrive at the hotel and get out room for four people on the top floor at a hotel close to the Place de la Republique.\\

November 8:\\
Typically people visit either the palace of Versailles or to lesser extend Fontainebleau. But only about an hour out of Paris is a third palace used by the emperors of the Bonaparte family, the palace of Compiegne. Almost all of the rooms have been decorated in the Empire style, they are in fact very beautiful too. The chateau might miss the old Renaissance or Baroque period rooms as you find them in the other two royal palaces, but what you find is definitely top notch too. It was a very rainy day, thus we didn't spend any time in the vast park. We visited the town church of St Jacques, which was a decent church. Plenty of churches and old houses belong to the buildings inscribed in the UNESCO world heritage as part of the pilgrim paths leading to Santiago de Compostela, and this church was one of them. Then we took a train to Chantilly. We walked through the little town along the forests and the horse riding race track to the castle of Chantilly. This chateau has very exquisite Baroque state rooms, which everybody was impressed by, particularly the large gallery, which has also paintings of the Three Graces by Raffael. Everybody else preferred skipping the French tour of the private apartments over a visit of the Baroque gardens. It wasn't raining as heavily anymore, so that was enjoyable too. And then we took the train back and did a late night visit of the Louvre. During the late night visits certain sections are usually closed off, in our case the Egyptian part of the exhibition. We saw though the Mona Lisa, the gallery of Apollon, the Venus de Milo, the Empire style rooms of the former ministry of finance of Napoleon III, as well as Assyrian and Babylonian artefacts. The Louvre is most probably my second most favourite museum I have seen so far (only beaten by the Vatican). \\

Compiegne: Chateau***, St Jacques*\\
Chantilly: Chateau***\\
Paris: Louvre***\\

November 9:\\
And our day started with a special visit of the Banque de France, or rather the representative rooms of the Hotel de Toulouse. Only a handful of tours are given by the French centre of national monuments, and the rooms can be visited during the European heritage weekend. I realised on the weekend I chose, the headquarters had a tour scheduled, and I successfully inscribed ourselves. Unfortunately at first they only claimed to accept payment by cheque, but my bank told me they wouldn't issue cheques anymore. Faced with that issue, I wrote the organisers an email, that either we would pay the exact amount of cash for all four of us, or we unfortunately would have to cancel our participation. This proposal was fortunately accepted, I handed over the cash to our guide, and we were ready to go. Indeed the tour was very nice, the golden gallery is a very fine and nice room with great frescoes and nice sculptures covered in gold leaf. Other meeting rooms were full with precious wood carving and tapestries. The Hotel de Toulouse is definitely worth the obstacles. After this visit both Chris's took of for the palace of Versailles. Both enjoyed it, took though the gold card to get through the vast park. Eric and I made out way to Malmaison. The castle of Malmaison had been the private home of Josephine de Beauharnais, the first wife of Napoleon I Bonaparte, and the last house Napoleon himself lived in, before being sent to exile on St Helena in 1815. The chateau sits in a nice small garden, with small but nice and finely equipped rooms, including the very beautiful library. Then we both had one grilled chicken from a food truck, and then we wanted to get to Maisons-Lafitte. That proved to be more difficult than we thought. The RER line which was supposed to get us there straight away had been closed for construction on that day, but another line was supposed to be used instead. After arriving there we found out, this line as well was closed for work and it suggested we use the previous one instead. Then we found out that a bus line was supposed to get us there, but we had to find out, indeed it has a final stop exactly at the castle we wanted to reach, but only later in the evening or early in the morning. We then had a coffee at McDonalds in order to get on a wireless network which could tell us what to do instead, and it provided us with a solution: We had to take a tram line up to its final stop and then transfer to another bus. And we finally arrived at Maisons-Lafitte, but over an hour later than originally planned. Now I decided to watch the state rooms. These were illuminated partly in green and red to test out the lightning for a private event which was planned to take place later that evening. While I managed to see all the state rooms, Eric decided to see the private rooms instead. Unfortunately we arrived that late, that we each didn't manage to see all the rooms. But what I saw was very nice, so not a wasted opportunity. Who knows, if I will come there again at some point (at least not until now in 2020). And then it was time to visit the observatory on the roof terrace of Tour Montparnasse. Still a very superb view, but instead of offering an unobstructed view of Paris, now large life-size glass windows had been installed, on some places with tiny gaps to take photos, clearly not improving the visitor experience. Still the view of Paris, particularly of the Eiffel tower is and remains magnificent (but a bit more expensive than walking up the second floor of the Eiffel Tower, but a better view since you see the Eiffel Tower). On our way back to the hotel Eric and I stopped for a short while by Notre Dame, took a view pictures of the cathedral, the river, as well as the Hotel de Ville by night and then had food.\\

Paris: Banque de France (Hotel de Toulouse)***, Tour Montparnasse***\\
Malmaison: Chateau**\\
Maisons-Lafitte: Chateau**\\

November 10:\\
Today we all planned to have an excursion day doing nothing in Paris downtown. The first stop was the chateau of Champs-sur-Marne. A wonderful small Baroque palace with several wonderful nice rococo salons, like the Chinese salon, it used to be the French guest house between 1959-1974. The chateau is surrounded by a French style park. After a short lunch we took the RER to Meudon, and then a bus to the large estate of Vaux-le-Vicomte. The Baroque chateau was built for Nicolas Fouquet, the superindent of finances during the times of Louis XIV. The decoration of the halls were done by Charles Le Brun, the park was designed by Andre le Notre, both later worked on the construction and planning for the palace of Versailles. Regarding the castle itself, particularly the state rooms on the lower floor are magnificent with the state bedroom of the King's apartment and the game cabinet. While we were there a chocolate festival was taking place, the palace was quite crowded, but the chocolate was very tasty. The park is quite extensive. It is the oldest Baroque garden in France and its layout was later used as basis for the gardens of Versailles with fountains, sculptures, vases, and a geometric parterre. The bus arrived pretty late and we had to ran to catch the train, Chris C fell, but besides a few bruises nothing drastic happened. The four of us just made the train a few seconds before it took off. Chris L had a long night out with his friends, while the three of us just had a dinner followed by a night cap.\\

Champs-sur-Marne: Chateau**\\
Vaux-le-Vicomte: Chateau***\\

November 11:\\
On our last day we stayed closed to the city centre. We started in St Denis with a visit of the basilica, enjoying once again the Royal tombs and the royal crypt. I am always impressed by the tombs of Louis XII. and Francois I, which show am impressive detail and skills of French Renaissance artists. The later tombs of the Bourbon kings in the crypt are in fact less sumptuous. The next highlight was a visit of the Opera Garnier, I am always happy to come back to what I consider the most beautiful theatre/opera/concert building I have been to so far. After the obligatorily stop by the Conciergerie, we were all impressed the stained glass of Sainte Chapelle (the renovation was still ongoing), Chris C and Eric bought little pieces of stained glass as memory and presents. And last but not least we stopped once again by Notre-Dame cathedral for a while before getting dinner, this time I opted for duck and fries. Then we had a photo stop by the Arc de Triomphe before getting a couple of snacks and drink for our 3 h long train ride back to Geneva.\\

St Denis: Basilica***\\
Paris: Opera Garnier***, Sainte-Chapelle***, Conciergerie*, Notre-Dame***

\section{November 30: Prangins}
\label{2013:Prangins}

Chateau de Prangins**

\section{December 27: Strassbourg}
\label{2013Strassbourg}

M\"unster***, Palais Rohan***
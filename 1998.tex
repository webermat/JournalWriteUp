\chapter{Year 1998}
\label{1998}

\section{April 17--April 18: Ulm}
\label{1998:Ulm}

%17 Ulm Rundgang
%18 Blaubeuren, Wiblingen

Ulm: M\"unster*****, Kornhaus*, Rathaus*, Schw\"orhaus*, Schiefes Haus*, Reichenauer Hof*, Einsteinbrunnen*, St Jakob*, Garnerkirche**
Blaubeuren: Blautopf***, Abtei****, Wiblingen: Klosterkirche****, Klosterbibliothek*****, Ulm M\"unsterturm****

\section{August 22--August 30: Paris}
\label{1998:Paris}

The yearly city trip of this year heads for Paris. Since Air France offers particularly cheap tickets for flights within France, we flew out of Basel-Mulhouse, by the way the one and only flight to Paris with the destination of Paris itself. The RER from Charles-de-Gaulle just stops at the Garde du Nord, where we had our hotel.\\

August 22: The Tour d'Eiffel:\\
Where would you go on your first day in Paris. My parents decided it should be the Eiffel Tower. Instead of taking the elevator up we took the stairs. These offer more detailed views of the structure of the pilons, it is a lot cheaper, and even faster during summer, when the queues for the elevators are quite sizeable. We enjoyed the views from the first two floors. The third floor can be only visited using an elevator.\\

Paris: Eiffel Tower*****\\

August 23: The Louvre\\
The Louvre: one of the oldest museums of the world, and one of the largest if not the largest of the world. Not convinced yet -- free for everyone up to and including age 18, even free up to and including age 25 should you be a resident or citizen of the European Union (Switzerland seems to work too). We first started with a visit by Mona Lisa, back then not that many people gathered around the painting, it was also behind a secured glass which was tinted a bit on the purple side. I was not impressed. On my later visit the presentation was largely improved, at the expense of more and more visitors heading for the painting, but still it leaves me a bit underwhelmed. I heard in 2019 the outfit of the room was once again changed though. I like old paintings, but after about an hour I get usually bored a bit by those. But the Louvre has much more to offer than just old paintings: the sculptures are amazing, particularly the ancient statues like the Venus de Milo, but also more classical pieces, like Michelangelo's slaves or Canova's Psyche and Amor. And all the beautifully decorated rooms. A couple even with the original ceilings, like the former Royal Bedchamber or the presence chamber of the queens.And as pinnacle of Baroque decorations -- the gallery of Apollo, one of the predecessors of the Hall of Mirrors in Versailles. Other ceilings have been remodelled later, but still with impressive carvings and frescoes, some by none other than Delacroix -- thus really amazing as well. During the second empire, a wing of the Louvre was used as meeting room for the Ministers, these so-called Napoleon III apartments with their gilded walls never fail to impress me. A couple of pieces from the former royal palace of the Tuileries are on display, notable the bedchamber and the throne. The British Museum has the Rosetta Stone, the Louvre has the Codex of Hammurabi, the eldest surviving scripture we know off. Should you care about ancient art, the exhibition of the old Assyrian palace walls are amazing, or the big top of a pillar in the audience hall of the imperial Persian palace of Persepolis. Naturally Egyptian mummies, statues and sculptures can be seen as well. Considering that I got into the museum for free I spend a couple of more afternoons and evenings in the museum in the next couple of days.\\

Paris: The Louvre*****\\

August 24: Ile de Cite:\\
The Ile de Cite is the larger of the two islands of old town Paris, the old centre of power with the remains of the old medieval royal palace -- Conciergerie and Sainte Chapelle, as well as the secular centre piece of Paris -- the Notre Dame de Paris. The guard hall of the former palace is one of the largest conserved gothic halls in France, and impressive considering that on top of it used to be the actual Great Hall of the Palace. A couple of windows of St Chapelle had been in renovation, but what could be seen was still very impressive. Being an early gothic cathedral I remember I thought that Notre Dame was quite dark for a gothic church, which I had come to know as particularly bright and illuminated. The Rose windows of Notre Dame have slighter darker colours than most other cathedrals I have been at as well. Oh and I remember the that all of Notre Dames photos got out in shiny red -- unfortunately there was no photoshop available to save these negatives and consequently analogue photos, the scanned versions well those are photoshopped to far better quality nowadays.\\ 

Paris: Notre Dame*****,Sainte Chapelle*****, Conciergerie***\\

August 25: Versailles:\\
Three former royal and imperial palaces are close to Paris: Versailles, Fontainebleau, and Compiegne. Clearly the most popular out of those is Versailles. Nowadays crowded to the maximum, particularly on weekends, when the fountains are running for about 2 hours during the day, come early or get stuck. Back in 1998 things were peaceful, the chateau was popular, but on a reasonable scale. Having read about the palace I was convinced I would love it, and indeed it was love at first sight, particularly the Hall of Mirrors and the chapel. The gardens are so large, that one needs more than a day to see all of it, I have never walked along the whole Grand Canal myself for example, but the plan shows only large forests once you cross the perpendicular canal. On the side is the private garden with the Grand Trianon (also with its own little Hall of Mirrors) and the private castle of Queen Marie Antoinette, the Petit Trianon (indeed far less sumptuous than the other two palaces). For those who don't feel like walking along the cobble stone, golf cards and a little train are available for your comfort.\\

Versailles: Chateau with both Trianons*****\\

August 26: St Denis:\\
Just a couple of weeks earlier France won the FIFA world champion title against Brazil, and this in the Stade de France which is close to Paris in the suburb of St Denis: Clearly I was excited to see the inside of a real large stadium for the first time. Impressive but maybe not the entrance fee, but none of the stadiums are particularly inexpensive. Anyway in St Denis itself is the basilica of St Denis, nowadays even a cathedral. So what's so special about this church: Besides the apparent collapsed left tower the basilica houses the tombs of most Kings and Queens in France, and the choir is the birthplace of gothic art. The stained glass is very impressive, but the tombs are magnificent, the only church I've been to which can compete in this respect is Westminster Abbey. Considering the French tombs are typically larger and more decorated I would opt to see those (particularly the ban on photography at Westminster Abbey adds one more icing on the cake). As if we didn't see enough on this day yet we continued with the Opera Garnier. One gigantic neo-Baroque theatre with modern elements like Marc Chagall's large ceiling fresco in the amphitheatre.\\

St Denis: Basilica and Royal Tombs*****,Stade de France**\\
Paris: Opera Garnier*****\\

August 27: The Pantheon:\\
On this rather quiet day we saw the Pantheon, originally a church build in the classical style during the reign of Louis XVI., the church was transformed into the French Pantheon holding the remains of the most important French people, among those e.g. Victor Hugo. The building itself was in a quite desperate state, a lot of ceilings were covered hidden behind nets due to falling rocks, scaffolding was all over the walls as well, clearly in dire need for renovation. Afterwards we spent some time in the vast park of the Jardin Luxembourg.\\

Paris: Pantheon****\\

August 28: Champs-Elysees\\
I never got the craze for fancy boulevards turned into shopping avenues. The Place de la Concorde is a nice superb square, also the church of Madelaine is just next to it. A bit on the side is the Palais d'Elysee. Back then you were at least allowed to view the gate and the courtyard. On normal days nowadays you even have to stick to the other side of the road. The only days when normal people like us can see the interior of the palace is the heritage weekend of the France (typically the third weekend in September). Since so many people want to see the palace on those two days the tickets are nowadays limited. Anyways back to the Champs-Elysees, the facades are nice, but otherwise almost all buildings are nothing else but shops (or McDonalds etc etc). The Arc de Triomphe is OK, but not really an absolute must either. SThe afternoon we spent by the quite interesting Parc des Buttes Chaumont.\\

Paris: Arc de Triomphe***, Place de la Concorde*****, Champs-Elysees**, Parc des Buttes Chaumont****\\

August 29: Montmartre \& Dome d'Invalides:\\
I will never understand what people see in going to Montmartre. I agree the Sacre-Couer is interesting from the outside, but an absolute plain uninteresting insides (maybe that's why photography is not allowed inside). Not to forget the zillions of street vendors hunting for possible customers around the stairs leading up to the mountain top. The views are nothing to get crazy about either. Anyways on to the next stop, the Galleries Lafayette is a giant shopping mall as well, but instead of others in a really old building, thus I go there to for looking at the building rather than shopping, don't forget to get up the roof terrace with an interesting view of the nearby Opera Garnier and the Eiffel Tower. And via the Place Vendome on to the Army Museum, where they exhibit all generations of French tanks (not that they had been leading to such an amazing success in most occasions). The centre piece of the Invalides complex is the Dome d'Invalides with Napoleon I's tomb, once again another occasion for a red tilted photograph.\\

Pais: Montmartre with Sacre-Coeur**, Galleries Lafayette***, Dome d'Invalides****\\

August 30: The last day:\\ %Park (Porte Ludovico Magno
Not that much to do on this very last day: we walked past Centre Pompidou, ate at a Pizza Hut (first time on that Paris trip), but then all the way back to the airport and home it goes. And then it took roughly 12 years for me to return, but more about that later on.


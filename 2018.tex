 \chapter{Year 2018}
\label{2018}

\section{China: December 29, 2017-January 21, 2018}
\label{2018:China}

January 1: Shanghai without the Smog:\\
China is infamous for its Smog, so I was very happy after two kind off rainy and very smoggy days to finally see blue skies. And what a nice view it was. Since we did celebrate into New Year the day before (not Chinese New Year, thus no fireworks) we got up late, had Dumplings by Nanjing Road and then walked along the Bund once more. This time we were able to see the top of the Shanghai Tower - really an iconic skyline. Then we went to Shanghai station, where I obtained (with lots of help from Siyi) all my reserved train tickets. The ticket counters are situated in a separate building, and it was all huge. Same for Shanghai train station, there is a check-in where everything has to go through security checks as well. Since we booked so late, we didn't get on the ultra high speed train to Kunshan/Suzhou but we had to use the old high-speed trains. And we had to stand, since no seats were free anymore. Siyi's mum was displeased to pick her up at the old Kunshan station, thus she had to take the bus. I took the metro and walked back to my hotel -- an absolutely amazing hotel that is, turned out to serve the best food I ever had in a hotel so far for the next couple of days. Only then I realised that all metro maps I used from Suzhou were outdated, and the newly opened metro line stopped a mere 50 m from my hotel (could have saved me that 1 km walk with all my luggage). I stayed in the middle of the city, so I went along the Guanquian Street and got some grilled meat and sugar-coated apples and hot tea, and then made it to the Xuanmiao Temple before calling it a day.\\

Shanghai: The Bund****, Suzhou: Xuanmiao Temple**\\

January 2: Hangzhou: Westlake:\\
I managed successfully to get on the correct ultra-high speed train, and since stations are announced in English as well, this went really smooth, coffee and sandwiches were served on the train as well, so I was ready to go (after another metro ride). Unfortunately the day was pretty foggy, so not everything looked as pretty around the West Lake as maybe in summer. I first climbed the little hill with the Baoshi Pagode and a rock garden with many little caves and ancient Buddha statues. There are many little bridges connecting little islands, many boats which operate between the islands on the lake, a nice museum about the Zhejiang province. Then I walked through the neighbouring Zongshan Park, only a few metres away from the temple and the tomb of General Yue Fei. Another boat-ride to get to the very beautiful island of the three lakes (peaceful too, the whole lake side was pretty deserted). And then the final stop: the reconstructed Leifeng Pagode, full of fantastic wood carvings, gold plated sculptures, a ceiling with 1000 little Buddha statues, really nicely done.\\

Hangzhou: West Lake*****, Baoshi Pagode and Rock Garden****, Zongshan Park****, Temple of General Yue Fei***, Zhejiang Provincial Museum***, Island of three lakes*****, Leifeng Pagode*****\\

January 3: Suzhou and meeting the parents:\\
This time it was a very rainy and cold day again. Unfortunate for visiting the famous classical gardens in Suzhou. The largest garden is the Humble Administrator's Garden with several little pavilions, a large lake with several little islands. The Lion Grove Garden has a nice little lake in its centre, with lots of rocks and little houses around it, similarly the Yipu garden, whereas the Mountain villa is dominated by an aritificial rocky mountain. Suzhou is located by the Grand Canal, with little side canals crossing the town in a fine grid. On my way I also went into two pretty modern temples. Then I was picked up by Siyi, her parents, and a family friend and we went to lunch, sitting around a very large round table, which could be easily rotated, to get access to any of the delicious food items - from mush rooms, pork belly, beef, as well as trout and eel, and many more items. Afterwards we went to the water town of Luzhi, with many canals, little market stands, bridges and private residences. In between we listened to a choir performing on the town stage. Afterwards we got a private tour through the textile factory of the family friend. I had dinner with other young family friends, Zhicheng (Siyi's cousin), Siyi and a young teenager, seems all of us either studied or worked for a US university at some point in our lives. The food was Western-Chinese fusion, with the local culinary highlight -- the Chinese mitten crab. After I was told how to actually eat a crab, having never had one before, I really enjoyed it -- very tasty. Nowadays this crab might be found in Europe too, considered as an invasive species. I can only recommend to well try to reduce its numbers by just eating it. Anyways after that I had lots of tea with Siyi and her parents, before heating up in a warm big bed. Since Siyi's parents realized how much I enjoyed the local tea, they offered me some bags as presents, which I gladly accepted.\\

Suzhou: Humble Administrator's Garden*****, Lion Grove Garden*****, Mountain Villa with Embracing Beauty****, Garden of Cultivation (Yipu Garden)*****, Chenghuan Temple**, Xuanmiao Temple***\\
Luzhi Town: Canals*****, Jiangnan Park***, Xiaozai Residence**, Residence of Shen**\\
Kunshan: Bacheng-Yangcheng Lake District***\\

January 4: Nanjing:
After stopping at a noodle restaurant - according to both Eric and Siyi one of the best they have been to -- it was time for Siyi to say goodbye to her mum, and she, Zhicheng, and I took the high speed train to Nanjing. It had snowed continuously throughout the night. I was told that much snow is very unusual for the region, in fact if was the largest amount of snow Nanjing had seen in a decade. Nanjing used to be one ancient capital of China, in the early 20th century it was the main political centre of the Republic of China and the headquarters of the Kuomintang. While Siyi went on to give a seminar talk about her work in Astronomy I went to the Chaotian palace, which houses the city museum of Nanjing by now. This was my first large palace in China - very impressive, everything was covered in snow too and looked very idyllic. Afterwards I went to the presidential palace - which was the seat of the government until the late 40s. If you plan to go inside, then don't forget your passports, otherwise you won't be allowed inside. It was very interesting to see how the Chinese government was fascinated by Western architecture at the begin of the 20th century, modelling most houses after European early 20th century architecture (not what I as European would consider particularly beautiful and elegant). The gardens of the Chinese presidents were very nice and beautiful, particularly in the snow fall. Part of the palace are the former quarters (and the throne hall) of the emperor of the short-lived heavenly Taiping empire. The throne was modelled after the imperial halls in the Forbidden City in Beijing. Afterwards I checked into the hotel. There was a bit of confusion as I had a room booked for two, but at that point I was by myself and I had a bit of trouble to make them understand the second person would show up at a later stage. I checked where we would meet for dinner, this time I also checked which floor and which building of the shopping centre I should go to. I took a picture of the name, that way I could recognise the patterns and make sure to show up at the right place. And once it was time to meet for dinner I did find the restaurant in time --- only to find out I was the only one to arrive. So I know I am at the correct place, I also wrote to Siyi to confirm this is indeed the place where I should be. Unfortunately the traffic was horrible due to the unusual level of snow, and everybody was projected to be at least over half an hour late. Unfortunately nobody at the restaurant knew any word of English, neither do I know to speak Mandarin or any of the local languages. It was a bit awkward, since I could not make them understand that I know I am right where i should be, but I still want to wait for my friends to arrive before ordering. Anyways 45 minutes later everybody else showed up, Siyi, her colleagues from Nanjing, Zhicheng, a professor from the US with family ties to China and his family. All in all a group of over 10 people who were very friendly and eager to hear, what I though so far about my time in China. The food was very good too, the first time I tried pigeon (tastes more like chicken than Ostrich or turkey). It was a fun evening and then it came to go back to the hotel. Zhicheng stayed with me and Siyi tried to check in - unfortunately they couldn't find her reservation thus we first went to a wine restaurant to have some local wine. Afterwards she could check in without any issues. What Siyi forgot to mention (and share) until the next morning: she was allowed to have whatever was in the minibar in her room for free, to make up for the delayed check in. She did make use of it, but unfortunately alone only. Anyways since Zhicheng studies physics too, he wondered what my specification was, and then he asked me if I ever had heard of CERN. And yes I obviously did (having worked there at that point by over 10 years).\\

Nanjing: Chatoian Palace (Nanjing City Museum)****, Presidential Palace in Nanjing*****\\

January 5: Nanjing:\\
And we started out with a local speciality -- duck blood soup, pretty tasty in fact. And then we three made our way to the purple mountain - on this day rather a white snowy mountain. We got our tickets and then we made out way to the imperial tomb of Xiaoling, after getting a hot coffee. The tomb starts with a giant gate, followed by an alley through guards - mythical creatures, e.g. unicorns, as well as soldiers, camels, elephants, and lions. After two more halls the actual tomb appeared - a snow-covered citadel with a wide ramp moving through the inside up to a temple like hall. I was impressed. Closeby is the tomb of Dr Sun Yat-sen, the father of the republic of China. The symbolism of the republic and the sun of the Kuomintang is clearly visible, still referencing Chinese classic buildings. Our last stop on the purple mountain was the Meilling Villa, a present for the first lady from the last president of the Chinese republic who had power over the mainland before fleeing onto the island of Taiwan. The Villa combines western and Chinese elements in a pretty harmonic way. At this point it was I had to say farewell to Siyi and Zhicheng, up to now (2020) the last time we've seen each other. I then made my way to the Jiming Temple, a pretty temple close to the city walls. So close that I decided to actually climb the Wall, and walk for about an hour. It is after the Xi'an city wall the best conserved city fortification in all of China. I looked up what was listed on the metro map as additional sight, and the city wall fortress of Zhonghuamen stuck out, so I took the metro to the closest station and then walked 20 minutes back, passing by the modern day replica of the famous Nanjing porcelain tower. \\

Nanjing: Xiaoling Mausoleum*****, Dr Sun Yat-sen Mausoleum****, Music Hall***, Meiling Villa****, Jiming Temple****, City Wall with Zhonghuamen*****\\

January 6: Nanjing\\
I spent the morning working on my presentation, running the latest version of the reconstruction algorithms. After 3 h of work -- clearly the connection was not the best -- I decided to visit the Nanjing Museum, with pieces of art, sculptures, Terracotta statues, silk clothing, bronze pots etc from prehistoric times up to the 20th century. In another wing contemporary paintings from Chinese artists were exhibited, as well as old steam trains and furniture from an old tea house. Only gates remain from the former imperial Ming palace. And then it was time to take the bullet train to Beijing. Due to the unexpected high levels of snow most trains were delayed by over an hour, just like mine. Once I arrived in Beijing I contacted my old co-worker Selena in order to organise where we should meet up the next day, and what we would try to see.\\

Nanjing: Nanjing Museum****, Ming Palace (Forbidden City)**\\

Up to now this has been the last time I saw Siyi in person, after three trips with overnight stays, and one more day trip.\\

January 7: Beijing\\
Selena suggested I get up early in the morning to watch the flag rising ceremony and so I did. A couple of snacks and coffee later I met up with Selena and her PhD student. We had to hand in all our baggage before going to Mao Zedongs mummified body inside a typical socialist style tomb (reminded me of Lenin's mausoleum). Then we spend two hours in the national museum of China, which houses a variety of art, from modern wood carving to african art, presents from Chinese state guests, as well as artefacts from the 2nd millennium before christ to the modern ara. The next item was the former imperial palace, the forbidden city. Selena bought three tickets for us online, technically those are then uploaded to the national ID though, my passport number was given in the booking, but naturally the ticket couldn't be uploaded. We were told that I could get a paper ticket (though not really indicated whereabouts), but seems they were also fine with me just showing my passport. The palace is gigantic, the main halls follow a rigid scheme, with many carved thrones, gilded halls, little bridges and terrasses. There are multiple courtyards each with their own private palaces. We also booked a visit to the treasury. One of their most valued pieces are western clocks, just like in European palaces Chinese porcelain was appreciated. After we enjoyed a bit longer in the imperial gardens we walked along the walls and moats, and then took the bus to get hot-pot in one of these big shopping streets. It was fantastic food and the personnel was very eager to get photos of me and my friends (and some special dessert for first customers). I was told that they try to make this dish for all new customers, including locals. \\

Beijing: Mao Zedong Mausoleum****, Chinese National Museum****, Forbidden City*****\\

January 8: Badaling\\
I informed myself previously what would be the easiest way to see the Great Wall starting from downtown Beijing. Most sites suggest to take buses, since these are far more frequent, or to use some local travel company to get there. I decided to use the train instead. It seemed pretty easy to do and the train times starting in 2018 had been posted two weeks before. The construction of a preliminary train station had just been finished (the new northern Beijing train station is completely remodelled for the winter olympics 2022). On the train I bought some snacks and water, and about 90 minutes later we arrived in Badaling (final stop, but all stops in between were announced in English as well). Now just a 20 min walk later, the Great Wall of Badaling opens up. There are two sections, I decided to walk over both. You might have seen photos of the Great Wall completely packed with people (typically those are taken over the Chinese New Year weekend so, thus not that representative). But seems in January with chilling wind and -14 C almost nobody wanted to walk along. I had multiple layers of closing on me so it was acceptable, and indeed at times I had the wall almost to myself. So I enjoyed it a lot, and then on the way back I got some dumplings at a local restaurant, where nobody spoke any English, but there is hardly anything you can do wrong with getting dumplings. And then back to Beijing again, where I decided to visit the Yonghe temple (also known as Lama temple). In the Great Hall of Harmony and Peace is the gigantic 18 m tall Maitreya Buddha statue on a 8 m pedestal, which is really breathtaking. Then I had dinner by the gates and towers of Zhengyangmen.\\   

Badaling: The Great Wall*****\\
Beijing: Yonghe Temple*****, Zhengyangmen****\\

January 9: Beijing\\
I started the morning enjoying the sunrise at the temple of heaven, free of any person. I was confused by the fact, that one has to leave the enclosure of the first hall, then enter another enclosure with the second prayer hall, and then one has to go back to the first enclosure. Anyway the whole temple complex includes several little palaces, prayer halls, kitchens, a wide park, meteors, so lots of walking (and then I had a short lunch snack before making my almost 2 h trip to the Summer Palace of Beijing. Getting in through the north gate, the first district is the Suzhou road, a re-enacting of of the grand canal by Suzhou. The canal was completely frozen and hardly any of the shops along the road were open. Then I reached a citadel which covers almost the full mountain leading up to a wooden large Buddha statue, with article rocks, grottoes, little pavilions surrounding the main hall. On the other side of the main hall towers a four floor pagoda, overlooking the lake of the summer palace. There are once again many little halls and palaces, as well as monasteries on the complex, a marble boat is sitting on the lake. Many pieces of art can be found in the Wenchang gallery. Some people were brave enough to go on the ice of the lake. I took instead the bridge over to the island. Once again there was an icy wind blowing over the lake, and it was too chilling to be comfortable. Since the Summer Palace is quite far away from the city centre, it can be combined with a visit of the Olympic park of the 2008 Summer Olympics. By now the night illumination was switched on. The stadium is very impressive, the wind was too strong for me to try the roof-top walk, but the restaurant in the stadium was nice.\

Beijing: Temple of Heaven*****, Summer Palace*****, Olympic Stadium*****\\

January 10:\\
Since I managed to see everything I really wanted to see in Beijing, I tried to see some more unusual sight, starting with the Bell and Drum Tower. Walking through the Hutongs (kind of the remains of the old city districts) I passed by the lakes which marked the beginnings of the Grand Canal, which I found out only a couple of months later. The Prince Gong House had a nice garden with the usual palace buildings. I was positively surprised about the Beihai Park. The nine-dragon-screen was praised everywhere, and it is very nice indeed, but far more impressive is an artificial mountain with thousands of statues and on top a golden buddha statue. The white Pagoda dominates the Jade Flower Island in the middle of the lake. Then i walked over to the Jingshan Park, with several pavilions, the main pavilion is just opposite of the forbidden city and offers a full view of the whole imperial palace. And then I got some early sleep for my next day.\\

Beijing: Bell \& Drum Tower****, Hutongs \& Grand Canal****, Prince Gong House****, Beihai Park*****, Jingshan Park****\\

January 11:\\
Getting up very early to reach Beijing West station, one of the largest stations of the country. I had quite a large breakfast there before taking the bullet train to Luoyang. Once I arrived at the train station I quickly found the bus stop, where two tourist buses were supposed to leave for the Longmen Grottoes (going every 20 minutes at least according to the English schedule, even in winter). After 30 minutes I started to get a bit nervous, taxi drivers around me had there apps talking English to me, telling me the next bus would come only in 3 h. I decided to wait a bit more, another 30 mins later, I was too annoyed and started walking. After all, the grottoes are only 3.5 km away from the train station. Once I reached the next large road (after a 10 min walk), there were plenty of buses going to the grottoes. The grottoes are really worth a trip from Beijing. Hundreds of Buddha statues carved out from caves, some even still painted in several colours. The western grottoes are far more interesting in my point of view, with many more surviving statues. One gets to each of the caves over several flights of ladders. Crossing over the Yi river one reaches the eastern grottoes (with a little museum). At the end of the grottoes is the Xiangshan temple, one of the few temples in China where photography is allowed. A villa of a former Kuomintang Governer is closeby as well, a bit boring, since it is built in Western fashion. Behind the temple is the cemetery and the tomb of the general Baijuyis. And once I arrived at the train station, I even managed to exchange my reserved ticket to two trains earlier. In Xi'an I found out, that I could not yet book the tourist bus from the Xi'an North station to the Terracotta army, but would need to come back the next days. I got almost lost on my way to my Ibis hotel, which had just opened a couple of weeks earlier (the lobby was still being furbished, also the clerk didn't know any English). I also had a huge room, with two sofas, a huge flat screen, free coffee and tea, I was impressed. Anyways I made my way to the city centre, where I walked along the Drum and Bell Towers, having the local dish of ..., and then I admired the Yongningmen gate and the city wall with its canals, and back to the Ibis hotel, where I clarified that the bus from the main bus station would be the quickest way to get to the Mausoleum (which would be the last stop too).\\

Luoyang: Longmen Grottoes*****, Xiangshan Temple****, Baijuyis Cemetery****\\
Xi'an: City Wall*****, Drum \& Bell Tower****\\

January 12:\\
Checking out really early I made my way to the main bus station. As people described online there are a lot of handlers trying to get you on their bus, claiming the public one would take ``too long'', but I was prepared for that and got on the correct bus. Once I was at the Mausoleum of Emperor Qing Shihuang I got my coffee, since I had to wait for another 15 minutes before the cash desks opened. I was the first one to arrive at the pits. It is really interesting to see how archaeologists still excavate, catalogue and clean remains they find there. It still seems many years to come before this project will get to completion. I enjoyed the army, but I agree that people should know what to expect. Particularly pit1 is gigantic, but still the soldiers are of similar type (though each with distinct faces), but I understand why people might ask themselves if it is worth a two day detour . I definitely think it is worth it, particularly if you stop in between in Luoyang or if you see the city of Xi'an as well. The museum is a bit of a let down, but the surrounding park is nice especially in a wintery scenery. Then i got back to the city again, walking around some local parks and temples, before going to the Dac'ien temple with the giant wild goose pagoda, one of two large pagodas in the city. The temple halls are full of woodcarvings, some of them plated in gold, and many golden Buddha statues. The pagoda has seven floors to climb. After walking along the city walls a bit I decided to stroll through the bazaar. Xi'an has quite a large century year old Muslim population, so I also went to the mosques. Considerably different than all the mosques in the mediterranean area, the design is rather reminiscent of Asian temples the prayer hall is of limits to non Muslims. And then i got back to the train station for my 4 1/2 h train ride back to Beijing.

Xi'an: Mausoleum of Emperor Qing Shihuang (Terracotta Army)*****, Daci'en Temple with Giant Wild Goose Pagoda*****, Mosque****, City Wall*****\\

January 13:\\
The Confucius temple and the Imperial Academy or next to each others, just a couple of hundred metres away from the Yonghe temple. Quite cute, but nothing which one has to have seen, although the forrest of in-scripted rocks and pillars is quite something. Then I decided to pass by the cathedral of Beijing where a wedding just ended, before getting to the QuanJuDe restaurant by the Qianmen street, where I had a really delicious half of a Beijing duck with pancakes and soup. I decided rather to finish early before having my flight over to Hong Kong the next day.\\

Beijing: Confucius Temple and Imperial Academy***, Cathedral**, Zhengyangmen****\\

January 14:\\
Getting to the airport lasted quite a bit, the plane was late by over an hour, but it was nice to see all these Air China Boeing 747s taking off and landing. Flying over several mountains, snowed in river planes I finally made it to Hong Kong International airport and got on my bus which brought me to downtown and my hotel (room without a window, well the main room, the bathroom did have a window though, so could get some fresh air in). I went over to the sea to enjoy the Harbour laser show. Unfortunately the lakefront walk had been closed down for construction, thus I had to stand at a non ideal spot. The laser show was a bit of a let down, OK a bit of flashing lasers, but very faint, not synchronous with the music (which was not that interesting either). Anyways the panoramic is fancy enough with all the lights, but still I had expected more.\

January 15:\\
After giving a rehearsal of my CLIC overview talk, I made it over to Hong Kong island and visited the museum of the Hong Kong currency in One International Finance Center. Since this museum is for free and situated quite high up the skyscraper you get a nice panoramic view of the island (and a bit of Kowloon). Then I walked through downtown a bit, and then along the shore line and by the piers. Then I took the metro back over to the peninsula and then visited the International Commerce Centre Observatory. Magnificent views of the sunset and the night panoramas of the Kowloon area as well as the harbour and the straight and Hong Kong Island. It is maybe a bit pricey, but I do enjoy observatories, so it was really fun.\\

Hong Kong: One International Finance Center Observatory****, International Commerce Centre Observatory*****\\

January 16:\\
Having purchased my tickets for the Hong Kong-Macau ferry starting in Kowloon I started walking through the parks over to the Terminal during sunrise, having gotten a window seat was nice for the to roughly one h boat ride to Macau. There I walked over to the old town. Macau had been a Portuguese colony for quite some time. Thus the churches and the cathedral are reminiscent of Portuguese styled baroque churches, though still far shy from the amazingly decorated ones in Coimbra, Porto and Lisbon themselves. The temples and private residences were alright, though I did in fact expect a bit more. I wouldn't call a single building (also none of the parks) in Macau a must see sight. The food I had was very nice (not that I had heard any of the names before, so I just went by their description of the ingredients. Restaurant claimed to be of Cantonese cuisine, and I was the only non south east asian around. Since I was finished far earlier than I thought, but still had quite some time to spend before my ferry would leave, I decided to check out the casinos, which are the reason why most tourists come to Macao. Most impressive among those was the Grand Lisboa. I made my way to the ferry terminal and had a snack break by Fisherman's Wharf, where i was greeted by their version of a ruined Roman theatre. Back in Hong Kong I went to the Botanical gardens, which is also home of a little Zoo with lots of birds, reptiles, monkeys, and turles. I enjoyed the Botanical gardens in dusk and stuck around for some downtown night photos before crossing over to Kowloon for dinner.\\

Macau: Cathedra**l, St Paul's****, Church of Antonius***, Domenican Church***, Sam Kai Vui Kun Temple**, St Lorenz Church**, Madarin's House****, A-Ma Temple**, City Office**, Lou Kau House**, Grand Lisboa***, Fisherman's Wharf**, Camoes Gardens****, Na Tcha Temple**, St Joseph***\\
Hong Kong: Botanical Gardens****, St John's Cathedral***

January 17:\\
The waiting time for the cable car up to Victoria Peak was very short in the morning, unlike in the afternoon and evenings before, when over an hour of waiting time was announced. I didn't bother to visit Madam Tussauds or the observatory of the peak tower, but decided to walk along the hiking paths instead. Indeed the panoramic views are very nice, although it was a bit hazy. Although it was only January it was already very warm and time to put on T-shirt and shorts (compare that to the -15 C in Badaling). Once I was down I walked over to the catholic cathedral, just realising that it was just a few metres away from the exit of the botanical gardens. I was still confused about the many layers of Hong Kong roads, where you can walk along roads on different levels just basically stucked on top of each other. After having a brief lunch in one of these pub type restaurants in downtown I got to the other side of the Peninsula to reach the HKUST Conference lodge. I was informed that unfortunately they were out of single rooms, so they could ``only" offer me a double room with the ocean view. I couldn't believe my luck and gladly accepted. Then I had dinner on the university premise (with dozens of restaurants, mensas, all very very fancy, even beating out what I was used to from Switzerland.

Hong Kong: Victoria Peak***, Catholic Cathedral*\\

January 18 \& 19:\\
The workshop on calorimeters was very interesting with very engaging discussions. It was fun to discuss the content of my two presentations with other experts in the field. In the evenings we had really nice food in local restaurants, also with a bit of tasting of local liquors. I did enjoy all of it.\\

January 20:\\
This was one item my sister recommended me to do. Located on an island close to the airport, I had to take a bus, transfer via two metro lines and then take a cable car over a mountainous island, crossing even a full bay. On the way you have a perfect panoramic view of the Hong Kong airport. It was really fun to see the movement and landings and starts from quite a distance. And then it appeared. A gigantic Tian Tha Buddha statue sitting on top of the hill. I walked up the stairs, walking around and inside the statue and then checking out the closeby Po Lin monastery with the many golden Buddha statue, since the hall was nicknamed hall of thousand buddhas, the number of statues was definitely huge. And then about 2 h to get back and another hour to get to the airport, and then having a long dinner and waiting for the departure of my plane very shortly before midnight on a Lufthansa Boeing 747-800.\\

Hong Kong: Tian Tha Buddha*****, Po Lin Monastery****\\

January 21:\\
Once arriving at Frankfurt airport I transferred back into the Schengen area. My camera was checked for remains of explosives, and tested positive. Naturally I had to have a special interview with the border police, aka where I come from, where I go to, why I go there, and I had to show them photos on the camera display. Seems I was not too suspicious and they let me go. Later I informed myself if that was a very rare occasion, or if that happens at a certain frequency, since you rather want your test to be sensitive than missing out a rel threat. It seems even some hand washing lotions can cause the alarm to be set off. I made it back to Geneva and straight away back to work for my first normal working day of the year.\\

\section{January 28: Lausanne}
\label{Lausanne2018}

Lausanne: Palais de Rumine (Cantonal Museum of Vaud)***, Cathedral****

\section{February 4: Nyon}
\label{Nyon} 

Nyon: Castle**, Roman Museum**, Roman Temple \& Terrace***

\section{March 3--March 4: Toulouse \& Albi}
\label{Toulouse2018}

March 3:\\
Albi: Cathedral*****, Palais de la Berbie****, Collegiale St Salvi***, Hotel Reynes**\\
Toulouse: Carmelite Chapel****, Basilica St-Sernin*****, Cathedral****, Canal du Midi****\\

March 4:\\
Canal du Midi****, Cathedral****, Hoel d'Assezat****, Basilique Notre-Dame de la Daurade**, Capitole*****, Jacobine Convent****, Basilica St-Sernin*****, Notre-Dame du Taur***

\section{March 9--March 11: Amsterdam \& Brussels}
\label{BrusselsAmsterdam2018}

Co-travellers:\\
Mohamed: directly responsible for this trip, but more about that later. Often, before I go on trips or start planning trips, I try to maybe get into contact with locals, sometimes it works out, often I get no reply back but without trying for sure you won't find any success. It has been a year-long dream of me to visit Egypt again, particularly to see more of Cairo or to plan getting inside one of the pyramids, and maybe to see more tombs in Luxor or even get to Alexandria -- well you get the idea. Mohamed works as software engineer in the steel industry, but has visited Germany for an extended time related to his work. And he loves to see places and visits interesting sights as well (being at least trilingual with English and German as additional languages is a big plus too). Thus when i started to talk to him he was quite eager to make it also work in his favour, and he asked me what I would suggest to do after another work related business trip to Western Germany, e.g. going to Amsterdam. Which brings me to the introduction of this trip:\\

How did we end up here this time?\\
Mohamed wanted to see Amsterdam. While Amsterdam has its charm (although a bit more in summer times and not March), I still would always rather visit Belgium and Brussels in particular. Thus I suggested that we could indeed see Amsterdam, but I would suggest another day in Brussels after. Mohamed agreed on that plan and we decided to meet up in Amsterdam, with him directly taking a bus from Germany, while I would get on a bus from Brussels.\\

March 9:\\
A typical late flight our of Geneva with Brussels, so delayed by a couple of minutes but nothing special. This time I stayed at an IBIS by the Place d'Agora, but since I had seen it previously and planned to go there the next day anyway I just decided to sleep early.\\

March 10: Amsterdam:\\
Since Mohamed wanted to take the bus instead of the train, I got to experience my first FlixBus rides. The bus left from Bruxelles Gare du Nord so jumping on an early train from Centraal getting myself some coffee and pastry before waiting for the bus. The bus was more or less on time and a stop by Antwerp Centraal later we were on our way to Amsterdam. The bus did even arrive a bit early, while Mohamed's bus from D\"usseldorf was late by about 25 minutes. We got ourselves day tickets, dropped his luggage by Amsterdam Centraal and off we went to the Royal Palace of Amsterdam. Originally built as city hall in Amsterdam's golden age, the city hall was later converted into a palace during Napoleon's I time. Although the king resides now in the Huis ten Bosch by Den Haag, the palace is still used by the royal family for official receptions and state visits. Typically the palace is open for visits all day long though. Although some rooms have been converted to dining rooms or guest bedrooms, they still carry on their names from the former use as city hall, but the style quite clearly changed to Empire furniture. The central hall, the Burgerzaal, covers three floors, dominated by statues of Atlas and virtues. A large fresco covers the wooden ceiling. Large galleries connect the central hall with the private quarters and rooms, like the throne hall or the Moses Salon where audiences take place nowadays. Walking along Amsterdam's canals and getting some more coffee and a short snack we got ourselves Tickets for the Van Gogh museum. Although it was winter season, we queued for about 20 minutes (tickets for the day were not sold online anymore), and we had to wait 90 minutes for the next free spot. I would recommend you buy tickets for this museum in advance. For how hyped it is, it is actually a tad disappointing. I had seen a couple of paintings already previously on special Van Gogh exhibitions, like the Potato Eaters in Milan. Photography was not allowed due to conservatory reasons. You might now think clearly if they would allow it, all paintings would be clogged with people, but I doubt this would get worse. The most important pieces had been surrounded by guides and their groups anyway, and in all other museums there is typically one Van Gogh painting which everybody wants to get a selfie off, while all other paintings just next to it are unfortunately completely neglected. Anyways I did learn a few details more about his life, although I knew quite a bit about what was told beforehand. So your experience might be different depending on how many of his paintings you saw before, or how many details of his life should be familiar to you. They did have a pretend to be in front of a painting selfie spot which was heavily used to have a copy of a self portrait in the background. Since we had now about an hour to spent we went back to Rijksmuseum which is just a couple of metres away from the Van Gogh museum. The main painting there is Rembrandt's Nightwatch. Still a huge painting even considering that parts had been cut out at some point. We also saw other paintings, period rooms, silver ware, carved statues, porcelain, quite different things which can be found here. After dinner we got on the bus back to Brussels (was about 30 mins late). Back in Brussels by night time, we visited the Galeries Royales St-Hubert and the Grand Place and Mohamed got himself some more food and drink.\\

Amsterdam: Royal Palace*****, Rijksmuseum****, Van-Gogh Museum***\\
Brussels: Grand Place*****\\

March 11:\\
I had bought our tickets for the Stadhuis (city hall) of Brussels, and I became quite nervous if I had explained it properly where and when the tour would start when Mohamed wasn't around five minutes before. He arrived on time anyway and I saw once more the beautiful inside of the building. Only this time photography was allowed, we also saw one more room (the anteroom of the Mayor's office). The first rooms are decorated with lots of tapestries, the later room have paintings on the wall as well as wood carvings. The festival hall is decorated with tapestries which are supposed to look like gothic characters, but since they were woven centuries later the style is rather a mock medieval one. After a short stop by St Nicholas we saw the gothic cathedral, which is one of many European cathedrals in such style with stain glass windows. For the first time I visited the Royal Museum of Fine Arts, concentrating on the old Flemish masters, like the Brughel family or Hieronymus van Bosch (pretty impressive). Walking over to the Palace of Justice we stopped inside the nice Notre-Dame au Sablon. And then we had steak frites and coffee by the Place d'Agora. For me it was then time to get home to Geneva, Mohammed stayed for another day, visiting the city of Namur in the Wallon\\

Brussels: Stadhuis*****, Cathedral****, St Nicholas***, St-Jacques-sur-Coudenberg***, Royal Museum of Fine Arts****, Notre-Dame au Sablon****

\section{March 28--March 31: Iceland}
\label{2018Iceland}

How did we all end up in iceland -- by coincidence. By end of 2011 Rachel mentioned that she would love to see the Northern Lights. So we checked how expensive it would be to get to Norway. And I just wondered if Iceland would be cheaper. Unexpectedly flying into Iceland was a lot cheaper, and considering that we also might want to do something during daylight, we decided that Iceland would be the more intriguing alternative. And it was also the easter weekend.

Co-travellers:\\
Rachel: indirectly responsible for this trip, as always on our trips together, Rachel will evaluate where we should have dinner.\\

Riju: by this point my most experienced co-traveller, it took nothing to convince Riju to tag along.\\

Reyer: secretly viking at heart, Reyer was up as well to find out what it's all about in Iceland.\\

March 28:\\

The flight from Geneva to Iceland takes about 4 hours, followed by a bus ride of about 1 h. The final bus stop was just a couple of metres away from our hotel, so we just dropped our stuff there and continued to explore the town. Iceland is only sparsely populated, Reykjavik is by far the most populous town in the country. As volcanic island geothermal heat is used to produce power, as well as heating up roads and water, unless the water is naturally warm enough. There are hot springs all over. The tab water does have a sulphuric taste to it, not dangerous for health reasons, but still something you first have to get used to. Anyways we walked along the coastline, passing by the sun bark??? monument by Jon Arnason, set up by the coast in 1986. Then we warmed ourselves up in the cafe and the foyer of the Harpa Concert Hall, a very modern hall with an exciting design. After that we had dinner by the harbour. Naturally it was already pretty dark, unfortunately also very cloudy, thus we decided to call it a day, since we had to get up pretty early the next day, when we also booked an excursion to bring us away from any settlement to watch the nightsky.\\

Reykjavik: Harpa Concert Hall****\\

March 29:\\
We were picked up by a small bus, in the end travelling with a small group of about 12 folks on what turned out to be the most beautiful day on the island. We first stopped by the huge waterfall of Seljalandsfoss. Most waterfalls are fed by glaciers covering the volcanic peaks of iceland. This waterfall is special in that sense, that just opposite of it you can walk up a little hill across the waterfall, the hilltop being just about halfway the height of the waterfall. It is also possible to walk behind the waterfall (getting soaked substantially though). A couple of minutes of driving a second large waterfall, the Skogafoss appears, with a slight rainbow hovering of the fall. A couple of ladders bring you up to the begin of the waterfall, and we walked a few hundred metres further along the river, seeing a couple of smaller waterfalls. Then we were brought to the Black Sand Beach. The volcanic remains were also visible in large rock islands just a bit after the beach line. This beach is notoriously dangerous due to waves pulling people inside the water with large power. It is advised to keep clear of these waves. Close to the beach is also a grotto, built up by basalt pillars, similarly to what can be found by the Giant's Causeway. We also had a small snack for lunch by the beach. Then we got to the highlight of this day: the large glacier of Solheimaj\"okull. Originally almost reaching the shore, the tongue of the glacier retreated quite substantially, now ending in a large lake of melting water. Getting equipped with crampons and ice axes we started our hike, quickly crossing a part, where the rock starts to become unstable due to melting permafrost. The glacier is covered substantially in volcanic ash originating from decade long eruptions. We tasted the crystal clear ice water running on the top of the ice, also we could look into crevasses in the distance. The walk last between 90-120 minutes. Then all the way back to Reykjavik, a short dinner, and then on to another excursion getting us to remote areas with almost no light pollution. The sky was not absolutely crystal clear, but anyways we were out of luck, and unfortunately no Northern lights.\\

Seljalandsfoss*****, Skogafoss*****, Vik i Myrdal: Black Sand Beach*****, Solheimaj\"okull*****\\

March 30:\\
The morning started with a thick dense layer of fog. We drove up to the upper plains just a bit above the fog, only to climb down into the dark again, this time into the Hafnarfjlorthur Lava Cave, one of the plenty lava tubes from former eruptions. Great to see the remnants of a tunnel carved out by hot magma. We also saw a couple of icicles and snow at the end of the cave. Back to Reykjavik where we got on a large bus for the golden tour. Starting out in the Thingvellier national park, where the American and the Eurasian plate move apart, opening up the middle Atlantic rift, which basically created the whole island of Iceland. We were given about an hour, to explore the rifts and to go down to the rock, where the ancient parliamentary meetings of the Icelandic vikings, the Thing, took place. Anyways an hour was pretty tight to make all of it, and after all it became a bit longer, waiting for Riju. The next stop was the giant Gullfoss waterfall, where the river drops down via two cascades into a canyon. Unlike close to Reykjavik the plains were still covered in snow. It started to rain, so we fled into the restaurant, getting soup for each of us. After this short lunch snack we did a 10 min drive to the Geysir Geyser basin. This is not a typo, Geysir is an ancient geysir, which named the whole phenomena, but it has been laying dormant for a couple of years by now. The close by Strokkur geysir is very active, erupting for a couple of seconds about each 10 minutes. Not as impressive as Old Faithful, but clearly far more regular. Our tour guide joked that every day hundreds of phones run out of battery trying to get that perfect photo of the eruption. On good Friday almost all shops and restaurants are closed for almost the whole day, and many restaurants open shortly before midnight. We managed to find a place though which opened up late in the evening already, and had some good local cheese and sausages (and beer and met).\\

Hafnarfjlorthur Lava Cave*****, Thingvellir National Park*****, Gullfoss*****, Strokkur Geyser*****\\

March 31:\\
We just had a couple of hours before jumping on the bus to the airport. We used it to see the modern Hallgrims Church, built by mid 20s century, The outside is dominated by the tower, whose decoration gives a nod to the basalt columns of volcanic eruptions. The interior reminded me of a modern interpretation of gothic style. Strolling through the city we also stopped by the catholic cathedral of Christ the King, once again built in gothic revival style. And then it was time to leave, transfer to the airport, where the plane had a delay of about an hour.\\

Reykjavik: Hallgrim's Church****, Christ The King Cathedral**

\section{April 7: Valais}
\label{Valais2018}

Sion: Chateau Tourbillon***, Chateau Valere****, Cathedral***\\
Vernayaz: Pissevache****

\section{April 13-April 15: Denmark}
\label{2018:Denmark}

This trip idea was related to Rachel as well: Rachel decided to get a tattoo depicting an event recorded by the Big European Bubble Chamber. Letting us decide if we would prefer to rather spent a long weekend in Copenhagen or Stuttgart, we voted for Copenhagen. When it came to the room, we had quite a bit of a discussion if we should get two rooms or one room for all four of us, Reyer putting an end to the discussion and deciding it should be one room, but more about that later.\\

April 13:\\
Geneva airport can be quite crowded, but this time it was really horrible. Almost every single flight had a delay of at least 20 minutes. Naturally my flight was delayed too. By the time it was supposed to depart, a flight leaving to Frankfurt just arrived for boarding, making it clear, that it would be tricky to transfer in Munich on time. My flight left with a delay of 50 minutes. Although I tried everything sprinting up and down 5 flight of escalators at Munich airport, with a short Monorail trip between satellite and main terminal, I missed my connection by a couple of minutes. I had to discuss to get on a flight by 7 am, and not at 11 am as was original proposed. Anyways this meant getting up the next day by about 4 am, hopefully without suffering the whole day being deprived of sleep.\\

April 14:\\
Getting on the early morning flight I finally made it to the hotel in Frederiksberg by around 9:30. Seems the room was quite crowded already with two bunk beds within 10 square metres, and the typical shower over the toilet, similarly to what I encountered in Copenhagen the last time around. Anyways we just made it in time for the tour of the city hall by 11 am. I was quite happy to make it this time, having missed out on a tour in 2013 by just 5 minutes. In fact the tour had just started but the guide told us he would wait for us to get tickets. The lady by the ticket counter though told us to wait since while she took over 5 minutes to explain another person how to read the metro map. Anyways we still made it onto the tour, so special thanks to the tour guide. The city hall was nice, many rooms decorated by murals, stuccos or tapestries, all in a style typical for the transition time between the 19th and 20th century. Then we visited the Rosenborg Castle with lots of rooms in Renaissance and Baroque style. The great hall houses two thrones, one made out of whale teeth, tapestries depicting important battles, and silver lions. The decoration of the glass cabinet is the highlight of the castle. In the basement of Rosersberg, the danish crowns, diamonds, diadems, sceptre, and imperial apple are displayed, together with other pieces of art of the royal treasury. It was time for a short snack, having Goat sausage this time, as well as my first try for a vegetarian sausage, which I thought was pretty tasty. The next item was the Royal Palace of Christiansborg. Christiansborg had been the main residence for Danish kings for a couple of centuries. After the Baroque palace burnt down, a classical palace had been erected, but that one burnt largely down by the end of 19th century. The next palace was built in the 20th century in a Neo Baroque revival style, this time only for representation purposes. It also houses the parliament, so both the legislative and executive branch of the country. The residence of the queen moved nowadays to the four palaces of Amalienborg. Parts of the former classical palace still exist, e.g. the royal chapel, as well as parts of the decoration of the Alexander Hall. The tapestries of the Great Hall have been woven in a modern style depicting recent historical events and the current royal couple. Seems Copenhagen was quite busy during this time of the year already, only our third try of a restaurant had an open spot. And then we made it to the amusement park of Tivoli. Unlike in summer no concert were going on, but we had a couple of snacks, and enjoyed the decorations, which are built in the style of a Mosque, or an artificial Matterhorn, lakes, pirate boats etc. Rachel and Reyer had fun on a rollercoaster, afterwards we witnessed the Fountain- \& Light show (still really nice). Then Reyer, RIju and I called it a day, while Rachel decided to stay out for a bit longer. So back to the hotel to distribute this time Riju and I on the upper bunk beds, and Reyer and Rachel to the lower floors.\\

Copenhagen: City Hall*****,  Rosenborg*****, Christiansborg*****, Tivoli*****\\

April 15:\\
After flooding the toilet (impossible to keep it dry, since the fixed shower head is directly over the shower, quite common in Denmark in fact), we wanted to get a longer brunch at one place Rachel found. We had to wait for about half an hour though, which Rachel and Reyer used to go hunting for Pokemons. Well anyways once the place was ready, then we had pancakes or salmon and all varities of coffee. Then we walked over to the central station and took the train to Roskilde. The cathedral of Roskilde houses the tombs of the Danish kings, built in brick gothic style (you find a similar style in other scandinavian countries, as well as in northern Germany). The tombs are in several chapels attached to the main nave, not quite as grandiose as those of French and English kings, but more impressive than e.g. Germany or Sweden, or Spain for that matter. The main altarpiece from 1560 is magnificent and most carved statues are covered in gold leafs. It is also possible to get to the second level, which gives you a perfect view of the tombs in the choir. And then it was time to take the trains back to the airport for myself, while Rachel, Riju \& Reyer had dinner and took another flight home to Geneva.\\

Roskilde: Cathedral*****

\section{April 27-May 1: Paris}
\label{2018:Paris}

Paris is amazing, but it is also useful to use as centre of a northern France trip. The French rail system is clearly focused on Paris, and to get from city to city you typically transfer in Paris anyway, so why not going to Paris right away and then doing train trips from there.\\

Co-travellers:\\
Reyer: After quite a couple of trips, Reyer decided we should do a bit of Paris and France, clearly I was on board. Reyer is one of the few people whose normal walking speed is even a bit faster than mine, clearly helped by his height which exceeds 2 m.\\
Christine: With her love for good restaurants and food France and Paris are always a nice place to visit again. Once Reyer and I told her that our trip is expected include Champagne tasting, she was even more eager to join.\\

Our Paris trip was already derailed by the months long train strike beforehand. We had booked a couple of train rides in order to see the major gothic cathedrals, which Reyer was very eager to see. So we had to reroute a lot of the originally planned itinerary. On the first day we originally planned to see Reims and Amiens, followed by Bourges on the second day. On the third day everything worked out as planned, besides taking the very last train out of Rouen (the one before was cancelled two weeks before). On the last day I wanted to see Chartres and that worked out too, but Reyer and I decided to spent a bit less time in Chartres and instead see the cathedral of Amiens beforehand.\\

April 28: Reims\\
After we found out that our TGV ride to Reims was cancelled due to the strike, we all decided that we should still go there, albeit a bit slower taking the Ouibus. We had a short breakfast and coffee at the Gare de Bercy before the bus ride of about 2 hours. Indeed even no single regional train would take us to the city centre of Reims, thus we got on the tram. Having arrived in old town we admired the many sculptures of the gothic facades and portals of the cathedral. Reims cathedral had been the church where the French kings were coronated, unfortunately it was hit very hard in World War I, and most of the stained glass had been lost forever. The interior of the cathedral is though still a magnificent sight, including the statues around the main portal on the inside of the church. Almost none of the original altars had survived, and the windows were mostly redone, also by renowned artists like Marc Chagall, others are even from the last decade, thus quite a nice experience to witness modern stained glass art.\\
 After a short lunch we took the bus over to the Basilica of St Remi, a large gothic church, where the French crown jewels had been kept for quite some time. From an arts perspective the tomb of St Remi is the clear highlight. And then it was time for our tour of the Champagne House of Taittinger. We were told about the history of the place, situated on a hill side on the remains of an old monastery. Long trails and caves have been put into the chalk ground, well suited to let champagne ripen. And then we got to taste a couple of champagnes, among them the best tasting one I ever had. It was really pricy, thus Reyer, Christine, and I decided to get three bottles, among them the almost 200 EUR expensive bottle and have all three of them later during the year. Since we still had about an hour left before taking the tram out to the train station, we saw the last part of the UNESCO heritage, the Palais du Tau, the former residence of the archbishops of Reinms. Here the tapestries of the Banquet Hall and statues from the cathedral are the things to see. And then once we arrived by the train station we had to sit on the bus for another 90 minutes, before we had a very nice dinner at a small restaurant close to the Place de la Bastille in Paris.\\

Reims: Cathedral*****, Basilica St Remi****, Champagne House of Taittinger*****, Palais du Tau*****\\

April 29: Versailles \& Paris\\
Originally I wanted to get to Bourges on this day, but due to the strike the only alternative would have been 6 h bus rides with Flixbus. Since I didn't want to do that, I wondered what we could do instead. Christine suggested to go to Versailles. Since the RER C train line was affected by the strike as well, we instead opted for a trip by metro and bus, starting about half an hour earlier than using the RER. Naturally we were the first ones in line for the palace. Even just when the doors opened the line was already hundreds of metres long. Operating just two security scanners out of the six available does lead to a speedy operation. Anyways we clearly were also the first ones in the state apartments. Thankfully I made sure that this time my white balance was set up properly unlike in 2015. Unfortunately the Room of Peace as well as the whole Queen's state rooms were in renovation, thus we saw about two thirds of the state apartments, albeit without any crowds. Thus we all got out panoramic shots of the hall of mirrors without anybody else. Christine decided for breakfast we should have pastries and the hot chocolate of Angelina (which we did). By the time we arrived at the exit lines had formed for the toilet, the waiting for the palace was over 1 h and the lines were over a km long (just a bit more than 1 h 30 after the opening of the palace). Next we visited the apartments des Madames (the rooms of the Dauphine were in renovation as well), then we decided to see the fountains in operation. \\
We had to find out that unlike the palace, the park is not for free for people below 25 during the weekends. Thus Reyer had to purchase his ticket, and then we were ready to go. On a normal weekend the fountains are switched on for an hour around lunch time, and again for an hour in the afternoon. During peak season fountains can also be seen illuminated during late evenings. Although I knew how to optimise my paths to see the most impressive fountains, and we started our fountain rush just when they started to operate, even with my walking speed I just made it to see all of the fountains. Clearly without preparation and less quick walking people have a hard time to see both sides of the fountains. A pity when you consider how beautiful both of them are. Clearly if you are not that speedy you need a full day, particularly if you want to see the Trianons too, which we decided to skip this time around. After our fountain rush, we had panini and drinks at one of the garden cafes before we went to the court of honour again going to the other pavilion, where guided tours are starting from.\\
 Only two security scanners were working here as well, but with less people queueing. This was my second time time to see the private apartments of the royal family after 2013. The rooms are smaller, but not any less sumptuous than the state apartments and clearly very impressive as well. A clear highlight is the royal opera house, our guide emphasised that the amphitheatre of Opera Garnier is roughly the same size than this spectacular theatre. \\
 After we decided to skip the Trianons we chose to see a bit of Paris instead, opting for Sainte-Chapelle as the clear highlight of the city. As impressive as always, this was the first time I had seen the upper chapel without scaffolding in a decade. One of the most impressive spaces of French gothic, a must-do when in Paris. The chapel is clearly small but still absolutely breathtaking every single time I step food into it. Unlike Reyer and Christine, I didn't want to miss my chance to see Notre Dame. Shortly before a service, the church was illuminated, and i got some nice night shots of the choir and the chapels. Little did I know that this would be my last visit for what seems to be maybe even for the next 10 years. I met up with Christine and Reyer at the Cafe les Deux Palais, getting myself an Irish coffee (it is great, do it if you have time). And then we had all the beef stew at the cafe xyz.\\

Versailles: Chateau***** (State Apartments*****, Private Apartments*****, Gardens*****)\\
Paris: Sainte-Chapelle*****, Notre Dame*****\\

April 30: Bayeux, Caen \& Rouen\\
After being only the two of us again and having dealt with the two days of trains not running, we finally had our first TGV ride out of Paris. Our destination was Bayeux, the home of the famed tapestry of Bayeux from the 11th century, depicting the conquest of England by the Normans led by William the Conquerer. The tapestry had been housed in the cathedral for the longest time, it is assumed that William's brother, who was the bishop at Bayeux, commissioned the tapestry. The tapestry is an almost 70 m long, 50 cm tall embroidered cloth, housed within a glasses case to keep the temperature constant. The story starts with event surrounding the last years of Edward the Confessor, his death, followed by the reign of Harold II Godwinson as King of England, and the preparation for the Norman invasion and the victory of William the Conqueror in the Battle of Hastings. A funny little detail is a star with a streaming tail, assumed to be Halley's Comet. After a visit of the tapestry we walked over to the cathedral, another nice gothic large church. And then back to the train station, arriving only a short while about 50 mins later in Caen.\\
 By that time it started to rain really heavily. Unfortunately Reyer didn't think of bringing an umbrella on this trip and neither a hat. My jacket had a hat, thus I gave him my umbrella. Due to our height difference, clearly only one of us could use the umbrella. As a result both of us were quite soaked once we arrived by the Abbey of Men. I hadn't had clothes wet like this since my visit of Stonehenge. While the church is large the decoration is more or less standard, but the interesting detail is the tomb of William the conquerer. By the time we got out the rain had slowed down considerably. We walked around the ruined church of St-Etienne-le-Vieux, and proceeded to St-Sauveur another gothic church in flamboyant high gothic style. The castle of Caen is impressive from the outside, since we were short on time, and reviews said the interior would be plain, we walked over to the train station, stopping in the courtyard of the Hotel d'Escovile and the church of St Jean with its unfinished square tower.\\\
  And off by bus to our last stop of the day: Rouen, the capital of Normandy. We shortly walked through the Foyer and Stair of Honour of the city hall, walked around the very impressive church of St-Ouen (unfortunately closed on this day), same for the church of St-Maclou. Our first large stop in Rouen was the large gothic cathedral with a nice baroque high altar, as well as impressive stone-carved tombs in the choir. The facade is considerably broader than usual for French cathedral with uneven towers, you might have seen Claude Monet's painting of it. The large square tower is rather untypical for French cathedrals, this style is more reminiscent of English cathedrals. The flamboyant outside of the high gothic Palais de Justice is one of the most impressive secular medieval buildings I have seen in a while. Then we admired the modern church of St-Jeanne-d'Arc, built on the square where she was burnt, the roof having a shape of a flame. Integrated into the modern church are windows of a former renaissance time church, which was completely destroyed in World War II, while the windows had been put at a safe place. After a short stop in the courtyard of the Hotel de Bourgtheroulde (nowdays a luxury hotel), we had dinner and then took the last train back to Paris.\\

Bayeux: Tapestry of Bayeux*****, Cathedral****\\
Caen: Abbey of Men****, Saint-Sauveur****, Hotel d'Escoville***, Saint Jean**\\
Rouen: Hotel de Ville*, St-Ouen*** (outside), Cathedral*****, St-Maclou** (outside), Palais de Justice*****, Sainte-Jeanne-d'Arc****, Hotel de Bourgtheroulde***\\

May 1: Amiens \& Chartres\\
Reyer and I started the day by dropping our luggage at Gare de Lyon before getting to the Gare du Nord. There our train was delayed by about half an hour, seems no train driver was available, maybe an aftermath of the train strike. Once we arrived in Amiens we walked the couple of minutes to the cathedral. The sculptures of the portals are outstanding, once brightly painted, nowadays the colour faded away almost completely. The original stained glass windows were lost, but the structure is considered the most beautiful high gothic cathedral of the country. The choir screens are very nice, most of the interior altars are from Baroque times. The belfry of Amiens is part of world unesco heritages as well, standalone a bit interesting, but nothing i would consider outstanding. Once we arrived by Gare du Nord we rushed to Gare de l'Est to catch the metro to Montparnasse, since the Gare du Nord metro station was in renovation. Once we arrived in Montparnasse we got a bit confused as which hall our train would leave from, we still got on the train in time just with only a few minutes to spare, but we made the transfer a lot shorter than SCNF or even google would suggest. \\
Chartres is one of the giant early gothic cathedrals in France, it seems far too big for the small town. By now the renovation had been finished, compared to my first visit in 2010 the walls had been cleaned considerably, and the inside had been repainted in beige. The choir screen is nice with tons of sculptures, but the most precious pieces of the cathedral are the absolutely beautiful and outstanding medieval stained glass windows. Nowadays after renovation again in their full shiny glory, detailing multiple stories of the bible, full of depictions of holy figures as well as old bishops, unmatched by most other churches I have been to. If you are a fan of stained glass, either visit Sainte Chapelle or this cathedral. While Sainte Chapelle impresses just by being absolutely dominated by the windows and almost no walls, in Chartres the glass windows are not that dominant, but the church is a lot larger, so there is a large variety of windows and stories to see. Having made it to Chartres earlier than I anticipated we also took an earlier train back and Reyer and I appreciated having another nice steak before getting on our TGV.\\

Amiens: Cathedral*****, Belfry**\\
Chartres: Cathedral****

\section{May 18--May 21: Northern Italy \& San Marino}
\label{2018:Italy}

Why I go to Italy this time: I planned to see the city of Bergamo for quite a while, I also wanted to go to San Marino, particularly after my office mate Rickard told me how much he loved to go there (OK I heard also opinions that it is a tourist trap). But time to find out myself. After previous considerations to go to San Marino in 2014 from Bologna,or later on from Venice never came to a successful conclusion, this time I made sure to have everything set up. Interesting fun fact: Pentecost Monday seems to be unknown in Italy, thus on Monday all buses and train run on a work day schedule.\\

Co-travellers:\\
Amin: Our first trip of this year, and once again to Italy combined with a micro state (Riju joked whenever I think about Italy I either ask him or Amin to join). But at least San Marino's claim to fame is, that it is the first Republic -- so maybe a bit more to it.\\

May 18:\\
We took the last train out of Geneva, once again staying close to the Milano Centrale at Hotel Bernina. This time we also had a quick Kebab before getting some sleep.\\

May 19: Pavia, and Brescia\\
We started the day at the Certosa di Pavia an ancient monastery which was partially in renovation at the time. The facade of its church is richly decorated, both with geometrical patterns, statues, marble, you name it. The interior is just as impressive, with tons of frescoes, murals, mosaics. Unfortunately it is not allowed to take photos, not that it is explained besides a written post next to the church door, which I overlooked. I got a very unfriendly reminder pretty quickly though that this is the case. The museum inside the former cloister is pretty nice too, with sculptures and reliefs from the former cells, as well as parts which had been taken off from the exterior for conservatory reasons. One studiolo of the monastery had been kept in its original form as part of the museum. Then we took the train back to Pavia, where we first saw the huge Duomo. This cathedral is huge and had been in construction for centuries, its size is impressive, but the inside is sparsely decorated, just as for other Italian towns the cathedral is not the highlight of the churches. After stopping for coffee on one Piazza we continued in the basilica of San Teodoro. The church was built from bricks, which is clearly visible on the inside, in the choir section old frescos from medieval times are conserved, and old Roman frescoes are on the floor. San Michele Maggiore is similarly built using bricks, but more of the medieval decoration can be seen, people set up the church for a concert of some kind, thus the crypt was closed. After a short stop by the baroque interior of Santa Maria del Carmine we stopped by the most beautiful church we saw in Pavia -- San Pietro in Ciel d'Oro. The clear highlight of this church is the sarcophagus of the holy Augustine, impressive all over depicting scenes from the bible, the life of the holy man, as well as further rich decoration in several floors. And we jumped on a train taking us to the town of Brescia. After a short lunch snack we had a short look into the baroque church of Santa Maria dei Miracoli. Unlike in most other baroque churches the ceiling, including the dome were left in plain white - I didn't find out if this was done intentionally or if they ran out of money during construction, or if the original decoration had been destroyed. While walking through old town we realised that this was the day of the Mille Miglia vintage car race. The town was full of little stages, and with time more and more vintage cars arrived and they were parked in prominent spots and the fans of the race started to come in as well for the big celebrations. The baroque church of San Maria Della Carita is built in an octagonal shape with an annexed chapel, replicating the interior of the Mary's house in Loreto. Brescia has been an important town already since Roman times. The remains of the capitoline temple and the theatre are still astounding, one of the oldest sanctuaries from the republic of Rome can be visited, going about 3 metres below road level of nowadays. Unlike in many other roman ruins, paintings and remains of the fries, but even the marble decoration is almost intact in two rooms of the sanctuary. By virtual reality you can bring the Capitol alive again as well. The floor in the main hall of the capitoline temple is still the original from Roman times. Even nowadays the square still holds a holy space in the form of the church of San Zeno. Besides the Roman ruins Brescia is famous for the Monastery of Santa Giulia. The monastery complex contains three churches, several cloisters, a really interesting museum with an ancient cross as old as the 8th century, as well as the remains of an old roman house. Afterwards we visited the old and the new cathedral and a modern art exhibit by Palazzo Martinengo. On this day the Mille Miglia race finished, thus the town was full of vintage cars, and little concerts and celebrations here and there, so really nice to spent an evening at.\\

Certosa di Pavia: Certosa*****\\
Pavia: Duomo***, San Teodoro****, San Michele Maggiore****, Santa Maria del Carmine***, San Pietro in Ciel d'Oro*****\\
Brescia: Santa Maria dei Miracoli****, Santa Maria Della Carita***, Capitol****, Roman Theatre****, San Zeno**, Santa Giulia*****, Duomo Vecchio****, Duomo Nuovo****, Palazzo Martinengo***\\

May 20: Bergamo, and Cremona\\
We walked through the lower town and took the cable car up to the upper city where we started with the very beautiful Baroque cathedral before heading to the even more beautiful basilica of Santa Maria Maggiore whose ceiling is filled with many frescoes and hundreds of little sculptures. The third little church on this square is the nice Cappella Colleoni with a large equestrian funerary monument. Next we enjoyed some city views from the tower of the Palazzo della Ragione. A little old Roman water reservoir can be found at the edge of the impressive city walls, which have been built in Venetian times. We walked around the city walls down to the lower city, saw a couple of fountains and churches on our way before having some pasta. A couple of trains later we arrived in Cremona, first visiting the Battistero and then climbing the cathedral tower. While we climbed the tower they had a last rehearsal of Jesus Christ Superstar (in Italian), which I would suppose was being performed later that day. I actually enjoyed the show knowing the Musical in its English version by heart. The Duomo is a very nice cathedral as well, not only the nice facade but also the interior where almost the whole walls are covered in large scale frescoes. And we had some more pasta and large cups of Gelato.\\ 

Bergamo: City Walls****, Battistero**, Duomo***** (lower church and treasury***) , Cappella Colleoni*****, Santa Maria Maggiore*****, San Michele al Pozzo Bianco****, Sant'Andrea***, Palazzo della Ragione****, Fontana del Lantro****, San Lorenzo alla Boccola**, Sant'Alessandro della Croce**, San Marco**, Sant'Alessandro in Colonna***\\
Cremona: Duomo*****, Battistero***\\

May 21: San Marino, and Milano\\
I asked Amin, if he wants to see quite a bit of Milano too, as this would mean getting up incredibly early. Amin didn't mind doing that, so we were getting up before 5 am, and then on a train to Rimini for almost 3 h, before taking a bus for about an hour until we reach the town of San Marino. San Marino is a fortified town on a hill top in the middle of the Republic. After a quick stop by the church of San Francesco we walked up to the Parliament which is located in the Palazzo Pubblico. Just a few metres beyond that is the classical basilica of San Marino. Then we walked along the walls up to the three towers of San Marino. The towers are in fact little castles, housing an armoury and special exhibitions nowadays. After spending some time at the National museum we had lunch by the city walls, before we made the way back to Milano. There we purchased the ticket for the whole Duomo complex. Once again the roof top is amazing and spectacular, as is the interior of the cathedral, unfortunately the middle nave is always off limits, thus one cannot get a close look on the choir, the main altar or the organ, or the little dome. The excavations are nothing special, once we got to the treasury -- we were informed that they just had closed (literally 2 seconds), and we should come back tomorrow (great if you are not in town then anymore). So I took Amin to the ossuary of San Bernardino alle Ossa, morbid but fascinating. And our last sight was the Galleria Vittorio Emanuele II, where we stopped before getting steak at Milano Centrale and getting home by the last train.\\

San Marino: San Francesco**, Palazzo Pubblico****, Basilica di San Marino***, Rocca della Guaita***, Rocca della Cesta****, National Museum**\\
Milano: Duomo***** (Roof Terrasse*****, Excavation***), Galleria Vittorio Emanuele II****, San Bernardino alle Ossa*****\\

This was the last trip I did with Amin before he moved back to California for his thesis write-up, after having done four trips with overnight stays. It was always great to see him enjoy Europe and its culture over and over again, considering he comes unlike most of my other co-travellers from a different background.\\

\section{May 25--May 28: Catalonia}
\label{Catalonia2018}

Once again Rachel had the idea for this long weekend: Rachel was scheduled to attend a workshop in the Pyrenees. The way to get there leads over Barcelona, and Rachel thought it is fun to spend a day in Barcelona, before sitting on the bus for the 3 h transfer. Seems she could only convince me to join her this time around.\\

May 25:\\
Usually I choose hotels based on location, price and rating, but often just normal style hotels. This time I got ourselves a pretty fancy hotel with balcony, roof top terrace, and roof top pool. We arrived late so no swimming in the pool, but we still enjoyed the bar and had snacks overlooking the city.\\

May 26:\\
I got up early, while Rachel decided to sleep in and checking out the pool and old town later on, until we would meet by Sagrada Familia. I took the train to Figueras instead. There Salvador Dali put up a museum to house his art-work, paintings, sculptures, diamonds, jewellery. All of that and plenty of it distributed over several floors. In fact I had seen the Teatre-Museu previously on my school trip to Barcelona, so I knew i would love to see it again.\\
After surrealist art, the next point focused on the ancient Roman era. Tarragona had plenty of those still conserved, an amphitheatre, a theatre, a circus, even the city walls are largely from Roman times. The cathedral was also a nice medieval building with later baroque additions as well. And the beach was also nice to see.\\
I met with Rachel by Sagrada Familia, unlike my previous visits the crypt had no religious service scheduled thus I finally got to see that part of the church and Antoni Gaudi's tomb. And then we were off to a fancy paella place. If you can choose to have a really expensive paella coming with chicken, pork, or beef. Or for a surplus of 7 EUR get the pasta with a full lobster, what would you do. I at least decided we should go for the lobster. Rachel was amused that I didn't really know how to eat it properly though. It is just not a normal dish for us Europeans. \\
For the second time I witnessed the light and music show on top of Casa Mila, where the roof top chimneys are used as projecting screen for nature, volcanoes, art videos (after a nightly tour through the house). Afterwards snacks are served with sparkling wine. Rachel did enjoy the event too so a good choice to go for it once more.\\

Figueras: Teatre-Museu Dali*****, Sant Pere**\\
Tarragona: Amphitheatre****, Roman Circus***, Torre del Pretori***, City Walls****, Cathedral*****, Forum Romanum***, Roman Theatre**\\
Barcelona: Sagrada Familia*****, Casa Mila (at night)*****\\

May 27:\\
Starting out our day with a long brunch in Barcelona Rachel got on here bus, which brought her to the workshop in the Pyrenees. The museums of Palau Nacional were my first stop of the day. Originally built for a world exhibition the palace had back then even a Throne hall for the king. Nowadays the building is used as arts museum. The Dome hall and the Oval hall are still very impressive rooms though. In Catalunya there are valleys with several old churches of the 12th and 13th century, some of those have been declared world unesco heritage by now. Some of the frescoes were deemed so important that they were cut out and transferred to the Palau Nacional. Thus I got to see the originals which clearly doesn't make the museum a UNESCO world heritage site itself.\\
Afterwards I walked over to the Olympic stadiums and the gardens, before visiting the Casa de les Punxes (far more impressive from the outside), and the first house built by Gaudi in Barcelona, the Casa Vincens. In fact since it was far more quiet, I appreciated it a bit more than Casa Mila or Casa Battlo. Since i had a bit of time left, I went back to old town and saw the remaining three columns of the Temple of Augustus.\\
Arriving at the airport my flight to Geneva was shown to arrive as scheduled. I had a some dinner and waited. Then I thought i could check my e-mails, and what did I see - an email which stated that my flight had been cancelled due to the strike of the French flight controllers in Marseille, and the proposed option for the next flight would be Wednesday morning. Clearly contradicting the airport screen stating my flight was supposed to be on time, I went to the Easyjet counter, where I was informed that indeed the flight was cancelled and I just should ignore the screen. So I had to get into the queue of people trying to get rerouted as soon as possible desperately (two people were working). About 2 h later I finally got to the counter, the last person of the line, since everybody behind me had given up anyway, and I managed to tell the actually really nice and polite lady my situation and she rerouted me over London Gatwick, which meant I would only need to take half a day off (which indeed was approved by my boss quickly, a good thing in a research environment is the flexibility of bosses and administration). I took a taxi to the hotel I was supposed to stay at...only to find out that the hotel had just given its last room to the person who just queued in front of me, who told them he would choose a fancier hotel originally, but then took the proposed room in the email anyway. The person on the reception told me to walk for another km to another hotel, where they might have more rooms available. Once I did arrive at that hotel (just 30 mins past midnight by now), I told the receptionist that I would need to get up at about 4:30, he gave me a whole holiday house with two floors, two bedrooms, a fully equipped kitchen and a large TV screen. Too bad I hardly made any use of it. I prepared myself a last tea, and then got myself less than 4 hours of sleep.\\

Barcelona: Palau Nacional*****, Olympic Stadium***, Jardins de Joan Maragall****, Casa Vincens****, Casa de les Punxes**, Temple of Augustus***\\

May 28:\\
Off to reception just before 4 am, getting a coffee and then back to the airport by taxi. This time the flight did leave on time, and then 3 more hours to spent at Gatwick Airport. Had another small snack while trying to work (not that successfully as the connection was not that amazing), on to my second flight of today and finally back in Geneva with over half a day delay.

\section{June 2--June 10: Great Britain}
\label{UK2018}

Many years ago I considered going to Scotland, mainly Edinburgh and maybe a trip from there. Back then I realised that both Durham and York were quite easily reachable from Edinburgh. Some nice estates are reachable from York as well, but then it would be already better to fly out from Manchester or Liverpool. Looking here and there the itinerary grew and grew until a couple of days were extended to a full week. Clearly this could be easily done by train, so getting a British Rail pass seemed the way to go. Since I didn't find friends who I could convince of my idea, my parents decided to join all a bit with the idea to return a day earlier since hotels in Manchester had gone up in price quite a bit (seems a music festival took place in the city on that date, and many Taylor Swift fans flooded the town too).\\

June 1:\\
The day my trip was supposed to start but it was already the second time in the same week that Easyjet didn't get me to my destination, since on the next morning nothing would enable to get me to Edinburgh early enough for something useful I accepted the proposed departure late Saturday. I called my parents to tell them that I wouldn't make it before late evening the next days and they would need to spend another day without me. We had gotten a Scottish heritage pass as well as tickets to Holyrood Palace so they could do that without me. Typically the one day tickets can be extended to a whole year entrance tickets via a stamp on the ticket. I hoped that they would do that at Holyrood even without me being present, and this was indeed the case, so back to my Geneva flat and waiting for another day.\\

June 2:\\
After spending another day in Geneva, participating in the Trivia world championship (not that successfully though), this time Easyjet got me to Edinburg where i met up with my parents, where we discussed our next day. I suggested they should have a quite morning stopping by Linlithgow Palace while I would do Edinburgh in the morning before getting to them later during the day.\\

June 3:\\
And I walked to the Castle old town enjoying the Royal Mile. St Giles cathedral was open, but not really. I got in and walked around and enjoyed it for what it was, but then a man told me it would actually be closed (strange considering about 10 other people walked around just like myself) and we were complimented out. Once again I felt VERY EXTREMELY welcome in Anglican churches. Seems to be a common theme, officially welcoming everybody, but not really - or only after a donation of 10 British pounds or so - but then still respecting of course the spirituality via not allowing photography. Once Holyrood Palace opened I got my ticket for the day. The palace is pretty decent, not as grandiose as Windsor Castle and Buckingham Palace, but nice for what it is. Some nice tapestries and nice ceilings, as well as old renaissance chambers of Mary Stuart times. Clearly the British Royal family doesn't mind getting paid by UK citizens but that doesn't mean you can take photographs of public funded art due to reasons. But that is handled in a similar manner in Spain, so what do I know. One of the rather curious parts of Holyrood Palace is the old ruined part of the abbey church, clearly fitting a stereotype. Edinburgh Castle is quite overrun for what it has to offer: A pretty decent great hall and a handful of rooms, old prison rooms as well as Scottish Crown Jewels which are though a tad underwhelming. It does look impressive on this old volcanic cone though. And then I got my rail pass stamped at the rail station and got on my train to Stirling. I enjoyed the Great Hall there more than the one in Edinburgh, the ceiling is far less decorated, but I just enjoyed the flair more. Just like for Dover Castle the remaining rooms (bar the castle chapel) are decorated in mock-up medieval pieces. After a short stop by the Church of the Holy Rood we had enough time to stop about an hour by Linlithgow. I had a short look at St Michael's before making my way to the palace. Once one of the main royal palaces in Scotland it burned out in 1746, still the walls are very impressive, one can climb up almost all four towers. My parents had seen the palace in the morning, so they had cake and coffee instead, now we took the train to South Queensferry, where we walked down to the village to have a nice view of the Forth Bridge. This Railway bridge was constructed in the late 19th century almost 2.5 km long. On the other side of the village is a large road bridge. And we walked back to the train station. There we did realize that the train coming over the bridge was delayed by about 10 minutes, but that the train to go over the bridge would arrive in just a minute. Thus we rushed over to the opposite platform, got on the train and crossed the bridge. Having arrived in North Queensferry we got off, crossed over to the other side and got on the train back. In Edinburgh I tried Haggis, actually quite tasty and some local beer. A nice dinner to end the day.\\

Edinburgh: Holyrood Palace*****, St Giles Cathedral***, Edinburgh Castle****\\
Stirling: Stirling Castle****, Church of the Holy Rude****\\
Linlithgow: St Michael's Church***, Linlithgow Palace*****\\
Queensferry: Forth Bridge****\\

June 4:\\
Glasgow: Cathedral****, City Chambers****\\
Fallkirk: Fallkirk Wheel****, Antonine Wall****\\
York: City Walls****\\

June 5:\\
We got up early and had a nice breakfast in a cafe, before walking over the city walls to the ruins of St Mary's abbey, which are located in one of the city parks, then we took a view outside photos of the Minster and waited for it to be opened. Already 5 minutes late into the officially stated opening time, the door finally opened and a Gentlemen posted a piece of paper, where the scribble said come back in an hour, training of the guards and the personnel. Firstly - no apology was given he rudely told one tourist that it would not be a big deal, once she asked if this couldn't have been communicated earlier - OR ONLINE, where they state special opening times (NOTHING was announced there), secondly he said - it was an unplanned training. This was a big FAT lie, other people who queued with us an hour later, told us they had been told not to come back too early the next day by guards, so unannounced my ass. With nothing better to do we walked through old town and over to Clifford's Tower. Unfortunately the tower didn't open its doors yet, so we could only see it from the outside. Then we arrived back at the minster, where a long queue had already be formed. 20 minutes after the scribbled one hour (so in fact 80!!!) minutes late the minster finally opened. Many people complained and were all rudely told off. By the way I wanted to know more about two things during going through the cathedral, and none of the guards knew even a hint of an answer, so seems the guide and guard training was useless after all. Anyways rant over -- the minster is a giant and very beautiful church, unfortunately I cannot comment on views from the tower, due to missing out on 80 minutes of time, I skipped that part. The large windows are one of the most beautiful medieval stained glass windows in England, some windows have been redone in a modern style. The choir screen is nice too. Unfortunately our time was cut short by the as usually very tourist unfriendly anglican church personnel. And once more we walked back over the city walls to the train station, getting over to Durham, after having a small snack in the train station. In Durham we walked up the hill to the university and got our tickets for Durham castle. My parents had been in Durham in the 70s and decided back then to rather see the cathedral than the castle, but then always wondered what they might have missed out on. Returning to Durham now over 30 years later, they were happy to finally get the chance to see it.\\

York: York Minster*****, City Walls****, St Mary's Abbey****, Clifford's Tower***\\
Durham: Cathedral*****, Castle****\\

June 6:\\
Knaresborough: Castle****\\
Ripon: Fountains Abbey***** and Studley Royal Water Garden***** (one ticket)\\
Kilburn: White Horse Walk****\\
York: Castle Howard*****\\

June 7:\\
Bakewell: Chatsworth House*****\\
Lincoln: Castle***, Cathedral*****\\

June 8:\\
Conwy: Castle*****, City Walls****\\
Caernarfon: Castle*****\\
Bangor: Cathedral**\\

June 9:\\
Liverpool: Cathedral*****, Metropolitan Cathedral****, St George's Hall****, Harbour Front Buildings***\\
Chester: St John the Baptist**, Cathedral***\\

June 10:\\
Manchester: Museum of Science and Industry****, Castlefield****, John Rydlands Library***, Cathedral***, St Mary's the Hidden Gem**, St Ann's Church*

\section{July 1--July 16: South Korea: ICHEP2018}
\label{Korea2018}

How this trip happened is easily explained, the largest important conference of high energy physics happened to take place in Seoul in 2018, I was among the people selected giving presentations on behalf of our collaboration. Naturally if one flies all the way to South Korea, it is only natural to spent a couple of more days in the country and explore it a big. Since I had a couple of friends and colleagues joining on this trip it was fun to take them to places too. In fact I did a trip within a trip with two other fellow-physicists who I only got to know during the conference.

Co-travellers on the first part:\\
Ulrike: a CERN fellow within the LCD group from Germany as well. \\
Eva: CERN stuff from the DP department but with strong ties to LCD and in fact working for CMS, which obviously holds a special place in my heart too\\

The plans:\\
We all were very excited to go to South Korea and since we have also many activities done outside of pure work, we decided to spend a tourist day before the conference, and clearly do other activities and dinners together. A complication which we faced is that in South Korea a concept of Vegetarian food is not that common, and since Eva is a vegetarian we had sometimes a bit of troubles to find a place which had the option. Most of the times they all then assumed we all would prefer a meat less diet. Not that this was a bad experience, since once a vegetarian option was on the menu, it was in fact tasty.\\

the trip within the trip to Busan:\\
just like I did in China from Beijing I had a side trip pretty much on short notice happening. Stephanie and Steffen, friends of Ulrike from her time at ATLAS, had already thought of getting a bit out of Seoul to the South East of the country to Busan. xyz, a professor who I worked with a couple of years earlier, with roots in Korea suggested I should try to see the ancient town of Gyeongju and its surroundings with many artefacts of the Goryo (xyz) kingdom. Since our high speed train stopped on this city too on our way to Busan, it was easy to get both Steffen and Stephanie on board for this place too.\\

July 1:\\
Oil Wells of Iraq*****\\

July 2:\\
Arriving when a typhoon hits the area can be interesting. I took a bus from Incheon airport to Seoul through heavy winds and large puddle of water and heavy rainfall, but guess that's what you expect in such conditions, but then the walk to the hotel was short, and I just went to bed right away, it was about 9 pm at this point anyway, so good enough to get some sleep. \\


Pakistan: Himalayas and Glaciers*****\\
China: Taklamakan Desert****\\

July 3:\\

The next day we realised that the only other international visitor on that tour through the modern palace besides us three has in fact been a physicist too, who attended the same conference like us.\\
After all of that we had seen enough of palaces thus we walked over to one of the city gates and the market area by Sungnyemun with little shops, little roads, and many food options.\\

Seoul: Changdeokgung Palace***** (Garden*****), Changgyeonggung Palace****, Deoksugung Palace**** (both traditional and modern palace), Sungnyemun City Gate****\\

July 4:\\
Seoul: Unhyeongung Palace**, Jongmyo Shrine***, Gyeongbokgung Palace****, National Palace Museum***, Gyeonghuigung Palace***, City Wall and Heunginjimun Gate****, Seolleung****\\

July 7:\\
Seoul: Lotte World Tower Observatory*****\\

July 8:\\
Paju: Imjingak Park****, Dora Observatory \& DMZ Tunnel****, Dorasan Station**\\
Seoul: Bongeunsa Temple***, Seoul N Tower*****, City Wall****\\

July 12:\\
Suwon: Hwaseong Fortress****\\
Gwangju: Namhansanseong Fortress****\\


July 13:\\
The starting day of our trip within the trip. We all had booked our rooms in the hotel in Busan separately, maybe not the smartest as we all could have had a large room for three too, but who cares, we also had gotten our train tickets three days before to be sure that we would face no bad surprises. The poor lady at the counter had troubles to understand our pronunciation of the train stations we wanted to move between too, but after all we managed to get it done, so early morning we met at the station got a snack and jumped on the Korean high speed train. Once we arrived in Gyeongju we got on the bus to Bulguksa Temple, one of the oldest and largest traditional Buddhist temple complexes in Korea. Very beautiful in a nice setting, also a bit of much needed shade, a lot of nice halls with wood carved statues and traditional architecture.\\
We finished a bit earlier than I thought so we went back to the town centre of Gyeongju to visit a market and getting a lunch snack. Since we had reached the temple an hour earlier than I thought (google maps didn't think we would make the first connection), I added the traditional Yangdong Village on our agenda. This village is one of the very few which is still conserved almost completely in its old traditional form with loads of wooden houses on a hill side. Since it was so hot (beyond 40 degrees and very humid), almost no other tourists made it there, particularly since we had to walk 50 minutes from the bus stop on the main road to get there. Coincidentally the only other tourists were three fellow Germans. We walked around for about an hour, watching some of the traditional houses from the inside and also getting something to drink, and then back by bus to Gyeongju.\\
There we went through parks which had loads of Royal Tombs of the former dynasty scattered all over, and an old observatory. Only at this point Steffen realised that in fact this was his second visit to Gyeongju, which had been the social program point on a workshop he attended previously a couple of years earlier. By that point we clearly needed to cool ourselves down, so we got into a local cafe getting their local cold freezing cold drinks enjoying the air conditioning as well.\\
Back to the train station and onwards to Busan, where our hotel was very close to the train station as well. And we walked over to Busan Tower which had though closed down for the day, but from the hillside we had also quite a view of the town, a park, and the bay. And then we did find a place which still served food and even plenty of vegetarian options for Stephanie, and good food options for myself and Steffen too.\\

Gyeongju: Bulguksa Temple*****, Yangdong Village****, Tombs \& Cheomseongdae***\\
Busan: Busan Tower**\\

July 14:\\

After having a traditional Korean breakfast (that's a no to sweet pastry etc which is what we Europeans typically have, but yes to soup, rice, or other ingredients), we took the bus out to Haedong Yonggung Temple. A temple sitting on cliffs overlooking the sea opening up from an adjacent forest. Indeed a very nice setting, and many locals were around to do exactly the same. Then we walked along the coast through the forest passing another small village temple, and a harbour with a rusty fisher boat. And then we took the bus halfway to the large sandy beach of Haeundae, where vapour clouds had formed within the city, which dissolved though while we walked through the sea water. We continued our walk along the beach up to the Nurimaru APEC House, which was a modern style congress centre set up for an APEC meeting in the early 2000s. Quite interesting to read up on that, and then we walked along the bay and got into a cafe for late lunch snacks and drinks cooling ourselves down in the air-conditioned area.\\
We looked into google which option was suggested to get to the train station: suggestion was to take the bus, since the metro line took quite a detour to reach Busan main station. We also made sure to add an additional 30 minutes on the projection just to be on the safe side. And this was indeed very needed, the traffic jam prediction from google for Busan failed totally on that very same day. With only minutes to spare we ran to our train and got on it just about 1-2 minutes before the projected exit time, and trains in Korea leave on time. And we had a large extensive last dinner together in Gangnam. Clearly this was the first and so far last trip I did both with Steffen and Stephanie. They flew out early morning the next day, while I had my flight scheduled around mid night the next day.\\

Busan: Haedong Yonggung Temple****, Haeundae Beach****, Nurimaru APEC House*****\\

July 15:\\
Seoul: National Museum*****, Myeong-dong Cathedral*, Dongdaemun Design Plaza*****, Olympic Park****\\

July 16:\\
Iraq: Mesopotamian Marshes****

\section{August 5: Diavolezza \& Chur}
\label{Diavolezza2018}

Pontresina: Diavolezza*****, R\"athische Eisenbahn****, Kathedrale Chur****

\section{August 12: Fort l'Ecluse}
\label{FortlEcluse}

Riju and I decided spontaneously that we should do something after an uneventful Saturday morning. Since it needed to be close we opted for the Fort l'Ecluse. This fort guards the only natural entrance from France into Switzerland close to Geneva sitting on a small canyon between the Vuache hills and the Jura mountains. The fort had originally been founded by the Savoys, extended by Vauban but then destroyed by the Austrians back in 1815. Between 1816 and 1828 the fort had been rebuilt. The fort is distributed over two sections. The lower fort is just next to the Rhone with many casemates a museum explaining the history of the fort, and another section dedicated to the bats which live in the rooms and cellars of the fort. Some of them endangered and thus the halls are closed off during some time of the year. Afterwards we started climbing up through stairs within a tunnel up the upper fort. There most of the fort had been closed off since the Fort is currently not safe to be visited, but some of the bastions were open and gave us fantastic views of the Rhone valley, and the Alps with Mont Blanc in the distance. There is also a Via Ferrate which can be used to climb from the lower to the upper fort. I have never done it myself, but several of my friends and colleagues did and they claim it is also suited for not so experienced people.\\

Leaz: Fort l'Ecluse****

\section{August 26: \"Oschinensee \& Bern}
\label{Oeschinen2018}

Kandersteg: \"Oschinensee*****, Bern: M\"unster****, Altstadt with Zytglogge \& Fountains****

\section{August 31--September 9: Germany}
\label{Germany2018}

At the time when we booked this trip, Reyer had never been to Germany in all his years in Geneva. By the time we did this trip Reyer had been on a workshop in Hamburg, but anyway for work purposes. Chris arrived only mid of this year. A newbie to most of Europe he decided to tag along, although he received a warning by Riju in advance. I regularly visit Bavaria. This time I put an itinerary together which was supposed to cover Bavaria as well as parts of Eastern Germany and Berlin, including Saxony-Anhalt which was one of the German states I had never been to.\\

Co-travellers:\\
Reyer and Chris B, both Californians doing their PhDs at UC Davis at the CMS experiment. Reyer has already quite an experience travelling with me, this time bringing his own umbrella along. Chris and Reyer will be the drivers on this trip, both already excited to see how fast our car will be able to go.\\

August 31: Flight to Munich:\\
Chris booked a couple of weeks after Reyer and I, by that time flying Easyjet was much more convenient than using Lufthansa, thus he arrived in Munich a couple of hours ahead, while me and Reyer took one of these delayed Friday flights out of Geneva. He also would stay one day longer in Berlin. Once having arrived in Munich it took almost an hour to get to the city (about a decade ago the mayor promised a fast S-Bahn would be built, but that never came even close to planning), and then we just had Kebabs and that was it.\\

September 1: Munich\\
Starting our day out by the largest palace of Germany, the Munich Residenz. After about a decade of renovation FINALLY the royal apartments had been reopened in 2018, at the expense of the Steinzimmer and the Kaisersaal (Hall of the Emperors) which were now being renovated. I was very happy to finally see these classical halls, planned by Leo von Klenze. The Throne halls of the king and the queen were particularly impressive, held in gold and white. The halls of the Nibelungs had been already built with the intention of being open to the public, romanticised frescoes of the late 19th century. The Baroque and Rokoko ``Rich Rooms'' show of the opulence of Bavarian Baroque which is closer to Italian than to French style. The treasury is full of artefacts spanning several centuries, don't forget to see the Cuvilies theatre as well. After my usual short lunch stop getting some of Grandma's meatballs we had a short look into some of Munich's churches before getting to Nymphenburg palace, the summer palace of the Bavarian rulers. The by far most impressive room of the palace is the Steinerne Saal (Stone Hall) with a large ceiling fresco and many chandeliers etc, some Chinese Cabinets and bedrooms. The heavy rain stopped us from enjoying most of the gardens, but we didn't want to miss out on the four little garden castles, among them the Amalienburg with the silver-blue Hall of Mirrors, or the Magdalenenklause whose chapel is decorated with sea shells, sea stars, and sea snails. Then we got our luggage at our hotel and off to the airport to get our rental car, quite a large SUV. Unfortunately other cars are not really suited for Reyer with his over 2 m. On our way to F\"ussen we got into an hour long traffic jam since a Porsche driver crashed his car (the driver seemed to be only mildly harmed). Once we arrived in F\"ussen we headed for Krone, unfortunately they were completely full so we ended up at the Kornhaus instead, still nice food.\\

Munich: Residenz***** (with Treasury***** and Cuvilies-Theatre****), Theatinerkirche****, Schloss Nymphenburg***** (with Amalienburg*****, Badenburg****, Pagodenburg**** \& Magadelenklause****), St Michael****, Frauenkirche***\\

September 2: Castles and Palaces in the alps and Regensburg.\\
After having an excellent breakfast buffet at the Luitpoldpark Hotel we had to queue about 25 mins to get handed over our reserved tickets for Neuschwanstein. I wonder how long it can take to check for those, but seems 3 mins per reservation is about the norm. Anyways we made our way up, naturally in bad weather, at least not in a snowstorm I thought. This time at least one could see the castle from the Marien bridge, but kinda disguised in fog and mist. Above it all unlike last time where we were told things about each room of the castle this time the guide only bothered to talk about four!!! of the rooms only - the indeed impressive throne hall, the bed room, and the singer's hall. But she couldn't be bothered to have any word about any of the other rooms, the bedroom was soon up for renovation and almost everything was covered in white sheets. Considering that it DID indeed feel like a vastly disappointing tourist trap nowadays. NOTHING really NOTHING about the magic the tour of this place once had. Oh yeah did I forget that photography is not permitted. While in early times the Bayrische Schl\"osserverwaltung (Bavarian palace administration) came up with the unscientific ``but it destroys the art'' excuse of a reason, this time they claim it disrupts the flow of the guided tours. I wonder how every other country manages otherwise, but maybe they have the same reasons like Austria. At least still feels like they pretend phones are at the same state like in the early 90s. Then we stopped by Wieskirche, where Sunday Mass just finished, thus you could still smell the candles and olibanum. The church was a tad fuller as usually, but still very great to see. And on to Linderhof palace. Unfortunately the Venus grotto was in renovation, but this time our guide was really great. I myself always prefer Linderhof to Neuschwanstein, and once again our guide confirmed how much more fun, cozy, and entertaining of a place Linderhof is. Unlike Neuschwanstein Linderhof was finished during Ludwig's II reign. Not as impressively situated like fairytale castle it is rather a summer pleasure palace type. The garden is sizeable with a Moroccan House and a Moorish Kiosk both originally constructed for a world fair in Paris and later bought by the King for his park. The fountain in front of the palace goes of every half an hour for about 1-2 minutes. After getting some sausages we had another short stop by the nice monastery of Ettal. About 90 minutes later we arrived by Regensburg old town. We had a stroll through the nice squares and old houses and stopped by the Gothic cathedral. Some of the stained glass windows of the Dom are from the 13th century and statues are also from medieval times. We crossed the Danube for some nice old town views with the Dom, and then on to our hotel. Back there we opted for an all you can eat buffet at a Chinese-Vietnamese restaurant, which also offer the odd choice of meet from Zebra to Ostrich to Kangaroo and a couple of other Asian cocktails, interesting but tasty selections -- enough to make us happy for sure. And these were the first two UNESCO world heritage sites of this trip.\\

Schwangau: Schloss Neuschwanstein****\\
Steingaden: Wieskirche*****\\
Ettal: Schloss Linderhof*****, Monastery****\\
Regensburg: Dom***** (cathedral), St Johann***, Old City Hall***\\

September 3: Danube Gorge, and Bamberg\\
The Liberation Hall is a classical style Rotunda commemorating the liberation of Germany from French influence among others, celebrating the German states as well as the generals who won crucial battles (among them e.g. Waterloo). It dominates the town and the Danube Gorge. From Kehlheim there is a ferry to reach the Weltenburg Monastery via the gorge in about 50 minutes. On our way we had the Weltenburger Klosterbier, the oldest monastery brewed beer, technically not the longest running still existing brewery, but the one by Tegernsee is not attached to a monastery anymore. The gorge itself is really nice with many rock walls nice forrests etc. Once we arrived at Weltenburg we first stopped by the late Baroque church, built by the Asam brothers, before having lunch at the monastery court. We opted to taste the Weltenburg monastery liquor as well, and we all enjoyed it. Our common friend Leona requested us to get her some J\"agermeister on our trip, but since all three of us agreed this liquor was so much better we got her this one and a bottle of J\"agermeister anyway, only to find out that she absolutely hated this one (our bad I guess). On the way back it started pouring heavily, so we enjoyed some coffee and tea while sitting on the lower deck inside. Then we drove along the Danube for another 20 km before we reached the greek mockup temple of Walhalla. Here the Bavarian king erected a temple to celebrate all significant Bavarian people, later this was extended to celebrate all famous German figures, among them Gauss and Einstein. On the village of Memmelsdorf is the summer palace of the bishops of Bamberg, the Schloss Seehof. Unfortunately the palace was already closed for the day, but the fountains and cascade were running just at this time instead. We might have crashed a couple of photos of a wedding party which took some photos at Seehof then, we did our best to avoid being in their photos at least. Then we dropped our luggage in Bamberg's hotel before walking via the old city hall to the Dom of Bamberg. This cathedral is one of the few medieval imperial cathedrals built in Romanesque style. It houses the tomb of German Emperor Henry II as well as the famous Bamberg rider. Across the cathedral is the archbishop's palace with a vast Rose garden where we spent some time in the evening sun. We found a nice restaurant where we had some good food. But then we wondered what Cold Duck in a Boot (yes that is the correct translation of Kalte Ente im Stiefel) would be. The waitress told us it is a type of Sangria which comes in a pitcher in the shape of a boot. Clearly we had to try, quite enjoying it anyway.\\

Kehlheim: Liberation Hall****, Danube Gorge*****, Weltenburg Monastery****\\
Donaustauf: Walhalla****\\
Memmelsdorf: Schloss Seehof**** (only park with switched on fountains)\\
Bamberg: Dom*****, Rose Garden****, Old City Hall***\\

September 4:\\
W\"urzburg: Residenz***** (with guided tour), Dom****\\
Bayreuth: Margravial Opera House*****, New Palace****, Eremitage*****\\

September 5:\\
Eisenach: Wartburg*****\\
Sch\"onstedt: Nationalpark Hainich****\\
Naumburg: Dom****\\

September 6:\\
Dessau-Rosslau: Bauhaus Masters' Houses***, Schloss Mosigkau****\\
Oranienbaum-W\"orlitz: Tabacco Factory Museum*, Schloss W\"orlitz***** (Guest Wing**** and State Apartments*****), Island Stein with Villa Hamilton*****, Park of W\"orlitz*****\\

September 7:\\
Wittenberg: City Church***, Palace Church****\\
Quedlinburg: Castle \& Collegiate Church****\\
Berlin: TV Tower****, Brandenburg Gate****\\

September 8:\\
Potsdam: Sanssouci***** (Schloss*****, New Palace*****, Orangery Palace*****, New Chambers*****, Picture Gallery*****, Roman Baths****, Charlottenhof Palace****)\\
Berlin: Philharmonie*****\\

September 9:\\
Berlin: New Museum*****, Pergamon Museum*****, Dom****\\

And this was the trip with most world unesco heritage sites so far, with 14 in total, beating even my 3 week trip in China. On top of that this number could have been very easily extended adding the modern housing estates of Berlin in, but I myself doubt I would really enjoy those on my own without an expert giving us the details that we could enjoy its significance in the proper context.\\

This was the last trip out of six trips with overnight stays with Reyer before he moved back all the way to California, watching 32 world unesco heritage sites together (second to Riju, excluding family members, who got a head start of about 20 years). And seems the trips were fun enough, that Reyer does come back to Europe for trips from time to time.\\

\section{September 21--September 23: Belgium}
\label{Belgium2018}

One more trip with Rachel, having been in Brussels three times before, didn't take much to convince me. This time Rachel came straight from a workshop in Germany.\\

September 22: Antwerp, Mechelen \& Brussels\\
Once again I got up early to do a morning adventure on my own, leading me to Antwerp for the printing workshop of Plantin-Moretus. There the old printing machines plus the old prints themselves were all on display. The rooms of the noble family itself are very nice to visit on their own as well. And then I went to see the Baroque church of Carolus Borromaeus which I missed out on during my last visit in 2017 as well. On my way to Antwerp I passed by Mechelen, and the tower of the cathedral looked really nice. Since I finished early in Antwerp I had more time to spare, thus I spontaneously got out by Mechelen train station and walked over to the Market Square and the cathedral. Not only interesting from the outside but also a nice gothic style church on the inside. I later found out the tower is also part of the famous collection of Belfried of Belgium and Northern France. Once I was back in Brussels Rachel informed me that she would be a tad late, thus I got myself waffles for lunch.\\
Then Rachel and I did the tour of the Cantillon Brewery followed by beer tasting, and then we got up on the Brussels Atomium. This time they had a nice visual-audio show running once again, and I am totally there for that. And clearly we had more beers and fried food close the Grand Place for dinner.\\

Antwerp: Museum Plantin-Moretus****, St Carolus Borromaeus****\\
Mechelen: Cathedral****, Grote Markt****\\
Brussels: Brewery Cantillon***, Atomium*****, St Catherine**, Grand Place*****\\

September 23: Leuven\\
Another day, another time to discover rain. It was raining cats and dogs. Originally we had wanted to have a slow start with walks through the Great Begijnhof, but then the rain didn't stop so we thought to change our program. One of Rachel's friends studied in Leuven, thus we met for Brunch in a cafe instead of touring the place until the rain would have faded away. Once it did we saw the medieval St Peter's church, followed by another lunch and beer tasting round in a local brewery. Then we did a guided tour of the city hall. From the outside it is very famous for its elaborate flamboyant gothic style facade with plenty of statues and stone carved decorations. On the inside most of the rooms are nowadays restyled in rich Baroque with lots of frescoes, paintings, and tapestries. And then off to the airport and more Belgium beer.\\

Leuven: Great Begijnhof****, Antonius Church**, St Peter's****, City Hall*****\\

This was the last multi-trip I did with Rachel before she moved back to the US for the write-up of her thesis, but it wasn't the last trip with Rachel, but more about that later on.

\section{November 4: Motiers \& Couvet}
\label{Motiers2018}

Did you know that absinthe was developed in the Jura mountains of Switzerland. After being prohibited for quite a few decades only recently it was allowed to produce and sell it again. Reason enough for Rachel to suggest a trip to the Absinthe Museum of Motiers. Since absinthe tasting was involved as well, it was not a problem to get several people on board which were Rachel, Christine, Grace E, Chris B, Kevin, Sasha and myself, all of us somewhat related to CERN. We all got our tickets beforehand, and decided to meet at Boreal for breakfast and coffee. Everybody arrived well on time, and after touring the absinthe museum we opted all for tasting three types of absinthe. We also got sausage and cheese which we were told would go well with it. It was a very nice experience and a nice welcoming snack going with it. Afterwards we had a short look in to the church of the village with its old wooden ceiling, before we had short stroll to the waterfall close to the river. Already on our way we realised that the river had completely dried up, and the waterfall well had no water running down as well. We used our cell phones to shine a bit of light in the little cave, the Grotte de Cascade, next to the waterfall. Unfortunately in the winery of Prieure St Pierre no tours were offered on the day, just wine for purchase. We were also too early for ordering dinner, seems 6 pm was just out of question. Thus we took the train to the next village of Couvet, where at least three restaurants were listed on google maps. Out of those only two were in fact open, and one indeed offered to serve us dinner, pizza, pasta and Flammkueche that is. Overall a nice relaxed day for an early November trip, and what Rachel and I thought might be our last trip together for a while, thankfully she already returned once again in 2019.\\

Motiers: Absinthe Museum*****, Church**, dried-up Waterfall and Grotte de Cascade****

\section{November 17--November 19: Provence}
\label{Provence2018} 

My most short term planned trip ever: Santona was in Geneva for the Heavy Ion runs. She wanted to make a short trip to somewhere, we originally though of Lisbon or Madrid. Unfortunately by the time we wanted to book the price for flight tickets skyrocketed up, thus we thought why not doing the trip by train. We still discussed about details until midnight, switching out possible destinations and decided to go for the order Arles, Marseille, Nimes, and Orange. Getting all high speed train tickets we still had the hotel to choose, so we went for the affordable option of the Ibis budget. Clearly not luxurious, but when booking on the same day, that's the best option, considering between Nimes and Arles as centre of operation.\\

Co-traveller:\\
Santona, Bangladeshi, but doing her PhD in the US, and working on my former experiment CMS. Whenever in Europe, Santona tries to see different countries and places,  e.g. Albania just a short while ago.\\

November 17:\\
Since Santona wanted to get some coffee before leaving we opted for the second earliest train out of Geneva (aka around 8 am). Two transfers later we arrived in Arles. The amphitheatre is one of the best conserved Roman arenas in the world, so we opted for a tour of that. The theatre is quite close to the road, so safe the couple of euros and just watch it from the fences. It still gives you most of the experience. Then we continued by the old cemetery of Alyscamps. While most of the old town of Arles is part of its own world UNESCO heritage site, the church in Alyscamps, St Honorat is part of the trails from France all the way to Santiago de Compostela. A large old romanesque church. Quite dark inside and not full of decoration, but particular in wintery limelight a certain mystique can be felt.\\
 Then we saw the old cathedral in Arles (nowadays the diocese was merged, and the cathedral is only a basilica minor anymore). The portal is richly decorated, the cloister is nice too, the inside offers a couple of tapestries. Afterwards we walked through the Marie-Desiree Smith exhibition they had on display in the old Bishop's palace. Remains of the old Roman forum can be visited as well, the underground storage area of the cryptoportikus, as well as the Baths of Constantine. Both cute but nothing special should you have seen other Roman remains previously. Similar things can be said about the Musee Reattu, but its modern art  exhibit offers quite a couple of nice drawings and paintings. We had to realise that many restaurants had closed for the season already, and we unfortunately had only a few places to choose from which were decent though. And then we made our way to Nimes after enjoying the illuminated Arena in Arles at night.\\ 

Arles: Amphitheatre*****, Roman Theatre****, Alyscamps with St Honorat*****, St Trophime \& Cloister*****, Bishop's Palace Exhibition***, Cryptoportikus***, Baths of Constantine***, Musee Reattu****\\

November 18:\\
Marseille is not high on many people's list when they consider destinations to see in France. But it does offer quite a few nice spots. We started our day getting large cakes and coffee to prepare us for more adventures to come. What is fascinating for myself, is the setting of the city built just by the sea with many lighthouses, old forts guarding the harbour, it even had an Imperial Palace (although that one looked pretty boring from the outside). For someone coming far away from any coastline the idea of a port, particularly an old idyllic harbour such as in Marseille is appealing. The cathedral is in fact an interesting large neo-byzantine church, not full of mosaics as other churches of that style, but with many columns or stone decorations in several geometrical patterns. The outside with two domed towers on its main facade and five domes around the choir is just spectacular.\\
 The abbey of St Victor has several layers with a large upper church and a lower section with several chapels and funery monuments all dating back to Romanesque medieval times. After having lunch in a small restaurant we walked up to the second neo-byzantine church of Notre-Dame de la Garde standing on the highest mountain overlooking the city. From there you have fantastic views over all of Marseille as well as the islands of the Frioul Archipelago with the Chateau d'If fortress. If you have time a boat ride over to the islands in summer time might be nice to do. In November those tours don't take place anymore. We finished the day by the museums of Palais Longchamp. The Palais itself is very eclectic 19th century with a long row of collonades, a large fountain with many layers, topped out by several groups of statues. The arts museum itself is alright, but far less spectacular than the outside. In fact lovers of brutalist architecture might want to check out Corbusier's Unit� d'Habitation.\\
I know about buildings, sculptures, art, also a bit about nature and hikes. I am definitely not the person to ask when it comes to food. Thankfully most of my travel companions can help out with this aspect. Santona was no exception to that rule having chosen a Michelin star restaurant for dinner in Nimes. Indeed a five course meal does come with its price, but it was still on an affordable scale and it was indeed fantastic. And once we had finished dinner we also did a small night walk to get some good view of the clock tower, the Maison Carree, and the amphitheatre.\\

Marseille: Old Harbour****, Cathedral****, Abbey St Victor*****, Notre-Dame de la Garde****, Palais Longchamp****\\

November 19:\\
Having stayed in Nimes for two nights by now, today we explored the town itself. The amphitheatre of Nimes rivals that of Arles, still in a fantastic state and still in use for performances. The romanesque cathedral is old but nothing to get too excited about. The Maison Carree is one of the best conserved Roman temples in the world, in fact it is almost fully intact. The inside from those times is though all gone, but a virtual tour gives you an idea what the temple looked like in antique times. Then we went to the Jardin de la Fontaine. The large fountain with its canals and statues is very pretty, that another old Roman building (in ruins this time) is just next to the fountain adds another idyllic point. A short hike up the hills behind the park gets you to Tour Magne, another relict from Roman times. Once you are up on the tower you have a superb view of the old town.\\
On our way back to Geneva we stopped in Orange. Orange is famous for its Roman theatre, one of the rare places where the stage of the theatre is still fully conserved. Conservations means also regular renovations, unfortunately for us most of the stage itself was behind scaffolding. It was nice though to walk around the catacombs and the seats of the theatre. After a short visit of the baroque cathedral we walked to the Triumphal Arch of Orange, one of the largest one in France, although not as precious as those in Rome or in Benevento. Since we still had time to spare we first checked out the local museum, getting some large hot chocolate by a local cafe next to the theatre before jumping on the train to Lyon and then to Geneva again. All in all a fantastic trip considering we did plan it just on the evening before going.\\

Nimes: Amphitheatre*****, Cathedral***, Maison Carree****, Jardin de la Fontaine with Temple of Diana \& Tour Magne*****\\
Orange: Theatre*** (in renovation), Cathedral***, Arc de Triomphe****

\section{November 25: Freiburg \& Basel}
\label{Freiburg2018}

We are all ready to see Christmas markets. Since Basel was listed as one of the most beautiful in Switzerland we decided to go there. And I suggested to do a detour to Freiburg im Breisgau, since it is so close to Basel and have a bit of German culture in between. Unfortunately I forgot to read the small print which stated that on this particular Saturday the Christmas market would open up only late in the evening.\\

Co-travellers:\\
Riju and Santona, both as excited as myself to have a good pre-Christmas time.\\

Once we arrived in Freiburg we quickly realised that my plan to have some nice local German food and drinks at the Christmas market wouldn't work out, we weren't the only ones having missed the small-print, since we met other friends of us from Geneva in Freiburg by coincidence. Thus I walked a bit through old town to find a Restaurant which would serve also local options (should Santona and Riju want to try some). And we found a nice place which offered for example Sauerbraten (which I chose myself) and also the Black Forrest Cherry Cake. Afterwards we saw the cathedral, the so-called M\"unster originally built by the citizens in gothic style with the transept in romanesque style. The tower is one of the very few big gothic church towers of Germany which was finished in medieval times. Santona and Riju opted for taking their time to climb the tower while i paid for the Choir chapels and then rushed up the tower (and yes I was pretty much out of breath afterwards), and then we jumped on the train back to Basel. The Christmas market in Basel is spread over a couple of squared and roads, the largest part being on M\"unsterhof. We did walk around the minster (clearly by that time it was closed). And then we started to walk along the stands having some Basler Leckerli as well as Raclette, Gl\"uhwein and/or Punsch as well as other nice things. In between we paid a visit to the big Christmas tree in the court of the city hall, and then about 2 1/2 h back on the train to Geneva.\\

Freiburg: M\"unster***** (with Choir***** and Tower Climb*****)\\
Basel: City Hall****, Christmas Market*****

\section{December 8: Burgundy}
\label{Burgundy2018}

This time we considered doing another UNESCO world heritage which was close and accessible to us: the wine region of Burgundy with Dijon as its main centre.\\

Co-travellers:\\
Santona and Chris B are on board, we also convinced two more Davis grad students, Graham (from the US) and YaoYao (her roots are in China) to join us on this trip.\\

We started the day in the Hotel-Dieu des Hospices Civil de Beaune, the former hospital for the poor in Beaune, one of the best examples of 15th century Burgundian gothic style, particularly famous for its glazed-tile roof. The museum describes the former life and activities of the hospice. While the buildings and ceilings are left in their original state, some of the furniture like tapestries or the main altar of the chapel have been moved into an adjacent museum nowadays. Some of the rooms had been refurbished later on, such as the Salon of Hugo, which had been refurbished with Baroque paintings. \\
We continued our trip to Dijon, due to the ongoing protests of the Yellow Vests our exit by the highway was blocked for over half an hour and we were given an improvised speech by a local activist why the protest was important and why we should be mad on the government instead of them. Once we did arrive in Dijon we visited the cathedral. While the cathedral itself is a normal decent sized gothic church, the romanesque crypt is very interesting with domed rotunda as center-peace and many small side chapels with nicely carved capitals on the sides. Then we had lunch with a nice bottle of white Burgundian vine. We originally wanted to visit the Ducal Palace which even should have offered a guided tour of the state rooms on that day but due to the ongoing Yellow Vest protests all the visits had been cancelled and the museum was closed. Instead we saw the churches of Notre-Dame and St Michel. While the exterior facades and portals of the churches were nice, the interiors were rather plain.\\
Thus we decided we had seen enough of Dijon and opted for impromptu visit of the Chateau du Clos de Vougeot. On the way out of Dijon we were once again stopped by Yellow Vest protestors, but this time only to receive candy. Anyways the chateau sits in the middle of the accompanying vineyards. Inside the castle they had a christmas fair going on, with two large decorated christmas trees and two people showing off their owls. The castle itself was pretty nice, the adjacent buildings around the courtyard had an old vine press in operation and showed how the vine had been processed in early times. A nice end of our touristic program before our ride home back to Geneva.\\

Beaune: Hotel-Dieu des Hospices Civil de Beaune*****\\
Dijon: Cathedral****, Hotel Aubriot***, Notre-Dame***, St Michel***\\
Vougeot: Chateau du Clos de Vougeot**

This was also the last trip I did with Santona. Indeed out trips were indeed only covering two months with two day trips and a multi day trip spanning 3 days, but we did manage to see four UNESCO world heritage sites, not too bad for 5 days of touristic outings.

\section{December 28--December 30: Ladenburg}
\label{2018Ladenburg}

December 28:\\
Maulbronn: Monastery*****\\
Ladenburg: Market Square****, Protestant City Church**, Dr-Carl-Benz-Garage***\\

December 29:\\
V\"olklingen: Ironworks*****\\
Saarbr\"ucken: Ludwig's Church****, Schloss***\\
Lorsch: Monastery****\\

December 30:\\
Since Ladenburg is so close to Mannheim and the weather forecast was nice, I convinced my parents to visit the large palace of Mannheim. Having missed out on the big staircase and the knight's hall a year before, this time every room could be visited. 

Mannheim: Palace*****\\
Rastatt: Alexander Church***, Palace*****, Palace Church****
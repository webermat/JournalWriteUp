\chapter{Year 2016}
\label{2016}

\section{January 2: Colmar}
\label{Colmar2016}

Colmar: Unterlindenmuseum*****, Martinsm\"unster***

\section{March 16--March 25: Sicily}
\label{Sicily2016}

I always hoped to see Sicily, in fact my mum had always the dream of seeing a volcano, so why not combine those ideas with a visit of Sicily. We realised flights from Geneva to Catania were over hundred Swiss Francs cheaper than from Zurich, so my parents decided to fly out from Geneva too, spending the night before the flight on the French side before. Since my brother unexpectedly happened to visit CERN in a work trip that very same day too we had a family dinner as well the night before. Since my brother anyway needed a car, he got my dad's car for the duration of the workshop, which also means we didn't have to pay any additional parking fees.\\

March 16: Drive to Piazza Armerina\\
We arrived in Catania late in the afternoon, got our (a bit large) Volvo at Hertz and then drove up to Piazza Armerina in the middle of the island. There we had booked a room in an old farm transformed into a hotel, we got Pizza at a local restaunr and then we already went to sleeep\\

March 17: Piazza Armerina, Cefalu \& Monreale:\\
We had a large breakfast in the main hall of the hotel overlooking Piazza Armerina old town on the opposite hill. The Villa Romana del Casale is a Roman Villa famous for its well-conserved extensive mosaic floors. Nothing much 

Piazza Armerina: Villa Romana del Casale*****\\
Cefalu: Duomo****\\
Monreale: Duomo*****\\

March 18: Palermo:\\
Palermo: Norman Palace****, Capella Palatina*****, San Giorgio in Kemonia**, San Giovanni degli Eremiti***, Chiesa del Gesu*****, San Cataldo****, La Martorana*****, Fontana Pretoria****, Piazza Vigliena****, Cathedral***, Castello della Zisa****, Castello della Cuba***, Teatro Massimo***, Giuseppe dei Teatini****

March 19: Palermo \& Segesta:\\
Palermo: Giuseppe dei Teatini****, Palazzo Mirto*****, Oratorio dei Bianchi***, Francesco dei Assisi****, Oratorio di San Lorenzo*****, San Matteo***
Segesta: Hera-Temple*****, Theatre****\\
Castellammare del Golfo: Church***, Harbour****\\

March 20: Selinunt, Scala dei Turchi \& Agrigento:\\
Selinunte: greek ruins in Selinunt****,\\
Realmonte: Scala dei Turchi*****\\
Agrigento: San Francesco***, San Lorenzo***\\

March 21: Agrigento:\\
Agrigento: Valley of the Temples*****, Church of S. Nicolas***, Archeological Museum****, San Spirito****\\

March 22: Val del Noto:\\
Comiso: St Blasius***, S. Maria Assunta****\\
Noto: San Domenico***, Cathedral****, Palazzo Ducezio***, San Carlo al Corso***, Montevergine***, Palazzo Nicolaci di Villadorata****, Santa Chiara****, San Francesco***\\
Siracusa: Apollo Temple***

March 23: Siracusa:\\
Siracusa: Santuario della Madonna delle Lacrime***, Duomo****, Apollo Temple***, Carmelite Church***, Amphitheatre***, Greek Theatre****, Necropolis****, Latomia del Paradiso*****\\
Catania: Francisco de Borja***, Basilica della Collegiata***, Cathedral***\\

March 24: Etna \& Catania:\\
My mum went to our hotel to heat herself up, while my dad and I paid a visit to the Roman Theatre and the Odeon. Those were hidden in the fundaments of old houses for centuries, before those were removed by and it was tried to give them back their original appearance, and one short visit to the Johannes-Bosco church.\\

Mount Etna Nationalpark: Mount Etna*****, Grotta dei tre Livelli****
Catania: Roman Theatre \& Odeon****, Johannes-Bosco**\\

March 25: Catania:\\
My mum didn't feel like going into the city so after breakfast she wanted to see the picture gallery inside of the Castello Ursino, my dad and I started with the large complex of the Monastery of San Nicolo l'Arena, nowadays part of the university of Catania. The church of San Francesco is a nice Baroque church too, even more impressive is the church of San Benedetto with a vastly decorated church with many paintings. Since we had still a bit of time left, I made it to the picture gallery of Castello Ursino, also nice there the Great Hall and the palace chapel, and then we took the bus to Catania airport. Now I transferred all photos to my laptop, gave my mum a UBS with all photos. On the flight to Geneva I sat on the side opposite of Mount Etna, I only got a glimpse of its snow covered top above the clouds. But then while continuing to fly north we passed the Aeolian Islands, also born in volcanic eruptions over millennia. Most impressive were the islands of Volcano and Stromboli, where an eruption send up clouds at that time. Once back in Geneva we had one more coffee at CERN, before my parents departed, I went home to prepare for my trip to Barcelona the next day, and stupid me deleted all the photos I had shot on the plane, thus no photos of all the volcano and islands from that flight.\\

Catania: Monastery of San Nicolo l'Arena****, San Francesco***, San Benedetto****, Chiesa della Badia***, Castello Ursino***

\section{March 26--March 29: Barcelona}
\label{Barcelona2016}

How this trip happened: Easter always offers a couple of days to spend somewhere else without the need to take any holidays out of your holiday budget. Not that this was of any matter at the end of a contract. Flights to Barcelona were pretty cheap, thus I decided we should check it out again. My second time since the year 2000, when I was in Barcelona during a high school cultural excursion. Since Barcelona is quite crowded nowadays we got a couple of tickets beforehand already, particularly for Sagrada Familia it is notoriously difficult to get tower tickets on the same day.\\

Co-trallers:\\
Riju: after over half a year since our quite turbulent last weekend trip, Riju feels his ready and more prepared for a long weekend, deciding not to join me on every single adventure, but to take it a bit slower than myself.\\

March 26: Barcelona:\\
After arriving and dropping our stuff in our hotel, we went to Casa Mila straight away, where long queues awaited us. Riju bought tickets using their free WIFI and we were ready to embark on our journey through Art Nouveau and modernist Barcelona. The inside of Casa Mila is not that amazing, but the decorations of the inner courts and particularly the roof are very spectacular. I then visited the Casa Llero Morera, the house with the most interesting inner decoration of the four houses I went on to see on that day, particularly the stain glass windows. The house had one of the first elevators in Barcelona, integrated right from the start. For Casa Battlo the facade and the roof are definitely the highlights, the ceiling are nice toowith their integrated large lights. Then the clear highlight of the whole trip was on our agenda: the Sagrada Familia, also the most reviewed building on tripadvisor. And man, did it change since 2001, when I last visited. Back then almost none of the interior had been finished, the choir walls and the two facades had just been finished by then. Now the interior of the church was complete, illuminated through colored glass, the gates decorated with little animals, like snails or beetles, and leaves, columns reminiscent of palm trees. Once more walking up the towers was very special, particularly seeing the decorations up-close. While Riju called it a day, I continued with the Casa Amtaller. A lot darker than the other houses, it was supposed to have a kind of medieval infused appearance. After a short dinner I went to a night show at Casa Mila, particularly solemn, with a movie projected onto the roof chimneys and towers. And as icing on the cake I got snacks and a glass of Cuvee.\\

Barcelona: Casa Mila*****, Casa Llero Morera****, Casa Batllo****, Sagrada Familia*****, Casa Amatller***\\

March 27: Barcelona:\\
Barcelona can get quite busy, but if you go to the old city very early in the morning, you are pretty much on your own. Even the cathedral is deserted. Mainly open for prayers in the morning, it gives you a really peaceful experience just walking through the nave and sitting below the Dome in solemness. Quite the opposite about what you might experience just a couple of hours later, when you rather walk through the packed hordes of people in a stop-and-go pedestrian jam. The cathedral itself is of gothic style, but with many baroque altars in its side chapels, as well as carved wooden seats in the choir. There are also two big other gothic style churches in Barcelona (Santa Maria del Pi and Santa Maria del Mar), but the cathedral is far more beautiful. On Saturdays the city hall, Casa de la Ciutat, can be visited -- back in 2016 for free. The main assembly meets in a room reminiscent of the time when the building was originally constructed. Many of the paintings in neighbouring rooms have been changed over times, thus you are offered an interesting mix of times. Then we took part in the first tour of the Palau de la Music Catalana of this day. Barcelona is famous for its Art Nouveau buildings, particularly by Antoni Gaudi. Only concentrating on Gaudi's building you might miss out on this beautiful piece of architecture, even a world unesco heritage building on its own account. The stain glass of the roof is absolutely breathtaking, the decoration and statues around the stage are magnificent. During the tour we even were given the opportunity to hear the organ play Bach's Toccata in d-minor. By that time we really were pretty tight on the schedule. Riju thought it would be hopeless to reach, but I ran as fast as I could over to the Palau de la Generalitat de Catalunya, the parliament of the Catalonian region. Originally I was informed that no tour was taking place on this day a couple of weeks before, but then a few days before the trip I received an email, that on short notice a tour would be available. Indeed just a minute after I arrived the doors were closed, Riju who also made it eventually was standing in front of closed doors. There are a lot of depictions of the national saint of Catalonia, St George or St Jordi in Catalonian, all over the place. His deeds and miracles are depicted in the giant hall of St Jordi, which houses also the most impressive fresco in the palace, covering all of the upper walls and the roof. I met up with Riju again to see Palau Guell, one of the big projects Gaudi did for one of his most important clients. I personally think it is the most beautiful house Gaudi built in Barcelona (out of the four I know), particularly the central hall with its built-in organ. Riju had decided to spend the rest of the afternoon in the Picasso Museum, which he was really impressed by. Since I had seen quite a couple of Picasso paintings already, I decided to see the grand theatre of Liceu instead in a cute tour. Then I saw the remains of the medieval royal palace of Palau Reial Major, and then went on a stroll through the Barro Gotico with a short visit of the baroque Basilica de la Merce, and then a walk along the alley up to the Columbus column. The Frederic Mares collectors museum was on my way, but nothing too interesting report from it. \\

Barcelona: Cathedral*****, Basilica Santa Maria del Pi***, Casa de la Ciutat****, Palau de la Musica Catalana*****, Palau de la Generalitat de Catalunya****, Palau Guell*****, Gran Teatre del Liceu****, Basilica Santa Maria del Mar***, Basilica de la Merce***, Palau Reial Major***,  Frederic Mares Museum*\\

March 28: Zaragoza:\\
Barcelona is a very exciting place to see, but it is as interesting to discover new places, this time Zaragoza. Close to the train station is the Aljaferia Palace, a fortified medieval Islamic palace. The Golden Hall and the northern halls reminded me a bit of the Alhambra. While the Alhambra is totally out of this world, the Aljaferia is still a really nice place to see, on the level of palaces in Marrakesh for example, another highlight is the old mosque. Once the islamic times in Spain had gone, the catholic Monarchs constructed another royal wing in the late 15th century. The ceilings are clearly the highlights of this wing, particularly in the new Throne room. In the following centuries the palace fell in despair and only in the 1980s the restoration was finished.
Afterwards we passed over the Renaissance courtyard Patio de la Infanta and the St Engracia with a beautiful facade. The most important buildings in Zaragoza are the cathedral and the Basilica del Pilar. The cathedral had been built throughout many years, the apsis and the choir are of particular interest in Mujedar style, the stuccos and sculptures alongside the chapels are really special, the veto on photography is also very rare for Spanish cathedrals though. The gigantic baroque basilica del Pilar had been build for about 200 years, largely finished in the late 19th century. The four towers and multiple domes clearly dominate the city scape of the old town. The domes of the basilica are partially decorated with ceiling paintings by Goya, the main building surrounds a baroque chapel building. 
On the way back a passenger complained about our ``loud english'', while his friend next to him didn't bother him talking on the phone just a bit later on. The family in front of us was quite lively too with one of the kids having to throw up just when before we arrived in Barcelona again - oh well. And the gentleman who complained about us, also lost some 10 or 20 EUR when getting up from his seat, and Riju had to run after him to give it back to him. Riju told me that a step tracker claimed we had walked about 23 km on this day.\\

Zaragoza: Palacio de la Aljaferia*****, Ibercaja Patio de la Infanta***, Basilica de Santa Engracia**, San Gil Abad***, Cathedral*****, Basilica del Pilar*****,San Juan de los Panetes*\\

March 29: Barcelona:\\
The Park Guell is a vast public park in the north of Barcelona. It had been developed shortly after 1900 by Antoni Gaudi, Nowadays the central part of the park, with multicolored mosaic Salamanders, terraces, and a hall with lots of columns and tile ceilings is so popular, that you have to pay to get in. The surrounding colonnaded foot trails and other terraces are not that popular, and nice to spend some time there. The former Hospital de la Santa Creu i Sant Pau was built by Lluis Domenech i Montaner, the architect of the Palau de la Musica. In 2003 the hospital moved out of the complex, and it has been transformed into a museum and cultural centre. The main entrance and former administration building is the most impressive part, other parts of the complex have been restored to their original appearance on the inside.\\

Barcelona: Park Guell*****, Hospital de la Santa Creu i Sant Pau****\\

And this concludes my travel chapters while working at UCLA: My life was supposed to change quite a bit, starting a new job in a new group and in a new (not yet existing) experiment, thus for at least the first couple of months I decided not to take too many days off.\\

\section{April 10: Gruyeres}
\label{2016Gruyeres}

Gruyeres: Musee HR Giger****, Chateau de Gruyeres****, Broc: Jaunbach Gorge****, Monsalvens***

\section{May 5--May 8: Piemonte}

May 5: Venaria Reale \& Torino\\
Venaria Reale: Reggia*****, Castello della Mandria****\\
Torino: Palazzo Reale*****, San Lorenzo****, Palazzo Carignano****, Palazzo Madama****, Duomo****\\

May 6: Racconigi \& Aglie\\
Racconigi: Castello****, S. Maria Assunta****\\
Aglie: S Maria***, Castello Ducale****\\

May 7: Torino \& Palazzo di Stupigini\\
Torino: Castello del Valentino*****, Borgo Medievale***, Villa della Regina****, San Filippo Neri***, Egypt Museum*****\\
Nichelino: Palazzina di Caccia di Stupinigi*****\\

May 8: Torino\\
Torino: Basilica di Superga****, Palazzo Accorsi****, Basilica del Corpus Domini****, San Domenico***, Palazzo Carignano****, Santuario della Consolata*****

\section{May 14--May 16: Liguria}
\label{2016Liguria}

Another long weekend, another trip to Italy: This time I wanted to see the town of Genoa with its many little palaces. I convinced Amin to join me on this trip, Eric had unfortunately no time, I decided to use a bus for once, trains just took too long. \\

Co-traveller:\\
Amin: UC Riverside grad student, Iranian. This is our first visit together, and I hope he will enjoy it just like other people did before. Iranians typically need a visa for almost every other country in the world, thus being in Switzerland on a Schengen visa opens up quite a couple of places to visit for him here.\\

May 14: Genoa\\
Starting our trip just shortly after midnight at the bus station of Geneva in a little bus with two other co-travellers. We crossed the Mont Blanc tunnel had a short stop for snacks in the Aosta valley before we had to exchange buses in Milan. We arrived on time by the harbour of Genoa, dropped our luggage by our hotel and started our day with a breakfast and some espresso. Our hotel was located very close to the train station, also not far off from old town. Amin found out that his pin code for his card was too long for the interface of the Italian ATM, so at least no money by the train station. We had our first tourist stop by the nice baroque basilica dell'Annunziata, before walking along the Rolli palaces to the cathedral of San Lorenzo. The choir was covered in scaffolding, but the side chapels and the nave were pretty nice. We then saw an arts exhibition by the Palazzo Ducale. Unfortunately most of the rooms were covered by panels for the exhibition, the large council hall was closed for the day, only the duke's chapel was free. Genova is full of Baroque churches, every major square has at least one of them, e.g. the Chiesa del Gesu, another superb church, with a nice frescoed main dome and also nice domes on the side naves even. \\
We just needed to walk along the streets, no matter which large church we stepped in, they all were pretty and beautiful. One of the famous palaces to see is the Palazzo Spinola di Pellicceria, the visit covers four floors, all of them have at least one nice room, like e.g. a Gallery of Mirrors on one of the upper floors, but also with nice paintings by Brueghel the younger and Rubens, as well as porcelain plates and vases. After a quick lunch we continued to see a couple of other small churches. In the afternoon we returned to the Palazzo Ducale, where a talk was presented in the Small Council Hall. And then we saw the highlight of the day - the Palazzo Reale, originally a palace of one of the local families which was bought by the Royal family of Savoy, which later became the kings of Italy. The rooms are richly decorated, including a nice gallery of mirrors, a throne hall in red (similar to all other Italian Throne Halls I have been at), a room with several nice tapestries and a roof terrace with a nice view of the harbour with the light house, as well as a superb view of the garden. The Nymphaeum with depictions of horses, peacocks, jaguars, and a weird looking depiction of elephans, was my personal highlight of the gardens. On the ground floor of the royal palace they had an exhibition about Antonio Canova. And then we had dinner and Amin watched the Eurovision Song Contest for the first time, not that we felt the need to vote for any of the songs.\\

Basilica dell'Annunziata*****, Cattedrale di San Lorenzo****, Chiesa del Gesu*****, Basilica di Santa Maria delle Vigne****, Palazzo Spinola di Pellicceria****, Santi Vittore e Carlo****,  San Luca***, San Pietro in Banchi***, Palazzo Ducale****, San Donato***, Palazzo Reale*****\\

May 15: Genoa\\
After a small breakfast we walked over to the Villa del Principe, which had a nice garden, and a nice decorated Loggia as well as other nice rooms with old tapestries depicting sea battles, and the legends of Perseus. Many of the ceilings were covered in nice frescoes as well. The gardens had two nice fountains as well. Then we walked up the hill to the romantic house of Castello d'Albertis. The palace had its own version of a mock-up turkish tent inside one of the rooms, as well as artefacts from the Andes as well as Mayan towns. After a nice stroll through some of the hill side quarters we ended up in old town again by Palazzo Rosso. Nowadays this palace is together with the Palazzo Bianco and the Palazzo Doria Tursi an arts museum, full of old masters. Some state rooms of the Palazzo Rossi are part of this museum as well, I liked the rooms of the four seasons in particular. From the roof terrace one has a nice 360 degree view of the whole city. Amin went back to the hotel to relax. I myself had read the day before that the coastline by Genoa was very nice, and an village with a nice old harbour and a small beach would be somewhere around there. So i just started to walk along the shore. The walk was nice, the weather was too, I got myself an ice-cream and continued to walk. But about over an hour in I got bored (not knowing how far longer it would take) and decided to walk back. Looking on a map later on, i missed a little bit less than 2 km to reach the village. Anyway I met up with Amin again and we had dinner and desert.\\

Villa del Principe****,Castello d'Albertis***, Palazzo Angelo Giovanni Spinola***, Palazzo Rosso****, Palazzo Bianco***, Palazzo Doria Tursi**, Loggia dei Mercanti***,  Basilica di San Siro****\\

May 16: Cinque Terre\\
Starting early in the morning getting to the coastline of Cinque Terre by train. We almost missed our train, as the reception was not occupied, even in a 24 h reception, some natural urges have to be followed up from time to time. Anyway we made it and arrived by Riomaggiore. Not too many people were around before 8 am, we went to the harbour and enjoyed the cliffs. Sadly the trail to Manarola is currently being refurbished (still not opened by begin of 2020). Then we had a short breakfast and took the next train to Manarola. The view of the tiny harbour and the village is really beautiful, for me the nicest view of all the villages. The beach of the village is tiny, but some people were already jumping into the water here too. By the time we got off the train in Corniglia considerably more people started to appear (around 9:30 am). Corniglia is the only village without a ferry stop, thus buses or trains are the only means by which it can be reached. Or by hike from the surrounding villages. And then we took the train all the way to Monterosso al Mare. This village is the only one in Cinque Terre with a large sand beach, also with a couple of smaller and larger rocks, which a few brave people did climb on. There are two little churches in the village, the oratory of the dead is more beautiful one of the two. And then we got on the ferry back along the coastline. The ferry is not large, thus the ride can be a bit bumpy, not that I myself had any issue with that, although almost everybody was fine with it. We saw the fifth village of Vernazza only from the sea side and the harbour.. But it looked nice too, the church is directly by the harbour, and a Castello is just a short walk by the harbour as well. We didn't get really close to Corniglia, although a trail exists there to get to the sea via a couple of stairs. Passing Manarola we once again stopped in Riomaggiore. There we had another lunch with pizza and desert. And then we took the train back to Genoa. Just as we arrived by our hotel I received a call of our bus-driver that they arrived and whenever we would be ready, we could start. Since we didn't have much planned to do anymore we went down to the tavern by the port, had another espresso and we (as only passengers from Genoa) started our way home about 40 minutes early. We did arrive in Milano on time, but unfortunately one of the passengers who was supposed to get on board there was nowhere to be found. Thus we left Milano with an additional delay of 35 minutes. After the usual traffic jam by Courmayeur before crossing the Mont Blanc tunnel we arrived in Geneva with about 45 minutes delay. Other folks remained on the bus on their way to Basel. It was even early enough for Amin to get back all the way to France.\\

Cliffs and coastline of Cinque Terre*****, Monterosso al Mare: John Baptists***, Oratory****

\section{May 28--June 3: Cantabria}
\label{Spain2016}

A new job comes with new colleagues, new tasks -- and in this case with a talk at a workshop just two months in. The workshop took place in Santander in the northern Spanish province of Cantabria. Over there Manuel and I had considered a short trip a couple of years ago, particularly to see the cave paintings which are some of the oldest known pieces of art of humankind. I realised that flying into Santander itself was cumbersome, but flights to Bilbao followed by an hour long bus ride would be an alternative, albeit this flight was only taking place on Saturdays.  \\

May 28:\\
Nothing much happened on that day, flight, bus, half an hour walk to the hotel, and then i walked along the sand beach of El Sardinero during flood times, thus a bit high water.\\

Santander: El Sardinero beach****\\

May 29: Puente Viesgo \& Santander:\\
A large fire destroyed most of Santander, thus almost nothing of old town remains nowadays, the cathedral was rebuilt, but just in a more modest form. Anyway today was the day to see the caves. I reserved my visits and even my place on the bus to get there. Above Puente Viesgo are several caves, two of which can be visited. Since cave painting are delicate, and too many visitors might be bad for the caves microclimate only a handful of visitors are allowed to go inside. I started with the Las Monedas cave, which has more figurative painting, but just younger. Both caves have also nice hallways with stalactites and stalagmites, El Castillo even more of those than Las Monedas. In order to not put too much stress on the art, only 5 guests at a time can come close, while the others remain in the neighbouring cavity. In El Castillo the cave art are full hands which have been place on the stones, thus the oldest known hand painting of the world. Despite not being allowed to take photographs (understandable for the cave paintings, maybe they could have allowed it in the normal halls), it was really exciting to see the oldest pieces of art on the continent. Since I had more time left, after being back in Santander I visited the exhibitions of the pre-historic archaeological museum where artefacts from the caves are shown nowadays.\\

Santander: Cathedral**, pre-historic archaeological Museum****\\
Puente Viesgo: El Castillo*****, Las Monedas*****\\

May 30: Burgos \& Santander\\
Burgos: Cathedral*****, San Nicola*****, San Gil****\\
Santander: Palacio de la Magdalena***\\

June 3: Bilbao\\
Bilbao: Guggenheim Museum*****, San Nicolas de Bari***, Cathedral***\\

June 4: Flight back:\\

\section{June 11: Lauterbrunnen \& Bern}
\label{LauterbrunnenBern2016}

Why this trip and co-traveller:\\
Andrew, my former co-worker from UCLA, hadn't been over in Europe for more than a year, and clearly once he is back we had to use the opportunity to visit the mountains once again, but this time I put not only nature, but also culture on the agenda. My attempt at putting local food on the agenda was thwarted by too expensive prices.\\

We took the car up to Lauterbrunnen. Usually I start my trip up to Jungfraujoch from there, but this time we had other plans: One of Switzerlands highest waterfalls is the Staubachfalls, but it also possible to hike up the cliffside and get behind the wild waters of the waterfall itself. Clearly one has to be willing and ready to get soaking wet, clearly you are guaranteed to be rained on just standing less than a metre behind a waterfall. Still feeling and hearing the power of the falling water is an experience you don't want to miss out on. Then we drove a bit further into the valley to the Tr\"ummelbach waterfalls. These are in total 10 waterfalls hidden within the mountain, where all water from the Jungfrau mountain massif falls down before ending up in the Weisse L\"utschine river. You can imagine that being fed by five glaciers the waterfalls are pretty strong and powerful. Being hidden within a cave like trail adds to the experience. If you are afraid to get wet or afraid in dark places, it might not be your thing to see. Don't fear of needing to walk up, a lift through the mountain takes you to the top, and you just shouldn't slip on your way down. All in all the waterfalls are spectacular and if you end your hike through the rock face with a valley overview in fog you understand that the Lauterbrunnen valley claims to be one major inspiration for J.R.R. Tolkien's Rivendell. And then it was time for local cake and coffee.\\
And on the way back we stopped in Bern. Bern's old town is located between a horseshoe bend of the Aare river. Thus it is a nice natural setting and the old town is more or less unchanged since the 18th century. The gothic style Berner M\"unster is the largest church in Switzerland with stained glass windows from mid 15th century and a main portal originally constructed in the 17th century. The town itself still has many medieval fountains scattered all over the roads, the gates are still in their original form as well. Then we wanted originally to taste Berner Geschnetzeltes, which is small cut veal served in a tasty sauce going with R\"osti or mashed potatoes. Unfortunately prices were south of 50 Swiss Francs thus we went back to Geneva and had some affordable pasta and pizza there instead.\\\

Lauterbrunnen: Staubachfall****\\
Stechelberg: Tr\"ummelbach Falls*****\\
Bern: M\"unster****, Fountains and Gates of Old Town****\\

\section{June 24--June 26: Ravenna \& Padua}
\label{RavennaPadua}

June 25: Ravenna:\\
Ravenna: Archbishop's Chapel and Diocesan Museum*****, San Francesco**, San Giovanni Evangelista**, Arian Battistero***, Sant'Apollinare Nuovo*****, Theoderic's Palace***, Dante's Tomb***, Neonian Battistero****, San Vitale*****, Galla Placidia Mausoleum*****, Santa Maria Maggiore***, Theoderic's Mausoleum***, Duomo***, San Lorenzo in Cesarea****\\
Classe: Sant'Apollinare in Calsse****\\

June 26: Padua\\
Padua: Emeritani Museum***, Scrovegni Chapel*****, Eremitani Church***, Oratorio di San Rocco****, Duomo***, Battistero****, Basilica di Sant'Antonio****, Palazzo della Ragione*****, Scuola del Santo*****, Basilica di Santa Giustina***, Maria dei Servi**

\section{July 1-July3: Bordeaux -- Eurocup 2016 Germany:Italy}
\label{2016:Bordeaux}

How did we end up here: The UEFA European Championship 2016 took place in France. Considering that it is just around the corner I created a UEFA fan account and tried to get tickets, but ended up getting none. During play off rounds, last minute tickets are up for sale a couple of days before, a maximum of four tickets per person (though not indicated, where the seats would be precisely. The quarter final match between Germany and Italy was scheduled on a Saturday, thus I asked if I could have the day off before, in case I would indeed get tickets (was granted). Moments before sales opened I kept refreshing and refreshing and ended up in a waiting room for 5 minutes. By that time I already informed myself how I could get there: flights were booked out, train to Paris and fan TGV seemed to be the option of choice; hotel rooms were available as well. So I called my sister if she wanted to join and she called my parents to see if they would be up to. After a quick decision (still 1 minute in the waiting room) everybody was on board, so as soon as i got out of the waiting room, I got myself two pairs of tickets, then my sister booked a large motel room for all of us. We decided my dad would pick my sister and me up from Basel train station and we would go with my dad's car.\\

July 1: Driving through France and Bordeaux evening\\
Getting up before 6 I made it to one of the first trains to Basel, met my sister, had some coffee and then getting into the car. Driving by the extinct volcanoes of the central massive, with a stop for snacks around Clermont-Ferrand we finally arrived in Bordeaux, and got to the stadium in order to exchange our voucher for the real tickets. In the evening we took the tram into old town, where we checked out the night views of the Place de la Bourse with the Miroir d'Eau. The square is filled regularly with water (only 2-5 cm), which mirrors the classical buildings of the square (thus Water Mirror). We then saw the last bits of the match Wales-Belgium on the Fanzone by Place des Quinconce. The city was already full by German and Italian fans, our motel was mainly filled by German fans.\\

Place de la Bourse \& Miroir d'Eau*****, Place des Quinconce***\\

July 2: Match day in Bordeaux\\
The old city of Bordeaux used to be an important port, and is of historical and culture significance and even considered UNESCO world heritage, thus enough to see for a day. The large churches had been stops on the pilgrimage to Santiago de Compostela as well. We started the day by the large basilica of St-Michel, a gothic church with a separate clock tower, continuing with the romanesque church of St-Croix. After lunch we walked past the remains of the city wall over to the Place de la Bourse. The water mirror is definitely more impressive at night. After another coffee stop we visited the cathedral. Although not necessarily bad, it is a bit plain particularly if you consider all the other cathedral which France has to offer. After getting a Bordeaux wine bottle in order to prepare for an after match night cap, we got on the bus, which should pass by the stadium. At some point we realised though that the bus was taking another route as typical, thus we got off and walked towards the next tram station. Seems trams and cars were the only means of accommodation to get to the stadium. The stadium was rebuilt in view of the euro cup, with 35 000 seats, one of the smaller stadiums in 2016. We got two seats in the German fan-block, just behind the goal. After 10 minutes of intro-programme, the match started. \"Ozil scored the first goal for Germany, Bonucci scored the 1:1 in a penalty for Italy. After overtime the result was still 1:1, thus penalties started. In between Schweinsteiger could have scored for Germany for the victory, but nope he failed. Thus it went on ... and on ... and on. Zaza tried to decay Neuer - and failed too. In the in total 18th! attempt, Jonas Hector finally scored for Germany for the final result of 6:5 after penalties. One of the rare victories of a German over an Italian team in play-off rounds. Celebrations and chanting all round us (and unintentional beer showers). As the stadium slowly emptied, tram after tram was leaving, each of them completely packed. We arrived at the motel roughly and hour later, and then had our Bordeaux wine.\\

St-Michel****, St-Croix****, Cathedral***, Stade Matmut-Atlantique***\\

July 3: Drive home\\
After sleeping a bit in and hours of driving, we all arrived home. A fun trip, but if I would go out of my way to just see Bordeaux in this region, the answer would rather be no.

\section{July 10: Nid d'Aigle}
\label{Niddaigle2016}

St Gervais-les-Bains: Glacier de Bionassay*****\\

This is so far the last time I went on a hike (out of four in total) with Pieter. A contributing factor was, that once you have two kids it can get tricky squeezing a fifth person into your car.\\

\section{July 16--July 17: Zermatt \&Fiesch}
\label{ZermattFiesch2016}

July 16: Zermatt:\\
Zermatt: Kleinmatterhorn*****, Matterhorn Glacier Hike*****\\

July 17: Fiesch:\\
Fiesch: Aletschgletscher*****, Fieschergletscher*****

\section{July 22--July 24: Brussels \& Aachen}
\label{Brussels2016}

July 22:\\
Brussels: Grand Place*****\\

July 23:\\
Brussels: Cathedral****, Palais Royal*****, Maison du Roi**\\
Aachen: Dom*****, Cathedral Treasury*****\\

July 24:\\
Brussels: San Nicolas***, National Basilica****, Stadthuis*****, St Catherine***, Notre-Dame du Sablon****,  Atomium*****, Notre-Dame de Laeken***

\section{August 12--August 29: US National Parks}
\label{US2016}

August 13: Chicago\\
Chicago: Cloud Gate****, Millenium Park***\\

August 14: Chicago\\
Chicago: Chicago Art Institute*****\\

August 15: Loosing the wind-safe hat at Hoover Dam : \\
And the day of our first inner US flight of this trip between Chicago and Las Vegas (United once again), and we had our transfer via a stretch limousine, actually the cheapest mode of transportation, it reminded me of my last visit to Chicago in 2012. This time I had the middle seat, so nothing to report about what route we took, my mum had to sit on a separate place from us. Then we got our gigantic rental car (gigantic at least for Europeans, no way it could have made it through old towns in Italy or France). We stopped shortly before the border from Nevada overlooking the low water levels of Lake Mead. We crossed the state border by the bridge spanning over the Colorado River with impressive views, and then it happened: Heavy winds and my mum's hat was set free, and made a happy trip down before landing on the Hoover Dam. She had asked specifically for a wind-safe hat, but seems windy times in Germany are nothing compared to windy times in the US. At least the hat was definitely gone. The first thing we did after arriving at our hotel was shopping for another sun hat, this time with laces, in order to be indeed wind safe. And yes - no wind was strong enough to let the hat sail away until now in 2019\\

Hoover Dam****\\

August 16: Grand Canyon:\\
Grand Canyon National Park: Grand Canyon View Points*****

August 17: Grand Canyon:\\
Grand Canyon National Park: South Kebab Trail*****, View Points*****\\

August 18: Horseshoe Bend \& Upper Antelope Canyon\\
Grand Canyon National Park: Grand Canyon Lipan Point*****\\
Cameron: Little Colorado River Gorge****\\
Page: Horseshoe Bend*****, Upper Antelope Canyon*****\\

August 19: Monument Valley \& Mesa Verde\\
Oljato-Monument Valley*****\\
Gooseneck State Park****\\
Mesa Verde National Park*****\\

August 20:Mesa Verde: \\
Mesa Verde National Park*****, including Cliff Palace*****\\

August 21: Arches National Park\\
Moab: Arches National Park*****\\

August 22: Bryce Canyon\\
Colorado River Rock Canyon***\\
Eagle Canyon****\\
Red Canyon*****\\
Bryce Canyon National Park*****\\

August 23: Bryce \& Zion Canyon\\
Bryce Canyon National Park*****\\
 Zion National Park*****\\

August 24: Zion Canyon \& Las Vegas:\\
Zion National Park*****, 
Las Vegas: Bellagio Fountain Show****, The Strip****, Volcano****\\

August 25: Las Vegas\\
Las Vegas: Luxor***, Excalibur***, Strip at Night****\\

August 26: Washington DC\\
Washington DC: White House***\\

August 27 Washington DC\\
Washington DC: United States Capitol*****,  Library of Congress****, Washington National Cathedral***, Washington Zoo****, Basilica of the National Shrine of the Immaculate Conception*****\\

August 28: Washington DC\\
Washington DC: World War II Memorial****, Lincoln Memorial****, Vietnam Memorial***, Korean War Memorial***, Martin Luther King Jr National Memorial***, Jefferson Memorial****, FDR Memorial***, Free Sackler Gallery***, Museum for African Art***, Hirshhorn Museum****, National Gallery****, Matthew Cathedral***\\

August 29: Washington DC:\\
On this day I met up with Luis, an actual tour and travel guide who has travelled the world far more than I did, and who I got in contact with on travel forums before. Once I told him I will be heading to DC, he suggested that we could meet up, should i have time on my hands. Since my parents weren't too keen on another US Capitol tour, but I still wanted to see the House Wing, I decided to suggest that to Luis. And indeed we met in the US Capitol lobby, and joined the tour of the House Wing. I was particularly impressed by the Old Hall of the House of Representatives, which houses nowadays statues of famous figures from all states, e.g. Rosa Parks. Once we finished the tour, he suggested to have a look into the Senate and House halls, which were up for few, since both the house and the senate were not in session. After this he guided me to the National Archives, which in fact I even didn't put on my own DC bucket list. Thanks to him I got to see the Rotunda for the Charters, where the US Constitution, the Bill of Rights, as well as the US Declaration of Indepence are put on display. After a coffee break we went on to the Botanical Gardens and the National Museum of Natural History (Diamonds and corll reefs on display), and then it was time to make it back home after a metro ride, followed by a seemingly endless bus ride to Dullas International.\\

Washington DC: United States Capitol*****, Botanical Gardens***, National Archive*****, National Museum of Natural History****

\section{September 10: Verbier}
\label{Verbier2018}

Verbier: Glacier de Corbassiere*****, Mont Fort*****

\section{September 11: Hermence}
\label{GrandeDixence2016}

For reasons still unknown to me -- let's call it my own stupidity -- I managed to upload photos on social networks but somehow also do delete all photos of that trip, even before making a safety copy, the second time of that very same year, but so far last time this happened. Usually photos are a good aid keeping things you experienced in memory, else the danger is that many things fade away with time without constant reminders through photos.\\

Hermence: Barrage de la Grande Dixence*****, Glacier de Cheillon*****

\section{September 13-- September 18: Arlington}
\label{Arlington2016}

\section{October 8--October 9: Santiago de Compostela}
\label{Santiago2016}

Santiago de Compostela: Cathedral*****, Pelagius Church****, Fructuosus Church***, University***, Francesco di Assisi***, Capilla San Roce***

\section{October 13--October 16: Cologne}
\label{Cologne2016}

October 13: Flight to Cologne:\\
Cologne: Dom*****\\

October 14: Cologne, Benrath \& Limburg:\\
Cologne: Dom*****\\
D\"usseldorf: Benrath Palace****\\
Limburg: Dom*****\\

October 15:\\
Br\"uhl: Schloss Augustusburg*****, Schloss Falkenlust****\\
Bonn: M\"unster****\\
Cologne: Dominican Church***

October 16: Flight Back:\\


\section{November 25--November 27: Milano, Mantua \& Verona}
\label{Milano2016}

November 25: train ride to Milan:\\

November 26: Milano \& Mantua\\
Milano: St Maria delle Grazi with Last Supper*****, Santuario di San Bernardino alle Ossa*****, Palazzo Reale*****, Duomo*****, Mantua: Palazzo Te*****, Basilica di Sant'Andrea*****, Duomo****\\

November 27: Mantua \& Verona\\
Mantua: Duomo****, Palazzo Ducale*****, Teatro Bibiena*****, Verona: Arena****, Basilica Sant'Anastasia****, Duomo****, Santa Maria Antica**

\section{December 4--December 17: Japan}
\label{Japan2016}

December 4: Sunday: flight to Paris\\
This time starting out in Zurich, with a planned transfer in Paris Charles-de-Gaulle. Unfortunately we started out with about an hour delay, after the deicing lasted quite a bit of time. With just 25 minutes to spare for transferring I was told to just start running to switch terminals. Once I started running to the departure gate for my flight to Tokyo, I heard already the last call for my next flight. I still made it through passport control and to the gate in time, even not being the last person to be allowed on board.\\

December 5: Monday: flight to Tokyo\\
I had troubles to stay awake on this 12 h flight, but flying over the Tundra of Russia and a couple of movies helped me to get through. The plane arrived so late, that the train office was closed, so no possibility to exchange a Japanese rail pass voucher on this day. Once I arrived in Japan I almost missed the fact that the train I originally got on only stopped at fast line stops, but a local resident made me aware to switch to the correct train. Anyways I finally arrived at my hotel, already quite late, greeted by Origami birds on the hotel bed.\\

December 6: Tuesday: train ride to Morioka\\
Waking up before 6 am, I got on the train too early to be able to exchange a voucher for a Japanese rail pass, thus full price for the first Shinkansen north. Then checking in at the hotel, which was just a 5 min walk away from the conference centre, and on to the first session of the workshop. It did start to snow on this day, and continued to do so for the remaining days of the week. The food was really great in Morioka, particularly the Sushi.\\

December 8: Hiraizumi\\
The conference advertised to see a bit of the surroundings, with a focus on nearby hot springs, the coast line (unfortunately not accessible by train anymore due to earthquakes destroying the connection), or the old temples of Hiraizumi. The most important artefact is the golden altar of Konjiki-do in the Chuson-ji temple complex. After passing by the remains of the Muryoko-in temple, I saw my first nice old Japanese garden with a rock monument in a little lake.\\

Hiraizumi: Chuson-ji Temple (with Golden Hall)****, Muryoko-in***, Motsuji Temple and Garden****\\

December 9: Tokyo\\
Tokyo: Tokyo Skytree*****, Senso-ji Temple***, Tokyo Tower****\\

December 10: Nikko\\
missing out on nature\\

Nikko: Shinkyo Bridge***, Rinno-ji****, Toshogu Shrine*****, Taiyuin Reibyo*****, Furarasan Shrine*\\

December 11: Kyoto\\
Kyoto: Imperial Palace and Gardens****, To-ji Temple*****, Nishi Hongwanji****, Taiyuin Reibyo****, Kosho-ji Temple***, Higashi Hongwanji***, JR Kyoto Station****, Nijo Castle*****\\

December 12: Kyoto\\
Kyoto: Kiyomizu-dera  Temple*****, Ginkakuji  Temple*****, Rokuon-ji  Temple*****, Ryoan-ji Temple****, Ninnaji Temple*****, Kamigamo-Shrine, Shimogamo Shrine***, Rishoin, Daigo-Ji*****, Sambo-in***\\

December 13: Nara \& Horyuji\\
Nara: Todai-ji Temple*****, Kasuga-taisha Shrine****, Shin-yakushi-ji Temple****, Gangoji Temple**, Kohfukuji Temple*****, Toshodaiji Temple*****, Yakusjiji Temple****\\
Ikaruga: Horyuji Temple****, Chuguji Temple***, Horinji Temple****, Hokiji Temple****\\

December 14: Hiroshima \& Himeji\\
rebooking etc\\

Hiroshima: Atom bomb Dome****, Peace Park \& Museum*****, Hiroshima Castle***\\
Himeji: Castle*****, Koko-en Garden****\\

December 15: Miyajima \& Osaka\\
rebooking etc\\

Hatsukaichi: Miyajima Island - Itsukushima Shrine*****,\\
Osaka: Castle***, Umeda Sky Building*****\\

December 16: Kyoto, Uji \& Tokyo\\
Kyoto: Fushimi Inari-Taisha Shrine****, Tenryuji Shrine***\\
Uji: Byodo-in Temple*****, Ujigami Shrine*\\
Tokyo: National Museum for Western Art***, Tokyo Metropolitan Art Museum****, National Museum Tokyo*****\\

December 17: Tokyo\\
Tokyo: Imperial Palace****, Metropolitan Government Building****, Teien Palace****
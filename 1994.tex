\chapter{Year 1994}
\label{1994}

\section{July 9--July 23: Westerwaldtreff in Oberlahr}
\label{1994:Westerwald}

In Daun we had close encounters with ants, this time heavy rain fall had flooded parts along the Rhine and we had to go on a daily hunt for flies, trust me I got really good at catching them after this trip. Our house had two floors, once again my brother and I occupied the upper floor.\\



%Limburg: Dom, Dietkirchen Lubentius
%Burg Runkel
%Montabaur: Schloss
%Bonn Haus der Geschichte, Beethoven
%Drachenfels
%Limes bei Rheinbrohl
%Linz, Burg Rathaus
%Grube Georg

\section{October-November: Telfs}
\label{1994:Telfs}

Once again we drove to Telf and Austria up the little house with the two floors, and the little tiny indoor pool. One of our visits lead us to the beautiful Renaissance castle of Ambras which is just on the suburbs of Innsbruck by a large forrest. The Knight's hall (Rittersaal) has many paintings of the Habsburg ancestors all over the place, as well as a beautiful courtyard in the main castle with an armoury. We visited once again an old mining town, this time the town of Hall. We walked up a hill to get to the church of Maria Locherboden, built in gothic revival style. The alps are a good place to find gorges with little waterfalls, also in Tirol we hiked through a slot canyon and enjoyed the little ladders and wooden paths over the rapid waters and the waterfalls between the rock walls. This time we took a cable car up the Karwendel massif. We even had first snow up there, which I was particularly excited about. On one of the last days we spent hours in the fortress of Kufstein just by the border of Austria and Bavaria.\\

Innsbruck: Schloss Ambras\\
Hall\\
Maria Locherboden\\
Karwendel Mountain peak\\
Fortress Kufstein\\

%Oct-Nov Telfs
%Schloss Ambras
%Hall
% Wallfahrtskirche Maria Locherboden
%Imst, Rosengartenschlucht
%Festung Kufstein
%Karwendel
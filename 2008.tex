\chapter{Year 2008}
\label{2008}

\section{January 2--January 4: Dresden}
\label{2008:Dresden}

January 2: Arriving in Dresden:\\

Kreuzkirche***, Frauenkirche****\\

January 3: Meissen:\\

Meissen: Dom***, Albrechtsburg****, Dresden: Gl\"aserne Manufaktur**

January 4: Dresden:\\

Zwinger*****, Hofkirche****, Frauenkirche****, Gr\"unes Gew\"olbe****, Festung Dresden***

\section{January 17: CHIPP School 2008 and St Gallen}
\label{2008:StGallen}

In January 2008 the first ever CHIPP winter school took place at the Linth Arena in Naefels with its own attached large indoor swimming pool. The lectures focused on LHC physics, but covered other topics as well, the neutrino lecture was given by Takaaki Kajita, who received the Nobel Prize in 2015 for his discovery of Neutrino Oscillations. On one of the days we had a free afternoon which I decided to use to go to St Gallen and see its beautiful cathedral. It had been over a decade since I had last seen this church. One of the late large Baroque churches in central Europe it is very ornate, classical influences can be already seen in the choir area. It was built as the church for the Abbey of St Gall, and later pronounced the cathedral of the newly founded Diocese of St Gallen. Unlike other Baroque churches a large wide rotunda dominates the main nave with a large fresco depicting the return of god.\\

St Gallen: Cathedral***\\

\section{May 3: Neuchatel, Solothurn, Bern, Sion and Lausanne}
\label{2008:Swisstrip}

Neuchatel: Collegiate****, Solothurn: Ursenkathedrale****, Jesuitenkirche******, Bern: M\"unster****, Sion: Tourbillon****, Lausanne: Hafen***

\section{July 18: Annecy}
\label{2008:Annecy}

Cathedral*

\section{July 27- August 6: Philadelphia}
\label{2008:Philadelphia}

How this trip came into existence:\\
My project was supposed to start in autumn 2008. Thus there was a vital interest in studies, which could be performed in first data. Each experiment of the project typically hands in a couple of abstracts. Some of those are accepted for a talk, others might be merged, or offered to be presented at poster. From my working group we had the offer of one talk covering several topics and one poster, covering one. Since the conference -- ICHEP 2008 -- was taking place in Philadelphia, I was eager to go to the US for the first time in my life. This meant though hard 3 months of very long hours, dedication and write up of public document, and we managed to get the approval from the collaboration that the results were mature enough to go public. Having started to work in early 2007 this was also the time to get my first own credit card (even late for European standards), but something got stuck and I received the card only two weeks after coming back from the US. Since the trip would be reimbursed later anyway, the costs were taken over directly by the institute. Thank God the institute's secretary was a very friendly lady, thus i wasn't too much yelled at for causing this extra work for her.\\

Co-travellers:\\
My first work trip, thus many many physicists, alone from my university my professor, one postdoc and two other grad students were attending.\\ 

July 28: My first ever transatlantic flight:\\
Connecting over Frankfurt, the first Lufthansa flight still had Sandwiches although Lufthansa catering was on strike. On the bus we passed and airplane and I thought already, that airplane looks pretty dirty compared to those around it -- and guess what, that was the US Airways Boeing, which I had to sit in later on. Up to now still the worst transatlantic flight I ever had, not that the crew was unfriendly or not attentive, they just claimed because of the Lufthansa catering strike they had for each passenger only one sandwich as food for the whole flight -- underwhelming. Immigration was pretty quick, by then the Immigration officer already even knew about our conference - I also got the obligatorily ``Swedish people are always so nice'', when I told the officer I came from Switzerland. Once I arrived at the hotel I was very surprised and pleased to see it had a rooftop pool.\\

July 29: Time to explore the city:\\
Philadelphia is the sixth largest city of the US. Pretty nice for a European to see a real skyline, which is something not many cities here back in Europe have to offer. That comes for most US cities usually at the expense of having any real old buildings or history of hundreds of years attached to it. I stopped by the cathedral, which was built in mid 19th century in a baroque revival style, and the city hall (known from many movies, e.g. 12 Monkeys). Philadelphia used to serve as first capital of the United States. It was also here, where the US Declaration of Independence had been signed. Unfortunately all slots to go inside Independence Hall had been filled, thus I didn't get to go inside. but saw e.g. Liberty Bell, the Second Bank of the United States and further buildings of the historical district. Most buildings there can be visited free of charge, check out Carpenter's Hall, the meeting place of the first Continental Congress, also home to one of the first US flags.\\

Liberty Bell***, Carpenter's Hall****, Historical District***, Cathedral***\\

July 30-August 5:\\
The conference itself took place at the University of Pennsylvania. In the evenings we hung out by the White Dog Cafe and the Black Cat tavern. This was my first major conference, so everything was new and exciting. I learnt a lot about all different aspects of my field, met many new interesting people. The conference has one day free for social excursions, some of my colleagues decided to rent a car and go up all the way to New York City, I decided to stay in Philadelphia and check out the Philadelphia Museum of Art (also the Rocky Statue on the side). It was a nice museum, but pretty much the same you can see in European places. The conference dinner was also a buffet in said museum. Overall I spend some time in coffee shops but rather focused on the conference itself than doing too much touristy things. Overall it was fun though. The flight home was nice and relaxing, particularly on the flight from Frankfurt there were only a dozen of people in this A320, so quite some space and nice views of the alps.\\

Philadelphia Museum of Art****

\section{September 3--September 9: Madrid}
\label{2008:Madrid}

September 4:\\
Royal Palace*****,San Isidro****, Almudena Cathedral*****, Campo del Moro***\\

September 5:\\
El Escorial: Habsburg Palace \& Monastery****\\

September 6:\\
Segovia: Aqueduct*****, Cathedral*****, Alcazar****, Madrid: Prado*****\\

September 7:\\
Monasterio de las Descalzas Reales****, Iglesia de San Miguel***, Museo Reina Sofia*****, San Francisco el Grande*****\\

September 8:\\
El Retiro Park****, San Jeronimo el Real***\\

September 9:\\
San Antonio de los Alemanes****, San Andrea****, Nicolas de los Servites***


\section{October 5: Jungfraujoch}
\label{2008:Jungfrau}

Even back in the 80s my parents always discussed if we should go up Jungfraujoch. Even back then being the most expensive ride in the alps, my parents decided it was just not affordable for a family of 6. Since I started now with my first regular job, I thought it would be a nice opportunity for a weekend out. I asked Sigi, Anne and Gianrico if they would have time as well. While Gianrico had others plans, Anne and Sigi were on board.\\

Co-travellers:\\
Sigi and Anne, while Sigi was now working as a PhD student in condensed matter physics, Anne had started here PhD in mathematics, so for all of us a day out in the mountains is a refreshing getaway thinking about work.\\

We were supposed to meet up in the train, but since the weekend was one of the late nice autumn weekends trains were expected to be so full, that an additional train was added for people coming from Bern. Thus we actually only met up in Interlaken where we decided to get to Grindelwald and get out return tickets up there. Even with the half fare card it is still 90 CHF for the last part (both from Grindelwald and Lauterbrunnen). I didn't inform myself properly about the procedure, thus at the Eiger Nordwand station I stayed on the train. On the way up the train stops at every station for around 10 minutes, this gives you time to leave the train take a couple of photos by the giant windows of the station and then getting back in time. Later trips up to Jungfrau always confirm the Nordwand station is not as impressive as the stop later - the Eismeer station. By this station you see the start and the icefall of the Grindwald-Fiescher glacier with the Eismeer glacier in the distance just below Schreckhorn. By the year 2000 both glaciers united to form the Lower Grindwald glacier by now this tongue has almost completely collapsed, thus this glacier will be only a note in history after the next 10 years or even earlier. Anyway we got to the final destination, the Jungfraujoch, on a saddle between the Jungfrau and the Moench mountain peaks. We first enjoyed the view down the valley as well as the impressive Aletsch glacier, with over 25 km length the largest of the alpine glaciers. Next we went up the Sphinx observatory, still used as a weather station nowadays, and then all the way down to the large glacier palace carved inside the ``eternal'' ice of the Jungfraufirn. Then we had lunch and a slow walk along the paved path. On our way back we asked at Kleine Scheidegg if we could exchange our ticket for Grindelwald to Lauterbrunnen instead. Indeed the valley of Lauterbrunnen is by far the more beautiful valley, in fact on most postcards and advertisements of Switzerland it is shown most prominently as THE alpine valley to see, dominated by falling cliffs and the Staubach waterfall. Rumour has it that Tolkien was inspired by the valley to come up with his descriptions of Rivendell. In Bern our paths split up again, but it was nice to see Anne and Sigi once again, not knowing that these occasions would get a bit more rare in the future to come.\\

Jungfraujoch*****, Kleine Scheidegg****

\section{November 29--November 31: N\"urnberg}
\label{2008:Nuernberg}

November 29:\\
Kaiserburg****, Sch\"oner Brunnen****\\

November 30:\\
Jakobskirche***, Elisabethkirche***, St Lorenz*****, Frauenkirche***, St Sebald****, Egidienkirche**, Lochgef\"angnis**, Weihnachtsmarkt****\\

November 31:\\
Kongresshalle*****, Haupttrib\"une Zepellinfeld****

\section{December 3: St Blasien}
\label{2008:StBlasien}

Dom****

\chapter{Year 2012}
\label{2012}

\section{February 11--February 26: Edit 2012}
\label{2012:EDIT}

February 12:\\
Batavia: Minos**, D0***, CDF***\\

February 13:\\
Wilson Hall**\\

February 18: Chicago:\\
Field Museum**, Willis Tower***, Cathedral**

\section{May 11--May 13: Prague}
\label{2012Prag}

May 11: Karlstejn\\
Burg Karlstejn**, Fanziskus von Assisi*, Nikolauskirche*, Alst\"adter Rathaus**\\

May 12: Prag:\\
Prague Castle: State Apartments***, Prague Castle: Old Royal Palace***, Vitus Cathedral***, Loreto Monastery**, Strahov Monastery***

May 13: Prague\\
St Jacob**, Teynchurch*, St Nikolaus (Kleinseite)***, Palais Waldstein**, Bethlehem Chapel

\section{May 27: Bernese Lakes}
\label{Brienz2012}

Lake Brienz**, Lake Thun**, Giessbachfalls***

\section{May 31--June 3: Loire Valley}
\label{2012Loire}

May 31:\\
Chateau Chenonceau***\\

June 1: Blois:\\
Blois: St Vincent \& Paul*, Chateau Blois***, Cathedrale*, Notre Dame de la Trinite**, St Nicolas*, Chateau Amboise**, Chateau Chaumont-sur-Loire**\\

June 2:\\
Chateau Cheverny***, Chateau Chambord***, Orleans: Cathedrale**, Hotel Groslot*\\

June 3:\\
Vezelay: Basilica Maria Magdalene***, Abbey of Cluny**

\section{June 17:Valais}
\label{Valais2012}

Sion: Valere**, Leukerbad*, Saas-Fee: Mittelallalin**

\section{July 22: Leuck \& Leukerbad}
\label{Leuk2012}

Leuck: Church*, Leuk: Gemmiwall**, Daubensee**, Wildstrubelglacier**

\section{July 24-July 31: Berlin}
\label{Berlin2012}

July 25:\\
Schloss Charlottenburg***, Mausoleum*, Belvedere*, New Pavillon**, Schloss Sch\"onhausen**, Franz\"osischer Dom*, Konzerthaus***, Hedwigskatherdrale*, Brandenburger Tor**\\

July 26: Potsdam\\
Schloss Cecilienhof**, Marmorpalais***, Belvedere*, Schloss Sanssouci*** (Gem\"aldegalerie***, Neue Kammern***), Orangerieschloss**, Charlottenhof*, Neues Palais: Schloss Sanssouci**\\

July 27:\\
Schloss Glienicke*, Schloss Pfaueninsel**, Nikolaikirche*, Schloss Babelsberg*, Flatow-Turm*\\

July 28:\\
Schloss Sanssouci*** (see above and Chinese Tea House*, Belvedere*, Norman Tower*), Machine House**\\

July 29:\\
Dom***, Nikolaikirche*, Graues Kloster*, Marienkirche*, Airport Berlin-Tempelhof*\\

July 30:\\
Schloss Bellevue***, Bunderpr\"asidialamt*, New Museum***, Pergamon Museum***, Rotes Rathaus*, Sophienkirche*, Reichstag***\\

July 31:
Kaiser-Wilhelms-Gedaechtniskirche**, Old National Gallery*, Bode Museum*, Ephraim-Palais, New National Gallery**, Olympic Stadium*

\section{August 11: Argentiere}
\label{Argentiere2012}

How this trip started: Nice weekends in August are always great to hike and explore the mountains. Considering one of us had a rental car, choice was to go to Chamonix, close enough to return early enough for the big fireworks of the Fete de Geneve 2012.\\

Co-travellers:\\
Jacob (US-American), UCLA undergrad, Dan and Brant (both US-Americans), North Eastern undergrads, and Jesse (US-American) UC Riverside grad student. All of us worked together on a muon chamber construction facility. Jacob had a car to transfer from his apartment to the Prevessin CERN side.\\

Jacob just had a couple of experience driving stick shift, at some point he decided he had enough of that stop and go traffic jams, and switched with Jesse, who then took the driver's seat. Once we arrived at Argentiere, we made it to the top of Grand Montets, enjoying for once a magnificent cloudless sky. It was even possible to look down on the Mer de Glace, as well as Mont Blanc. Then we went down to the middle station of Lognan. From there it is a pretty easy path over to the icefall of the Glacier d'Argentiere. We had to pass by herds of mountain goats and sheep, though all seemed not to bother too much about hikers or were curious enough to lick hands etc. Once we reached the glacier, Brant and Jacob decided to walk on the ice, just like most of the other tourists did. By that time even at an altitude of about 2000 m it was still around 20 degrees, so thank god we all were in T-shirts or shorts. The hike back was even faster, so we were back in Geneva early enough to still get perfect spots for another great 45 min fireworks over the lake.\\

Glacier d'Argentiere***, Glacier des Grands Montets**

\section{August 18: Jungfraujoch \& Grindelwald}
\label{Jsungfrau2012}

How this trip started: Brant and Jacob wanted to do Canyoning in Switzerland. They found a pretty decent offer close to Interlaken and decided to spend the weekend there. We got a newcomer in Muon Chamber land -- Indara -- who also expressed interest to see some mountains. I thought it could be nice to get to Jungfrau again. Indara and I decided to go there, and later meeting up with Jacob and Brant for dinner and a train ride back.\\

Co-travellers:\\
Indara (Mexican), physics grad student, also part of the Muon chambers crew. Having not seen most of the mountains, glaciers, or snow she is very excited to see what mountains have to offer.\\

 We got ourselves day tickets for the demi-tarif on swiss trains and decided to meet at the train station. I arrived there 15 minutes early, and waited, and waited, and waited. No Indara to be seen anywhere. I then realised that we forgot to switch phone numbers. So I thought well i go ahead and let Eric (my grad student) know what i went ahead. After all he was smarter to exchange phone numbers with any newcomer. Indara had the same idea and 20 minutes later we were in contact: the following had happened -- Indara wanted to park her car at CERN, but then the guard gave her car a very thorough check, when entering CERN. Thus she missed the tram, and arrived about 25 minutes later than planned, so she jumped on the next train. I decided that we would could meet up by the middle station of Kleine Scheidegg, which has great views, where I could also grab stuff to eat before moving on. Switching to the cog rail at Lauterbrunnen, Indara calls again which train to take at Interlaken. I tell her to either take the one to Grindelwald or Lauterbrunnen, in fact it is the same train -- the front part goes to Lauterbrunnen, the back part to Grindelwald -- I prefer to take the one to Lauterbrunnen. That valley is far more breath-taking than the boring ride to Grindelwald. Thus Indara decided she wants to have that view too, but messed up and ended up in the Grindelwald part. Unfortunately she was not fast enough to switch between the parts, thus half an hour more of waiting time. Anyways I don't mind taking more time eating my R\"osti at Kleine Scheidegg, at this point, we would skip the originally planned hike from Pfingsteg to B\"aregg and just stick to Jungfraujoch. Finally we met up, and onwards to the mountain top. Of the stops along the way the Eismeer station offers the best experience, with an overlook of the Grindelwald-Fieschergletscher and the Eismeer in the distance. Up on Jungfraujoch we enjoyed the glacier cave, and the views of the Aletschgletscher. We walked a bit on the trail, and Indara decided to try sledding for the first time in her life. An area was set up for tourists to book sleds, and so she did. Once she sat on it, she wasn't sure anymore if it was a good idea, so I gave her a gentle push. Now 20 mins later she decided she had enough of it, and we took the cog rail down to Grindelwald. There we decided to take the cable car up to First, to enjoy a better view of the Upper and Lower Grindelwald glaciers. And while we were up there we treated ourselves with a giant cup of ice-cream. Then we made our way to the youth hostel of Interlaken, where we met up with Brant and Jacob. There we also met their friends Kate and Stan, who worked at UN organisations in Geneva. After a snack for dinner, we enjoyed a parade of cows and took the last trains out of Interlaken, arriving after midnight in Geneva.\\

Jungfraujoch***, First***

\section{August 26: Meersburg}
\label{Meersburg2018}

Neues Schloss Meersburg**, Altes Schloss Meersburg**, Unteruhldingen: Pile Dwellings**, Birnau**

\section{August 30--September 9: Portugal}
\label{Portugal2012}

August 30: Lisbon\\
Santo Antonio da Se*, Se Cathedral**, Church of de Craca**, Sao Vincente de Fora**,  Panteon Santa Engracia*, San Martires*, Church da Encarnacao*, San Loreto, Sao Roque***\\

August 31:\\
Palacio de Mafra***, Palacio de Queluz***, Sintra: National Palace**, Palacio de Monserrate**, Quinta de Regaleira**\\

September 1:\\
Sintra: Palacio Pena**, Castello di Mauro*, Lisbon: Sao Roque***, Santa Catarina*, Sao Domingos**\\

September 2: Belem\\
Palacio de Ajuda***, Monasteiro Jeronimo***, Monument of the Explorers**, Palacio de Belem**\\

September 5:\\
Technology Museum*\\

September 6:\\
Tomar: Convent***, Lisbon: Palacio do Conde d'Obidos*\\

September 7:\\
Boca do Inferno***\\

September 8: Porto \& Coimbra\\
Porto: Se Cathedral**, Church dos Grilos*, Palacio do Bolsa***, Sao Francisco***, Church dos Clerigos**, Church dos Congregados*, Coimbra: Monastery Santa Cruz**, University**, New Cathedral*, Old Cathedral**\\

September 9: Lisbon\\
Estadio da Luz**, Basilica de Estrela**, Oceanarium**

\section{September 16: Grimsel}
\label{Grimsel2012}

Sidelhorn***, Unteraargletscher***, Oberaargletscher***, Rhonegletscher**

\section{September 28--October 1: Poland}
\label{Poland2012}

I wanted to visit Poland and particularly Warsaw already for quite a couple of year. My family wasn't so keen on a trip to Poland around New Year's (most probably rightfully so). So I finally decided that I should try it late September. My boss agreed that it would be fine to leave my hardware test stand in the hands of my grad student Eric who has shown some good work already. Once I told Eric I would go to Poland and I thought he would be now ready to take over operating the test stand for two days on his own, he asked me if he could rather join. I was obviously fine with it, and so was our boss and that's how I started taking friends and colleagues on my trips.\\

Co-traveller:\\
Eric: Japanese-American, physics grad student at UCLA. Eric had already seen a bit of Europe before during an exchange semester in Budapest. He is clearly up to see more of Europe, so off we go.\\

September 28: Warsaw:\\
We flew to Warsaw via Vienna with Austrian airlines. I checked in a suitcase, Eric preferred to have everything on him. In Vienna Eric got himself a book and once we arrived in Warsaw my suitcase had been lost, in fact it was in Catania. I was promised it would arrive a day later, but then it only made it to Rome and then all traces had been lost from thereon. Besides having to get myself a couple of new underwear my battery charger for my camera had been lost too, so unfortunately I had to make sure that it would last until the end of our trip. And so it did indeed. Thus after about 45 minutes of an delay we finally got out of the airport. We did go to the Lazienkowski Park, but due to our time constraints we didn't see all of the pavilions and little palaces which we wanted to see. We started with the Palace on the Water, a classicist palace, which was rebuilt after it was burnt down by Germans in 1944. The Solomon room, as well as the ballroom, and the bathing room are still impressive, but most of the frescoes as well as the gold leave decorations are lost. The Myslewicki Palace is an nice classical palace with several nicely decorated rooms, it had been used as Poland's guest house during communist times. Our hotel was close-by the Palace of Culture and Science, a ``present'' by Stalin to the Polish people's republic. Unfortunately the roof terrace had closed down, but the Foyer was still nice to see, as well as the colourful lighting of the facades.\\

Lazienkowski Park: Palace on the Water***, Myslewicki Palace**, Palace of Culture and Science**\\

September 29: Krakow\\
Wawel Castle***, Cathedral***, Dragon's Den**, Berhardine Church**, Mary's Church***, St Francesco*\\

September 30: Warsaw\\
Early in the morning we took the bus to the Wilanow Palace, the Baroque summer residence of the Polish kings, and one of the few palaces not to be destroyed by the German army. The royal apartments are richly decorated and nice to see, particularly the state bed rooms, as well as many paintings in several galleries. Some paintings of the White Hall have been lost though. The French gardens are nice as well. On our way back we realised that the bus didn't follow the typical paths at all, due to the Marathon taking place on this day. Thus we got off somewhere, but not close to old town where we wanted to get to actually. We had lunch there, and then got on the metro, which was not affected by the ongoing Marathon. The royal castle was completely burnt down by Germans in 1944, reconstructed in 1971-1984 according to photographs done shortly before outbreak of World War II.\\ 

Wilanow Palace***, Royal Palace***, Jesuit Church*, St John's Cathedral**, Mary's Church, Holy-Ghost-Church*, Cathedral of the polish Army**, St Anna**, Carmelite Church*, Cathedral St Michael \& St Florian*\\

October 1:\\
Museum of the Warsaw Uprising**, Lazienkowski Park: White House*, Old Orangerie*

\chapter{Year 2012}
\label{2012}

\section{February 11--February 26: Edit 2012}
\label{2012:EDIT}

A new year a new challenge: my boss decided that it would be nice for me to get a hands on overview on a multitude of detector technology, something provided by the EDIT school which is a two week long school to get young physicists interested and educated in current and past technologies to detect particles. This 2012 edition it would be hosted at Fermilab. Realising that my passport was close to expiration I exchanged it just shortly before Christmas time, and it did arrive just in time for this visit (reminder to my 2021 self, not to forget to extend my current passport as quickly as possible). We all stayed at one resort with a large bar, where we enjoyed long nice discussions (after eating at one of the many large restaurants in the vicinity). The resort had also two indoor pools which we made heavy use off too.\\

February 12: Fermilab\\
During the school we had the opportunity to see some of detectors at Fermilab, among those the large neutrino experiment of Minos, and the two multi-purpose detectors of the Tevatron, which just had been decommissioned a couple of months before: CDF and D0. While D0 was stilled closed and taking cosmic data, CDF had already been opened which gave us unique views of its interior.\\

Batavia: Minos****, D0*****, CDF*****\\

February 13: Femilab\\
I put that data as random, in fact on all days of the EDIT school we had been at Wilson hall. The hall was inspired by the large French gothic cathedrals, it particularly has a setting where the building gets smaller the higher it gets. Thus it creates the feeling of the high naves of French gothic churches.\\

Batavia: Wilson Hall****\\

February 18: Chicago:\\
When in Chicago you go to downtown and the loop. This time I started out the day by Field museum with its giant Tyrannosaurus Rex Skeleton and the remains of old Mayan towns. Then Jingtuan, Ina, and I walked to John Hancock Center enjoying the Chicago skyline from the observatory before going over to Chicago's cathedral, going alongside the Chicago river and its art deco skyscrapers. Once we arrived by Willis tower we were told it might take up to 2 hours to get up, but we only had about 2 h time to get back to the Limousine all of us ordered to get us back to the resort. The others didn't want to risk it but I decided to go for it to get some nice night views out. Once I got in to watch the movie in the waiting hall i placed myself by the exit, thus I skipped already a bit. Then we were offered the opportunity to get up earlier at the expense to walk up the last 10 floors. I obviously volunteered and one lift switch later mid tower I was up there walking the last 10 floors up. Instead of 2 hours it took me about 40 minutes to get up after all. Thus enough time to get night photos out of the sea of houses up to the horizon (Greater Chicago is gigantic, maybe not as large as Greater LA, but still). Anyways I got down early enough joined the others who had a large dinner, had just a snack instead but I clearly made it on the limousine too AND got my photos - so win-win after all.\\

Chicago: Field Museum****, Willis Tower*****, Cathedral***

\section{A new camera: an DSLR Canon EOS 600D}
\label{canon600D}

After all years with my faithful companion Olympus Superzoom I realised its limitations particularly for interior photography and in the white balance settings, which just had become outdated at that point. Whenever inside I needed a stable surface, and tripods were not allowed in many places either. Thus it was time to retire my first digital camera and go for a larger DSLR. Since my brother had already a Nikkon I got the complimentary Canon and the model I thought would be right for myself, and EOS 600D.

\section{May 11--May 13: Prague}
\label{2012Prag}

My first trip with my new camera. I tried to get people on board for this trip, but there was no interest. Thus I asked my parents, who toured northern Germany the week before. Since we still had our coupon from Bamberg for a free overnight stay, my parents opted to use that before moving to Prague already on May 10, moving into the private large apartment we rented for three nights.\\

May 11: Karlstejn \& Prague\\
Once I arrived by the airport the sole of my shoe started to disintegrate. I had not used them in the winter months and something seemed to have happened to their structure. I had swiftly to change to my second pairs of shoes instead and wear those for all the remaining days. My parents picked me up at the airport and we drove to Karlstejn.\\

The castle of Karlstejn was listed as one of the most beautiful medieval castles in central Europe in a couple of books I owned as kid. I also had a toy paper castle version of it which I had put together years ago. There are several tours you can participate in. The royal palace of Emperor Charles IV from the house of Luxembourg is interesting, but most pieces of furniture on this tour have been put there from other castles, since hardly anything had survived the times. We were also told that most towers, including the large belfry had undergone quite some restoration effort in the 19th century, one of the reasons why the castle itself has not been added onto the UNESCO world heritage list despite the Czech Republic trying to achieve that. The highlight of the castle can be seen on another tour though, leading through the former Imperial chapels. Those are really amazing with medieval frescoes, particularly impressive is the upper golden chapel, which once used to house the Bohemian Royal Crown. These three chapels are really outstanding, unfortunately back then photography wasn't allowed anywhere.\\

And then we drove back to Karlstejn, where I dropped my suitcase by the apartment, had a short coffee break in the kitchen, before having a walk through old town. We started out by the old town hall with the astronomical clock, then had a short look into the close by Baroque churches of Nicolas church and Francis of Assisi, both with nicely painted domes. Then we crossed over Charles Bridge, walked around Prague castle and then had dinner on this part of the river.\\

Karlstejn: Burg Karlstejn****\\
Prague: Church Francis of Assisi***, Nicolas Church***, Alst\"adter Rathaus****\\

May 12: Prague:\\
Prague: Castle: State Apartments*****, Prague Castle: Old Royal Palace*****, Vitus Cathedral*****, Loreto Monastery****, Strahov Monastery*****

May 13: Prague:\\
Prague: St Jacob****, Teynchurch***, St Nikolaus (Kleinseite)*****, Palais Waldstein****, Bethlehem Chapel*

\section{May 27: Bernese Lakes}
\label{Brienz2012}

Lake Brienz****, Lake Thun****, Giessbachfalls*****

\section{May 31--June 3: Loire Valley}
\label{2012Loire}

May 31: Chenonceaux\\
Chenonceaux: Chateau Chenonceau*****\\

June 1: Blois, Amboise \& Chaumont-sur-Loire:\\
Blois: St Vincent \& Paul***, Chateau Blois*****, Cathedrale***, Notre Dame de la Trinite****, St Nicolas***\\
Amboise: Chateau Amboise****\\
Chaumont-sur-Loire: Chateau Chaumont-sur-Loire****\\

June 2:  Cheverny, Chambord \& Orleans\\
Cheverny: Chateau Cheverny*****\\
Chambord: Chateau Chambord*****\\\
Orleans: Cathedrale****, Hotel Groslot***\\

June 3: Vezelay \& Cluny\\
My mum and I had discussed a couple of options where we could stop on our way back, one of those was the cathedral of Bourges (took me in the end seven more years to get there). In the end we decided to do for sure the Basilica of Vezelay and then maybe more afterwards, since this place was more remote than the other options. In Vezelay is the tomb of Mary Magdalene according to legends. Wether that is true or not the church itself is a beautiful large Romanesque basilica with some of the best conserved early medieval vaults. The stone carvings in the entrance hall are outstanding too.\\
Since it wasn't that late still we continued our tourist program with a visit of the remains of the abbey of Cluny. Once the largest monastery in the world with the back then second largest church in the world during revolutionary times most of the basilica was destroyed, while the monastery itself is still there. Only one half of the one of the two original transept survived destruction, and is now the parish church of the village. Other remains, columns, pillars, mosaics, portals, statues, can be found in the museum, as well as remains of the ground walls of the basilica, which give you an idea how huge the building had been prior to destruction.\\

Vezelay: Basilica Maria Magdalene*****
Cluny: Abbey****

\section{June 17:Valais}
\label{Valais2012}

Starting out the day in Sion we did see the usual castles and the cathedral. Then we asked ourselves what we could do then. Since neither of us had been in Leukerbad we jumped on the bus, only to realise that even in mid June not a single chairlift or cable car was yet in operation. The cliffs and rocks around the village were though impressive enough. So instead we got on a bus again and a train and another bus later we arrived in Saas-Fee, where we took the cable car up the mountain to get some late afternoon views of the Feegletscher bei the Mittelallalin station. My brother was not too happy about the fact that we just had 45 minutes of time to spent on the peak station though.\\

Sion: Valere***, Leukerbad: village**, Saas-Fee: Mittelallalin****

\section{July 22: Leuk \& Leukerbad}
\label{Leuk2012}

This time around we knew that the cable cars of Leukerbad would be running. Since the bus to Leukerbad would only leave almost an hour later, we took a bus to Leuk first, saw the church there, the medieval town hall and the castle from the outside. In Leukerbad we took the cable car up to the Gemmi pass, in fact we could have hiked through the Gemmiwall as well, some brave folks even took the mountain bike down the very narrow path. Up on the high plain we walked down to the Daubensee mountain lake, enjoying views of the Wildstrubelgletscher as well. It was very chilly and cold up there, even with a little drizzle of rain. Then we walked on one of those stairs to nowhere over the cliff (not that I get why people are so amazed about those), before taking the cable car down again.\\

Leuk: Church**\\
Leukerbad: Gemmiwall****, Daubensee****, Wildstrubelglacier****

\section{July 24-July 31: Berlin}
\label{Berlin2012}

How this trip happened:\\
Finally I had no obligation to block July holidays due to possible exam correction or exam protocol days. It had been years since I visited Berlin, and I always wanted to see also the ``little'' palaces of the Prussian kings and German emperors, particularly since the UNESCO awarded them World Heritage status too. Thus I purchased the yearly ticket of the palaces of Berlin and Brandenburg. This would also cover the special exhibit on Friedrich II der Grosse of Prussia (unfortunately this also meant no photos in the New Palace of Sanssouci which hosted that exhibit). I also found out that it is in fact possible to visit the presidential palace of Bellevue (a bit hidden on the president's webpage), thus I inquired and was given a couple of dates, which then decided after all when I would visit the city.\\

Co-Travellers:\\
I didn't manage to get people from Geneva on board, so I tried to engage my friends from my physics studies at ETH to join me on this trip. Gianrico decided that Berlin is always worth a short trip and he would join me for a couple of days. He also suggested a Motel One hotel for the stay, which was very central and since I booked a couple of months in advance also not too expensive.\\

July 24: Berlin\\
I took a late evening flight. In fact it was supposed to leave at a different time and to land in Berlin-Brandenburg. But then we all know the story by now, just a couple of weeks before inauguration major problems with the fire extinguisher system were announced and the whole airport terminal had to be revised, and only almost a decade later the airport finally was opened. So old GDR style airport Sch\"onefeld it was for me instead. After arriving I just arrived in the hotel and then did a small walk over to KDW and Kaiser-Wilhelms-Ged\"achtniskirche which was hidden under tons of scaffolding.\\

July 25: Berlin\\
Originally I feared the mornings would be cold and then it would get too warm in the afternoons, but then already in the morning stepping out in long jeans I felt it was a bit too warm, so changed to shorts right away (right choice!). Anyways off to the very first palace of Charlottenburg. This is the largest palace in Berlin which wasn't completely destroyed in the war (or blown up like Berlin City Palace). Since this palace was used by multiple monarchs, it contains rooms from Baroque, Rokoko, and Classicist style. It used to be home of the Amber room too before it was given to the Russian Emperor as a present. I was particularly impressed by the Porcelain room and the palace chapel. Then there is the new wing which has a flight of rooms from the time of Friedrich II with lots of silver decorations. The park is of French Baroque style with a revival style mausoleum, the Belvedere which exhibits Porcelain nowadays, and the Neue Pavilion, built as leisure palace for the crown-prince of that era by Schinkel.\\
Then I moved over to the eastern part of Berlin, visiting the Schloss Sch\"onhausen. This palace had been the main residence of the wife of Friedrich II. The two didn't really have a harmonic marriage to say the least. Only a couple of rooms remain intact from this era, as the palace had been refurbished as the state guest house of the German Democratic Republic. Although this was done in a communist era, the rooms actual respect the appearance of the overall place, and it fits quite well with the other rooms.\\
And then back to Gendarmenmarkt, one of the most beautiful squares of Berlin with the classic style churches and the classicist Konzerthaus as main focus. The tour of the concert hall was quite informative, it is also a far more classical setting with a large organ in the main hall than the modern style building of the Berlin philharmonics. Thus depending on your taste you have the possibility to witness concerts traditional style or in a more modern environment. \\
I ended the day walking across the Lindenstrasse through the Brandenburg Gate with a short stop by Hedwigskathedrale, the catholic cathedral of Berlin. Largely destroyed in WWII, this baroque building was re-interpreted in a modern manner, but opening up the crypt to the main central rotunda. Now, by the begin of the 2020s a new renovation of the cathedral will change the appearance of the cathedral once again, closing down the crypt.\\

Berlin: Schloss Charlottenburg*****( Mausoleum***, Belvedere***, New Pavillon****), Schloss Sch\"onhausen****, Franz\"osischer Dom***, Konzerthaus*****, Hedwigskathedrale***, Brandenburger Tor****\\

July 26: Potsdam\\
Today I went to the so-called New Garden area, a vast area by beautiful lakes which had been set up by the late kings of Prussia for their private residences. I started the day with a visit of Schloss Cecilienhof. This palace was built in a tudor landhouse style for the last crown prince of the German Empire. Since it was actually built that late, electricity, elevators as well as modern type bath rooms have been set up for personal comfort right from the get go. There are two parts of the visits, one general one addressing the later fate of the castle, and a special guided tour only visit of the former private quarters of the royal couple with a visit of their living rooms, the bed rooms and the bathrooms on the upper floor of the castle. The castle has been the setting of the Potsdam conference where the four allied forces discussed what to do with Germany after the defeat of WWII. In fact some parts of the palace are more traditional classical style, one room is set up in style of a boat cabin. It feels at least more modest than usual royal palaces.\\

In the New Garden is also the classicist Marmorpalais, a summer palace built along the shores of a lake, with beautiful restored halls. Some of those had been only reconstructed lately. And then I walked up to the Belvedere on Pfingstberg and then over to Sanssouci. Since I had a yearly ticket I visited the New Chambers, the picture gallery and the main palace. The gallery is particularly impressive with walls covered in classical paintings. I did participate in tours of the classicist Charlottenhof \& Orangerieschloss with its main gallery, which houses copies of many famous paintings, particularly by Raffael. Last but not least I visited the large Friederisko exhibition in the New Palace, unfortunately this meant a ban on photography. But I got at least to see the upper gallery, the palace theatre, and the royal apartment itself, before all these place had been closed off for renovation (some are nowadays not part of normal tour itineraries anymore).\\

Potsdam: Schloss Cecilienhof****, Marmorpalais*****, Belvedere Pfingstberg***, Schloss Sanssouci***** (Gem\"aldegalerie*****, Neue Kammern*****, Orangerieschloss****, Charlottenhof***, Neues Palais****)\\

July 27: Berlin \& Potsdam\\
Berlin: Schloss Glienicke***, Schloss Pfaueninsel****
Potsdam: Nikolaikirche***, Schloss Babelsberg***, Flatow-Turm***\\

July 28: Potsdam\\
Potsdam: Schloss Sanssouci***** (see above and Chinese Tea House***, Belvedere***, Norman Tower***), Machine House****\\

July 29: Berlin\\
Berlin: Dom*****, Nikolaikirche***, Graues Kloster***, Marienkirche***, Airport Berlin-Tempelhof***\\

July 30: Berlin\\
Berlin: Schloss Bellevue*****, Bunderpr\"asidialamt***, New Museum*****, Pergamon Museum*****, Rotes Rathaus***, Sophienkirche***, Reichstag*****\\

July 31: Berlin:\\
Berlin: Kaiser-Wilhelms-Gedaechtniskirche****, Old National Gallery***, Bode Museum***, Ephraim-Palais*, New National Gallery****, Olympic Stadium***

\section{August 11: Argentiere}
\label{Argentiere2012}

How this trip started: Nice weekends in August are always great to hike and explore the mountains. Considering one of us had a rental car, choice was to go to Chamonix, close enough to return early enough for the big fireworks of the Fete de Geneve 2012.\\

Co-travellers:\\
Jacob (US-American), UCLA undergrad, Dan and Brant (both US-Americans), North Eastern undergrads, and Jesse (US-American) UC Riverside grad student. All of us worked together on a muon chamber construction facility. Jacob had a car to transfer from his apartment to the Prevessin CERN side.\\

Jacob just had a couple of experience driving stick shift, at some point he decided he had enough of that stop and go traffic jams, and switched with Jesse, who then took the driver's seat. Once we arrived at Argentiere, we made it to the top of Grand Montets, enjoying for once a magnificent cloudless sky. It was even possible to look down on the Mer de Glace, as well as Mont Blanc. Then we went down to the middle station of Lognan. From there it is a pretty easy path over to the icefall of the Glacier d'Argentiere. We had to pass by herds of mountain goats and sheep, though all seemed not to bother too much about hikers or were curious enough to lick hands etc. Once we reached the glacier, Brant and Jacob decided to walk on the ice, just like most of the other tourists did. By that time even at an altitude of about 2000 m it was still around 20 degrees, so thank god we all were in T-shirts or shorts. The hike back was even faster, so we were back in Geneva early enough to still get perfect spots for another great 45 min fireworks over the lake. But only Dan in fact chose to see the fireworks on my spot, whereas the others opted for a spot by Perle de Lac with UN folks. We did meet up later to celebrate further on.\\

Argentiere: Glacier d'Argentiere*****, Glacier des Grands Montets****

\section{August 18: Jungfraujoch \& Grindelwald}
\label{Jsungfrau2012}

How this trip started: Brant and Jacob wanted to do Canyoning in Switzerland. They found a pretty decent offer close to Interlaken and decided to spend the weekend there. We got a newcomer in Muon Chamber land -- Indara -- who also expressed interest to see some mountains. I thought it could be nice to get to Jungfrau again. Indara and I decided to go there, and later meeting up with Jacob and Brant for dinner and a train ride back.\\

Co-travellers:\\
Indara (Mexican), physics grad student, also part of the Muon chambers crew. Having not seen most of the mountains, glaciers, or snow she is very excited to see what mountains have to offer.\\

 We got ourselves day tickets for the demi-tarif on swiss trains and decided to meet at the train station. I arrived there 15 minutes early, and waited, and waited, and waited. No Indara to be seen anywhere. I then realised that we forgot to switch phone numbers. So I thought well i go ahead and let Eric (my grad student) know what i went ahead. After all he was smarter to exchange phone numbers with any newcomer. Indara had the same idea and 20 minutes later we were in contact: the following had happened -- Indara wanted to park her car at CERN, but then the guard gave her car a very thorough check, when entering CERN. Thus she missed the tram, and arrived about 25 minutes later than planned, so she jumped on the next train.\\
 I decided that we would could meet up by the middle station of Kleine Scheidegg, which has great views, where I could also grab stuff to eat before moving on. Switching to the cog rail at Lauterbrunnen, Indara calls again which train to take at Interlaken. I tell her to either take the one to Grindelwald or Lauterbrunnen, in fact it is the same train -- the front part goes to Lauterbrunnen, the back part to Grindelwald -- I prefer to take the one to Lauterbrunnen. That valley is far more breath-taking than the boring ride to Grindelwald. Thus Indara decided she wants to have that view too, but messed up and ended up in the Grindelwald part. Unfortunately she was not fast enough to switch between the parts, thus half an hour more of waiting time. Anyways I don't mind taking more time eating my R\"osti at Kleine Scheidegg, at this point, we would skip the originally planned hike from Pfingsteg to B\"aregg and just stick to Jungfraujoch. Finally we met up, and onwards to the mountain top. Of the stops along the way the Eismeer station offers the best experience, with an overlook of the Grindelwald-Fieschergletscher and the Eismeer in the distance. Up on Jungfraujoch we enjoyed the glacier cave, and the views of the Aletschgletscher. We walked a bit on the trail, and Indara decided to try sledding for the first time in her life. An area was set up for tourists to book sleds, and so she did. Once she sat on it, she wasn't sure anymore if it was a good idea, so I gave her a gentle push. Now 20 mins later she decided she had enough of it, and we took the cog rail down to Grindelwald. \\
There we decided to take the cable car up to First, to enjoy a better view of the Upper and Lower Grindelwald glaciers. And while we were up there we treated ourselves with a giant cup of ice-cream. Back in 2012 the tongue of the Upper Grindelwaldglacier still reached almost down to the valley, but the middle section had already melted quite a bit. In the late 2010s the lower part separated from the upper part of the glacier completely. The Lower Grindelwaldglacier was already on the verge of collapse int he early 2010s, now having reached 2021 it completely disappeared. So my recommendation wherever there's a glacier close to you make the effort to see it. I know many of us try nowadays to keep our emissions low, e.g. walking instead of taking the car for short distances. But unfortunately even if we would stop our carbon emissions right now it will be too late for glaciers to recover in our lifetime again most probably. Thus we have to enjoy them now before it is completely too late. And that's what we did enjoy ice cream and enjoy the view. Then we made our way to the youth hostel of Interlaken, where we met up with Brant and Jacob. There we also met their friends Kate and Stan, who worked at UN organisations in Geneva. in fact we might have overdone ourselves with that gigantic cup, once we were offered fries we only had a handful of those. After having that as snack for dinner, we enjoyed a parade of cows and the sunset views of Jungfrau, before taking the last train out of Interlaken, arriving after midnight in Geneva.\\

Jungfrau region: Jungfraujoch*****, First*****

\section{August 26: Meersburg}
\label{Meersburg2018}

I was home for a weekend once again, shortly before my first -- and only -- CMS week outside of CERN. Only recently the renovated New Castle of Meersburg was open for tourists, so I convinced my family to join me on that adventure. The large staircase is indeed impressive, the festival hall is nice to see as well, the private apartment of the bishop was smaller than I expected. Back then no photography was allowed and overall the explanations were not up to point shortly after opening. Thus a nice place to see but could do better I think. Then my mum and I visited the old castle, one of the oldest non ruined medieval castles in Germany, albeit with lots of modifications done during the times.\\
Since we were done a lot quicker than though in Meersburg we drove to the neighbouring village of Unteruhldingen. Here a lot of archaeological remains of pile dwellings have been found. Nearby a pile dwelling village has been reconstructed as open air museum, with displays of life back then happening from time to time as well. On our way back we stopped by the beautiful baroque Pilgrimage church.\\
Considering my 2020/2021 situation: IF covid19 wouldn't have happened this would have been a perfect affordable day to spend in my region, but even such things are nowadays a luxurious thing of the past.

Meersburg: Neues Schloss Meersburg****, Altes Schloss Meersburg****\\
Unteruhldingen: Pile Dwellings****\\
Birnau: Pilgrimage church***

\section{August 30--September 9: Portugal}
\label{Portugal2012}

August 30: Lisbon\\
Santo Antonio da Se***, Se Cathedral****, Church of de Craca****, Sao Vincente de Fora****,  Panteon Santa Engracia***, San Martires***, Church da Encarnacao***, San Loreto**, Sao Roque*****\\

August 31: Mafra, Queluz \& Sintra\\
Mafra: Palacio de Mafra*****\\
Queluz: Palacio de Queluz*****\\
Sintra: National Palace****, Palacio de Monserrate****, Quinta de Regaleira****\\

September 1: Sintra \& Lisbon\\
Sintra: Palacio Pena****, Castello di Mauro***\\
Lisbon: Sao Roque*****, Santa Catarina***, Sao Domingos****\\

September 2: Belem: Sunday\\
Co-Traveller for the day: David, US-American, a professor from UCLA and my boss. David decided to come to this year's CMS week as well. In order to get not into the worst jet-lag he arrived a day early and decided I should take him on a tourist program such that he wouldn't get too tempted to sleep in or get to sleep far too early. I clearly took on the challenge to have an interesting day set up for another person, it might have been a tad too much.\\

We started the day in the Palacio de Ajuda. After the palace in Lisbon had been destroyed in the large earth quake, finally a spot in Ajuda in the outskirts had been chosen as site for the new grand palace. But then in 1807 the Royal family had to flee to Brazil. That's also the reason why in Brazil you in fact have real royal palaces where indeed royalty lived and was not only represented by a governer. Only in 1821 the king returned to Portugal and only then the palace became a royal residence. When in 1910 the monarchy was abolished only a small part of the palace had been finished. Indeed until now you can see the unfinished walls of one of the wings. The state apartments which were already finished are decorated mainly in classical style, but also winter gardens with fountains or a Chinese style room can be found there. All in all definitely a palace I recommend you to see. \\
Then we visited the gothic Monasteiro Jeronimo in Belem. Besides the church with its small filigree pillars and elaborate vaults, the monastery is particularly famous for its highly decorated cloister considered to be the highlight of Portuguese gothic architecture together with the monastery of Batalha. Although California is famous for its sun, the climate is far more temperate than the Mediterranean. David already complained the heat was getting unbearable. We still went up on the monument of explorers. Besides the good view of Belems and its garden, and the famous tower of Belem, you also have literally the world at your feet, since a giant world map covers the square in front of the monument. Still completely exhausted and overheated David decided it was long time that we get dinner and particularly tons of water. \\
After having refreshed ourselves we went to the last stop of our tourist day, the Palacio de Belem, once a royal palace which is nowadays the presidential palace. The tour was given in Portuguese, the only word we understood was, you are allowed to take photos. Clearly I was happy about that and started to take photos. But then unlike we were told it was supposed to mean you CANNOT, obviously that one word changes the meaning to the opposite. Still interesting to see in what wonderful richly ornate rooms the President of Portugal can hold his meeting, quite the opposite compared to the cold white of the German presidential palace or the rather modest outfit of the White House. Once in the gardens we could indeed take photos. The garden is a very geometric French style gardens with little Pavilions for more private meetings. At the end of the tour David called it a day and went back to Lisbon, and even I agreed that it was a little bit too hot for my taste as well.\\ 

Belem: Palacio de Ajuda*****, Monasteiro Jeronimo*****, Monument of the Explorers****, Palacio de Belem****\\

September 5:\\
This day of the workshop ended with a reception by the technology museum in Lisbon. I was too enthusiastic using one of the pumps to produce big bubbles in a water column that I hurt my wrist that much, that the strain made it hurt for two months later when bent backwards. So be enthusiastic when in museums, but keep your excitement under control. Don't be me!\\

Lisbon: Technology Museum**\\

September 6: Tomar \& Lisbon\\
Once again a day full of meetings, and the afternoon focused on the heavy ion runs, which wasn't my field of study at all. Thus I decided to take the afternoon off, and by suggestion of a colleague I took the regional train to Tomar. In Tomar you have one of the largest old monasteries in Portugal. The convent was built throughout many centuries. Some parts, such as the old gothic chapter house, are nowadays in ruins, but most of the cloisters survived all troubles. The most beautiful of those multiple courtyards is the large Renaissance one constructed during the times, when Felipe II of Spain was also king of Portugal. The church itself is much older, the Rotunda houses a giant masterpiece of wood and stone carving with multiple coloured statues with impressive details. \\
And back to Lisbon for the workshop dinner in the Palacio do Conde d'Obidos. Not only the food was nice, but also the view of the harbour and the Christ Redeemer Statue on the other side of the river. I did enjoy the baroque library of the palace as well.\\

Tomar: Convent*****\\
Lisbon: Palacio do Conde d'Obidos***\\

September 7: Cascais\\
Cascais: Boca do Inferno*****\\

September 8: Porto \& Coimbra\\
Porto: Se Cathedral****, Church dos Grilos***, Palacio do Bolsa*****, Sao Francisco*****, Church dos Clerigos****, Church dos Congregados***\\
Coimbra: Monastery Santa Cruz****, University****, New Cathedral***, Old Cathedral****\\

September 9: Lisbon\\
Estadio da Luz****, Basilica de Estrela****, Oceanarium****

\section{September 16: Grimsel}
\label{Grimsel2012}

Grimsel: Sidelhorn*****, Unteraargletscher*****, Oberaargletscher*****,
Oberwald: Rhonegletscher****

\section{September 28--October 1: Poland}
\label{Poland2012}

I wanted to visit Poland and particularly Warsaw already for quite a couple of year. My family wasn't so keen on a trip to Poland around New Year's (most probably rightfully so). So I finally decided that I should try it late September. My boss agreed that it would be fine to leave my hardware test stand in the hands of my grad student Eric who has shown some good work already. Once I told Eric I would go to Poland and I thought he would be now ready to take over operating the test stand for two days on his own, he asked me if he could rather join. I was obviously fine with it, and so was our boss and that's how I started taking friends and colleagues on my trips.\\

Co-traveller:\\
Eric: Japanese-American, physics grad student at UCLA. Eric had already seen a bit of Europe before during an exchange semester in Budapest. He is clearly up to see more of Europe, so off we go.\\

September 28: Warsaw:\\
We flew to Warsaw via Vienna with Austrian airlines. I checked in a suitcase, Eric preferred to have everything on him. In Vienna Eric got himself a book and once we arrived in Warsaw my suitcase had been lost, in fact it was in Catania. I was promised it would arrive a day later, but then it only made it to Rome and then all traces had been lost from thereon. Besides having to get myself a couple of new underwear my battery charger for my camera had been lost too, so unfortunately I had to make sure that it would last until the end of our trip. And so it did indeed. Thus after about 45 minutes of an delay we finally got out of the airport. We did go to the Lazienkowski Park, but due to our time constraints we didn't see all of the pavilions and little palaces which we wanted to see. We started with the Palace on the Water, a classicist palace, which was rebuilt after it was burnt down by Germans in 1944. The Solomon room, as well as the ballroom, and the bathing room are still impressive, but most of the frescoes as well as the gold leave decorations are lost. The Myslewicki Palace is an nice classical palace with several nicely decorated rooms, it had been used as Poland's guest house during communist times. Our hotel was close-by the Palace of Culture and Science, a ``present'' by Stalin to the Polish people's republic. Unfortunately the roof terrace had closed down, but the Foyer was still nice to see, as well as the colourful lighting of the facades. But then we had drinks by the hotel while waiting to hear back where my bag was, but no success so more alcohol it was.\\

Warsaw: Lazienkowski Park: Palace on the Water*****, Myslewicki Palace****, Palace of Culture and Science****\\

September 29: Krakow\\
We wanted to have all the time we need in Krakow so we booked an early 6 am train. But our train was sad to be 30 minutes late, nope 60 minutes, nope 90 minutes. Finally it arrived and we were told that they had to swap the trains due to an engine failure. Anyways now having been late on arrival by about 2 hours we rushed over to the castle. Clearly the line to the castle tours was long, and we wanted to see both the standard and the private apartment tours. We still managed to get tickets for both. I didn't enjoy the fact that on the whole premise, on both tours and in the cathedral photography is forbidden (didn't change until 2020). Some remarks on tripadvisor state such that people won't see how disappointing the palace is. While the palace is not up there with Versailles, it does have some cute and nice moments, mainly tapestries, coffered ceilings. The cathedral is outstanding with many royal tombs in many different styles surrounding the gothic main nave. The cave below the castle once was occupied by the local dragon. Another highlight of Brick Gothic is the breathtaking Mary's Church. The most beautiful basilica in Poland I have seen up to 2021. Richly decorated with a masterpiece of gothic wood carving as high altar piece. The choir stalls are outstanding, and the blue ceiling with its golden star gives it a nice flair too. Then we had another good large dinner in one of the side roads by the market square.\\

Krakow: Wawel Castle*****, Cathedral*****, Dragon's Den***, Bernhardine Church****, Mary's Church*****, St Francesco***\\

September 30: Warsaw\\
Early in the morning we took the bus to the Wilanow Palace, the Baroque summer residence of the Polish kings, and one of the few palaces not to be destroyed by the German army. The royal apartments are richly decorated and nice to see, particularly the state bed rooms, as well as many paintings in several galleries. Some paintings of the White Hall have been lost though. The French gardens are nice as well. On our way back we realised that the bus didn't follow the typical paths at all, due to the Marathon taking place on this day. Thus we got off somewhere, but not close to old town where we wanted to get to actually. We had lunch there, and then got on the metro, which was not affected by the ongoing Marathon. The royal castle was completely burnt down by Germans in 1944, reconstructed in 1971-1984 according to photographs done shortly before outbreak of World War II. They did a really nice job with that. Then we walked through old town stopping here and there for many churches, all with detailed reconstruction work going into it. The polish army cathedral had though many modern style additions, but all of them done in a fitting tone. Then we walked over the large river and visited the revival style cathedral on the other side of the river. And then more time for a big dinner and tasting of several Polish vodkas in a bar afterwards.\\ 

Warsaw: Wilanow Palace*****, Royal Palace*****, Jesuit Church***, St John's Cathedral****, Mary's Church**, Holy-Ghost-Church***, Cathedral of the polish Army****, St Anna****, Carmelite Church***, Cathedral St Michael \& St Florian***\\

October 1: Warsaw\\
Poland was the country most hit by the Wehrmacht in WWII. One very good museum dedicated to the Warsaw Uprising illustrated the efforts normal local citizens undertook to fight the Nazis, albeit with devastating failed results. Since our trip to Lazienkowski Park had been cut short by my lost bag on the first day (which had been found in Catania, and got lost AGAIN on its way to Warsaw in Roma Fiumicino, got reimbursed for it later on), we made a second visit to the park, watching the White House and the Old Orangerie with the cute Baroque Court theatre this time around. And on our flight back we switched at Brussels airport, enjoying a Belgium beer in Belgium while we had the chance. In the end it took me until 2016 before I visited Brussels itself, but then my visits to Brussels stand at 4 all up to 2021.\\

Warsaw: Museum of the Warsaw Uprising****, Lazienkowski Park: White House***, Old Orangerie****

\chapter{Year 2005}
\label{2005}

\section{February:Diavolezza}
\label{2005:Diavolezza}

Each winter ETH Z\"urich offers the possibility to observe the sky for a week with fancy little telescopes under the supervision of experts to get practical experience on how to perform little projects. Many fellow students in my friends group were interested too. Sigi, Sibylle and I set up an observation plan, we learnt to get to know the telescopes previously as well. This year's observation point was Diavolezza, a mountain hut at almost 3000 m in the canton of Graub\"unden high in the alps. Chances for a clear sky are higher in winter, the Bernina massive is also far away from big cities, though only a distant light cone of Milan should spoil the sky. Most of the groups had projects during night time, while other projects studied the sun (thus day time projects). Anyway since we were quite a bunch of people who knew each other I set up a little group booking, which should have given us a nice ride in free seats etc -- or that was the plan. My dad got me to Zurich main station and there we realised that nothing would be as straight forward as hoped. While most of us were indeed around at the planned meeting point, the traffic control system of Zurich main station had major issues on this very day (and Sigi had a major cold, so he was out of the trip as well). Mind you Zurich is one of central europe's busiest train station, thus imagine the gigantic chaos when almost no train can be routed to go in and out as planned. Some of my friends were stuck for over an hour, although being merely a few km away from the central station (second biggest train chaos I have ever seen in Switzerland, where things go pretty smoothly almost always, the largest chaos being a failure of the whole power network for the full train system of Switzerland). Also we others were stuck as well, as almost no train left the station either -- not that we were too worried, as long as it would be possible to get up to the hut once we would arrive (and yes we were even interviewed by Swiss newspapers how we felt about it). Now at some point a train was leaving for Chur, and we jumped on it. From there train traffic was just working fine again, we (actually Frank) had to organise a cable car ride up since we would miss the regular hours for a cable car ride. Since it was not in our hands to arrive though thankfully SBB (Swiss Train company) agreed to take over the typically extra fee for the out of hours ride.\\
 Once we arrived in the hut we were split up into the hostel rooms, had dinner and a first view at the sky, which was though a bit cloudy on that night, thus not ideal for observations. The hut was very cute, Diavolezza offers a very popular skiing piste as well as superb views of the Bernina massif (highest peaks in the eastern alps) and the Pers and Moreratsch glacier. Being a non-skier myself I rather enjoyed the mountain scenery (and sleeping in on mornings, since the observations were going on at night). During day time we had set up other telescopes for the sun observations, which we also showed to other tourists after all it is great to convince others (also young kids coming for skiing) that astrophysics is a pretty great thing to do. I also learnt about all playing modes of the traditional Swiss card game Jassen (the one we played at home in Germany didn't include all options Swiss people came up with).\\
 Other unexpected issues popped up during our observations run - it is pretty cold high up in the alps at night (no shit -- not THAT unexpected I know). At about - 20 degrees C some mechanism to automatically move the telescope to tackle earth's rotation froze in, and laptops went on strike as well. Thus we had to construct heating devices (aka enclosures which we heated up using hot water packets, which we had to exchange regularly) and we used hair driers to blow hot air at the mechanics of the telescope on a regular basis as well. Thank god the air was dry, thus the other end didn't freeze over with a film of ice like your windshield might do in winter. The next couple of nights were pretty clear, at one point we also saw the ISS pass over us (quite awesome if you ask me). Since we had to deal with these telescopes in a manner that we didn't expect to be that elaborate and two nights were lost to none-clear skies, some groups -- including mine -- merged with others who had a similar project in mind in order to distribute the tasks and speed up things. 
 %something about the project, brightness curve
 Later we found out that although we were convinced to have caught the star constellation we wanted to study we had in fact not but studied another double-star system close-by. Thus it was not the ideal conditions for the curves as we had hoped for, but it was still doable (tough luck).\\
 What happened otherwise: the showers broke for a few days and we had to use the sink (plumbing doesn't get easier up there than down in the valley). Gianrico visited as well over the weekend and we hiked over to a nearby mountain, my first time crossing the 3000 m altitude (Germany's highest mountain peak -- the Zugspitze -- is only 2963 m or 2964 m, depending if you ask Germany or Austria). By now (begin of 2020) I crossed 3000 m a couple of times, in Switzerland, France, Italy, as well as in the US). Swiss alps are always very cute and particularly magical in wintery white snow coverage, nevertheless also very impressive in Summer, when you can still admire all the glaciers (please do that while you can, in 50-100 years from now ALMOST all of them will be sadly gone). And then it was time to get back home. For the train ride from Diavolezzo to Chur we decided to take the route of the R\"athische Eisenbahn (Raethian Railways) which has been included into World Unesco Heritage since - with many tunnels and viaducts crossing canyons and valleys. This time all trains were running on their regular schedule.\\
 
 Diavolezza \& Bernina Massif with Pers and Morteratsch Glaciers***, R\"athische Eisenbahn**

\section{March 20--March 28: Egypt}
\label{2005:Egypt}

Getting to Egypt has been a dream for me and my parents for quite some time. In my last years at highschool I wondered if I should inscribe for archaeology, but the dire job situation made me reconsider. How I managed to end up in particle physics instead with uncertain job aspects instead still remains a mystery. Anyway finally my dad decided that we should tackle the organisational matters and go for it. He found a tour provider for Egypt in Zurich with who he set up an itinerary. We had to hand in our passports to check visa regulations. It was a bit chaotic but then we were ready to go. This time it was my parents, my younger brother and I.\\

March 20: Arriving in Cairo:\\
Having arrived in Cairo we met our contact, who welcomed us and handed over our flight tickets and our tickets for the Nile cruise. But something didn't really work out. We were supposed to fly from Cairo to Aswan and back from Luxor to Cairo, but the boat was going the other way, from Luxor to Aswan. Something for sure was messed up. A couple of calls later we were told that something went wrong with the inner Egyptian flights and in a couple of hours we would get the proper tickets, thus we decided to walk around Cairo for a bit. Our hotel was located just next to the Hilton Ramses directly by the river nile and its island. Back in pre google times we had to rely on a rough city map we had received at the hotel and we chose to explore a bit the Nile Island. Which turned out to be a nice walk also with a view of Tower. Once we came back from our walk, the new flight tickets were awaiting us.\\

Nile Island**\\

March 21: Pyramids and the Egyptian Museum:\\
Our guide took us first to the graveyard of Sakkara. Here we saw the complex of Djoser's pyramid, built in several steps it is the first of the large Egyptian pyramids. Sakkara was located south of the ancient capital of Memphis. Not only Djoser's pryamid is located there, but we saw also the pyramids of Userkaf and Unas as well as the walls where old boats had been buried with ancient treasures. After Sakkara we moved on to the necropolis of Gizeh with the famous large three pyramids from Chufu (also known as Cheops in Germany), Chephren and Menkaure. We walked around all of the pyramids also sitting one one of the stones just in front of the big pyramid. Our guide took us inside one of the queen's pyramids which are located next to the Pharaoh's pyramid. Chufu's pyramid is the largest but Chephrens pyramid is built on a slightly higher plateau, thus it appears to be a bit higher. On top of that the polished surface of the latter is still conserved. One of Cairo's later rulers tried to destroy the pyramids starting with the smallest one of Menkaure, he didn't get very far but a big hole can be seen about midway up where stones had been removed. Although the Sphinx of Gizeh is gigantic it just pales in comparison to the size of the pyramids which completely dominate the view. Not to forget that this was my very first time in a desert. It was very impressive to see nothing but sand and rock as far as your ice could see. After our visit by the pyramids we headed back to our hotel which was just a couple of metres away from the Egyptian Museum. It is absolutely amazing what sculptures, gift, treasures etc are housed in this rather tiny building. The whole floor and foyer were full of art of thousands of years. And the out of this world gilded chamber of Tunankamun with the sarcophagus and the gold mask. Even photographs don't do justice to the amazing incredible piece of world history you can find here.\\

Pyramids of Sakkara***, Pyramids of Gizeh***, Egyptian Museum***\\

March 22: Karnak and Luxor:\\
We got up very early on to make our flight to Luxor. When our flight was called we realised that our tickets had a different colour than most other passengers, and indeed we were told not to follow everybody else but move to the left. We got our own private van bringing us to our Air Egypt flight. Seems the mess up with our tickets lead to us flying in Business class on this flight. Unfortunately it was just a bit less than an hour. But we tasted all nice pastries and jam and fruits and juices which we were served. After all it didn't happen to me that often to be upgraded to Business at least up to now. Once we arrived in Luxor we met our guide for the Nile cruise part of this trip who brought us first to our ship. And then we went to the largest temple of Egypt, the Amun temple in Karnak. After an alley of sphinx, a forrest of columns appears with lots of depicted religious and battle scenes, some even with their colour preserved. There is a vast area of courtyards and a little lake behind the main halls, where excavations are still going on. The second large temple of the region is located in the South of Luxor, both temples are thought to have been once fix points of a procession street through the old Egyptian capital of Thebes. The temple of Luxor has a large hall with columns as well, inside the temple is also an old mosque, and a lot of colossal statues.\\

Temple of Luxor***, Temple of Karnak***\\

March 23: Valley of the Kings:\\
Crossing over the other shore of the Nile, we visited the valley of the kings and one of the tombs of a Pharaoh, which was very richly decorated. It might have been the tomb of Seti I. Afterwards we stopped by the giant mortuary temple of Hatshepsut, one of the view female Pharaohs, which rised up from the desert in terraces leading to the main chamber. Nearby another temple is being excavated right now. On our way back to the ship we passed the temple of Ramses and stopped by the so-called Colossi of Memnon, the giant status of Pharaoh Amenhotep III in front of his temple, which was destroyed in an earth quake. And then we got on the boat and the cruise started. We spent the afternoon playing cards on the sun deck, the pool was a tad small. In the evening we passed a lock where already other ships had lined up ahead of us.\\
  
Valley of the Kings***, Temple of Hatshepsut***, Colossal statues of Amenhotep III**\\

March 24: Edfu and Khom Ombo\\
One of the late large temples was built in Edfu during the Ptolemaic period. Already the front depicts battle scenes. The temple is in a superb shape, but the frieze, sculptures, and paintings are less colourful, but still very impressive to see. And on to the boat again, where my mum decided to give the tiny pool a chance. But there wasn't much swimming to do, due to it only being about 10 m long. It was already dark when we arrived by
 the double temple of Khom Ombo. Brightly illuminated blue seemed to be the dominating colour of the ceiling decorations. It was a real nice experience visiting the sight by night. Coming back to our cabin we realised that our towels had been folded into Elephants, butterflies, and crocodiles, very cute and nice to look at.\\
 
 Temple of Edfu***, Temple of Khom Ombo***\\
 

%25 Aswan Temple
%26 Staudamm, bootsfahrt
%27 palast bei zitadelle Zitadelle

Citadel***
Aswan Temple of Isis***

\section{August 28--September 4: Lisbon}
\label{2005:lisbon}

%28:
% grosser platz: strand/hafen mit Bruecke und Christus im hintergrund
%29
%Castelo/Le Seo
%30 Belem, Weltkarte
%31 Aquarium
%01 Queluz & Sintra
%02 Stadion, Altstadt
%03 Christus
%04 heimflug

Castelo de Sao Jorge*, Monasteiro de Geronimo***, Tower of Belem**, Aquarium***, Estadio Jose Avalade**, Estadio da Luz*
Palacio de Queluz***, Palacio de Sintra**
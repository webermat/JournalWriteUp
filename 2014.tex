\chapter{Year 2014}
\label{2014}

\section{April 18--April 22: Bavaria}
\label{2014:Germany}

Co-travellers: \\
Eric:\\
After Eric was back from his semester in Los Angeles, it was time to go on a trip again. What could be better than showing him one of the beautiful regions of Germany. Little did he know that a year later he would relocate to Garching.\\

Indara:\\
Not only Eric is back, but Indara as well. After we did a couple of trips to France and Switzerland, Indara is ready for a longer trip, provided that the itinerary includes shopping or coffee time.\\

How this trip came into existence:\\
Bavaria is what most international tourists think of first, when asked about Germany. Since I had been in Bavaria multiple times, I felt confident enough to set up a nice list of places to see. Since this weekend was over Easter holidays, we just had to take 1 day off work for a 5 day little tour. And there is no speed limit on the Autobahn - that's a thrill for all drivers as well. I more or less pressure talked Eric into that trip -- now knowing that his alternative would be visiting his girlfriend in Boston -- only for two nights, but that's better than nothing. Thank god later on I did make up for it, by planning another good trip for the three of us, but she told me she was pretty mad on me for keeping him for my trip weekend.\\

April 18: Neuschwanstein in a snow storm:\\
Starting the day with the earliest flight to Munich. Having two drivers is always a good idea. Since both Indara and Eric were excited to drive in a place without a speed limit, it was clear who would be both drivers. We originally reserved a VW Golf type of car. Since they were out of those, we got a Mercedes B class instead. Eric wasn't displeased. And off we went deep south to Schwangau.\\
 Although we only would be in Bavaria for five days, our plan included to visit quite a few palaces, so we got the two week palace tickets by the Bavarian palace administration (one single, and one couple version). It gives you access to almost all important Bavarian palaces and castles (also some monasteries) for only 24 EUR. We still opted to reserve Neuschwanstein tickets, since without a reservation it can happen that you have to wait more than 4 hours to get inside, or even for the next day (so a well invested 3 EUR reservation fee). Hohenschwangau is still owned by the former Bavarian Royal family, thus you still have to pay for that one. So Hohenschwangau was our first stop (no photos). Rebuilt by Maximilian II of Bavaria as hunting palace, it is built in a neo gothic style with lots of frescoes in most of the rooms. Some of which were restyled by his son Ludwig II later on. Not as grandiose as Neuschwanstein, but a finished interior and still quite cute. After touring Hohenschwangau we went up to Neuschwanstein. I hoped we would have a good view of the castle from the bridge over the P\"ollat gorge, but we ended up in a snow storm. Thank god our tour started soon after and we were in a warm place, and a nice place, not as crowded as German speaking tours a couple of years earlier. Once we were done with Neuschwanstein, we drove over to the Lechfall and the Lechklamm, and checked in. Then we had a really nice dinner at Gasthaus Krone, and afterwards ice cream. Since we shared a room, Indara decided to clean up the mess we made while unpacking.\\

Schwangau: Hohenschwangau****, Neuschwanstein*****\\
F\"ussen: Lechfall****, Lechklamm****\\

April 19: Palaces, Churches, and Dirndls\\
We started the day early driving over the snow-covered Autobahn to Kempten. We first had coffee at a market stand, and then started out in the Basilica. A really large, maybe even over dimensioned church for Kempton, built in baroque style. Then we continued with a guided tour of the Residenz. Unfortunately it had to be in German, the leaf lets in English were pretty sparse, and the fellow guests (only two) complained that we would be too loud, or take too many photos (which is allowed). The tour encompasses the seven rooms of the state apartment, with the audience room and the throne room as highlights of late Bavarian baroque. The other rooms are nowadays used for local administration.\\
Then we had to deal with Indara's main point of the day -- Dirndl shopping. Closeby of the Residenz is a Dirndl shop, where modern versions of the traditional Bavarian dresses can be bought (not as overpriced as in some shops in Munich). In the end we had to decide between a more classical blue checkered dress, and a more modern dark green dress. And we decided to go for the more fancy modern version (after revisiting each of the dresses several times again, combining it with appropriate belts and aprons). By that time I realised I didn't have my mobile phone in my pocket. The previous day I had given it to Indara, since her CERN phone died and my phone's battery was still fine. She gave it back to me at the end of the day, but it was gone. Notifying the hotel, they claimed the room was empty, only to find the phone over two months later in one of the wardrobe drawers (and sending me the phone card back to me).\\
Anyways the next stop of the day was Wieskirche -- the highlight of Bavarian Rokoko, standing in the middle of nowhere among green meadows. It was my fourth visit -- and each time i am amazed about its beauty again. The most beautiful of Ludwig's II castles is Schloss Linderhof -- also the only palace which was finished during the King's lifetime. The park is large, the setting in the alps is amazing; even the park houses - in Moroccan and Moorish style. The king loved peacocks, thus many statues of those can be found throughout the palace, the main rooms are in blue colors, also the king's favorite. The dining room has a table which can be lowered to the lower floor (similar features can be found in Neuschwanstein too). The king was a huge fan of Richard Wagner, thus a grotto with a Swan Boat and an artificial waterfall was constructed illuminated in multiple colors while playing Wagner's music to entertain the king. The large fountain operated by a natural gradient pressure, pushing it up about every hour on the hour. Quite close to the Schloss is the famous monastery of Ettal, check out the monastery church if you have time.\\ 
The final stop of the day was Garmisch-Partenkirchen, the host of the Winter Olympics in 1936. The ski jumping hill has been renovated multiple times, the last time in 2007. A few minutes later the entry to the slot canyon of the Partnachklamm starts, one of Germany's most famous Canyons. We were all very impressed, but then we still had to make our way to the hotel in Rosenheim, where we ordered the Bavarian version of sausage salad with Apfelstrudel.\\

Kempten: Residenz****, Basilica St Lorenz****, Steingaden: Wieskirche*****, \\
Ettal: Schloss Linderhof***** (Venusgrotte*****), Monastery****,\\
 Garmisch-Partenkirchen: Ski Jumping Stadium***, Partnachklamm*****\\

April 20: Barbara oh Barbara\\
Today the largest lake of Bavaria was our main destination - the Chiemsee. Getting on a lovely little boat -- Barbara -- we set over to the larger of two islands in the lake -- Herreninsel. On this island Ludwig II built a replica of Versailles together with a vast French garden with many baroque style fountain, including a copy of the Latona fountain of Versailles. The centre piece of Herrenchiemsee is the Hall of Mirrors - slightly larger than the original in Versailles, more in a golden tone than the original blue. The state bed room is in fact the most expensive room built in the 19th century. Since Ludwig II died by committing suicide in a lake, the palace was left unfinished, the second stair case is still a brick shell. The side wings have not been realised as well. Afterwards we walked over to the old palace, in fact an old monastery. At this place Germany's constitution had been formulated back in August 1949. And back to our car, once again with Barbara.\\
Now we continued our trip to Landshut. There we visited first the Residenz, the old part being heavily inspired by Italian Renaissance Palazzi. The inner decoration is heavily reminiscent of Italian stucco ceilings, including the frescoes. The later king Ludwig I studied in Landshut, during this time a couple of rooms were redecorated, including Wallpapers in bright green, the Paris green, in German called poisonous green (``giftgr\"un''), since it contains arsenate. Afterwards we paid a visit to the old gothic brick church of St Martin, and then we dropped the car at Munich airport. There another passenger recommended us to get on a group ride ticket with him to the city centre, which turned out to be cheaper for all of us.\\
Since we arrived early enough in Munich we went into old town to enjoy some night views of the squares including Marienplatz, looked at the sculptures in the pedestrian zone, and climbed the lions of Feldherrenhalle.\\

Herrenchiemsee: New Palace*****, Old Palace****, \\
Landshut: Residenz***, St Martin***\\

April 21: Palaced out\\
In Munich there are two real nice palaces -- and we wanted to see both of them on this very day. Starting the day in Nymphenburg, the summer palace of Munich, as well as the birthplace of Ludwig II. The main hall (Steinerner Saal) and the adjacent bedrooms and cabinets are sumptuous baroque rooms, The vast garden is of English style with four park palaces. Amalienburg is considered the highlight of secular Bavarian Rococo. The chapel of Magdalenenklause is decorated with shells and old snail houses. Afterwards we had M\"unchener Schnitzel closeby, which had a filling with leek and spinash. We did stop by Max Krug and obtained Steins for Eric (and Siyi) and Indara.\\
Then we visited the giant palace of the Residenz, the largest palace of Germany. The royal apartments were still in renovation (still after over 6 years, after all they reopened in 2019), but the rooms of the Prince-electro times, and the rich rooms (Reiche Zimmer) - still make up over 80 rooms. After going through all those rooms, Eric and Indara were clearly palaced out, and they had no more attention span for the dozen rooms of the treasury. As usually when in Munich we had Grandma's meatballs (``Omas Fleischpflanzl'') at the N\"urnberger Bratwurst Gl\"ockl am Dom (really amazing once more).\\

Munich: Nymphenburg***** (Badenburg****, Amalienburg****, Pagodenburg****, Magdalenenklause****), Residenz***** (Cuvilies-Theater****, Treasury*****), St Michael****\\

April 22:\\
Indara decided to meet up with one of her old friends in Munich (and his new born daughter). She also had to deal with slides for a meeting. Thus Eric and I spent most of the day alone. This time we went to Schleissheim, a village where the Bavarian rulers built their summer residence. The original plan of the complex has not been realised, since the war of Spanish succession put the planned extension of the main wing on hold. The Corps de Logis is though very impressive. Some of the rooms have been transformed into a gallery of Baroque paintings, but most of the halls are decorated with impressive stuccos, gilded wood carvings, tapestries, and large frescoes. Period furniture is still available, particularly in the state bedrooms. The French garden is nice, at the end is the Schloss Lustheim. The main hall of this pleasure palace was unfortunately in renovation at this point.\\
In Munich we rushed a bit through the three main churches of Munich (little did we know that Eric would move to Garching for two years just a bit more than a year later). The inner city of Munich is not that large, in fact you are able to cross it within at most half an hour.\\ 
We met up with Indara and her friends by the Chinese Tower in the English Garden as previously planned. And we had lunch and a Mass of beer there. After her second Mass, wanting to taste both Pils and Starkbier, Indara attended her meeting, and then it was already time to say goodbye to Munich once more. On our flight we transferred in Zurich, getting a couple of chocolate bars from Swiss on our way, while Indara lost track of her hat at some point.\\

Schleissheim: Old Palace***, New Palace*****, Lustheim Palace****, \\
Munich: St Michael****, Frauenkirche***, Theatinerkirche****, English Garden with Chinese Tower****

\section{May 1: Neuchatel}
\label{Neuchatel2014}

What should you do on a free day -- preferably not work. I didn't plan any large outing beforehand, thus my brother and I decided to get ourselves to Neuchatel in order to visit the castle. The castle is one of the largest in Switzerland, it is nowadays used to host the cantonal parliament of Neuchatel. The interior has been clearly adapted for that purpose, though also keeping a couple of representative rooms. The Collegiate church next to the castle is a medieval church with a couple of statues, portals and a nice decorated roof top. Other than that Neuchatel is a cute little town just by the lake of Neuchatel, in good weather conditions one can see as far as the Bernese alps, dominated by Eiger, M\"onch, and Jungfrau.\\

Neuchatel: Castle****, Collegiate****

\section{May 4: Stresa}
\label{Stresa2014}

Stresa: Isola Superiore****, Isola Madre****, Isola Bella*****, Baveno: Santi Gervasio e Protasio***, Domodossola:  St. Gervasius \& Protasius**

\section{May 18: Baden-W\"urttemberg}
\label{Donaueschingen2014}

For once I planned to visit my parents at home again. In order to choose the day I checked what could be done in the area. I found out that the Palace in Donaueschingen offered tours on a selected couple of days. Thus I planned to be there on one of these days. But then once we drove up to Donaueschingen we just found a paper leaf hanging on the palace gates that unfortunately due to a wedding the originally scheduled tour had to be cancelled. AS IF weddings are not planned well into advance. Clearly the ducal family doesn't care to put such stuff online, neither did the people of the adjacent museum which run the tours offer any apology, just a snarky better luck next time. Since we were there already we visited the colllection of the Ducal family, indeed some fossils were nice, but also a visit of the town church couldn't cheer us up. To make things worse the Donauquelle was in renovation and surrounded by scaffolding (claimed to be the spring of the Danube, which is fake, it is anyway created by the confluence of two rivers and most of the time in the Danube sinkhole this part of the Danube disappears completely to reappear by the Aach Spring). \\
Having had really bad luck so far we decided to just drive on to the next larger town, which is Sigmaringen. One side branch of the imperial German family of Hohenzollern, built their large castle in that town. The rooms range from Baroque to classic and revival styles, so quite a variety to see, unfortunately it is not permitted to take any photos. On the way back we stopped by the abbey church of Beuron, once again a large Baroque church but with an attached chapel built in the Beuron interpretation of neo-byzantine style. And then we did another walk along the Danube gorge around the abbey, so at least the trip was worthwhile.\\

Donaueschingen: F\"urstenberg Museum***, St Johann**, Sigmaringen: Castle****, Beuron: Abbey****

\section{May 29: Gorge du Durnand}
\label{Durnand2014}

Gorge du Durnand*****

\section{June 8: Grindelwald}
\label{Grindelwald2014}

Co-traveller:\\
Chris M: US-American, PhD student at Johns Hopkin's University (yes that one with the worldwide covid dashboard among other things). Chris and I knew each other for about two years but surprisingly enough although we both enjoy hikes, we never did one together so far. \\

Clearly we had to change that, thus since summer was coming we wanted to go up to the Alps. Still early in the season I chose a spot which should give us good views of the mountains, still be a bit of a challenge, but have a final spot we could in fact reach. Thus we decided to go to Grindelwald, take the cable car to Pfingstegg and this time make it to the B\"areggh\"utte unlike on my last attempt with my brother. Starting out early, we arrived well in time, enjoyed walks along some rocks, little creeks and waterfalls, passing meadows and sheep and looking at the debris of the landslide of a couple of years ago. Due to the glacier melting the moraines and side walls had become unstable and a big chunk of the mountain collapsed creating a large glacier lake. This lake created an even more rapid melt of the glacier tongue, and an artificial tunnel was dug to prevent flooding of the valley. The glacier itself was even more doomed though. At that point only the upper part of the tongue still existed, now in 2021 even that part is more or less gone and the Lower Grindelwaldglacier is effectively gone (the upper part, called the Grindelwald-Fieschergletscher still exists, don't worry). There were multiple glacier lakes and remains of ice surrounding those by the tongue end of the glacier. Since it was early enough we decided to climb a bit higher but we quickly realised there were a couple of snowfields on the trail and we decided to turn back, have a short snack and walk down. Although it was only about 1 pm at that point and I suggested to think about doing more stuff just somewhere else, Chris decided he had enough for the day and we should rather go back to Geneva and enjoy a late afternoon/early evening by the lake instead.\\

Pfingsthorn \& Unterer Grindelwaldgletscher*****

\section{June 22: Meiringen}
\label{Meiringen2014}

Another weekend, another trip to the alps with my younger brother, once again close to the Interlaken region. This time we wanted to admire the famous Reichenbachfalls, immortalised by the final fight between Moriati and Sherlock Holmes. Well we took the cable car up and walked down. But it was a bit of a let down. Having seen many alpine waterfalls, at least in June the Reichenbachfalls didn't appear overly impressive for out tastes. A bit disappointed we continued to hike along the trails to reach the Aareschlucht. Now THAT is an impressive slot canyon, a gorge which definitely deserves a visit. The trail takes about an hour, both ends can be reached with a little train from Meiringen (else it is also less than a km away from the train station, otherwise you can also go through the gorge twice). Then we took the train back to Spiez and got on a boat over to Oberhofen. A while ago we considered getting off at Oberhofen while on a cruise of Lake Thun, but decided against. This time we thought we had enough time for a short visit. The castle is nicely located on a peninsula reaching into Lake Thun. Not only cute from the outside, the inside has carved rooms, more of a neo baroque or classicist interior, as was en vogue in the late 19th or early 20th century. The castle (and the adjacent gardens) for sure deserve a visit.\\

Meiringen: Reichenbachfall***, Aareschlucht*****\\
Oberhofen: Castle****

\section{July 24--August 4: South England}
\label{southengland2014}

I enjoy going to London, but by now it had been quite a while since I had been there the last time. This time I planned to do also the ``small places'' and sights in London, maybe also with small trips out of London. My sister enjoys being in London too, albeit spending times with friends in the London greater area as well. Since she wanted to leave late July and Buckingham palace opens up only in August, I extended the trip by a couple of days. Thus for once we all were up for a big family trip again with my parents, my little brother, and my sister. Our planning made the list of places larger and larger, and we decided to get the London Plus Railpass, which would also allow us to see most of South East England. Particularly my mum was very excited about this proposal, since she had wanted to do a whole organised tour of the Southern part of Great Britain for quite a while. Since we wanted to see all former Royal Palaces of London we got ourselves a partner ticket for two including the four old big palaces of London, as well as a discount on Kew Gardens.\\

July 24: London\\
My brother and I took the earliest flight out of Geneva to Gatwick and we met our parents at our hotel, where we all stayed in a four-bed room together. They had already flown in the previous day, seeing some museums with my sister during the afternoon. While my dad and my brother wanted to have a longer lunch, my mum and I decided to go to Syon House instead. Syon House is a large noble house in the suburbs of London with lots of nice classically decorated rooms, Queen Victoria had spent parts of her child hood in this house. The large greenhouse is quite famous as well with nice orchids. Then we took the train to Kew Gardens, where we met up with the rest of the family. Kew gardens is one of the largest botanical gardens in Europe, also one of the oldest. Besides giant greenhouses with palm trees, orchids, cactuses, as well as alpine plants, you can enjoy yourself on a tree top path, stroll across a Japanese Garden with a Japanese style pavilion, and a Chinese style pagoda. Inside the gardens is also the rather modest Kew Palace, once a private getaway for the Royal Family. Some of the rooms are still in their original state, as well as the tiny French style palace garden. Nice and cute, but since it was a private home, don't expect to see any large state rooms. We did enjoy our summer afternoon and evening out in this vast park. And then we had dinner at a fish restaurant.\\

London: Syon House****, Kew Gardens****\\

July 25: Windsor \& Hatfield\\
My parents and my brother joined me on this trip, but my dad and my brother decided they rather want to see Eton College than Windsor Castle, so it was me and my mum alone for the castle. The core of Windsor castle is from medieval times, the state apartments are from Baroque times onwards. Parts of the castle contain the private and semi-private chambers of the Royal family. While the private rooms are clearly off-limits all around the year, three of the semi-private rooms can be visited in winter time. The state apartments are very impressive, the King's dining room and the audience chamber of the queen are particularly impressive from these times. The Waterloo hall is the most impressive part of the Gregorian times, the Reception Hall, which had to be largely rebuilt after the devastating fire of the 1990s is an amazing room with gilded walls, tapestries and large windows overlooking the river. The second part of the castle is dominated by the large gothic St George's Chapel with nice wood-carved choir stalls, well-sculptured altar pieces and tombs in the side-chapels and stained glass windows. Photography is forbidden inside the castle and the chapel, but the ticket can be converted into a year ticket. We had a little snack by the train station, also taking a view photos of the steam engine ``The Queen''. Then we took trains over to Hatfield. My highschool English lessons book centred around the town of Hatfield and what the kids there were doing. Among those stories was a visit to Hatfield House. Remembering those stories I decided to just enquire what there would be to see. I liked what I found and convinced everybody to join me. The suggestion to visit St Alban's afterwards was voted down though. From the train station it is just a short 15 minute walk to Hatfield House. The Marble Hall is home to one of the rainbow portraits of Queen Elisabath I, the King James room is nice as well. The long gallery has a nice ceiling, wood carved walls, marble fireplaces, it has appeared in historical movies as well. Then we walked through the large French style gardens of the house, also having a short look into the Great Hall of the old palace, a big gothic hall which was being prepared for a Banquet, thus we couldn't walk through it on this day. And then we took the train back to London where we had dinner at the cafe of the Victoria \& Albert Museum, where my mum couldn't deal with the spiciness of the Chicken curry (it was in fact not very spicy). The museum had its late opening day, it does house artefacts from Nepal, India, China, Silverware, paintings from the 19th century, the Devonshire tapestries from the 15th century, as well as copies of famous artefacts, just like Trajan's column or Marc Aurel's column in Rome. I thought the museum was a bit hit and miss, clearly not my favourite in London itself even.\\
  
Windsor: Windsor Castle*****\\
Hatfield: Hatfield House*****\\
London: Victoria \& Albert Museum***\\
 
July 26: Brighton\\
Seaford: Seven Sisters*****\\
Brighton: Royal Pavillon****, Brighton Pier***\\

July 27: London\\
London: Eltham Palace****, Kensington Palace****, British Museum*****\\

July 28: Isle of Wight:\\

Coming back to London I convinced my mum to try a Turkish restaurant. She rather wanted to go to an Italian restaurant, claiming she wouldn't know what could be a good dish from Turkey she'd enjoy. Seems a mixed place of Kebab and Koefte was just down her alley. She enjoyed it that much, that we ate there twice more on this trip.\\

East Cowes: Osborne House****\\
Totland: Alum Bay with Needles*****\\

July 29: London:\\
London: All Hallows-by-the-Tower***, Tower of London****, St Paul's Cathedral*****, Mansion House****, Freemason's Hall***\\

July 30: Oxford \& Blenheim Palace:\\
My mum and I started our day early, since we wanted to participate in a special tour of Bodleian Library which would also include a sneak peak into the Radcliffe Camera Reading rooms. The tour started in the magnificent Divinity School, a large gothic room with lots of sculptures, nice elaborate vaults, before continuing into the old reading rooms of Bodleian Library with its wood carved shelves and paintings. The Radcliffe Camera is more of a classical building. We also walked though the modern extension of the library.Then we met up with my dad and my brother, which had arrived in Oxford on a later train. Altogether we visited Christ Church College with its large hall, which inspired the big hall in the Harry Potter movie series. The cloisters and stair cases played a role in those movies too. The church of the college is in fact the cathedral of the diocese of Oxford as well, albeit a smallish cathedral. Another church which dominates the main road in Oxford is the gothic Virgin Mary church. After a small snack we took the bus over to Woodstock which is home of Blenheim Palace. Blenheim Palace is one of the largest palaces in England, built in Baroque style by the Churchill family. The state rooms can be visited, the private apartments of the dukes on special occasions. The rooms are filled with frescoes, marble statues, paintings and tapestries. The birth room of Winston Churchill is part of the tour, as well as a small exhibition of his private items. The library is one of the largest private classicist library in the country. The gardens are large and vast with a couple of pavilions, fountains, a cascade as well as some little flower gardens. We took the bus back to Oxford, had dinner there in a small basement restaurant/pub having also some local beers, and then took the train back to London.\\
 
Oxford: Bodleian Library with Divinity School*****, Radcliffe Camera***, Christ Church College*****, Virgin Mary Church***\\
Woodstock: Blenheim Palace*****\\

July 31: London:\\
London: Houses of Parliament*****, Banqueting House****, St Margaret's***, Westminster Abbey*****, Jewel Tower*, Westminster Cathedral****, Apsley House***, British Museum*****\\

August 1: Rochester, Canterbury \& Dover\\
We originally wanted to take the train from Victoria to Rochester. But then the train was delayed indefinitely, thus we took the metro over to Paddington and went to Rochester from there. We started our day with the cathedral of Rochester, a decent cathedral which looked like a romanesque-gothic mixture including modern murals from 2004, but in a rather romanticised style. Next to the cathedral is the old castle, which looks still almost intact from the outside, but is clearly ruined for centuries, it had been used as Robin Hood's castle in the 90s Kevin Costner movie. Quite interesting to climb the upper floors with superb views of the cathedral, the river, and the rusty old submarine of U-475 Black Widdow. And on by train to the second town of the day: Canterbury. The ruined remains of the keep are all that is left from Canterbury Castles, but still just a short detour on our way to the cathedral, which is the seat of the spiritual leader of the anglican church. A really beautiful impressive gothic church, with many tombs, stained glass windows and beautiful screens, a cloister with ruins of a monastery and a chapter house with a nice ceiling. And then it was time to take the train to Dover. We walked up to the castle, where folks tried to replicate the original medieval interior, which falls a bit flat though and rather looks like a Disney replicate. Dover castle had been used as a hospital during the Blitz, we took a guided tour of what remains of the former rooms. Then we walked over to the harbour, where ferries from France were landing and departing. In the distance one could see the start of the cliffs of Dover.\\

Rochester: Cathedral***, Castle****\\
Canterbury: Castle***, Cathedral*****\\
 Dover: Castle***\\

August 2: Stonehenge, Salisbury \& Bath\\
This day was supposed to be one of the highlights of our trip, it clearly turned out to be the day with one of the heaviest rainfalls I walked through. We took the train out to Salisbury and got on the bus there, it did rain quite a bit on the ride, but once we switched to the bus which was bringing us from the parking lot closer to the site that is Stonehenge, the rain clearly got worse. Some people preferred to stay on the bus, me, my parents, and my brother decided that we didn't get there just to have one photo and then just turn our backs. So we walked around the circle: clearly it is an impressive sight to see something that old just standing there in front of you. In order to prevent tourists from damaging the site, you don't get any closer than 5 or 10 m, but still impressive. Unfortunately the rain was coming from all sides, even an Umbrella was useless. Thus we all rushed into the museum and warmed ourselves up and tried to dry up by a bit. Unfortunately not with much success, but we got on the bus to Salisbury. The gothic cathedral is one of the famous English cathedrals, completely dominated by its impressive spire. Most probably the cathedral survived the war without any damage based on the fact, that the spire was a useful landmark for the german air force. The interior of the cathedral is one of the finest in England as well. It even turned out to be a sunny afternoon which helped us to further dry up walking around town a bit. We got back to Salisbury, had a coffee break, and wondered if our London plus pass would cover actually the trip from Salisbury to Bath. Bath and Salisbury were clearly mentioned as cities, which could be reached directly from London with the pass, but the direct way to Bath doesn't lead over Salisbury, although the time spent via Salisbury is roughly the same as it turned out later. Once we arrived in Bath we first stepped into the Abbey. While it is a decent church (pales a bit compared to Salisbury), the way the guards tried to force you into donating 2 pounds was appalling. I usually donate, but AFTER I have seen the place, otherwise just charge, I am more than happy to pay a fee, which I know goes into the conservation efforts of the place. At least we got to hear an organ recital which was very nice too. Afterwards we visited the Roman Bath, in fact the Bath has been maintained over many centuries, thus it is still in a functioning superb state. Bath is also famous for its squares, though the Circus and the Royal Crescent failed to impress me at all. OK they are nice shaped squares, and the facades of the houses are not shabby, but that can be said about many squares in many European cities.\\

Amesbury: Stonehenge*****\\
Salisbury: Cathedral*****\\
Bath: Abbey****, Roman Bath****, The Circus**, Royal Crescent**\\

August 3: London\\
London: Hampton Court Palace*****, Strawberry Hill House***, Chiswick House***\\

August 4: London\\
And we started our day at Buckingham Palace. Clearly one of the most famous palaces in the world, it is also one of the most beautiful. The state rooms are large classicist halls, a large gallery is part of the state apartments as well, also the large ball room. As in all palaces still used by the royal family, photos are not permitted inside, but you can take some on the vast grounds of the palace, although parts of the gardens not accessible either. Then our parents left for their flight, while me and my brother wondered if we should do Clarence House. Deciding against that we instead saw the arts exhibitions in Somerset House, which was more or less OK. Then we took a metro to the Guildhall, which turned out to be a nice neogothic large hall.\\

London: Buckingham Palace*****, Somerset House***, Guildhall****\\

\section{August 22-August 31: Vienna, Bratislava \& Budapest}
\label{austria2014}

August 22: Vienna:\\
Vienna: Hofburg: State Apartments****, National Library*****, Treasury*****, Silver Chamber***, Garden Palace Liechtenstein****, City Palace Liechtenstein*****, Stephansdom****\\

August 23: Eisenstadt \& Fert\"od:\\
Eisenstadt: Esterhazy Palace****, Fert\"od: Esterhazy Palace*****\\
Vienna: Hofburg: Great Redoutensaal****\\

August 24: Bratislava \& Schlosshof:\\
Bratislava: Palais Grassalkovich**, Bratislava Castle**, Primatialpalais****, Palais Apponyi***, Old City Hall***, Martin's Cathedral****\\
Schlosshof: Schloss Hof*****\\

August 25: Vienna:\\
The first day of the workshop: Most of it was spent in the seminar room of the Technical University, discussing technical details and performances of Jet and MET reconstruction. We had lunch by Naschmarkt choosing some of the nice food, and followed by desert somewhere else. On the same square of TU is the Baroque church of Karlskirche, one of the many beautiful spots in Vienna. The interior is very nice -- in theory -- in practise a really ugly elevator obstructs the view of the dome, as well as the alter piece. Originally it had been installed to carry a platform for the renovation of the dome fresco. Then the church realised it could be made into a money making machine, since the dome fresco is only visible from up there anymore. This atrocious construction completely destroys any of the flair the church once had. Even years later it still wasn't removed unfortunately. At least fine enough for a post coffee stop.\\

Vienna: Karlskirche***\\

August 27: Vienna:\\
Peterskirche****, Albertina****\\

August 28: Vienna:\\
Schloss Sch\"onbrunn*****, Hermesvilla****, Staatsoper****, Prince Eugen Winterpalais****, Stephansdom****, Schlosstheater Sch\"onbrunn****\\

August 29: Vienna\\
Sezession's Building*, Upper Belvedere***, Lower Belvedere***, Old Danube**\\

August 30: Budapest \& G\"od\"ollo:\\
Budapest: Parliament***, Stephen's Basilica**, State Opera***, Fischerbastei**, G\"od\"ollo: Palace**\\

August 31: Budapest:\\
Castle with National Gallery***, Matthias Church***, Stephen's Basilica**, Great Synagoge \& Jewish Museum**

\section{September 6--September 14: Tuscany}
\label{Tuscany2014}

September 6: Firenze\\
Palazzo Pitti*****, Boboli Gardens with Buontalenti Grotta*****\\

September 7: Pisa \& Firenze:
Pisa: Duomo*****, Battisterio***, Palazzo Blu***, San Michele in Borgo**, Firenze: Orsanmichele***, Palazzo Medici Riccardi*****, Galleria Academia*****, San Gaetano****\\

September 8: Monday: Bologna:\\
In fact the day the ISMD2014 conference started. As you might expect in Italy conferences happen in style, thus the conference venue was an old oratory. It still contained the beautiful baroque altarpiece, but the ceiling clearly showed the damage air raids of WWII had caused. Still an interesting mix of the original oratory and modern attributes to make the building fit for conferences.\\

Oratorio San Filippo Neri***

September 9: Bologna:\\
Cathedral***

September 10: Bologna:\\
Basilica San Petronio***, Archiginnasio****, Basilica San Stefano****, Palazzo d'Accursio****

September 11: Bologna:\\
Palazzo d'Accursio****, Santa Maria della Vita***, Basilica San Domenico****, Santuario de Corpus Domini****, Palazzo Pepoli Campogrande****, Basilica Santo Stefano****, San Giacomo Maggiore****, Basilica Madonna di San Luca***, Cathedral***\\

September 12: Firenze:\\
San Lorenzo***, Battisterio*****,  Palazzo Vecchio*****, Medici Chapel*****\\

September 13: Saturday: Firenze \& Siena:\\
Florence: Palazzo Davanzati****, Duomo**** (Dome*****, without it a tad plaine, I was disappointed), Casa Martelli****, San Spirito***, Santa Croce*****
Siena: Palazzo Pubblico*****, Duomo*****\\

September 14: Firenze:\\
Villa della Petraia*****, Villa Corsini***, Palazzo del Bargello****, San Lorenzo***, Santa Maria Novella*****

\section{September 19--September 21: Paris}
\label{Paris2014}

September 20:\\
Palais d'Elysee*****, Hotel de Beauvau***, Hotel de Charost****, Hotel de Monaco*****, Hotel du Chatelet****, Hotel de Clermont****, Hotel de Villeroy***, Hotel de Castries***, Hotel de Matignon*****\\

September 21:\\
Palais du Luxembourg*****, Petit Luxembourg****, Hotel de Ville*****, Palais Royal*****, Hotel de Bourvallais****, Hotel de Talleyrand****, Hotel de la Marine*****, Hotel de Salm****, Hotel d'Estrees****, Hotel de Roquelaure****

\section{October 4: Creux du Van}
\label{Creuxduvan2014}

The Creux du Van is called the Grand Canyon of Switzerland (obviously not as large as the Grand Canyon at all). Located in the Jura Mountains, the walls of the crescent shaped mountain fall straight down on the inner semi-circle by about 200 m. Still very impreszsive isn't it. My first hike in a while without my brother, who moved away to the Chicago area, a full ocean away from me.\\

Co-hikers:\\
Manuel: US-American, physics grad student at UC Riverside. After hearing a lot about trips and travels with me, Manuel thinks it is time to experience this adventure himself.\\
Bing: Chinese, physics grad student at Ohio State University. Bing loves hiking too (typically longer and more difficult hikes than I do though). He enjoys a good panoramic view, so Creux du Van seemed to be ideal. He in fact joined the hike just the day before when we had beers on a nice sunny terrace and people commented on how he might enjoy the hike too.\\

October 4:\\
We started out by CERN just after lunch time, and we had to suffer getting up to the village of Noirague. Manuel's car was an automatic, but pretty old, thus the submission never worked higher than 4th gear, but it wouldn't just stuck to 4th, it tried and continuously failed to switch to 5th gear shortly above 90 km/h. Thus our average cruising speed was about 90 km/h. Anyways arriving a bit more than 2 h later, we started to hike along cow filled meadows, reaching soon the forrest and slowly climbing up the mountain, until we reached the cliffs of the Creux du Van. It was a beautiful sunny early October day, so pretty nice to hike along the cliffs. The way down was a bit more scenic, also since the trees started already slightly to change colour. Compared to the previous times we hiked pretty relaxed, since we knew we could take our time. The hike is assigned a time of about 3 h, so pretty nice and chill. The view of rocky cliffs is always an amazing sight, indeed some of the most impressive in Switzerland\\

Creux du Van*****

\section{October 23--October 27: Istanbul}
\label{Istanbul2014}

Co-traveller:\\
Chris M.: US-American from Kansas. Chris hasn't been travelling a lot with me so far (besides that one hike to Grindelwald). Considering my claim to fame to do a lot during one trip, he was already a bit on the edge if our way to travel would be compatible. The day programme that is, during the evenings most of our interests line up pretty well. So we decided in case we realise that it gets too much, each of us can do part of the day alone and meet up at pre-defined places after, something which I did for most other trips with people from then on as well. Chris has never been to Turkey or even to Asia, so we plan to have two firsts here. As far as I remember it was also his first time in a Muslim country.\\

October 23 (Thursday):\\
Turkish Airlines provided excellent service (great dinner, nice on-board entertainment). But 20 mins in the person in front of Chris decided to incline their seat to the max. Thus Chris requested if he could have another seat (which was possible). Once we got to the airport and got through immigration (US-Americans need online e-visas, Germans just need to show IDs), we took a taxi to our hotel. This was located in the middle of old town with very nice cafes, restaurants, bars etc around. But it was also not renovated for quite some time. I didn't bother it that much, beds were large and comfy enough, and I didn't plan to spend ages in the shower anyway.\\

October 24 (Friday):\\
We started out having breakfast and coffee by a cafe by the Domabahce Mosque sitting next to the sea. The pier for ferries crossing over to the Asian side was just a couple of metres away. At one point Chris shouted watch out, and I jumped up quickly enough to avoid getting hit by a wave created by one of the large freight ships crossing nearby. Else it would have been a half wet pants day. After that adventure we started the day by Dolmabahce Palace. A giant palace used by the Sultans as new home, after they decided that Topkapi Palace was too old-fashioned as main residence. We decided to get the full ticket, which gave us access to the private Harem quarters and the gardens as well. The state rooms were clearly set up to impress visitors, the staircase alone is decorated with countless crystals. We were also told how proud the Sultan was to be able to show his guest the exotic styles of Europe (kind of just the opposite of what Chinese Saloons or Turkish Salons do in European palaces, whatever is considered exotic is the thing to get). The huge memorial hall is dominated by a giant chandelier, the largest one of its time to be installed, and we were told that the electric light is still the original one installed. The Blue and Pink Halls impressed me most in the Harem quarters. In those quarters one can also visit the office of Kemal Atat\"urk, the founder of Turkey. There is the little downside that photography was not allowed in the whole palace area, including the garden pavilions. I would advice you to visit at least the Crystal Pavilion.\\
Then we planned to visit the Yildiz Palace, walking and enjoying the peaceful Yildiz Park. Once we reached the palace we had to find out that it had been closed for just a couple of day due to a planned renovation (unfortunately none of my online sources had that announced). Anyways we walked down to another small park, where one can find the two Ihlamur Pavilions. Also a perfect stop to have another snack at one of the cafes in the park. And we were shown the inside of both pavilions (should have been a guided tour, but trust me mine and Chris's Turkish is close to non-existent). And then back to see the interior of the Dolmabahce Mosque and back by tram to the other side of the Golden Horn.\\
Clearly I thought we might need more time for what we had seen up to now, but since we had time I decided we should see the Hagia Sophia. Once the largest church of Christianity after Istanbul was conquered by the Ottomans, the church was converted into a Mosque, and all mosaics had been covered. Later transformed into a museum the covers had been largely removed to bring out the mosaics again. Protected by the layer of paintings nowadays they all appear in the glory again. Breathtaking and worth the wait for getting tickets. The main dome is still under renovation (a bit less though than back in 2001). Then we went to the Great Bazaar stopping by Nuruosmaniye Mosque in between. Then we strolled through the many roads and shops of the Great Bazaar, where Chris got his parents some souvenirs. Ending up by the other side of the complex we went to the New Mosque, another mosque built in the typical Ottoman style. And then we had a large dinner in the street in front of our hotel, and later a few drinks in one of the plenty of bars and enjoyed Shisha.\\

Istanbul: Dolmabahce Palace***** (Private Quarters*****), Ihlamur Pavillons****, Domabahce Mosque***, Hagia Sophia*****, Nuruosmaniye Mosque***, Great Bazaar***, New Mosque****\\

October 25 (Saturday):\\
We had chosen a hotel very close to the old town, thus we walked over to the Topkapi Palace, where we didn't have to queue that much being there just before opening time. Unlike last time I decided that we should see the Harem quarters as well. Unlike for European Palaces the rooms are scattered in a vast area around many courts and gardens. We started in the treasury building, the most important piece are old artefacts of Mohammed, including his traditional dress and cape. They also set up a portrait gallery of the sultans, for those whose picture they didn't have they made up portraits (reminded me of what the Venetians did for the Palazzo Ducale, or the Papal Portrait Freeze of St Paul's in Rome). The Harem quarters were very interesting as well, exquisite decorations with many colourful tiles, windows and landscape paintings. If you have time you should definitely see that part of the palace too. Besides being allowed to take photos the Topkapi palace also makes you feel in another world, while in Dolmabahce the palace orients its style quite a bit on European equivalents.\\
On the premise of Tokpapi is also the large old byzantine church of Hagia Irene, once used as palace church. Next to Hagia Sophia are the tombs of a couple of sultans, the so-called T\"urbes with traditional Ottoman Domes. Another highlight of Islamic Ottoman art is the Sultan Ahmed Mosque, the so-called Blue Mosque. Still being an active mosque, visitors are requested to leave their shoes outside and those of non-muslim faith are asked to remain on the western side of the mosque. It is still quite a sight to see all geometrical decorative tiles and the many large chandeliers scattered all around the full and semi-domes of the mosque, clearly the stained-glass windows had a lot of nice patterns on display too. We had now seen a lot of Istanbul old town, thus it was time to jump on one ferry taking us to the other side of the golden horn sea inlet. There the sultans had once built a larger summer palace, but nowdays only one pavilion remains, the Aynalikavak Pavillon. By now this part of town has quite a morbid appeal. Situated by the harbour you find a lot of rusty equipment, decaying factories and ware houses in dire need of work. Still we made it to the pavilion without any troubles, and I did enjoy that this is a unique spot where we can see a traditional Diwan from a time between the construction of Topkapi and Dolmabahce thus closing a gap of Ottoman architecture. And then we took another ferry up to the tip of the Golden horn.\\
 There we walked along the late Roman Theodosian City Walls, until we reached the remains of the once most impressive Imperial palace of the former Byzantine Empire. The Tekfur palace has been undergoing reconstruction since the early 2010s and it is supposed to be set up as museum regarding the Byzantine era of Istanbul. And then we got to our last spot, the Chora Church. Unlike in Hagia Sophia in this church all mosaics are open for you to see, one more beautiful than the other, with a big Christ Pantokrator mosaic held in gold as centrepiece. Unfortunately the apsis was in renovation and thus hidden from view, I imagine the experience would have been otherworldly otherwise. Then we took a bus back to our part of oldtown. For whatever reason my public transportation ticket had stopped working, but the bus driver let us on anyway and pay in cash instead. On our way we passed by the Sultan Suleiman mosque which I still miss to do in Istanbul, and we drove below the large old Roman aqueduct of Istanbul.\\
 Having done quite a large program we decided to have dinner by a fish restaurant by Galata Bridge. There we had a couple of local fish which we were told they fished on that day. Don't ask me if that was true, it tasted great anyways. Having night views of the new mosque, the sparkling Bosporus bridge, the ferries running at night, as well as the Topkapi Palace did add the the vibes as well. Afterwards we walked past Hagia Sophia and Sultan Ahmed Mosque for some night photos ending up in a bar for Sisha and couple of beers.\\

Istanbul: Topkapi Palace***** (Harem*****), Hagia Irene***, T\"urbe of Sultans****, Sultan Ahmed Mosque****, Aynalikavak Pavillon****, Theodosian Walls and Tekfur Palace****, Chora Church*****\\

October 26 (Sunday):\\
Due to the heavy pace of our touristic program so far Chris decided to go for a rather slow day, taking the Bosporus ferry up to Yoros castle for the day, while I wanted to take the ferry only for the way back instead. Thus I got myself a new day ticket, jumped on the newly opened Metro which crosses below the Bosporus ending up on the Asian side Istanbul. There i had a short snack, went into the Mosque, and jumped on a bus which was taking me to Beylerbeyi Palace. Seems hardly any tourist usually takes this way, thus I was asked by Turkish folks at which stop they had to get off for their destination. Although I did have the schedule at hand, unfortunately my non-existent Turkish was clearly of no help. Once I made it to the palace I signed up for an English tour, it was a bit chaotic since a couple of cruise-ship tours were going on at the same time. The palace is pretty nice, with a mix of classic Ottoman style and European influences, which can be seen in the Hall of Fountains or the Egyptian Hall. Unfortunately the no photo policy is true in that place too (it damages the gold we were told - which is just a big fat lie, but OK).\\
I continued on the bus to K\"uc\"uksu Kiosk. This is another little summer pavilion of the sultans, you might also know it as stand-in for Elektra King's palace in Baku in James Bond: The World Is Not Enough. It is a prime example of Turkish Rococo with mainly green windows and elaborately decorated rooms. Then I stopped by Anadolu Hisari for a couple of photos of that fortress and walked over to the road leading to Beykoz Kiosk. This Kiosk had been a museum up to 2014, and just a few weeks before I visited Erdogan decided to use it for himself again, thus it was closed. So I continued on another bus up to the end of the Bosporus.\\
On my way up to Yoros castle I met Chris who was already on his way down and we decided to meet once I would be down for lunch. While Yoros castle itself is closed for tourists from up there you still have a superb view of the Bosporus and particularly the Black Sea. It is here that the third bridge crossing the Bosporus was in construction at that point. Looking down the Bosporus one could see a lot of fisher-boats and large Freighters making their way through the strait. Once down Chris and I enjoyed another delicious dinner before getting on the ferry back to Istanbul, passing along many villas, villages, the fortress of Rumeli Hisaris and the two large bridges over the Bosporus, and last but not least the last big palace of the Ottoman sultan, which is nowadays a high class hotel, the Ciragan Palace. Chris took some rest in our hotel, while I continued to visit the gigantic Roman Basilica Cistern, which had been enlarged multiple times and had been in use up to the early 20th century. An amazing place of ancient civil engineering, also the capitals originate partially from old temples and palaces.\\

Atik Valide Kui\"olliyesi Mosque***, Beylerbeyi Palace*****, K\"uc\"uksu Kiosk****, Yoros Castle***, Rumeli Hisari*****, Anadolu Hisari****, Basilica Cistern*****\\

October 27 (Monday):\\
We got up early had still turkish tea, and then got on our flight. Having flown into and out of Geneva multiple times by now, this was the one and only time I crossed Mont Blanc on this side with absolutely fantastic breathtaking views of the Glacier de la Brenva, Glacier du Geant and the Vallee Blanche with its magnificent icefalls, so far a once in a lifetime.\\

Flying over Mont Blanc with its Glaciers*****\\

This already concludes my adventures with Chris M, we did one hike, one multi-day trip (and a really fun one that is, most probably the week I smoked most in my life) including two UNESCO world heritage sites. Having known Chris for many years with many late evenings, may it be Trivia Pub Nights, dancing to DJ events, Premier \& Champions League Games, BBQs, or just Board Game evenings (and yes even seeing Donald Trump win the 2016 elections), it is always fun to spend time with him. Thus I would be definitely up for a trip again sometime, somewhere. If Chris can deal with my hardcore way of travel (maybe sitting out half of the items while enjoying good times in cafes etc), that's another thing to consider.

\section{November 2: Saas Fee}
\label{SaasFee2014}

Once again Andrew was in town and had the desire to go to the mountains. Since I had been in Grindelwald, Zermatt, and Chamonix pretty often by that point, I thought it could be nice to go to Saas Fee again.\\

Co-traveller:\\
Andrew: Californian, a UCLA electronics engineer based in Los Angeles, just around for a bit more than a week, thus using the one and only weekend in Europe in a while to see the mountains.\\

While the temperature was pretty cold, there had been almost no snow yet. Thus we decided that we could still go to Saas-Fee and hike quite a bit up. We started very early in the morning and arrived in Saas Fee shortly after sunrise. We started to hike up the ski piste to L\"angfl\"uh. The moraine of the glacier is very impressive, even with a little lake. The Feeglacier starts at the Allalin mountain and splits at L\"angfl\"uh into two parts. We walked up the ridge up to Spielboden, where we stopped for a short snack. Although the temperature was in single digits, the hiking heated up so much that we walked up in t-shirts after a while. Since my last visit in 2010 the glacier had been receded quite substantially, the glacier tongue had detached and a few hundred metres were just dead ice nowadays. The day was very clear in sunny, thus the views of the surrounding scenery and the mountains was very nice. Considering that it starts to get dark already by 4 pm we stopped about 30 metres altitude short of our original planned final stop (we missed basically one corner) and decided to turn around and get to the valley again. There we had coffee and a regional cake before getting back to Geneva, where we had some R\"osti.\\

Feegletscher*****\\

\section{December 12-14: London}
\label{London2014}

Co-traveller:\\
Manuel: US-American, first-generation immigrant with roots in Mexico. Manuel wanted to see London, and I love to go to London anyway (in fact I did just a few months before). Thus I still had even valid tickets for some of the places we wanted to see, so didn't take much convincing to see those practically for free once again. At that point our common friend Hossein was in town as well (with another friend of his), thus the ideal weekend to have a bit more fun in London. Also it is one of those weekends where part of the group is on its way back home so not too much to do at home in Geneva.\\

December 12 (Friday):\\
There are multiple flights leaving Geneva for London regulary (even holds in pandemic times). On this day we decided to get on a plane to London Gatwick. Unfortunately on this very same day the UK flight control had a major incident where their computing system went down for a while. All flights from and to the UK had a major delay, we talking about delays of over an h here. We still made it out of Geneva, but arriving in Gatwick very shortly before midnight, thus with the transfer over to Paddington it was shortly before 1 am when we got to the (not so much party for tonight) Kings Cross Keystone hostel.\\

December 13 (Saturday):\\
Getting up really early we arrived by Westminster Abbey just as it opened. Clearly less crowded than in the middle of the day in Summer I enjoyed once again seeing all the Royal tombs, or the fantastic chapels, the tomb of Isaac Newton, the choir, transept, just everything. Then our next point was the gigantic venue which is the Royal Albert Hall (even a first time for me). We had an interesting tour (after having a coffee in the cafe of the venue), through the Foyers and the main hall, with a sneak peak on the Queen's seating area. As the Albert Memorial is so close to the Royal Albert Hall, we just had a short detour to take some snap shots. As most museums in London, the Museum of Natural History can be visited for free, once more I was fascinated by the vast collection of dinosaur skeletons (and the gigantic skeleton of whales and sharks as well). \\ 
And then it was time to take the British Rail out to the vast complex that is Hampton Court Palace, where I still had the yearly dual ticket for - so yeah for free palace visits. Clearly Hampton Court is one of the most amazing palaces in London, both the huge Gothic Great Hall from the times of Henry VIII., as well as the baroque State Apartments built by Christopher Wren for William III and Mary II. Unlike in summer the eldest part of the palace, Cardinal Wolsey's Cabinet had be reopened. By the time we left the light festival had started. The facades had been illuminated in multiple colours, and we were very eager about what was to come. But then we were told the yearly ticket wasn't valid for this type of event.\\
Thus we rather went back to London, where we went to the Houses of Parliament for some night photos, where I tripped and fell over just in front of the London Eye (without any further consequences). Manuel claimed it was a "nice fall". Then we took some more photos of the Tower of London and the Tower Bridge, before we met up with Hossein and his friend for dinner, and early ending pub visit (seems those are rarely open even until midnight).\\

Westminster Abbey*****, Museum of Natural History****, Royal Albert Hall***, Albert Memorial***, Hampton Court Palace*****\\

December 14 (Sunday):\\
On this Sunday we started the day in Windsor Castle, for which I had a yearly ticket as well. In winter not only the State Apartments are open, but as well the semi-private State Rooms which I hadn't seen on my previous two trips. These include private dining rooms, and the green and crimson drawing rooms. I am still amazed how the queen's great reception room had been remodelled after the destruction by the big fire in the early 90s. After our visit to Windsor Manuel and I visited the Wallace House with its gigantic private collection of paintings, silver ware, glasses, pottery and porcelain as well as baroque style State Rooms and galleries. Since we still had a bit of time left we spent a bit of time at the British Museum, particularly the mexican, egyptian and assyrian artefacts.\\
At Gatwick we had Mexican food, but they messed up our order and first saved us a different order. Thus both of us got 1 1/2 dinners out after the mistake had been realised and corrected for. At least the flight out was only delayed by about a quarter.\\\

Windsor Castle*****, Wallace House****, British Museum*****
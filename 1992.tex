\chapter{Year 1992}
\label{1992}

\section{July: Daun}
\label{1992:Daun}

My parents decided to go for our family trip to a place in Germany once again, this time opting for a place in Rhineland-Palatine with the option to see also a big of the northern Rhineland. \\ 

The town where we stayed at was Daun, a town with an old castle and three close-by Maars, lakes which are remains of a volcanic steam explosions. In fact the local german word Maar were used to scientifically name those type of lakes.. 

%Weinfeldener Maar, Totenmaar, Schalkenmeerer Maar
%7.7 Maria Laach
%8.7. Koblenz, Ehrenbreitstein
%Strohn: Lavabombe, ROemergrag
%koeln: Dom, Aachen: Dom
%Manderscheid (Burgen), Saegewerg
%Kasselburg
%wWallenderborn




\section{October 25--November 1: Lechbruck}
\label{1992:Lechbruck}

Since we loved it in Lechbruck two years ago this town was chosen again as late autumn destination.\\

Once more we all enjoyed the large indoor pool with the adjacent hot outdoor pool - still fascinating to watch the steam over the pool. Other than that the church of Lechbruck received a couple of new bells which were lifted up to the tower on this weekend with a large celebration going with it. This time the renovation of Wieskirche had been finished, thus at least something could be seen of the church. We once again visited one of Ludwig's II castles, this time the beautiful little palace of Linderhof, which is in fact the only of his large projects which was finished. On our way back we stopped by the close-by Baroque monastery of Ettal. And I bought my first guidebook about castles myself, outlining all realised project and plans the king had in mind concerning castles, theatres etc. Naturally we also did a little hike, once again going up the Auerberg. And we visited F\"ussen with its castle and churches. I thought back then the church of St Mang was really big, little did I know how large churches can be at this point. Don't get me wrong, St Mang is indeed a fine church but when I visited about 20 years later it was just a normal sized Baroque Bavarian church. Anyways having read up a bit on the projects of Ludwig II., I convinced my parents that it would be nice to hike up the ruin of Falkenstein by Pfronten, the place where the king started his last castle project. Besides building a road and some foundations for a future castle nothing more had been finished. The ruin itself is one of the highest located in Germany, so it was quite a substantial hike for us kids, but I enjoyed it.\\

Lechbruck: Church\\
Ettal: Monastery \& Linderhof\\
Auerberg\\
Steingaden: Wieskirche\\
F\"ussen: St Mang, Hohes Schloss (Castle), Heilig-Geist-Kirche, Franziskanerkirche\\
Pfronten: Burg Falkenstein*\\


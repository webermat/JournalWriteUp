\chapter{Year 1991}
\label{1991}

\section{May18--June 1; Kell am See}
\label{1991:Kell}

%21 Porta Nigra, Kaiserthermen, Aula, Dom, Matthias
%23 Flugzeugmuseum Hermeskeil --Concorde
%24 Luxembourg 
%26 Saarburg: old castle, waterfall
%28 hike to Eltz and Cochem
%31 ruwerquelle (second attempt)
This summer holiday brought us to a holiday village by Kell am See in the Hunsbr\"uck mountains. We played also a couple of times with dutch kids next door, although we didn't understand all which they tried to tell us.\\
We did also plenty of trips to nearby Trier with its famous Roman gateway, the Porta Nigra, and other ruins such as the impressive Imperial Baths or the former audience chamber of Emperor Contantine. We also saw other medieval churches such as the cathedral and the basilica of apostle Matthias. In Luxembourg we walked along the city walls, which rather looked like an old ruined castle to me. We did also more castles, such as Burg Eltz, famous since it featured prominently on Germany's 500 DM bank note. I remember that we had to walk quite a bit from parking to reach the castle through forests and then we took part of the tour. Later on we stopped on one day as well by the former Imperial castle of Cochem. We did also two hikes to reach the sspring of the Ruwer river. On our first attempt we took a wrong turn and then ended up doing a forest lake walk instead, which was still pretty enough. On our second attempt we made it to the spring which was in summer rather a small creek than a river. A further trip lead us to the town of Saarburg where we walked up on the castle hill, and enjoyed view of the waterfall which sits in the middle of town.\\

Trier: Porta Nigra*****, Constantine Basilica***, Imperial Baths****, Dom*****, Matthiasbasilika****\\
Luxemburg: Fortifications***, Cathedral**\\
Wierschem: Burg Eltz*****\\

\section{August 3--August 13: \"Uberlingen}
\label{1991:Uberlingen}

And we had another camping holiday, this time with the extended family including my aunt, two uncles, three cousins, and my grand parents. In \"Uberlingen we had a couple of walks through the parks of the town and the minster, then we did walk through the reconstructed pile dwellings of Unteruhldingen, had a tour of the Old Castle in Meersburg, one of the oldest still inhabited buildings in the country with walls which are more than one m thick on average. Last but not least we took a ferry across the lake to reach the gardens and parks of the Mainau island.\\

Meersburg: Old Castle****\\
Ueberlingen: M\"unster***\\
Unteruhldingen: Pile Dwelling*****\\
 Mainau: Gardens*****
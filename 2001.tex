\chapter{Year 2001}
\label{2001}

\section{Spring 2001 -- Barcelona}
\label{2001:Barcelona}

Montserrat***, Sagrada Familia*****, Cathedral*****, Aquarium***, Casa Mila****, Beach in Sitges**, Miro Museum***, Figures: Teatro Dali*****, Pont du Gard****

\section{May 25 - May 31 -- Berlin}
\label{2001:Berlin}

Reichstag*****, Sanssouci*****, Egypt Museum****, Charlottenburg*****, Kaiser-Wilhelm-Gedaechtniskirche***, Dom****, Deutscher Dom***, Zitadelle Spandau**

\section{August 22 - September 3: Andalusia}
\label{2001:Andalusia}

My family really loves spending time in Spain, both to see the culture and to be on the beach. Thus Andalusia offers both of those worlds and thus one destination which seemed to be ideal. After we had good experience the two years before we opted for a guided tour of Andalusia.\\

August 22: Where is our hotel?\\
Arriving late in the evening on we were already awaited by a shuttle, which was supposed to bring my family and two fellow travellers to our hotel. We got out and proceeded to check-in -- AND were informed that we were dropped of by the wrong hotel. Unfortunately nobody knew though were the correct hotel would have been. Thus my father called my older brother, who had opted out of the trip and decided to stay at home. Calling the agency in Germany forth and back, detailing that we were lost at this particular hotel - some 30 minutes later another shuttle arrived taking us to the correct hotel. Our tour guide for the next few days claimed she was already looking for us, when in fact she even didn't initiate anything and most probably even didn't bother to check if everybody was properly picked up. At least in the following days she showed once and once again that she wasn't up to the standards we had so far travelling with this agency.\\

August 23: Nerja\\
Our first stop was the Balcon de Europa, a viewpoint over the Costa del Sol in the village of Nerja, it has a couple of cliffs around, but the weather was not that sunny, thus maybe not as impressive as it could have been. Then we stopped by the very impressive caves of Nerja, which stretch for a couple of km. Huge halls with gigantic stalactites and Stalagmites. Then our bus ride continued to Granada, where we explored the arab quarter of Albayzin with the best view of the Alhambra. Our guide claimed that every day tourist would get lost in the labyrinth of the quarter, which was for sure a bit exaggerated. We didn't go into any of the old houses, courtyards etc, if I should be back at some point, that's something I'd like to explore again. Nowadays google would lead the way anyway.\\

%Arcos de la Frontera/Ronda 1, Marbella 28 etc ? 14 nights

Nerja: Balcon de Europa***, Caves of Nerja*****\\
Granada: Albayzin***\\

August 24: Granada\\
The Alhambra is one of the most impressive palaces in Europe. The palace was built over centuries, with the most important part being build by the Nasrid Kings of Granada. After the conquest of Granada by the Spanish, king Carlos V built a renaissance palace on the plateau. The palace was left unfinished though. We had a tour of the Nasrid palace with its many courtyards, tile and carved stone decorations. Most of the paint faded away through the times, I was most impressed by the Lion Courtyard. Afterwards we went over to the gardens of the Generalife palace. Since we had a short coffee stop by the cathedral, we asked out guide if we could have a self payed look into the cathedral as well. We were brushed off that 15 minutes wouldn't be enough and it would be utterly boring for anybody to see (Once again she was wrong and proofed here incompetence). Once we arrived in Cordoba we stopped by the Roman bridge for a photo.\\

Granada: Alhambra*****, Generalife*****\\

August 25: Cordoba\\
The Mezquita-Cathedral is two sights for the price of one. It is the old mosque of the Caliphate of Cordoba, and if you look from most sides, it still looks to be completely intact. The Mihrab is still there, the forrest of columns with white and red double arches is not that much affected by the modifications of the Spanish kings, it is brilliantly done. In the middle of the mosque is the cathedral part, built in Baroque and Renaissance style. Obviously a completely different style and not going well together, the king is said to have announced his displeasure of not being told how awesome the mosque was, before agreeing to the construction of the cathedral. Nonetheless the cathedral is very beautiful itself as well. Back in 2001 though in a dire state and in clear need for renovation (which was finished by mid 2010s though). Impressive choir stalls with lots of wood carving, impressive main altar piece, beautiful ceiling. The side chapels are held in moorish style and blend in with the mosque. While my dad took a photo of the rood screen and the retro choir his flash decided to go up with a bang (scary but nothing more happened). The combination of the Mezquita and Cathedral is clearly my most favourite religious building i have been to so far (status being of 2020). Afterwards we walked through old town having a look at courtyards and flowers over the whitewashed walls. And then off to Sevilla, where we took a bus tour of the remaining pavilions of the EXPO 1992 (500 years of the Columbus Journey to the Americas) and the Puente del Alamillo\\ 

Cordoba: Mezquita-Cathedral*****\\

August 26: Sevilla\\
We started our day by the Plaza de Espana, a square set up in the 1920s as exhibition space for the Ibero-American exposition in 1929. Nowadays the square with its canal and it towers is used as prominent background for filming. Then we visited the Alcazar, the Royal Spanish palace in Sevilla, with vast gardens and even rooms which have been built in the 11th century by the Abbabid dynasty while Sevilla was the seat of muslim kings (really beautiful). Only a couple of metres away is the gigantic gothic cathedral of Sevilla, still the largest cathedral in the world and only a bit smaller than St Peter's, absolutely magnificent. It is also the home of the tomb of Christopher Columbus. The Retablo of the main altar is the largest in the world, and the pinnacle of Spanish medieval wood carving. Then we climbed the Giralda, the bell tower which is in fact the former Minaret of the previous mosque with Renaissance additions. \\

Sevilla: Alcazar*****, Cathedral*****\\

August 27: Cadiz, Jerez and Arcos de la Frontera\\
Driving along the Guadalquivir and the Isla Del Trocadero we arrived by the old city of Cadiz, where we stopped over lunch time. The cathedral was closed at that point and there wasn't much more to do than getting a snack. Then we drove through olive fields over to Jerez de la Frontera. On our way our guide praised the quality of Andalusian olives and talked down the horrible olives from Italy and Greece. Then she continued to tell us how amazing the Formula One Grand Prix of 2000 had been. Talking to German formula one fans, who all clearly do remember the Jerez Grand Prix in 1997, when Schumacher tried to kick out Villeneuve unsuccessfully and lost the championship, she clearly had no idea ONCE again. Or she just believed we would be stupid. After being told that she got her facts wrong she insisted that she is right. Clearly once again showing her incompetence. And we stopped by the cathedral and the Alcazar (not that we got to see any of them), but we had a sherry tasting by the Gonzalez Byass Sherry Bodegas, which was alright (having not developed a proper taste for alcoholic drinks at that point). The final stop was Arcos de la Frontera with a visit of the Santa Maria basilica.\\

Jerez de la Frontera: Gonzalez Byass Sherry Bodegas***, Arcos de la Frontera: Santa Maria***\\

August 28: Ronda:\\
The last stop of the Andalusia tour where we walked over the El Tajo Canyon, saw a couple of gardens, the church of Santa Maria la Mayor and one of the oldest bullfighting arenas in Spain. And then back to Marbella, where we transferred to our holiday club for the next couple of days.

Ronda: El Tajo Canyon****,  Santa Maria la Mayor***, Bullfighting Arena*\\

September 1: Marbella\\
What do you do as good German in a club full of tourists, pools, drinks and a closeby beach: watch football (just like the English families in that club did too). But what a horrible match it was: Germany 1 - England 5. Half of the holiday resort was celebrating and running around singing, whereas the other half is dead silent. All English fans felt pretty sorry for us. But then in the aftermath: Germany finished second in the group, in fact Germany qualified against Ukraine in the relegation matches, and made it into the final of the world cup 2002 - only losing out to Brazil.
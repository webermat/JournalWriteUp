\chapter{Year 2010: the year without any flight}
\label{2010}

Since 1996 i had at least one flight per year but this year was one without any flight. This was the year when i stayed so close to home, that no flight had been necessary. Something which I never repeated up to now (2020), although if we use maybe a more appropriate metric such as a distance between two flights, then a time span surpassing 365 or 366 days happened later on too.

\section{January 19--January 25: Ancona, CHIPP 2010}
\label{Ancona2010}

This is the second ever CHIPP school taking place in the Monte Verita centre on a hillside over Ancona. Almost all fellow ETH PhD students were going as well. The lectures focused on neutrino physics this time.\\

January 19: Ancona arriving at Monte Verita\\
Starting out early in the morning I took hopped on the train with direction to Milan, only to get off at Domodossola transferring to the Cento Valli train over to Locarno. The wintery landscape was very nice and snow was all over. By the time we arrived in Locarno it was already night, then a short bus ride and a bit of a walk later I arrived up on the Monte Verita centre. It was not too late though to go down to the town for dinner, enjoying the nightly views of the harbour and Lago Maggiore.\\

Ancona: Monte Verita***, Harbour***\\

January 22: Bellinzona\\
The social program included a trip to Bellinzona, where we could either have the day to explore the town, or spent the day at the SBB Cargo centre. I opted for the town: Bellinzona is dominated by three castles which are part of the World UNESCO heritage, the largest being the quite impressive Castelgrande. Castel Montebello has a more interesting layout, unfortunately the interior is closed in winter, but it is still possible to climb the walls, at Castel Sasso Corbaro even that is not possible. The church of Lugano is an ordinary Baroque church.\\

Bellinzona: Castelgrande****, Castel Montebello****, Collegiata dei Santi Pietro i Stefano***, Castel Sasso Corbaro***\\

January 255: Lugano \& Como\\
Since a day ticket was only 2 CHF more expensive than a direct ticket back to Geneva, I decided to see a bit more of Ticino, this time Lugano the largest city of the canton. It was very rainy, the mountain overlooking Lago Lugano was covered in clouds. Thus I just stepped into the cathedral (as most of Switzerland's cathedrals, rather small and nothing special), and then wondered what I could do next. Checking the schedule I realised I would have enough time to get to Como and from there back to Geneva. Getting rid of my stuff in one of the lockers I jumped on the next train to Chiasso and from there on to Como. Clearly the weather didn't improve on this short ride, so I didn't get any good views out of the Lago di Como. But the cathedral in Como is a lot larger and more stunning than the one in Lugano, the Duomo is a richly decorated Baroque church wtih one large Dome. Afterwards I was pleasantly surprised by the church of San Fedele. And then I got on my way home including five transfers (and getting my luggage in Lugano) as well as a ride over the (nowadays old) Gotthard tunnel.\\

Lugano: Cattedrale**\\
Como: Duomo****, San Fedele****

\section{February 6--February 7: Chamonix}
\label{Chamonix2010}

Not the first time I was in Chamonix and it wasn't the last time, but the one and only time I stayed over the night. ETH folks had planned a skiing weekend. Despite not skiing for over 10 years and neither planning to pick it up again at that point I decided to join, after all Chamonix offers nice landscape views and many cable car rides to take. Clearly that fun relies more on excellent weather than skiing does.\\

February 5:\\
Grand Montets (too foggy)*, Mer de Glace (too foggy)*, Church*, Museum of Crystals***\\

February 6:\\
Aiguille du Midi*****

\section{April 5--April 9: Bavaria}
\label{2010:Bavaria}

April 5: Schwangau \& F\"ussen:\\
Schwangau: Hohenschwangau***, Steingaden: Wieskirche*****, Neuschwanstein****, F\"ussen: St Mang****, Spitalkirche***, Hohes Schloss***, Lechfall***, Lechklamm****\\

April 6: Garmisch-Partenkirchen \& Ettal:\\
Ehrwald: Zugspitze*****, Ettal: Kloster****, Linderhof*****, Venusgrotte*****, Garmisch-Partenkirchen: Eibsee*****, Skisprung-Schanze***\\

April 7: M\"unchen\\
Nymphenburg mit Badenburg,Amalienburg,Pagodenburg \& Magdalenenklause*****, Olympiastadion****, Asamkirche*****, St Peter****, Heilig-Geist-Kirche***, Michaelskirche****, Frauenkirche***, Theatinerkirche***\\

April 8: M\"unchen\\
Residenz*****, Schatzkammer*****, Cuviliestheater****, Deutsches Museum*****, St Anna im Lehel**, Klosterkirche St Anna im Lehel***, Dreifaltigkeitskirche***\\

April 9: Augsburg \& Ulm:\\
Augsburg: Rathaus*****, St Afra** (in renovation), Baroque Gallery in Palais Sch\"atzlein**, Dom***, Ulm: M\"unster*****

\section{May 23: Luzern \& Rosenlaui}
\label{2010:Luzern}

Luzern: Spreuerbr\"ucke****, Jesuitenkirche****, Kapellbr\"ucke****, Weeping Lion***, Rosenlaui: Rosenlauischlucht*****

\section{June 27: Fiesch}
\label{Fiesch2010}

Aletschgletscher*****, Eggishorn*****, Fieschergletscher*****, Moosfluh*****

\section{July 11: Rhonegletscher}
\label{Rhonegletscher2010}

Rhonegletscher*****

\section{July 21--August 4: Paris, ICHEP2010}
\label{Paris2010}

The largest conference of the year happened to be taking place in Paris. After having presented a poster on simulated data in 2008, I made sure to have my analysis ready on real data in early 2010. Naturally my poster abstract had been accepted without any problems. Since Paris is quite close to Geneva as well, my boss encouraged me to show my results. The results were also shown by more senior people in other talks. Since I had not been in Paris since 1998 I asked for a couple of days off, in order to see the city as well as the surrounding place once again.\\

Co-travellers:\\
For the conference part most of my colleagues and friends had been around, but other than that I spent most of the other days with my mum and dad.\\

July 21: Paris:\\
Having arrived at Gare de Lyon with the earliest train possible I decided to walk the two stations to my hotel, where I had a room to myself just below the roof-top (elevator was not existing), so just walked up the four floors, dropped my stuff and on to discover Paris. The hotel was in the quarter of Marais, thus only a short walk away from the Ile de Cite. On my way to the island I stopped by the church of St Paul \& St Louis. Then we got to real business: The Conciergerie is the only surviving part of the old royal medieval palace, the hall of the guards is still impressive, but it is more or less one interesting room. OK there is also the cell of Marie Antoinette and other people where they spent their last days before beheading by the Guillotine. One of the most beautiful gems of Paris is just a couple of metres near the Conciergerie, the Sainte Chapelle, also part of the former royal palace. The stained glass is just breathtaking, absolutely magnificent. Still after all these centuries a place to impress people. Many times people tried to replicate this beauty, Stephen's Hall in the Houses of Parliament is claimed to be modelled after Sainte Chapelle for example. But all of those attempts cannot recreate the special atmosphere of the upper chapel. The lower chapel is nice, pales though compare to the upper chapel. And once we saw the little gem, right on to the large cathedral, located on the island as well. Notre Dame is an early gothic cathedral, which was largely renovated in the 19th century. Whenever I was in Paris I paid my visit - until disaster struck and the whole roof went up in flames in 2019. Whatever happens I will make sure to be back once it should be possible again. Still refusing to take the Metro I made my way to the Pantheon, at least unlike in 1998 no scaffolding had to be placed on certain parts due to instability of the ceilings. It is one of the late classical buildings in Paris, but the more I visited the less impress I was. On my way to the Dome d'Invalides I strolled through the Jardins du Luxembourg and had a short look into St Francois-Xavier, before getting to the Hotel d'Invalides. Originally built for the wounded of the wars it now hosts the Army Museum (a bit meeh, was more impressed by the tanks back as teenager in 98). The Dome d'Invalides houses Napoleon I's tomb, as well as remains of his family members. The main altar is nice, as well as the large dome. The cathedral of the army, St Louis d'Invalides is attached to the Dome and has a lot of flags being flown. On my way to the old opera house I stopped shortly by St Madelaine, a church in classic style, modelled after ancient temples on the outside. Opera Garnier never fails to disappoint, the staircase, the eclectic upper Foyer as well as the amazing amphitheatre, particularly the giant fresco by Marc Chagall, one item more beautiful than the other. The most beautiful opera house I have seen so far.\\

St Paul \& St Louis**, Conciergerie***, Sainte Chapelle*****, Notre Dame****, Pantheon****, St Francois-Xavier***, Dome d'Invalides****, Hotel d'Invalides: Musee d'Armee***, St Louis d'Invalides***, St Madeleine****, Opera Garnier*****, St. Germain l'Auxerrois***\\

July 25: Versailles:\\
Chateau*****, Trianons*****, Cathedrale***\\

July 26: ICHEP 2010\\
President Sarkozy decided that ICHEP was important enough to give a speech. Rumour has it that the organisers wanted the minister of science to give a speech, but she was quite reluctant. Once Sarkozy heard about it, he decided to give the speech himself and the minister was relegated to sitting in the audience and clap. For the speech itself it was energetic, obviously in French, we had to go through security and metal detectors. Sarkozy told us how important basic science is. how much his government increased the funding. Our French colleagues commented on that by saying he counts also the military developments as part of that so the increase was far lower than claimed though. It was nice at least to witness a speech by a world leader for once.\\

Paris: Presidential Speech****\\

July 29: Chartres:\\
Chartres: Cathedrale*****, St Pierre****, St Aignan****\\

July 30: Fontainebleau:\\
Fontainebleau: Chateau*****\\
Paris: Tour Montparnasse*****\\

July 31: \\
Vincennes: Chateau***\\
St Denis: Stade de France***, Basilica*****,\\
Puteaux: La Defense****\\
Paris: Arc de Triomphe****\\

August 1: Paris\\
Paris: Louvre*****, Musee d'Orsay***, Musee de l'Orangerie****, Centre Pompidou*****\\

August 2: Paris\\
Paris: Sacre-Coeur**, Madelaine****, Place de la Concorde*****, Grand Palais***,  St Sulpice****, St Germain des Pres****, Tour Eiffel*****\\

August 3: Reims\\
Reims: Palais du Tau****, Cathedrale*****, Roof \& Towers of the Cathedral*****, St Remi****\\

August 4: Paris\\
Paris: Petit Palais****, Hotel de Cluny****, St-Gervais-et-St-Protais*, St Eustache****

\section{August 8: Grindelwald}
\label{2010Grindelwald}

Weather forecast claimed the day could be cloudy, but it should be a dry day and the clouds would be high up. Indeed most of the clouds were high up, thus one could see most of the North face of the Eiger. I took the gondola up to First and walked over to Bachalpsee. It is a pretty easy hike, which definitely can be done in a family outing or for non-experienced hikers. On the way one can enjoy the views of the Lower Grindelwaldglacier, while on the Firstbahn the Upper Grindelwaldglacier is in perfect view. From First itself the view over to Titlis and Grosse Scheidegg is nice too. On the way back I got into a 5 minute heavy rain fall, even my jacket didn't keep me fully dry. Still I took the bus to the Gletscherschlucht restaurant (had a little snack there) and walked through the Glacier Gorge of the Lower Grindelwaldglacier. A pretty nice cute canyon, where some people even took a zip line down the rock face, or climbed up. Some people even cross the touristic area and try to climb up to the glacier. That might have been suspended now that the glacier walls became unstable. As last bit I took the bus to the Upper Grindelwaldglacier.. Back in the 80s my parents took us kids to the glacier cave of the upper Grindelwaldglacier, back then the glacier had reached a maximum of the last 60 years and was very impressive. Digging a glacier cave had been given up since, even climbing multiple ladders and crossing over the deep gorge of the melting water, the glacier tongue was unreachable. Ten years later the lower part of the glacier has split of at the lower icefall and the glacier lost another 2 km of its length, so go there while you still can.\\

First \& Bachalpsee*****, Glacier Gorge of the Lower Grindelwaldglacier****, Ladders up the Upper Grindelwaldglacier****

\section{August 21: Hasslachh\"ohle}
\label{2010Hasslach}

Erdmannsh\"ohle****

\section{August 22: Hohenzollern Castle \& Rottweil}
\label{2010Hohenzollern}

Hohenzollern Castle***, Rottweil: Kapellenkirche**, Predigerkirche****, Heilig-Kreuz-M\"unster****

\section{August 27: Jura}
\label{jura2010}

The trip where I really messed up: The 2010 official ETH particle physics institute outing to the Jura.\\
Both the Geneva and Zurich crowd met at the Neuchatel station switching to trains going up the Jura. In Brenets we all got on the boats leading through the narrow Lac des Brenets. This natural lake rather feels like a wide river inside an impressive nice little canyon with high grey rock walls. At the end of the lake, a small 15-20 minute walk leads to the impressive Saut du Doubs waterfall. The Swiss side had two platform just by the waterfall but the river of Doubs forms much of the border between Switzerland and France in this place. I decided that I would love to get the view from the French side from up the canyon. Said and done I got my photos ran back to the harbour, saw the boat but didn't speed up to reach it. Previously there had been 3 boats leaving within 10 minutes and I thought this would be the first one leaving. Turns out I messed up with the time and this was in fact the last boat leaving, with the next boat leaving not any earlier than 2 hours later.\\
Thus I decided to not give up and just walk back to the bus stop at the other side of the lake. Having arrived there I figured out also the next bus would take about an hour to leave. I called my postdoc, clarified I messed up and most probably should just skip lunch and get myself to the next touristic stop we were supposed to get to next (and turns out the bus passed it). Thus no large four course dinner for me, but rather a small not that much filling snack, but what can you do once you messed up.\\
And then on to the bus over to a mountain pass close to Le Locle. Obviously my professor and most of the others made fun how I managed to mess up with times, since I was rather known to keep accurate track of dates and times, and still only messing up the fun relaxing part of lunch while not sacrificing any of the tourist program. Back to the mills: They had been installed underground in one of the many cavities in this karst region and renovated lately. We had a guided tour through mills also going along the shafts which had to be cleaned from dirt and gravel, quite interesting to see.\\
This is so far also the only time I walked through the towns of Le Locle or La-Chaux-de-Fonds, both assigned a UNESCO world heritage status, but not knowing or caring about this back in 2010 I didn't bother to take a photograph. Thus this remains to this day the only UNESCO world heritage site I saw after getting my camera which I didn't take any single photograph of. So nobody else to blame but myself. Maybe still adds to the mistakes from my side for this particular trip.

Brenets: Lac des Brenets*****, Saut du Doubs*****\\
Le Locle: Moulins Souterrains***\\
La-Chaux-de-Fonds**

\section{August 28: Zermatt}
\label{Zermatt2010}

Unterrothorn*****, Kleinmatterhorn*****, Matterhorn Glacier Walk*****, Gornerschlucht****

\section{September 4: Saas Fee}
\label{2010SaasFee}

Saas-Fee: L\"angfluh*****, Mittelallalin*****, Eisgrotte Feegletscher*****, Saas Grund: Triftgletscher*****

\section{September 5: Engelberg}
\label{2010Titlis}

Titlis \& Titlisgletscher*****, Kloster Engelberg****

\section{September 12: Chamonix I}
\label{2010ChamonixI}

Aiguille du Midi*****, Panoramique Mont-Blanc*****, Pointe Helbronner*****, Glacier d'Argentiere*****, Glacier des Grands Montets****

\section{October 2: Chamonix II}
\label{2010ChamonixII}

Aiguille du Midi*****

\section{November 13: St Blasien}
\label{2010StBlasien}

Dom****

\section{November 27: Zermatt \& Bern}
\label{2010ZermattBern}

Swiss Meteo claimed after snow fall really early in the morning everything should clear up and a we could have a beautiful afternoon in Zermatt, so time to take my brother to Zermatt. We decided that we should go up to Unterrothorn. The side of Strahlhorn and the Findelgletscher were in superb view, but the Matterhorn was completely covered in clouds. So time to built a snow man (which proofed to be pretty hard considering the consistency of the snow). Still no Matterhorn to be seen, so we were mad and gave up and decided to take the train to Bern, enjoying at least free views of the Bisgletscher by Randa. In Bern they just had set up the Christmas market, but not too many stands had been opened thus we just went to the M\"unster instead, Switzerlands largest church in gothic style, always cute to see.\\

Zermatt: Unterrothorn****, Bisgletscher****\\
Bern: M\"unster****

\section{December 4: Zermatt}
\label{2010ZermattII}

Gornergrat*****, Kleinmatterhorn*****